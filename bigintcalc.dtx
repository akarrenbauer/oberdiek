% \iffalse meta-comment
%
% File: bigintcalc.dtx
% Version: 2012/04/08 v1.3
% Info: Expandable calculations on big integers
%
% Copyright (C) 2007, 2011, 2012 by
%    Heiko Oberdiek <heiko.oberdiek at googlemail.com>
%
% This work may be distributed and/or modified under the
% conditions of the LaTeX Project Public License, either
% version 1.3c of this license or (at your option) any later
% version. This version of this license is in
%    http://www.latex-project.org/lppl/lppl-1-3c.txt
% and the latest version of this license is in
%    http://www.latex-project.org/lppl.txt
% and version 1.3 or later is part of all distributions of
% LaTeX version 2005/12/01 or later.
%
% This work has the LPPL maintenance status "maintained".
%
% This Current Maintainer of this work is Heiko Oberdiek.
%
% The Base Interpreter refers to any `TeX-Format',
% because some files are installed in TDS:tex/generic//.
%
% This work consists of the main source file bigintcalc.dtx
% and the derived files
%    bigintcalc.sty, bigintcalc.pdf, bigintcalc.ins, bigintcalc.drv,
%    bigintcalc-test1.tex, bigintcalc-test2.tex,
%    bigintcalc-test3.tex.
%
% Distribution:
%    CTAN:macros/latex/contrib/oberdiek/bigintcalc.dtx
%    CTAN:macros/latex/contrib/oberdiek/bigintcalc.pdf
%
% Unpacking:
%    (a) If bigintcalc.ins is present:
%           tex bigintcalc.ins
%    (b) Without bigintcalc.ins:
%           tex bigintcalc.dtx
%    (c) If you insist on using LaTeX
%           latex \let\install=y% \iffalse meta-comment
%
% File: bigintcalc.dtx
% Version: 2016/05/16 v1.4
% Info: Expandable calculations on big integers
%
% Copyright (C) 2007, 2011, 2012 by
%    Heiko Oberdiek <heiko.oberdiek at googlemail.com>
%    2016
%    https://github.com/ho-tex/oberdiek/issues
%
% This work may be distributed and/or modified under the
% conditions of the LaTeX Project Public License, either
% version 1.3c of this license or (at your option) any later
% version. This version of this license is in
%    http://www.latex-project.org/lppl/lppl-1-3c.txt
% and the latest version of this license is in
%    http://www.latex-project.org/lppl.txt
% and version 1.3 or later is part of all distributions of
% LaTeX version 2005/12/01 or later.
%
% This work has the LPPL maintenance status "maintained".
%
% This Current Maintainer of this work is Heiko Oberdiek.
%
% The Base Interpreter refers to any `TeX-Format',
% because some files are installed in TDS:tex/generic//.
%
% This work consists of the main source file bigintcalc.dtx
% and the derived files
%    bigintcalc.sty, bigintcalc.pdf, bigintcalc.ins, bigintcalc.drv,
%    bigintcalc-test1.tex, bigintcalc-test2.tex,
%    bigintcalc-test3.tex.
%
% Distribution:
%    CTAN:macros/latex/contrib/oberdiek/bigintcalc.dtx
%    CTAN:macros/latex/contrib/oberdiek/bigintcalc.pdf
%
% Unpacking:
%    (a) If bigintcalc.ins is present:
%           tex bigintcalc.ins
%    (b) Without bigintcalc.ins:
%           tex bigintcalc.dtx
%    (c) If you insist on using LaTeX
%           latex \let\install=y% \iffalse meta-comment
%
% File: bigintcalc.dtx
% Version: 2016/05/16 v1.4
% Info: Expandable calculations on big integers
%
% Copyright (C) 2007, 2011, 2012 by
%    Heiko Oberdiek <heiko.oberdiek at googlemail.com>
%    2016
%    https://github.com/ho-tex/oberdiek/issues
%
% This work may be distributed and/or modified under the
% conditions of the LaTeX Project Public License, either
% version 1.3c of this license or (at your option) any later
% version. This version of this license is in
%    http://www.latex-project.org/lppl/lppl-1-3c.txt
% and the latest version of this license is in
%    http://www.latex-project.org/lppl.txt
% and version 1.3 or later is part of all distributions of
% LaTeX version 2005/12/01 or later.
%
% This work has the LPPL maintenance status "maintained".
%
% This Current Maintainer of this work is Heiko Oberdiek.
%
% The Base Interpreter refers to any `TeX-Format',
% because some files are installed in TDS:tex/generic//.
%
% This work consists of the main source file bigintcalc.dtx
% and the derived files
%    bigintcalc.sty, bigintcalc.pdf, bigintcalc.ins, bigintcalc.drv,
%    bigintcalc-test1.tex, bigintcalc-test2.tex,
%    bigintcalc-test3.tex.
%
% Distribution:
%    CTAN:macros/latex/contrib/oberdiek/bigintcalc.dtx
%    CTAN:macros/latex/contrib/oberdiek/bigintcalc.pdf
%
% Unpacking:
%    (a) If bigintcalc.ins is present:
%           tex bigintcalc.ins
%    (b) Without bigintcalc.ins:
%           tex bigintcalc.dtx
%    (c) If you insist on using LaTeX
%           latex \let\install=y% \iffalse meta-comment
%
% File: bigintcalc.dtx
% Version: 2016/05/16 v1.4
% Info: Expandable calculations on big integers
%
% Copyright (C) 2007, 2011, 2012 by
%    Heiko Oberdiek <heiko.oberdiek at googlemail.com>
%    2016
%    https://github.com/ho-tex/oberdiek/issues
%
% This work may be distributed and/or modified under the
% conditions of the LaTeX Project Public License, either
% version 1.3c of this license or (at your option) any later
% version. This version of this license is in
%    http://www.latex-project.org/lppl/lppl-1-3c.txt
% and the latest version of this license is in
%    http://www.latex-project.org/lppl.txt
% and version 1.3 or later is part of all distributions of
% LaTeX version 2005/12/01 or later.
%
% This work has the LPPL maintenance status "maintained".
%
% This Current Maintainer of this work is Heiko Oberdiek.
%
% The Base Interpreter refers to any `TeX-Format',
% because some files are installed in TDS:tex/generic//.
%
% This work consists of the main source file bigintcalc.dtx
% and the derived files
%    bigintcalc.sty, bigintcalc.pdf, bigintcalc.ins, bigintcalc.drv,
%    bigintcalc-test1.tex, bigintcalc-test2.tex,
%    bigintcalc-test3.tex.
%
% Distribution:
%    CTAN:macros/latex/contrib/oberdiek/bigintcalc.dtx
%    CTAN:macros/latex/contrib/oberdiek/bigintcalc.pdf
%
% Unpacking:
%    (a) If bigintcalc.ins is present:
%           tex bigintcalc.ins
%    (b) Without bigintcalc.ins:
%           tex bigintcalc.dtx
%    (c) If you insist on using LaTeX
%           latex \let\install=y\input{bigintcalc.dtx}
%        (quote the arguments according to the demands of your shell)
%
% Documentation:
%    (a) If bigintcalc.drv is present:
%           latex bigintcalc.drv
%    (b) Without bigintcalc.drv:
%           latex bigintcalc.dtx; ...
%    The class ltxdoc loads the configuration file ltxdoc.cfg
%    if available. Here you can specify further options, e.g.
%    use A4 as paper format:
%       \PassOptionsToClass{a4paper}{article}
%
%    Programm calls to get the documentation (example):
%       pdflatex bigintcalc.dtx
%       makeindex -s gind.ist bigintcalc.idx
%       pdflatex bigintcalc.dtx
%       makeindex -s gind.ist bigintcalc.idx
%       pdflatex bigintcalc.dtx
%
% Installation:
%    TDS:tex/generic/oberdiek/bigintcalc.sty
%    TDS:doc/latex/oberdiek/bigintcalc.pdf
%    TDS:doc/latex/oberdiek/test/bigintcalc-test1.tex
%    TDS:doc/latex/oberdiek/test/bigintcalc-test2.tex
%    TDS:doc/latex/oberdiek/test/bigintcalc-test3.tex
%    TDS:source/latex/oberdiek/bigintcalc.dtx
%
%<*ignore>
\begingroup
  \catcode123=1 %
  \catcode125=2 %
  \def\x{LaTeX2e}%
\expandafter\endgroup
\ifcase 0\ifx\install y1\fi\expandafter
         \ifx\csname processbatchFile\endcsname\relax\else1\fi
         \ifx\fmtname\x\else 1\fi\relax
\else\csname fi\endcsname
%</ignore>
%<*install>
\input docstrip.tex
\Msg{************************************************************************}
\Msg{* Installation}
\Msg{* Package: bigintcalc 2016/05/16 v1.4 Expandable calculations on big integers (HO)}
\Msg{************************************************************************}

\keepsilent
\askforoverwritefalse

\let\MetaPrefix\relax
\preamble

This is a generated file.

Project: bigintcalc
Version: 2016/05/16 v1.4

Copyright (C) 2007, 2011, 2012 by
   Heiko Oberdiek <heiko.oberdiek at googlemail.com>

This work may be distributed and/or modified under the
conditions of the LaTeX Project Public License, either
version 1.3c of this license or (at your option) any later
version. This version of this license is in
   http://www.latex-project.org/lppl/lppl-1-3c.txt
and the latest version of this license is in
   http://www.latex-project.org/lppl.txt
and version 1.3 or later is part of all distributions of
LaTeX version 2005/12/01 or later.

This work has the LPPL maintenance status "maintained".

This Current Maintainer of this work is Heiko Oberdiek.

The Base Interpreter refers to any `TeX-Format',
because some files are installed in TDS:tex/generic//.

This work consists of the main source file bigintcalc.dtx
and the derived files
   bigintcalc.sty, bigintcalc.pdf, bigintcalc.ins, bigintcalc.drv,
   bigintcalc-test1.tex, bigintcalc-test2.tex,
   bigintcalc-test3.tex.

\endpreamble
\let\MetaPrefix\DoubleperCent

\generate{%
  \file{bigintcalc.ins}{\from{bigintcalc.dtx}{install}}%
  \file{bigintcalc.drv}{\from{bigintcalc.dtx}{driver}}%
  \usedir{tex/generic/oberdiek}%
  \file{bigintcalc.sty}{\from{bigintcalc.dtx}{package}}%
%  \usedir{doc/latex/oberdiek/test}%
%  \file{bigintcalc-test1.tex}{\from{bigintcalc.dtx}{test1}}%
%  \file{bigintcalc-test2.tex}{\from{bigintcalc.dtx}{test2,etex}}%
%  \file{bigintcalc-test3.tex}{\from{bigintcalc.dtx}{test2,noetex}}%
  \nopreamble
  \nopostamble
  \usedir{source/latex/oberdiek/catalogue}%
  \file{bigintcalc.xml}{\from{bigintcalc.dtx}{catalogue}}%
}

\catcode32=13\relax% active space
\let =\space%
\Msg{************************************************************************}
\Msg{*}
\Msg{* To finish the installation you have to move the following}
\Msg{* file into a directory searched by TeX:}
\Msg{*}
\Msg{*     bigintcalc.sty}
\Msg{*}
\Msg{* To produce the documentation run the file `bigintcalc.drv'}
\Msg{* through LaTeX.}
\Msg{*}
\Msg{* Happy TeXing!}
\Msg{*}
\Msg{************************************************************************}

\endbatchfile
%</install>
%<*ignore>
\fi
%</ignore>
%<*driver>
\NeedsTeXFormat{LaTeX2e}
\ProvidesFile{bigintcalc.drv}%
  [2016/05/16 v1.4 Expandable calculations on big integers (HO)]%
\documentclass{ltxdoc}
\usepackage{wasysym}
\let\iint\relax
\let\iiint\relax
\usepackage[fleqn]{amsmath}

\DeclareMathOperator{\opInv}{Inv}
\DeclareMathOperator{\opAbs}{Abs}
\DeclareMathOperator{\opSgn}{Sgn}
\DeclareMathOperator{\opMin}{Min}
\DeclareMathOperator{\opMax}{Max}
\DeclareMathOperator{\opCmp}{Cmp}
\DeclareMathOperator{\opOdd}{Odd}
\DeclareMathOperator{\opInc}{Inc}
\DeclareMathOperator{\opDec}{Dec}
\DeclareMathOperator{\opAdd}{Add}
\DeclareMathOperator{\opSub}{Sub}
\DeclareMathOperator{\opShl}{Shl}
\DeclareMathOperator{\opShr}{Shr}
\DeclareMathOperator{\opMul}{Mul}
\DeclareMathOperator{\opSqr}{Sqr}
\DeclareMathOperator{\opFac}{Fac}
\DeclareMathOperator{\opPow}{Pow}
\DeclareMathOperator{\opDiv}{Div}
\DeclareMathOperator{\opMod}{Mod}
\DeclareMathOperator{\opInt}{Int}
\DeclareMathOperator{\opODD}{ifodd}

\newcommand*{\Def}{%
  \ensuremath{%
    \mathrel{\mathop{:}}=%
  }%
}
\newcommand*{\op}[1]{%
  \textsf{#1}%
}
\usepackage{holtxdoc}[2011/11/22]
\begin{document}
  \DocInput{bigintcalc.dtx}%
\end{document}
%</driver>
% \fi
%
%
% \CharacterTable
%  {Upper-case    \A\B\C\D\E\F\G\H\I\J\K\L\M\N\O\P\Q\R\S\T\U\V\W\X\Y\Z
%   Lower-case    \a\b\c\d\e\f\g\h\i\j\k\l\m\n\o\p\q\r\s\t\u\v\w\x\y\z
%   Digits        \0\1\2\3\4\5\6\7\8\9
%   Exclamation   \!     Double quote  \"     Hash (number) \#
%   Dollar        \$     Percent       \%     Ampersand     \&
%   Acute accent  \'     Left paren    \(     Right paren   \)
%   Asterisk      \*     Plus          \+     Comma         \,
%   Minus         \-     Point         \.     Solidus       \/
%   Colon         \:     Semicolon     \;     Less than     \<
%   Equals        \=     Greater than  \>     Question mark \?
%   Commercial at \@     Left bracket  \[     Backslash     \\
%   Right bracket \]     Circumflex    \^     Underscore    \_
%   Grave accent  \`     Left brace    \{     Vertical bar  \|
%   Right brace   \}     Tilde         \~}
%
% \GetFileInfo{bigintcalc.drv}
%
% \title{The \xpackage{bigintcalc} package}
% \date{2016/05/16 v1.4}
% \author{Heiko Oberdiek\thanks
% {Please report any issues at https://github.com/ho-tex/oberdiek/issues}\\
% \xemail{heiko.oberdiek at googlemail.com}}
%
% \maketitle
%
% \begin{abstract}
% This package provides expandable arithmetic operations
% with big integers that can exceed \TeX's number limits.
% \end{abstract}
%
% \tableofcontents
%
% \section{Documentation}
%
% \subsection{Introduction}
%
% Package \xpackage{bigintcalc} defines arithmetic operations
% that deal with big integers. Big integers can be given
% either as explicit integer number or as macro code
% that expands to an explicit number. \emph{Big} means that
% there is no limit on the size of the number. Big integers
% may exceed \TeX's range limitation of -2147483647 and 2147483647.
% Only memory issues will limit the usable range.
%
% In opposite to package \xpackage{intcalc}
% unexpandable command tokens are not supported, even if
% they are valid \TeX\ numbers like count registers or
% commands created by \cs{chardef}. Nevertheless they may
% be used, if they are prefixed by \cs{number}.
%
% Also \eTeX's \cs{numexpr} expressions are not supported
% directly in the manner of package \xpackage{intcalc}.
% However they can be given if \cs{the}\cs{numexpr}
% or \cs{number}\cs{numexpr} are used.
%
% The operations have the form of macros that take one or
% two integers as parameter and return the integer result.
% The macro name is a three letter operation name prefixed
% by the package name, e.g. \cs{bigintcalcAdd}|{10}{43}| returns
% |53|.
%
% The macros are fully expandable, exactly two expansion
% steps generate the result. Therefore the operations
% may be used nearly everywhere in \TeX, even inside
% \cs{csname}, file names,  or other
% expandable contexts.
%
% \subsection{Conditions}
%
% \subsubsection{Preconditions}
%
% \begin{itemize}
% \item
%   Arguments can be anything that expands to a number that consists
%   of optional signs and digits.
% \item
%   The arguments and return values must be sound.
%   Zero as divisor or factorials of negative numbers will cause errors.
% \end{itemize}
%
% \subsubsection{Postconditions}
%
% Additional properties of the macros apart from calculating
% a correct result (of course \smiley):
% \begin{itemize}
% \item
%   The macros are fully expandable. Thus they can
%   be used inside \cs{edef}, \cs{csname},
%   for example.
% \item
%   Furthermore exactly two expansion steps calculate the result.
% \item
%   The number consists of one optional minus sign and one or more
%   digits. The first digit is larger than zero for
%   numbers that consists of more than one digit.
%
%   In short, the number format is exactly the same as
%   \cs{number} generates, but without its range limitation.
%   And the tokens (minus sign, digits)
%   have catcode 12 (other).
% \item
%   Call by value is simulated. First the arguments are
%   converted to numbers. Then these numbers are used
%   in the calculations.
%
%   Remember that arguments
%   may contain expensive macros or \eTeX\ expressions.
%   This strategy avoids multiple evaluations of such
%   arguments.
% \end{itemize}
%
% \subsection{Error handling}
%
% Some errors are detected by the macros, example: division by zero.
% In this cases an undefined control sequence is called and causes
% a TeX error message, example: \cs{BigIntCalcError:DivisionByZero}.
% The name of the control sequence contains
% the reason for the error. The \TeX\ error may be ignored.
% Then the operation returns zero as result.
% Because the macros are supposed to work in expandible contexts.
% An traditional error message, however, is not expandable and
% would break these contexts.
%
% \subsection{Operations}
%
% Some definition equations below use the function $\opInt$
% that converts a real number to an integer. The number
% is truncated that means rounding to zero:
% \begin{gather*}
%   \opInt(x) \Def
%   \begin{cases}
%     \lfloor x\rfloor & \text{if $x\geq0$}\\
%     \lceil x\rceil & \text{otherwise}
%   \end{cases}
% \end{gather*}
%
% \subsubsection{\op{Num}}
%
% \begin{declcs}{bigintcalcNum} \M{x}
% \end{declcs}
% Macro \cs{bigintcalcNum} converts its argument to a normalized integer
% number without unnecessary leading zeros or signs.
% The result matches the regular expression:
%\begin{quote}
%\begin{verbatim}
%0|-?[1-9][0-9]*
%\end{verbatim}
%\end{quote}
%
% \subsubsection{\op{Inv}, \op{Abs}, \op{Sgn}}
%
% \begin{declcs}{bigintcalcInv} \M{x}
% \end{declcs}
% Macro \cs{bigintcalcInv} switches the sign.
% \begin{gather*}
%   \opInv(x) \Def -x
% \end{gather*}
%
% \begin{declcs}{bigintcalcAbs} \M{x}
% \end{declcs}
% Macro \cs{bigintcalcAbs} returns the absolute value
% of integer \meta{x}.
% \begin{gather*}
%   \opAbs(x) \Def \vert x\vert
% \end{gather*}
%
% \begin{declcs}{bigintcalcSgn} \M{x}
% \end{declcs}
% Macro \cs{bigintcalcSgn} encodes the sign of \meta{x} as number.
% \begin{gather*}
%   \opSgn(x) \Def
%   \begin{cases}
%     -1& \text{if $x<0$}\\
%      0& \text{if $x=0$}\\
%      1& \text{if $x>0$}
%   \end{cases}
% \end{gather*}
% These return values can easily be distinguished by \cs{ifcase}:
%\begin{quote}
%\begin{verbatim}
%\ifcase\bigintcalcSgn{<x>}
%  $x=0$
%\or
%  $x>0$
%\else
%  $x<0$
%\fi
%\end{verbatim}
%\end{quote}
%
% \subsubsection{\op{Min}, \op{Max}, \op{Cmp}}
%
% \begin{declcs}{bigintcalcMin} \M{x} \M{y}
% \end{declcs}
% Macro \cs{bigintcalcMin} returns the smaller of the two integers.
% \begin{gather*}
%   \opMin(x,y) \Def
%   \begin{cases}
%     x & \text{if $x<y$}\\
%     y & \text{otherwise}
%   \end{cases}
% \end{gather*}
%
% \begin{declcs}{bigintcalcMax} \M{x} \M{y}
% \end{declcs}
% Macro \cs{bigintcalcMax} returns the larger of the two integers.
% \begin{gather*}
%   \opMax(x,y) \Def
%   \begin{cases}
%     x & \text{if $x>y$}\\
%     y & \text{otherwise}
%   \end{cases}
% \end{gather*}
%
% \begin{declcs}{bigintcalcCmp} \M{x} \M{y}
% \end{declcs}
% Macro \cs{bigintcalcCmp} encodes the comparison result as number:
% \begin{gather*}
%   \opCmp(x,y) \Def
%   \begin{cases}
%     -1 & \text{if $x<y$}\\
%      0 & \text{if $x=y$}\\
%      1 & \text{if $x>y$}
%   \end{cases}
% \end{gather*}
% These values can be distinguished by \cs{ifcase}:
%\begin{quote}
%\begin{verbatim}
%\ifcase\bigintcalcCmp{<x>}{<y>}
%  $x=y$
%\or
%  $x>y$
%\else
%  $x<y$
%\fi
%\end{verbatim}
%\end{quote}
%
% \subsubsection{\op{Odd}}
%
% \begin{declcs}{bigintcalcOdd} \M{x}
% \end{declcs}
% \begin{gather*}
%   \opOdd(x) \Def
%   \begin{cases}
%     1 & \text{if $x$ is odd}\\
%     0 & \text{if $x$ is even}
%   \end{cases}
% \end{gather*}
%
% \subsubsection{\op{Inc}, \op{Dec}, \op{Add}, \op{Sub}}
%
% \begin{declcs}{bigintcalcInc} \M{x}
% \end{declcs}
% Macro \cs{bigintcalcInc} increments \meta{x} by one.
% \begin{gather*}
%   \opInc(x) \Def x + 1
% \end{gather*}
%
% \begin{declcs}{bigintcalcDec} \M{x}
% \end{declcs}
% Macro \cs{bigintcalcDec} decrements \meta{x} by one.
% \begin{gather*}
%   \opDec(x) \Def x - 1
% \end{gather*}
%
% \begin{declcs}{bigintcalcAdd} \M{x} \M{y}
% \end{declcs}
% Macro \cs{bigintcalcAdd} adds the two numbers.
% \begin{gather*}
%   \opAdd(x, y) \Def x + y
% \end{gather*}
%
% \begin{declcs}{bigintcalcSub} \M{x} \M{y}
% \end{declcs}
% Macro \cs{bigintcalcSub} calculates the difference.
% \begin{gather*}
%   \opSub(x, y) \Def x - y
% \end{gather*}
%
% \subsubsection{\op{Shl}, \op{Shr}}
%
% \begin{declcs}{bigintcalcShl} \M{x}
% \end{declcs}
% Macro \cs{bigintcalcShl} implements shifting to the left that
% means the number is multiplied by two.
% The sign is preserved.
% \begin{gather*}
%   \opShl(x) \Def x*2
% \end{gather*}
%
% \begin{declcs}{bigintcalcShr} \M{x}
% \end{declcs}
% Macro \cs{bigintcalcShr} implements shifting to the right.
% That is equivalent to an integer division by two.
% The sign is preserved.
% \begin{gather*}
%   \opShr(x) \Def \opInt(x/2)
% \end{gather*}
%
% \subsubsection{\op{Mul}, \op{Sqr}, \op{Fac}, \op{Pow}}
%
% \begin{declcs}{bigintcalcMul} \M{x} \M{y}
% \end{declcs}
% Macro \cs{bigintcalcMul} calculates the product of
% \meta{x} and \meta{y}.
% \begin{gather*}
%   \opMul(x,y) \Def x*y
% \end{gather*}
%
% \begin{declcs}{bigintcalcSqr} \M{x}
% \end{declcs}
% Macro \cs{bigintcalcSqr} returns the square product.
% \begin{gather*}
%   \opSqr(x) \Def x^2
% \end{gather*}
%
% \begin{declcs}{bigintcalcFac} \M{x}
% \end{declcs}
% Macro \cs{bigintcalcFac} returns the factorial of \meta{x}.
% Negative numbers are not permitted.
% \begin{gather*}
%   \opFac(x) \Def x!\qquad\text{for $x\geq0$}
% \end{gather*}
% ($0! = 1$)
%
% \begin{declcs}{bigintcalcPow} M{x} M{y}
% \end{declcs}
% Macro \cs{bigintcalcPow} calculates the value of \meta{x} to the
% power of \meta{y}. The error ``division by zero'' is thrown
% if \meta{x} is zero and \meta{y} is negative.
% permitted:
% \begin{gather*}
%   \opPow(x,y) \Def
%   \opInt(x^y)\qquad\text{for $x\neq0$ or $y\geq0$}
% \end{gather*}
% ($0^0 = 1$)
%
% \subsubsection{\op{Div}, \op{Mul}}
%
% \begin{declcs}{bigintcalcDiv} \M{x} \M{y}
% \end{declcs}
% Macro \cs{bigintcalcDiv} performs an integer division.
% Argument \meta{y} must not be zero.
% \begin{gather*}
%   \opDiv(x,y) \Def \opInt(x/y)\qquad\text{for $y\neq0$}
% \end{gather*}
%
% \begin{declcs}{bigintcalcMod} \M{x} \M{y}
% \end{declcs}
% Macro \cs{bigintcalcMod} gets the remainder of the integer
% division. The sign follows the divisor \meta{y}.
% Argument \meta{y} must not be zero.
% \begin{gather*}
%   \opMod(x,y) \Def x\mathrel{\%}y\qquad\text{for $y\neq0$}
% \end{gather*}
% The result ranges:
% \begin{gather*}
%   -\vert y\vert < \opMod(x,y) \leq 0\qquad\text{for $y<0$}\\
%   0 \leq \opMod(x,y) < y\qquad\text{for $y\geq0$}
% \end{gather*}
%
% \subsection{Interface for programmers}
%
% If the programmer can ensure some more properties about
% the arguments of the operations, then the following
% macros are a little more efficient.
%
% In general numbers must obey the following constraints:
% \begin{itemize}
% \item Plain number: digit tokens only, no command tokens.
% \item Non-negative. Signs are forbidden.
% \item Delimited by exclamation mark. Curly braces
%       around the number are not allowed and will
%       break the code.
% \end{itemize}
%
% \begin{declcs}{BigIntCalcOdd} \meta{number} |!|
% \end{declcs}
% |1|/|0| is returned if \meta{number} is odd/even.
%
% \begin{declcs}{BigIntCalcInc} \meta{number} |!|
% \end{declcs}
% Incrementation.
%
% \begin{declcs}{BigIntCalcDec} \meta{number} |!|
% \end{declcs}
% Decrementation, positive number without zero.
%
% \begin{declcs}{BigIntCalcAdd} \meta{number A} |!| \meta{number B} |!|
% \end{declcs}
% Addition, $A\geq B$.
%
% \begin{declcs}{BigIntCalcSub} \meta{number A} |!| \meta{number B} |!|
% \end{declcs}
% Subtraction, $A\geq B$.
%
% \begin{declcs}{BigIntCalcShl} \meta{number} |!|
% \end{declcs}
% Left shift (multiplication with two).
%
% \begin{declcs}{BigIntCalcShr} \meta{number} |!|
% \end{declcs}
% Right shift (integer division by two).
%
% \begin{declcs}{BigIntCalcMul} \meta{number A} |!| \meta{number B} |!|
% \end{declcs}
% Multiplication, $A\geq B$.
%
% \begin{declcs}{BigIntCalcDiv} \meta{number A} |!| \meta{number B} |!|
% \end{declcs}
% Division operation.
%
% \begin{declcs}{BigIntCalcMod} \meta{number A} |!| \meta{number B} |!|
% \end{declcs}
% Modulo operation.
%
%
% \StopEventually{
% }
%
% \section{Implementation}
%
%    \begin{macrocode}
%<*package>
%    \end{macrocode}
%
% \subsection{Reload check and package identification}
%    Reload check, especially if the package is not used with \LaTeX.
%    \begin{macrocode}
\begingroup\catcode61\catcode48\catcode32=10\relax%
  \catcode13=5 % ^^M
  \endlinechar=13 %
  \catcode35=6 % #
  \catcode39=12 % '
  \catcode44=12 % ,
  \catcode45=12 % -
  \catcode46=12 % .
  \catcode58=12 % :
  \catcode64=11 % @
  \catcode123=1 % {
  \catcode125=2 % }
  \expandafter\let\expandafter\x\csname ver@bigintcalc.sty\endcsname
  \ifx\x\relax % plain-TeX, first loading
  \else
    \def\empty{}%
    \ifx\x\empty % LaTeX, first loading,
      % variable is initialized, but \ProvidesPackage not yet seen
    \else
      \expandafter\ifx\csname PackageInfo\endcsname\relax
        \def\x#1#2{%
          \immediate\write-1{Package #1 Info: #2.}%
        }%
      \else
        \def\x#1#2{\PackageInfo{#1}{#2, stopped}}%
      \fi
      \x{bigintcalc}{The package is already loaded}%
      \aftergroup\endinput
    \fi
  \fi
\endgroup%
%    \end{macrocode}
%    Package identification:
%    \begin{macrocode}
\begingroup\catcode61\catcode48\catcode32=10\relax%
  \catcode13=5 % ^^M
  \endlinechar=13 %
  \catcode35=6 % #
  \catcode39=12 % '
  \catcode40=12 % (
  \catcode41=12 % )
  \catcode44=12 % ,
  \catcode45=12 % -
  \catcode46=12 % .
  \catcode47=12 % /
  \catcode58=12 % :
  \catcode64=11 % @
  \catcode91=12 % [
  \catcode93=12 % ]
  \catcode123=1 % {
  \catcode125=2 % }
  \expandafter\ifx\csname ProvidesPackage\endcsname\relax
    \def\x#1#2#3[#4]{\endgroup
      \immediate\write-1{Package: #3 #4}%
      \xdef#1{#4}%
    }%
  \else
    \def\x#1#2[#3]{\endgroup
      #2[{#3}]%
      \ifx#1\@undefined
        \xdef#1{#3}%
      \fi
      \ifx#1\relax
        \xdef#1{#3}%
      \fi
    }%
  \fi
\expandafter\x\csname ver@bigintcalc.sty\endcsname
\ProvidesPackage{bigintcalc}%
  [2016/05/16 v1.4 Expandable calculations on big integers (HO)]%
%    \end{macrocode}
%
% \subsection{Catcodes}
%
%    \begin{macrocode}
\begingroup\catcode61\catcode48\catcode32=10\relax%
  \catcode13=5 % ^^M
  \endlinechar=13 %
  \catcode123=1 % {
  \catcode125=2 % }
  \catcode64=11 % @
  \def\x{\endgroup
    \expandafter\edef\csname BIC@AtEnd\endcsname{%
      \endlinechar=\the\endlinechar\relax
      \catcode13=\the\catcode13\relax
      \catcode32=\the\catcode32\relax
      \catcode35=\the\catcode35\relax
      \catcode61=\the\catcode61\relax
      \catcode64=\the\catcode64\relax
      \catcode123=\the\catcode123\relax
      \catcode125=\the\catcode125\relax
    }%
  }%
\x\catcode61\catcode48\catcode32=10\relax%
\catcode13=5 % ^^M
\endlinechar=13 %
\catcode35=6 % #
\catcode64=11 % @
\catcode123=1 % {
\catcode125=2 % }
\def\TMP@EnsureCode#1#2{%
  \edef\BIC@AtEnd{%
    \BIC@AtEnd
    \catcode#1=\the\catcode#1\relax
  }%
  \catcode#1=#2\relax
}
\TMP@EnsureCode{33}{12}% !
\TMP@EnsureCode{36}{14}% $ (comment!)
\TMP@EnsureCode{38}{14}% & (comment!)
\TMP@EnsureCode{40}{12}% (
\TMP@EnsureCode{41}{12}% )
\TMP@EnsureCode{42}{12}% *
\TMP@EnsureCode{43}{12}% +
\TMP@EnsureCode{45}{12}% -
\TMP@EnsureCode{46}{12}% .
\TMP@EnsureCode{47}{12}% /
\TMP@EnsureCode{58}{11}% : (letter!)
\TMP@EnsureCode{60}{12}% <
\TMP@EnsureCode{62}{12}% >
\TMP@EnsureCode{63}{14}% ? (comment!)
\TMP@EnsureCode{91}{12}% [
\TMP@EnsureCode{93}{12}% ]
\edef\BIC@AtEnd{\BIC@AtEnd\noexpand\endinput}
\begingroup\expandafter\expandafter\expandafter\endgroup
\expandafter\ifx\csname BIC@TestMode\endcsname\relax
\else
  \catcode63=9 % ? (ignore)
\fi
? \let\BIC@@TestMode\BIC@TestMode
%    \end{macrocode}
%
% \subsection{\eTeX\ detection}
%
%    \begin{macrocode}
\begingroup\expandafter\expandafter\expandafter\endgroup
\expandafter\ifx\csname numexpr\endcsname\relax
  \catcode36=9 % $ (ignore)
\else
  \catcode38=9 % & (ignore)
\fi
%    \end{macrocode}
%
% \subsection{Help macros}
%
%    \begin{macro}{\BIC@Fi}
%    \begin{macrocode}
\let\BIC@Fi\fi
%    \end{macrocode}
%    \end{macro}
%    \begin{macro}{\BIC@AfterFi}
%    \begin{macrocode}
\def\BIC@AfterFi#1#2\BIC@Fi{\fi#1}%
%    \end{macrocode}
%    \end{macro}
%    \begin{macro}{\BIC@AfterFiFi}
%    \begin{macrocode}
\def\BIC@AfterFiFi#1#2\BIC@Fi{\fi\fi#1}%
%    \end{macrocode}
%    \end{macro}
%    \begin{macro}{\BIC@AfterFiFiFi}
%    \begin{macrocode}
\def\BIC@AfterFiFiFi#1#2\BIC@Fi{\fi\fi\fi#1}%
%    \end{macrocode}
%    \end{macro}
%
%    \begin{macro}{\BIC@Space}
%    \begin{macrocode}
\begingroup
  \def\x#1{\endgroup
    \let\BIC@Space= #1%
  }%
\x{ }
%    \end{macrocode}
%    \end{macro}
%
% \subsection{Expand number}
%
%    \begin{macrocode}
\begingroup\expandafter\expandafter\expandafter\endgroup
\expandafter\ifx\csname RequirePackage\endcsname\relax
  \def\TMP@RequirePackage#1[#2]{%
    \begingroup\expandafter\expandafter\expandafter\endgroup
    \expandafter\ifx\csname ver@#1.sty\endcsname\relax
      \input #1.sty\relax
    \fi
  }%
  \TMP@RequirePackage{pdftexcmds}[2007/11/11]%
\else
  \RequirePackage{pdftexcmds}[2007/11/11]%
\fi
%    \end{macrocode}
%
%    \begin{macrocode}
\begingroup\expandafter\expandafter\expandafter\endgroup
\expandafter\ifx\csname pdf@escapehex\endcsname\relax
%    \end{macrocode}
%
%    \begin{macro}{\BIC@Expand}
%    \begin{macrocode}
  \def\BIC@Expand#1{%
    \romannumeral0%
    \BIC@@Expand#1!\@nil{}%
  }%
%    \end{macrocode}
%    \end{macro}
%    \begin{macro}{\BIC@@Expand}
%    \begin{macrocode}
  \def\BIC@@Expand#1#2\@nil#3{%
    \expandafter\ifcat\noexpand#1\relax
      \expandafter\@firstoftwo
    \else
      \expandafter\@secondoftwo
    \fi
    {%
      \expandafter\BIC@@Expand#1#2\@nil{#3}%
    }{%
      \ifx#1!%
        \expandafter\@firstoftwo
      \else
        \expandafter\@secondoftwo
      \fi
      { #3}{%
        \BIC@@Expand#2\@nil{#3#1}%
      }%
    }%
  }%
%    \end{macrocode}
%    \end{macro}
%    \begin{macro}{\@firstoftwo}
%    \begin{macrocode}
  \expandafter\ifx\csname @firstoftwo\endcsname\relax
    \long\def\@firstoftwo#1#2{#1}%
  \fi
%    \end{macrocode}
%    \end{macro}
%    \begin{macro}{\@secondoftwo}
%    \begin{macrocode}
  \expandafter\ifx\csname @secondoftwo\endcsname\relax
    \long\def\@secondoftwo#1#2{#2}%
  \fi
%    \end{macrocode}
%    \end{macro}
%
%    \begin{macrocode}
\else
%    \end{macrocode}
%    \begin{macro}{\BIC@Expand}
%    \begin{macrocode}
  \def\BIC@Expand#1{%
    \romannumeral0\expandafter\expandafter\expandafter\BIC@Space
    \pdf@unescapehex{%
      \expandafter\expandafter\expandafter
      \BIC@StripHexSpace\pdf@escapehex{#1}20\@nil
    }%
  }%
%    \end{macrocode}
%    \end{macro}
%    \begin{macro}{\BIC@StripHexSpace}
%    \begin{macrocode}
  \def\BIC@StripHexSpace#120#2\@nil{%
    #1%
    \ifx\\#2\\%
    \else
      \BIC@AfterFi{%
        \BIC@StripHexSpace#2\@nil
      }%
    \BIC@Fi
  }%
%    \end{macrocode}
%    \end{macro}
%    \begin{macrocode}
\fi
%    \end{macrocode}
%
% \subsection{Normalize expanded number}
%
%    \begin{macro}{\BIC@Normalize}
%    |#1|: result sign\\
%    |#2|: first token of number
%    \begin{macrocode}
\def\BIC@Normalize#1#2{%
  \ifx#2-%
    \ifx\\#1\\%
      \BIC@AfterFiFi{%
        \BIC@Normalize-%
      }%
    \else
      \BIC@AfterFiFi{%
        \BIC@Normalize{}%
      }%
    \fi
  \else
    \ifx#2+%
      \BIC@AfterFiFi{%
        \BIC@Normalize{#1}%
      }%
    \else
      \ifx#20%
        \BIC@AfterFiFiFi{%
          \BIC@NormalizeZero{#1}%
        }%
      \else
        \BIC@AfterFiFiFi{%
          \BIC@NormalizeDigits#1#2%
        }%
      \fi
    \fi
  \BIC@Fi
}
%    \end{macrocode}
%    \end{macro}
%    \begin{macro}{\BIC@NormalizeZero}
%    \begin{macrocode}
\def\BIC@NormalizeZero#1#2{%
  \ifx#2!%
    \BIC@AfterFi{ 0}%
  \else
    \ifx#20%
      \BIC@AfterFiFi{%
        \BIC@NormalizeZero{#1}%
      }%
    \else
      \BIC@AfterFiFi{%
        \BIC@NormalizeDigits#1#2%
      }%
    \fi
  \BIC@Fi
}
%    \end{macrocode}
%    \end{macro}
%    \begin{macro}{\BIC@NormalizeDigits}
%    \begin{macrocode}
\def\BIC@NormalizeDigits#1!{ #1}
%    \end{macrocode}
%    \end{macro}
%
% \subsection{\op{Num}}
%
%    \begin{macro}{\bigintcalcNum}
%    \begin{macrocode}
\def\bigintcalcNum#1{%
  \romannumeral0%
  \expandafter\expandafter\expandafter\BIC@Normalize
  \expandafter\expandafter\expandafter{%
  \expandafter\expandafter\expandafter}%
  \BIC@Expand{#1}!%
}
%    \end{macrocode}
%    \end{macro}
%
% \subsection{\op{Inv}, \op{Abs}, \op{Sgn}}
%
%
%    \begin{macro}{\bigintcalcInv}
%    \begin{macrocode}
\def\bigintcalcInv#1{%
  \romannumeral0\expandafter\expandafter\expandafter\BIC@Space
  \bigintcalcNum{-#1}%
}
%    \end{macrocode}
%    \end{macro}
%
%    \begin{macro}{\bigintcalcAbs}
%    \begin{macrocode}
\def\bigintcalcAbs#1{%
  \romannumeral0%
  \expandafter\expandafter\expandafter\BIC@Abs
  \bigintcalcNum{#1}%
}
%    \end{macrocode}
%    \end{macro}
%    \begin{macro}{\BIC@Abs}
%    \begin{macrocode}
\def\BIC@Abs#1{%
  \ifx#1-%
    \expandafter\BIC@Space
  \else
    \expandafter\BIC@Space
    \expandafter#1%
  \fi
}
%    \end{macrocode}
%    \end{macro}
%
%    \begin{macro}{\bigintcalcSgn}
%    \begin{macrocode}
\def\bigintcalcSgn#1{%
  \number
  \expandafter\expandafter\expandafter\BIC@Sgn
  \bigintcalcNum{#1}! %
}
%    \end{macrocode}
%    \end{macro}
%    \begin{macro}{\BIC@Sgn}
%    \begin{macrocode}
\def\BIC@Sgn#1#2!{%
  \ifx#1-%
    -1%
  \else
    \ifx#10%
      0%
    \else
      1%
    \fi
  \fi
}
%    \end{macrocode}
%    \end{macro}
%
% \subsection{\op{Cmp}, \op{Min}, \op{Max}}
%
%    \begin{macro}{\bigintcalcCmp}
%    \begin{macrocode}
\def\bigintcalcCmp#1#2{%
  \number
  \expandafter\expandafter\expandafter\BIC@Cmp
  \bigintcalcNum{#2}!{#1}%
}
%    \end{macrocode}
%    \end{macro}
%    \begin{macro}{\BIC@Cmp}
%    \begin{macrocode}
\def\BIC@Cmp#1!#2{%
  \expandafter\expandafter\expandafter\BIC@@Cmp
  \bigintcalcNum{#2}!#1!%
}
%    \end{macrocode}
%    \end{macro}
%    \begin{macro}{\BIC@@Cmp}
%    \begin{macrocode}
\def\BIC@@Cmp#1#2!#3#4!{%
  \ifx#1-%
    \ifx#3-%
      \BIC@AfterFiFi{%
        \BIC@@Cmp#4!#2!%
      }%
    \else
      \BIC@AfterFiFi{%
        -1 %
      }%
    \fi
  \else
    \ifx#3-%
      \BIC@AfterFiFi{%
        1 %
      }%
    \else
      \BIC@AfterFiFi{%
        \BIC@CmpLength#1#2!#3#4!#1#2!#3#4!%
      }%
    \fi
  \BIC@Fi
}
%    \end{macrocode}
%    \end{macro}
%    \begin{macro}{\BIC@PosCmp}
%    \begin{macrocode}
\def\BIC@PosCmp#1!#2!{%
  \BIC@CmpLength#1!#2!#1!#2!%
}
%    \end{macrocode}
%    \end{macro}
%    \begin{macro}{\BIC@CmpLength}
%    \begin{macrocode}
\def\BIC@CmpLength#1#2!#3#4!{%
  \ifx\\#2\\%
    \ifx\\#4\\%
      \BIC@AfterFiFi\BIC@CmpDiff
    \else
      \BIC@AfterFiFi{%
        \BIC@CmpResult{-1}%
      }%
    \fi
  \else
    \ifx\\#4\\%
      \BIC@AfterFiFi{%
        \BIC@CmpResult1%
      }%
    \else
      \BIC@AfterFiFi{%
        \BIC@CmpLength#2!#4!%
      }%
    \fi
  \BIC@Fi
}
%    \end{macrocode}
%    \end{macro}
%    \begin{macro}{\BIC@CmpResult}
%    \begin{macrocode}
\def\BIC@CmpResult#1#2!#3!{#1 }
%    \end{macrocode}
%    \end{macro}
%    \begin{macro}{\BIC@CmpDiff}
%    \begin{macrocode}
\def\BIC@CmpDiff#1#2!#3#4!{%
  \ifnum#1<#3 %
    \BIC@AfterFi{%
      -1 %
    }%
  \else
    \ifnum#1>#3 %
      \BIC@AfterFiFi{%
        1 %
      }%
    \else
      \ifx\\#2\\%
        \BIC@AfterFiFiFi{%
          0 %
        }%
      \else
        \BIC@AfterFiFiFi{%
          \BIC@CmpDiff#2!#4!%
        }%
      \fi
    \fi
  \BIC@Fi
}
%    \end{macrocode}
%    \end{macro}
%
%    \begin{macro}{\bigintcalcMin}
%    \begin{macrocode}
\def\bigintcalcMin#1{%
  \romannumeral0%
  \expandafter\expandafter\expandafter\BIC@MinMax
  \bigintcalcNum{#1}!-!%
}
%    \end{macrocode}
%    \end{macro}
%    \begin{macro}{\bigintcalcMax}
%    \begin{macrocode}
\def\bigintcalcMax#1{%
  \romannumeral0%
  \expandafter\expandafter\expandafter\BIC@MinMax
  \bigintcalcNum{#1}!!%
}
%    \end{macrocode}
%    \end{macro}
%    \begin{macro}{\BIC@MinMax}
%    |#1|: $x$\\
%    |#2|: sign for comparison\\
%    |#3|: $y$
%    \begin{macrocode}
\def\BIC@MinMax#1!#2!#3{%
  \expandafter\expandafter\expandafter\BIC@@MinMax
  \bigintcalcNum{#3}!#1!#2!%
}
%    \end{macrocode}
%    \end{macro}
%    \begin{macro}{\BIC@@MinMax}
%    |#1|: $y$\\
%    |#2|: $x$\\
%    |#3|: sign for comparison
%    \begin{macrocode}
\def\BIC@@MinMax#1!#2!#3!{%
  \ifnum\BIC@@Cmp#1!#2!=#31 %
    \BIC@AfterFi{ #1}%
  \else
    \BIC@AfterFi{ #2}%
  \BIC@Fi
}
%    \end{macrocode}
%    \end{macro}
%
% \subsection{\op{Odd}}
%
%    \begin{macro}{\bigintcalcOdd}
%    \begin{macrocode}
\def\bigintcalcOdd#1{%
  \romannumeral0%
  \expandafter\expandafter\expandafter\BIC@Odd
  \bigintcalcAbs{#1}!%
}
%    \end{macrocode}
%    \end{macro}
%    \begin{macro}{\BigIntCalcOdd}
%    \begin{macrocode}
\def\BigIntCalcOdd#1!{%
  \romannumeral0%
  \BIC@Odd#1!%
}
%    \end{macrocode}
%    \end{macro}
%    \begin{macro}{\BIC@Odd}
%    |#1|: $x$
%    \begin{macrocode}
\def\BIC@Odd#1#2{%
  \ifx#2!%
    \ifodd#1 %
      \BIC@AfterFiFi{ 1}%
    \else
      \BIC@AfterFiFi{ 0}%
    \fi
  \else
    \expandafter\BIC@Odd\expandafter#2%
  \BIC@Fi
}
%    \end{macrocode}
%    \end{macro}
%
% \subsection{\op{Inc}, \op{Dec}}
%
%    \begin{macro}{\bigintcalcInc}
%    \begin{macrocode}
\def\bigintcalcInc#1{%
  \romannumeral0%
  \expandafter\expandafter\expandafter\BIC@IncSwitch
  \bigintcalcNum{#1}!%
}
%    \end{macrocode}
%    \end{macro}
%    \begin{macro}{\BIC@IncSwitch}
%    \begin{macrocode}
\def\BIC@IncSwitch#1#2!{%
  \ifcase\BIC@@Cmp#1#2!-1!%
    \BIC@AfterFi{ 0}%
  \or
    \BIC@AfterFi{%
      \BIC@Inc#1#2!{}%
    }%
  \else
    \BIC@AfterFi{%
      \expandafter-\romannumeral0%
      \BIC@Dec#2!{}%
    }%
  \BIC@Fi
}
%    \end{macrocode}
%    \end{macro}
%
%    \begin{macro}{\bigintcalcDec}
%    \begin{macrocode}
\def\bigintcalcDec#1{%
  \romannumeral0%
  \expandafter\expandafter\expandafter\BIC@DecSwitch
  \bigintcalcNum{#1}!%
}
%    \end{macrocode}
%    \end{macro}
%    \begin{macro}{\BIC@DecSwitch}
%    \begin{macrocode}
\def\BIC@DecSwitch#1#2!{%
  \ifcase\BIC@Sgn#1#2! %
    \BIC@AfterFi{ -1}%
  \or
    \BIC@AfterFi{%
      \BIC@Dec#1#2!{}%
    }%
  \else
    \BIC@AfterFi{%
      \expandafter-\romannumeral0%
      \BIC@Inc#2!{}%
    }%
  \BIC@Fi
}
%    \end{macrocode}
%    \end{macro}
%
%    \begin{macro}{\BigIntCalcInc}
%    \begin{macrocode}
\def\BigIntCalcInc#1!{%
  \romannumeral0\BIC@Inc#1!{}%
}
%    \end{macrocode}
%    \end{macro}
%    \begin{macro}{\BigIntCalcDec}
%    \begin{macrocode}
\def\BigIntCalcDec#1!{%
  \romannumeral0\BIC@Dec#1!{}%
}
%    \end{macrocode}
%    \end{macro}
%
%    \begin{macro}{\BIC@Inc}
%    \begin{macrocode}
\def\BIC@Inc#1#2!#3{%
  \ifx\\#2\\%
    \BIC@AfterFi{%
      \BIC@@Inc1#1#3!{}%
    }%
  \else
    \BIC@AfterFi{%
      \BIC@Inc#2!{#1#3}%
    }%
  \BIC@Fi
}
%    \end{macrocode}
%    \end{macro}
%    \begin{macro}{\BIC@@Inc}
%    \begin{macrocode}
\def\BIC@@Inc#1#2#3!#4{%
  \ifcase#1 %
    \ifx\\#3\\%
      \BIC@AfterFiFi{ #2#4}%
    \else
      \BIC@AfterFiFi{%
        \BIC@@Inc0#3!{#2#4}%
      }%
    \fi
  \else
    \ifnum#2<9 %
      \BIC@AfterFiFi{%
&       \expandafter\BIC@@@Inc\the\numexpr#2+1\relax
$       \expandafter\expandafter\expandafter\BIC@@@Inc
$       \ifcase#2 \expandafter1%
$       \or\expandafter2%
$       \or\expandafter3%
$       \or\expandafter4%
$       \or\expandafter5%
$       \or\expandafter6%
$       \or\expandafter7%
$       \or\expandafter8%
$       \or\expandafter9%
$?      \else\BigIntCalcError:ThisCannotHappen%
$       \fi
        0#3!{#4}%
      }%
    \else
      \BIC@AfterFiFi{%
        \BIC@@@Inc01#3!{#4}%
      }%
    \fi
  \BIC@Fi
}
%    \end{macrocode}
%    \end{macro}
%    \begin{macro}{\BIC@@@Inc}
%    \begin{macrocode}
\def\BIC@@@Inc#1#2#3!#4{%
  \ifx\\#3\\%
    \ifnum#2=1 %
      \BIC@AfterFiFi{ 1#1#4}%
    \else
      \BIC@AfterFiFi{ #1#4}%
    \fi
  \else
    \BIC@AfterFi{%
      \BIC@@Inc#2#3!{#1#4}%
    }%
  \BIC@Fi
}
%    \end{macrocode}
%    \end{macro}
%
%    \begin{macro}{\BIC@Dec}
%    \begin{macrocode}
\def\BIC@Dec#1#2!#3{%
  \ifx\\#2\\%
    \BIC@AfterFi{%
      \BIC@@Dec1#1#3!{}%
    }%
  \else
    \BIC@AfterFi{%
      \BIC@Dec#2!{#1#3}%
    }%
  \BIC@Fi
}
%    \end{macrocode}
%    \end{macro}
%    \begin{macro}{\BIC@@Dec}
%    \begin{macrocode}
\def\BIC@@Dec#1#2#3!#4{%
  \ifcase#1 %
    \ifx\\#3\\%
      \BIC@AfterFiFi{ #2#4}%
    \else
      \BIC@AfterFiFi{%
        \BIC@@Dec0#3!{#2#4}%
      }%
    \fi
  \else
    \ifnum#2>0 %
      \BIC@AfterFiFi{%
&       \expandafter\BIC@@@Dec\the\numexpr#2-1\relax
$       \expandafter\expandafter\expandafter\BIC@@@Dec
$       \ifcase#2
$?        \BigIntCalcError:ThisCannotHappen%
$       \or\expandafter0%
$       \or\expandafter1%
$       \or\expandafter2%
$       \or\expandafter3%
$       \or\expandafter4%
$       \or\expandafter5%
$       \or\expandafter6%
$       \or\expandafter7%
$       \or\expandafter8%
$?      \else\BigIntCalcError:ThisCannotHappen%
$       \fi
        0#3!{#4}%
      }%
    \else
      \BIC@AfterFiFi{%
        \BIC@@@Dec91#3!{#4}%
      }%
    \fi
  \BIC@Fi
}
%    \end{macrocode}
%    \end{macro}
%    \begin{macro}{\BIC@@@Dec}
%    \begin{macrocode}
\def\BIC@@@Dec#1#2#3!#4{%
  \ifx\\#3\\%
    \ifcase#1 %
      \ifx\\#4\\%
        \BIC@AfterFiFiFi{ 0}%
      \else
        \BIC@AfterFiFiFi{ #4}%
      \fi
    \else
      \BIC@AfterFiFi{ #1#4}%
    \fi
  \else
    \BIC@AfterFi{%
      \BIC@@Dec#2#3!{#1#4}%
    }%
  \BIC@Fi
}
%    \end{macrocode}
%    \end{macro}
%
% \subsection{\op{Add}, \op{Sub}}
%
%    \begin{macro}{\bigintcalcAdd}
%    \begin{macrocode}
\def\bigintcalcAdd#1{%
  \romannumeral0%
  \expandafter\expandafter\expandafter\BIC@Add
  \bigintcalcNum{#1}!%
}
%    \end{macrocode}
%    \end{macro}
%    \begin{macro}{\BIC@Add}
%    \begin{macrocode}
\def\BIC@Add#1!#2{%
  \expandafter\expandafter\expandafter
  \BIC@AddSwitch\bigintcalcNum{#2}!#1!%
}
%    \end{macrocode}
%    \end{macro}
%    \begin{macro}{\bigintcalcSub}
%    \begin{macrocode}
\def\bigintcalcSub#1#2{%
  \romannumeral0%
  \expandafter\expandafter\expandafter\BIC@Add
  \bigintcalcNum{-#2}!{#1}%
}
%    \end{macrocode}
%    \end{macro}
%
%    \begin{macro}{\BIC@AddSwitch}
%    Decision table for \cs{BIC@AddSwitch}.
%    \begin{quote}
%    \begin{tabular}[t]{@{}|l|l|l|l|l|@{}}
%      \hline
%      $x<0$ & $y<0$ & $-x>-y$ & $-$ & $\opAdd(-x,-y)$\\
%      \cline{3-3}\cline{5-5}
%            &       & else    &     & $\opAdd(-y,-x)$\\
%      \cline{2-5}
%            & else  & $-x> y$ & $-$ & $\opSub(-x, y)$\\
%      \cline{3-5}
%            &       & $-x= y$ &     & $0$\\
%      \cline{3-5}
%            &       & else    & $+$ & $\opSub( y,-x)$\\
%      \hline
%      else  & $y<0$ & $ x>-y$ & $+$ & $\opSub( x,-y)$\\
%      \cline{3-5}
%            &       & $ x=-y$ &     & $0$\\
%      \cline{3-5}
%            &       & else    & $-$ & $\opSub(-y, x)$\\
%      \cline{2-5}
%            & else  & $ x> y$ & $+$ & $\opAdd( x, y)$\\
%      \cline{3-3}\cline{5-5}
%            &       & else    &     & $\opAdd( y, x)$\\
%      \hline
%    \end{tabular}
%    \end{quote}
%    \begin{macrocode}
\def\BIC@AddSwitch#1#2!#3#4!{%
  \ifx#1-% x < 0
    \ifx#3-% y < 0
      \expandafter-\romannumeral0%
      \ifnum\BIC@PosCmp#2!#4!=1 % -x > -y
        \BIC@AfterFiFiFi{%
          \BIC@AddXY#2!#4!!!%
        }%
      \else % -x <= -y
        \BIC@AfterFiFiFi{%
          \BIC@AddXY#4!#2!!!%
        }%
      \fi
    \else % y >= 0
      \ifcase\BIC@PosCmp#2!#3#4!% -x = y
        \BIC@AfterFiFiFi{ 0}%
      \or % -x > y
        \expandafter-\romannumeral0%
        \BIC@AfterFiFiFi{%
          \BIC@SubXY#2!#3#4!!!%
        }%
      \else % -x <= y
        \BIC@AfterFiFiFi{%
          \BIC@SubXY#3#4!#2!!!%
        }%
      \fi
    \fi
  \else % x >= 0
    \ifx#3-% y < 0
      \ifcase\BIC@PosCmp#1#2!#4!% x = -y
        \BIC@AfterFiFiFi{ 0}%
      \or % x > -y
        \BIC@AfterFiFiFi{%
          \BIC@SubXY#1#2!#4!!!%
        }%
      \else % x <= -y
        \expandafter-\romannumeral0%
        \BIC@AfterFiFiFi{%
          \BIC@SubXY#4!#1#2!!!%
        }%
      \fi
    \else % y >= 0
      \ifnum\BIC@PosCmp#1#2!#3#4!=1 % x > y
        \BIC@AfterFiFiFi{%
          \BIC@AddXY#1#2!#3#4!!!%
        }%
      \else % x <= y
        \BIC@AfterFiFiFi{%
          \BIC@AddXY#3#4!#1#2!!!%
        }%
      \fi
    \fi
  \BIC@Fi
}
%    \end{macrocode}
%    \end{macro}
%
%    \begin{macro}{\BigIntCalcAdd}
%    \begin{macrocode}
\def\BigIntCalcAdd#1!#2!{%
  \romannumeral0\BIC@AddXY#1!#2!!!%
}
%    \end{macrocode}
%    \end{macro}
%    \begin{macro}{\BigIntCalcSub}
%    \begin{macrocode}
\def\BigIntCalcSub#1!#2!{%
  \romannumeral0\BIC@SubXY#1!#2!!!%
}
%    \end{macrocode}
%    \end{macro}
%
%    \begin{macro}{\BIC@AddXY}
%    \begin{macrocode}
\def\BIC@AddXY#1#2!#3#4!#5!#6!{%
  \ifx\\#2\\%
    \ifx\\#3\\%
      \BIC@AfterFiFi{%
        \BIC@DoAdd0!#1#5!#60!%
      }%
    \else
      \BIC@AfterFiFi{%
        \BIC@DoAdd0!#1#5!#3#6!%
      }%
    \fi
  \else
    \ifx\\#4\\%
      \ifx\\#3\\%
        \BIC@AfterFiFiFi{%
          \BIC@AddXY#2!{}!#1#5!#60!%
        }%
      \else
        \BIC@AfterFiFiFi{%
          \BIC@AddXY#2!{}!#1#5!#3#6!%
        }%
      \fi
    \else
      \BIC@AfterFiFi{%
        \BIC@AddXY#2!#4!#1#5!#3#6!%
      }%
    \fi
  \BIC@Fi
}
%    \end{macrocode}
%    \end{macro}
%    \begin{macro}{\BIC@DoAdd}
%    |#1|: carry\\
%    |#2|: reverted result\\
%    |#3#4|: reverted $x$\\
%    |#5#6|: reverted $y$
%    \begin{macrocode}
\def\BIC@DoAdd#1#2!#3#4!#5#6!{%
  \ifx\\#4\\%
    \BIC@AfterFi{%
&     \expandafter\BIC@Space
&     \the\numexpr#1+#3+#5\relax#2%
$     \expandafter\expandafter\expandafter\BIC@AddResult
$     \BIC@AddDigit#1#3#5#2%
    }%
  \else
    \BIC@AfterFi{%
      \expandafter\expandafter\expandafter\BIC@DoAdd
      \BIC@AddDigit#1#3#5#2!#4!#6!%
    }%
  \BIC@Fi
}
%    \end{macrocode}
%    \end{macro}
%    \begin{macro}{\BIC@AddResult}
%    \begin{macrocode}
$ \def\BIC@AddResult#1{%
$   \ifx#10%
$     \expandafter\BIC@Space
$   \else
$     \expandafter\BIC@Space\expandafter#1%
$   \fi
$ }%
%    \end{macrocode}
%    \end{macro}
%    \begin{macro}{\BIC@AddDigit}
%    |#1|: carry\\
%    |#2|: digit of $x$\\
%    |#3|: digit of $y$
%    \begin{macrocode}
\def\BIC@AddDigit#1#2#3{%
  \romannumeral0%
& \expandafter\BIC@@AddDigit\the\numexpr#1+#2+#3!%
$ \expandafter\BIC@@AddDigit\number%
$ \csname
$   BIC@AddCarry%
$   \ifcase#1 %
$     #2%
$   \else
$     \ifcase#2 1\or2\or3\or4\or5\or6\or7\or8\or9\or10\fi
$   \fi
$ \endcsname#3!%
}
%    \end{macrocode}
%    \end{macro}
%    \begin{macro}{\BIC@@AddDigit}
%    \begin{macrocode}
\def\BIC@@AddDigit#1!{%
  \ifnum#1<10 %
    \BIC@AfterFi{ 0#1}%
  \else
    \BIC@AfterFi{ #1}%
  \BIC@Fi
}
%    \end{macrocode}
%    \end{macro}
%    \begin{macro}{\BIC@AddCarry0}
%    \begin{macrocode}
$ \expandafter\def\csname BIC@AddCarry0\endcsname#1{#1}%
%    \end{macrocode}
%    \end{macro}
%    \begin{macro}{\BIC@AddCarry10}
%    \begin{macrocode}
$ \expandafter\def\csname BIC@AddCarry10\endcsname#1{1#1}%
%    \end{macrocode}
%    \end{macro}
%    \begin{macro}{\BIC@AddCarry[1-9]}
%    \begin{macrocode}
$ \def\BIC@Temp#1#2{%
$   \expandafter\def\csname BIC@AddCarry#1\endcsname##1{%
$     \ifcase##1 #1\or
$     #2%
$?    \else\BigIntCalcError:ThisCannotHappen%
$     \fi
$   }%
$ }%
$ \BIC@Temp 0{1\or2\or3\or4\or5\or6\or7\or8\or9}%
$ \BIC@Temp 1{2\or3\or4\or5\or6\or7\or8\or9\or10}%
$ \BIC@Temp 2{3\or4\or5\or6\or7\or8\or9\or10\or11}%
$ \BIC@Temp 3{4\or5\or6\or7\or8\or9\or10\or11\or12}%
$ \BIC@Temp 4{5\or6\or7\or8\or9\or10\or11\or12\or13}%
$ \BIC@Temp 5{6\or7\or8\or9\or10\or11\or12\or13\or14}%
$ \BIC@Temp 6{7\or8\or9\or10\or11\or12\or13\or14\or15}%
$ \BIC@Temp 7{8\or9\or10\or11\or12\or13\or14\or15\or16}%
$ \BIC@Temp 8{9\or10\or11\or12\or13\or14\or15\or16\or17}%
$ \BIC@Temp 9{10\or11\or12\or13\or14\or15\or16\or17\or18}%
%    \end{macrocode}
%    \end{macro}
%
%    \begin{macro}{\BIC@SubXY}
%    Preconditions:
%    \begin{itemize}
%    \item $x > y$, $x \geq 0$, and $y >= 0$
%    \item digits($x$) = digits($y$)
%    \end{itemize}
%    \begin{macrocode}
\def\BIC@SubXY#1#2!#3#4!#5!#6!{%
  \ifx\\#2\\%
    \ifx\\#3\\%
      \BIC@AfterFiFi{%
        \BIC@DoSub0!#1#5!#60!%
      }%
    \else
      \BIC@AfterFiFi{%
        \BIC@DoSub0!#1#5!#3#6!%
      }%
    \fi
  \else
    \ifx\\#4\\%
      \ifx\\#3\\%
        \BIC@AfterFiFiFi{%
          \BIC@SubXY#2!{}!#1#5!#60!%
        }%
      \else
        \BIC@AfterFiFiFi{%
          \BIC@SubXY#2!{}!#1#5!#3#6!%
        }%
      \fi
    \else
      \BIC@AfterFiFi{%
        \BIC@SubXY#2!#4!#1#5!#3#6!%
      }%
    \fi
  \BIC@Fi
}
%    \end{macrocode}
%    \end{macro}
%    \begin{macro}{\BIC@DoSub}
%    |#1|: carry\\
%    |#2|: reverted result\\
%    |#3#4|: reverted x\\
%    |#5#6|: reverted y
%    \begin{macrocode}
\def\BIC@DoSub#1#2!#3#4!#5#6!{%
  \ifx\\#4\\%
    \BIC@AfterFi{%
      \expandafter\expandafter\expandafter\BIC@SubResult
      \BIC@SubDigit#1#3#5#2%
    }%
  \else
    \BIC@AfterFi{%
      \expandafter\expandafter\expandafter\BIC@DoSub
      \BIC@SubDigit#1#3#5#2!#4!#6!%
    }%
  \BIC@Fi
}
%    \end{macrocode}
%    \end{macro}
%    \begin{macro}{\BIC@SubResult}
%    \begin{macrocode}
\def\BIC@SubResult#1{%
  \ifx#10%
    \expandafter\BIC@SubResult
  \else
    \expandafter\BIC@Space\expandafter#1%
  \fi
}
%    \end{macrocode}
%    \end{macro}
%    \begin{macro}{\BIC@SubDigit}
%    |#1|: carry\\
%    |#2|: digit of $x$\\
%    |#3|: digit of $y$
%    \begin{macrocode}
\def\BIC@SubDigit#1#2#3{%
  \romannumeral0%
& \expandafter\BIC@@SubDigit\the\numexpr#2-#3-#1!%
$ \expandafter\BIC@@AddDigit\number
$   \csname
$     BIC@SubCarry%
$     \ifcase#1 %
$       #3%
$     \else
$       \ifcase#3 1\or2\or3\or4\or5\or6\or7\or8\or9\or10\fi
$     \fi
$   \endcsname#2!%
}
%    \end{macrocode}
%    \end{macro}
%    \begin{macro}{\BIC@@SubDigit}
%    \begin{macrocode}
& \def\BIC@@SubDigit#1!{%
&   \ifnum#1<0 %
&     \BIC@AfterFi{%
&       \expandafter\BIC@Space
&       \expandafter1\the\numexpr#1+10\relax
&     }%
&   \else
&     \BIC@AfterFi{ 0#1}%
&   \BIC@Fi
& }%
%    \end{macrocode}
%    \end{macro}
%    \begin{macro}{\BIC@SubCarry0}
%    \begin{macrocode}
$ \expandafter\def\csname BIC@SubCarry0\endcsname#1{#1}%
%    \end{macrocode}
%    \end{macro}
%    \begin{macro}{\BIC@SubCarry10}%
%    \begin{macrocode}
$ \expandafter\def\csname BIC@SubCarry10\endcsname#1{1#1}%
%    \end{macrocode}
%    \end{macro}
%    \begin{macro}{\BIC@SubCarry[1-9]}
%    \begin{macrocode}
$ \def\BIC@Temp#1#2{%
$   \expandafter\def\csname BIC@SubCarry#1\endcsname##1{%
$     \ifcase##1 #2%
$?    \else\BigIntCalcError:ThisCannotHappen%
$     \fi
$   }%
$ }%
$ \BIC@Temp 1{19\or0\or1\or2\or3\or4\or5\or6\or7\or8}%
$ \BIC@Temp 2{18\or19\or0\or1\or2\or3\or4\or5\or6\or7}%
$ \BIC@Temp 3{17\or18\or19\or0\or1\or2\or3\or4\or5\or6}%
$ \BIC@Temp 4{16\or17\or18\or19\or0\or1\or2\or3\or4\or5}%
$ \BIC@Temp 5{15\or16\or17\or18\or19\or0\or1\or2\or3\or4}%
$ \BIC@Temp 6{14\or15\or16\or17\or18\or19\or0\or1\or2\or3}%
$ \BIC@Temp 7{13\or14\or15\or16\or17\or18\or19\or0\or1\or2}%
$ \BIC@Temp 8{12\or13\or14\or15\or16\or17\or18\or19\or0\or1}%
$ \BIC@Temp 9{11\or12\or13\or14\or15\or16\or17\or18\or19\or0}%
%    \end{macrocode}
%    \end{macro}
%
% \subsection{\op{Shl}, \op{Shr}}
%
%    \begin{macro}{\bigintcalcShl}
%    \begin{macrocode}
\def\bigintcalcShl#1{%
  \romannumeral0%
  \expandafter\expandafter\expandafter\BIC@Shl
  \bigintcalcNum{#1}!%
}
%    \end{macrocode}
%    \end{macro}
%    \begin{macro}{\BIC@Shl}
%    \begin{macrocode}
\def\BIC@Shl#1#2!{%
  \ifx#1-%
    \BIC@AfterFi{%
      \expandafter-\romannumeral0%
&     \BIC@@Shl#2!!%
$     \BIC@AddXY#2!#2!!!%
    }%
  \else
    \BIC@AfterFi{%
&     \BIC@@Shl#1#2!!%
$     \BIC@AddXY#1#2!#1#2!!!%
    }%
  \BIC@Fi
}
%    \end{macrocode}
%    \end{macro}
%    \begin{macro}{\BigIntCalcShl}
%    \begin{macrocode}
\def\BigIntCalcShl#1!{%
  \romannumeral0%
& \BIC@@Shl#1!!%
$ \BIC@AddXY#1!#1!!!%
}
%    \end{macrocode}
%    \end{macro}
%    \begin{macro}{\BIC@@Shl}
%    \begin{macrocode}
& \def\BIC@@Shl#1#2!{%
&   \ifx\\#2\\%
&     \BIC@AfterFi{%
&       \BIC@@@Shl0!#1%
&     }%
&   \else
&     \BIC@AfterFi{%
&       \BIC@@Shl#2!#1%
&     }%
&   \BIC@Fi
& }%
%    \end{macrocode}
%    \end{macro}
%    \begin{macro}{\BIC@@@Shl}
%    |#1|: carry\\
%    |#2|: result\\
%    |#3#4|: reverted number
%    \begin{macrocode}
& \def\BIC@@@Shl#1#2!#3#4!{%
&   \ifx\\#4\\%
&     \BIC@AfterFi{%
&       \expandafter\BIC@Space
&       \the\numexpr#3*2+#1\relax#2%
&     }%
&   \else
&     \BIC@AfterFi{%
&       \expandafter\BIC@@@@Shl\the\numexpr#3*2+#1!#2!#4!%
&     }%
&   \BIC@Fi
& }%
%    \end{macrocode}
%    \end{macro}
%    \begin{macro}{\BIC@@@@Shl}
%    \begin{macrocode}
& \def\BIC@@@@Shl#1!{%
&   \ifnum#1<10 %
&     \BIC@AfterFi{%
&       \BIC@@@Shl0#1%
&     }%
&   \else
&     \BIC@AfterFi{%
&       \BIC@@@Shl#1%
&     }%
&   \BIC@Fi
& }%
%    \end{macrocode}
%    \end{macro}
%
%    \begin{macro}{\bigintcalcShr}
%    \begin{macrocode}
\def\bigintcalcShr#1{%
  \romannumeral0%
  \expandafter\expandafter\expandafter\BIC@Shr
  \bigintcalcNum{#1}!%
}
%    \end{macrocode}
%    \end{macro}
%    \begin{macro}{\BIC@Shr}
%    \begin{macrocode}
\def\BIC@Shr#1#2!{%
  \ifx#1-%
    \expandafter-\romannumeral0%
    \BIC@AfterFi{%
      \BIC@@Shr#2!%
    }%
  \else
    \BIC@AfterFi{%
      \BIC@@Shr#1#2!%
    }%
  \BIC@Fi
}
%    \end{macrocode}
%    \end{macro}
%    \begin{macro}{\BigIntCalcShr}
%    \begin{macrocode}
\def\BigIntCalcShr#1!{%
  \romannumeral0%
  \BIC@@Shr#1!%
}
%    \end{macrocode}
%    \end{macro}
%    \begin{macro}{\BIC@@Shr}
%    \begin{macrocode}
\def\BIC@@Shr#1#2!{%
  \ifcase#1 %
    \BIC@AfterFi{ 0}%
  \or
    \ifx\\#2\\%
      \BIC@AfterFiFi{ 0}%
    \else
      \BIC@AfterFiFi{%
        \BIC@@@Shr#1#2!!%
      }%
    \fi
  \else
    \BIC@AfterFi{%
      \BIC@@@Shr0#1#2!!%
    }%
  \BIC@Fi
}
%    \end{macrocode}
%    \end{macro}
%    \begin{macro}{\BIC@@@Shr}
%    |#1|: carry\\
%    |#2#3|: number\\
%    |#4|: result
%    \begin{macrocode}
\def\BIC@@@Shr#1#2#3!#4!{%
  \ifx\\#3\\%
    \ifodd#1#2 %
      \BIC@AfterFiFi{%
&       \expandafter\BIC@ShrResult\the\numexpr(#1#2-1)/2\relax
$       \expandafter\expandafter\expandafter\BIC@ShrResult
$       \csname BIC@ShrDigit#1#2\endcsname
        #4!%
      }%
    \else
      \BIC@AfterFiFi{%
&       \expandafter\BIC@ShrResult\the\numexpr#1#2/2\relax
$       \expandafter\expandafter\expandafter\BIC@ShrResult
$       \csname BIC@ShrDigit#1#2\endcsname
        #4!%
      }%
    \fi
  \else
    \ifodd#1#2 %
      \BIC@AfterFiFi{%
&       \expandafter\BIC@@@@Shr\the\numexpr(#1#2-1)/2\relax1%
$       \expandafter\expandafter\expandafter\BIC@@@@Shr
$       \csname BIC@ShrDigit#1#2\endcsname
        #3!#4!%
      }%
    \else
      \BIC@AfterFiFi{%
&       \expandafter\BIC@@@@Shr\the\numexpr#1#2/2\relax0%
$       \expandafter\expandafter\expandafter\BIC@@@@Shr
$       \csname BIC@ShrDigit#1#2\endcsname
        #3!#4!%
      }%
    \fi
  \BIC@Fi
}
%    \end{macrocode}
%    \end{macro}
%    \begin{macro}{\BIC@ShrResult}
%    \begin{macrocode}
& \def\BIC@ShrResult#1#2!{ #2#1}%
$ \def\BIC@ShrResult#1#2#3!{ #3#1}%
%    \end{macrocode}
%    \end{macro}
%    \begin{macro}{\BIC@@@@Shr}
%    |#1|: new digit\\
%    |#2|: carry\\
%    |#3|: remaining number\\
%    |#4|: result
%    \begin{macrocode}
\def\BIC@@@@Shr#1#2#3!#4!{%
  \BIC@@@Shr#2#3!#4#1!%
}
%    \end{macrocode}
%    \end{macro}
%    \begin{macro}{\BIC@ShrDigit[00-19]}
%    \begin{macrocode}
$ \def\BIC@Temp#1#2#3#4{%
$   \expandafter\def\csname BIC@ShrDigit#1#2\endcsname{#3#4}%
$ }%
$ \BIC@Temp 0000%
$ \BIC@Temp 0101%
$ \BIC@Temp 0210%
$ \BIC@Temp 0311%
$ \BIC@Temp 0420%
$ \BIC@Temp 0521%
$ \BIC@Temp 0630%
$ \BIC@Temp 0731%
$ \BIC@Temp 0840%
$ \BIC@Temp 0941%
$ \BIC@Temp 1050%
$ \BIC@Temp 1151%
$ \BIC@Temp 1260%
$ \BIC@Temp 1361%
$ \BIC@Temp 1470%
$ \BIC@Temp 1571%
$ \BIC@Temp 1680%
$ \BIC@Temp 1781%
$ \BIC@Temp 1890%
$ \BIC@Temp 1991%
%    \end{macrocode}
%    \end{macro}
%
% \subsection{\cs{BIC@Tim}}
%
%    \begin{macro}{\BIC@Tim}
%    Macro \cs{BIC@Tim} implements ``Number \emph{tim}es digit''.\\
%    |#1|: plain number without sign\\
%    |#2|: digit
\def\BIC@Tim#1!#2{%
  \romannumeral0%
  \ifcase#2 % 0
    \BIC@AfterFi{ 0}%
  \or % 1
    \BIC@AfterFi{ #1}%
  \or % 2
    \BIC@AfterFi{%
      \BIC@Shl#1!%
    }%
  \else % 3-9
    \BIC@AfterFi{%
      \BIC@@Tim#1!!#2%
    }%
  \BIC@Fi
}
%    \begin{macrocode}
%    \end{macrocode}
%    \end{macro}
%    \begin{macro}{\BIC@@Tim}
%    |#1#2|: number\\
%    |#3|: reverted number
%    \begin{macrocode}
\def\BIC@@Tim#1#2!{%
  \ifx\\#2\\%
    \BIC@AfterFi{%
      \BIC@ProcessTim0!#1%
    }%
  \else
    \BIC@AfterFi{%
      \BIC@@Tim#2!#1%
    }%
  \BIC@Fi
}
%    \end{macrocode}
%    \end{macro}
%    \begin{macro}{\BIC@ProcessTim}
%    |#1|: carry\\
%    |#2|: result\\
%    |#3#4|: reverted number\\
%    |#5|: digit
%    \begin{macrocode}
\def\BIC@ProcessTim#1#2!#3#4!#5{%
  \ifx\\#4\\%
    \BIC@AfterFi{%
      \expandafter\BIC@Space
&     \the\numexpr#3*#5+#1\relax
$     \romannumeral0\BIC@TimDigit#3#5#1%
      #2%
    }%
  \else
    \BIC@AfterFi{%
      \expandafter\BIC@@ProcessTim
&     \the\numexpr#3*#5+#1%
$     \romannumeral0\BIC@TimDigit#3#5#1%
      !#2!#4!#5%
    }%
  \BIC@Fi
}
%    \end{macrocode}
%    \end{macro}
%    \begin{macro}{\BIC@@ProcessTim}
%    |#1#2|: carry?, new digit\\
%    |#3|: new number\\
%    |#4|: old number\\
%    |#5|: digit
%    \begin{macrocode}
\def\BIC@@ProcessTim#1#2!{%
  \ifx\\#2\\%
    \BIC@AfterFi{%
      \BIC@ProcessTim0#1%
    }%
  \else
    \BIC@AfterFi{%
      \BIC@ProcessTim#1#2%
    }%
  \BIC@Fi
}
%    \end{macrocode}
%    \end{macro}
%    \begin{macro}{\BIC@TimDigit}
%    |#1|: digit 0--9\\
%    |#2|: digit 3--9\\
%    |#3|: carry 0--9
%    \begin{macrocode}
$ \def\BIC@TimDigit#1#2#3{%
$   \ifcase#1 % 0
$     \BIC@AfterFi{ #3}%
$   \or % 1
$     \BIC@AfterFi{%
$       \expandafter\BIC@Space
$       \number\csname BIC@AddCarry#2\endcsname#3 %
$     }%
$   \else
$     \ifcase#3 %
$       \BIC@AfterFiFi{%
$         \expandafter\BIC@Space
$         \number\csname BIC@MulDigit#2\endcsname#1 %
$       }%
$     \else
$       \BIC@AfterFiFi{%
$         \expandafter\BIC@Space
$         \romannumeral0%
$         \expandafter\BIC@AddXY
$         \number\csname BIC@MulDigit#2\endcsname#1!%
$         #3!!!%
$       }%
$     \fi
$   \BIC@Fi
$ }%
%    \end{macrocode}
%    \end{macro}
%    \begin{macro}{\BIC@MulDigit[3-9]}
%    \begin{macrocode}
$ \def\BIC@Temp#1#2{%
$   \expandafter\def\csname BIC@MulDigit#1\endcsname##1{%
$     \ifcase##1 0%
$     \or ##1%
$     \or #2%
$?    \else\BigIntCalcError:ThisCannotHappen%
$     \fi
$   }%
$ }%
$ \BIC@Temp 3{6\or9\or12\or15\or18\or21\or24\or27}%
$ \BIC@Temp 4{8\or12\or16\or20\or24\or28\or32\or36}%
$ \BIC@Temp 5{10\or15\or20\or25\or30\or35\or40\or45}%
$ \BIC@Temp 6{12\or18\or24\or30\or36\or42\or48\or54}%
$ \BIC@Temp 7{14\or21\or28\or35\or42\or49\or56\or63}%
$ \BIC@Temp 8{16\or24\or32\or40\or48\or56\or64\or72}%
$ \BIC@Temp 9{18\or27\or36\or45\or54\or63\or72\or81}%
%    \end{macrocode}
%    \end{macro}
%
% \subsection{\op{Mul}}
%
%    \begin{macro}{\bigintcalcMul}
%    \begin{macrocode}
\def\bigintcalcMul#1#2{%
  \romannumeral0%
  \expandafter\expandafter\expandafter\BIC@Mul
  \bigintcalcNum{#1}!{#2}%
}
%    \end{macrocode}
%    \end{macro}
%    \begin{macro}{\BIC@Mul}
%    \begin{macrocode}
\def\BIC@Mul#1!#2{%
  \expandafter\expandafter\expandafter\BIC@MulSwitch
  \bigintcalcNum{#2}!#1!%
}
%    \end{macrocode}
%    \end{macro}
%    \begin{macro}{\BIC@MulSwitch}
%    Decision table for \cs{BIC@MulSwitch}.
%    \begin{quote}
%    \begin{tabular}[t]{@{}|l|l|l|l|l|@{}}
%      \hline
%      $x=0$ &\multicolumn{3}{l}{}    &$0$\\
%      \hline
%      $x>0$ & $y=0$ &\multicolumn2l{}&$0$\\
%      \cline{2-5}
%            & $y>0$ & $ x> y$ & $+$ & $\opMul( x, y)$\\
%      \cline{3-3}\cline{5-5}
%            &       & else    &     & $\opMul( y, x)$\\
%      \cline{2-5}
%            & $y<0$ & $ x>-y$ & $-$ & $\opMul( x,-y)$\\
%      \cline{3-3}\cline{5-5}
%            &       & else    &     & $\opMul(-y, x)$\\
%      \hline
%      $x<0$ & $y=0$ &\multicolumn2l{}&$0$\\
%      \cline{2-5}
%            & $y>0$ & $-x> y$ & $-$ & $\opMul(-x, y)$\\
%      \cline{3-3}\cline{5-5}
%            &       & else    &     & $\opMul( y,-x)$\\
%      \cline{2-5}
%            & $y<0$ & $-x>-y$ & $+$ & $\opMul(-x,-y)$\\
%      \cline{3-3}\cline{5-5}
%            &       & else    &     & $\opMul(-y,-x)$\\
%      \hline
%    \end{tabular}
%    \end{quote}
%    \begin{macrocode}
\def\BIC@MulSwitch#1#2!#3#4!{%
  \ifcase\BIC@Sgn#1#2! % x = 0
    \BIC@AfterFi{ 0}%
  \or % x > 0
    \ifcase\BIC@Sgn#3#4! % y = 0
      \BIC@AfterFiFi{ 0}%
    \or % y > 0
      \ifnum\BIC@PosCmp#1#2!#3#4!=1 % x > y
        \BIC@AfterFiFiFi{%
          \BIC@ProcessMul0!#1#2!#3#4!%
        }%
      \else % x <= y
        \BIC@AfterFiFiFi{%
          \BIC@ProcessMul0!#3#4!#1#2!%
        }%
      \fi
    \else % y < 0
      \expandafter-\romannumeral0%
      \ifnum\BIC@PosCmp#1#2!#4!=1 % x > -y
        \BIC@AfterFiFiFi{%
          \BIC@ProcessMul0!#1#2!#4!%
        }%
      \else % x <= -y
        \BIC@AfterFiFiFi{%
          \BIC@ProcessMul0!#4!#1#2!%
        }%
      \fi
    \fi
  \else % x < 0
    \ifcase\BIC@Sgn#3#4! % y = 0
      \BIC@AfterFiFi{ 0}%
    \or % y > 0
      \expandafter-\romannumeral0%
      \ifnum\BIC@PosCmp#2!#3#4!=1 % -x > y
        \BIC@AfterFiFiFi{%
          \BIC@ProcessMul0!#2!#3#4!%
        }%
      \else % -x <= y
        \BIC@AfterFiFiFi{%
          \BIC@ProcessMul0!#3#4!#2!%
        }%
      \fi
    \else % y < 0
      \ifnum\BIC@PosCmp#2!#4!=1 % -x > -y
        \BIC@AfterFiFiFi{%
          \BIC@ProcessMul0!#2!#4!%
        }%
      \else % -x <= -y
        \BIC@AfterFiFiFi{%
          \BIC@ProcessMul0!#4!#2!%
        }%
      \fi
    \fi
  \BIC@Fi
}
%    \end{macrocode}
%    \end{macro}
%    \begin{macro}{\BigIntCalcMul}
%    \begin{macrocode}
\def\BigIntCalcMul#1!#2!{%
  \romannumeral0%
  \BIC@ProcessMul0!#1!#2!%
}
%    \end{macrocode}
%    \end{macro}
%    \begin{macro}{\BIC@ProcessMul}
%    |#1|: result\\
%    |#2|: number $x$\\
%    |#3#4|: number $y$
%    \begin{macrocode}
\def\BIC@ProcessMul#1!#2!#3#4!{%
  \ifx\\#4\\%
    \BIC@AfterFi{%
      \expandafter\expandafter\expandafter\BIC@Space
      \bigintcalcAdd{\BIC@Tim#2!#3}{#10}%
    }%
  \else
    \BIC@AfterFi{%
      \expandafter\expandafter\expandafter\BIC@ProcessMul
      \bigintcalcAdd{\BIC@Tim#2!#3}{#10}!#2!#4!%
    }%
  \BIC@Fi
}
%    \end{macrocode}
%    \end{macro}
%
% \subsection{\op{Sqr}}
%
%    \begin{macro}{\bigintcalcSqr}
%    \begin{macrocode}
\def\bigintcalcSqr#1{%
  \romannumeral0%
  \expandafter\expandafter\expandafter\BIC@Sqr
  \bigintcalcNum{#1}!%
}
%    \end{macrocode}
%    \end{macro}
%    \begin{macro}{\BIC@Sqr}
%    \begin{macrocode}
\def\BIC@Sqr#1{%
   \ifx#1-%
     \expandafter\BIC@@Sqr
   \else
     \expandafter\BIC@@Sqr\expandafter#1%
   \fi
}
%    \end{macrocode}
%    \end{macro}
%    \begin{macro}{\BIC@@Sqr}
%    \begin{macrocode}
\def\BIC@@Sqr#1!{%
  \BIC@ProcessMul0!#1!#1!%
}
%    \end{macrocode}
%    \end{macro}
%
% \subsection{\op{Fac}}
%
%    \begin{macro}{\bigintcalcFac}
%    \begin{macrocode}
\def\bigintcalcFac#1{%
  \romannumeral0%
  \expandafter\expandafter\expandafter\BIC@Fac
  \bigintcalcNum{#1}!%
}
%    \end{macrocode}
%    \end{macro}
%    \begin{macro}{\BIC@Fac}
%    \begin{macrocode}
\def\BIC@Fac#1#2!{%
  \ifx#1-%
    \BIC@AfterFi{ 0\BigIntCalcError:FacNegative}%
  \else
    \ifnum\BIC@PosCmp#1#2!13!<0 %
      \ifcase#1#2 %
         \BIC@AfterFiFiFi{ 1}% 0!
      \or\BIC@AfterFiFiFi{ 1}% 1!
      \or\BIC@AfterFiFiFi{ 2}% 2!
      \or\BIC@AfterFiFiFi{ 6}% 3!
      \or\BIC@AfterFiFiFi{ 24}% 4!
      \or\BIC@AfterFiFiFi{ 120}% 5!
      \or\BIC@AfterFiFiFi{ 720}% 6!
      \or\BIC@AfterFiFiFi{ 5040}% 7!
      \or\BIC@AfterFiFiFi{ 40320}% 8!
      \or\BIC@AfterFiFiFi{ 362880}% 9!
      \or\BIC@AfterFiFiFi{ 3628800}% 10!
      \or\BIC@AfterFiFiFi{ 39916800}% 11!
      \or\BIC@AfterFiFiFi{ 479001600}% 12!
?     \else\BigIntCalcError:ThisCannotHappen%
      \fi
    \else
      \BIC@AfterFiFi{%
        \BIC@ProcessFac#1#2!479001600!%
      }%
    \fi
  \BIC@Fi
}
%    \end{macrocode}
%    \end{macro}
%    \begin{macro}{\BIC@ProcessFac}
%    |#1|: $n$\\
%    |#2|: result
%    \begin{macrocode}
\def\BIC@ProcessFac#1!#2!{%
  \ifnum\BIC@PosCmp#1!12!=0 %
    \BIC@AfterFi{ #2}%
  \else
    \BIC@AfterFi{%
      \expandafter\BIC@@ProcessFac
      \romannumeral0\BIC@ProcessMul0!#2!#1!%
      !#1!%
    }%
  \BIC@Fi
}
%    \end{macrocode}
%    \end{macro}
%    \begin{macro}{\BIC@@ProcessFac}
%    |#1|: result\\
%    |#2|: $n$
%    \begin{macrocode}
\def\BIC@@ProcessFac#1!#2!{%
  \expandafter\BIC@ProcessFac
  \romannumeral0\BIC@Dec#2!{}%
  !#1!%
}
%    \end{macrocode}
%    \end{macro}
%
% \subsection{\op{Pow}}
%
%    \begin{macro}{\bigintcalcPow}
%    |#1|: basis\\
%    |#2|: power
%    \begin{macrocode}
\def\bigintcalcPow#1{%
  \romannumeral0%
  \expandafter\expandafter\expandafter\BIC@Pow
  \bigintcalcNum{#1}!%
}
%    \end{macrocode}
%    \end{macro}
%    \begin{macro}{\BIC@Pow}
%    |#1|: basis\\
%    |#2|: power
%    \begin{macrocode}
\def\BIC@Pow#1!#2{%
  \expandafter\expandafter\expandafter\BIC@PowSwitch
  \bigintcalcNum{#2}!#1!%
}
%    \end{macrocode}
%    \end{macro}
%    \begin{macro}{\BIC@PowSwitch}
%    |#1#2|: power $y$\\
%    |#3#4|: basis $x$\\
%    Decision table for \cs{BIC@PowSwitch}.
%    \begin{quote}
%    \def\M#1{\multicolumn{#1}{l}{}}%
%    \begin{tabular}[t]{@{}|l|l|l|l|@{}}
%    \hline
%    $y=0$ & \M{2}                        & $1$\\
%    \hline
%    $y=1$ & \M{2}                        & $x$\\
%    \hline
%    $y=2$ & $x<0$          & \M{1}       & $\opMul(-x,-x)$\\
%    \cline{2-4}
%          & else           & \M{1}       & $\opMul(x,x)$\\
%    \hline
%    $y<0$ & $x=0$          & \M{1}       & DivisionByZero\\
%    \cline{2-4}
%          & $x=1$          & \M{1}       & $1$\\
%    \cline{2-4}
%          & $x=-1$         & $\opODD(y)$ & $-1$\\
%    \cline{3-4}
%          &                & else        & $1$\\
%    \cline{2-4}
%          & else ($\string|x\string|>1$) & \M{1}       & $0$\\
%    \hline
%    $y>2$ & $x=0$          & \M{1}       & $0$\\
%    \cline{2-4}
%          & $x=1$          & \M{1}       & $1$\\
%    \cline{2-4}
%          & $x=-1$         & $\opODD(y)$ & $-1$\\
%    \cline{3-4}
%          &                & else        & $1$\\
%    \cline{2-4}
%          & $x<-1$ ($x<0$) & $\opODD(y)$ & $-\opPow(-x,y)$\\
%    \cline{3-4}
%          &                & else        & $\opPow(-x,y)$\\
%    \cline{2-4}
%          & else ($x>1$)   & \M{1}       & $\opPow(x,y)$\\
%    \hline
%    \end{tabular}
%    \end{quote}
%    \begin{macrocode}
\def\BIC@PowSwitch#1#2!#3#4!{%
  \ifcase\ifx\\#2\\%
           \ifx#100 % y = 0
           \else\ifx#111 % y = 1
           \else\ifx#122 % y = 2
           \else4 % y > 2
           \fi\fi\fi
         \else
           \ifx#1-3 % y < 0
           \else4 % y > 2
           \fi
         \fi
    \BIC@AfterFi{ 1}% y = 0
  \or % y = 1
    \BIC@AfterFi{ #3#4}%
  \or % y = 2
    \ifx#3-% x < 0
      \BIC@AfterFiFi{%
        \BIC@ProcessMul0!#4!#4!%
      }%
    \else % x >= 0
      \BIC@AfterFiFi{%
        \BIC@ProcessMul0!#3#4!#3#4!%
      }%
    \fi
  \or % y < 0
    \ifcase\ifx\\#4\\%
             \ifx#300 % x = 0
             \else\ifx#311 % x = 1
             \else3 % x > 1
             \fi\fi
           \else
             \ifcase\BIC@MinusOne#3#4! %
               3 % |x| > 1
             \or
               2 % x = -1
?            \else\BigIntCalcError:ThisCannotHappen%
             \fi
           \fi
      \BIC@AfterFiFi{ 0\BigIntCalcError:DivisionByZero}% x = 0
    \or % x = 1
      \BIC@AfterFiFi{ 1}% x = 1
    \or % x = -1
      \ifcase\BIC@ModTwo#2! % even(y)
        \BIC@AfterFiFiFi{ 1}%
      \or % odd(y)
        \BIC@AfterFiFiFi{ -1}%
?     \else\BigIntCalcError:ThisCannotHappen%
      \fi
    \or % |x| > 1
      \BIC@AfterFiFi{ 0}%
?   \else\BigIntCalcError:ThisCannotHappen%
    \fi
  \or % y > 2
    \ifcase\ifx\\#4\\%
             \ifx#300 % x = 0
             \else\ifx#311 % x = 1
             \else4 % x > 1
             \fi\fi
           \else
             \ifx#3-%
               \ifcase\BIC@MinusOne#3#4! %
                 3 % x < -1
               \else
                 2 % x = -1
               \fi
             \else
               4 % x > 1
             \fi
           \fi
      \BIC@AfterFiFi{ 0}% x = 0
    \or % x = 1
      \BIC@AfterFiFi{ 1}% x = 1
    \or % x = -1
      \ifcase\BIC@ModTwo#1#2! % even(y)
        \BIC@AfterFiFiFi{ 1}%
      \or % odd(y)
        \BIC@AfterFiFiFi{ -1}%
?     \else\BigIntCalcError:ThisCannotHappen%
      \fi
    \or % x < -1
      \ifcase\BIC@ModTwo#1#2! % even(y)
        \BIC@AfterFiFiFi{%
          \BIC@PowRec#4!#1#2!1!%
        }%
      \or % odd(y)
        \expandafter-\romannumeral0%
        \BIC@AfterFiFiFi{%
          \BIC@PowRec#4!#1#2!1!%
        }%
?     \else\BigIntCalcError:ThisCannotHappen%
      \fi
    \or % x > 1
      \BIC@AfterFiFi{%
        \BIC@PowRec#3#4!#1#2!1!%
      }%
?   \else\BigIntCalcError:ThisCannotHappen%
    \fi
? \else\BigIntCalcError:ThisCannotHappen%
  \BIC@Fi
}
%    \end{macrocode}
%    \end{macro}
%
% \subsubsection{Help macros}
%
%    \begin{macro}{\BIC@ModTwo}
%    Macro \cs{BIC@ModTwo} expects a number without sign
%    and returns digit |1| or |0| if the number is odd or even.
%    \begin{macrocode}
\def\BIC@ModTwo#1#2!{%
  \ifx\\#2\\%
    \ifodd#1 %
      \BIC@AfterFiFi1%
    \else
      \BIC@AfterFiFi0%
    \fi
  \else
    \BIC@AfterFi{%
      \BIC@ModTwo#2!%
    }%
  \BIC@Fi
}
%    \end{macrocode}
%    \end{macro}
%
%    \begin{macro}{\BIC@MinusOne}
%    Macro \cs{BIC@MinusOne} expects a number and returns
%    digit |1| if the number equals minus one and returns |0| otherwise.
%    \begin{macrocode}
\def\BIC@MinusOne#1#2!{%
  \ifx#1-%
    \BIC@@MinusOne#2!%
  \else
    0%
  \fi
}
%    \end{macrocode}
%    \end{macro}
%    \begin{macro}{\BIC@@MinusOne}
%    \begin{macrocode}
\def\BIC@@MinusOne#1#2!{%
  \ifx#11%
    \ifx\\#2\\%
      1%
    \else
      0%
    \fi
  \else
    0%
  \fi
}
%    \end{macrocode}
%    \end{macro}
%
% \subsubsection{Recursive calculation}
%
%    \begin{macro}{\BIC@PowRec}
%\begin{quote}
%\begin{verbatim}
%Pow(x, y) {
%  PowRec(x, y, 1)
%}
%PowRec(x, y, r) {
%  if y == 1 then
%    return r
%  else
%    ifodd y then
%      return PowRec(x*x, y div 2, r*x) % y div 2 = (y-1)/2
%    else
%      return PowRec(x*x, y div 2, r)
%    fi
%  fi
%}
%\end{verbatim}
%\end{quote}
%    |#1|: $x$ (basis)\\
%    |#2#3|: $y$ (power)\\
%    |#4|: $r$ (result)
%    \begin{macrocode}
\def\BIC@PowRec#1!#2#3!#4!{%
  \ifcase\ifx#21\ifx\\#3\\0 \else1 \fi\else1 \fi % y = 1
    \ifnum\BIC@PosCmp#1!#4!=1 % x > r
      \BIC@AfterFiFi{%
        \BIC@ProcessMul0!#1!#4!%
      }%
    \else
      \BIC@AfterFiFi{%
        \BIC@ProcessMul0!#4!#1!%
      }%
    \fi
  \or
    \ifcase\BIC@ModTwo#2#3! % even(y)
      \BIC@AfterFiFi{%
        \expandafter\BIC@@PowRec\romannumeral0%
        \BIC@@Shr#2#3!%
        !#1!#4!%
      }%
    \or % odd(y)
      \ifnum\BIC@PosCmp#1!#4!=1 % x > r
        \BIC@AfterFiFiFi{%
          \expandafter\BIC@@@PowRec\romannumeral0%
          \BIC@ProcessMul0!#1!#4!%
          !#1!#2#3!%
        }%
      \else
        \BIC@AfterFiFiFi{%
          \expandafter\BIC@@@PowRec\romannumeral0%
          \BIC@ProcessMul0!#1!#4!%
          !#1!#2#3!%
        }%
      \fi
?   \else\BigIntCalcError:ThisCannotHappen%
    \fi
? \else\BigIntCalcError:ThisCannotHappen%
  \BIC@Fi
}
%    \end{macrocode}
%    \end{macro}
%    \begin{macro}{\BIC@@PowRec}
%    |#1|: $y/2$\\
%    |#2|: $x$\\
%    |#3|: new $r$ ($r$ or $r*x$)
%    \begin{macrocode}
\def\BIC@@PowRec#1!#2!#3!{%
  \expandafter\BIC@PowRec\romannumeral0%
  \BIC@ProcessMul0!#2!#2!%
  !#1!#3!%
}
%    \end{macrocode}
%    \end{macro}
%    \begin{macro}{\BIC@@@PowRec}
%    |#1|: $r*x$
%    |#2|: $x$
%    |#3|: $y$
%    \begin{macrocode}
\def\BIC@@@PowRec#1!#2!#3!{%
  \expandafter\BIC@@PowRec\romannumeral0%
  \BIC@@Shr#3!%
  !#2!#1!%
}
%    \end{macrocode}
%    \end{macro}
%
% \subsection{\op{Div}}
%
%    \begin{macro}{\bigintcalcDiv}
%    |#1|: $x$\\
%    |#2|: $y$ (divisor)
%    \begin{macrocode}
\def\bigintcalcDiv#1{%
  \romannumeral0%
  \expandafter\expandafter\expandafter\BIC@Div
  \bigintcalcNum{#1}!%
}
%    \end{macrocode}
%    \end{macro}
%    \begin{macro}{\BIC@Div}
%    |#1|: $x$\\
%    |#2|: $y$
%    \begin{macrocode}
\def\BIC@Div#1!#2{%
  \expandafter\expandafter\expandafter\BIC@DivSwitchSign
  \bigintcalcNum{#2}!#1!%
}
%    \end{macrocode}
%    \end{macro}
%    \begin{macro}{\BigIntCalcDiv}
%    \begin{macrocode}
\def\BigIntCalcDiv#1!#2!{%
  \romannumeral0%
  \BIC@DivSwitchSign#2!#1!%
}
%    \end{macrocode}
%    \end{macro}
%    \begin{macro}{\BIC@DivSwitchSign}
%    Decision table for \cs{BIC@DivSwitchSign}.
%    \begin{quote}
%    \begin{tabular}{@{}|l|l|l|@{}}
%    \hline
%    $y=0$ & \multicolumn{1}{l}{} & DivisionByZero\\
%    \hline
%    $y>0$ & $x=0$ & $0$\\
%    \cline{2-3}
%          & $x>0$ & DivSwitch$(+,x,y)$\\
%    \cline{2-3}
%          & $x<0$ & DivSwitch$(-,-x,y)$\\
%    \hline
%    $y<0$ & $x=0$ & $0$\\
%    \cline{2-3}
%          & $x>0$ & DivSwitch$(-,x,-y)$\\
%    \cline{2-3}
%          & $x<0$ & DivSwitch$(+,-x,-y)$\\
%    \hline
%    \end{tabular}
%    \end{quote}
%    |#1|: $y$ (divisor)\\
%    |#2|: $x$
%    \begin{macrocode}
\def\BIC@DivSwitchSign#1#2!#3#4!{%
  \ifcase\BIC@Sgn#1#2! % y = 0
    \BIC@AfterFi{ 0\BigIntCalcError:DivisionByZero}%
  \or % y > 0
    \ifcase\BIC@Sgn#3#4! % x = 0
      \BIC@AfterFiFi{ 0}%
    \or % x > 0
      \BIC@AfterFiFi{%
        \BIC@DivSwitch{}#3#4!#1#2!%
      }%
    \else % x < 0
      \BIC@AfterFiFi{%
        \BIC@DivSwitch-#4!#1#2!%
      }%
    \fi
  \else % y < 0
    \ifcase\BIC@Sgn#3#4! % x = 0
      \BIC@AfterFiFi{ 0}%
    \or % x > 0
      \BIC@AfterFiFi{%
        \BIC@DivSwitch-#3#4!#2!%
      }%
    \else % x < 0
      \BIC@AfterFiFi{%
        \BIC@DivSwitch{}#4!#2!%
      }%
    \fi
  \BIC@Fi
}
%    \end{macrocode}
%    \end{macro}
%    \begin{macro}{\BIC@DivSwitch}
%    Decision table for \cs{BIC@DivSwitch}.
%    \begin{quote}
%    \begin{tabular}{@{}|l|l|l|@{}}
%    \hline
%    $y=x$ & \multicolumn{1}{l}{} & sign $1$\\
%    \hline
%    $y>x$ & \multicolumn{1}{l}{} & $0$\\
%    \hline
%    $y<x$ & $y=1$                & sign $x$\\
%    \cline{2-3}
%          & $y=2$                & sign Shr$(x)$\\
%    \cline{2-3}
%          & $y=4$                & sign Shr(Shr$(x)$)\\
%    \cline{2-3}
%          & else                 & sign ProcessDiv$(x,y)$\\
%    \hline
%    \end{tabular}
%    \end{quote}
%    |#1|: sign\\
%    |#2|: $x$\\
%    |#3#4|: $y$ ($y\ne 0$)
%    \begin{macrocode}
\def\BIC@DivSwitch#1#2!#3#4!{%
  \ifcase\BIC@PosCmp#3#4!#2!% y = x
    \BIC@AfterFi{ #11}%
  \or % y > x
    \BIC@AfterFi{ 0}%
  \else % y < x
    \ifx\\#1\\%
    \else
      \expandafter-\romannumeral0%
    \fi
    \ifcase\ifx\\#4\\%
             \ifx#310 % y = 1
             \else\ifx#321 % y = 2
             \else\ifx#342 % y = 4
             \else3 % y > 2
             \fi\fi\fi
           \else
             3 % y > 2
           \fi
      \BIC@AfterFiFi{ #2}% y = 1
    \or % y = 2
      \BIC@AfterFiFi{%
        \BIC@@Shr#2!%
      }%
    \or % y = 4
      \BIC@AfterFiFi{%
        \expandafter\BIC@@Shr\romannumeral0%
          \BIC@@Shr#2!!%
      }%
    \or % y > 2
      \BIC@AfterFiFi{%
        \BIC@DivStartX#2!#3#4!!!%
      }%
?   \else\BigIntCalcError:ThisCannotHappen%
    \fi
  \BIC@Fi
}
%    \end{macrocode}
%    \end{macro}
%    \begin{macro}{\BIC@ProcessDiv}
%    |#1#2|: $x$\\
%    |#3#4|: $y$\\
%    |#5|: collect first digits of $x$\\
%    |#6|: corresponding digits of $y$
%    \begin{macrocode}
\def\BIC@DivStartX#1#2!#3#4!#5!#6!{%
  \ifx\\#4\\%
    \BIC@AfterFi{%
      \BIC@DivStartYii#6#3#4!{#5#1}#2=!%
    }%
  \else
    \BIC@AfterFi{%
      \BIC@DivStartX#2!#4!#5#1!#6#3!%
    }%
  \BIC@Fi
}
%    \end{macrocode}
%    \end{macro}
%    \begin{macro}{\BIC@DivStartYii}
%    |#1|: $y$\\
%    |#2|: $x$, |=|
%    \begin{macrocode}
\def\BIC@DivStartYii#1!{%
  \expandafter\BIC@DivStartYiv\romannumeral0%
  \BIC@Shl#1!%
  !#1!%
}
%    \end{macrocode}
%    \end{macro}
%    \begin{macro}{\BIC@DivStartYiv}
%    |#1|: $2y$\\
%    |#2|: $y$\\
%    |#3|: $x$, |=|
%    \begin{macrocode}
\def\BIC@DivStartYiv#1!{%
  \expandafter\BIC@DivStartYvi\romannumeral0%
  \BIC@Shl#1!%
  !#1!%
}
%    \end{macrocode}
%    \end{macro}
%    \begin{macro}{\BIC@DivStartYvi}
%    |#1|: $4y$\\
%    |#2|: $2y$\\
%    |#3|: $y$\\
%    |#4|: $x$, |=|
%    \begin{macrocode}
\def\BIC@DivStartYvi#1!#2!{%
  \expandafter\BIC@DivStartYviii\romannumeral0%
  \BIC@AddXY#1!#2!!!%
  !#1!#2!%
}
%    \end{macrocode}
%    \end{macro}
%    \begin{macro}{\BIC@DivStartYviii}
%    |#1|: $6y$\\
%    |#2|: $4y$\\
%    |#3|: $2y$\\
%    |#4|: $y$\\
%    |#5|: $x$, |=|
%    \begin{macrocode}
\def\BIC@DivStartYviii#1!#2!{%
  \expandafter\BIC@DivStart\romannumeral0%
  \BIC@Shl#2!%
  !#1!#2!%
}
%    \end{macrocode}
%    \end{macro}
%    \begin{macro}{\BIC@DivStart}
%    |#1|: $8y$\\
%    |#2|: $6y$\\
%    |#3|: $4y$\\
%    |#4|: $2y$\\
%    |#5|: $y$\\
%    |#6|: $x$, |=|
%    \begin{macrocode}
\def\BIC@DivStart#1!#2!#3!#4!#5!#6!{%
  \BIC@ProcessDiv#6!!#5!#4!#3!#2!#1!=%
}
%    \end{macrocode}
%    \end{macro}
%    \begin{macro}{\BIC@ProcessDiv}
%    |#1#2#3|: $x$, |=|\\
%    |#4|: result\\
%    |#5|: $y$\\
%    |#6|: $2y$\\
%    |#7|: $4y$\\
%    |#8|: $6y$\\
%    |#9|: $8y$
%    \begin{macrocode}
\def\BIC@ProcessDiv#1#2#3!#4!#5!{%
  \ifcase\BIC@PosCmp#5!#1!% y = #1
    \ifx#2=%
      \BIC@AfterFiFi{\BIC@DivCleanup{#41}}%
    \else
      \BIC@AfterFiFi{%
        \BIC@ProcessDiv#2#3!#41!#5!%
      }%
    \fi
  \or % y > #1
    \ifx#2=%
      \BIC@AfterFiFi{\BIC@DivCleanup{#40}}%
    \else
      \ifx\\#4\\%
        \BIC@AfterFiFiFi{%
          \BIC@ProcessDiv{#1#2}#3!!#5!%
        }%
      \else
        \BIC@AfterFiFiFi{%
          \BIC@ProcessDiv{#1#2}#3!#40!#5!%
        }%
      \fi
    \fi
  \else % y < #1
    \BIC@AfterFi{%
      \BIC@@ProcessDiv{#1}#2#3!#4!#5!%
    }%
  \BIC@Fi
}
%    \end{macrocode}
%    \end{macro}
%    \begin{macro}{\BIC@DivCleanup}
%    |#1|: result\\
%    |#2|: garbage
%    \begin{macrocode}
\def\BIC@DivCleanup#1#2={ #1}%
%    \end{macrocode}
%    \end{macro}
%    \begin{macro}{\BIC@@ProcessDiv}
%    \begin{macrocode}
\def\BIC@@ProcessDiv#1#2#3!#4!#5!#6!#7!{%
  \ifcase\BIC@PosCmp#7!#1!% 4y = #1
    \ifx#2=%
      \BIC@AfterFiFi{\BIC@DivCleanup{#44}}%
    \else
     \BIC@AfterFiFi{%
       \BIC@ProcessDiv#2#3!#44!#5!#6!#7!%
     }%
    \fi
  \or % 4y > #1
    \ifcase\BIC@PosCmp#6!#1!% 2y = #1
      \ifx#2=%
        \BIC@AfterFiFiFi{\BIC@DivCleanup{#42}}%
      \else
        \BIC@AfterFiFiFi{%
          \BIC@ProcessDiv#2#3!#42!#5!#6!#7!%
        }%
      \fi
    \or % 2y > #1
      \ifx#2=%
        \BIC@AfterFiFiFi{\BIC@DivCleanup{#41}}%
      \else
        \BIC@AfterFiFiFi{%
          \BIC@DivSub#1!#5!#2#3!#41!#5!#6!#7!%
        }%
      \fi
    \else % 2y < #1
      \BIC@AfterFiFi{%
        \expandafter\BIC@ProcessDivII\romannumeral0%
        \BIC@SubXY#1!#6!!!%
        !#2#3!#4!#5!23%
        #6!#7!%
      }%
    \fi
  \else % 4y < #1
    \BIC@AfterFi{%
      \BIC@@@ProcessDiv{#1}#2#3!#4!#5!#6!#7!%
    }%
  \BIC@Fi
}
%    \end{macrocode}
%    \end{macro}
%    \begin{macro}{\BIC@DivSub}
%    Next token group: |#1|-|#2| and next digit |#3|.
%    \begin{macrocode}
\def\BIC@DivSub#1!#2!#3{%
  \expandafter\BIC@ProcessDiv\expandafter{%
    \romannumeral0%
    \BIC@SubXY#1!#2!!!%
    #3%
  }%
}
%    \end{macrocode}
%    \end{macro}
%    \begin{macro}{\BIC@ProcessDivII}
%    |#1|: $x'-2y$\\
%    |#2#3|: remaining $x$, |=|\\
%    |#4|: result\\
%    |#5|: $y$\\
%    |#6|: first possible result digit\\
%    |#7|: second possible result digit
%    \begin{macrocode}
\def\BIC@ProcessDivII#1!#2#3!#4!#5!#6#7{%
  \ifcase\BIC@PosCmp#5!#1!% y = #1
    \ifx#2=%
      \BIC@AfterFiFi{\BIC@DivCleanup{#4#7}}%
    \else
      \BIC@AfterFiFi{%
        \BIC@ProcessDiv#2#3!#4#7!#5!%
      }%
    \fi
  \or % y > #1
    \ifx#2=%
      \BIC@AfterFiFi{\BIC@DivCleanup{#4#6}}%
    \else
      \BIC@AfterFiFi{%
        \BIC@ProcessDiv{#1#2}#3!#4#6!#5!%
      }%
    \fi
  \else % y < #1
    \ifx#2=%
      \BIC@AfterFiFi{\BIC@DivCleanup{#4#7}}%
    \else
      \BIC@AfterFiFi{%
        \BIC@DivSub#1!#5!#2#3!#4#7!#5!%
      }%
    \fi
  \BIC@Fi
}
%    \end{macrocode}
%    \end{macro}
%    \begin{macro}{\BIC@ProcessDivIV}
%    |#1#2#3|: $x$, |=|, $x>4y$\\
%    |#4|: result\\
%    |#5|: $y$\\
%    |#6|: $2y$\\
%    |#7|: $4y$\\
%    |#8|: $6y$\\
%    |#9|: $8y$
%    \begin{macrocode}
\def\BIC@@@ProcessDiv#1#2#3!#4!#5!#6!#7!#8!#9!{%
  \ifcase\BIC@PosCmp#8!#1!% 6y = #1
    \ifx#2=%
      \BIC@AfterFiFi{\BIC@DivCleanup{#46}}%
    \else
      \BIC@AfterFiFi{%
        \BIC@ProcessDiv#2#3!#46!#5!#6!#7!#8!#9!%
      }%
    \fi
  \or % 6y > #1
    \BIC@AfterFi{%
      \expandafter\BIC@ProcessDivII\romannumeral0%
      \BIC@SubXY#1!#7!!!%
      !#2#3!#4!#5!45%
      #6!#7!#8!#9!%
    }%
  \else % 6y < #1
    \ifcase\BIC@PosCmp#9!#1!% 8y = #1
      \ifx#2=%
        \BIC@AfterFiFiFi{\BIC@DivCleanup{#48}}%
      \else
        \BIC@AfterFiFiFi{%
          \BIC@ProcessDiv#2#3!#48!#5!#6!#7!#8!#9!%
        }%
      \fi
    \or % 8y > #1
      \BIC@AfterFiFi{%
        \expandafter\BIC@ProcessDivII\romannumeral0%
        \BIC@SubXY#1!#8!!!%
        !#2#3!#4!#5!67%
        #6!#7!#8!#9!%
      }%
    \else % 8y < #1
      \BIC@AfterFiFi{%
        \expandafter\BIC@ProcessDivII\romannumeral0%
        \BIC@SubXY#1!#9!!!%
        !#2#3!#4!#5!89%
        #6!#7!#8!#9!%
      }%
    \fi
  \BIC@Fi
}
%    \end{macrocode}
%    \end{macro}
%
% \subsection{\op{Mod}}
%
%    \begin{macro}{\bigintcalcMod}
%    |#1|: $x$\\
%    |#2|: $y$
%    \begin{macrocode}
\def\bigintcalcMod#1{%
  \romannumeral0%
  \expandafter\expandafter\expandafter\BIC@Mod
  \bigintcalcNum{#1}!%
}
%    \end{macrocode}
%    \end{macro}
%    \begin{macro}{\BIC@Mod}
%    |#1|: $x$\\
%    |#2|: $y$
%    \begin{macrocode}
\def\BIC@Mod#1!#2{%
  \expandafter\expandafter\expandafter\BIC@ModSwitchSign
  \bigintcalcNum{#2}!#1!%
}
%    \end{macrocode}
%    \end{macro}
%    \begin{macro}{\BigIntCalcMod}
%    \begin{macrocode}
\def\BigIntCalcMod#1!#2!{%
  \romannumeral0%
  \BIC@ModSwitchSign#2!#1!%
}
%    \end{macrocode}
%    \end{macro}
%    \begin{macro}{\BIC@ModSwitchSign}
%    Decision table for \cs{BIC@ModSwitchSign}.
%    \begin{quote}
%    \begin{tabular}{@{}|l|l|l|@{}}
%    \hline
%    $y=0$ & \multicolumn{1}{l}{} & DivisionByZero\\
%    \hline
%    $y>0$ & $x=0$ & $0$\\
%    \cline{2-3}
%          & else  & ModSwitch$(+,x,y)$\\
%    \hline
%    $y<0$ & \multicolumn{1}{l}{} & ModSwitch$(-,-x,-y)$\\
%    \hline
%    \end{tabular}
%    \end{quote}
%    |#1#2|: $y$\\
%    |#3#4|: $x$
%    \begin{macrocode}
\def\BIC@ModSwitchSign#1#2!#3#4!{%
  \ifcase\ifx\\#2\\%
           \ifx#100 % y = 0
           \else1 % y > 0
           \fi
          \else
            \ifx#1-2 % y < 0
            \else1 % y > 0
            \fi
          \fi
    \BIC@AfterFi{ 0\BigIntCalcError:DivisionByZero}%
  \or % y > 0
    \ifcase\ifx\\#4\\\ifx#300 \else1 \fi\else1 \fi % x = 0
      \BIC@AfterFiFi{ 0}%
    \else
      \BIC@AfterFiFi{%
        \BIC@ModSwitch{}#3#4!#1#2!%
      }%
    \fi
  \else % y < 0
    \ifcase\ifx\\#4\\%
             \ifx#300 % x = 0
             \else1 % x > 0
             \fi
           \else
             \ifx#3-2 % x < 0
             \else1 % x > 0
             \fi
           \fi
      \BIC@AfterFiFi{ 0}%
    \or % x > 0
      \BIC@AfterFiFi{%
        \BIC@ModSwitch--#3#4!#2!%
      }%
    \else % x < 0
      \BIC@AfterFiFi{%
        \BIC@ModSwitch-#4!#2!%
      }%
    \fi
  \BIC@Fi
}
%    \end{macrocode}
%    \end{macro}
%    \begin{macro}{\BIC@ModSwitch}
%    Decision table for \cs{BIC@ModSwitch}.
%    \begin{quote}
%    \begin{tabular}{@{}|l|l|l|@{}}
%    \hline
%    $y=1$ & \multicolumn{1}{l}{} & $0$\\
%    \hline
%    $y=2$ & ifodd$(x)$ & sign $1$\\
%    \cline{2-3}
%          & else       & $0$\\
%    \hline
%    $y>2$ & $x<0$      &
%        $z\leftarrow x-(x/y)*y;\quad (z<0)\mathbin{?}z+y\mathbin{:}z$\\
%    \cline{2-3}
%          & $x>0$      & $x - (x/y) * y$\\
%    \hline
%    \end{tabular}
%    \end{quote}
%    |#1|: sign\\
%    |#2#3|: $x$\\
%    |#4#5|: $y$
%    \begin{macrocode}
\def\BIC@ModSwitch#1#2#3!#4#5!{%
  \ifcase\ifx\\#5\\%
           \ifx#410 % y = 1
           \else\ifx#421 % y = 2
           \else2 % y > 2
           \fi\fi
         \else2 % y > 2
         \fi
    \BIC@AfterFi{ 0}% y = 1
  \or % y = 2
    \ifcase\BIC@ModTwo#2#3! % even(x)
      \BIC@AfterFiFi{ 0}%
    \or % odd(x)
      \BIC@AfterFiFi{ #11}%
?   \else\BigIntCalcError:ThisCannotHappen%
    \fi
  \or % y > 2
    \ifx\\#1\\%
    \else
      \expandafter\BIC@Space\romannumeral0%
      \expandafter\BIC@ModMinus\romannumeral0%
    \fi
    \ifx#2-% x < 0
      \BIC@AfterFiFi{%
        \expandafter\expandafter\expandafter\BIC@ModX
        \bigintcalcSub{#2#3}{%
          \bigintcalcMul{#4#5}{\bigintcalcDiv{#2#3}{#4#5}}%
        }!#4#5!%
      }%
    \else % x > 0
      \BIC@AfterFiFi{%
        \expandafter\expandafter\expandafter\BIC@Space
        \bigintcalcSub{#2#3}{%
          \bigintcalcMul{#4#5}{\bigintcalcDiv{#2#3}{#4#5}}%
        }%
      }%
    \fi
? \else\BigIntCalcError:ThisCannotHappen%
  \BIC@Fi
}
%    \end{macrocode}
%    \end{macro}
%    \begin{macro}{\BIC@ModMinus}
%    \begin{macrocode}
\def\BIC@ModMinus#1{%
  \ifx#10%
    \BIC@AfterFi{ 0}%
  \else
    \BIC@AfterFi{ -#1}%
  \BIC@Fi
}
%    \end{macrocode}
%    \end{macro}
%    \begin{macro}{\BIC@ModX}
%    |#1#2|: $z$\\
%    |#3|: $x$
%    \begin{macrocode}
\def\BIC@ModX#1#2!#3!{%
  \ifx#1-% z < 0
    \BIC@AfterFi{%
      \expandafter\BIC@Space\romannumeral0%
      \BIC@SubXY#3!#2!!!%
    }%
  \else % z >= 0
    \BIC@AfterFi{ #1#2}%
  \BIC@Fi
}
%    \end{macrocode}
%    \end{macro}
%
%    \begin{macrocode}
\BIC@AtEnd%
%    \end{macrocode}
%
%    \begin{macrocode}
%</package>
%    \end{macrocode}
%
% \section{Test}
%
% \subsection{Catcode checks for loading}
%
%    \begin{macrocode}
%<*test1>
%    \end{macrocode}
%    \begin{macrocode}
\catcode`\{=1 %
\catcode`\}=2 %
\catcode`\#=6 %
\catcode`\@=11 %
\expandafter\ifx\csname count@\endcsname\relax
  \countdef\count@=255 %
\fi
\expandafter\ifx\csname @gobble\endcsname\relax
  \long\def\@gobble#1{}%
\fi
\expandafter\ifx\csname @firstofone\endcsname\relax
  \long\def\@firstofone#1{#1}%
\fi
\expandafter\ifx\csname loop\endcsname\relax
  \expandafter\@firstofone
\else
  \expandafter\@gobble
\fi
{%
  \def\loop#1\repeat{%
    \def\body{#1}%
    \iterate
  }%
  \def\iterate{%
    \body
      \let\next\iterate
    \else
      \let\next\relax
    \fi
    \next
  }%
  \let\repeat=\fi
}%
\def\RestoreCatcodes{}
\count@=0 %
\loop
  \edef\RestoreCatcodes{%
    \RestoreCatcodes
    \catcode\the\count@=\the\catcode\count@\relax
  }%
\ifnum\count@<255 %
  \advance\count@ 1 %
\repeat

\def\RangeCatcodeInvalid#1#2{%
  \count@=#1\relax
  \loop
    \catcode\count@=15 %
  \ifnum\count@<#2\relax
    \advance\count@ 1 %
  \repeat
}
\def\RangeCatcodeCheck#1#2#3{%
  \count@=#1\relax
  \loop
    \ifnum#3=\catcode\count@
    \else
      \errmessage{%
        Character \the\count@\space
        with wrong catcode \the\catcode\count@\space
        instead of \number#3%
      }%
    \fi
  \ifnum\count@<#2\relax
    \advance\count@ 1 %
  \repeat
}
\def\space{ }
\expandafter\ifx\csname LoadCommand\endcsname\relax
  \def\LoadCommand{\input bigintcalc.sty\relax}%
\fi
\def\Test{%
  \RangeCatcodeInvalid{0}{47}%
  \RangeCatcodeInvalid{58}{64}%
  \RangeCatcodeInvalid{91}{96}%
  \RangeCatcodeInvalid{123}{255}%
  \catcode`\@=12 %
  \catcode`\\=0 %
  \catcode`\%=14 %
  \LoadCommand
  \RangeCatcodeCheck{0}{36}{15}%
  \RangeCatcodeCheck{37}{37}{14}%
  \RangeCatcodeCheck{38}{47}{15}%
  \RangeCatcodeCheck{48}{57}{12}%
  \RangeCatcodeCheck{58}{63}{15}%
  \RangeCatcodeCheck{64}{64}{12}%
  \RangeCatcodeCheck{65}{90}{11}%
  \RangeCatcodeCheck{91}{91}{15}%
  \RangeCatcodeCheck{92}{92}{0}%
  \RangeCatcodeCheck{93}{96}{15}%
  \RangeCatcodeCheck{97}{122}{11}%
  \RangeCatcodeCheck{123}{255}{15}%
  \RestoreCatcodes
}
\Test
\csname @@end\endcsname
\end
%    \end{macrocode}
%    \begin{macrocode}
%</test1>
%    \end{macrocode}
%
% \subsection{Macro tests}
%
% \subsubsection{Preamble with test macro definitions}
%
%    \begin{macrocode}
%<*test2>
\NeedsTeXFormat{LaTeX2e}
\nofiles
\documentclass{article}
%<noetex>\let\SavedNumexpr\numexpr
%<noetex>\let\numexpr\UNDEFINED
\makeatletter
\chardef\BIC@TestMode=1 %
\makeatother
\usepackage{bigintcalc}[2016/05/16]
%<noetex>\let\numexpr\SavedNumexpr
\usepackage{qstest}
\IncludeTests{*}
\LogTests{log}{*}{*}
\newcommand*{\TestSpaceAtEnd}[1]{%
%<noetex>  \let\SavedNumexpr\numexpr
%<noetex>  \let\numexpr\UNDEFINED
  \edef\resultA{#1}%
  \edef\resultB{#1 }%
%<noetex>  \let\numexpr\SavedNumexpr
  \Expect*{\resultA\space}*{\resultB}%
}
\newcommand*{\TestResult}[2]{%
%<noetex>  \let\SavedNumexpr\numexpr
%<noetex>  \let\numexpr\UNDEFINED
  \edef\result{#1}%
%<noetex>  \let\numexpr\SavedNumexpr
  \Expect*{\result}{#2}%
}
\newcommand*{\TestResultTwoExpansions}[2]{%
%<*noetex>
  \begingroup
    \let\numexpr\UNDEFINED
    \expandafter\expandafter\expandafter
  \endgroup
%</noetex>
  \expandafter\expandafter\expandafter\Expect
  \expandafter\expandafter\expandafter{#1}{#2}%
}
\newcount\TestCount
%<etex>\newcommand*{\TestArg}[1]{\numexpr#1\relax}
%<noetex>\newcommand*{\TestArg}[1]{#1}
\newcommand*{\TestTeXDivide}[2]{%
  \TestCount=\TestArg{#1}\relax
  \divide\TestCount by \TestArg{#2}\relax
  \Expect*{\bigintcalcDiv{#1}{#2}}*{\the\TestCount}%
}
\newcommand*{\Test}[2]{%
  \TestResult{#1}{#2}%
  \TestResultTwoExpansions{#1}{#2}%
  \TestSpaceAtEnd{#1}%
}
\newcommand*{\TestExch}[2]{\Test{#2}{#1}}
\newcommand*{\TestInv}[2]{%
  \Test{\bigintcalcInv{#1}}{#2}%
}
\newcommand*{\TestAbs}[2]{%
  \Test{\bigintcalcAbs{#1}}{#2}%
}
\newcommand*{\TestSgn}[2]{%
  \Test{\bigintcalcSgn{#1}}{#2}%
}
\newcommand*{\TestMin}[3]{%
  \Test{\bigintcalcMin{#1}{#2}}{#3}%
}
\newcommand*{\TestMax}[3]{%
  \Test{\bigintcalcMax{#1}{#2}}{#3}%
}
\newcommand*{\TestCmp}[3]{%
  \Test{\bigintcalcCmp{#1}{#2}}{#3}%
}
\newcommand*{\TestOdd}[2]{%
  \Test{\bigintcalcOdd{#1}}{#2}%
  \edef\x{%
    \noexpand\Test{%
      \noexpand\BigIntCalcOdd
      \bigintcalcAbs{#1}!%
    }{#2}%
  }%
  \x
}
\newcommand*{\TestInc}[2]{%
  \Test{\bigintcalcInc{#1}}{#2}%
  \ifnum\bigintcalcSgn{#1}>-1 %
    \edef\x{%
      \noexpand\Test{%
        \noexpand\BigIntCalcInc\bigintcalcNum{#1}!%
      }{#2}%
    }%
    \x
  \fi
}
\newcommand*{\TestDec}[2]{%
  \Test{\bigintcalcDec{#1}}{#2}%
  \ifnum\bigintcalcSgn{#1}>0 %
    \edef\x{%
      \noexpand\Test{%
        \noexpand\BigIntCalcDec\bigintcalcNum{#1}!%
      }{#2}%
    }%
    \x
  \fi
}
\newcommand*{\TestAdd}[3]{%
  \Test{\bigintcalcAdd{#1}{#2}}{#3}%
  \ifnum\bigintcalcSgn{#1}>0 %
    \ifnum\bigintcalcSgn{#2}> 0 %
      \ifnum\bigintcalcCmp{#1}{#2}>0 %
        \edef\x{%
          \noexpand\Test{%
            \noexpand\BigIntCalcAdd
            \bigintcalcNum{#1}!\bigintcalcNum{#2}!%
          }{#3}%
        }%
        \x
      \else
        \edef\x{%
          \noexpand\Test{%
            \noexpand\BigIntCalcAdd
            \bigintcalcNum{#2}!\bigintcalcNum{#1}!%
          }{#3}%
        }%
        \x
      \fi
    \fi
  \fi
}
\newcommand*{\TestSub}[3]{%
  \Test{\bigintcalcSub{#1}{#2}}{#3}%
  \ifnum\bigintcalcSgn{#1}>0 %
    \ifnum\bigintcalcSgn{#2}> 0 %
      \ifnum\bigintcalcCmp{#1}{#2}>0 %
        \edef\x{%
          \noexpand\Test{%
            \noexpand\BigIntCalcSub
            \bigintcalcNum{#1}!\bigintcalcNum{#2}!%
          }{#3}%
        }%
        \x
      \fi
    \fi
  \fi
}
\newcommand*{\TestShl}[2]{%
  \Test{\bigintcalcShl{#1}}{#2}%
  \edef\x{%
    \noexpand\Test{%
      \noexpand\BigIntCalcShl\bigintcalcAbs{#1}!%
    }{\bigintcalcAbs{#2}}%
  }%
  \x
}
\newcommand*{\TestShr}[2]{%
  \Test{\bigintcalcShr{#1}}{#2}%
  \edef\x{%
    \noexpand\Test{%
      \noexpand\BigIntCalcShr\bigintcalcAbs{#1}!%
    }{\bigintcalcAbs{#2}}%
  }%
  \x
}
\newcommand*{\TestMul}[3]{%
  \Test{\bigintcalcMul{#1}{#2}}{#3}%
  \edef\x{%
    \noexpand\Test{%
      \noexpand\BigIntCalcMul
      \bigintcalcAbs{#1}!\bigintcalcAbs{#2}!%
    }{\bigintcalcAbs{#3}}%
  }%
  \x
}
\newcommand*{\TestSqr}[2]{%
  \Test{\bigintcalcSqr{#1}}{#2}%
}
\newcommand*{\TestFac}[2]{%
  \expandafter\TestExch\expandafter{%
    \the\numexpr#2%
  }{\bigintcalcFac{#1}}%
}
\newcommand*{\TestFacBig}[2]{%
  \Test{\bigintcalcFac{#1}}{#2}%
}
\newcommand*{\TestPow}[3]{%
  \Test{\bigintcalcPow{#1}{#2}}{#3}%
}
\newcommand*{\TestDiv}[3]{%
  \Test{\bigintcalcDiv{#1}{#2}}{#3}%
  \TestTeXDivide{#1}{#2}%
}
\newcommand*{\TestDivBig}[3]{%
  \Test{\bigintcalcDiv{#1}{#2}}{#3}%
  \edef\x{%
    \noexpand\Test{%
      \noexpand\BigIntCalcDiv\bigintcalcAbs{#1}!\bigintcalcAbs{#2}!%
    }{\bigintcalcAbs{#3}}%
  }%
}
\newcommand*{\TestMod}[3]{%
  \Test{\bigintcalcMod{#1}{#2}}{#3}%
  \ifcase\ifcase\bigintcalcSgn{#1} 0%
         \or
           \ifcase\bigintcalcSgn{#2} 1%
           \or 0%
           \else 1%
           \fi
         \else
           \ifcase\bigintcalcSgn{#2} 1%
           \or 1%
           \else 0%
           \fi
         \fi\relax
    \edef\x{%
      \noexpand\Test{%
        \noexpand\BigIntCalcMod
        \bigintcalcAbs{#1}!\bigintcalcAbs{#2}!%
      }{\bigintcalcAbs{#3}}%
    }%
    \x
  \fi
}
%    \end{macrocode}
%
% \subsubsection{Time}
%
%    \begin{macrocode}
\begingroup\expandafter\expandafter\expandafter\endgroup
\expandafter\ifx\csname pdfresettimer\endcsname\relax
\else
  \makeatletter
  \newcount\SummaryTime
  \newcount\TestTime
  \SummaryTime=\z@
  \newcommand*{\PrintTime}[2]{%
    \typeout{%
      [Time #1: \strip@pt\dimexpr\number#2sp\relax\space s]%
    }%
  }%
  \newcommand*{\StartTime}[1]{%
    \renewcommand*{\TimeDescription}{#1}%
    \pdfresettimer
  }%
  \newcommand*{\TimeDescription}{}%
  \newcommand*{\StopTime}{%
    \TestTime=\pdfelapsedtime
    \global\advance\SummaryTime\TestTime
    \PrintTime\TimeDescription\TestTime
  }%
  \let\saved@qstest\qstest
  \let\saved@endqstest\endqstest
  \def\qstest#1#2{%
    \saved@qstest{#1}{#2}%
    \StartTime{#1}%
  }%
  \def\endqstest{%
    \StopTime
    \saved@endqstest
  }%
  \AtEndDocument{%
    \PrintTime{summary}\SummaryTime
  }%
  \makeatother
\fi
%    \end{macrocode}
%
% \subsubsection{Test sets}
%
%    \begin{macrocode}
\makeatletter

\begin{qstest}{inv}{inv}%
  \TestInv{0}{0}%
  \TestInv{1}{-1}%
  \TestInv{-1}{1}%
  \TestInv{10}{-10}%
  \TestInv{-10}{10}%
  \TestInv{2147483647}{-2147483647}%
  \TestInv{-2147483647}{2147483647}%
  \TestInv{12345678901234567890}{-12345678901234567890}%
  \TestInv{-12345678901234567890}{12345678901234567890}%
  \TestInv{ 0 }{0}%
  \TestInv{ 1 }{-1}%
  \TestInv{--1}{-1}%
  \TestInv{\number\z@}{0}%
  \TestInv{\ifx\relax\relax1\fi}{-1}%
  \TestInv{\ifx\relax\relax-\fi\ifx234\else1\fi}{1}%
\end{qstest}

\begin{qstest}{abs}{abs}%
  \TestAbs{0}{0}%
  \TestAbs{1}{1}%
  \TestAbs{-1}{1}%
  \TestAbs{10}{10}%
  \TestAbs{-10}{10}%
  \TestAbs{2147483647}{2147483647}%
  \TestAbs{-2147483647}{2147483647}%
  \TestAbs{12345678901234567890}{12345678901234567890}%
  \TestAbs{-12345678901234567890}{12345678901234567890}%
  \TestAbs{ 0 }{0}%
  \TestAbs{ 1 }{1}%
  \TestAbs{--1}{1}%
  \TestAbs{-+-+1}{1}%
  \TestAbs{00000000000}{0}%
  \TestAbs{00000001000}{1000}%
  \TestAbs{\ifx\relax\relax 0\else 1\fi}{0}%
\end{qstest}

\begin{qstest}{sign}{sign}%
  \TestSgn{0}{0}%
  \TestSgn{1}{1}%
  \TestSgn{-1}{-1}%
  \TestSgn{10}{1}%
  \TestSgn{-10}{-1}%
  \TestSgn{2147483647}{1}%
  \TestSgn{-2147483647}{-1}%
  \TestSgn{12345678901234567890}{1}%
  \TestSgn{-12345678901234567890}{-1}%
  \TestSgn{ 0 }{0}%
  \TestSgn{ 2 }{1}%
  \TestSgn{ -2 }{-1}%
  \TestSgn{--2}{1}%
  \TestSgn{\number\z@}{0}%
  \TestSgn{\number\@ne}{1}%
  \TestSgn{\number\m@ne}{-1}%
  \TestSgn{%
    -+-+\number\z@\number\z@
    \iftrue1\fi\iftrue2\fi\iftrue3\fi
  }{1}%
\end{qstest}

\begin{qstest}{min}{min}%
  \TestMin{0}{1}{0}%
  \TestMin{1}{0}{0}%
  \TestMin{-10}{-20}{-20}%
  \TestMin{ 1 }{ 2 }{1}%
  \TestMin{ 2 }{ 1 }{1}%
  \TestMin{1}{1}{1}%
  \TestMin{\number\z@}{\number\@ne}{0}%
  \TestMin{\number\@ne}{\number\m@ne}{-1}%
\end{qstest}

\begin{qstest}{max}{max}%
  \TestMax{0}{1}{1}%
  \TestMax{1}{0}{1}%
  \TestMax{-10}{-20}{-10}%
  \TestMax{ 1 }{ 2 }{2}%
  \TestMax{ 2 }{ 1 }{2}%
  \TestMax{1}{1}{1}%
  \TestMax{\number\z@}{\number\@ne}{1}%
  \TestMax{\number\@ne}{\number\m@ne}{1}%
\end{qstest}

\begin{qstest}{cmp}{cmp}%
  \TestCmp{0}{0}{0}%
  \TestCmp{-21}{17}{-1}%
  \TestCmp{3}{4}{-1}%
  \TestCmp{-10}{-10}{0}%
  \TestCmp{-10}{-11}{1}%
  \TestCmp{100}{5}{1}%
  \TestCmp{9}{10}{-1}%
  \TestCmp{10}{9}{1}%
  \TestCmp{ 3 }{ 3 }{0}%
  \TestCmp{-9}{-10}{1}%
  \TestCmp{-10}{-9}{-1}%
  \TestCmp{-3}{-3}{0}%
  \TestCmp{0}{-2}{1}%
  \TestCmp{0}{2}{-1}%
  \TestCmp{2}{0}{1}%
  \TestCmp{-2}{0}{-1}%
  \TestCmp{12}{11}{1}%
  \TestCmp{11}{12}{-1}%
  \TestCmp{2147483647}{-2147483647}{1}%
  \TestCmp{-2147483647}{2147483647}{-1}%
  \TestCmp{2147483647}{2147483647}{0}%
  \TestCmp{\number\z@}{\number\@ne}{-1}%
  \TestCmp{\number\@ne}{\number\m@ne}{1}%
  \TestCmp{ 4 }{ 5 }{-1}%
  \TestCmp{ -3 }{ -7 }{1}%
\end{qstest}

\begin{qstest}{odd}{odd}
\tracingmacros=1
  \TestOdd{0}{0}%
  \TestOdd{1}{1}%
  \TestOdd{2}{0}%
  \TestOdd{3}{1}%
  \TestOdd{14}{0}%
  \TestOdd{15}{1}%
  \TestOdd{12345678901234567896}{0}%
  \TestOdd{12345678901234567897}{1}%
\end{qstest}

\begin{qstest}{inc}{inc}%
  \TestInc{0}{1}%
  \TestInc{1}{2}%
  \TestInc{-1}{0}%
  \TestInc{10}{11}%
  \TestInc{-10}{-9}%
  \TestInc{ 3 }{4}%
  \TestInc{999}{1000}%
  \TestInc{-1000}{-999}%
  \TestInc{129}{130}%
  \TestInc{2147483646}{2147483647}%
  \TestInc{-2147483647}{-2147483646}%
  \TestInc{12345678901234567890}{12345678901234567891}%
  \TestInc{99999999999999999999}{100000000000000000000}%
  \TestInc{-12345678901234567891}{-12345678901234567890}%
  \TestInc{-100000000000000000000}{-99999999999999999999}%
\end{qstest}

\begin{qstest}{dec}{dec}%
  \TestDec{0}{-1}%
  \TestDec{1}{0}%
  \TestDec{-1}{-2}%
  \TestDec{10}{9}%
  \TestDec{-10}{-11}%
  \TestDec{1000}{999}%
  \TestDec{-999}{-1000}%
  \TestDec{130}{129}%
  \TestDec{2147483647}{2147483646}%
  \TestDec{-2147483646}{-2147483647}%
  \TestDec{12345678901234567891}{12345678901234567890}%
  \TestDec{100000000000000000000}{99999999999999999999}%
  \TestDec{-12345678901234567890}{-12345678901234567891}%
  \TestDec{-99999999999999999999}{-100000000000000000000}%
\end{qstest}

\begin{qstest}{add}{add}%
  \TestAdd{0}{0}{0}%
  \TestAdd{1}{0}{1}%
  \TestAdd{0}{1}{1}%
  \TestAdd{1}{2}{3}%
  \TestAdd{-1}{-1}{-2}%
  \TestAdd{2147483646}{1}{2147483647}%
  \TestAdd{-2147483647}{2147483647}{0}%
  \TestAdd{20}{-5}{15}%
  \TestAdd{-4}{-1}{-5}%
  \TestAdd{-1}{-4}{-5}%
  \TestAdd{-4}{1}{-3}%
  \TestAdd{-1}{4}{3}%
  \TestAdd{4}{-1}{3}%
  \TestAdd{1}{-4}{-3}%
  \TestAdd{-4}{-1}{-5}%
  \TestAdd{-1}{-4}{-5}%
  \TestAdd{ -4 }{ -1 }{-5}%
  \TestAdd{ -1 }{ -4 }{-5}%
  \TestAdd{ -4 }{ 1 }{-3}%
  \TestAdd{ -1 }{ 4 }{3}%
  \TestAdd{ 4 }{ -1 }{3}%
  \TestAdd{ 1 }{ -4 }{-3}%
  \TestAdd{ -4 }{ -1 }{-5}%
  \TestAdd{ -1 }{ -4 }{-5}%
  \TestAdd{876543210}{111111111}{987654321}%
  \TestAdd{999999999}{2}{1000000001}%
\end{qstest}

\begin{qstest}{sub}{sub}
  \TestSub{0}{0}{0}%
  \TestSub{1}{0}{1}%
  \TestSub{1}{2}{-1}%
  \TestSub{-1}{-1}{0}%
  \TestSub{2147483646}{-1}{2147483647}%
  \TestSub{-2147483647}{-2147483647}{0}%
  \TestSub{-4}{-1}{-3}%
  \TestSub{-1}{-4}{3}%
  \TestSub{-4}{1}{-5}%
  \TestSub{-1}{4}{-5}%
  \TestSub{4}{-1}{5}%
  \TestSub{1}{-4}{5}%
  \TestSub{-4}{-1}{-3}%
  \TestSub{-1}{-4}{3}%
  \TestSub{ -4 }{ -1 }{-3}%
  \TestSub{ -1 }{ -4 }{3}%
  \TestSub{ -4 }{ 1 }{-5}%
  \TestSub{ -1 }{ 4 }{-5}%
  \TestSub{ 4 }{ -1 }{5}%
  \TestSub{ 1 }{ -4 }{5}%
  \TestSub{ -4 }{ -1 }{-3}%
  \TestSub{ -1 }{ -4 }{3}%
  \TestSub{1000000000}{2}{999999998}%
  \TestSub{987654321}{111111111}{876543210}%
\end{qstest}

\begin{qstest}{shl}{shl}
  \TestShl{0}{0}%
  \TestShl{1}{2}%
  \TestShl{2}{4}%
  \TestShl{5621}{11242}%
  \TestShl{1073741823}{2147483646}%
\end{qstest}

\begin{qstest}{shr}{shr}
  \TestShr{0}{0}%
  \TestShr{1}{0}%
  \TestShr{2}{1}%
  \TestShr{3}{1}%
  \TestShr{4}{2}%
  \TestShr{5}{2}%
  \TestShr{6}{3}%
  \TestShr{7}{3}%
  \TestShr{8}{4}%
  \TestShr{9}{4}%
  \TestShr{10}{5}%
  \TestShr{11}{5}%
  \TestShr{12}{6}%
  \TestShr{13}{6}%
  \TestShr{14}{7}%
  \TestShr{15}{7}%
  \TestShr{16}{8}%
  \TestShr{17}{8}%
  \TestShr{18}{9}%
  \TestShr{19}{9}%
  \TestShr{20}{10}%
  \TestShr{21}{10}%
  \TestShr{22}{11}%
  \TestShr{11241}{5620}%
  \TestShr{73054202}{36527101}%
  \TestShr{2147483646}{1073741823}%
\end{qstest}

\begin{qstest}{mul}{mul}
  \TestMul{0}{0}{0}%
  \TestMul{1}{0}{0}%
  \TestMul{0}{1}{0}%
  \TestMul{1}{1}{1}%
  \TestMul{3}{1}{3}%
  \TestMul{1}{-3}{-3}%
  \TestMul{-4}{-5}{20}%
  \TestMul{3}{7}{21}%
  \TestMul{7}{3}{21}%
  \TestMul{3}{-7}{-21}%
  \TestMul{7}{-3}{-21}%
  \TestMul{-3}{7}{-21}%
  \TestMul{-7}{3}{-21}%
  \TestMul{-3}{-7}{21}%
  \TestMul{-7}{-3}{21}%
  \TestMul{12}{11}{132}%
  \TestMul{999}{333}{332667}%
  \TestMul{1000}{4321}{4321000}%
  \TestMul{12345}{173955}{2147474475}%
  \TestMul{1073741823}{2}{2147483646}%
  \TestMul{2}{1073741823}{2147483646}%
  \TestMul{-1073741823}{2}{-2147483646}%
  \TestMul{2}{-1073741823}{-2147483646}%
  \TestMul{6706022400}{13}{87178291200}%
\end{qstest}

\begin{qstest}{sqr}{sqr}
  \TestSqr{0}{0}%
  \TestSqr{1}{1}%
  \TestSqr{2}{4}%
  \TestSqr{3}{9}%
  \TestSqr{4}{16}%
  \TestSqr{9}{81}%
  \TestSqr{10}{100}%
  \TestSqr{46340}{2147395600}%
  \TestSqr{-1}{1}%
  \TestSqr{-2}{4}%
  \TestSqr{-46340}{2147395600}%
\end{qstest}

\begin{qstest}{fac}{fac}
  \TestFac{0}{1}%
  \TestFac{1}{1}%
  \TestFac{2}{2}%
  \TestFac{3}{2*3}%
  \TestFac{4}{2*3*4}%
  \TestFac{5}{2*3*4*5}%
  \TestFac{6}{2*3*4*5*6}%
  \TestFac{7}{2*3*4*5*6*7}%
  \TestFac{8}{2*3*4*5*6*7*8}%
  \TestFac{9}{2*3*4*5*6*7*8*9}%
  \TestFac{10}{2*3*4*5*6*7*8*9*10}%
  \TestFac{11}{2*3*4*5*6*7*8*9*10*11}%
  \TestFac{12}{2*3*4*5*6*7*8*9*10*11*12}%
  \TestFacBig{13}{6227020800}%
  \TestFacBig{14}{87178291200}%
  \TestFacBig{15}{1307674368000}%
  \TestFacBig{16}{20922789888000}%
  \TestFacBig{17}{355687428096000}%
  \TestFacBig{18}{6402373705728000}%
  \TestFacBig{19}{121645100408832000}%
  \TestFacBig{20}{2432902008176640000}%
  \TestFacBig{21}{51090942171709440000}%
  \TestFacBig{22}{1124000727777607680000}%
\end{qstest}

\begin{qstest}{pow}{pow}
  \TestPow{-2}{0}{1}%
  \TestPow{-1}{0}{1}%
  \TestPow{0}{0}{1}%
  \TestPow{1}{0}{1}%
  \TestPow{2}{0}{1}%
  \TestPow{3}{0}{1}%
  \TestPow{-2}{1}{-2}%
  \TestPow{-1}{1}{-1}%
  \TestPow{1}{1}{1}%
  \TestPow{2}{1}{2}%
  \TestPow{3}{1}{3}%
  \TestPow{-2}{2}{4}%
  \TestPow{-1}{2}{1}%
  \TestPow{0}{2}{0}%
  \TestPow{1}{2}{1}%
  \TestPow{2}{2}{4}%
  \TestPow{3}{2}{9}%
  \TestPow{0}{1}{0}%
  \TestPow{1}{-2}{1}%
  \TestPow{1}{-1}{1}%
  \TestPow{-1}{-2}{1}%
  \TestPow{-1}{-1}{-1}%
  \TestPow{-1}{3}{-1}%
  \TestPow{-1}{4}{1}%
  \TestPow{-2}{-1}{0}%
  \TestPow{-2}{-2}{0}%
  \TestPow{2}{3}{8}%
  \TestPow{2}{4}{16}%
  \TestPow{2}{5}{32}%
  \TestPow{2}{6}{64}%
  \TestPow{2}{7}{128}%
  \TestPow{2}{8}{256}%
  \TestPow{2}{9}{512}%
  \TestPow{2}{10}{1024}%
  \TestPow{-2}{3}{-8}%
  \TestPow{-2}{4}{16}%
  \TestPow{-2}{5}{-32}%
  \TestPow{-2}{6}{64}%
  \TestPow{-2}{7}{-128}%
  \TestPow{-2}{8}{256}%
  \TestPow{-2}{9}{-512}%
  \TestPow{-2}{10}{1024}%
  \TestPow{3}{3}{27}%
  \TestPow{3}{4}{81}%
  \TestPow{3}{5}{243}%
  \TestPow{-3}{3}{-27}%
  \TestPow{-3}{4}{81}%
  \TestPow{-3}{5}{-243}%
  \TestPow{2}{30}{1073741824}%
  \TestPow{-3}{19}{-1162261467}%
  \TestPow{5}{13}{1220703125}%
  \TestPow{-7}{11}{-1977326743}%
\end{qstest}

\begin{qstest}{div}{div}
  \TestDiv{1}{1}{1}%
  \TestDiv{2}{1}{2}%
  \TestDiv{-2}{1}{-2}%
  \TestDiv{2}{-1}{-2}%
  \TestDiv{-2}{-1}{2}%
  \TestDiv{15}{2}{7}%
  \TestDiv{-16}{2}{-8}%
  \TestDiv{1}{2}{0}%
  \TestDiv{1}{3}{0}%
  \TestDiv{2}{3}{0}%
  \TestDiv{-2}{3}{0}%
  \TestDiv{2}{-3}{0}%
  \TestDiv{-2}{-3}{0}%
  \TestDiv{13}{3}{4}%
  \TestDiv{-13}{-3}{4}%
  \TestDiv{-13}{3}{-4}%
  \TestDiv{-6}{5}{-1}%
  \TestDiv{-5}{5}{-1}%
  \TestDiv{-4}{5}{0}%
  \TestDiv{-3}{5}{0}%
  \TestDiv{-2}{5}{0}%
  \TestDiv{-1}{5}{0}%
  \TestDiv{0}{5}{0}%
  \TestDiv{1}{5}{0}%
  \TestDiv{2}{5}{0}%
  \TestDiv{3}{5}{0}%
  \TestDiv{4}{5}{0}%
  \TestDiv{5}{5}{1}%
  \TestDiv{6}{5}{1}%
  \TestDiv{-5}{4}{-1}%
  \TestDiv{-4}{4}{-1}%
  \TestDiv{-3}{4}{0}%
  \TestDiv{-2}{4}{0}%
  \TestDiv{-1}{4}{0}%
  \TestDiv{0}{4}{0}%
  \TestDiv{1}{4}{0}%
  \TestDiv{2}{4}{0}%
  \TestDiv{3}{4}{0}%
  \TestDiv{4}{4}{1}%
  \TestDiv{5}{4}{1}%
  \TestDiv{12345}{678}{18}%
  \TestDiv{32372}{5952}{5}%
  \TestDiv{284271294}{18162}{15651}%
  \TestDiv{217652429}{12561}{17327}%
  \TestDiv{462028434}{5439}{84947}%
  \TestDiv{2147483647}{1000}{2147483}%
  \TestDiv{2147483647}{-1000}{-2147483}%
  \TestDiv{-2147483647}{1000}{-2147483}%
  \TestDiv{-2147483647}{-1000}{2147483}%
  \TestDiv{0}{3}{0}%
  \TestDiv{1}{3}{0}%
  \TestDiv{2}{3}{0}%
  \TestDiv{3}{3}{1}%
  \TestDiv{4}{3}{1}%
  \TestDiv{5}{3}{1}%
  \TestDiv{6}{3}{2}%
  \TestDiv{7}{3}{2}%
  \TestDiv{8}{3}{2}%
  \TestDiv{9}{3}{3}%
  \TestDiv{10}{3}{3}%
  \TestDiv{11}{3}{3}%
  \TestDiv{12}{3}{4}%
  \TestDiv{13}{3}{4}%
  \TestDiv{14}{3}{4}%
  \TestDiv{15}{3}{5}%
  \TestDiv{16}{3}{5}%
  \TestDiv{17}{3}{5}%
  \TestDiv{18}{3}{6}%
  \TestDiv{19}{3}{6}%
  \TestDiv{20}{3}{6}%
  \TestDiv{21}{3}{7}%
  \TestDiv{22}{3}{7}%
  \TestDiv{23}{3}{7}%
  \TestDiv{24}{3}{8}%
  \TestDiv{25}{3}{8}%
  \TestDiv{26}{3}{8}%
  \TestDiv{27}{3}{9}%
  \TestDiv{28}{3}{9}%
  \TestDiv{29}{3}{9}%
  \TestDiv{30}{3}{10}%
  \TestDiv{31}{3}{10}%
  \TestDivBig{17363436332507}{24702}{702916214}%
\end{qstest}

\begin{qstest}{mod}{mod}
  \TestMod{-6}{5}{4}%
  \TestMod{-5}{5}{0}%
  \TestMod{-4}{5}{1}%
  \TestMod{-3}{5}{2}%
  \TestMod{-2}{5}{3}%
  \TestMod{-1}{5}{4}%
  \TestMod{0}{5}{0}%
  \TestMod{1}{5}{1}%
  \TestMod{2}{5}{2}%
  \TestMod{3}{5}{3}%
  \TestMod{4}{5}{4}%
  \TestMod{5}{5}{0}%
  \TestMod{6}{5}{1}%
  \TestMod{-5}{4}{3}%
  \TestMod{-4}{4}{0}%
  \TestMod{-3}{4}{1}%
  \TestMod{-2}{4}{2}%
  \TestMod{-1}{4}{3}%
  \TestMod{0}{4}{0}%
  \TestMod{1}{4}{1}%
  \TestMod{2}{4}{2}%
  \TestMod{3}{4}{3}%
  \TestMod{4}{4}{0}%
  \TestMod{5}{4}{1}%
  \TestMod{-6}{-5}{-1}%
  \TestMod{-5}{-5}{0}%
  \TestMod{-4}{-5}{-4}%
  \TestMod{-3}{-5}{-3}%
  \TestMod{-2}{-5}{-2}%
  \TestMod{-1}{-5}{-1}%
  \TestMod{0}{-5}{0}%
  \TestMod{1}{-5}{-4}%
  \TestMod{2}{-5}{-3}%
  \TestMod{3}{-5}{-2}%
  \TestMod{4}{-5}{-1}%
  \TestMod{5}{-5}{0}%
  \TestMod{6}{-5}{-4}%
  \TestMod{-5}{-4}{-1}%
  \TestMod{-4}{-4}{0}%
  \TestMod{-3}{-4}{-3}%
  \TestMod{-2}{-4}{-2}%
  \TestMod{-1}{-4}{-1}%
  \TestMod{0}{-4}{0}%
  \TestMod{1}{-4}{-3}%
  \TestMod{2}{-4}{-2}%
  \TestMod{3}{-4}{-1}%
  \TestMod{4}{-4}{0}%
  \TestMod{5}{-4}{-3}%
  \TestMod{2147483647}{1000}{647}%
  \TestMod{2147483647}{-1000}{-353}%
  \TestMod{-2147483647}{1000}{353}%
  \TestMod{-2147483647}{-1000}{-647}%
  \TestMod{ 0 }{ 4 }{0}%
  \TestMod{ 1 }{ 4 }{1}%
  \TestMod{ -1 }{ 4 }{3}%
  \TestMod{ 0 }{ -4 }{0}%
  \TestMod{ 1 }{ -4 }{-3}%
  \TestMod{ -1 }{ -4 }{-1}%
  \TestMod{18362}{25}{12}%
\end{qstest}

\newcommand*{\TestError}[2]{%
  \begingroup
    \expandafter\def\csname BigIntCalcError:#1\endcsname{}%
    \Expect*{#2}{0}%
    \expandafter\def\csname BigIntCalcError:#1\endcsname{ERROR}%
    \Expect*{#2}{0ERROR}%
  \endgroup
}
\begin{qstest}{error}{error}
  \TestError{FacNegative}{\bigintcalcFac{-1}}%
  \TestError{FacNegative}{\bigintcalcFac{-2147483647}}%
  \TestError{DivisionByZero}{\bigintcalcPow{0}{-1}}%
  \TestError{DivisionByZero}{\bigintcalcDiv{1}{0}}%
  \TestError{DivisionByZero}{\bigintcalcMod{1}{0}}%
\end{qstest}

\begin{document}
\end{document}
%</test2>
%    \end{macrocode}
%
% \section{Installation}
%
% \subsection{Download}
%
% \paragraph{Package.} This package is available on
% CTAN\footnote{\url{http://ctan.org/pkg/bigintcalc}}:
% \begin{description}
% \item[\CTAN{macros/latex/contrib/oberdiek/bigintcalc.dtx}] The source file.
% \item[\CTAN{macros/latex/contrib/oberdiek/bigintcalc.pdf}] Documentation.
% \end{description}
%
%
% \paragraph{Bundle.} All the packages of the bundle `oberdiek'
% are also available in a TDS compliant ZIP archive. There
% the packages are already unpacked and the documentation files
% are generated. The files and directories obey the TDS standard.
% \begin{description}
% \item[\CTAN{install/macros/latex/contrib/oberdiek.tds.zip}]
% \end{description}
% \emph{TDS} refers to the standard ``A Directory Structure
% for \TeX\ Files'' (\CTAN{tds/tds.pdf}). Directories
% with \xfile{texmf} in their name are usually organized this way.
%
% \subsection{Bundle installation}
%
% \paragraph{Unpacking.} Unpack the \xfile{oberdiek.tds.zip} in the
% TDS tree (also known as \xfile{texmf} tree) of your choice.
% Example (linux):
% \begin{quote}
%   |unzip oberdiek.tds.zip -d ~/texmf|
% \end{quote}
%
% \paragraph{Script installation.}
% Check the directory \xfile{TDS:scripts/oberdiek/} for
% scripts that need further installation steps.
% Package \xpackage{attachfile2} comes with the Perl script
% \xfile{pdfatfi.pl} that should be installed in such a way
% that it can be called as \texttt{pdfatfi}.
% Example (linux):
% \begin{quote}
%   |chmod +x scripts/oberdiek/pdfatfi.pl|\\
%   |cp scripts/oberdiek/pdfatfi.pl /usr/local/bin/|
% \end{quote}
%
% \subsection{Package installation}
%
% \paragraph{Unpacking.} The \xfile{.dtx} file is a self-extracting
% \docstrip\ archive. The files are extracted by running the
% \xfile{.dtx} through \plainTeX:
% \begin{quote}
%   \verb|tex bigintcalc.dtx|
% \end{quote}
%
% \paragraph{TDS.} Now the different files must be moved into
% the different directories in your installation TDS tree
% (also known as \xfile{texmf} tree):
% \begin{quote}
% \def\t{^^A
% \begin{tabular}{@{}>{\ttfamily}l@{ $\rightarrow$ }>{\ttfamily}l@{}}
%   bigintcalc.sty & tex/generic/oberdiek/bigintcalc.sty\\
%   bigintcalc.pdf & doc/latex/oberdiek/bigintcalc.pdf\\
%   test/bigintcalc-test1.tex & doc/latex/oberdiek/test/bigintcalc-test1.tex\\
%   test/bigintcalc-test2.tex & doc/latex/oberdiek/test/bigintcalc-test2.tex\\
%   test/bigintcalc-test3.tex & doc/latex/oberdiek/test/bigintcalc-test3.tex\\
%   bigintcalc.dtx & source/latex/oberdiek/bigintcalc.dtx\\
% \end{tabular}^^A
% }^^A
% \sbox0{\t}^^A
% \ifdim\wd0>\linewidth
%   \begingroup
%     \advance\linewidth by\leftmargin
%     \advance\linewidth by\rightmargin
%   \edef\x{\endgroup
%     \def\noexpand\lw{\the\linewidth}^^A
%   }\x
%   \def\lwbox{^^A
%     \leavevmode
%     \hbox to \linewidth{^^A
%       \kern-\leftmargin\relax
%       \hss
%       \usebox0
%       \hss
%       \kern-\rightmargin\relax
%     }^^A
%   }^^A
%   \ifdim\wd0>\lw
%     \sbox0{\small\t}^^A
%     \ifdim\wd0>\linewidth
%       \ifdim\wd0>\lw
%         \sbox0{\footnotesize\t}^^A
%         \ifdim\wd0>\linewidth
%           \ifdim\wd0>\lw
%             \sbox0{\scriptsize\t}^^A
%             \ifdim\wd0>\linewidth
%               \ifdim\wd0>\lw
%                 \sbox0{\tiny\t}^^A
%                 \ifdim\wd0>\linewidth
%                   \lwbox
%                 \else
%                   \usebox0
%                 \fi
%               \else
%                 \lwbox
%               \fi
%             \else
%               \usebox0
%             \fi
%           \else
%             \lwbox
%           \fi
%         \else
%           \usebox0
%         \fi
%       \else
%         \lwbox
%       \fi
%     \else
%       \usebox0
%     \fi
%   \else
%     \lwbox
%   \fi
% \else
%   \usebox0
% \fi
% \end{quote}
% If you have a \xfile{docstrip.cfg} that configures and enables \docstrip's
% TDS installing feature, then some files can already be in the right
% place, see the documentation of \docstrip.
%
% \subsection{Refresh file name databases}
%
% If your \TeX~distribution
% (\teTeX, \mikTeX, \dots) relies on file name databases, you must refresh
% these. For example, \teTeX\ users run \verb|texhash| or
% \verb|mktexlsr|.
%
% \subsection{Some details for the interested}
%
% \paragraph{Attached source.}
%
% The PDF documentation on CTAN also includes the
% \xfile{.dtx} source file. It can be extracted by
% AcrobatReader 6 or higher. Another option is \textsf{pdftk},
% e.g. unpack the file into the current directory:
% \begin{quote}
%   \verb|pdftk bigintcalc.pdf unpack_files output .|
% \end{quote}
%
% \paragraph{Unpacking with \LaTeX.}
% The \xfile{.dtx} chooses its action depending on the format:
% \begin{description}
% \item[\plainTeX:] Run \docstrip\ and extract the files.
% \item[\LaTeX:] Generate the documentation.
% \end{description}
% If you insist on using \LaTeX\ for \docstrip\ (really,
% \docstrip\ does not need \LaTeX), then inform the autodetect routine
% about your intention:
% \begin{quote}
%   \verb|latex \let\install=y\input{bigintcalc.dtx}|
% \end{quote}
% Do not forget to quote the argument according to the demands
% of your shell.
%
% \paragraph{Generating the documentation.}
% You can use both the \xfile{.dtx} or the \xfile{.drv} to generate
% the documentation. The process can be configured by the
% configuration file \xfile{ltxdoc.cfg}. For instance, put this
% line into this file, if you want to have A4 as paper format:
% \begin{quote}
%   \verb|\PassOptionsToClass{a4paper}{article}|
% \end{quote}
% An example follows how to generate the
% documentation with pdf\LaTeX:
% \begin{quote}
%\begin{verbatim}
%pdflatex bigintcalc.dtx
%makeindex -s gind.ist bigintcalc.idx
%pdflatex bigintcalc.dtx
%makeindex -s gind.ist bigintcalc.idx
%pdflatex bigintcalc.dtx
%\end{verbatim}
% \end{quote}
%
% \section{Catalogue}
%
% The following XML file can be used as source for the
% \href{http://mirror.ctan.org/help/Catalogue/catalogue.html}{\TeX\ Catalogue}.
% The elements \texttt{caption} and \texttt{description} are imported
% from the original XML file from the Catalogue.
% The name of the XML file in the Catalogue is \xfile{bigintcalc.xml}.
%    \begin{macrocode}
%<*catalogue>
<?xml version='1.0' encoding='us-ascii'?>
<!DOCTYPE entry SYSTEM 'catalogue.dtd'>
<entry datestamp='$Date$' modifier='$Author$' id='bigintcalc'>
  <name>bigintcalc</name>
  <caption>Integer calculations on very large numbers.</caption>
  <authorref id='auth:oberdiek'/>
  <copyright owner='Heiko Oberdiek' year='2007,2011,2012'/>
  <license type='lppl1.3'/>
  <version number='1.4'/>
  <description>
    This package provides expandable arithmetic operations
    with big integers that can exceed TeX's number limits.
    <p/>
    The package is part of the <xref refid='oberdiek'>oberdiek</xref> bundle.
  </description>
  <documentation details='Package documentation'
      href='ctan:/macros/latex/contrib/oberdiek/bigintcalc.pdf'/>
  <ctan file='true' path='/macros/latex/contrib/oberdiek/bigintcalc.dtx'/>
  <miktex location='oberdiek'/>
  <texlive location='oberdiek'/>
  <install path='/macros/latex/contrib/oberdiek/oberdiek.tds.zip'/>
</entry>
%</catalogue>
%    \end{macrocode}
%
% \begin{History}
%   \begin{Version}{2007/09/27 v1.0}
%   \item
%     First version.
%   \end{Version}
%   \begin{Version}{2007/11/11 v1.1}
%   \item
%     Use of package \xpackage{pdftexcmds} for \LuaTeX\ support.
%   \end{Version}
%   \begin{Version}{2011/01/30 v1.2}
%   \item
%     Already loaded package files are not input in \hologo{plainTeX}.
%   \end{Version}
%   \begin{Version}{2012/04/08 v1.3}
%   \item
%     Fix: \xpackage{pdftexcmds} wasn't loaded in case of \hologo{LaTeX}.
%   \end{Version}
%   \begin{Version}{2016/05/16 v1.4}
%   \item
%     Documentation updates.
%   \end{Version}
% \end{History}
%
% \PrintIndex
%
% \Finale
\endinput

%        (quote the arguments according to the demands of your shell)
%
% Documentation:
%    (a) If bigintcalc.drv is present:
%           latex bigintcalc.drv
%    (b) Without bigintcalc.drv:
%           latex bigintcalc.dtx; ...
%    The class ltxdoc loads the configuration file ltxdoc.cfg
%    if available. Here you can specify further options, e.g.
%    use A4 as paper format:
%       \PassOptionsToClass{a4paper}{article}
%
%    Programm calls to get the documentation (example):
%       pdflatex bigintcalc.dtx
%       makeindex -s gind.ist bigintcalc.idx
%       pdflatex bigintcalc.dtx
%       makeindex -s gind.ist bigintcalc.idx
%       pdflatex bigintcalc.dtx
%
% Installation:
%    TDS:tex/generic/oberdiek/bigintcalc.sty
%    TDS:doc/latex/oberdiek/bigintcalc.pdf
%    TDS:doc/latex/oberdiek/test/bigintcalc-test1.tex
%    TDS:doc/latex/oberdiek/test/bigintcalc-test2.tex
%    TDS:doc/latex/oberdiek/test/bigintcalc-test3.tex
%    TDS:source/latex/oberdiek/bigintcalc.dtx
%
%<*ignore>
\begingroup
  \catcode123=1 %
  \catcode125=2 %
  \def\x{LaTeX2e}%
\expandafter\endgroup
\ifcase 0\ifx\install y1\fi\expandafter
         \ifx\csname processbatchFile\endcsname\relax\else1\fi
         \ifx\fmtname\x\else 1\fi\relax
\else\csname fi\endcsname
%</ignore>
%<*install>
\input docstrip.tex
\Msg{************************************************************************}
\Msg{* Installation}
\Msg{* Package: bigintcalc 2016/05/16 v1.4 Expandable calculations on big integers (HO)}
\Msg{************************************************************************}

\keepsilent
\askforoverwritefalse

\let\MetaPrefix\relax
\preamble

This is a generated file.

Project: bigintcalc
Version: 2016/05/16 v1.4

Copyright (C) 2007, 2011, 2012 by
   Heiko Oberdiek <heiko.oberdiek at googlemail.com>

This work may be distributed and/or modified under the
conditions of the LaTeX Project Public License, either
version 1.3c of this license or (at your option) any later
version. This version of this license is in
   http://www.latex-project.org/lppl/lppl-1-3c.txt
and the latest version of this license is in
   http://www.latex-project.org/lppl.txt
and version 1.3 or later is part of all distributions of
LaTeX version 2005/12/01 or later.

This work has the LPPL maintenance status "maintained".

This Current Maintainer of this work is Heiko Oberdiek.

The Base Interpreter refers to any `TeX-Format',
because some files are installed in TDS:tex/generic//.

This work consists of the main source file bigintcalc.dtx
and the derived files
   bigintcalc.sty, bigintcalc.pdf, bigintcalc.ins, bigintcalc.drv,
   bigintcalc-test1.tex, bigintcalc-test2.tex,
   bigintcalc-test3.tex.

\endpreamble
\let\MetaPrefix\DoubleperCent

\generate{%
  \file{bigintcalc.ins}{\from{bigintcalc.dtx}{install}}%
  \file{bigintcalc.drv}{\from{bigintcalc.dtx}{driver}}%
  \usedir{tex/generic/oberdiek}%
  \file{bigintcalc.sty}{\from{bigintcalc.dtx}{package}}%
%  \usedir{doc/latex/oberdiek/test}%
%  \file{bigintcalc-test1.tex}{\from{bigintcalc.dtx}{test1}}%
%  \file{bigintcalc-test2.tex}{\from{bigintcalc.dtx}{test2,etex}}%
%  \file{bigintcalc-test3.tex}{\from{bigintcalc.dtx}{test2,noetex}}%
  \nopreamble
  \nopostamble
  \usedir{source/latex/oberdiek/catalogue}%
  \file{bigintcalc.xml}{\from{bigintcalc.dtx}{catalogue}}%
}

\catcode32=13\relax% active space
\let =\space%
\Msg{************************************************************************}
\Msg{*}
\Msg{* To finish the installation you have to move the following}
\Msg{* file into a directory searched by TeX:}
\Msg{*}
\Msg{*     bigintcalc.sty}
\Msg{*}
\Msg{* To produce the documentation run the file `bigintcalc.drv'}
\Msg{* through LaTeX.}
\Msg{*}
\Msg{* Happy TeXing!}
\Msg{*}
\Msg{************************************************************************}

\endbatchfile
%</install>
%<*ignore>
\fi
%</ignore>
%<*driver>
\NeedsTeXFormat{LaTeX2e}
\ProvidesFile{bigintcalc.drv}%
  [2016/05/16 v1.4 Expandable calculations on big integers (HO)]%
\documentclass{ltxdoc}
\usepackage{wasysym}
\let\iint\relax
\let\iiint\relax
\usepackage[fleqn]{amsmath}

\DeclareMathOperator{\opInv}{Inv}
\DeclareMathOperator{\opAbs}{Abs}
\DeclareMathOperator{\opSgn}{Sgn}
\DeclareMathOperator{\opMin}{Min}
\DeclareMathOperator{\opMax}{Max}
\DeclareMathOperator{\opCmp}{Cmp}
\DeclareMathOperator{\opOdd}{Odd}
\DeclareMathOperator{\opInc}{Inc}
\DeclareMathOperator{\opDec}{Dec}
\DeclareMathOperator{\opAdd}{Add}
\DeclareMathOperator{\opSub}{Sub}
\DeclareMathOperator{\opShl}{Shl}
\DeclareMathOperator{\opShr}{Shr}
\DeclareMathOperator{\opMul}{Mul}
\DeclareMathOperator{\opSqr}{Sqr}
\DeclareMathOperator{\opFac}{Fac}
\DeclareMathOperator{\opPow}{Pow}
\DeclareMathOperator{\opDiv}{Div}
\DeclareMathOperator{\opMod}{Mod}
\DeclareMathOperator{\opInt}{Int}
\DeclareMathOperator{\opODD}{ifodd}

\newcommand*{\Def}{%
  \ensuremath{%
    \mathrel{\mathop{:}}=%
  }%
}
\newcommand*{\op}[1]{%
  \textsf{#1}%
}
\usepackage{holtxdoc}[2011/11/22]
\begin{document}
  \DocInput{bigintcalc.dtx}%
\end{document}
%</driver>
% \fi
%
%
% \CharacterTable
%  {Upper-case    \A\B\C\D\E\F\G\H\I\J\K\L\M\N\O\P\Q\R\S\T\U\V\W\X\Y\Z
%   Lower-case    \a\b\c\d\e\f\g\h\i\j\k\l\m\n\o\p\q\r\s\t\u\v\w\x\y\z
%   Digits        \0\1\2\3\4\5\6\7\8\9
%   Exclamation   \!     Double quote  \"     Hash (number) \#
%   Dollar        \$     Percent       \%     Ampersand     \&
%   Acute accent  \'     Left paren    \(     Right paren   \)
%   Asterisk      \*     Plus          \+     Comma         \,
%   Minus         \-     Point         \.     Solidus       \/
%   Colon         \:     Semicolon     \;     Less than     \<
%   Equals        \=     Greater than  \>     Question mark \?
%   Commercial at \@     Left bracket  \[     Backslash     \\
%   Right bracket \]     Circumflex    \^     Underscore    \_
%   Grave accent  \`     Left brace    \{     Vertical bar  \|
%   Right brace   \}     Tilde         \~}
%
% \GetFileInfo{bigintcalc.drv}
%
% \title{The \xpackage{bigintcalc} package}
% \date{2016/05/16 v1.4}
% \author{Heiko Oberdiek\thanks
% {Please report any issues at https://github.com/ho-tex/oberdiek/issues}\\
% \xemail{heiko.oberdiek at googlemail.com}}
%
% \maketitle
%
% \begin{abstract}
% This package provides expandable arithmetic operations
% with big integers that can exceed \TeX's number limits.
% \end{abstract}
%
% \tableofcontents
%
% \section{Documentation}
%
% \subsection{Introduction}
%
% Package \xpackage{bigintcalc} defines arithmetic operations
% that deal with big integers. Big integers can be given
% either as explicit integer number or as macro code
% that expands to an explicit number. \emph{Big} means that
% there is no limit on the size of the number. Big integers
% may exceed \TeX's range limitation of -2147483647 and 2147483647.
% Only memory issues will limit the usable range.
%
% In opposite to package \xpackage{intcalc}
% unexpandable command tokens are not supported, even if
% they are valid \TeX\ numbers like count registers or
% commands created by \cs{chardef}. Nevertheless they may
% be used, if they are prefixed by \cs{number}.
%
% Also \eTeX's \cs{numexpr} expressions are not supported
% directly in the manner of package \xpackage{intcalc}.
% However they can be given if \cs{the}\cs{numexpr}
% or \cs{number}\cs{numexpr} are used.
%
% The operations have the form of macros that take one or
% two integers as parameter and return the integer result.
% The macro name is a three letter operation name prefixed
% by the package name, e.g. \cs{bigintcalcAdd}|{10}{43}| returns
% |53|.
%
% The macros are fully expandable, exactly two expansion
% steps generate the result. Therefore the operations
% may be used nearly everywhere in \TeX, even inside
% \cs{csname}, file names,  or other
% expandable contexts.
%
% \subsection{Conditions}
%
% \subsubsection{Preconditions}
%
% \begin{itemize}
% \item
%   Arguments can be anything that expands to a number that consists
%   of optional signs and digits.
% \item
%   The arguments and return values must be sound.
%   Zero as divisor or factorials of negative numbers will cause errors.
% \end{itemize}
%
% \subsubsection{Postconditions}
%
% Additional properties of the macros apart from calculating
% a correct result (of course \smiley):
% \begin{itemize}
% \item
%   The macros are fully expandable. Thus they can
%   be used inside \cs{edef}, \cs{csname},
%   for example.
% \item
%   Furthermore exactly two expansion steps calculate the result.
% \item
%   The number consists of one optional minus sign and one or more
%   digits. The first digit is larger than zero for
%   numbers that consists of more than one digit.
%
%   In short, the number format is exactly the same as
%   \cs{number} generates, but without its range limitation.
%   And the tokens (minus sign, digits)
%   have catcode 12 (other).
% \item
%   Call by value is simulated. First the arguments are
%   converted to numbers. Then these numbers are used
%   in the calculations.
%
%   Remember that arguments
%   may contain expensive macros or \eTeX\ expressions.
%   This strategy avoids multiple evaluations of such
%   arguments.
% \end{itemize}
%
% \subsection{Error handling}
%
% Some errors are detected by the macros, example: division by zero.
% In this cases an undefined control sequence is called and causes
% a TeX error message, example: \cs{BigIntCalcError:DivisionByZero}.
% The name of the control sequence contains
% the reason for the error. The \TeX\ error may be ignored.
% Then the operation returns zero as result.
% Because the macros are supposed to work in expandible contexts.
% An traditional error message, however, is not expandable and
% would break these contexts.
%
% \subsection{Operations}
%
% Some definition equations below use the function $\opInt$
% that converts a real number to an integer. The number
% is truncated that means rounding to zero:
% \begin{gather*}
%   \opInt(x) \Def
%   \begin{cases}
%     \lfloor x\rfloor & \text{if $x\geq0$}\\
%     \lceil x\rceil & \text{otherwise}
%   \end{cases}
% \end{gather*}
%
% \subsubsection{\op{Num}}
%
% \begin{declcs}{bigintcalcNum} \M{x}
% \end{declcs}
% Macro \cs{bigintcalcNum} converts its argument to a normalized integer
% number without unnecessary leading zeros or signs.
% The result matches the regular expression:
%\begin{quote}
%\begin{verbatim}
%0|-?[1-9][0-9]*
%\end{verbatim}
%\end{quote}
%
% \subsubsection{\op{Inv}, \op{Abs}, \op{Sgn}}
%
% \begin{declcs}{bigintcalcInv} \M{x}
% \end{declcs}
% Macro \cs{bigintcalcInv} switches the sign.
% \begin{gather*}
%   \opInv(x) \Def -x
% \end{gather*}
%
% \begin{declcs}{bigintcalcAbs} \M{x}
% \end{declcs}
% Macro \cs{bigintcalcAbs} returns the absolute value
% of integer \meta{x}.
% \begin{gather*}
%   \opAbs(x) \Def \vert x\vert
% \end{gather*}
%
% \begin{declcs}{bigintcalcSgn} \M{x}
% \end{declcs}
% Macro \cs{bigintcalcSgn} encodes the sign of \meta{x} as number.
% \begin{gather*}
%   \opSgn(x) \Def
%   \begin{cases}
%     -1& \text{if $x<0$}\\
%      0& \text{if $x=0$}\\
%      1& \text{if $x>0$}
%   \end{cases}
% \end{gather*}
% These return values can easily be distinguished by \cs{ifcase}:
%\begin{quote}
%\begin{verbatim}
%\ifcase\bigintcalcSgn{<x>}
%  $x=0$
%\or
%  $x>0$
%\else
%  $x<0$
%\fi
%\end{verbatim}
%\end{quote}
%
% \subsubsection{\op{Min}, \op{Max}, \op{Cmp}}
%
% \begin{declcs}{bigintcalcMin} \M{x} \M{y}
% \end{declcs}
% Macro \cs{bigintcalcMin} returns the smaller of the two integers.
% \begin{gather*}
%   \opMin(x,y) \Def
%   \begin{cases}
%     x & \text{if $x<y$}\\
%     y & \text{otherwise}
%   \end{cases}
% \end{gather*}
%
% \begin{declcs}{bigintcalcMax} \M{x} \M{y}
% \end{declcs}
% Macro \cs{bigintcalcMax} returns the larger of the two integers.
% \begin{gather*}
%   \opMax(x,y) \Def
%   \begin{cases}
%     x & \text{if $x>y$}\\
%     y & \text{otherwise}
%   \end{cases}
% \end{gather*}
%
% \begin{declcs}{bigintcalcCmp} \M{x} \M{y}
% \end{declcs}
% Macro \cs{bigintcalcCmp} encodes the comparison result as number:
% \begin{gather*}
%   \opCmp(x,y) \Def
%   \begin{cases}
%     -1 & \text{if $x<y$}\\
%      0 & \text{if $x=y$}\\
%      1 & \text{if $x>y$}
%   \end{cases}
% \end{gather*}
% These values can be distinguished by \cs{ifcase}:
%\begin{quote}
%\begin{verbatim}
%\ifcase\bigintcalcCmp{<x>}{<y>}
%  $x=y$
%\or
%  $x>y$
%\else
%  $x<y$
%\fi
%\end{verbatim}
%\end{quote}
%
% \subsubsection{\op{Odd}}
%
% \begin{declcs}{bigintcalcOdd} \M{x}
% \end{declcs}
% \begin{gather*}
%   \opOdd(x) \Def
%   \begin{cases}
%     1 & \text{if $x$ is odd}\\
%     0 & \text{if $x$ is even}
%   \end{cases}
% \end{gather*}
%
% \subsubsection{\op{Inc}, \op{Dec}, \op{Add}, \op{Sub}}
%
% \begin{declcs}{bigintcalcInc} \M{x}
% \end{declcs}
% Macro \cs{bigintcalcInc} increments \meta{x} by one.
% \begin{gather*}
%   \opInc(x) \Def x + 1
% \end{gather*}
%
% \begin{declcs}{bigintcalcDec} \M{x}
% \end{declcs}
% Macro \cs{bigintcalcDec} decrements \meta{x} by one.
% \begin{gather*}
%   \opDec(x) \Def x - 1
% \end{gather*}
%
% \begin{declcs}{bigintcalcAdd} \M{x} \M{y}
% \end{declcs}
% Macro \cs{bigintcalcAdd} adds the two numbers.
% \begin{gather*}
%   \opAdd(x, y) \Def x + y
% \end{gather*}
%
% \begin{declcs}{bigintcalcSub} \M{x} \M{y}
% \end{declcs}
% Macro \cs{bigintcalcSub} calculates the difference.
% \begin{gather*}
%   \opSub(x, y) \Def x - y
% \end{gather*}
%
% \subsubsection{\op{Shl}, \op{Shr}}
%
% \begin{declcs}{bigintcalcShl} \M{x}
% \end{declcs}
% Macro \cs{bigintcalcShl} implements shifting to the left that
% means the number is multiplied by two.
% The sign is preserved.
% \begin{gather*}
%   \opShl(x) \Def x*2
% \end{gather*}
%
% \begin{declcs}{bigintcalcShr} \M{x}
% \end{declcs}
% Macro \cs{bigintcalcShr} implements shifting to the right.
% That is equivalent to an integer division by two.
% The sign is preserved.
% \begin{gather*}
%   \opShr(x) \Def \opInt(x/2)
% \end{gather*}
%
% \subsubsection{\op{Mul}, \op{Sqr}, \op{Fac}, \op{Pow}}
%
% \begin{declcs}{bigintcalcMul} \M{x} \M{y}
% \end{declcs}
% Macro \cs{bigintcalcMul} calculates the product of
% \meta{x} and \meta{y}.
% \begin{gather*}
%   \opMul(x,y) \Def x*y
% \end{gather*}
%
% \begin{declcs}{bigintcalcSqr} \M{x}
% \end{declcs}
% Macro \cs{bigintcalcSqr} returns the square product.
% \begin{gather*}
%   \opSqr(x) \Def x^2
% \end{gather*}
%
% \begin{declcs}{bigintcalcFac} \M{x}
% \end{declcs}
% Macro \cs{bigintcalcFac} returns the factorial of \meta{x}.
% Negative numbers are not permitted.
% \begin{gather*}
%   \opFac(x) \Def x!\qquad\text{for $x\geq0$}
% \end{gather*}
% ($0! = 1$)
%
% \begin{declcs}{bigintcalcPow} M{x} M{y}
% \end{declcs}
% Macro \cs{bigintcalcPow} calculates the value of \meta{x} to the
% power of \meta{y}. The error ``division by zero'' is thrown
% if \meta{x} is zero and \meta{y} is negative.
% permitted:
% \begin{gather*}
%   \opPow(x,y) \Def
%   \opInt(x^y)\qquad\text{for $x\neq0$ or $y\geq0$}
% \end{gather*}
% ($0^0 = 1$)
%
% \subsubsection{\op{Div}, \op{Mul}}
%
% \begin{declcs}{bigintcalcDiv} \M{x} \M{y}
% \end{declcs}
% Macro \cs{bigintcalcDiv} performs an integer division.
% Argument \meta{y} must not be zero.
% \begin{gather*}
%   \opDiv(x,y) \Def \opInt(x/y)\qquad\text{for $y\neq0$}
% \end{gather*}
%
% \begin{declcs}{bigintcalcMod} \M{x} \M{y}
% \end{declcs}
% Macro \cs{bigintcalcMod} gets the remainder of the integer
% division. The sign follows the divisor \meta{y}.
% Argument \meta{y} must not be zero.
% \begin{gather*}
%   \opMod(x,y) \Def x\mathrel{\%}y\qquad\text{for $y\neq0$}
% \end{gather*}
% The result ranges:
% \begin{gather*}
%   -\vert y\vert < \opMod(x,y) \leq 0\qquad\text{for $y<0$}\\
%   0 \leq \opMod(x,y) < y\qquad\text{for $y\geq0$}
% \end{gather*}
%
% \subsection{Interface for programmers}
%
% If the programmer can ensure some more properties about
% the arguments of the operations, then the following
% macros are a little more efficient.
%
% In general numbers must obey the following constraints:
% \begin{itemize}
% \item Plain number: digit tokens only, no command tokens.
% \item Non-negative. Signs are forbidden.
% \item Delimited by exclamation mark. Curly braces
%       around the number are not allowed and will
%       break the code.
% \end{itemize}
%
% \begin{declcs}{BigIntCalcOdd} \meta{number} |!|
% \end{declcs}
% |1|/|0| is returned if \meta{number} is odd/even.
%
% \begin{declcs}{BigIntCalcInc} \meta{number} |!|
% \end{declcs}
% Incrementation.
%
% \begin{declcs}{BigIntCalcDec} \meta{number} |!|
% \end{declcs}
% Decrementation, positive number without zero.
%
% \begin{declcs}{BigIntCalcAdd} \meta{number A} |!| \meta{number B} |!|
% \end{declcs}
% Addition, $A\geq B$.
%
% \begin{declcs}{BigIntCalcSub} \meta{number A} |!| \meta{number B} |!|
% \end{declcs}
% Subtraction, $A\geq B$.
%
% \begin{declcs}{BigIntCalcShl} \meta{number} |!|
% \end{declcs}
% Left shift (multiplication with two).
%
% \begin{declcs}{BigIntCalcShr} \meta{number} |!|
% \end{declcs}
% Right shift (integer division by two).
%
% \begin{declcs}{BigIntCalcMul} \meta{number A} |!| \meta{number B} |!|
% \end{declcs}
% Multiplication, $A\geq B$.
%
% \begin{declcs}{BigIntCalcDiv} \meta{number A} |!| \meta{number B} |!|
% \end{declcs}
% Division operation.
%
% \begin{declcs}{BigIntCalcMod} \meta{number A} |!| \meta{number B} |!|
% \end{declcs}
% Modulo operation.
%
%
% \StopEventually{
% }
%
% \section{Implementation}
%
%    \begin{macrocode}
%<*package>
%    \end{macrocode}
%
% \subsection{Reload check and package identification}
%    Reload check, especially if the package is not used with \LaTeX.
%    \begin{macrocode}
\begingroup\catcode61\catcode48\catcode32=10\relax%
  \catcode13=5 % ^^M
  \endlinechar=13 %
  \catcode35=6 % #
  \catcode39=12 % '
  \catcode44=12 % ,
  \catcode45=12 % -
  \catcode46=12 % .
  \catcode58=12 % :
  \catcode64=11 % @
  \catcode123=1 % {
  \catcode125=2 % }
  \expandafter\let\expandafter\x\csname ver@bigintcalc.sty\endcsname
  \ifx\x\relax % plain-TeX, first loading
  \else
    \def\empty{}%
    \ifx\x\empty % LaTeX, first loading,
      % variable is initialized, but \ProvidesPackage not yet seen
    \else
      \expandafter\ifx\csname PackageInfo\endcsname\relax
        \def\x#1#2{%
          \immediate\write-1{Package #1 Info: #2.}%
        }%
      \else
        \def\x#1#2{\PackageInfo{#1}{#2, stopped}}%
      \fi
      \x{bigintcalc}{The package is already loaded}%
      \aftergroup\endinput
    \fi
  \fi
\endgroup%
%    \end{macrocode}
%    Package identification:
%    \begin{macrocode}
\begingroup\catcode61\catcode48\catcode32=10\relax%
  \catcode13=5 % ^^M
  \endlinechar=13 %
  \catcode35=6 % #
  \catcode39=12 % '
  \catcode40=12 % (
  \catcode41=12 % )
  \catcode44=12 % ,
  \catcode45=12 % -
  \catcode46=12 % .
  \catcode47=12 % /
  \catcode58=12 % :
  \catcode64=11 % @
  \catcode91=12 % [
  \catcode93=12 % ]
  \catcode123=1 % {
  \catcode125=2 % }
  \expandafter\ifx\csname ProvidesPackage\endcsname\relax
    \def\x#1#2#3[#4]{\endgroup
      \immediate\write-1{Package: #3 #4}%
      \xdef#1{#4}%
    }%
  \else
    \def\x#1#2[#3]{\endgroup
      #2[{#3}]%
      \ifx#1\@undefined
        \xdef#1{#3}%
      \fi
      \ifx#1\relax
        \xdef#1{#3}%
      \fi
    }%
  \fi
\expandafter\x\csname ver@bigintcalc.sty\endcsname
\ProvidesPackage{bigintcalc}%
  [2016/05/16 v1.4 Expandable calculations on big integers (HO)]%
%    \end{macrocode}
%
% \subsection{Catcodes}
%
%    \begin{macrocode}
\begingroup\catcode61\catcode48\catcode32=10\relax%
  \catcode13=5 % ^^M
  \endlinechar=13 %
  \catcode123=1 % {
  \catcode125=2 % }
  \catcode64=11 % @
  \def\x{\endgroup
    \expandafter\edef\csname BIC@AtEnd\endcsname{%
      \endlinechar=\the\endlinechar\relax
      \catcode13=\the\catcode13\relax
      \catcode32=\the\catcode32\relax
      \catcode35=\the\catcode35\relax
      \catcode61=\the\catcode61\relax
      \catcode64=\the\catcode64\relax
      \catcode123=\the\catcode123\relax
      \catcode125=\the\catcode125\relax
    }%
  }%
\x\catcode61\catcode48\catcode32=10\relax%
\catcode13=5 % ^^M
\endlinechar=13 %
\catcode35=6 % #
\catcode64=11 % @
\catcode123=1 % {
\catcode125=2 % }
\def\TMP@EnsureCode#1#2{%
  \edef\BIC@AtEnd{%
    \BIC@AtEnd
    \catcode#1=\the\catcode#1\relax
  }%
  \catcode#1=#2\relax
}
\TMP@EnsureCode{33}{12}% !
\TMP@EnsureCode{36}{14}% $ (comment!)
\TMP@EnsureCode{38}{14}% & (comment!)
\TMP@EnsureCode{40}{12}% (
\TMP@EnsureCode{41}{12}% )
\TMP@EnsureCode{42}{12}% *
\TMP@EnsureCode{43}{12}% +
\TMP@EnsureCode{45}{12}% -
\TMP@EnsureCode{46}{12}% .
\TMP@EnsureCode{47}{12}% /
\TMP@EnsureCode{58}{11}% : (letter!)
\TMP@EnsureCode{60}{12}% <
\TMP@EnsureCode{62}{12}% >
\TMP@EnsureCode{63}{14}% ? (comment!)
\TMP@EnsureCode{91}{12}% [
\TMP@EnsureCode{93}{12}% ]
\edef\BIC@AtEnd{\BIC@AtEnd\noexpand\endinput}
\begingroup\expandafter\expandafter\expandafter\endgroup
\expandafter\ifx\csname BIC@TestMode\endcsname\relax
\else
  \catcode63=9 % ? (ignore)
\fi
? \let\BIC@@TestMode\BIC@TestMode
%    \end{macrocode}
%
% \subsection{\eTeX\ detection}
%
%    \begin{macrocode}
\begingroup\expandafter\expandafter\expandafter\endgroup
\expandafter\ifx\csname numexpr\endcsname\relax
  \catcode36=9 % $ (ignore)
\else
  \catcode38=9 % & (ignore)
\fi
%    \end{macrocode}
%
% \subsection{Help macros}
%
%    \begin{macro}{\BIC@Fi}
%    \begin{macrocode}
\let\BIC@Fi\fi
%    \end{macrocode}
%    \end{macro}
%    \begin{macro}{\BIC@AfterFi}
%    \begin{macrocode}
\def\BIC@AfterFi#1#2\BIC@Fi{\fi#1}%
%    \end{macrocode}
%    \end{macro}
%    \begin{macro}{\BIC@AfterFiFi}
%    \begin{macrocode}
\def\BIC@AfterFiFi#1#2\BIC@Fi{\fi\fi#1}%
%    \end{macrocode}
%    \end{macro}
%    \begin{macro}{\BIC@AfterFiFiFi}
%    \begin{macrocode}
\def\BIC@AfterFiFiFi#1#2\BIC@Fi{\fi\fi\fi#1}%
%    \end{macrocode}
%    \end{macro}
%
%    \begin{macro}{\BIC@Space}
%    \begin{macrocode}
\begingroup
  \def\x#1{\endgroup
    \let\BIC@Space= #1%
  }%
\x{ }
%    \end{macrocode}
%    \end{macro}
%
% \subsection{Expand number}
%
%    \begin{macrocode}
\begingroup\expandafter\expandafter\expandafter\endgroup
\expandafter\ifx\csname RequirePackage\endcsname\relax
  \def\TMP@RequirePackage#1[#2]{%
    \begingroup\expandafter\expandafter\expandafter\endgroup
    \expandafter\ifx\csname ver@#1.sty\endcsname\relax
      \input #1.sty\relax
    \fi
  }%
  \TMP@RequirePackage{pdftexcmds}[2007/11/11]%
\else
  \RequirePackage{pdftexcmds}[2007/11/11]%
\fi
%    \end{macrocode}
%
%    \begin{macrocode}
\begingroup\expandafter\expandafter\expandafter\endgroup
\expandafter\ifx\csname pdf@escapehex\endcsname\relax
%    \end{macrocode}
%
%    \begin{macro}{\BIC@Expand}
%    \begin{macrocode}
  \def\BIC@Expand#1{%
    \romannumeral0%
    \BIC@@Expand#1!\@nil{}%
  }%
%    \end{macrocode}
%    \end{macro}
%    \begin{macro}{\BIC@@Expand}
%    \begin{macrocode}
  \def\BIC@@Expand#1#2\@nil#3{%
    \expandafter\ifcat\noexpand#1\relax
      \expandafter\@firstoftwo
    \else
      \expandafter\@secondoftwo
    \fi
    {%
      \expandafter\BIC@@Expand#1#2\@nil{#3}%
    }{%
      \ifx#1!%
        \expandafter\@firstoftwo
      \else
        \expandafter\@secondoftwo
      \fi
      { #3}{%
        \BIC@@Expand#2\@nil{#3#1}%
      }%
    }%
  }%
%    \end{macrocode}
%    \end{macro}
%    \begin{macro}{\@firstoftwo}
%    \begin{macrocode}
  \expandafter\ifx\csname @firstoftwo\endcsname\relax
    \long\def\@firstoftwo#1#2{#1}%
  \fi
%    \end{macrocode}
%    \end{macro}
%    \begin{macro}{\@secondoftwo}
%    \begin{macrocode}
  \expandafter\ifx\csname @secondoftwo\endcsname\relax
    \long\def\@secondoftwo#1#2{#2}%
  \fi
%    \end{macrocode}
%    \end{macro}
%
%    \begin{macrocode}
\else
%    \end{macrocode}
%    \begin{macro}{\BIC@Expand}
%    \begin{macrocode}
  \def\BIC@Expand#1{%
    \romannumeral0\expandafter\expandafter\expandafter\BIC@Space
    \pdf@unescapehex{%
      \expandafter\expandafter\expandafter
      \BIC@StripHexSpace\pdf@escapehex{#1}20\@nil
    }%
  }%
%    \end{macrocode}
%    \end{macro}
%    \begin{macro}{\BIC@StripHexSpace}
%    \begin{macrocode}
  \def\BIC@StripHexSpace#120#2\@nil{%
    #1%
    \ifx\\#2\\%
    \else
      \BIC@AfterFi{%
        \BIC@StripHexSpace#2\@nil
      }%
    \BIC@Fi
  }%
%    \end{macrocode}
%    \end{macro}
%    \begin{macrocode}
\fi
%    \end{macrocode}
%
% \subsection{Normalize expanded number}
%
%    \begin{macro}{\BIC@Normalize}
%    |#1|: result sign\\
%    |#2|: first token of number
%    \begin{macrocode}
\def\BIC@Normalize#1#2{%
  \ifx#2-%
    \ifx\\#1\\%
      \BIC@AfterFiFi{%
        \BIC@Normalize-%
      }%
    \else
      \BIC@AfterFiFi{%
        \BIC@Normalize{}%
      }%
    \fi
  \else
    \ifx#2+%
      \BIC@AfterFiFi{%
        \BIC@Normalize{#1}%
      }%
    \else
      \ifx#20%
        \BIC@AfterFiFiFi{%
          \BIC@NormalizeZero{#1}%
        }%
      \else
        \BIC@AfterFiFiFi{%
          \BIC@NormalizeDigits#1#2%
        }%
      \fi
    \fi
  \BIC@Fi
}
%    \end{macrocode}
%    \end{macro}
%    \begin{macro}{\BIC@NormalizeZero}
%    \begin{macrocode}
\def\BIC@NormalizeZero#1#2{%
  \ifx#2!%
    \BIC@AfterFi{ 0}%
  \else
    \ifx#20%
      \BIC@AfterFiFi{%
        \BIC@NormalizeZero{#1}%
      }%
    \else
      \BIC@AfterFiFi{%
        \BIC@NormalizeDigits#1#2%
      }%
    \fi
  \BIC@Fi
}
%    \end{macrocode}
%    \end{macro}
%    \begin{macro}{\BIC@NormalizeDigits}
%    \begin{macrocode}
\def\BIC@NormalizeDigits#1!{ #1}
%    \end{macrocode}
%    \end{macro}
%
% \subsection{\op{Num}}
%
%    \begin{macro}{\bigintcalcNum}
%    \begin{macrocode}
\def\bigintcalcNum#1{%
  \romannumeral0%
  \expandafter\expandafter\expandafter\BIC@Normalize
  \expandafter\expandafter\expandafter{%
  \expandafter\expandafter\expandafter}%
  \BIC@Expand{#1}!%
}
%    \end{macrocode}
%    \end{macro}
%
% \subsection{\op{Inv}, \op{Abs}, \op{Sgn}}
%
%
%    \begin{macro}{\bigintcalcInv}
%    \begin{macrocode}
\def\bigintcalcInv#1{%
  \romannumeral0\expandafter\expandafter\expandafter\BIC@Space
  \bigintcalcNum{-#1}%
}
%    \end{macrocode}
%    \end{macro}
%
%    \begin{macro}{\bigintcalcAbs}
%    \begin{macrocode}
\def\bigintcalcAbs#1{%
  \romannumeral0%
  \expandafter\expandafter\expandafter\BIC@Abs
  \bigintcalcNum{#1}%
}
%    \end{macrocode}
%    \end{macro}
%    \begin{macro}{\BIC@Abs}
%    \begin{macrocode}
\def\BIC@Abs#1{%
  \ifx#1-%
    \expandafter\BIC@Space
  \else
    \expandafter\BIC@Space
    \expandafter#1%
  \fi
}
%    \end{macrocode}
%    \end{macro}
%
%    \begin{macro}{\bigintcalcSgn}
%    \begin{macrocode}
\def\bigintcalcSgn#1{%
  \number
  \expandafter\expandafter\expandafter\BIC@Sgn
  \bigintcalcNum{#1}! %
}
%    \end{macrocode}
%    \end{macro}
%    \begin{macro}{\BIC@Sgn}
%    \begin{macrocode}
\def\BIC@Sgn#1#2!{%
  \ifx#1-%
    -1%
  \else
    \ifx#10%
      0%
    \else
      1%
    \fi
  \fi
}
%    \end{macrocode}
%    \end{macro}
%
% \subsection{\op{Cmp}, \op{Min}, \op{Max}}
%
%    \begin{macro}{\bigintcalcCmp}
%    \begin{macrocode}
\def\bigintcalcCmp#1#2{%
  \number
  \expandafter\expandafter\expandafter\BIC@Cmp
  \bigintcalcNum{#2}!{#1}%
}
%    \end{macrocode}
%    \end{macro}
%    \begin{macro}{\BIC@Cmp}
%    \begin{macrocode}
\def\BIC@Cmp#1!#2{%
  \expandafter\expandafter\expandafter\BIC@@Cmp
  \bigintcalcNum{#2}!#1!%
}
%    \end{macrocode}
%    \end{macro}
%    \begin{macro}{\BIC@@Cmp}
%    \begin{macrocode}
\def\BIC@@Cmp#1#2!#3#4!{%
  \ifx#1-%
    \ifx#3-%
      \BIC@AfterFiFi{%
        \BIC@@Cmp#4!#2!%
      }%
    \else
      \BIC@AfterFiFi{%
        -1 %
      }%
    \fi
  \else
    \ifx#3-%
      \BIC@AfterFiFi{%
        1 %
      }%
    \else
      \BIC@AfterFiFi{%
        \BIC@CmpLength#1#2!#3#4!#1#2!#3#4!%
      }%
    \fi
  \BIC@Fi
}
%    \end{macrocode}
%    \end{macro}
%    \begin{macro}{\BIC@PosCmp}
%    \begin{macrocode}
\def\BIC@PosCmp#1!#2!{%
  \BIC@CmpLength#1!#2!#1!#2!%
}
%    \end{macrocode}
%    \end{macro}
%    \begin{macro}{\BIC@CmpLength}
%    \begin{macrocode}
\def\BIC@CmpLength#1#2!#3#4!{%
  \ifx\\#2\\%
    \ifx\\#4\\%
      \BIC@AfterFiFi\BIC@CmpDiff
    \else
      \BIC@AfterFiFi{%
        \BIC@CmpResult{-1}%
      }%
    \fi
  \else
    \ifx\\#4\\%
      \BIC@AfterFiFi{%
        \BIC@CmpResult1%
      }%
    \else
      \BIC@AfterFiFi{%
        \BIC@CmpLength#2!#4!%
      }%
    \fi
  \BIC@Fi
}
%    \end{macrocode}
%    \end{macro}
%    \begin{macro}{\BIC@CmpResult}
%    \begin{macrocode}
\def\BIC@CmpResult#1#2!#3!{#1 }
%    \end{macrocode}
%    \end{macro}
%    \begin{macro}{\BIC@CmpDiff}
%    \begin{macrocode}
\def\BIC@CmpDiff#1#2!#3#4!{%
  \ifnum#1<#3 %
    \BIC@AfterFi{%
      -1 %
    }%
  \else
    \ifnum#1>#3 %
      \BIC@AfterFiFi{%
        1 %
      }%
    \else
      \ifx\\#2\\%
        \BIC@AfterFiFiFi{%
          0 %
        }%
      \else
        \BIC@AfterFiFiFi{%
          \BIC@CmpDiff#2!#4!%
        }%
      \fi
    \fi
  \BIC@Fi
}
%    \end{macrocode}
%    \end{macro}
%
%    \begin{macro}{\bigintcalcMin}
%    \begin{macrocode}
\def\bigintcalcMin#1{%
  \romannumeral0%
  \expandafter\expandafter\expandafter\BIC@MinMax
  \bigintcalcNum{#1}!-!%
}
%    \end{macrocode}
%    \end{macro}
%    \begin{macro}{\bigintcalcMax}
%    \begin{macrocode}
\def\bigintcalcMax#1{%
  \romannumeral0%
  \expandafter\expandafter\expandafter\BIC@MinMax
  \bigintcalcNum{#1}!!%
}
%    \end{macrocode}
%    \end{macro}
%    \begin{macro}{\BIC@MinMax}
%    |#1|: $x$\\
%    |#2|: sign for comparison\\
%    |#3|: $y$
%    \begin{macrocode}
\def\BIC@MinMax#1!#2!#3{%
  \expandafter\expandafter\expandafter\BIC@@MinMax
  \bigintcalcNum{#3}!#1!#2!%
}
%    \end{macrocode}
%    \end{macro}
%    \begin{macro}{\BIC@@MinMax}
%    |#1|: $y$\\
%    |#2|: $x$\\
%    |#3|: sign for comparison
%    \begin{macrocode}
\def\BIC@@MinMax#1!#2!#3!{%
  \ifnum\BIC@@Cmp#1!#2!=#31 %
    \BIC@AfterFi{ #1}%
  \else
    \BIC@AfterFi{ #2}%
  \BIC@Fi
}
%    \end{macrocode}
%    \end{macro}
%
% \subsection{\op{Odd}}
%
%    \begin{macro}{\bigintcalcOdd}
%    \begin{macrocode}
\def\bigintcalcOdd#1{%
  \romannumeral0%
  \expandafter\expandafter\expandafter\BIC@Odd
  \bigintcalcAbs{#1}!%
}
%    \end{macrocode}
%    \end{macro}
%    \begin{macro}{\BigIntCalcOdd}
%    \begin{macrocode}
\def\BigIntCalcOdd#1!{%
  \romannumeral0%
  \BIC@Odd#1!%
}
%    \end{macrocode}
%    \end{macro}
%    \begin{macro}{\BIC@Odd}
%    |#1|: $x$
%    \begin{macrocode}
\def\BIC@Odd#1#2{%
  \ifx#2!%
    \ifodd#1 %
      \BIC@AfterFiFi{ 1}%
    \else
      \BIC@AfterFiFi{ 0}%
    \fi
  \else
    \expandafter\BIC@Odd\expandafter#2%
  \BIC@Fi
}
%    \end{macrocode}
%    \end{macro}
%
% \subsection{\op{Inc}, \op{Dec}}
%
%    \begin{macro}{\bigintcalcInc}
%    \begin{macrocode}
\def\bigintcalcInc#1{%
  \romannumeral0%
  \expandafter\expandafter\expandafter\BIC@IncSwitch
  \bigintcalcNum{#1}!%
}
%    \end{macrocode}
%    \end{macro}
%    \begin{macro}{\BIC@IncSwitch}
%    \begin{macrocode}
\def\BIC@IncSwitch#1#2!{%
  \ifcase\BIC@@Cmp#1#2!-1!%
    \BIC@AfterFi{ 0}%
  \or
    \BIC@AfterFi{%
      \BIC@Inc#1#2!{}%
    }%
  \else
    \BIC@AfterFi{%
      \expandafter-\romannumeral0%
      \BIC@Dec#2!{}%
    }%
  \BIC@Fi
}
%    \end{macrocode}
%    \end{macro}
%
%    \begin{macro}{\bigintcalcDec}
%    \begin{macrocode}
\def\bigintcalcDec#1{%
  \romannumeral0%
  \expandafter\expandafter\expandafter\BIC@DecSwitch
  \bigintcalcNum{#1}!%
}
%    \end{macrocode}
%    \end{macro}
%    \begin{macro}{\BIC@DecSwitch}
%    \begin{macrocode}
\def\BIC@DecSwitch#1#2!{%
  \ifcase\BIC@Sgn#1#2! %
    \BIC@AfterFi{ -1}%
  \or
    \BIC@AfterFi{%
      \BIC@Dec#1#2!{}%
    }%
  \else
    \BIC@AfterFi{%
      \expandafter-\romannumeral0%
      \BIC@Inc#2!{}%
    }%
  \BIC@Fi
}
%    \end{macrocode}
%    \end{macro}
%
%    \begin{macro}{\BigIntCalcInc}
%    \begin{macrocode}
\def\BigIntCalcInc#1!{%
  \romannumeral0\BIC@Inc#1!{}%
}
%    \end{macrocode}
%    \end{macro}
%    \begin{macro}{\BigIntCalcDec}
%    \begin{macrocode}
\def\BigIntCalcDec#1!{%
  \romannumeral0\BIC@Dec#1!{}%
}
%    \end{macrocode}
%    \end{macro}
%
%    \begin{macro}{\BIC@Inc}
%    \begin{macrocode}
\def\BIC@Inc#1#2!#3{%
  \ifx\\#2\\%
    \BIC@AfterFi{%
      \BIC@@Inc1#1#3!{}%
    }%
  \else
    \BIC@AfterFi{%
      \BIC@Inc#2!{#1#3}%
    }%
  \BIC@Fi
}
%    \end{macrocode}
%    \end{macro}
%    \begin{macro}{\BIC@@Inc}
%    \begin{macrocode}
\def\BIC@@Inc#1#2#3!#4{%
  \ifcase#1 %
    \ifx\\#3\\%
      \BIC@AfterFiFi{ #2#4}%
    \else
      \BIC@AfterFiFi{%
        \BIC@@Inc0#3!{#2#4}%
      }%
    \fi
  \else
    \ifnum#2<9 %
      \BIC@AfterFiFi{%
&       \expandafter\BIC@@@Inc\the\numexpr#2+1\relax
$       \expandafter\expandafter\expandafter\BIC@@@Inc
$       \ifcase#2 \expandafter1%
$       \or\expandafter2%
$       \or\expandafter3%
$       \or\expandafter4%
$       \or\expandafter5%
$       \or\expandafter6%
$       \or\expandafter7%
$       \or\expandafter8%
$       \or\expandafter9%
$?      \else\BigIntCalcError:ThisCannotHappen%
$       \fi
        0#3!{#4}%
      }%
    \else
      \BIC@AfterFiFi{%
        \BIC@@@Inc01#3!{#4}%
      }%
    \fi
  \BIC@Fi
}
%    \end{macrocode}
%    \end{macro}
%    \begin{macro}{\BIC@@@Inc}
%    \begin{macrocode}
\def\BIC@@@Inc#1#2#3!#4{%
  \ifx\\#3\\%
    \ifnum#2=1 %
      \BIC@AfterFiFi{ 1#1#4}%
    \else
      \BIC@AfterFiFi{ #1#4}%
    \fi
  \else
    \BIC@AfterFi{%
      \BIC@@Inc#2#3!{#1#4}%
    }%
  \BIC@Fi
}
%    \end{macrocode}
%    \end{macro}
%
%    \begin{macro}{\BIC@Dec}
%    \begin{macrocode}
\def\BIC@Dec#1#2!#3{%
  \ifx\\#2\\%
    \BIC@AfterFi{%
      \BIC@@Dec1#1#3!{}%
    }%
  \else
    \BIC@AfterFi{%
      \BIC@Dec#2!{#1#3}%
    }%
  \BIC@Fi
}
%    \end{macrocode}
%    \end{macro}
%    \begin{macro}{\BIC@@Dec}
%    \begin{macrocode}
\def\BIC@@Dec#1#2#3!#4{%
  \ifcase#1 %
    \ifx\\#3\\%
      \BIC@AfterFiFi{ #2#4}%
    \else
      \BIC@AfterFiFi{%
        \BIC@@Dec0#3!{#2#4}%
      }%
    \fi
  \else
    \ifnum#2>0 %
      \BIC@AfterFiFi{%
&       \expandafter\BIC@@@Dec\the\numexpr#2-1\relax
$       \expandafter\expandafter\expandafter\BIC@@@Dec
$       \ifcase#2
$?        \BigIntCalcError:ThisCannotHappen%
$       \or\expandafter0%
$       \or\expandafter1%
$       \or\expandafter2%
$       \or\expandafter3%
$       \or\expandafter4%
$       \or\expandafter5%
$       \or\expandafter6%
$       \or\expandafter7%
$       \or\expandafter8%
$?      \else\BigIntCalcError:ThisCannotHappen%
$       \fi
        0#3!{#4}%
      }%
    \else
      \BIC@AfterFiFi{%
        \BIC@@@Dec91#3!{#4}%
      }%
    \fi
  \BIC@Fi
}
%    \end{macrocode}
%    \end{macro}
%    \begin{macro}{\BIC@@@Dec}
%    \begin{macrocode}
\def\BIC@@@Dec#1#2#3!#4{%
  \ifx\\#3\\%
    \ifcase#1 %
      \ifx\\#4\\%
        \BIC@AfterFiFiFi{ 0}%
      \else
        \BIC@AfterFiFiFi{ #4}%
      \fi
    \else
      \BIC@AfterFiFi{ #1#4}%
    \fi
  \else
    \BIC@AfterFi{%
      \BIC@@Dec#2#3!{#1#4}%
    }%
  \BIC@Fi
}
%    \end{macrocode}
%    \end{macro}
%
% \subsection{\op{Add}, \op{Sub}}
%
%    \begin{macro}{\bigintcalcAdd}
%    \begin{macrocode}
\def\bigintcalcAdd#1{%
  \romannumeral0%
  \expandafter\expandafter\expandafter\BIC@Add
  \bigintcalcNum{#1}!%
}
%    \end{macrocode}
%    \end{macro}
%    \begin{macro}{\BIC@Add}
%    \begin{macrocode}
\def\BIC@Add#1!#2{%
  \expandafter\expandafter\expandafter
  \BIC@AddSwitch\bigintcalcNum{#2}!#1!%
}
%    \end{macrocode}
%    \end{macro}
%    \begin{macro}{\bigintcalcSub}
%    \begin{macrocode}
\def\bigintcalcSub#1#2{%
  \romannumeral0%
  \expandafter\expandafter\expandafter\BIC@Add
  \bigintcalcNum{-#2}!{#1}%
}
%    \end{macrocode}
%    \end{macro}
%
%    \begin{macro}{\BIC@AddSwitch}
%    Decision table for \cs{BIC@AddSwitch}.
%    \begin{quote}
%    \begin{tabular}[t]{@{}|l|l|l|l|l|@{}}
%      \hline
%      $x<0$ & $y<0$ & $-x>-y$ & $-$ & $\opAdd(-x,-y)$\\
%      \cline{3-3}\cline{5-5}
%            &       & else    &     & $\opAdd(-y,-x)$\\
%      \cline{2-5}
%            & else  & $-x> y$ & $-$ & $\opSub(-x, y)$\\
%      \cline{3-5}
%            &       & $-x= y$ &     & $0$\\
%      \cline{3-5}
%            &       & else    & $+$ & $\opSub( y,-x)$\\
%      \hline
%      else  & $y<0$ & $ x>-y$ & $+$ & $\opSub( x,-y)$\\
%      \cline{3-5}
%            &       & $ x=-y$ &     & $0$\\
%      \cline{3-5}
%            &       & else    & $-$ & $\opSub(-y, x)$\\
%      \cline{2-5}
%            & else  & $ x> y$ & $+$ & $\opAdd( x, y)$\\
%      \cline{3-3}\cline{5-5}
%            &       & else    &     & $\opAdd( y, x)$\\
%      \hline
%    \end{tabular}
%    \end{quote}
%    \begin{macrocode}
\def\BIC@AddSwitch#1#2!#3#4!{%
  \ifx#1-% x < 0
    \ifx#3-% y < 0
      \expandafter-\romannumeral0%
      \ifnum\BIC@PosCmp#2!#4!=1 % -x > -y
        \BIC@AfterFiFiFi{%
          \BIC@AddXY#2!#4!!!%
        }%
      \else % -x <= -y
        \BIC@AfterFiFiFi{%
          \BIC@AddXY#4!#2!!!%
        }%
      \fi
    \else % y >= 0
      \ifcase\BIC@PosCmp#2!#3#4!% -x = y
        \BIC@AfterFiFiFi{ 0}%
      \or % -x > y
        \expandafter-\romannumeral0%
        \BIC@AfterFiFiFi{%
          \BIC@SubXY#2!#3#4!!!%
        }%
      \else % -x <= y
        \BIC@AfterFiFiFi{%
          \BIC@SubXY#3#4!#2!!!%
        }%
      \fi
    \fi
  \else % x >= 0
    \ifx#3-% y < 0
      \ifcase\BIC@PosCmp#1#2!#4!% x = -y
        \BIC@AfterFiFiFi{ 0}%
      \or % x > -y
        \BIC@AfterFiFiFi{%
          \BIC@SubXY#1#2!#4!!!%
        }%
      \else % x <= -y
        \expandafter-\romannumeral0%
        \BIC@AfterFiFiFi{%
          \BIC@SubXY#4!#1#2!!!%
        }%
      \fi
    \else % y >= 0
      \ifnum\BIC@PosCmp#1#2!#3#4!=1 % x > y
        \BIC@AfterFiFiFi{%
          \BIC@AddXY#1#2!#3#4!!!%
        }%
      \else % x <= y
        \BIC@AfterFiFiFi{%
          \BIC@AddXY#3#4!#1#2!!!%
        }%
      \fi
    \fi
  \BIC@Fi
}
%    \end{macrocode}
%    \end{macro}
%
%    \begin{macro}{\BigIntCalcAdd}
%    \begin{macrocode}
\def\BigIntCalcAdd#1!#2!{%
  \romannumeral0\BIC@AddXY#1!#2!!!%
}
%    \end{macrocode}
%    \end{macro}
%    \begin{macro}{\BigIntCalcSub}
%    \begin{macrocode}
\def\BigIntCalcSub#1!#2!{%
  \romannumeral0\BIC@SubXY#1!#2!!!%
}
%    \end{macrocode}
%    \end{macro}
%
%    \begin{macro}{\BIC@AddXY}
%    \begin{macrocode}
\def\BIC@AddXY#1#2!#3#4!#5!#6!{%
  \ifx\\#2\\%
    \ifx\\#3\\%
      \BIC@AfterFiFi{%
        \BIC@DoAdd0!#1#5!#60!%
      }%
    \else
      \BIC@AfterFiFi{%
        \BIC@DoAdd0!#1#5!#3#6!%
      }%
    \fi
  \else
    \ifx\\#4\\%
      \ifx\\#3\\%
        \BIC@AfterFiFiFi{%
          \BIC@AddXY#2!{}!#1#5!#60!%
        }%
      \else
        \BIC@AfterFiFiFi{%
          \BIC@AddXY#2!{}!#1#5!#3#6!%
        }%
      \fi
    \else
      \BIC@AfterFiFi{%
        \BIC@AddXY#2!#4!#1#5!#3#6!%
      }%
    \fi
  \BIC@Fi
}
%    \end{macrocode}
%    \end{macro}
%    \begin{macro}{\BIC@DoAdd}
%    |#1|: carry\\
%    |#2|: reverted result\\
%    |#3#4|: reverted $x$\\
%    |#5#6|: reverted $y$
%    \begin{macrocode}
\def\BIC@DoAdd#1#2!#3#4!#5#6!{%
  \ifx\\#4\\%
    \BIC@AfterFi{%
&     \expandafter\BIC@Space
&     \the\numexpr#1+#3+#5\relax#2%
$     \expandafter\expandafter\expandafter\BIC@AddResult
$     \BIC@AddDigit#1#3#5#2%
    }%
  \else
    \BIC@AfterFi{%
      \expandafter\expandafter\expandafter\BIC@DoAdd
      \BIC@AddDigit#1#3#5#2!#4!#6!%
    }%
  \BIC@Fi
}
%    \end{macrocode}
%    \end{macro}
%    \begin{macro}{\BIC@AddResult}
%    \begin{macrocode}
$ \def\BIC@AddResult#1{%
$   \ifx#10%
$     \expandafter\BIC@Space
$   \else
$     \expandafter\BIC@Space\expandafter#1%
$   \fi
$ }%
%    \end{macrocode}
%    \end{macro}
%    \begin{macro}{\BIC@AddDigit}
%    |#1|: carry\\
%    |#2|: digit of $x$\\
%    |#3|: digit of $y$
%    \begin{macrocode}
\def\BIC@AddDigit#1#2#3{%
  \romannumeral0%
& \expandafter\BIC@@AddDigit\the\numexpr#1+#2+#3!%
$ \expandafter\BIC@@AddDigit\number%
$ \csname
$   BIC@AddCarry%
$   \ifcase#1 %
$     #2%
$   \else
$     \ifcase#2 1\or2\or3\or4\or5\or6\or7\or8\or9\or10\fi
$   \fi
$ \endcsname#3!%
}
%    \end{macrocode}
%    \end{macro}
%    \begin{macro}{\BIC@@AddDigit}
%    \begin{macrocode}
\def\BIC@@AddDigit#1!{%
  \ifnum#1<10 %
    \BIC@AfterFi{ 0#1}%
  \else
    \BIC@AfterFi{ #1}%
  \BIC@Fi
}
%    \end{macrocode}
%    \end{macro}
%    \begin{macro}{\BIC@AddCarry0}
%    \begin{macrocode}
$ \expandafter\def\csname BIC@AddCarry0\endcsname#1{#1}%
%    \end{macrocode}
%    \end{macro}
%    \begin{macro}{\BIC@AddCarry10}
%    \begin{macrocode}
$ \expandafter\def\csname BIC@AddCarry10\endcsname#1{1#1}%
%    \end{macrocode}
%    \end{macro}
%    \begin{macro}{\BIC@AddCarry[1-9]}
%    \begin{macrocode}
$ \def\BIC@Temp#1#2{%
$   \expandafter\def\csname BIC@AddCarry#1\endcsname##1{%
$     \ifcase##1 #1\or
$     #2%
$?    \else\BigIntCalcError:ThisCannotHappen%
$     \fi
$   }%
$ }%
$ \BIC@Temp 0{1\or2\or3\or4\or5\or6\or7\or8\or9}%
$ \BIC@Temp 1{2\or3\or4\or5\or6\or7\or8\or9\or10}%
$ \BIC@Temp 2{3\or4\or5\or6\or7\or8\or9\or10\or11}%
$ \BIC@Temp 3{4\or5\or6\or7\or8\or9\or10\or11\or12}%
$ \BIC@Temp 4{5\or6\or7\or8\or9\or10\or11\or12\or13}%
$ \BIC@Temp 5{6\or7\or8\or9\or10\or11\or12\or13\or14}%
$ \BIC@Temp 6{7\or8\or9\or10\or11\or12\or13\or14\or15}%
$ \BIC@Temp 7{8\or9\or10\or11\or12\or13\or14\or15\or16}%
$ \BIC@Temp 8{9\or10\or11\or12\or13\or14\or15\or16\or17}%
$ \BIC@Temp 9{10\or11\or12\or13\or14\or15\or16\or17\or18}%
%    \end{macrocode}
%    \end{macro}
%
%    \begin{macro}{\BIC@SubXY}
%    Preconditions:
%    \begin{itemize}
%    \item $x > y$, $x \geq 0$, and $y >= 0$
%    \item digits($x$) = digits($y$)
%    \end{itemize}
%    \begin{macrocode}
\def\BIC@SubXY#1#2!#3#4!#5!#6!{%
  \ifx\\#2\\%
    \ifx\\#3\\%
      \BIC@AfterFiFi{%
        \BIC@DoSub0!#1#5!#60!%
      }%
    \else
      \BIC@AfterFiFi{%
        \BIC@DoSub0!#1#5!#3#6!%
      }%
    \fi
  \else
    \ifx\\#4\\%
      \ifx\\#3\\%
        \BIC@AfterFiFiFi{%
          \BIC@SubXY#2!{}!#1#5!#60!%
        }%
      \else
        \BIC@AfterFiFiFi{%
          \BIC@SubXY#2!{}!#1#5!#3#6!%
        }%
      \fi
    \else
      \BIC@AfterFiFi{%
        \BIC@SubXY#2!#4!#1#5!#3#6!%
      }%
    \fi
  \BIC@Fi
}
%    \end{macrocode}
%    \end{macro}
%    \begin{macro}{\BIC@DoSub}
%    |#1|: carry\\
%    |#2|: reverted result\\
%    |#3#4|: reverted x\\
%    |#5#6|: reverted y
%    \begin{macrocode}
\def\BIC@DoSub#1#2!#3#4!#5#6!{%
  \ifx\\#4\\%
    \BIC@AfterFi{%
      \expandafter\expandafter\expandafter\BIC@SubResult
      \BIC@SubDigit#1#3#5#2%
    }%
  \else
    \BIC@AfterFi{%
      \expandafter\expandafter\expandafter\BIC@DoSub
      \BIC@SubDigit#1#3#5#2!#4!#6!%
    }%
  \BIC@Fi
}
%    \end{macrocode}
%    \end{macro}
%    \begin{macro}{\BIC@SubResult}
%    \begin{macrocode}
\def\BIC@SubResult#1{%
  \ifx#10%
    \expandafter\BIC@SubResult
  \else
    \expandafter\BIC@Space\expandafter#1%
  \fi
}
%    \end{macrocode}
%    \end{macro}
%    \begin{macro}{\BIC@SubDigit}
%    |#1|: carry\\
%    |#2|: digit of $x$\\
%    |#3|: digit of $y$
%    \begin{macrocode}
\def\BIC@SubDigit#1#2#3{%
  \romannumeral0%
& \expandafter\BIC@@SubDigit\the\numexpr#2-#3-#1!%
$ \expandafter\BIC@@AddDigit\number
$   \csname
$     BIC@SubCarry%
$     \ifcase#1 %
$       #3%
$     \else
$       \ifcase#3 1\or2\or3\or4\or5\or6\or7\or8\or9\or10\fi
$     \fi
$   \endcsname#2!%
}
%    \end{macrocode}
%    \end{macro}
%    \begin{macro}{\BIC@@SubDigit}
%    \begin{macrocode}
& \def\BIC@@SubDigit#1!{%
&   \ifnum#1<0 %
&     \BIC@AfterFi{%
&       \expandafter\BIC@Space
&       \expandafter1\the\numexpr#1+10\relax
&     }%
&   \else
&     \BIC@AfterFi{ 0#1}%
&   \BIC@Fi
& }%
%    \end{macrocode}
%    \end{macro}
%    \begin{macro}{\BIC@SubCarry0}
%    \begin{macrocode}
$ \expandafter\def\csname BIC@SubCarry0\endcsname#1{#1}%
%    \end{macrocode}
%    \end{macro}
%    \begin{macro}{\BIC@SubCarry10}%
%    \begin{macrocode}
$ \expandafter\def\csname BIC@SubCarry10\endcsname#1{1#1}%
%    \end{macrocode}
%    \end{macro}
%    \begin{macro}{\BIC@SubCarry[1-9]}
%    \begin{macrocode}
$ \def\BIC@Temp#1#2{%
$   \expandafter\def\csname BIC@SubCarry#1\endcsname##1{%
$     \ifcase##1 #2%
$?    \else\BigIntCalcError:ThisCannotHappen%
$     \fi
$   }%
$ }%
$ \BIC@Temp 1{19\or0\or1\or2\or3\or4\or5\or6\or7\or8}%
$ \BIC@Temp 2{18\or19\or0\or1\or2\or3\or4\or5\or6\or7}%
$ \BIC@Temp 3{17\or18\or19\or0\or1\or2\or3\or4\or5\or6}%
$ \BIC@Temp 4{16\or17\or18\or19\or0\or1\or2\or3\or4\or5}%
$ \BIC@Temp 5{15\or16\or17\or18\or19\or0\or1\or2\or3\or4}%
$ \BIC@Temp 6{14\or15\or16\or17\or18\or19\or0\or1\or2\or3}%
$ \BIC@Temp 7{13\or14\or15\or16\or17\or18\or19\or0\or1\or2}%
$ \BIC@Temp 8{12\or13\or14\or15\or16\or17\or18\or19\or0\or1}%
$ \BIC@Temp 9{11\or12\or13\or14\or15\or16\or17\or18\or19\or0}%
%    \end{macrocode}
%    \end{macro}
%
% \subsection{\op{Shl}, \op{Shr}}
%
%    \begin{macro}{\bigintcalcShl}
%    \begin{macrocode}
\def\bigintcalcShl#1{%
  \romannumeral0%
  \expandafter\expandafter\expandafter\BIC@Shl
  \bigintcalcNum{#1}!%
}
%    \end{macrocode}
%    \end{macro}
%    \begin{macro}{\BIC@Shl}
%    \begin{macrocode}
\def\BIC@Shl#1#2!{%
  \ifx#1-%
    \BIC@AfterFi{%
      \expandafter-\romannumeral0%
&     \BIC@@Shl#2!!%
$     \BIC@AddXY#2!#2!!!%
    }%
  \else
    \BIC@AfterFi{%
&     \BIC@@Shl#1#2!!%
$     \BIC@AddXY#1#2!#1#2!!!%
    }%
  \BIC@Fi
}
%    \end{macrocode}
%    \end{macro}
%    \begin{macro}{\BigIntCalcShl}
%    \begin{macrocode}
\def\BigIntCalcShl#1!{%
  \romannumeral0%
& \BIC@@Shl#1!!%
$ \BIC@AddXY#1!#1!!!%
}
%    \end{macrocode}
%    \end{macro}
%    \begin{macro}{\BIC@@Shl}
%    \begin{macrocode}
& \def\BIC@@Shl#1#2!{%
&   \ifx\\#2\\%
&     \BIC@AfterFi{%
&       \BIC@@@Shl0!#1%
&     }%
&   \else
&     \BIC@AfterFi{%
&       \BIC@@Shl#2!#1%
&     }%
&   \BIC@Fi
& }%
%    \end{macrocode}
%    \end{macro}
%    \begin{macro}{\BIC@@@Shl}
%    |#1|: carry\\
%    |#2|: result\\
%    |#3#4|: reverted number
%    \begin{macrocode}
& \def\BIC@@@Shl#1#2!#3#4!{%
&   \ifx\\#4\\%
&     \BIC@AfterFi{%
&       \expandafter\BIC@Space
&       \the\numexpr#3*2+#1\relax#2%
&     }%
&   \else
&     \BIC@AfterFi{%
&       \expandafter\BIC@@@@Shl\the\numexpr#3*2+#1!#2!#4!%
&     }%
&   \BIC@Fi
& }%
%    \end{macrocode}
%    \end{macro}
%    \begin{macro}{\BIC@@@@Shl}
%    \begin{macrocode}
& \def\BIC@@@@Shl#1!{%
&   \ifnum#1<10 %
&     \BIC@AfterFi{%
&       \BIC@@@Shl0#1%
&     }%
&   \else
&     \BIC@AfterFi{%
&       \BIC@@@Shl#1%
&     }%
&   \BIC@Fi
& }%
%    \end{macrocode}
%    \end{macro}
%
%    \begin{macro}{\bigintcalcShr}
%    \begin{macrocode}
\def\bigintcalcShr#1{%
  \romannumeral0%
  \expandafter\expandafter\expandafter\BIC@Shr
  \bigintcalcNum{#1}!%
}
%    \end{macrocode}
%    \end{macro}
%    \begin{macro}{\BIC@Shr}
%    \begin{macrocode}
\def\BIC@Shr#1#2!{%
  \ifx#1-%
    \expandafter-\romannumeral0%
    \BIC@AfterFi{%
      \BIC@@Shr#2!%
    }%
  \else
    \BIC@AfterFi{%
      \BIC@@Shr#1#2!%
    }%
  \BIC@Fi
}
%    \end{macrocode}
%    \end{macro}
%    \begin{macro}{\BigIntCalcShr}
%    \begin{macrocode}
\def\BigIntCalcShr#1!{%
  \romannumeral0%
  \BIC@@Shr#1!%
}
%    \end{macrocode}
%    \end{macro}
%    \begin{macro}{\BIC@@Shr}
%    \begin{macrocode}
\def\BIC@@Shr#1#2!{%
  \ifcase#1 %
    \BIC@AfterFi{ 0}%
  \or
    \ifx\\#2\\%
      \BIC@AfterFiFi{ 0}%
    \else
      \BIC@AfterFiFi{%
        \BIC@@@Shr#1#2!!%
      }%
    \fi
  \else
    \BIC@AfterFi{%
      \BIC@@@Shr0#1#2!!%
    }%
  \BIC@Fi
}
%    \end{macrocode}
%    \end{macro}
%    \begin{macro}{\BIC@@@Shr}
%    |#1|: carry\\
%    |#2#3|: number\\
%    |#4|: result
%    \begin{macrocode}
\def\BIC@@@Shr#1#2#3!#4!{%
  \ifx\\#3\\%
    \ifodd#1#2 %
      \BIC@AfterFiFi{%
&       \expandafter\BIC@ShrResult\the\numexpr(#1#2-1)/2\relax
$       \expandafter\expandafter\expandafter\BIC@ShrResult
$       \csname BIC@ShrDigit#1#2\endcsname
        #4!%
      }%
    \else
      \BIC@AfterFiFi{%
&       \expandafter\BIC@ShrResult\the\numexpr#1#2/2\relax
$       \expandafter\expandafter\expandafter\BIC@ShrResult
$       \csname BIC@ShrDigit#1#2\endcsname
        #4!%
      }%
    \fi
  \else
    \ifodd#1#2 %
      \BIC@AfterFiFi{%
&       \expandafter\BIC@@@@Shr\the\numexpr(#1#2-1)/2\relax1%
$       \expandafter\expandafter\expandafter\BIC@@@@Shr
$       \csname BIC@ShrDigit#1#2\endcsname
        #3!#4!%
      }%
    \else
      \BIC@AfterFiFi{%
&       \expandafter\BIC@@@@Shr\the\numexpr#1#2/2\relax0%
$       \expandafter\expandafter\expandafter\BIC@@@@Shr
$       \csname BIC@ShrDigit#1#2\endcsname
        #3!#4!%
      }%
    \fi
  \BIC@Fi
}
%    \end{macrocode}
%    \end{macro}
%    \begin{macro}{\BIC@ShrResult}
%    \begin{macrocode}
& \def\BIC@ShrResult#1#2!{ #2#1}%
$ \def\BIC@ShrResult#1#2#3!{ #3#1}%
%    \end{macrocode}
%    \end{macro}
%    \begin{macro}{\BIC@@@@Shr}
%    |#1|: new digit\\
%    |#2|: carry\\
%    |#3|: remaining number\\
%    |#4|: result
%    \begin{macrocode}
\def\BIC@@@@Shr#1#2#3!#4!{%
  \BIC@@@Shr#2#3!#4#1!%
}
%    \end{macrocode}
%    \end{macro}
%    \begin{macro}{\BIC@ShrDigit[00-19]}
%    \begin{macrocode}
$ \def\BIC@Temp#1#2#3#4{%
$   \expandafter\def\csname BIC@ShrDigit#1#2\endcsname{#3#4}%
$ }%
$ \BIC@Temp 0000%
$ \BIC@Temp 0101%
$ \BIC@Temp 0210%
$ \BIC@Temp 0311%
$ \BIC@Temp 0420%
$ \BIC@Temp 0521%
$ \BIC@Temp 0630%
$ \BIC@Temp 0731%
$ \BIC@Temp 0840%
$ \BIC@Temp 0941%
$ \BIC@Temp 1050%
$ \BIC@Temp 1151%
$ \BIC@Temp 1260%
$ \BIC@Temp 1361%
$ \BIC@Temp 1470%
$ \BIC@Temp 1571%
$ \BIC@Temp 1680%
$ \BIC@Temp 1781%
$ \BIC@Temp 1890%
$ \BIC@Temp 1991%
%    \end{macrocode}
%    \end{macro}
%
% \subsection{\cs{BIC@Tim}}
%
%    \begin{macro}{\BIC@Tim}
%    Macro \cs{BIC@Tim} implements ``Number \emph{tim}es digit''.\\
%    |#1|: plain number without sign\\
%    |#2|: digit
\def\BIC@Tim#1!#2{%
  \romannumeral0%
  \ifcase#2 % 0
    \BIC@AfterFi{ 0}%
  \or % 1
    \BIC@AfterFi{ #1}%
  \or % 2
    \BIC@AfterFi{%
      \BIC@Shl#1!%
    }%
  \else % 3-9
    \BIC@AfterFi{%
      \BIC@@Tim#1!!#2%
    }%
  \BIC@Fi
}
%    \begin{macrocode}
%    \end{macrocode}
%    \end{macro}
%    \begin{macro}{\BIC@@Tim}
%    |#1#2|: number\\
%    |#3|: reverted number
%    \begin{macrocode}
\def\BIC@@Tim#1#2!{%
  \ifx\\#2\\%
    \BIC@AfterFi{%
      \BIC@ProcessTim0!#1%
    }%
  \else
    \BIC@AfterFi{%
      \BIC@@Tim#2!#1%
    }%
  \BIC@Fi
}
%    \end{macrocode}
%    \end{macro}
%    \begin{macro}{\BIC@ProcessTim}
%    |#1|: carry\\
%    |#2|: result\\
%    |#3#4|: reverted number\\
%    |#5|: digit
%    \begin{macrocode}
\def\BIC@ProcessTim#1#2!#3#4!#5{%
  \ifx\\#4\\%
    \BIC@AfterFi{%
      \expandafter\BIC@Space
&     \the\numexpr#3*#5+#1\relax
$     \romannumeral0\BIC@TimDigit#3#5#1%
      #2%
    }%
  \else
    \BIC@AfterFi{%
      \expandafter\BIC@@ProcessTim
&     \the\numexpr#3*#5+#1%
$     \romannumeral0\BIC@TimDigit#3#5#1%
      !#2!#4!#5%
    }%
  \BIC@Fi
}
%    \end{macrocode}
%    \end{macro}
%    \begin{macro}{\BIC@@ProcessTim}
%    |#1#2|: carry?, new digit\\
%    |#3|: new number\\
%    |#4|: old number\\
%    |#5|: digit
%    \begin{macrocode}
\def\BIC@@ProcessTim#1#2!{%
  \ifx\\#2\\%
    \BIC@AfterFi{%
      \BIC@ProcessTim0#1%
    }%
  \else
    \BIC@AfterFi{%
      \BIC@ProcessTim#1#2%
    }%
  \BIC@Fi
}
%    \end{macrocode}
%    \end{macro}
%    \begin{macro}{\BIC@TimDigit}
%    |#1|: digit 0--9\\
%    |#2|: digit 3--9\\
%    |#3|: carry 0--9
%    \begin{macrocode}
$ \def\BIC@TimDigit#1#2#3{%
$   \ifcase#1 % 0
$     \BIC@AfterFi{ #3}%
$   \or % 1
$     \BIC@AfterFi{%
$       \expandafter\BIC@Space
$       \number\csname BIC@AddCarry#2\endcsname#3 %
$     }%
$   \else
$     \ifcase#3 %
$       \BIC@AfterFiFi{%
$         \expandafter\BIC@Space
$         \number\csname BIC@MulDigit#2\endcsname#1 %
$       }%
$     \else
$       \BIC@AfterFiFi{%
$         \expandafter\BIC@Space
$         \romannumeral0%
$         \expandafter\BIC@AddXY
$         \number\csname BIC@MulDigit#2\endcsname#1!%
$         #3!!!%
$       }%
$     \fi
$   \BIC@Fi
$ }%
%    \end{macrocode}
%    \end{macro}
%    \begin{macro}{\BIC@MulDigit[3-9]}
%    \begin{macrocode}
$ \def\BIC@Temp#1#2{%
$   \expandafter\def\csname BIC@MulDigit#1\endcsname##1{%
$     \ifcase##1 0%
$     \or ##1%
$     \or #2%
$?    \else\BigIntCalcError:ThisCannotHappen%
$     \fi
$   }%
$ }%
$ \BIC@Temp 3{6\or9\or12\or15\or18\or21\or24\or27}%
$ \BIC@Temp 4{8\or12\or16\or20\or24\or28\or32\or36}%
$ \BIC@Temp 5{10\or15\or20\or25\or30\or35\or40\or45}%
$ \BIC@Temp 6{12\or18\or24\or30\or36\or42\or48\or54}%
$ \BIC@Temp 7{14\or21\or28\or35\or42\or49\or56\or63}%
$ \BIC@Temp 8{16\or24\or32\or40\or48\or56\or64\or72}%
$ \BIC@Temp 9{18\or27\or36\or45\or54\or63\or72\or81}%
%    \end{macrocode}
%    \end{macro}
%
% \subsection{\op{Mul}}
%
%    \begin{macro}{\bigintcalcMul}
%    \begin{macrocode}
\def\bigintcalcMul#1#2{%
  \romannumeral0%
  \expandafter\expandafter\expandafter\BIC@Mul
  \bigintcalcNum{#1}!{#2}%
}
%    \end{macrocode}
%    \end{macro}
%    \begin{macro}{\BIC@Mul}
%    \begin{macrocode}
\def\BIC@Mul#1!#2{%
  \expandafter\expandafter\expandafter\BIC@MulSwitch
  \bigintcalcNum{#2}!#1!%
}
%    \end{macrocode}
%    \end{macro}
%    \begin{macro}{\BIC@MulSwitch}
%    Decision table for \cs{BIC@MulSwitch}.
%    \begin{quote}
%    \begin{tabular}[t]{@{}|l|l|l|l|l|@{}}
%      \hline
%      $x=0$ &\multicolumn{3}{l}{}    &$0$\\
%      \hline
%      $x>0$ & $y=0$ &\multicolumn2l{}&$0$\\
%      \cline{2-5}
%            & $y>0$ & $ x> y$ & $+$ & $\opMul( x, y)$\\
%      \cline{3-3}\cline{5-5}
%            &       & else    &     & $\opMul( y, x)$\\
%      \cline{2-5}
%            & $y<0$ & $ x>-y$ & $-$ & $\opMul( x,-y)$\\
%      \cline{3-3}\cline{5-5}
%            &       & else    &     & $\opMul(-y, x)$\\
%      \hline
%      $x<0$ & $y=0$ &\multicolumn2l{}&$0$\\
%      \cline{2-5}
%            & $y>0$ & $-x> y$ & $-$ & $\opMul(-x, y)$\\
%      \cline{3-3}\cline{5-5}
%            &       & else    &     & $\opMul( y,-x)$\\
%      \cline{2-5}
%            & $y<0$ & $-x>-y$ & $+$ & $\opMul(-x,-y)$\\
%      \cline{3-3}\cline{5-5}
%            &       & else    &     & $\opMul(-y,-x)$\\
%      \hline
%    \end{tabular}
%    \end{quote}
%    \begin{macrocode}
\def\BIC@MulSwitch#1#2!#3#4!{%
  \ifcase\BIC@Sgn#1#2! % x = 0
    \BIC@AfterFi{ 0}%
  \or % x > 0
    \ifcase\BIC@Sgn#3#4! % y = 0
      \BIC@AfterFiFi{ 0}%
    \or % y > 0
      \ifnum\BIC@PosCmp#1#2!#3#4!=1 % x > y
        \BIC@AfterFiFiFi{%
          \BIC@ProcessMul0!#1#2!#3#4!%
        }%
      \else % x <= y
        \BIC@AfterFiFiFi{%
          \BIC@ProcessMul0!#3#4!#1#2!%
        }%
      \fi
    \else % y < 0
      \expandafter-\romannumeral0%
      \ifnum\BIC@PosCmp#1#2!#4!=1 % x > -y
        \BIC@AfterFiFiFi{%
          \BIC@ProcessMul0!#1#2!#4!%
        }%
      \else % x <= -y
        \BIC@AfterFiFiFi{%
          \BIC@ProcessMul0!#4!#1#2!%
        }%
      \fi
    \fi
  \else % x < 0
    \ifcase\BIC@Sgn#3#4! % y = 0
      \BIC@AfterFiFi{ 0}%
    \or % y > 0
      \expandafter-\romannumeral0%
      \ifnum\BIC@PosCmp#2!#3#4!=1 % -x > y
        \BIC@AfterFiFiFi{%
          \BIC@ProcessMul0!#2!#3#4!%
        }%
      \else % -x <= y
        \BIC@AfterFiFiFi{%
          \BIC@ProcessMul0!#3#4!#2!%
        }%
      \fi
    \else % y < 0
      \ifnum\BIC@PosCmp#2!#4!=1 % -x > -y
        \BIC@AfterFiFiFi{%
          \BIC@ProcessMul0!#2!#4!%
        }%
      \else % -x <= -y
        \BIC@AfterFiFiFi{%
          \BIC@ProcessMul0!#4!#2!%
        }%
      \fi
    \fi
  \BIC@Fi
}
%    \end{macrocode}
%    \end{macro}
%    \begin{macro}{\BigIntCalcMul}
%    \begin{macrocode}
\def\BigIntCalcMul#1!#2!{%
  \romannumeral0%
  \BIC@ProcessMul0!#1!#2!%
}
%    \end{macrocode}
%    \end{macro}
%    \begin{macro}{\BIC@ProcessMul}
%    |#1|: result\\
%    |#2|: number $x$\\
%    |#3#4|: number $y$
%    \begin{macrocode}
\def\BIC@ProcessMul#1!#2!#3#4!{%
  \ifx\\#4\\%
    \BIC@AfterFi{%
      \expandafter\expandafter\expandafter\BIC@Space
      \bigintcalcAdd{\BIC@Tim#2!#3}{#10}%
    }%
  \else
    \BIC@AfterFi{%
      \expandafter\expandafter\expandafter\BIC@ProcessMul
      \bigintcalcAdd{\BIC@Tim#2!#3}{#10}!#2!#4!%
    }%
  \BIC@Fi
}
%    \end{macrocode}
%    \end{macro}
%
% \subsection{\op{Sqr}}
%
%    \begin{macro}{\bigintcalcSqr}
%    \begin{macrocode}
\def\bigintcalcSqr#1{%
  \romannumeral0%
  \expandafter\expandafter\expandafter\BIC@Sqr
  \bigintcalcNum{#1}!%
}
%    \end{macrocode}
%    \end{macro}
%    \begin{macro}{\BIC@Sqr}
%    \begin{macrocode}
\def\BIC@Sqr#1{%
   \ifx#1-%
     \expandafter\BIC@@Sqr
   \else
     \expandafter\BIC@@Sqr\expandafter#1%
   \fi
}
%    \end{macrocode}
%    \end{macro}
%    \begin{macro}{\BIC@@Sqr}
%    \begin{macrocode}
\def\BIC@@Sqr#1!{%
  \BIC@ProcessMul0!#1!#1!%
}
%    \end{macrocode}
%    \end{macro}
%
% \subsection{\op{Fac}}
%
%    \begin{macro}{\bigintcalcFac}
%    \begin{macrocode}
\def\bigintcalcFac#1{%
  \romannumeral0%
  \expandafter\expandafter\expandafter\BIC@Fac
  \bigintcalcNum{#1}!%
}
%    \end{macrocode}
%    \end{macro}
%    \begin{macro}{\BIC@Fac}
%    \begin{macrocode}
\def\BIC@Fac#1#2!{%
  \ifx#1-%
    \BIC@AfterFi{ 0\BigIntCalcError:FacNegative}%
  \else
    \ifnum\BIC@PosCmp#1#2!13!<0 %
      \ifcase#1#2 %
         \BIC@AfterFiFiFi{ 1}% 0!
      \or\BIC@AfterFiFiFi{ 1}% 1!
      \or\BIC@AfterFiFiFi{ 2}% 2!
      \or\BIC@AfterFiFiFi{ 6}% 3!
      \or\BIC@AfterFiFiFi{ 24}% 4!
      \or\BIC@AfterFiFiFi{ 120}% 5!
      \or\BIC@AfterFiFiFi{ 720}% 6!
      \or\BIC@AfterFiFiFi{ 5040}% 7!
      \or\BIC@AfterFiFiFi{ 40320}% 8!
      \or\BIC@AfterFiFiFi{ 362880}% 9!
      \or\BIC@AfterFiFiFi{ 3628800}% 10!
      \or\BIC@AfterFiFiFi{ 39916800}% 11!
      \or\BIC@AfterFiFiFi{ 479001600}% 12!
?     \else\BigIntCalcError:ThisCannotHappen%
      \fi
    \else
      \BIC@AfterFiFi{%
        \BIC@ProcessFac#1#2!479001600!%
      }%
    \fi
  \BIC@Fi
}
%    \end{macrocode}
%    \end{macro}
%    \begin{macro}{\BIC@ProcessFac}
%    |#1|: $n$\\
%    |#2|: result
%    \begin{macrocode}
\def\BIC@ProcessFac#1!#2!{%
  \ifnum\BIC@PosCmp#1!12!=0 %
    \BIC@AfterFi{ #2}%
  \else
    \BIC@AfterFi{%
      \expandafter\BIC@@ProcessFac
      \romannumeral0\BIC@ProcessMul0!#2!#1!%
      !#1!%
    }%
  \BIC@Fi
}
%    \end{macrocode}
%    \end{macro}
%    \begin{macro}{\BIC@@ProcessFac}
%    |#1|: result\\
%    |#2|: $n$
%    \begin{macrocode}
\def\BIC@@ProcessFac#1!#2!{%
  \expandafter\BIC@ProcessFac
  \romannumeral0\BIC@Dec#2!{}%
  !#1!%
}
%    \end{macrocode}
%    \end{macro}
%
% \subsection{\op{Pow}}
%
%    \begin{macro}{\bigintcalcPow}
%    |#1|: basis\\
%    |#2|: power
%    \begin{macrocode}
\def\bigintcalcPow#1{%
  \romannumeral0%
  \expandafter\expandafter\expandafter\BIC@Pow
  \bigintcalcNum{#1}!%
}
%    \end{macrocode}
%    \end{macro}
%    \begin{macro}{\BIC@Pow}
%    |#1|: basis\\
%    |#2|: power
%    \begin{macrocode}
\def\BIC@Pow#1!#2{%
  \expandafter\expandafter\expandafter\BIC@PowSwitch
  \bigintcalcNum{#2}!#1!%
}
%    \end{macrocode}
%    \end{macro}
%    \begin{macro}{\BIC@PowSwitch}
%    |#1#2|: power $y$\\
%    |#3#4|: basis $x$\\
%    Decision table for \cs{BIC@PowSwitch}.
%    \begin{quote}
%    \def\M#1{\multicolumn{#1}{l}{}}%
%    \begin{tabular}[t]{@{}|l|l|l|l|@{}}
%    \hline
%    $y=0$ & \M{2}                        & $1$\\
%    \hline
%    $y=1$ & \M{2}                        & $x$\\
%    \hline
%    $y=2$ & $x<0$          & \M{1}       & $\opMul(-x,-x)$\\
%    \cline{2-4}
%          & else           & \M{1}       & $\opMul(x,x)$\\
%    \hline
%    $y<0$ & $x=0$          & \M{1}       & DivisionByZero\\
%    \cline{2-4}
%          & $x=1$          & \M{1}       & $1$\\
%    \cline{2-4}
%          & $x=-1$         & $\opODD(y)$ & $-1$\\
%    \cline{3-4}
%          &                & else        & $1$\\
%    \cline{2-4}
%          & else ($\string|x\string|>1$) & \M{1}       & $0$\\
%    \hline
%    $y>2$ & $x=0$          & \M{1}       & $0$\\
%    \cline{2-4}
%          & $x=1$          & \M{1}       & $1$\\
%    \cline{2-4}
%          & $x=-1$         & $\opODD(y)$ & $-1$\\
%    \cline{3-4}
%          &                & else        & $1$\\
%    \cline{2-4}
%          & $x<-1$ ($x<0$) & $\opODD(y)$ & $-\opPow(-x,y)$\\
%    \cline{3-4}
%          &                & else        & $\opPow(-x,y)$\\
%    \cline{2-4}
%          & else ($x>1$)   & \M{1}       & $\opPow(x,y)$\\
%    \hline
%    \end{tabular}
%    \end{quote}
%    \begin{macrocode}
\def\BIC@PowSwitch#1#2!#3#4!{%
  \ifcase\ifx\\#2\\%
           \ifx#100 % y = 0
           \else\ifx#111 % y = 1
           \else\ifx#122 % y = 2
           \else4 % y > 2
           \fi\fi\fi
         \else
           \ifx#1-3 % y < 0
           \else4 % y > 2
           \fi
         \fi
    \BIC@AfterFi{ 1}% y = 0
  \or % y = 1
    \BIC@AfterFi{ #3#4}%
  \or % y = 2
    \ifx#3-% x < 0
      \BIC@AfterFiFi{%
        \BIC@ProcessMul0!#4!#4!%
      }%
    \else % x >= 0
      \BIC@AfterFiFi{%
        \BIC@ProcessMul0!#3#4!#3#4!%
      }%
    \fi
  \or % y < 0
    \ifcase\ifx\\#4\\%
             \ifx#300 % x = 0
             \else\ifx#311 % x = 1
             \else3 % x > 1
             \fi\fi
           \else
             \ifcase\BIC@MinusOne#3#4! %
               3 % |x| > 1
             \or
               2 % x = -1
?            \else\BigIntCalcError:ThisCannotHappen%
             \fi
           \fi
      \BIC@AfterFiFi{ 0\BigIntCalcError:DivisionByZero}% x = 0
    \or % x = 1
      \BIC@AfterFiFi{ 1}% x = 1
    \or % x = -1
      \ifcase\BIC@ModTwo#2! % even(y)
        \BIC@AfterFiFiFi{ 1}%
      \or % odd(y)
        \BIC@AfterFiFiFi{ -1}%
?     \else\BigIntCalcError:ThisCannotHappen%
      \fi
    \or % |x| > 1
      \BIC@AfterFiFi{ 0}%
?   \else\BigIntCalcError:ThisCannotHappen%
    \fi
  \or % y > 2
    \ifcase\ifx\\#4\\%
             \ifx#300 % x = 0
             \else\ifx#311 % x = 1
             \else4 % x > 1
             \fi\fi
           \else
             \ifx#3-%
               \ifcase\BIC@MinusOne#3#4! %
                 3 % x < -1
               \else
                 2 % x = -1
               \fi
             \else
               4 % x > 1
             \fi
           \fi
      \BIC@AfterFiFi{ 0}% x = 0
    \or % x = 1
      \BIC@AfterFiFi{ 1}% x = 1
    \or % x = -1
      \ifcase\BIC@ModTwo#1#2! % even(y)
        \BIC@AfterFiFiFi{ 1}%
      \or % odd(y)
        \BIC@AfterFiFiFi{ -1}%
?     \else\BigIntCalcError:ThisCannotHappen%
      \fi
    \or % x < -1
      \ifcase\BIC@ModTwo#1#2! % even(y)
        \BIC@AfterFiFiFi{%
          \BIC@PowRec#4!#1#2!1!%
        }%
      \or % odd(y)
        \expandafter-\romannumeral0%
        \BIC@AfterFiFiFi{%
          \BIC@PowRec#4!#1#2!1!%
        }%
?     \else\BigIntCalcError:ThisCannotHappen%
      \fi
    \or % x > 1
      \BIC@AfterFiFi{%
        \BIC@PowRec#3#4!#1#2!1!%
      }%
?   \else\BigIntCalcError:ThisCannotHappen%
    \fi
? \else\BigIntCalcError:ThisCannotHappen%
  \BIC@Fi
}
%    \end{macrocode}
%    \end{macro}
%
% \subsubsection{Help macros}
%
%    \begin{macro}{\BIC@ModTwo}
%    Macro \cs{BIC@ModTwo} expects a number without sign
%    and returns digit |1| or |0| if the number is odd or even.
%    \begin{macrocode}
\def\BIC@ModTwo#1#2!{%
  \ifx\\#2\\%
    \ifodd#1 %
      \BIC@AfterFiFi1%
    \else
      \BIC@AfterFiFi0%
    \fi
  \else
    \BIC@AfterFi{%
      \BIC@ModTwo#2!%
    }%
  \BIC@Fi
}
%    \end{macrocode}
%    \end{macro}
%
%    \begin{macro}{\BIC@MinusOne}
%    Macro \cs{BIC@MinusOne} expects a number and returns
%    digit |1| if the number equals minus one and returns |0| otherwise.
%    \begin{macrocode}
\def\BIC@MinusOne#1#2!{%
  \ifx#1-%
    \BIC@@MinusOne#2!%
  \else
    0%
  \fi
}
%    \end{macrocode}
%    \end{macro}
%    \begin{macro}{\BIC@@MinusOne}
%    \begin{macrocode}
\def\BIC@@MinusOne#1#2!{%
  \ifx#11%
    \ifx\\#2\\%
      1%
    \else
      0%
    \fi
  \else
    0%
  \fi
}
%    \end{macrocode}
%    \end{macro}
%
% \subsubsection{Recursive calculation}
%
%    \begin{macro}{\BIC@PowRec}
%\begin{quote}
%\begin{verbatim}
%Pow(x, y) {
%  PowRec(x, y, 1)
%}
%PowRec(x, y, r) {
%  if y == 1 then
%    return r
%  else
%    ifodd y then
%      return PowRec(x*x, y div 2, r*x) % y div 2 = (y-1)/2
%    else
%      return PowRec(x*x, y div 2, r)
%    fi
%  fi
%}
%\end{verbatim}
%\end{quote}
%    |#1|: $x$ (basis)\\
%    |#2#3|: $y$ (power)\\
%    |#4|: $r$ (result)
%    \begin{macrocode}
\def\BIC@PowRec#1!#2#3!#4!{%
  \ifcase\ifx#21\ifx\\#3\\0 \else1 \fi\else1 \fi % y = 1
    \ifnum\BIC@PosCmp#1!#4!=1 % x > r
      \BIC@AfterFiFi{%
        \BIC@ProcessMul0!#1!#4!%
      }%
    \else
      \BIC@AfterFiFi{%
        \BIC@ProcessMul0!#4!#1!%
      }%
    \fi
  \or
    \ifcase\BIC@ModTwo#2#3! % even(y)
      \BIC@AfterFiFi{%
        \expandafter\BIC@@PowRec\romannumeral0%
        \BIC@@Shr#2#3!%
        !#1!#4!%
      }%
    \or % odd(y)
      \ifnum\BIC@PosCmp#1!#4!=1 % x > r
        \BIC@AfterFiFiFi{%
          \expandafter\BIC@@@PowRec\romannumeral0%
          \BIC@ProcessMul0!#1!#4!%
          !#1!#2#3!%
        }%
      \else
        \BIC@AfterFiFiFi{%
          \expandafter\BIC@@@PowRec\romannumeral0%
          \BIC@ProcessMul0!#1!#4!%
          !#1!#2#3!%
        }%
      \fi
?   \else\BigIntCalcError:ThisCannotHappen%
    \fi
? \else\BigIntCalcError:ThisCannotHappen%
  \BIC@Fi
}
%    \end{macrocode}
%    \end{macro}
%    \begin{macro}{\BIC@@PowRec}
%    |#1|: $y/2$\\
%    |#2|: $x$\\
%    |#3|: new $r$ ($r$ or $r*x$)
%    \begin{macrocode}
\def\BIC@@PowRec#1!#2!#3!{%
  \expandafter\BIC@PowRec\romannumeral0%
  \BIC@ProcessMul0!#2!#2!%
  !#1!#3!%
}
%    \end{macrocode}
%    \end{macro}
%    \begin{macro}{\BIC@@@PowRec}
%    |#1|: $r*x$
%    |#2|: $x$
%    |#3|: $y$
%    \begin{macrocode}
\def\BIC@@@PowRec#1!#2!#3!{%
  \expandafter\BIC@@PowRec\romannumeral0%
  \BIC@@Shr#3!%
  !#2!#1!%
}
%    \end{macrocode}
%    \end{macro}
%
% \subsection{\op{Div}}
%
%    \begin{macro}{\bigintcalcDiv}
%    |#1|: $x$\\
%    |#2|: $y$ (divisor)
%    \begin{macrocode}
\def\bigintcalcDiv#1{%
  \romannumeral0%
  \expandafter\expandafter\expandafter\BIC@Div
  \bigintcalcNum{#1}!%
}
%    \end{macrocode}
%    \end{macro}
%    \begin{macro}{\BIC@Div}
%    |#1|: $x$\\
%    |#2|: $y$
%    \begin{macrocode}
\def\BIC@Div#1!#2{%
  \expandafter\expandafter\expandafter\BIC@DivSwitchSign
  \bigintcalcNum{#2}!#1!%
}
%    \end{macrocode}
%    \end{macro}
%    \begin{macro}{\BigIntCalcDiv}
%    \begin{macrocode}
\def\BigIntCalcDiv#1!#2!{%
  \romannumeral0%
  \BIC@DivSwitchSign#2!#1!%
}
%    \end{macrocode}
%    \end{macro}
%    \begin{macro}{\BIC@DivSwitchSign}
%    Decision table for \cs{BIC@DivSwitchSign}.
%    \begin{quote}
%    \begin{tabular}{@{}|l|l|l|@{}}
%    \hline
%    $y=0$ & \multicolumn{1}{l}{} & DivisionByZero\\
%    \hline
%    $y>0$ & $x=0$ & $0$\\
%    \cline{2-3}
%          & $x>0$ & DivSwitch$(+,x,y)$\\
%    \cline{2-3}
%          & $x<0$ & DivSwitch$(-,-x,y)$\\
%    \hline
%    $y<0$ & $x=0$ & $0$\\
%    \cline{2-3}
%          & $x>0$ & DivSwitch$(-,x,-y)$\\
%    \cline{2-3}
%          & $x<0$ & DivSwitch$(+,-x,-y)$\\
%    \hline
%    \end{tabular}
%    \end{quote}
%    |#1|: $y$ (divisor)\\
%    |#2|: $x$
%    \begin{macrocode}
\def\BIC@DivSwitchSign#1#2!#3#4!{%
  \ifcase\BIC@Sgn#1#2! % y = 0
    \BIC@AfterFi{ 0\BigIntCalcError:DivisionByZero}%
  \or % y > 0
    \ifcase\BIC@Sgn#3#4! % x = 0
      \BIC@AfterFiFi{ 0}%
    \or % x > 0
      \BIC@AfterFiFi{%
        \BIC@DivSwitch{}#3#4!#1#2!%
      }%
    \else % x < 0
      \BIC@AfterFiFi{%
        \BIC@DivSwitch-#4!#1#2!%
      }%
    \fi
  \else % y < 0
    \ifcase\BIC@Sgn#3#4! % x = 0
      \BIC@AfterFiFi{ 0}%
    \or % x > 0
      \BIC@AfterFiFi{%
        \BIC@DivSwitch-#3#4!#2!%
      }%
    \else % x < 0
      \BIC@AfterFiFi{%
        \BIC@DivSwitch{}#4!#2!%
      }%
    \fi
  \BIC@Fi
}
%    \end{macrocode}
%    \end{macro}
%    \begin{macro}{\BIC@DivSwitch}
%    Decision table for \cs{BIC@DivSwitch}.
%    \begin{quote}
%    \begin{tabular}{@{}|l|l|l|@{}}
%    \hline
%    $y=x$ & \multicolumn{1}{l}{} & sign $1$\\
%    \hline
%    $y>x$ & \multicolumn{1}{l}{} & $0$\\
%    \hline
%    $y<x$ & $y=1$                & sign $x$\\
%    \cline{2-3}
%          & $y=2$                & sign Shr$(x)$\\
%    \cline{2-3}
%          & $y=4$                & sign Shr(Shr$(x)$)\\
%    \cline{2-3}
%          & else                 & sign ProcessDiv$(x,y)$\\
%    \hline
%    \end{tabular}
%    \end{quote}
%    |#1|: sign\\
%    |#2|: $x$\\
%    |#3#4|: $y$ ($y\ne 0$)
%    \begin{macrocode}
\def\BIC@DivSwitch#1#2!#3#4!{%
  \ifcase\BIC@PosCmp#3#4!#2!% y = x
    \BIC@AfterFi{ #11}%
  \or % y > x
    \BIC@AfterFi{ 0}%
  \else % y < x
    \ifx\\#1\\%
    \else
      \expandafter-\romannumeral0%
    \fi
    \ifcase\ifx\\#4\\%
             \ifx#310 % y = 1
             \else\ifx#321 % y = 2
             \else\ifx#342 % y = 4
             \else3 % y > 2
             \fi\fi\fi
           \else
             3 % y > 2
           \fi
      \BIC@AfterFiFi{ #2}% y = 1
    \or % y = 2
      \BIC@AfterFiFi{%
        \BIC@@Shr#2!%
      }%
    \or % y = 4
      \BIC@AfterFiFi{%
        \expandafter\BIC@@Shr\romannumeral0%
          \BIC@@Shr#2!!%
      }%
    \or % y > 2
      \BIC@AfterFiFi{%
        \BIC@DivStartX#2!#3#4!!!%
      }%
?   \else\BigIntCalcError:ThisCannotHappen%
    \fi
  \BIC@Fi
}
%    \end{macrocode}
%    \end{macro}
%    \begin{macro}{\BIC@ProcessDiv}
%    |#1#2|: $x$\\
%    |#3#4|: $y$\\
%    |#5|: collect first digits of $x$\\
%    |#6|: corresponding digits of $y$
%    \begin{macrocode}
\def\BIC@DivStartX#1#2!#3#4!#5!#6!{%
  \ifx\\#4\\%
    \BIC@AfterFi{%
      \BIC@DivStartYii#6#3#4!{#5#1}#2=!%
    }%
  \else
    \BIC@AfterFi{%
      \BIC@DivStartX#2!#4!#5#1!#6#3!%
    }%
  \BIC@Fi
}
%    \end{macrocode}
%    \end{macro}
%    \begin{macro}{\BIC@DivStartYii}
%    |#1|: $y$\\
%    |#2|: $x$, |=|
%    \begin{macrocode}
\def\BIC@DivStartYii#1!{%
  \expandafter\BIC@DivStartYiv\romannumeral0%
  \BIC@Shl#1!%
  !#1!%
}
%    \end{macrocode}
%    \end{macro}
%    \begin{macro}{\BIC@DivStartYiv}
%    |#1|: $2y$\\
%    |#2|: $y$\\
%    |#3|: $x$, |=|
%    \begin{macrocode}
\def\BIC@DivStartYiv#1!{%
  \expandafter\BIC@DivStartYvi\romannumeral0%
  \BIC@Shl#1!%
  !#1!%
}
%    \end{macrocode}
%    \end{macro}
%    \begin{macro}{\BIC@DivStartYvi}
%    |#1|: $4y$\\
%    |#2|: $2y$\\
%    |#3|: $y$\\
%    |#4|: $x$, |=|
%    \begin{macrocode}
\def\BIC@DivStartYvi#1!#2!{%
  \expandafter\BIC@DivStartYviii\romannumeral0%
  \BIC@AddXY#1!#2!!!%
  !#1!#2!%
}
%    \end{macrocode}
%    \end{macro}
%    \begin{macro}{\BIC@DivStartYviii}
%    |#1|: $6y$\\
%    |#2|: $4y$\\
%    |#3|: $2y$\\
%    |#4|: $y$\\
%    |#5|: $x$, |=|
%    \begin{macrocode}
\def\BIC@DivStartYviii#1!#2!{%
  \expandafter\BIC@DivStart\romannumeral0%
  \BIC@Shl#2!%
  !#1!#2!%
}
%    \end{macrocode}
%    \end{macro}
%    \begin{macro}{\BIC@DivStart}
%    |#1|: $8y$\\
%    |#2|: $6y$\\
%    |#3|: $4y$\\
%    |#4|: $2y$\\
%    |#5|: $y$\\
%    |#6|: $x$, |=|
%    \begin{macrocode}
\def\BIC@DivStart#1!#2!#3!#4!#5!#6!{%
  \BIC@ProcessDiv#6!!#5!#4!#3!#2!#1!=%
}
%    \end{macrocode}
%    \end{macro}
%    \begin{macro}{\BIC@ProcessDiv}
%    |#1#2#3|: $x$, |=|\\
%    |#4|: result\\
%    |#5|: $y$\\
%    |#6|: $2y$\\
%    |#7|: $4y$\\
%    |#8|: $6y$\\
%    |#9|: $8y$
%    \begin{macrocode}
\def\BIC@ProcessDiv#1#2#3!#4!#5!{%
  \ifcase\BIC@PosCmp#5!#1!% y = #1
    \ifx#2=%
      \BIC@AfterFiFi{\BIC@DivCleanup{#41}}%
    \else
      \BIC@AfterFiFi{%
        \BIC@ProcessDiv#2#3!#41!#5!%
      }%
    \fi
  \or % y > #1
    \ifx#2=%
      \BIC@AfterFiFi{\BIC@DivCleanup{#40}}%
    \else
      \ifx\\#4\\%
        \BIC@AfterFiFiFi{%
          \BIC@ProcessDiv{#1#2}#3!!#5!%
        }%
      \else
        \BIC@AfterFiFiFi{%
          \BIC@ProcessDiv{#1#2}#3!#40!#5!%
        }%
      \fi
    \fi
  \else % y < #1
    \BIC@AfterFi{%
      \BIC@@ProcessDiv{#1}#2#3!#4!#5!%
    }%
  \BIC@Fi
}
%    \end{macrocode}
%    \end{macro}
%    \begin{macro}{\BIC@DivCleanup}
%    |#1|: result\\
%    |#2|: garbage
%    \begin{macrocode}
\def\BIC@DivCleanup#1#2={ #1}%
%    \end{macrocode}
%    \end{macro}
%    \begin{macro}{\BIC@@ProcessDiv}
%    \begin{macrocode}
\def\BIC@@ProcessDiv#1#2#3!#4!#5!#6!#7!{%
  \ifcase\BIC@PosCmp#7!#1!% 4y = #1
    \ifx#2=%
      \BIC@AfterFiFi{\BIC@DivCleanup{#44}}%
    \else
     \BIC@AfterFiFi{%
       \BIC@ProcessDiv#2#3!#44!#5!#6!#7!%
     }%
    \fi
  \or % 4y > #1
    \ifcase\BIC@PosCmp#6!#1!% 2y = #1
      \ifx#2=%
        \BIC@AfterFiFiFi{\BIC@DivCleanup{#42}}%
      \else
        \BIC@AfterFiFiFi{%
          \BIC@ProcessDiv#2#3!#42!#5!#6!#7!%
        }%
      \fi
    \or % 2y > #1
      \ifx#2=%
        \BIC@AfterFiFiFi{\BIC@DivCleanup{#41}}%
      \else
        \BIC@AfterFiFiFi{%
          \BIC@DivSub#1!#5!#2#3!#41!#5!#6!#7!%
        }%
      \fi
    \else % 2y < #1
      \BIC@AfterFiFi{%
        \expandafter\BIC@ProcessDivII\romannumeral0%
        \BIC@SubXY#1!#6!!!%
        !#2#3!#4!#5!23%
        #6!#7!%
      }%
    \fi
  \else % 4y < #1
    \BIC@AfterFi{%
      \BIC@@@ProcessDiv{#1}#2#3!#4!#5!#6!#7!%
    }%
  \BIC@Fi
}
%    \end{macrocode}
%    \end{macro}
%    \begin{macro}{\BIC@DivSub}
%    Next token group: |#1|-|#2| and next digit |#3|.
%    \begin{macrocode}
\def\BIC@DivSub#1!#2!#3{%
  \expandafter\BIC@ProcessDiv\expandafter{%
    \romannumeral0%
    \BIC@SubXY#1!#2!!!%
    #3%
  }%
}
%    \end{macrocode}
%    \end{macro}
%    \begin{macro}{\BIC@ProcessDivII}
%    |#1|: $x'-2y$\\
%    |#2#3|: remaining $x$, |=|\\
%    |#4|: result\\
%    |#5|: $y$\\
%    |#6|: first possible result digit\\
%    |#7|: second possible result digit
%    \begin{macrocode}
\def\BIC@ProcessDivII#1!#2#3!#4!#5!#6#7{%
  \ifcase\BIC@PosCmp#5!#1!% y = #1
    \ifx#2=%
      \BIC@AfterFiFi{\BIC@DivCleanup{#4#7}}%
    \else
      \BIC@AfterFiFi{%
        \BIC@ProcessDiv#2#3!#4#7!#5!%
      }%
    \fi
  \or % y > #1
    \ifx#2=%
      \BIC@AfterFiFi{\BIC@DivCleanup{#4#6}}%
    \else
      \BIC@AfterFiFi{%
        \BIC@ProcessDiv{#1#2}#3!#4#6!#5!%
      }%
    \fi
  \else % y < #1
    \ifx#2=%
      \BIC@AfterFiFi{\BIC@DivCleanup{#4#7}}%
    \else
      \BIC@AfterFiFi{%
        \BIC@DivSub#1!#5!#2#3!#4#7!#5!%
      }%
    \fi
  \BIC@Fi
}
%    \end{macrocode}
%    \end{macro}
%    \begin{macro}{\BIC@ProcessDivIV}
%    |#1#2#3|: $x$, |=|, $x>4y$\\
%    |#4|: result\\
%    |#5|: $y$\\
%    |#6|: $2y$\\
%    |#7|: $4y$\\
%    |#8|: $6y$\\
%    |#9|: $8y$
%    \begin{macrocode}
\def\BIC@@@ProcessDiv#1#2#3!#4!#5!#6!#7!#8!#9!{%
  \ifcase\BIC@PosCmp#8!#1!% 6y = #1
    \ifx#2=%
      \BIC@AfterFiFi{\BIC@DivCleanup{#46}}%
    \else
      \BIC@AfterFiFi{%
        \BIC@ProcessDiv#2#3!#46!#5!#6!#7!#8!#9!%
      }%
    \fi
  \or % 6y > #1
    \BIC@AfterFi{%
      \expandafter\BIC@ProcessDivII\romannumeral0%
      \BIC@SubXY#1!#7!!!%
      !#2#3!#4!#5!45%
      #6!#7!#8!#9!%
    }%
  \else % 6y < #1
    \ifcase\BIC@PosCmp#9!#1!% 8y = #1
      \ifx#2=%
        \BIC@AfterFiFiFi{\BIC@DivCleanup{#48}}%
      \else
        \BIC@AfterFiFiFi{%
          \BIC@ProcessDiv#2#3!#48!#5!#6!#7!#8!#9!%
        }%
      \fi
    \or % 8y > #1
      \BIC@AfterFiFi{%
        \expandafter\BIC@ProcessDivII\romannumeral0%
        \BIC@SubXY#1!#8!!!%
        !#2#3!#4!#5!67%
        #6!#7!#8!#9!%
      }%
    \else % 8y < #1
      \BIC@AfterFiFi{%
        \expandafter\BIC@ProcessDivII\romannumeral0%
        \BIC@SubXY#1!#9!!!%
        !#2#3!#4!#5!89%
        #6!#7!#8!#9!%
      }%
    \fi
  \BIC@Fi
}
%    \end{macrocode}
%    \end{macro}
%
% \subsection{\op{Mod}}
%
%    \begin{macro}{\bigintcalcMod}
%    |#1|: $x$\\
%    |#2|: $y$
%    \begin{macrocode}
\def\bigintcalcMod#1{%
  \romannumeral0%
  \expandafter\expandafter\expandafter\BIC@Mod
  \bigintcalcNum{#1}!%
}
%    \end{macrocode}
%    \end{macro}
%    \begin{macro}{\BIC@Mod}
%    |#1|: $x$\\
%    |#2|: $y$
%    \begin{macrocode}
\def\BIC@Mod#1!#2{%
  \expandafter\expandafter\expandafter\BIC@ModSwitchSign
  \bigintcalcNum{#2}!#1!%
}
%    \end{macrocode}
%    \end{macro}
%    \begin{macro}{\BigIntCalcMod}
%    \begin{macrocode}
\def\BigIntCalcMod#1!#2!{%
  \romannumeral0%
  \BIC@ModSwitchSign#2!#1!%
}
%    \end{macrocode}
%    \end{macro}
%    \begin{macro}{\BIC@ModSwitchSign}
%    Decision table for \cs{BIC@ModSwitchSign}.
%    \begin{quote}
%    \begin{tabular}{@{}|l|l|l|@{}}
%    \hline
%    $y=0$ & \multicolumn{1}{l}{} & DivisionByZero\\
%    \hline
%    $y>0$ & $x=0$ & $0$\\
%    \cline{2-3}
%          & else  & ModSwitch$(+,x,y)$\\
%    \hline
%    $y<0$ & \multicolumn{1}{l}{} & ModSwitch$(-,-x,-y)$\\
%    \hline
%    \end{tabular}
%    \end{quote}
%    |#1#2|: $y$\\
%    |#3#4|: $x$
%    \begin{macrocode}
\def\BIC@ModSwitchSign#1#2!#3#4!{%
  \ifcase\ifx\\#2\\%
           \ifx#100 % y = 0
           \else1 % y > 0
           \fi
          \else
            \ifx#1-2 % y < 0
            \else1 % y > 0
            \fi
          \fi
    \BIC@AfterFi{ 0\BigIntCalcError:DivisionByZero}%
  \or % y > 0
    \ifcase\ifx\\#4\\\ifx#300 \else1 \fi\else1 \fi % x = 0
      \BIC@AfterFiFi{ 0}%
    \else
      \BIC@AfterFiFi{%
        \BIC@ModSwitch{}#3#4!#1#2!%
      }%
    \fi
  \else % y < 0
    \ifcase\ifx\\#4\\%
             \ifx#300 % x = 0
             \else1 % x > 0
             \fi
           \else
             \ifx#3-2 % x < 0
             \else1 % x > 0
             \fi
           \fi
      \BIC@AfterFiFi{ 0}%
    \or % x > 0
      \BIC@AfterFiFi{%
        \BIC@ModSwitch--#3#4!#2!%
      }%
    \else % x < 0
      \BIC@AfterFiFi{%
        \BIC@ModSwitch-#4!#2!%
      }%
    \fi
  \BIC@Fi
}
%    \end{macrocode}
%    \end{macro}
%    \begin{macro}{\BIC@ModSwitch}
%    Decision table for \cs{BIC@ModSwitch}.
%    \begin{quote}
%    \begin{tabular}{@{}|l|l|l|@{}}
%    \hline
%    $y=1$ & \multicolumn{1}{l}{} & $0$\\
%    \hline
%    $y=2$ & ifodd$(x)$ & sign $1$\\
%    \cline{2-3}
%          & else       & $0$\\
%    \hline
%    $y>2$ & $x<0$      &
%        $z\leftarrow x-(x/y)*y;\quad (z<0)\mathbin{?}z+y\mathbin{:}z$\\
%    \cline{2-3}
%          & $x>0$      & $x - (x/y) * y$\\
%    \hline
%    \end{tabular}
%    \end{quote}
%    |#1|: sign\\
%    |#2#3|: $x$\\
%    |#4#5|: $y$
%    \begin{macrocode}
\def\BIC@ModSwitch#1#2#3!#4#5!{%
  \ifcase\ifx\\#5\\%
           \ifx#410 % y = 1
           \else\ifx#421 % y = 2
           \else2 % y > 2
           \fi\fi
         \else2 % y > 2
         \fi
    \BIC@AfterFi{ 0}% y = 1
  \or % y = 2
    \ifcase\BIC@ModTwo#2#3! % even(x)
      \BIC@AfterFiFi{ 0}%
    \or % odd(x)
      \BIC@AfterFiFi{ #11}%
?   \else\BigIntCalcError:ThisCannotHappen%
    \fi
  \or % y > 2
    \ifx\\#1\\%
    \else
      \expandafter\BIC@Space\romannumeral0%
      \expandafter\BIC@ModMinus\romannumeral0%
    \fi
    \ifx#2-% x < 0
      \BIC@AfterFiFi{%
        \expandafter\expandafter\expandafter\BIC@ModX
        \bigintcalcSub{#2#3}{%
          \bigintcalcMul{#4#5}{\bigintcalcDiv{#2#3}{#4#5}}%
        }!#4#5!%
      }%
    \else % x > 0
      \BIC@AfterFiFi{%
        \expandafter\expandafter\expandafter\BIC@Space
        \bigintcalcSub{#2#3}{%
          \bigintcalcMul{#4#5}{\bigintcalcDiv{#2#3}{#4#5}}%
        }%
      }%
    \fi
? \else\BigIntCalcError:ThisCannotHappen%
  \BIC@Fi
}
%    \end{macrocode}
%    \end{macro}
%    \begin{macro}{\BIC@ModMinus}
%    \begin{macrocode}
\def\BIC@ModMinus#1{%
  \ifx#10%
    \BIC@AfterFi{ 0}%
  \else
    \BIC@AfterFi{ -#1}%
  \BIC@Fi
}
%    \end{macrocode}
%    \end{macro}
%    \begin{macro}{\BIC@ModX}
%    |#1#2|: $z$\\
%    |#3|: $x$
%    \begin{macrocode}
\def\BIC@ModX#1#2!#3!{%
  \ifx#1-% z < 0
    \BIC@AfterFi{%
      \expandafter\BIC@Space\romannumeral0%
      \BIC@SubXY#3!#2!!!%
    }%
  \else % z >= 0
    \BIC@AfterFi{ #1#2}%
  \BIC@Fi
}
%    \end{macrocode}
%    \end{macro}
%
%    \begin{macrocode}
\BIC@AtEnd%
%    \end{macrocode}
%
%    \begin{macrocode}
%</package>
%    \end{macrocode}
%
% \section{Test}
%
% \subsection{Catcode checks for loading}
%
%    \begin{macrocode}
%<*test1>
%    \end{macrocode}
%    \begin{macrocode}
\catcode`\{=1 %
\catcode`\}=2 %
\catcode`\#=6 %
\catcode`\@=11 %
\expandafter\ifx\csname count@\endcsname\relax
  \countdef\count@=255 %
\fi
\expandafter\ifx\csname @gobble\endcsname\relax
  \long\def\@gobble#1{}%
\fi
\expandafter\ifx\csname @firstofone\endcsname\relax
  \long\def\@firstofone#1{#1}%
\fi
\expandafter\ifx\csname loop\endcsname\relax
  \expandafter\@firstofone
\else
  \expandafter\@gobble
\fi
{%
  \def\loop#1\repeat{%
    \def\body{#1}%
    \iterate
  }%
  \def\iterate{%
    \body
      \let\next\iterate
    \else
      \let\next\relax
    \fi
    \next
  }%
  \let\repeat=\fi
}%
\def\RestoreCatcodes{}
\count@=0 %
\loop
  \edef\RestoreCatcodes{%
    \RestoreCatcodes
    \catcode\the\count@=\the\catcode\count@\relax
  }%
\ifnum\count@<255 %
  \advance\count@ 1 %
\repeat

\def\RangeCatcodeInvalid#1#2{%
  \count@=#1\relax
  \loop
    \catcode\count@=15 %
  \ifnum\count@<#2\relax
    \advance\count@ 1 %
  \repeat
}
\def\RangeCatcodeCheck#1#2#3{%
  \count@=#1\relax
  \loop
    \ifnum#3=\catcode\count@
    \else
      \errmessage{%
        Character \the\count@\space
        with wrong catcode \the\catcode\count@\space
        instead of \number#3%
      }%
    \fi
  \ifnum\count@<#2\relax
    \advance\count@ 1 %
  \repeat
}
\def\space{ }
\expandafter\ifx\csname LoadCommand\endcsname\relax
  \def\LoadCommand{\input bigintcalc.sty\relax}%
\fi
\def\Test{%
  \RangeCatcodeInvalid{0}{47}%
  \RangeCatcodeInvalid{58}{64}%
  \RangeCatcodeInvalid{91}{96}%
  \RangeCatcodeInvalid{123}{255}%
  \catcode`\@=12 %
  \catcode`\\=0 %
  \catcode`\%=14 %
  \LoadCommand
  \RangeCatcodeCheck{0}{36}{15}%
  \RangeCatcodeCheck{37}{37}{14}%
  \RangeCatcodeCheck{38}{47}{15}%
  \RangeCatcodeCheck{48}{57}{12}%
  \RangeCatcodeCheck{58}{63}{15}%
  \RangeCatcodeCheck{64}{64}{12}%
  \RangeCatcodeCheck{65}{90}{11}%
  \RangeCatcodeCheck{91}{91}{15}%
  \RangeCatcodeCheck{92}{92}{0}%
  \RangeCatcodeCheck{93}{96}{15}%
  \RangeCatcodeCheck{97}{122}{11}%
  \RangeCatcodeCheck{123}{255}{15}%
  \RestoreCatcodes
}
\Test
\csname @@end\endcsname
\end
%    \end{macrocode}
%    \begin{macrocode}
%</test1>
%    \end{macrocode}
%
% \subsection{Macro tests}
%
% \subsubsection{Preamble with test macro definitions}
%
%    \begin{macrocode}
%<*test2>
\NeedsTeXFormat{LaTeX2e}
\nofiles
\documentclass{article}
%<noetex>\let\SavedNumexpr\numexpr
%<noetex>\let\numexpr\UNDEFINED
\makeatletter
\chardef\BIC@TestMode=1 %
\makeatother
\usepackage{bigintcalc}[2016/05/16]
%<noetex>\let\numexpr\SavedNumexpr
\usepackage{qstest}
\IncludeTests{*}
\LogTests{log}{*}{*}
\newcommand*{\TestSpaceAtEnd}[1]{%
%<noetex>  \let\SavedNumexpr\numexpr
%<noetex>  \let\numexpr\UNDEFINED
  \edef\resultA{#1}%
  \edef\resultB{#1 }%
%<noetex>  \let\numexpr\SavedNumexpr
  \Expect*{\resultA\space}*{\resultB}%
}
\newcommand*{\TestResult}[2]{%
%<noetex>  \let\SavedNumexpr\numexpr
%<noetex>  \let\numexpr\UNDEFINED
  \edef\result{#1}%
%<noetex>  \let\numexpr\SavedNumexpr
  \Expect*{\result}{#2}%
}
\newcommand*{\TestResultTwoExpansions}[2]{%
%<*noetex>
  \begingroup
    \let\numexpr\UNDEFINED
    \expandafter\expandafter\expandafter
  \endgroup
%</noetex>
  \expandafter\expandafter\expandafter\Expect
  \expandafter\expandafter\expandafter{#1}{#2}%
}
\newcount\TestCount
%<etex>\newcommand*{\TestArg}[1]{\numexpr#1\relax}
%<noetex>\newcommand*{\TestArg}[1]{#1}
\newcommand*{\TestTeXDivide}[2]{%
  \TestCount=\TestArg{#1}\relax
  \divide\TestCount by \TestArg{#2}\relax
  \Expect*{\bigintcalcDiv{#1}{#2}}*{\the\TestCount}%
}
\newcommand*{\Test}[2]{%
  \TestResult{#1}{#2}%
  \TestResultTwoExpansions{#1}{#2}%
  \TestSpaceAtEnd{#1}%
}
\newcommand*{\TestExch}[2]{\Test{#2}{#1}}
\newcommand*{\TestInv}[2]{%
  \Test{\bigintcalcInv{#1}}{#2}%
}
\newcommand*{\TestAbs}[2]{%
  \Test{\bigintcalcAbs{#1}}{#2}%
}
\newcommand*{\TestSgn}[2]{%
  \Test{\bigintcalcSgn{#1}}{#2}%
}
\newcommand*{\TestMin}[3]{%
  \Test{\bigintcalcMin{#1}{#2}}{#3}%
}
\newcommand*{\TestMax}[3]{%
  \Test{\bigintcalcMax{#1}{#2}}{#3}%
}
\newcommand*{\TestCmp}[3]{%
  \Test{\bigintcalcCmp{#1}{#2}}{#3}%
}
\newcommand*{\TestOdd}[2]{%
  \Test{\bigintcalcOdd{#1}}{#2}%
  \edef\x{%
    \noexpand\Test{%
      \noexpand\BigIntCalcOdd
      \bigintcalcAbs{#1}!%
    }{#2}%
  }%
  \x
}
\newcommand*{\TestInc}[2]{%
  \Test{\bigintcalcInc{#1}}{#2}%
  \ifnum\bigintcalcSgn{#1}>-1 %
    \edef\x{%
      \noexpand\Test{%
        \noexpand\BigIntCalcInc\bigintcalcNum{#1}!%
      }{#2}%
    }%
    \x
  \fi
}
\newcommand*{\TestDec}[2]{%
  \Test{\bigintcalcDec{#1}}{#2}%
  \ifnum\bigintcalcSgn{#1}>0 %
    \edef\x{%
      \noexpand\Test{%
        \noexpand\BigIntCalcDec\bigintcalcNum{#1}!%
      }{#2}%
    }%
    \x
  \fi
}
\newcommand*{\TestAdd}[3]{%
  \Test{\bigintcalcAdd{#1}{#2}}{#3}%
  \ifnum\bigintcalcSgn{#1}>0 %
    \ifnum\bigintcalcSgn{#2}> 0 %
      \ifnum\bigintcalcCmp{#1}{#2}>0 %
        \edef\x{%
          \noexpand\Test{%
            \noexpand\BigIntCalcAdd
            \bigintcalcNum{#1}!\bigintcalcNum{#2}!%
          }{#3}%
        }%
        \x
      \else
        \edef\x{%
          \noexpand\Test{%
            \noexpand\BigIntCalcAdd
            \bigintcalcNum{#2}!\bigintcalcNum{#1}!%
          }{#3}%
        }%
        \x
      \fi
    \fi
  \fi
}
\newcommand*{\TestSub}[3]{%
  \Test{\bigintcalcSub{#1}{#2}}{#3}%
  \ifnum\bigintcalcSgn{#1}>0 %
    \ifnum\bigintcalcSgn{#2}> 0 %
      \ifnum\bigintcalcCmp{#1}{#2}>0 %
        \edef\x{%
          \noexpand\Test{%
            \noexpand\BigIntCalcSub
            \bigintcalcNum{#1}!\bigintcalcNum{#2}!%
          }{#3}%
        }%
        \x
      \fi
    \fi
  \fi
}
\newcommand*{\TestShl}[2]{%
  \Test{\bigintcalcShl{#1}}{#2}%
  \edef\x{%
    \noexpand\Test{%
      \noexpand\BigIntCalcShl\bigintcalcAbs{#1}!%
    }{\bigintcalcAbs{#2}}%
  }%
  \x
}
\newcommand*{\TestShr}[2]{%
  \Test{\bigintcalcShr{#1}}{#2}%
  \edef\x{%
    \noexpand\Test{%
      \noexpand\BigIntCalcShr\bigintcalcAbs{#1}!%
    }{\bigintcalcAbs{#2}}%
  }%
  \x
}
\newcommand*{\TestMul}[3]{%
  \Test{\bigintcalcMul{#1}{#2}}{#3}%
  \edef\x{%
    \noexpand\Test{%
      \noexpand\BigIntCalcMul
      \bigintcalcAbs{#1}!\bigintcalcAbs{#2}!%
    }{\bigintcalcAbs{#3}}%
  }%
  \x
}
\newcommand*{\TestSqr}[2]{%
  \Test{\bigintcalcSqr{#1}}{#2}%
}
\newcommand*{\TestFac}[2]{%
  \expandafter\TestExch\expandafter{%
    \the\numexpr#2%
  }{\bigintcalcFac{#1}}%
}
\newcommand*{\TestFacBig}[2]{%
  \Test{\bigintcalcFac{#1}}{#2}%
}
\newcommand*{\TestPow}[3]{%
  \Test{\bigintcalcPow{#1}{#2}}{#3}%
}
\newcommand*{\TestDiv}[3]{%
  \Test{\bigintcalcDiv{#1}{#2}}{#3}%
  \TestTeXDivide{#1}{#2}%
}
\newcommand*{\TestDivBig}[3]{%
  \Test{\bigintcalcDiv{#1}{#2}}{#3}%
  \edef\x{%
    \noexpand\Test{%
      \noexpand\BigIntCalcDiv\bigintcalcAbs{#1}!\bigintcalcAbs{#2}!%
    }{\bigintcalcAbs{#3}}%
  }%
}
\newcommand*{\TestMod}[3]{%
  \Test{\bigintcalcMod{#1}{#2}}{#3}%
  \ifcase\ifcase\bigintcalcSgn{#1} 0%
         \or
           \ifcase\bigintcalcSgn{#2} 1%
           \or 0%
           \else 1%
           \fi
         \else
           \ifcase\bigintcalcSgn{#2} 1%
           \or 1%
           \else 0%
           \fi
         \fi\relax
    \edef\x{%
      \noexpand\Test{%
        \noexpand\BigIntCalcMod
        \bigintcalcAbs{#1}!\bigintcalcAbs{#2}!%
      }{\bigintcalcAbs{#3}}%
    }%
    \x
  \fi
}
%    \end{macrocode}
%
% \subsubsection{Time}
%
%    \begin{macrocode}
\begingroup\expandafter\expandafter\expandafter\endgroup
\expandafter\ifx\csname pdfresettimer\endcsname\relax
\else
  \makeatletter
  \newcount\SummaryTime
  \newcount\TestTime
  \SummaryTime=\z@
  \newcommand*{\PrintTime}[2]{%
    \typeout{%
      [Time #1: \strip@pt\dimexpr\number#2sp\relax\space s]%
    }%
  }%
  \newcommand*{\StartTime}[1]{%
    \renewcommand*{\TimeDescription}{#1}%
    \pdfresettimer
  }%
  \newcommand*{\TimeDescription}{}%
  \newcommand*{\StopTime}{%
    \TestTime=\pdfelapsedtime
    \global\advance\SummaryTime\TestTime
    \PrintTime\TimeDescription\TestTime
  }%
  \let\saved@qstest\qstest
  \let\saved@endqstest\endqstest
  \def\qstest#1#2{%
    \saved@qstest{#1}{#2}%
    \StartTime{#1}%
  }%
  \def\endqstest{%
    \StopTime
    \saved@endqstest
  }%
  \AtEndDocument{%
    \PrintTime{summary}\SummaryTime
  }%
  \makeatother
\fi
%    \end{macrocode}
%
% \subsubsection{Test sets}
%
%    \begin{macrocode}
\makeatletter

\begin{qstest}{inv}{inv}%
  \TestInv{0}{0}%
  \TestInv{1}{-1}%
  \TestInv{-1}{1}%
  \TestInv{10}{-10}%
  \TestInv{-10}{10}%
  \TestInv{2147483647}{-2147483647}%
  \TestInv{-2147483647}{2147483647}%
  \TestInv{12345678901234567890}{-12345678901234567890}%
  \TestInv{-12345678901234567890}{12345678901234567890}%
  \TestInv{ 0 }{0}%
  \TestInv{ 1 }{-1}%
  \TestInv{--1}{-1}%
  \TestInv{\number\z@}{0}%
  \TestInv{\ifx\relax\relax1\fi}{-1}%
  \TestInv{\ifx\relax\relax-\fi\ifx234\else1\fi}{1}%
\end{qstest}

\begin{qstest}{abs}{abs}%
  \TestAbs{0}{0}%
  \TestAbs{1}{1}%
  \TestAbs{-1}{1}%
  \TestAbs{10}{10}%
  \TestAbs{-10}{10}%
  \TestAbs{2147483647}{2147483647}%
  \TestAbs{-2147483647}{2147483647}%
  \TestAbs{12345678901234567890}{12345678901234567890}%
  \TestAbs{-12345678901234567890}{12345678901234567890}%
  \TestAbs{ 0 }{0}%
  \TestAbs{ 1 }{1}%
  \TestAbs{--1}{1}%
  \TestAbs{-+-+1}{1}%
  \TestAbs{00000000000}{0}%
  \TestAbs{00000001000}{1000}%
  \TestAbs{\ifx\relax\relax 0\else 1\fi}{0}%
\end{qstest}

\begin{qstest}{sign}{sign}%
  \TestSgn{0}{0}%
  \TestSgn{1}{1}%
  \TestSgn{-1}{-1}%
  \TestSgn{10}{1}%
  \TestSgn{-10}{-1}%
  \TestSgn{2147483647}{1}%
  \TestSgn{-2147483647}{-1}%
  \TestSgn{12345678901234567890}{1}%
  \TestSgn{-12345678901234567890}{-1}%
  \TestSgn{ 0 }{0}%
  \TestSgn{ 2 }{1}%
  \TestSgn{ -2 }{-1}%
  \TestSgn{--2}{1}%
  \TestSgn{\number\z@}{0}%
  \TestSgn{\number\@ne}{1}%
  \TestSgn{\number\m@ne}{-1}%
  \TestSgn{%
    -+-+\number\z@\number\z@
    \iftrue1\fi\iftrue2\fi\iftrue3\fi
  }{1}%
\end{qstest}

\begin{qstest}{min}{min}%
  \TestMin{0}{1}{0}%
  \TestMin{1}{0}{0}%
  \TestMin{-10}{-20}{-20}%
  \TestMin{ 1 }{ 2 }{1}%
  \TestMin{ 2 }{ 1 }{1}%
  \TestMin{1}{1}{1}%
  \TestMin{\number\z@}{\number\@ne}{0}%
  \TestMin{\number\@ne}{\number\m@ne}{-1}%
\end{qstest}

\begin{qstest}{max}{max}%
  \TestMax{0}{1}{1}%
  \TestMax{1}{0}{1}%
  \TestMax{-10}{-20}{-10}%
  \TestMax{ 1 }{ 2 }{2}%
  \TestMax{ 2 }{ 1 }{2}%
  \TestMax{1}{1}{1}%
  \TestMax{\number\z@}{\number\@ne}{1}%
  \TestMax{\number\@ne}{\number\m@ne}{1}%
\end{qstest}

\begin{qstest}{cmp}{cmp}%
  \TestCmp{0}{0}{0}%
  \TestCmp{-21}{17}{-1}%
  \TestCmp{3}{4}{-1}%
  \TestCmp{-10}{-10}{0}%
  \TestCmp{-10}{-11}{1}%
  \TestCmp{100}{5}{1}%
  \TestCmp{9}{10}{-1}%
  \TestCmp{10}{9}{1}%
  \TestCmp{ 3 }{ 3 }{0}%
  \TestCmp{-9}{-10}{1}%
  \TestCmp{-10}{-9}{-1}%
  \TestCmp{-3}{-3}{0}%
  \TestCmp{0}{-2}{1}%
  \TestCmp{0}{2}{-1}%
  \TestCmp{2}{0}{1}%
  \TestCmp{-2}{0}{-1}%
  \TestCmp{12}{11}{1}%
  \TestCmp{11}{12}{-1}%
  \TestCmp{2147483647}{-2147483647}{1}%
  \TestCmp{-2147483647}{2147483647}{-1}%
  \TestCmp{2147483647}{2147483647}{0}%
  \TestCmp{\number\z@}{\number\@ne}{-1}%
  \TestCmp{\number\@ne}{\number\m@ne}{1}%
  \TestCmp{ 4 }{ 5 }{-1}%
  \TestCmp{ -3 }{ -7 }{1}%
\end{qstest}

\begin{qstest}{odd}{odd}
\tracingmacros=1
  \TestOdd{0}{0}%
  \TestOdd{1}{1}%
  \TestOdd{2}{0}%
  \TestOdd{3}{1}%
  \TestOdd{14}{0}%
  \TestOdd{15}{1}%
  \TestOdd{12345678901234567896}{0}%
  \TestOdd{12345678901234567897}{1}%
\end{qstest}

\begin{qstest}{inc}{inc}%
  \TestInc{0}{1}%
  \TestInc{1}{2}%
  \TestInc{-1}{0}%
  \TestInc{10}{11}%
  \TestInc{-10}{-9}%
  \TestInc{ 3 }{4}%
  \TestInc{999}{1000}%
  \TestInc{-1000}{-999}%
  \TestInc{129}{130}%
  \TestInc{2147483646}{2147483647}%
  \TestInc{-2147483647}{-2147483646}%
  \TestInc{12345678901234567890}{12345678901234567891}%
  \TestInc{99999999999999999999}{100000000000000000000}%
  \TestInc{-12345678901234567891}{-12345678901234567890}%
  \TestInc{-100000000000000000000}{-99999999999999999999}%
\end{qstest}

\begin{qstest}{dec}{dec}%
  \TestDec{0}{-1}%
  \TestDec{1}{0}%
  \TestDec{-1}{-2}%
  \TestDec{10}{9}%
  \TestDec{-10}{-11}%
  \TestDec{1000}{999}%
  \TestDec{-999}{-1000}%
  \TestDec{130}{129}%
  \TestDec{2147483647}{2147483646}%
  \TestDec{-2147483646}{-2147483647}%
  \TestDec{12345678901234567891}{12345678901234567890}%
  \TestDec{100000000000000000000}{99999999999999999999}%
  \TestDec{-12345678901234567890}{-12345678901234567891}%
  \TestDec{-99999999999999999999}{-100000000000000000000}%
\end{qstest}

\begin{qstest}{add}{add}%
  \TestAdd{0}{0}{0}%
  \TestAdd{1}{0}{1}%
  \TestAdd{0}{1}{1}%
  \TestAdd{1}{2}{3}%
  \TestAdd{-1}{-1}{-2}%
  \TestAdd{2147483646}{1}{2147483647}%
  \TestAdd{-2147483647}{2147483647}{0}%
  \TestAdd{20}{-5}{15}%
  \TestAdd{-4}{-1}{-5}%
  \TestAdd{-1}{-4}{-5}%
  \TestAdd{-4}{1}{-3}%
  \TestAdd{-1}{4}{3}%
  \TestAdd{4}{-1}{3}%
  \TestAdd{1}{-4}{-3}%
  \TestAdd{-4}{-1}{-5}%
  \TestAdd{-1}{-4}{-5}%
  \TestAdd{ -4 }{ -1 }{-5}%
  \TestAdd{ -1 }{ -4 }{-5}%
  \TestAdd{ -4 }{ 1 }{-3}%
  \TestAdd{ -1 }{ 4 }{3}%
  \TestAdd{ 4 }{ -1 }{3}%
  \TestAdd{ 1 }{ -4 }{-3}%
  \TestAdd{ -4 }{ -1 }{-5}%
  \TestAdd{ -1 }{ -4 }{-5}%
  \TestAdd{876543210}{111111111}{987654321}%
  \TestAdd{999999999}{2}{1000000001}%
\end{qstest}

\begin{qstest}{sub}{sub}
  \TestSub{0}{0}{0}%
  \TestSub{1}{0}{1}%
  \TestSub{1}{2}{-1}%
  \TestSub{-1}{-1}{0}%
  \TestSub{2147483646}{-1}{2147483647}%
  \TestSub{-2147483647}{-2147483647}{0}%
  \TestSub{-4}{-1}{-3}%
  \TestSub{-1}{-4}{3}%
  \TestSub{-4}{1}{-5}%
  \TestSub{-1}{4}{-5}%
  \TestSub{4}{-1}{5}%
  \TestSub{1}{-4}{5}%
  \TestSub{-4}{-1}{-3}%
  \TestSub{-1}{-4}{3}%
  \TestSub{ -4 }{ -1 }{-3}%
  \TestSub{ -1 }{ -4 }{3}%
  \TestSub{ -4 }{ 1 }{-5}%
  \TestSub{ -1 }{ 4 }{-5}%
  \TestSub{ 4 }{ -1 }{5}%
  \TestSub{ 1 }{ -4 }{5}%
  \TestSub{ -4 }{ -1 }{-3}%
  \TestSub{ -1 }{ -4 }{3}%
  \TestSub{1000000000}{2}{999999998}%
  \TestSub{987654321}{111111111}{876543210}%
\end{qstest}

\begin{qstest}{shl}{shl}
  \TestShl{0}{0}%
  \TestShl{1}{2}%
  \TestShl{2}{4}%
  \TestShl{5621}{11242}%
  \TestShl{1073741823}{2147483646}%
\end{qstest}

\begin{qstest}{shr}{shr}
  \TestShr{0}{0}%
  \TestShr{1}{0}%
  \TestShr{2}{1}%
  \TestShr{3}{1}%
  \TestShr{4}{2}%
  \TestShr{5}{2}%
  \TestShr{6}{3}%
  \TestShr{7}{3}%
  \TestShr{8}{4}%
  \TestShr{9}{4}%
  \TestShr{10}{5}%
  \TestShr{11}{5}%
  \TestShr{12}{6}%
  \TestShr{13}{6}%
  \TestShr{14}{7}%
  \TestShr{15}{7}%
  \TestShr{16}{8}%
  \TestShr{17}{8}%
  \TestShr{18}{9}%
  \TestShr{19}{9}%
  \TestShr{20}{10}%
  \TestShr{21}{10}%
  \TestShr{22}{11}%
  \TestShr{11241}{5620}%
  \TestShr{73054202}{36527101}%
  \TestShr{2147483646}{1073741823}%
\end{qstest}

\begin{qstest}{mul}{mul}
  \TestMul{0}{0}{0}%
  \TestMul{1}{0}{0}%
  \TestMul{0}{1}{0}%
  \TestMul{1}{1}{1}%
  \TestMul{3}{1}{3}%
  \TestMul{1}{-3}{-3}%
  \TestMul{-4}{-5}{20}%
  \TestMul{3}{7}{21}%
  \TestMul{7}{3}{21}%
  \TestMul{3}{-7}{-21}%
  \TestMul{7}{-3}{-21}%
  \TestMul{-3}{7}{-21}%
  \TestMul{-7}{3}{-21}%
  \TestMul{-3}{-7}{21}%
  \TestMul{-7}{-3}{21}%
  \TestMul{12}{11}{132}%
  \TestMul{999}{333}{332667}%
  \TestMul{1000}{4321}{4321000}%
  \TestMul{12345}{173955}{2147474475}%
  \TestMul{1073741823}{2}{2147483646}%
  \TestMul{2}{1073741823}{2147483646}%
  \TestMul{-1073741823}{2}{-2147483646}%
  \TestMul{2}{-1073741823}{-2147483646}%
  \TestMul{6706022400}{13}{87178291200}%
\end{qstest}

\begin{qstest}{sqr}{sqr}
  \TestSqr{0}{0}%
  \TestSqr{1}{1}%
  \TestSqr{2}{4}%
  \TestSqr{3}{9}%
  \TestSqr{4}{16}%
  \TestSqr{9}{81}%
  \TestSqr{10}{100}%
  \TestSqr{46340}{2147395600}%
  \TestSqr{-1}{1}%
  \TestSqr{-2}{4}%
  \TestSqr{-46340}{2147395600}%
\end{qstest}

\begin{qstest}{fac}{fac}
  \TestFac{0}{1}%
  \TestFac{1}{1}%
  \TestFac{2}{2}%
  \TestFac{3}{2*3}%
  \TestFac{4}{2*3*4}%
  \TestFac{5}{2*3*4*5}%
  \TestFac{6}{2*3*4*5*6}%
  \TestFac{7}{2*3*4*5*6*7}%
  \TestFac{8}{2*3*4*5*6*7*8}%
  \TestFac{9}{2*3*4*5*6*7*8*9}%
  \TestFac{10}{2*3*4*5*6*7*8*9*10}%
  \TestFac{11}{2*3*4*5*6*7*8*9*10*11}%
  \TestFac{12}{2*3*4*5*6*7*8*9*10*11*12}%
  \TestFacBig{13}{6227020800}%
  \TestFacBig{14}{87178291200}%
  \TestFacBig{15}{1307674368000}%
  \TestFacBig{16}{20922789888000}%
  \TestFacBig{17}{355687428096000}%
  \TestFacBig{18}{6402373705728000}%
  \TestFacBig{19}{121645100408832000}%
  \TestFacBig{20}{2432902008176640000}%
  \TestFacBig{21}{51090942171709440000}%
  \TestFacBig{22}{1124000727777607680000}%
\end{qstest}

\begin{qstest}{pow}{pow}
  \TestPow{-2}{0}{1}%
  \TestPow{-1}{0}{1}%
  \TestPow{0}{0}{1}%
  \TestPow{1}{0}{1}%
  \TestPow{2}{0}{1}%
  \TestPow{3}{0}{1}%
  \TestPow{-2}{1}{-2}%
  \TestPow{-1}{1}{-1}%
  \TestPow{1}{1}{1}%
  \TestPow{2}{1}{2}%
  \TestPow{3}{1}{3}%
  \TestPow{-2}{2}{4}%
  \TestPow{-1}{2}{1}%
  \TestPow{0}{2}{0}%
  \TestPow{1}{2}{1}%
  \TestPow{2}{2}{4}%
  \TestPow{3}{2}{9}%
  \TestPow{0}{1}{0}%
  \TestPow{1}{-2}{1}%
  \TestPow{1}{-1}{1}%
  \TestPow{-1}{-2}{1}%
  \TestPow{-1}{-1}{-1}%
  \TestPow{-1}{3}{-1}%
  \TestPow{-1}{4}{1}%
  \TestPow{-2}{-1}{0}%
  \TestPow{-2}{-2}{0}%
  \TestPow{2}{3}{8}%
  \TestPow{2}{4}{16}%
  \TestPow{2}{5}{32}%
  \TestPow{2}{6}{64}%
  \TestPow{2}{7}{128}%
  \TestPow{2}{8}{256}%
  \TestPow{2}{9}{512}%
  \TestPow{2}{10}{1024}%
  \TestPow{-2}{3}{-8}%
  \TestPow{-2}{4}{16}%
  \TestPow{-2}{5}{-32}%
  \TestPow{-2}{6}{64}%
  \TestPow{-2}{7}{-128}%
  \TestPow{-2}{8}{256}%
  \TestPow{-2}{9}{-512}%
  \TestPow{-2}{10}{1024}%
  \TestPow{3}{3}{27}%
  \TestPow{3}{4}{81}%
  \TestPow{3}{5}{243}%
  \TestPow{-3}{3}{-27}%
  \TestPow{-3}{4}{81}%
  \TestPow{-3}{5}{-243}%
  \TestPow{2}{30}{1073741824}%
  \TestPow{-3}{19}{-1162261467}%
  \TestPow{5}{13}{1220703125}%
  \TestPow{-7}{11}{-1977326743}%
\end{qstest}

\begin{qstest}{div}{div}
  \TestDiv{1}{1}{1}%
  \TestDiv{2}{1}{2}%
  \TestDiv{-2}{1}{-2}%
  \TestDiv{2}{-1}{-2}%
  \TestDiv{-2}{-1}{2}%
  \TestDiv{15}{2}{7}%
  \TestDiv{-16}{2}{-8}%
  \TestDiv{1}{2}{0}%
  \TestDiv{1}{3}{0}%
  \TestDiv{2}{3}{0}%
  \TestDiv{-2}{3}{0}%
  \TestDiv{2}{-3}{0}%
  \TestDiv{-2}{-3}{0}%
  \TestDiv{13}{3}{4}%
  \TestDiv{-13}{-3}{4}%
  \TestDiv{-13}{3}{-4}%
  \TestDiv{-6}{5}{-1}%
  \TestDiv{-5}{5}{-1}%
  \TestDiv{-4}{5}{0}%
  \TestDiv{-3}{5}{0}%
  \TestDiv{-2}{5}{0}%
  \TestDiv{-1}{5}{0}%
  \TestDiv{0}{5}{0}%
  \TestDiv{1}{5}{0}%
  \TestDiv{2}{5}{0}%
  \TestDiv{3}{5}{0}%
  \TestDiv{4}{5}{0}%
  \TestDiv{5}{5}{1}%
  \TestDiv{6}{5}{1}%
  \TestDiv{-5}{4}{-1}%
  \TestDiv{-4}{4}{-1}%
  \TestDiv{-3}{4}{0}%
  \TestDiv{-2}{4}{0}%
  \TestDiv{-1}{4}{0}%
  \TestDiv{0}{4}{0}%
  \TestDiv{1}{4}{0}%
  \TestDiv{2}{4}{0}%
  \TestDiv{3}{4}{0}%
  \TestDiv{4}{4}{1}%
  \TestDiv{5}{4}{1}%
  \TestDiv{12345}{678}{18}%
  \TestDiv{32372}{5952}{5}%
  \TestDiv{284271294}{18162}{15651}%
  \TestDiv{217652429}{12561}{17327}%
  \TestDiv{462028434}{5439}{84947}%
  \TestDiv{2147483647}{1000}{2147483}%
  \TestDiv{2147483647}{-1000}{-2147483}%
  \TestDiv{-2147483647}{1000}{-2147483}%
  \TestDiv{-2147483647}{-1000}{2147483}%
  \TestDiv{0}{3}{0}%
  \TestDiv{1}{3}{0}%
  \TestDiv{2}{3}{0}%
  \TestDiv{3}{3}{1}%
  \TestDiv{4}{3}{1}%
  \TestDiv{5}{3}{1}%
  \TestDiv{6}{3}{2}%
  \TestDiv{7}{3}{2}%
  \TestDiv{8}{3}{2}%
  \TestDiv{9}{3}{3}%
  \TestDiv{10}{3}{3}%
  \TestDiv{11}{3}{3}%
  \TestDiv{12}{3}{4}%
  \TestDiv{13}{3}{4}%
  \TestDiv{14}{3}{4}%
  \TestDiv{15}{3}{5}%
  \TestDiv{16}{3}{5}%
  \TestDiv{17}{3}{5}%
  \TestDiv{18}{3}{6}%
  \TestDiv{19}{3}{6}%
  \TestDiv{20}{3}{6}%
  \TestDiv{21}{3}{7}%
  \TestDiv{22}{3}{7}%
  \TestDiv{23}{3}{7}%
  \TestDiv{24}{3}{8}%
  \TestDiv{25}{3}{8}%
  \TestDiv{26}{3}{8}%
  \TestDiv{27}{3}{9}%
  \TestDiv{28}{3}{9}%
  \TestDiv{29}{3}{9}%
  \TestDiv{30}{3}{10}%
  \TestDiv{31}{3}{10}%
  \TestDivBig{17363436332507}{24702}{702916214}%
\end{qstest}

\begin{qstest}{mod}{mod}
  \TestMod{-6}{5}{4}%
  \TestMod{-5}{5}{0}%
  \TestMod{-4}{5}{1}%
  \TestMod{-3}{5}{2}%
  \TestMod{-2}{5}{3}%
  \TestMod{-1}{5}{4}%
  \TestMod{0}{5}{0}%
  \TestMod{1}{5}{1}%
  \TestMod{2}{5}{2}%
  \TestMod{3}{5}{3}%
  \TestMod{4}{5}{4}%
  \TestMod{5}{5}{0}%
  \TestMod{6}{5}{1}%
  \TestMod{-5}{4}{3}%
  \TestMod{-4}{4}{0}%
  \TestMod{-3}{4}{1}%
  \TestMod{-2}{4}{2}%
  \TestMod{-1}{4}{3}%
  \TestMod{0}{4}{0}%
  \TestMod{1}{4}{1}%
  \TestMod{2}{4}{2}%
  \TestMod{3}{4}{3}%
  \TestMod{4}{4}{0}%
  \TestMod{5}{4}{1}%
  \TestMod{-6}{-5}{-1}%
  \TestMod{-5}{-5}{0}%
  \TestMod{-4}{-5}{-4}%
  \TestMod{-3}{-5}{-3}%
  \TestMod{-2}{-5}{-2}%
  \TestMod{-1}{-5}{-1}%
  \TestMod{0}{-5}{0}%
  \TestMod{1}{-5}{-4}%
  \TestMod{2}{-5}{-3}%
  \TestMod{3}{-5}{-2}%
  \TestMod{4}{-5}{-1}%
  \TestMod{5}{-5}{0}%
  \TestMod{6}{-5}{-4}%
  \TestMod{-5}{-4}{-1}%
  \TestMod{-4}{-4}{0}%
  \TestMod{-3}{-4}{-3}%
  \TestMod{-2}{-4}{-2}%
  \TestMod{-1}{-4}{-1}%
  \TestMod{0}{-4}{0}%
  \TestMod{1}{-4}{-3}%
  \TestMod{2}{-4}{-2}%
  \TestMod{3}{-4}{-1}%
  \TestMod{4}{-4}{0}%
  \TestMod{5}{-4}{-3}%
  \TestMod{2147483647}{1000}{647}%
  \TestMod{2147483647}{-1000}{-353}%
  \TestMod{-2147483647}{1000}{353}%
  \TestMod{-2147483647}{-1000}{-647}%
  \TestMod{ 0 }{ 4 }{0}%
  \TestMod{ 1 }{ 4 }{1}%
  \TestMod{ -1 }{ 4 }{3}%
  \TestMod{ 0 }{ -4 }{0}%
  \TestMod{ 1 }{ -4 }{-3}%
  \TestMod{ -1 }{ -4 }{-1}%
  \TestMod{18362}{25}{12}%
\end{qstest}

\newcommand*{\TestError}[2]{%
  \begingroup
    \expandafter\def\csname BigIntCalcError:#1\endcsname{}%
    \Expect*{#2}{0}%
    \expandafter\def\csname BigIntCalcError:#1\endcsname{ERROR}%
    \Expect*{#2}{0ERROR}%
  \endgroup
}
\begin{qstest}{error}{error}
  \TestError{FacNegative}{\bigintcalcFac{-1}}%
  \TestError{FacNegative}{\bigintcalcFac{-2147483647}}%
  \TestError{DivisionByZero}{\bigintcalcPow{0}{-1}}%
  \TestError{DivisionByZero}{\bigintcalcDiv{1}{0}}%
  \TestError{DivisionByZero}{\bigintcalcMod{1}{0}}%
\end{qstest}

\begin{document}
\end{document}
%</test2>
%    \end{macrocode}
%
% \section{Installation}
%
% \subsection{Download}
%
% \paragraph{Package.} This package is available on
% CTAN\footnote{\url{http://ctan.org/pkg/bigintcalc}}:
% \begin{description}
% \item[\CTAN{macros/latex/contrib/oberdiek/bigintcalc.dtx}] The source file.
% \item[\CTAN{macros/latex/contrib/oberdiek/bigintcalc.pdf}] Documentation.
% \end{description}
%
%
% \paragraph{Bundle.} All the packages of the bundle `oberdiek'
% are also available in a TDS compliant ZIP archive. There
% the packages are already unpacked and the documentation files
% are generated. The files and directories obey the TDS standard.
% \begin{description}
% \item[\CTAN{install/macros/latex/contrib/oberdiek.tds.zip}]
% \end{description}
% \emph{TDS} refers to the standard ``A Directory Structure
% for \TeX\ Files'' (\CTAN{tds/tds.pdf}). Directories
% with \xfile{texmf} in their name are usually organized this way.
%
% \subsection{Bundle installation}
%
% \paragraph{Unpacking.} Unpack the \xfile{oberdiek.tds.zip} in the
% TDS tree (also known as \xfile{texmf} tree) of your choice.
% Example (linux):
% \begin{quote}
%   |unzip oberdiek.tds.zip -d ~/texmf|
% \end{quote}
%
% \paragraph{Script installation.}
% Check the directory \xfile{TDS:scripts/oberdiek/} for
% scripts that need further installation steps.
% Package \xpackage{attachfile2} comes with the Perl script
% \xfile{pdfatfi.pl} that should be installed in such a way
% that it can be called as \texttt{pdfatfi}.
% Example (linux):
% \begin{quote}
%   |chmod +x scripts/oberdiek/pdfatfi.pl|\\
%   |cp scripts/oberdiek/pdfatfi.pl /usr/local/bin/|
% \end{quote}
%
% \subsection{Package installation}
%
% \paragraph{Unpacking.} The \xfile{.dtx} file is a self-extracting
% \docstrip\ archive. The files are extracted by running the
% \xfile{.dtx} through \plainTeX:
% \begin{quote}
%   \verb|tex bigintcalc.dtx|
% \end{quote}
%
% \paragraph{TDS.} Now the different files must be moved into
% the different directories in your installation TDS tree
% (also known as \xfile{texmf} tree):
% \begin{quote}
% \def\t{^^A
% \begin{tabular}{@{}>{\ttfamily}l@{ $\rightarrow$ }>{\ttfamily}l@{}}
%   bigintcalc.sty & tex/generic/oberdiek/bigintcalc.sty\\
%   bigintcalc.pdf & doc/latex/oberdiek/bigintcalc.pdf\\
%   test/bigintcalc-test1.tex & doc/latex/oberdiek/test/bigintcalc-test1.tex\\
%   test/bigintcalc-test2.tex & doc/latex/oberdiek/test/bigintcalc-test2.tex\\
%   test/bigintcalc-test3.tex & doc/latex/oberdiek/test/bigintcalc-test3.tex\\
%   bigintcalc.dtx & source/latex/oberdiek/bigintcalc.dtx\\
% \end{tabular}^^A
% }^^A
% \sbox0{\t}^^A
% \ifdim\wd0>\linewidth
%   \begingroup
%     \advance\linewidth by\leftmargin
%     \advance\linewidth by\rightmargin
%   \edef\x{\endgroup
%     \def\noexpand\lw{\the\linewidth}^^A
%   }\x
%   \def\lwbox{^^A
%     \leavevmode
%     \hbox to \linewidth{^^A
%       \kern-\leftmargin\relax
%       \hss
%       \usebox0
%       \hss
%       \kern-\rightmargin\relax
%     }^^A
%   }^^A
%   \ifdim\wd0>\lw
%     \sbox0{\small\t}^^A
%     \ifdim\wd0>\linewidth
%       \ifdim\wd0>\lw
%         \sbox0{\footnotesize\t}^^A
%         \ifdim\wd0>\linewidth
%           \ifdim\wd0>\lw
%             \sbox0{\scriptsize\t}^^A
%             \ifdim\wd0>\linewidth
%               \ifdim\wd0>\lw
%                 \sbox0{\tiny\t}^^A
%                 \ifdim\wd0>\linewidth
%                   \lwbox
%                 \else
%                   \usebox0
%                 \fi
%               \else
%                 \lwbox
%               \fi
%             \else
%               \usebox0
%             \fi
%           \else
%             \lwbox
%           \fi
%         \else
%           \usebox0
%         \fi
%       \else
%         \lwbox
%       \fi
%     \else
%       \usebox0
%     \fi
%   \else
%     \lwbox
%   \fi
% \else
%   \usebox0
% \fi
% \end{quote}
% If you have a \xfile{docstrip.cfg} that configures and enables \docstrip's
% TDS installing feature, then some files can already be in the right
% place, see the documentation of \docstrip.
%
% \subsection{Refresh file name databases}
%
% If your \TeX~distribution
% (\teTeX, \mikTeX, \dots) relies on file name databases, you must refresh
% these. For example, \teTeX\ users run \verb|texhash| or
% \verb|mktexlsr|.
%
% \subsection{Some details for the interested}
%
% \paragraph{Attached source.}
%
% The PDF documentation on CTAN also includes the
% \xfile{.dtx} source file. It can be extracted by
% AcrobatReader 6 or higher. Another option is \textsf{pdftk},
% e.g. unpack the file into the current directory:
% \begin{quote}
%   \verb|pdftk bigintcalc.pdf unpack_files output .|
% \end{quote}
%
% \paragraph{Unpacking with \LaTeX.}
% The \xfile{.dtx} chooses its action depending on the format:
% \begin{description}
% \item[\plainTeX:] Run \docstrip\ and extract the files.
% \item[\LaTeX:] Generate the documentation.
% \end{description}
% If you insist on using \LaTeX\ for \docstrip\ (really,
% \docstrip\ does not need \LaTeX), then inform the autodetect routine
% about your intention:
% \begin{quote}
%   \verb|latex \let\install=y% \iffalse meta-comment
%
% File: bigintcalc.dtx
% Version: 2016/05/16 v1.4
% Info: Expandable calculations on big integers
%
% Copyright (C) 2007, 2011, 2012 by
%    Heiko Oberdiek <heiko.oberdiek at googlemail.com>
%    2016
%    https://github.com/ho-tex/oberdiek/issues
%
% This work may be distributed and/or modified under the
% conditions of the LaTeX Project Public License, either
% version 1.3c of this license or (at your option) any later
% version. This version of this license is in
%    http://www.latex-project.org/lppl/lppl-1-3c.txt
% and the latest version of this license is in
%    http://www.latex-project.org/lppl.txt
% and version 1.3 or later is part of all distributions of
% LaTeX version 2005/12/01 or later.
%
% This work has the LPPL maintenance status "maintained".
%
% This Current Maintainer of this work is Heiko Oberdiek.
%
% The Base Interpreter refers to any `TeX-Format',
% because some files are installed in TDS:tex/generic//.
%
% This work consists of the main source file bigintcalc.dtx
% and the derived files
%    bigintcalc.sty, bigintcalc.pdf, bigintcalc.ins, bigintcalc.drv,
%    bigintcalc-test1.tex, bigintcalc-test2.tex,
%    bigintcalc-test3.tex.
%
% Distribution:
%    CTAN:macros/latex/contrib/oberdiek/bigintcalc.dtx
%    CTAN:macros/latex/contrib/oberdiek/bigintcalc.pdf
%
% Unpacking:
%    (a) If bigintcalc.ins is present:
%           tex bigintcalc.ins
%    (b) Without bigintcalc.ins:
%           tex bigintcalc.dtx
%    (c) If you insist on using LaTeX
%           latex \let\install=y\input{bigintcalc.dtx}
%        (quote the arguments according to the demands of your shell)
%
% Documentation:
%    (a) If bigintcalc.drv is present:
%           latex bigintcalc.drv
%    (b) Without bigintcalc.drv:
%           latex bigintcalc.dtx; ...
%    The class ltxdoc loads the configuration file ltxdoc.cfg
%    if available. Here you can specify further options, e.g.
%    use A4 as paper format:
%       \PassOptionsToClass{a4paper}{article}
%
%    Programm calls to get the documentation (example):
%       pdflatex bigintcalc.dtx
%       makeindex -s gind.ist bigintcalc.idx
%       pdflatex bigintcalc.dtx
%       makeindex -s gind.ist bigintcalc.idx
%       pdflatex bigintcalc.dtx
%
% Installation:
%    TDS:tex/generic/oberdiek/bigintcalc.sty
%    TDS:doc/latex/oberdiek/bigintcalc.pdf
%    TDS:doc/latex/oberdiek/test/bigintcalc-test1.tex
%    TDS:doc/latex/oberdiek/test/bigintcalc-test2.tex
%    TDS:doc/latex/oberdiek/test/bigintcalc-test3.tex
%    TDS:source/latex/oberdiek/bigintcalc.dtx
%
%<*ignore>
\begingroup
  \catcode123=1 %
  \catcode125=2 %
  \def\x{LaTeX2e}%
\expandafter\endgroup
\ifcase 0\ifx\install y1\fi\expandafter
         \ifx\csname processbatchFile\endcsname\relax\else1\fi
         \ifx\fmtname\x\else 1\fi\relax
\else\csname fi\endcsname
%</ignore>
%<*install>
\input docstrip.tex
\Msg{************************************************************************}
\Msg{* Installation}
\Msg{* Package: bigintcalc 2016/05/16 v1.4 Expandable calculations on big integers (HO)}
\Msg{************************************************************************}

\keepsilent
\askforoverwritefalse

\let\MetaPrefix\relax
\preamble

This is a generated file.

Project: bigintcalc
Version: 2016/05/16 v1.4

Copyright (C) 2007, 2011, 2012 by
   Heiko Oberdiek <heiko.oberdiek at googlemail.com>

This work may be distributed and/or modified under the
conditions of the LaTeX Project Public License, either
version 1.3c of this license or (at your option) any later
version. This version of this license is in
   http://www.latex-project.org/lppl/lppl-1-3c.txt
and the latest version of this license is in
   http://www.latex-project.org/lppl.txt
and version 1.3 or later is part of all distributions of
LaTeX version 2005/12/01 or later.

This work has the LPPL maintenance status "maintained".

This Current Maintainer of this work is Heiko Oberdiek.

The Base Interpreter refers to any `TeX-Format',
because some files are installed in TDS:tex/generic//.

This work consists of the main source file bigintcalc.dtx
and the derived files
   bigintcalc.sty, bigintcalc.pdf, bigintcalc.ins, bigintcalc.drv,
   bigintcalc-test1.tex, bigintcalc-test2.tex,
   bigintcalc-test3.tex.

\endpreamble
\let\MetaPrefix\DoubleperCent

\generate{%
  \file{bigintcalc.ins}{\from{bigintcalc.dtx}{install}}%
  \file{bigintcalc.drv}{\from{bigintcalc.dtx}{driver}}%
  \usedir{tex/generic/oberdiek}%
  \file{bigintcalc.sty}{\from{bigintcalc.dtx}{package}}%
%  \usedir{doc/latex/oberdiek/test}%
%  \file{bigintcalc-test1.tex}{\from{bigintcalc.dtx}{test1}}%
%  \file{bigintcalc-test2.tex}{\from{bigintcalc.dtx}{test2,etex}}%
%  \file{bigintcalc-test3.tex}{\from{bigintcalc.dtx}{test2,noetex}}%
  \nopreamble
  \nopostamble
  \usedir{source/latex/oberdiek/catalogue}%
  \file{bigintcalc.xml}{\from{bigintcalc.dtx}{catalogue}}%
}

\catcode32=13\relax% active space
\let =\space%
\Msg{************************************************************************}
\Msg{*}
\Msg{* To finish the installation you have to move the following}
\Msg{* file into a directory searched by TeX:}
\Msg{*}
\Msg{*     bigintcalc.sty}
\Msg{*}
\Msg{* To produce the documentation run the file `bigintcalc.drv'}
\Msg{* through LaTeX.}
\Msg{*}
\Msg{* Happy TeXing!}
\Msg{*}
\Msg{************************************************************************}

\endbatchfile
%</install>
%<*ignore>
\fi
%</ignore>
%<*driver>
\NeedsTeXFormat{LaTeX2e}
\ProvidesFile{bigintcalc.drv}%
  [2016/05/16 v1.4 Expandable calculations on big integers (HO)]%
\documentclass{ltxdoc}
\usepackage{wasysym}
\let\iint\relax
\let\iiint\relax
\usepackage[fleqn]{amsmath}

\DeclareMathOperator{\opInv}{Inv}
\DeclareMathOperator{\opAbs}{Abs}
\DeclareMathOperator{\opSgn}{Sgn}
\DeclareMathOperator{\opMin}{Min}
\DeclareMathOperator{\opMax}{Max}
\DeclareMathOperator{\opCmp}{Cmp}
\DeclareMathOperator{\opOdd}{Odd}
\DeclareMathOperator{\opInc}{Inc}
\DeclareMathOperator{\opDec}{Dec}
\DeclareMathOperator{\opAdd}{Add}
\DeclareMathOperator{\opSub}{Sub}
\DeclareMathOperator{\opShl}{Shl}
\DeclareMathOperator{\opShr}{Shr}
\DeclareMathOperator{\opMul}{Mul}
\DeclareMathOperator{\opSqr}{Sqr}
\DeclareMathOperator{\opFac}{Fac}
\DeclareMathOperator{\opPow}{Pow}
\DeclareMathOperator{\opDiv}{Div}
\DeclareMathOperator{\opMod}{Mod}
\DeclareMathOperator{\opInt}{Int}
\DeclareMathOperator{\opODD}{ifodd}

\newcommand*{\Def}{%
  \ensuremath{%
    \mathrel{\mathop{:}}=%
  }%
}
\newcommand*{\op}[1]{%
  \textsf{#1}%
}
\usepackage{holtxdoc}[2011/11/22]
\begin{document}
  \DocInput{bigintcalc.dtx}%
\end{document}
%</driver>
% \fi
%
%
% \CharacterTable
%  {Upper-case    \A\B\C\D\E\F\G\H\I\J\K\L\M\N\O\P\Q\R\S\T\U\V\W\X\Y\Z
%   Lower-case    \a\b\c\d\e\f\g\h\i\j\k\l\m\n\o\p\q\r\s\t\u\v\w\x\y\z
%   Digits        \0\1\2\3\4\5\6\7\8\9
%   Exclamation   \!     Double quote  \"     Hash (number) \#
%   Dollar        \$     Percent       \%     Ampersand     \&
%   Acute accent  \'     Left paren    \(     Right paren   \)
%   Asterisk      \*     Plus          \+     Comma         \,
%   Minus         \-     Point         \.     Solidus       \/
%   Colon         \:     Semicolon     \;     Less than     \<
%   Equals        \=     Greater than  \>     Question mark \?
%   Commercial at \@     Left bracket  \[     Backslash     \\
%   Right bracket \]     Circumflex    \^     Underscore    \_
%   Grave accent  \`     Left brace    \{     Vertical bar  \|
%   Right brace   \}     Tilde         \~}
%
% \GetFileInfo{bigintcalc.drv}
%
% \title{The \xpackage{bigintcalc} package}
% \date{2016/05/16 v1.4}
% \author{Heiko Oberdiek\thanks
% {Please report any issues at https://github.com/ho-tex/oberdiek/issues}\\
% \xemail{heiko.oberdiek at googlemail.com}}
%
% \maketitle
%
% \begin{abstract}
% This package provides expandable arithmetic operations
% with big integers that can exceed \TeX's number limits.
% \end{abstract}
%
% \tableofcontents
%
% \section{Documentation}
%
% \subsection{Introduction}
%
% Package \xpackage{bigintcalc} defines arithmetic operations
% that deal with big integers. Big integers can be given
% either as explicit integer number or as macro code
% that expands to an explicit number. \emph{Big} means that
% there is no limit on the size of the number. Big integers
% may exceed \TeX's range limitation of -2147483647 and 2147483647.
% Only memory issues will limit the usable range.
%
% In opposite to package \xpackage{intcalc}
% unexpandable command tokens are not supported, even if
% they are valid \TeX\ numbers like count registers or
% commands created by \cs{chardef}. Nevertheless they may
% be used, if they are prefixed by \cs{number}.
%
% Also \eTeX's \cs{numexpr} expressions are not supported
% directly in the manner of package \xpackage{intcalc}.
% However they can be given if \cs{the}\cs{numexpr}
% or \cs{number}\cs{numexpr} are used.
%
% The operations have the form of macros that take one or
% two integers as parameter and return the integer result.
% The macro name is a three letter operation name prefixed
% by the package name, e.g. \cs{bigintcalcAdd}|{10}{43}| returns
% |53|.
%
% The macros are fully expandable, exactly two expansion
% steps generate the result. Therefore the operations
% may be used nearly everywhere in \TeX, even inside
% \cs{csname}, file names,  or other
% expandable contexts.
%
% \subsection{Conditions}
%
% \subsubsection{Preconditions}
%
% \begin{itemize}
% \item
%   Arguments can be anything that expands to a number that consists
%   of optional signs and digits.
% \item
%   The arguments and return values must be sound.
%   Zero as divisor or factorials of negative numbers will cause errors.
% \end{itemize}
%
% \subsubsection{Postconditions}
%
% Additional properties of the macros apart from calculating
% a correct result (of course \smiley):
% \begin{itemize}
% \item
%   The macros are fully expandable. Thus they can
%   be used inside \cs{edef}, \cs{csname},
%   for example.
% \item
%   Furthermore exactly two expansion steps calculate the result.
% \item
%   The number consists of one optional minus sign and one or more
%   digits. The first digit is larger than zero for
%   numbers that consists of more than one digit.
%
%   In short, the number format is exactly the same as
%   \cs{number} generates, but without its range limitation.
%   And the tokens (minus sign, digits)
%   have catcode 12 (other).
% \item
%   Call by value is simulated. First the arguments are
%   converted to numbers. Then these numbers are used
%   in the calculations.
%
%   Remember that arguments
%   may contain expensive macros or \eTeX\ expressions.
%   This strategy avoids multiple evaluations of such
%   arguments.
% \end{itemize}
%
% \subsection{Error handling}
%
% Some errors are detected by the macros, example: division by zero.
% In this cases an undefined control sequence is called and causes
% a TeX error message, example: \cs{BigIntCalcError:DivisionByZero}.
% The name of the control sequence contains
% the reason for the error. The \TeX\ error may be ignored.
% Then the operation returns zero as result.
% Because the macros are supposed to work in expandible contexts.
% An traditional error message, however, is not expandable and
% would break these contexts.
%
% \subsection{Operations}
%
% Some definition equations below use the function $\opInt$
% that converts a real number to an integer. The number
% is truncated that means rounding to zero:
% \begin{gather*}
%   \opInt(x) \Def
%   \begin{cases}
%     \lfloor x\rfloor & \text{if $x\geq0$}\\
%     \lceil x\rceil & \text{otherwise}
%   \end{cases}
% \end{gather*}
%
% \subsubsection{\op{Num}}
%
% \begin{declcs}{bigintcalcNum} \M{x}
% \end{declcs}
% Macro \cs{bigintcalcNum} converts its argument to a normalized integer
% number without unnecessary leading zeros or signs.
% The result matches the regular expression:
%\begin{quote}
%\begin{verbatim}
%0|-?[1-9][0-9]*
%\end{verbatim}
%\end{quote}
%
% \subsubsection{\op{Inv}, \op{Abs}, \op{Sgn}}
%
% \begin{declcs}{bigintcalcInv} \M{x}
% \end{declcs}
% Macro \cs{bigintcalcInv} switches the sign.
% \begin{gather*}
%   \opInv(x) \Def -x
% \end{gather*}
%
% \begin{declcs}{bigintcalcAbs} \M{x}
% \end{declcs}
% Macro \cs{bigintcalcAbs} returns the absolute value
% of integer \meta{x}.
% \begin{gather*}
%   \opAbs(x) \Def \vert x\vert
% \end{gather*}
%
% \begin{declcs}{bigintcalcSgn} \M{x}
% \end{declcs}
% Macro \cs{bigintcalcSgn} encodes the sign of \meta{x} as number.
% \begin{gather*}
%   \opSgn(x) \Def
%   \begin{cases}
%     -1& \text{if $x<0$}\\
%      0& \text{if $x=0$}\\
%      1& \text{if $x>0$}
%   \end{cases}
% \end{gather*}
% These return values can easily be distinguished by \cs{ifcase}:
%\begin{quote}
%\begin{verbatim}
%\ifcase\bigintcalcSgn{<x>}
%  $x=0$
%\or
%  $x>0$
%\else
%  $x<0$
%\fi
%\end{verbatim}
%\end{quote}
%
% \subsubsection{\op{Min}, \op{Max}, \op{Cmp}}
%
% \begin{declcs}{bigintcalcMin} \M{x} \M{y}
% \end{declcs}
% Macro \cs{bigintcalcMin} returns the smaller of the two integers.
% \begin{gather*}
%   \opMin(x,y) \Def
%   \begin{cases}
%     x & \text{if $x<y$}\\
%     y & \text{otherwise}
%   \end{cases}
% \end{gather*}
%
% \begin{declcs}{bigintcalcMax} \M{x} \M{y}
% \end{declcs}
% Macro \cs{bigintcalcMax} returns the larger of the two integers.
% \begin{gather*}
%   \opMax(x,y) \Def
%   \begin{cases}
%     x & \text{if $x>y$}\\
%     y & \text{otherwise}
%   \end{cases}
% \end{gather*}
%
% \begin{declcs}{bigintcalcCmp} \M{x} \M{y}
% \end{declcs}
% Macro \cs{bigintcalcCmp} encodes the comparison result as number:
% \begin{gather*}
%   \opCmp(x,y) \Def
%   \begin{cases}
%     -1 & \text{if $x<y$}\\
%      0 & \text{if $x=y$}\\
%      1 & \text{if $x>y$}
%   \end{cases}
% \end{gather*}
% These values can be distinguished by \cs{ifcase}:
%\begin{quote}
%\begin{verbatim}
%\ifcase\bigintcalcCmp{<x>}{<y>}
%  $x=y$
%\or
%  $x>y$
%\else
%  $x<y$
%\fi
%\end{verbatim}
%\end{quote}
%
% \subsubsection{\op{Odd}}
%
% \begin{declcs}{bigintcalcOdd} \M{x}
% \end{declcs}
% \begin{gather*}
%   \opOdd(x) \Def
%   \begin{cases}
%     1 & \text{if $x$ is odd}\\
%     0 & \text{if $x$ is even}
%   \end{cases}
% \end{gather*}
%
% \subsubsection{\op{Inc}, \op{Dec}, \op{Add}, \op{Sub}}
%
% \begin{declcs}{bigintcalcInc} \M{x}
% \end{declcs}
% Macro \cs{bigintcalcInc} increments \meta{x} by one.
% \begin{gather*}
%   \opInc(x) \Def x + 1
% \end{gather*}
%
% \begin{declcs}{bigintcalcDec} \M{x}
% \end{declcs}
% Macro \cs{bigintcalcDec} decrements \meta{x} by one.
% \begin{gather*}
%   \opDec(x) \Def x - 1
% \end{gather*}
%
% \begin{declcs}{bigintcalcAdd} \M{x} \M{y}
% \end{declcs}
% Macro \cs{bigintcalcAdd} adds the two numbers.
% \begin{gather*}
%   \opAdd(x, y) \Def x + y
% \end{gather*}
%
% \begin{declcs}{bigintcalcSub} \M{x} \M{y}
% \end{declcs}
% Macro \cs{bigintcalcSub} calculates the difference.
% \begin{gather*}
%   \opSub(x, y) \Def x - y
% \end{gather*}
%
% \subsubsection{\op{Shl}, \op{Shr}}
%
% \begin{declcs}{bigintcalcShl} \M{x}
% \end{declcs}
% Macro \cs{bigintcalcShl} implements shifting to the left that
% means the number is multiplied by two.
% The sign is preserved.
% \begin{gather*}
%   \opShl(x) \Def x*2
% \end{gather*}
%
% \begin{declcs}{bigintcalcShr} \M{x}
% \end{declcs}
% Macro \cs{bigintcalcShr} implements shifting to the right.
% That is equivalent to an integer division by two.
% The sign is preserved.
% \begin{gather*}
%   \opShr(x) \Def \opInt(x/2)
% \end{gather*}
%
% \subsubsection{\op{Mul}, \op{Sqr}, \op{Fac}, \op{Pow}}
%
% \begin{declcs}{bigintcalcMul} \M{x} \M{y}
% \end{declcs}
% Macro \cs{bigintcalcMul} calculates the product of
% \meta{x} and \meta{y}.
% \begin{gather*}
%   \opMul(x,y) \Def x*y
% \end{gather*}
%
% \begin{declcs}{bigintcalcSqr} \M{x}
% \end{declcs}
% Macro \cs{bigintcalcSqr} returns the square product.
% \begin{gather*}
%   \opSqr(x) \Def x^2
% \end{gather*}
%
% \begin{declcs}{bigintcalcFac} \M{x}
% \end{declcs}
% Macro \cs{bigintcalcFac} returns the factorial of \meta{x}.
% Negative numbers are not permitted.
% \begin{gather*}
%   \opFac(x) \Def x!\qquad\text{for $x\geq0$}
% \end{gather*}
% ($0! = 1$)
%
% \begin{declcs}{bigintcalcPow} M{x} M{y}
% \end{declcs}
% Macro \cs{bigintcalcPow} calculates the value of \meta{x} to the
% power of \meta{y}. The error ``division by zero'' is thrown
% if \meta{x} is zero and \meta{y} is negative.
% permitted:
% \begin{gather*}
%   \opPow(x,y) \Def
%   \opInt(x^y)\qquad\text{for $x\neq0$ or $y\geq0$}
% \end{gather*}
% ($0^0 = 1$)
%
% \subsubsection{\op{Div}, \op{Mul}}
%
% \begin{declcs}{bigintcalcDiv} \M{x} \M{y}
% \end{declcs}
% Macro \cs{bigintcalcDiv} performs an integer division.
% Argument \meta{y} must not be zero.
% \begin{gather*}
%   \opDiv(x,y) \Def \opInt(x/y)\qquad\text{for $y\neq0$}
% \end{gather*}
%
% \begin{declcs}{bigintcalcMod} \M{x} \M{y}
% \end{declcs}
% Macro \cs{bigintcalcMod} gets the remainder of the integer
% division. The sign follows the divisor \meta{y}.
% Argument \meta{y} must not be zero.
% \begin{gather*}
%   \opMod(x,y) \Def x\mathrel{\%}y\qquad\text{for $y\neq0$}
% \end{gather*}
% The result ranges:
% \begin{gather*}
%   -\vert y\vert < \opMod(x,y) \leq 0\qquad\text{for $y<0$}\\
%   0 \leq \opMod(x,y) < y\qquad\text{for $y\geq0$}
% \end{gather*}
%
% \subsection{Interface for programmers}
%
% If the programmer can ensure some more properties about
% the arguments of the operations, then the following
% macros are a little more efficient.
%
% In general numbers must obey the following constraints:
% \begin{itemize}
% \item Plain number: digit tokens only, no command tokens.
% \item Non-negative. Signs are forbidden.
% \item Delimited by exclamation mark. Curly braces
%       around the number are not allowed and will
%       break the code.
% \end{itemize}
%
% \begin{declcs}{BigIntCalcOdd} \meta{number} |!|
% \end{declcs}
% |1|/|0| is returned if \meta{number} is odd/even.
%
% \begin{declcs}{BigIntCalcInc} \meta{number} |!|
% \end{declcs}
% Incrementation.
%
% \begin{declcs}{BigIntCalcDec} \meta{number} |!|
% \end{declcs}
% Decrementation, positive number without zero.
%
% \begin{declcs}{BigIntCalcAdd} \meta{number A} |!| \meta{number B} |!|
% \end{declcs}
% Addition, $A\geq B$.
%
% \begin{declcs}{BigIntCalcSub} \meta{number A} |!| \meta{number B} |!|
% \end{declcs}
% Subtraction, $A\geq B$.
%
% \begin{declcs}{BigIntCalcShl} \meta{number} |!|
% \end{declcs}
% Left shift (multiplication with two).
%
% \begin{declcs}{BigIntCalcShr} \meta{number} |!|
% \end{declcs}
% Right shift (integer division by two).
%
% \begin{declcs}{BigIntCalcMul} \meta{number A} |!| \meta{number B} |!|
% \end{declcs}
% Multiplication, $A\geq B$.
%
% \begin{declcs}{BigIntCalcDiv} \meta{number A} |!| \meta{number B} |!|
% \end{declcs}
% Division operation.
%
% \begin{declcs}{BigIntCalcMod} \meta{number A} |!| \meta{number B} |!|
% \end{declcs}
% Modulo operation.
%
%
% \StopEventually{
% }
%
% \section{Implementation}
%
%    \begin{macrocode}
%<*package>
%    \end{macrocode}
%
% \subsection{Reload check and package identification}
%    Reload check, especially if the package is not used with \LaTeX.
%    \begin{macrocode}
\begingroup\catcode61\catcode48\catcode32=10\relax%
  \catcode13=5 % ^^M
  \endlinechar=13 %
  \catcode35=6 % #
  \catcode39=12 % '
  \catcode44=12 % ,
  \catcode45=12 % -
  \catcode46=12 % .
  \catcode58=12 % :
  \catcode64=11 % @
  \catcode123=1 % {
  \catcode125=2 % }
  \expandafter\let\expandafter\x\csname ver@bigintcalc.sty\endcsname
  \ifx\x\relax % plain-TeX, first loading
  \else
    \def\empty{}%
    \ifx\x\empty % LaTeX, first loading,
      % variable is initialized, but \ProvidesPackage not yet seen
    \else
      \expandafter\ifx\csname PackageInfo\endcsname\relax
        \def\x#1#2{%
          \immediate\write-1{Package #1 Info: #2.}%
        }%
      \else
        \def\x#1#2{\PackageInfo{#1}{#2, stopped}}%
      \fi
      \x{bigintcalc}{The package is already loaded}%
      \aftergroup\endinput
    \fi
  \fi
\endgroup%
%    \end{macrocode}
%    Package identification:
%    \begin{macrocode}
\begingroup\catcode61\catcode48\catcode32=10\relax%
  \catcode13=5 % ^^M
  \endlinechar=13 %
  \catcode35=6 % #
  \catcode39=12 % '
  \catcode40=12 % (
  \catcode41=12 % )
  \catcode44=12 % ,
  \catcode45=12 % -
  \catcode46=12 % .
  \catcode47=12 % /
  \catcode58=12 % :
  \catcode64=11 % @
  \catcode91=12 % [
  \catcode93=12 % ]
  \catcode123=1 % {
  \catcode125=2 % }
  \expandafter\ifx\csname ProvidesPackage\endcsname\relax
    \def\x#1#2#3[#4]{\endgroup
      \immediate\write-1{Package: #3 #4}%
      \xdef#1{#4}%
    }%
  \else
    \def\x#1#2[#3]{\endgroup
      #2[{#3}]%
      \ifx#1\@undefined
        \xdef#1{#3}%
      \fi
      \ifx#1\relax
        \xdef#1{#3}%
      \fi
    }%
  \fi
\expandafter\x\csname ver@bigintcalc.sty\endcsname
\ProvidesPackage{bigintcalc}%
  [2016/05/16 v1.4 Expandable calculations on big integers (HO)]%
%    \end{macrocode}
%
% \subsection{Catcodes}
%
%    \begin{macrocode}
\begingroup\catcode61\catcode48\catcode32=10\relax%
  \catcode13=5 % ^^M
  \endlinechar=13 %
  \catcode123=1 % {
  \catcode125=2 % }
  \catcode64=11 % @
  \def\x{\endgroup
    \expandafter\edef\csname BIC@AtEnd\endcsname{%
      \endlinechar=\the\endlinechar\relax
      \catcode13=\the\catcode13\relax
      \catcode32=\the\catcode32\relax
      \catcode35=\the\catcode35\relax
      \catcode61=\the\catcode61\relax
      \catcode64=\the\catcode64\relax
      \catcode123=\the\catcode123\relax
      \catcode125=\the\catcode125\relax
    }%
  }%
\x\catcode61\catcode48\catcode32=10\relax%
\catcode13=5 % ^^M
\endlinechar=13 %
\catcode35=6 % #
\catcode64=11 % @
\catcode123=1 % {
\catcode125=2 % }
\def\TMP@EnsureCode#1#2{%
  \edef\BIC@AtEnd{%
    \BIC@AtEnd
    \catcode#1=\the\catcode#1\relax
  }%
  \catcode#1=#2\relax
}
\TMP@EnsureCode{33}{12}% !
\TMP@EnsureCode{36}{14}% $ (comment!)
\TMP@EnsureCode{38}{14}% & (comment!)
\TMP@EnsureCode{40}{12}% (
\TMP@EnsureCode{41}{12}% )
\TMP@EnsureCode{42}{12}% *
\TMP@EnsureCode{43}{12}% +
\TMP@EnsureCode{45}{12}% -
\TMP@EnsureCode{46}{12}% .
\TMP@EnsureCode{47}{12}% /
\TMP@EnsureCode{58}{11}% : (letter!)
\TMP@EnsureCode{60}{12}% <
\TMP@EnsureCode{62}{12}% >
\TMP@EnsureCode{63}{14}% ? (comment!)
\TMP@EnsureCode{91}{12}% [
\TMP@EnsureCode{93}{12}% ]
\edef\BIC@AtEnd{\BIC@AtEnd\noexpand\endinput}
\begingroup\expandafter\expandafter\expandafter\endgroup
\expandafter\ifx\csname BIC@TestMode\endcsname\relax
\else
  \catcode63=9 % ? (ignore)
\fi
? \let\BIC@@TestMode\BIC@TestMode
%    \end{macrocode}
%
% \subsection{\eTeX\ detection}
%
%    \begin{macrocode}
\begingroup\expandafter\expandafter\expandafter\endgroup
\expandafter\ifx\csname numexpr\endcsname\relax
  \catcode36=9 % $ (ignore)
\else
  \catcode38=9 % & (ignore)
\fi
%    \end{macrocode}
%
% \subsection{Help macros}
%
%    \begin{macro}{\BIC@Fi}
%    \begin{macrocode}
\let\BIC@Fi\fi
%    \end{macrocode}
%    \end{macro}
%    \begin{macro}{\BIC@AfterFi}
%    \begin{macrocode}
\def\BIC@AfterFi#1#2\BIC@Fi{\fi#1}%
%    \end{macrocode}
%    \end{macro}
%    \begin{macro}{\BIC@AfterFiFi}
%    \begin{macrocode}
\def\BIC@AfterFiFi#1#2\BIC@Fi{\fi\fi#1}%
%    \end{macrocode}
%    \end{macro}
%    \begin{macro}{\BIC@AfterFiFiFi}
%    \begin{macrocode}
\def\BIC@AfterFiFiFi#1#2\BIC@Fi{\fi\fi\fi#1}%
%    \end{macrocode}
%    \end{macro}
%
%    \begin{macro}{\BIC@Space}
%    \begin{macrocode}
\begingroup
  \def\x#1{\endgroup
    \let\BIC@Space= #1%
  }%
\x{ }
%    \end{macrocode}
%    \end{macro}
%
% \subsection{Expand number}
%
%    \begin{macrocode}
\begingroup\expandafter\expandafter\expandafter\endgroup
\expandafter\ifx\csname RequirePackage\endcsname\relax
  \def\TMP@RequirePackage#1[#2]{%
    \begingroup\expandafter\expandafter\expandafter\endgroup
    \expandafter\ifx\csname ver@#1.sty\endcsname\relax
      \input #1.sty\relax
    \fi
  }%
  \TMP@RequirePackage{pdftexcmds}[2007/11/11]%
\else
  \RequirePackage{pdftexcmds}[2007/11/11]%
\fi
%    \end{macrocode}
%
%    \begin{macrocode}
\begingroup\expandafter\expandafter\expandafter\endgroup
\expandafter\ifx\csname pdf@escapehex\endcsname\relax
%    \end{macrocode}
%
%    \begin{macro}{\BIC@Expand}
%    \begin{macrocode}
  \def\BIC@Expand#1{%
    \romannumeral0%
    \BIC@@Expand#1!\@nil{}%
  }%
%    \end{macrocode}
%    \end{macro}
%    \begin{macro}{\BIC@@Expand}
%    \begin{macrocode}
  \def\BIC@@Expand#1#2\@nil#3{%
    \expandafter\ifcat\noexpand#1\relax
      \expandafter\@firstoftwo
    \else
      \expandafter\@secondoftwo
    \fi
    {%
      \expandafter\BIC@@Expand#1#2\@nil{#3}%
    }{%
      \ifx#1!%
        \expandafter\@firstoftwo
      \else
        \expandafter\@secondoftwo
      \fi
      { #3}{%
        \BIC@@Expand#2\@nil{#3#1}%
      }%
    }%
  }%
%    \end{macrocode}
%    \end{macro}
%    \begin{macro}{\@firstoftwo}
%    \begin{macrocode}
  \expandafter\ifx\csname @firstoftwo\endcsname\relax
    \long\def\@firstoftwo#1#2{#1}%
  \fi
%    \end{macrocode}
%    \end{macro}
%    \begin{macro}{\@secondoftwo}
%    \begin{macrocode}
  \expandafter\ifx\csname @secondoftwo\endcsname\relax
    \long\def\@secondoftwo#1#2{#2}%
  \fi
%    \end{macrocode}
%    \end{macro}
%
%    \begin{macrocode}
\else
%    \end{macrocode}
%    \begin{macro}{\BIC@Expand}
%    \begin{macrocode}
  \def\BIC@Expand#1{%
    \romannumeral0\expandafter\expandafter\expandafter\BIC@Space
    \pdf@unescapehex{%
      \expandafter\expandafter\expandafter
      \BIC@StripHexSpace\pdf@escapehex{#1}20\@nil
    }%
  }%
%    \end{macrocode}
%    \end{macro}
%    \begin{macro}{\BIC@StripHexSpace}
%    \begin{macrocode}
  \def\BIC@StripHexSpace#120#2\@nil{%
    #1%
    \ifx\\#2\\%
    \else
      \BIC@AfterFi{%
        \BIC@StripHexSpace#2\@nil
      }%
    \BIC@Fi
  }%
%    \end{macrocode}
%    \end{macro}
%    \begin{macrocode}
\fi
%    \end{macrocode}
%
% \subsection{Normalize expanded number}
%
%    \begin{macro}{\BIC@Normalize}
%    |#1|: result sign\\
%    |#2|: first token of number
%    \begin{macrocode}
\def\BIC@Normalize#1#2{%
  \ifx#2-%
    \ifx\\#1\\%
      \BIC@AfterFiFi{%
        \BIC@Normalize-%
      }%
    \else
      \BIC@AfterFiFi{%
        \BIC@Normalize{}%
      }%
    \fi
  \else
    \ifx#2+%
      \BIC@AfterFiFi{%
        \BIC@Normalize{#1}%
      }%
    \else
      \ifx#20%
        \BIC@AfterFiFiFi{%
          \BIC@NormalizeZero{#1}%
        }%
      \else
        \BIC@AfterFiFiFi{%
          \BIC@NormalizeDigits#1#2%
        }%
      \fi
    \fi
  \BIC@Fi
}
%    \end{macrocode}
%    \end{macro}
%    \begin{macro}{\BIC@NormalizeZero}
%    \begin{macrocode}
\def\BIC@NormalizeZero#1#2{%
  \ifx#2!%
    \BIC@AfterFi{ 0}%
  \else
    \ifx#20%
      \BIC@AfterFiFi{%
        \BIC@NormalizeZero{#1}%
      }%
    \else
      \BIC@AfterFiFi{%
        \BIC@NormalizeDigits#1#2%
      }%
    \fi
  \BIC@Fi
}
%    \end{macrocode}
%    \end{macro}
%    \begin{macro}{\BIC@NormalizeDigits}
%    \begin{macrocode}
\def\BIC@NormalizeDigits#1!{ #1}
%    \end{macrocode}
%    \end{macro}
%
% \subsection{\op{Num}}
%
%    \begin{macro}{\bigintcalcNum}
%    \begin{macrocode}
\def\bigintcalcNum#1{%
  \romannumeral0%
  \expandafter\expandafter\expandafter\BIC@Normalize
  \expandafter\expandafter\expandafter{%
  \expandafter\expandafter\expandafter}%
  \BIC@Expand{#1}!%
}
%    \end{macrocode}
%    \end{macro}
%
% \subsection{\op{Inv}, \op{Abs}, \op{Sgn}}
%
%
%    \begin{macro}{\bigintcalcInv}
%    \begin{macrocode}
\def\bigintcalcInv#1{%
  \romannumeral0\expandafter\expandafter\expandafter\BIC@Space
  \bigintcalcNum{-#1}%
}
%    \end{macrocode}
%    \end{macro}
%
%    \begin{macro}{\bigintcalcAbs}
%    \begin{macrocode}
\def\bigintcalcAbs#1{%
  \romannumeral0%
  \expandafter\expandafter\expandafter\BIC@Abs
  \bigintcalcNum{#1}%
}
%    \end{macrocode}
%    \end{macro}
%    \begin{macro}{\BIC@Abs}
%    \begin{macrocode}
\def\BIC@Abs#1{%
  \ifx#1-%
    \expandafter\BIC@Space
  \else
    \expandafter\BIC@Space
    \expandafter#1%
  \fi
}
%    \end{macrocode}
%    \end{macro}
%
%    \begin{macro}{\bigintcalcSgn}
%    \begin{macrocode}
\def\bigintcalcSgn#1{%
  \number
  \expandafter\expandafter\expandafter\BIC@Sgn
  \bigintcalcNum{#1}! %
}
%    \end{macrocode}
%    \end{macro}
%    \begin{macro}{\BIC@Sgn}
%    \begin{macrocode}
\def\BIC@Sgn#1#2!{%
  \ifx#1-%
    -1%
  \else
    \ifx#10%
      0%
    \else
      1%
    \fi
  \fi
}
%    \end{macrocode}
%    \end{macro}
%
% \subsection{\op{Cmp}, \op{Min}, \op{Max}}
%
%    \begin{macro}{\bigintcalcCmp}
%    \begin{macrocode}
\def\bigintcalcCmp#1#2{%
  \number
  \expandafter\expandafter\expandafter\BIC@Cmp
  \bigintcalcNum{#2}!{#1}%
}
%    \end{macrocode}
%    \end{macro}
%    \begin{macro}{\BIC@Cmp}
%    \begin{macrocode}
\def\BIC@Cmp#1!#2{%
  \expandafter\expandafter\expandafter\BIC@@Cmp
  \bigintcalcNum{#2}!#1!%
}
%    \end{macrocode}
%    \end{macro}
%    \begin{macro}{\BIC@@Cmp}
%    \begin{macrocode}
\def\BIC@@Cmp#1#2!#3#4!{%
  \ifx#1-%
    \ifx#3-%
      \BIC@AfterFiFi{%
        \BIC@@Cmp#4!#2!%
      }%
    \else
      \BIC@AfterFiFi{%
        -1 %
      }%
    \fi
  \else
    \ifx#3-%
      \BIC@AfterFiFi{%
        1 %
      }%
    \else
      \BIC@AfterFiFi{%
        \BIC@CmpLength#1#2!#3#4!#1#2!#3#4!%
      }%
    \fi
  \BIC@Fi
}
%    \end{macrocode}
%    \end{macro}
%    \begin{macro}{\BIC@PosCmp}
%    \begin{macrocode}
\def\BIC@PosCmp#1!#2!{%
  \BIC@CmpLength#1!#2!#1!#2!%
}
%    \end{macrocode}
%    \end{macro}
%    \begin{macro}{\BIC@CmpLength}
%    \begin{macrocode}
\def\BIC@CmpLength#1#2!#3#4!{%
  \ifx\\#2\\%
    \ifx\\#4\\%
      \BIC@AfterFiFi\BIC@CmpDiff
    \else
      \BIC@AfterFiFi{%
        \BIC@CmpResult{-1}%
      }%
    \fi
  \else
    \ifx\\#4\\%
      \BIC@AfterFiFi{%
        \BIC@CmpResult1%
      }%
    \else
      \BIC@AfterFiFi{%
        \BIC@CmpLength#2!#4!%
      }%
    \fi
  \BIC@Fi
}
%    \end{macrocode}
%    \end{macro}
%    \begin{macro}{\BIC@CmpResult}
%    \begin{macrocode}
\def\BIC@CmpResult#1#2!#3!{#1 }
%    \end{macrocode}
%    \end{macro}
%    \begin{macro}{\BIC@CmpDiff}
%    \begin{macrocode}
\def\BIC@CmpDiff#1#2!#3#4!{%
  \ifnum#1<#3 %
    \BIC@AfterFi{%
      -1 %
    }%
  \else
    \ifnum#1>#3 %
      \BIC@AfterFiFi{%
        1 %
      }%
    \else
      \ifx\\#2\\%
        \BIC@AfterFiFiFi{%
          0 %
        }%
      \else
        \BIC@AfterFiFiFi{%
          \BIC@CmpDiff#2!#4!%
        }%
      \fi
    \fi
  \BIC@Fi
}
%    \end{macrocode}
%    \end{macro}
%
%    \begin{macro}{\bigintcalcMin}
%    \begin{macrocode}
\def\bigintcalcMin#1{%
  \romannumeral0%
  \expandafter\expandafter\expandafter\BIC@MinMax
  \bigintcalcNum{#1}!-!%
}
%    \end{macrocode}
%    \end{macro}
%    \begin{macro}{\bigintcalcMax}
%    \begin{macrocode}
\def\bigintcalcMax#1{%
  \romannumeral0%
  \expandafter\expandafter\expandafter\BIC@MinMax
  \bigintcalcNum{#1}!!%
}
%    \end{macrocode}
%    \end{macro}
%    \begin{macro}{\BIC@MinMax}
%    |#1|: $x$\\
%    |#2|: sign for comparison\\
%    |#3|: $y$
%    \begin{macrocode}
\def\BIC@MinMax#1!#2!#3{%
  \expandafter\expandafter\expandafter\BIC@@MinMax
  \bigintcalcNum{#3}!#1!#2!%
}
%    \end{macrocode}
%    \end{macro}
%    \begin{macro}{\BIC@@MinMax}
%    |#1|: $y$\\
%    |#2|: $x$\\
%    |#3|: sign for comparison
%    \begin{macrocode}
\def\BIC@@MinMax#1!#2!#3!{%
  \ifnum\BIC@@Cmp#1!#2!=#31 %
    \BIC@AfterFi{ #1}%
  \else
    \BIC@AfterFi{ #2}%
  \BIC@Fi
}
%    \end{macrocode}
%    \end{macro}
%
% \subsection{\op{Odd}}
%
%    \begin{macro}{\bigintcalcOdd}
%    \begin{macrocode}
\def\bigintcalcOdd#1{%
  \romannumeral0%
  \expandafter\expandafter\expandafter\BIC@Odd
  \bigintcalcAbs{#1}!%
}
%    \end{macrocode}
%    \end{macro}
%    \begin{macro}{\BigIntCalcOdd}
%    \begin{macrocode}
\def\BigIntCalcOdd#1!{%
  \romannumeral0%
  \BIC@Odd#1!%
}
%    \end{macrocode}
%    \end{macro}
%    \begin{macro}{\BIC@Odd}
%    |#1|: $x$
%    \begin{macrocode}
\def\BIC@Odd#1#2{%
  \ifx#2!%
    \ifodd#1 %
      \BIC@AfterFiFi{ 1}%
    \else
      \BIC@AfterFiFi{ 0}%
    \fi
  \else
    \expandafter\BIC@Odd\expandafter#2%
  \BIC@Fi
}
%    \end{macrocode}
%    \end{macro}
%
% \subsection{\op{Inc}, \op{Dec}}
%
%    \begin{macro}{\bigintcalcInc}
%    \begin{macrocode}
\def\bigintcalcInc#1{%
  \romannumeral0%
  \expandafter\expandafter\expandafter\BIC@IncSwitch
  \bigintcalcNum{#1}!%
}
%    \end{macrocode}
%    \end{macro}
%    \begin{macro}{\BIC@IncSwitch}
%    \begin{macrocode}
\def\BIC@IncSwitch#1#2!{%
  \ifcase\BIC@@Cmp#1#2!-1!%
    \BIC@AfterFi{ 0}%
  \or
    \BIC@AfterFi{%
      \BIC@Inc#1#2!{}%
    }%
  \else
    \BIC@AfterFi{%
      \expandafter-\romannumeral0%
      \BIC@Dec#2!{}%
    }%
  \BIC@Fi
}
%    \end{macrocode}
%    \end{macro}
%
%    \begin{macro}{\bigintcalcDec}
%    \begin{macrocode}
\def\bigintcalcDec#1{%
  \romannumeral0%
  \expandafter\expandafter\expandafter\BIC@DecSwitch
  \bigintcalcNum{#1}!%
}
%    \end{macrocode}
%    \end{macro}
%    \begin{macro}{\BIC@DecSwitch}
%    \begin{macrocode}
\def\BIC@DecSwitch#1#2!{%
  \ifcase\BIC@Sgn#1#2! %
    \BIC@AfterFi{ -1}%
  \or
    \BIC@AfterFi{%
      \BIC@Dec#1#2!{}%
    }%
  \else
    \BIC@AfterFi{%
      \expandafter-\romannumeral0%
      \BIC@Inc#2!{}%
    }%
  \BIC@Fi
}
%    \end{macrocode}
%    \end{macro}
%
%    \begin{macro}{\BigIntCalcInc}
%    \begin{macrocode}
\def\BigIntCalcInc#1!{%
  \romannumeral0\BIC@Inc#1!{}%
}
%    \end{macrocode}
%    \end{macro}
%    \begin{macro}{\BigIntCalcDec}
%    \begin{macrocode}
\def\BigIntCalcDec#1!{%
  \romannumeral0\BIC@Dec#1!{}%
}
%    \end{macrocode}
%    \end{macro}
%
%    \begin{macro}{\BIC@Inc}
%    \begin{macrocode}
\def\BIC@Inc#1#2!#3{%
  \ifx\\#2\\%
    \BIC@AfterFi{%
      \BIC@@Inc1#1#3!{}%
    }%
  \else
    \BIC@AfterFi{%
      \BIC@Inc#2!{#1#3}%
    }%
  \BIC@Fi
}
%    \end{macrocode}
%    \end{macro}
%    \begin{macro}{\BIC@@Inc}
%    \begin{macrocode}
\def\BIC@@Inc#1#2#3!#4{%
  \ifcase#1 %
    \ifx\\#3\\%
      \BIC@AfterFiFi{ #2#4}%
    \else
      \BIC@AfterFiFi{%
        \BIC@@Inc0#3!{#2#4}%
      }%
    \fi
  \else
    \ifnum#2<9 %
      \BIC@AfterFiFi{%
&       \expandafter\BIC@@@Inc\the\numexpr#2+1\relax
$       \expandafter\expandafter\expandafter\BIC@@@Inc
$       \ifcase#2 \expandafter1%
$       \or\expandafter2%
$       \or\expandafter3%
$       \or\expandafter4%
$       \or\expandafter5%
$       \or\expandafter6%
$       \or\expandafter7%
$       \or\expandafter8%
$       \or\expandafter9%
$?      \else\BigIntCalcError:ThisCannotHappen%
$       \fi
        0#3!{#4}%
      }%
    \else
      \BIC@AfterFiFi{%
        \BIC@@@Inc01#3!{#4}%
      }%
    \fi
  \BIC@Fi
}
%    \end{macrocode}
%    \end{macro}
%    \begin{macro}{\BIC@@@Inc}
%    \begin{macrocode}
\def\BIC@@@Inc#1#2#3!#4{%
  \ifx\\#3\\%
    \ifnum#2=1 %
      \BIC@AfterFiFi{ 1#1#4}%
    \else
      \BIC@AfterFiFi{ #1#4}%
    \fi
  \else
    \BIC@AfterFi{%
      \BIC@@Inc#2#3!{#1#4}%
    }%
  \BIC@Fi
}
%    \end{macrocode}
%    \end{macro}
%
%    \begin{macro}{\BIC@Dec}
%    \begin{macrocode}
\def\BIC@Dec#1#2!#3{%
  \ifx\\#2\\%
    \BIC@AfterFi{%
      \BIC@@Dec1#1#3!{}%
    }%
  \else
    \BIC@AfterFi{%
      \BIC@Dec#2!{#1#3}%
    }%
  \BIC@Fi
}
%    \end{macrocode}
%    \end{macro}
%    \begin{macro}{\BIC@@Dec}
%    \begin{macrocode}
\def\BIC@@Dec#1#2#3!#4{%
  \ifcase#1 %
    \ifx\\#3\\%
      \BIC@AfterFiFi{ #2#4}%
    \else
      \BIC@AfterFiFi{%
        \BIC@@Dec0#3!{#2#4}%
      }%
    \fi
  \else
    \ifnum#2>0 %
      \BIC@AfterFiFi{%
&       \expandafter\BIC@@@Dec\the\numexpr#2-1\relax
$       \expandafter\expandafter\expandafter\BIC@@@Dec
$       \ifcase#2
$?        \BigIntCalcError:ThisCannotHappen%
$       \or\expandafter0%
$       \or\expandafter1%
$       \or\expandafter2%
$       \or\expandafter3%
$       \or\expandafter4%
$       \or\expandafter5%
$       \or\expandafter6%
$       \or\expandafter7%
$       \or\expandafter8%
$?      \else\BigIntCalcError:ThisCannotHappen%
$       \fi
        0#3!{#4}%
      }%
    \else
      \BIC@AfterFiFi{%
        \BIC@@@Dec91#3!{#4}%
      }%
    \fi
  \BIC@Fi
}
%    \end{macrocode}
%    \end{macro}
%    \begin{macro}{\BIC@@@Dec}
%    \begin{macrocode}
\def\BIC@@@Dec#1#2#3!#4{%
  \ifx\\#3\\%
    \ifcase#1 %
      \ifx\\#4\\%
        \BIC@AfterFiFiFi{ 0}%
      \else
        \BIC@AfterFiFiFi{ #4}%
      \fi
    \else
      \BIC@AfterFiFi{ #1#4}%
    \fi
  \else
    \BIC@AfterFi{%
      \BIC@@Dec#2#3!{#1#4}%
    }%
  \BIC@Fi
}
%    \end{macrocode}
%    \end{macro}
%
% \subsection{\op{Add}, \op{Sub}}
%
%    \begin{macro}{\bigintcalcAdd}
%    \begin{macrocode}
\def\bigintcalcAdd#1{%
  \romannumeral0%
  \expandafter\expandafter\expandafter\BIC@Add
  \bigintcalcNum{#1}!%
}
%    \end{macrocode}
%    \end{macro}
%    \begin{macro}{\BIC@Add}
%    \begin{macrocode}
\def\BIC@Add#1!#2{%
  \expandafter\expandafter\expandafter
  \BIC@AddSwitch\bigintcalcNum{#2}!#1!%
}
%    \end{macrocode}
%    \end{macro}
%    \begin{macro}{\bigintcalcSub}
%    \begin{macrocode}
\def\bigintcalcSub#1#2{%
  \romannumeral0%
  \expandafter\expandafter\expandafter\BIC@Add
  \bigintcalcNum{-#2}!{#1}%
}
%    \end{macrocode}
%    \end{macro}
%
%    \begin{macro}{\BIC@AddSwitch}
%    Decision table for \cs{BIC@AddSwitch}.
%    \begin{quote}
%    \begin{tabular}[t]{@{}|l|l|l|l|l|@{}}
%      \hline
%      $x<0$ & $y<0$ & $-x>-y$ & $-$ & $\opAdd(-x,-y)$\\
%      \cline{3-3}\cline{5-5}
%            &       & else    &     & $\opAdd(-y,-x)$\\
%      \cline{2-5}
%            & else  & $-x> y$ & $-$ & $\opSub(-x, y)$\\
%      \cline{3-5}
%            &       & $-x= y$ &     & $0$\\
%      \cline{3-5}
%            &       & else    & $+$ & $\opSub( y,-x)$\\
%      \hline
%      else  & $y<0$ & $ x>-y$ & $+$ & $\opSub( x,-y)$\\
%      \cline{3-5}
%            &       & $ x=-y$ &     & $0$\\
%      \cline{3-5}
%            &       & else    & $-$ & $\opSub(-y, x)$\\
%      \cline{2-5}
%            & else  & $ x> y$ & $+$ & $\opAdd( x, y)$\\
%      \cline{3-3}\cline{5-5}
%            &       & else    &     & $\opAdd( y, x)$\\
%      \hline
%    \end{tabular}
%    \end{quote}
%    \begin{macrocode}
\def\BIC@AddSwitch#1#2!#3#4!{%
  \ifx#1-% x < 0
    \ifx#3-% y < 0
      \expandafter-\romannumeral0%
      \ifnum\BIC@PosCmp#2!#4!=1 % -x > -y
        \BIC@AfterFiFiFi{%
          \BIC@AddXY#2!#4!!!%
        }%
      \else % -x <= -y
        \BIC@AfterFiFiFi{%
          \BIC@AddXY#4!#2!!!%
        }%
      \fi
    \else % y >= 0
      \ifcase\BIC@PosCmp#2!#3#4!% -x = y
        \BIC@AfterFiFiFi{ 0}%
      \or % -x > y
        \expandafter-\romannumeral0%
        \BIC@AfterFiFiFi{%
          \BIC@SubXY#2!#3#4!!!%
        }%
      \else % -x <= y
        \BIC@AfterFiFiFi{%
          \BIC@SubXY#3#4!#2!!!%
        }%
      \fi
    \fi
  \else % x >= 0
    \ifx#3-% y < 0
      \ifcase\BIC@PosCmp#1#2!#4!% x = -y
        \BIC@AfterFiFiFi{ 0}%
      \or % x > -y
        \BIC@AfterFiFiFi{%
          \BIC@SubXY#1#2!#4!!!%
        }%
      \else % x <= -y
        \expandafter-\romannumeral0%
        \BIC@AfterFiFiFi{%
          \BIC@SubXY#4!#1#2!!!%
        }%
      \fi
    \else % y >= 0
      \ifnum\BIC@PosCmp#1#2!#3#4!=1 % x > y
        \BIC@AfterFiFiFi{%
          \BIC@AddXY#1#2!#3#4!!!%
        }%
      \else % x <= y
        \BIC@AfterFiFiFi{%
          \BIC@AddXY#3#4!#1#2!!!%
        }%
      \fi
    \fi
  \BIC@Fi
}
%    \end{macrocode}
%    \end{macro}
%
%    \begin{macro}{\BigIntCalcAdd}
%    \begin{macrocode}
\def\BigIntCalcAdd#1!#2!{%
  \romannumeral0\BIC@AddXY#1!#2!!!%
}
%    \end{macrocode}
%    \end{macro}
%    \begin{macro}{\BigIntCalcSub}
%    \begin{macrocode}
\def\BigIntCalcSub#1!#2!{%
  \romannumeral0\BIC@SubXY#1!#2!!!%
}
%    \end{macrocode}
%    \end{macro}
%
%    \begin{macro}{\BIC@AddXY}
%    \begin{macrocode}
\def\BIC@AddXY#1#2!#3#4!#5!#6!{%
  \ifx\\#2\\%
    \ifx\\#3\\%
      \BIC@AfterFiFi{%
        \BIC@DoAdd0!#1#5!#60!%
      }%
    \else
      \BIC@AfterFiFi{%
        \BIC@DoAdd0!#1#5!#3#6!%
      }%
    \fi
  \else
    \ifx\\#4\\%
      \ifx\\#3\\%
        \BIC@AfterFiFiFi{%
          \BIC@AddXY#2!{}!#1#5!#60!%
        }%
      \else
        \BIC@AfterFiFiFi{%
          \BIC@AddXY#2!{}!#1#5!#3#6!%
        }%
      \fi
    \else
      \BIC@AfterFiFi{%
        \BIC@AddXY#2!#4!#1#5!#3#6!%
      }%
    \fi
  \BIC@Fi
}
%    \end{macrocode}
%    \end{macro}
%    \begin{macro}{\BIC@DoAdd}
%    |#1|: carry\\
%    |#2|: reverted result\\
%    |#3#4|: reverted $x$\\
%    |#5#6|: reverted $y$
%    \begin{macrocode}
\def\BIC@DoAdd#1#2!#3#4!#5#6!{%
  \ifx\\#4\\%
    \BIC@AfterFi{%
&     \expandafter\BIC@Space
&     \the\numexpr#1+#3+#5\relax#2%
$     \expandafter\expandafter\expandafter\BIC@AddResult
$     \BIC@AddDigit#1#3#5#2%
    }%
  \else
    \BIC@AfterFi{%
      \expandafter\expandafter\expandafter\BIC@DoAdd
      \BIC@AddDigit#1#3#5#2!#4!#6!%
    }%
  \BIC@Fi
}
%    \end{macrocode}
%    \end{macro}
%    \begin{macro}{\BIC@AddResult}
%    \begin{macrocode}
$ \def\BIC@AddResult#1{%
$   \ifx#10%
$     \expandafter\BIC@Space
$   \else
$     \expandafter\BIC@Space\expandafter#1%
$   \fi
$ }%
%    \end{macrocode}
%    \end{macro}
%    \begin{macro}{\BIC@AddDigit}
%    |#1|: carry\\
%    |#2|: digit of $x$\\
%    |#3|: digit of $y$
%    \begin{macrocode}
\def\BIC@AddDigit#1#2#3{%
  \romannumeral0%
& \expandafter\BIC@@AddDigit\the\numexpr#1+#2+#3!%
$ \expandafter\BIC@@AddDigit\number%
$ \csname
$   BIC@AddCarry%
$   \ifcase#1 %
$     #2%
$   \else
$     \ifcase#2 1\or2\or3\or4\or5\or6\or7\or8\or9\or10\fi
$   \fi
$ \endcsname#3!%
}
%    \end{macrocode}
%    \end{macro}
%    \begin{macro}{\BIC@@AddDigit}
%    \begin{macrocode}
\def\BIC@@AddDigit#1!{%
  \ifnum#1<10 %
    \BIC@AfterFi{ 0#1}%
  \else
    \BIC@AfterFi{ #1}%
  \BIC@Fi
}
%    \end{macrocode}
%    \end{macro}
%    \begin{macro}{\BIC@AddCarry0}
%    \begin{macrocode}
$ \expandafter\def\csname BIC@AddCarry0\endcsname#1{#1}%
%    \end{macrocode}
%    \end{macro}
%    \begin{macro}{\BIC@AddCarry10}
%    \begin{macrocode}
$ \expandafter\def\csname BIC@AddCarry10\endcsname#1{1#1}%
%    \end{macrocode}
%    \end{macro}
%    \begin{macro}{\BIC@AddCarry[1-9]}
%    \begin{macrocode}
$ \def\BIC@Temp#1#2{%
$   \expandafter\def\csname BIC@AddCarry#1\endcsname##1{%
$     \ifcase##1 #1\or
$     #2%
$?    \else\BigIntCalcError:ThisCannotHappen%
$     \fi
$   }%
$ }%
$ \BIC@Temp 0{1\or2\or3\or4\or5\or6\or7\or8\or9}%
$ \BIC@Temp 1{2\or3\or4\or5\or6\or7\or8\or9\or10}%
$ \BIC@Temp 2{3\or4\or5\or6\or7\or8\or9\or10\or11}%
$ \BIC@Temp 3{4\or5\or6\or7\or8\or9\or10\or11\or12}%
$ \BIC@Temp 4{5\or6\or7\or8\or9\or10\or11\or12\or13}%
$ \BIC@Temp 5{6\or7\or8\or9\or10\or11\or12\or13\or14}%
$ \BIC@Temp 6{7\or8\or9\or10\or11\or12\or13\or14\or15}%
$ \BIC@Temp 7{8\or9\or10\or11\or12\or13\or14\or15\or16}%
$ \BIC@Temp 8{9\or10\or11\or12\or13\or14\or15\or16\or17}%
$ \BIC@Temp 9{10\or11\or12\or13\or14\or15\or16\or17\or18}%
%    \end{macrocode}
%    \end{macro}
%
%    \begin{macro}{\BIC@SubXY}
%    Preconditions:
%    \begin{itemize}
%    \item $x > y$, $x \geq 0$, and $y >= 0$
%    \item digits($x$) = digits($y$)
%    \end{itemize}
%    \begin{macrocode}
\def\BIC@SubXY#1#2!#3#4!#5!#6!{%
  \ifx\\#2\\%
    \ifx\\#3\\%
      \BIC@AfterFiFi{%
        \BIC@DoSub0!#1#5!#60!%
      }%
    \else
      \BIC@AfterFiFi{%
        \BIC@DoSub0!#1#5!#3#6!%
      }%
    \fi
  \else
    \ifx\\#4\\%
      \ifx\\#3\\%
        \BIC@AfterFiFiFi{%
          \BIC@SubXY#2!{}!#1#5!#60!%
        }%
      \else
        \BIC@AfterFiFiFi{%
          \BIC@SubXY#2!{}!#1#5!#3#6!%
        }%
      \fi
    \else
      \BIC@AfterFiFi{%
        \BIC@SubXY#2!#4!#1#5!#3#6!%
      }%
    \fi
  \BIC@Fi
}
%    \end{macrocode}
%    \end{macro}
%    \begin{macro}{\BIC@DoSub}
%    |#1|: carry\\
%    |#2|: reverted result\\
%    |#3#4|: reverted x\\
%    |#5#6|: reverted y
%    \begin{macrocode}
\def\BIC@DoSub#1#2!#3#4!#5#6!{%
  \ifx\\#4\\%
    \BIC@AfterFi{%
      \expandafter\expandafter\expandafter\BIC@SubResult
      \BIC@SubDigit#1#3#5#2%
    }%
  \else
    \BIC@AfterFi{%
      \expandafter\expandafter\expandafter\BIC@DoSub
      \BIC@SubDigit#1#3#5#2!#4!#6!%
    }%
  \BIC@Fi
}
%    \end{macrocode}
%    \end{macro}
%    \begin{macro}{\BIC@SubResult}
%    \begin{macrocode}
\def\BIC@SubResult#1{%
  \ifx#10%
    \expandafter\BIC@SubResult
  \else
    \expandafter\BIC@Space\expandafter#1%
  \fi
}
%    \end{macrocode}
%    \end{macro}
%    \begin{macro}{\BIC@SubDigit}
%    |#1|: carry\\
%    |#2|: digit of $x$\\
%    |#3|: digit of $y$
%    \begin{macrocode}
\def\BIC@SubDigit#1#2#3{%
  \romannumeral0%
& \expandafter\BIC@@SubDigit\the\numexpr#2-#3-#1!%
$ \expandafter\BIC@@AddDigit\number
$   \csname
$     BIC@SubCarry%
$     \ifcase#1 %
$       #3%
$     \else
$       \ifcase#3 1\or2\or3\or4\or5\or6\or7\or8\or9\or10\fi
$     \fi
$   \endcsname#2!%
}
%    \end{macrocode}
%    \end{macro}
%    \begin{macro}{\BIC@@SubDigit}
%    \begin{macrocode}
& \def\BIC@@SubDigit#1!{%
&   \ifnum#1<0 %
&     \BIC@AfterFi{%
&       \expandafter\BIC@Space
&       \expandafter1\the\numexpr#1+10\relax
&     }%
&   \else
&     \BIC@AfterFi{ 0#1}%
&   \BIC@Fi
& }%
%    \end{macrocode}
%    \end{macro}
%    \begin{macro}{\BIC@SubCarry0}
%    \begin{macrocode}
$ \expandafter\def\csname BIC@SubCarry0\endcsname#1{#1}%
%    \end{macrocode}
%    \end{macro}
%    \begin{macro}{\BIC@SubCarry10}%
%    \begin{macrocode}
$ \expandafter\def\csname BIC@SubCarry10\endcsname#1{1#1}%
%    \end{macrocode}
%    \end{macro}
%    \begin{macro}{\BIC@SubCarry[1-9]}
%    \begin{macrocode}
$ \def\BIC@Temp#1#2{%
$   \expandafter\def\csname BIC@SubCarry#1\endcsname##1{%
$     \ifcase##1 #2%
$?    \else\BigIntCalcError:ThisCannotHappen%
$     \fi
$   }%
$ }%
$ \BIC@Temp 1{19\or0\or1\or2\or3\or4\or5\or6\or7\or8}%
$ \BIC@Temp 2{18\or19\or0\or1\or2\or3\or4\or5\or6\or7}%
$ \BIC@Temp 3{17\or18\or19\or0\or1\or2\or3\or4\or5\or6}%
$ \BIC@Temp 4{16\or17\or18\or19\or0\or1\or2\or3\or4\or5}%
$ \BIC@Temp 5{15\or16\or17\or18\or19\or0\or1\or2\or3\or4}%
$ \BIC@Temp 6{14\or15\or16\or17\or18\or19\or0\or1\or2\or3}%
$ \BIC@Temp 7{13\or14\or15\or16\or17\or18\or19\or0\or1\or2}%
$ \BIC@Temp 8{12\or13\or14\or15\or16\or17\or18\or19\or0\or1}%
$ \BIC@Temp 9{11\or12\or13\or14\or15\or16\or17\or18\or19\or0}%
%    \end{macrocode}
%    \end{macro}
%
% \subsection{\op{Shl}, \op{Shr}}
%
%    \begin{macro}{\bigintcalcShl}
%    \begin{macrocode}
\def\bigintcalcShl#1{%
  \romannumeral0%
  \expandafter\expandafter\expandafter\BIC@Shl
  \bigintcalcNum{#1}!%
}
%    \end{macrocode}
%    \end{macro}
%    \begin{macro}{\BIC@Shl}
%    \begin{macrocode}
\def\BIC@Shl#1#2!{%
  \ifx#1-%
    \BIC@AfterFi{%
      \expandafter-\romannumeral0%
&     \BIC@@Shl#2!!%
$     \BIC@AddXY#2!#2!!!%
    }%
  \else
    \BIC@AfterFi{%
&     \BIC@@Shl#1#2!!%
$     \BIC@AddXY#1#2!#1#2!!!%
    }%
  \BIC@Fi
}
%    \end{macrocode}
%    \end{macro}
%    \begin{macro}{\BigIntCalcShl}
%    \begin{macrocode}
\def\BigIntCalcShl#1!{%
  \romannumeral0%
& \BIC@@Shl#1!!%
$ \BIC@AddXY#1!#1!!!%
}
%    \end{macrocode}
%    \end{macro}
%    \begin{macro}{\BIC@@Shl}
%    \begin{macrocode}
& \def\BIC@@Shl#1#2!{%
&   \ifx\\#2\\%
&     \BIC@AfterFi{%
&       \BIC@@@Shl0!#1%
&     }%
&   \else
&     \BIC@AfterFi{%
&       \BIC@@Shl#2!#1%
&     }%
&   \BIC@Fi
& }%
%    \end{macrocode}
%    \end{macro}
%    \begin{macro}{\BIC@@@Shl}
%    |#1|: carry\\
%    |#2|: result\\
%    |#3#4|: reverted number
%    \begin{macrocode}
& \def\BIC@@@Shl#1#2!#3#4!{%
&   \ifx\\#4\\%
&     \BIC@AfterFi{%
&       \expandafter\BIC@Space
&       \the\numexpr#3*2+#1\relax#2%
&     }%
&   \else
&     \BIC@AfterFi{%
&       \expandafter\BIC@@@@Shl\the\numexpr#3*2+#1!#2!#4!%
&     }%
&   \BIC@Fi
& }%
%    \end{macrocode}
%    \end{macro}
%    \begin{macro}{\BIC@@@@Shl}
%    \begin{macrocode}
& \def\BIC@@@@Shl#1!{%
&   \ifnum#1<10 %
&     \BIC@AfterFi{%
&       \BIC@@@Shl0#1%
&     }%
&   \else
&     \BIC@AfterFi{%
&       \BIC@@@Shl#1%
&     }%
&   \BIC@Fi
& }%
%    \end{macrocode}
%    \end{macro}
%
%    \begin{macro}{\bigintcalcShr}
%    \begin{macrocode}
\def\bigintcalcShr#1{%
  \romannumeral0%
  \expandafter\expandafter\expandafter\BIC@Shr
  \bigintcalcNum{#1}!%
}
%    \end{macrocode}
%    \end{macro}
%    \begin{macro}{\BIC@Shr}
%    \begin{macrocode}
\def\BIC@Shr#1#2!{%
  \ifx#1-%
    \expandafter-\romannumeral0%
    \BIC@AfterFi{%
      \BIC@@Shr#2!%
    }%
  \else
    \BIC@AfterFi{%
      \BIC@@Shr#1#2!%
    }%
  \BIC@Fi
}
%    \end{macrocode}
%    \end{macro}
%    \begin{macro}{\BigIntCalcShr}
%    \begin{macrocode}
\def\BigIntCalcShr#1!{%
  \romannumeral0%
  \BIC@@Shr#1!%
}
%    \end{macrocode}
%    \end{macro}
%    \begin{macro}{\BIC@@Shr}
%    \begin{macrocode}
\def\BIC@@Shr#1#2!{%
  \ifcase#1 %
    \BIC@AfterFi{ 0}%
  \or
    \ifx\\#2\\%
      \BIC@AfterFiFi{ 0}%
    \else
      \BIC@AfterFiFi{%
        \BIC@@@Shr#1#2!!%
      }%
    \fi
  \else
    \BIC@AfterFi{%
      \BIC@@@Shr0#1#2!!%
    }%
  \BIC@Fi
}
%    \end{macrocode}
%    \end{macro}
%    \begin{macro}{\BIC@@@Shr}
%    |#1|: carry\\
%    |#2#3|: number\\
%    |#4|: result
%    \begin{macrocode}
\def\BIC@@@Shr#1#2#3!#4!{%
  \ifx\\#3\\%
    \ifodd#1#2 %
      \BIC@AfterFiFi{%
&       \expandafter\BIC@ShrResult\the\numexpr(#1#2-1)/2\relax
$       \expandafter\expandafter\expandafter\BIC@ShrResult
$       \csname BIC@ShrDigit#1#2\endcsname
        #4!%
      }%
    \else
      \BIC@AfterFiFi{%
&       \expandafter\BIC@ShrResult\the\numexpr#1#2/2\relax
$       \expandafter\expandafter\expandafter\BIC@ShrResult
$       \csname BIC@ShrDigit#1#2\endcsname
        #4!%
      }%
    \fi
  \else
    \ifodd#1#2 %
      \BIC@AfterFiFi{%
&       \expandafter\BIC@@@@Shr\the\numexpr(#1#2-1)/2\relax1%
$       \expandafter\expandafter\expandafter\BIC@@@@Shr
$       \csname BIC@ShrDigit#1#2\endcsname
        #3!#4!%
      }%
    \else
      \BIC@AfterFiFi{%
&       \expandafter\BIC@@@@Shr\the\numexpr#1#2/2\relax0%
$       \expandafter\expandafter\expandafter\BIC@@@@Shr
$       \csname BIC@ShrDigit#1#2\endcsname
        #3!#4!%
      }%
    \fi
  \BIC@Fi
}
%    \end{macrocode}
%    \end{macro}
%    \begin{macro}{\BIC@ShrResult}
%    \begin{macrocode}
& \def\BIC@ShrResult#1#2!{ #2#1}%
$ \def\BIC@ShrResult#1#2#3!{ #3#1}%
%    \end{macrocode}
%    \end{macro}
%    \begin{macro}{\BIC@@@@Shr}
%    |#1|: new digit\\
%    |#2|: carry\\
%    |#3|: remaining number\\
%    |#4|: result
%    \begin{macrocode}
\def\BIC@@@@Shr#1#2#3!#4!{%
  \BIC@@@Shr#2#3!#4#1!%
}
%    \end{macrocode}
%    \end{macro}
%    \begin{macro}{\BIC@ShrDigit[00-19]}
%    \begin{macrocode}
$ \def\BIC@Temp#1#2#3#4{%
$   \expandafter\def\csname BIC@ShrDigit#1#2\endcsname{#3#4}%
$ }%
$ \BIC@Temp 0000%
$ \BIC@Temp 0101%
$ \BIC@Temp 0210%
$ \BIC@Temp 0311%
$ \BIC@Temp 0420%
$ \BIC@Temp 0521%
$ \BIC@Temp 0630%
$ \BIC@Temp 0731%
$ \BIC@Temp 0840%
$ \BIC@Temp 0941%
$ \BIC@Temp 1050%
$ \BIC@Temp 1151%
$ \BIC@Temp 1260%
$ \BIC@Temp 1361%
$ \BIC@Temp 1470%
$ \BIC@Temp 1571%
$ \BIC@Temp 1680%
$ \BIC@Temp 1781%
$ \BIC@Temp 1890%
$ \BIC@Temp 1991%
%    \end{macrocode}
%    \end{macro}
%
% \subsection{\cs{BIC@Tim}}
%
%    \begin{macro}{\BIC@Tim}
%    Macro \cs{BIC@Tim} implements ``Number \emph{tim}es digit''.\\
%    |#1|: plain number without sign\\
%    |#2|: digit
\def\BIC@Tim#1!#2{%
  \romannumeral0%
  \ifcase#2 % 0
    \BIC@AfterFi{ 0}%
  \or % 1
    \BIC@AfterFi{ #1}%
  \or % 2
    \BIC@AfterFi{%
      \BIC@Shl#1!%
    }%
  \else % 3-9
    \BIC@AfterFi{%
      \BIC@@Tim#1!!#2%
    }%
  \BIC@Fi
}
%    \begin{macrocode}
%    \end{macrocode}
%    \end{macro}
%    \begin{macro}{\BIC@@Tim}
%    |#1#2|: number\\
%    |#3|: reverted number
%    \begin{macrocode}
\def\BIC@@Tim#1#2!{%
  \ifx\\#2\\%
    \BIC@AfterFi{%
      \BIC@ProcessTim0!#1%
    }%
  \else
    \BIC@AfterFi{%
      \BIC@@Tim#2!#1%
    }%
  \BIC@Fi
}
%    \end{macrocode}
%    \end{macro}
%    \begin{macro}{\BIC@ProcessTim}
%    |#1|: carry\\
%    |#2|: result\\
%    |#3#4|: reverted number\\
%    |#5|: digit
%    \begin{macrocode}
\def\BIC@ProcessTim#1#2!#3#4!#5{%
  \ifx\\#4\\%
    \BIC@AfterFi{%
      \expandafter\BIC@Space
&     \the\numexpr#3*#5+#1\relax
$     \romannumeral0\BIC@TimDigit#3#5#1%
      #2%
    }%
  \else
    \BIC@AfterFi{%
      \expandafter\BIC@@ProcessTim
&     \the\numexpr#3*#5+#1%
$     \romannumeral0\BIC@TimDigit#3#5#1%
      !#2!#4!#5%
    }%
  \BIC@Fi
}
%    \end{macrocode}
%    \end{macro}
%    \begin{macro}{\BIC@@ProcessTim}
%    |#1#2|: carry?, new digit\\
%    |#3|: new number\\
%    |#4|: old number\\
%    |#5|: digit
%    \begin{macrocode}
\def\BIC@@ProcessTim#1#2!{%
  \ifx\\#2\\%
    \BIC@AfterFi{%
      \BIC@ProcessTim0#1%
    }%
  \else
    \BIC@AfterFi{%
      \BIC@ProcessTim#1#2%
    }%
  \BIC@Fi
}
%    \end{macrocode}
%    \end{macro}
%    \begin{macro}{\BIC@TimDigit}
%    |#1|: digit 0--9\\
%    |#2|: digit 3--9\\
%    |#3|: carry 0--9
%    \begin{macrocode}
$ \def\BIC@TimDigit#1#2#3{%
$   \ifcase#1 % 0
$     \BIC@AfterFi{ #3}%
$   \or % 1
$     \BIC@AfterFi{%
$       \expandafter\BIC@Space
$       \number\csname BIC@AddCarry#2\endcsname#3 %
$     }%
$   \else
$     \ifcase#3 %
$       \BIC@AfterFiFi{%
$         \expandafter\BIC@Space
$         \number\csname BIC@MulDigit#2\endcsname#1 %
$       }%
$     \else
$       \BIC@AfterFiFi{%
$         \expandafter\BIC@Space
$         \romannumeral0%
$         \expandafter\BIC@AddXY
$         \number\csname BIC@MulDigit#2\endcsname#1!%
$         #3!!!%
$       }%
$     \fi
$   \BIC@Fi
$ }%
%    \end{macrocode}
%    \end{macro}
%    \begin{macro}{\BIC@MulDigit[3-9]}
%    \begin{macrocode}
$ \def\BIC@Temp#1#2{%
$   \expandafter\def\csname BIC@MulDigit#1\endcsname##1{%
$     \ifcase##1 0%
$     \or ##1%
$     \or #2%
$?    \else\BigIntCalcError:ThisCannotHappen%
$     \fi
$   }%
$ }%
$ \BIC@Temp 3{6\or9\or12\or15\or18\or21\or24\or27}%
$ \BIC@Temp 4{8\or12\or16\or20\or24\or28\or32\or36}%
$ \BIC@Temp 5{10\or15\or20\or25\or30\or35\or40\or45}%
$ \BIC@Temp 6{12\or18\or24\or30\or36\or42\or48\or54}%
$ \BIC@Temp 7{14\or21\or28\or35\or42\or49\or56\or63}%
$ \BIC@Temp 8{16\or24\or32\or40\or48\or56\or64\or72}%
$ \BIC@Temp 9{18\or27\or36\or45\or54\or63\or72\or81}%
%    \end{macrocode}
%    \end{macro}
%
% \subsection{\op{Mul}}
%
%    \begin{macro}{\bigintcalcMul}
%    \begin{macrocode}
\def\bigintcalcMul#1#2{%
  \romannumeral0%
  \expandafter\expandafter\expandafter\BIC@Mul
  \bigintcalcNum{#1}!{#2}%
}
%    \end{macrocode}
%    \end{macro}
%    \begin{macro}{\BIC@Mul}
%    \begin{macrocode}
\def\BIC@Mul#1!#2{%
  \expandafter\expandafter\expandafter\BIC@MulSwitch
  \bigintcalcNum{#2}!#1!%
}
%    \end{macrocode}
%    \end{macro}
%    \begin{macro}{\BIC@MulSwitch}
%    Decision table for \cs{BIC@MulSwitch}.
%    \begin{quote}
%    \begin{tabular}[t]{@{}|l|l|l|l|l|@{}}
%      \hline
%      $x=0$ &\multicolumn{3}{l}{}    &$0$\\
%      \hline
%      $x>0$ & $y=0$ &\multicolumn2l{}&$0$\\
%      \cline{2-5}
%            & $y>0$ & $ x> y$ & $+$ & $\opMul( x, y)$\\
%      \cline{3-3}\cline{5-5}
%            &       & else    &     & $\opMul( y, x)$\\
%      \cline{2-5}
%            & $y<0$ & $ x>-y$ & $-$ & $\opMul( x,-y)$\\
%      \cline{3-3}\cline{5-5}
%            &       & else    &     & $\opMul(-y, x)$\\
%      \hline
%      $x<0$ & $y=0$ &\multicolumn2l{}&$0$\\
%      \cline{2-5}
%            & $y>0$ & $-x> y$ & $-$ & $\opMul(-x, y)$\\
%      \cline{3-3}\cline{5-5}
%            &       & else    &     & $\opMul( y,-x)$\\
%      \cline{2-5}
%            & $y<0$ & $-x>-y$ & $+$ & $\opMul(-x,-y)$\\
%      \cline{3-3}\cline{5-5}
%            &       & else    &     & $\opMul(-y,-x)$\\
%      \hline
%    \end{tabular}
%    \end{quote}
%    \begin{macrocode}
\def\BIC@MulSwitch#1#2!#3#4!{%
  \ifcase\BIC@Sgn#1#2! % x = 0
    \BIC@AfterFi{ 0}%
  \or % x > 0
    \ifcase\BIC@Sgn#3#4! % y = 0
      \BIC@AfterFiFi{ 0}%
    \or % y > 0
      \ifnum\BIC@PosCmp#1#2!#3#4!=1 % x > y
        \BIC@AfterFiFiFi{%
          \BIC@ProcessMul0!#1#2!#3#4!%
        }%
      \else % x <= y
        \BIC@AfterFiFiFi{%
          \BIC@ProcessMul0!#3#4!#1#2!%
        }%
      \fi
    \else % y < 0
      \expandafter-\romannumeral0%
      \ifnum\BIC@PosCmp#1#2!#4!=1 % x > -y
        \BIC@AfterFiFiFi{%
          \BIC@ProcessMul0!#1#2!#4!%
        }%
      \else % x <= -y
        \BIC@AfterFiFiFi{%
          \BIC@ProcessMul0!#4!#1#2!%
        }%
      \fi
    \fi
  \else % x < 0
    \ifcase\BIC@Sgn#3#4! % y = 0
      \BIC@AfterFiFi{ 0}%
    \or % y > 0
      \expandafter-\romannumeral0%
      \ifnum\BIC@PosCmp#2!#3#4!=1 % -x > y
        \BIC@AfterFiFiFi{%
          \BIC@ProcessMul0!#2!#3#4!%
        }%
      \else % -x <= y
        \BIC@AfterFiFiFi{%
          \BIC@ProcessMul0!#3#4!#2!%
        }%
      \fi
    \else % y < 0
      \ifnum\BIC@PosCmp#2!#4!=1 % -x > -y
        \BIC@AfterFiFiFi{%
          \BIC@ProcessMul0!#2!#4!%
        }%
      \else % -x <= -y
        \BIC@AfterFiFiFi{%
          \BIC@ProcessMul0!#4!#2!%
        }%
      \fi
    \fi
  \BIC@Fi
}
%    \end{macrocode}
%    \end{macro}
%    \begin{macro}{\BigIntCalcMul}
%    \begin{macrocode}
\def\BigIntCalcMul#1!#2!{%
  \romannumeral0%
  \BIC@ProcessMul0!#1!#2!%
}
%    \end{macrocode}
%    \end{macro}
%    \begin{macro}{\BIC@ProcessMul}
%    |#1|: result\\
%    |#2|: number $x$\\
%    |#3#4|: number $y$
%    \begin{macrocode}
\def\BIC@ProcessMul#1!#2!#3#4!{%
  \ifx\\#4\\%
    \BIC@AfterFi{%
      \expandafter\expandafter\expandafter\BIC@Space
      \bigintcalcAdd{\BIC@Tim#2!#3}{#10}%
    }%
  \else
    \BIC@AfterFi{%
      \expandafter\expandafter\expandafter\BIC@ProcessMul
      \bigintcalcAdd{\BIC@Tim#2!#3}{#10}!#2!#4!%
    }%
  \BIC@Fi
}
%    \end{macrocode}
%    \end{macro}
%
% \subsection{\op{Sqr}}
%
%    \begin{macro}{\bigintcalcSqr}
%    \begin{macrocode}
\def\bigintcalcSqr#1{%
  \romannumeral0%
  \expandafter\expandafter\expandafter\BIC@Sqr
  \bigintcalcNum{#1}!%
}
%    \end{macrocode}
%    \end{macro}
%    \begin{macro}{\BIC@Sqr}
%    \begin{macrocode}
\def\BIC@Sqr#1{%
   \ifx#1-%
     \expandafter\BIC@@Sqr
   \else
     \expandafter\BIC@@Sqr\expandafter#1%
   \fi
}
%    \end{macrocode}
%    \end{macro}
%    \begin{macro}{\BIC@@Sqr}
%    \begin{macrocode}
\def\BIC@@Sqr#1!{%
  \BIC@ProcessMul0!#1!#1!%
}
%    \end{macrocode}
%    \end{macro}
%
% \subsection{\op{Fac}}
%
%    \begin{macro}{\bigintcalcFac}
%    \begin{macrocode}
\def\bigintcalcFac#1{%
  \romannumeral0%
  \expandafter\expandafter\expandafter\BIC@Fac
  \bigintcalcNum{#1}!%
}
%    \end{macrocode}
%    \end{macro}
%    \begin{macro}{\BIC@Fac}
%    \begin{macrocode}
\def\BIC@Fac#1#2!{%
  \ifx#1-%
    \BIC@AfterFi{ 0\BigIntCalcError:FacNegative}%
  \else
    \ifnum\BIC@PosCmp#1#2!13!<0 %
      \ifcase#1#2 %
         \BIC@AfterFiFiFi{ 1}% 0!
      \or\BIC@AfterFiFiFi{ 1}% 1!
      \or\BIC@AfterFiFiFi{ 2}% 2!
      \or\BIC@AfterFiFiFi{ 6}% 3!
      \or\BIC@AfterFiFiFi{ 24}% 4!
      \or\BIC@AfterFiFiFi{ 120}% 5!
      \or\BIC@AfterFiFiFi{ 720}% 6!
      \or\BIC@AfterFiFiFi{ 5040}% 7!
      \or\BIC@AfterFiFiFi{ 40320}% 8!
      \or\BIC@AfterFiFiFi{ 362880}% 9!
      \or\BIC@AfterFiFiFi{ 3628800}% 10!
      \or\BIC@AfterFiFiFi{ 39916800}% 11!
      \or\BIC@AfterFiFiFi{ 479001600}% 12!
?     \else\BigIntCalcError:ThisCannotHappen%
      \fi
    \else
      \BIC@AfterFiFi{%
        \BIC@ProcessFac#1#2!479001600!%
      }%
    \fi
  \BIC@Fi
}
%    \end{macrocode}
%    \end{macro}
%    \begin{macro}{\BIC@ProcessFac}
%    |#1|: $n$\\
%    |#2|: result
%    \begin{macrocode}
\def\BIC@ProcessFac#1!#2!{%
  \ifnum\BIC@PosCmp#1!12!=0 %
    \BIC@AfterFi{ #2}%
  \else
    \BIC@AfterFi{%
      \expandafter\BIC@@ProcessFac
      \romannumeral0\BIC@ProcessMul0!#2!#1!%
      !#1!%
    }%
  \BIC@Fi
}
%    \end{macrocode}
%    \end{macro}
%    \begin{macro}{\BIC@@ProcessFac}
%    |#1|: result\\
%    |#2|: $n$
%    \begin{macrocode}
\def\BIC@@ProcessFac#1!#2!{%
  \expandafter\BIC@ProcessFac
  \romannumeral0\BIC@Dec#2!{}%
  !#1!%
}
%    \end{macrocode}
%    \end{macro}
%
% \subsection{\op{Pow}}
%
%    \begin{macro}{\bigintcalcPow}
%    |#1|: basis\\
%    |#2|: power
%    \begin{macrocode}
\def\bigintcalcPow#1{%
  \romannumeral0%
  \expandafter\expandafter\expandafter\BIC@Pow
  \bigintcalcNum{#1}!%
}
%    \end{macrocode}
%    \end{macro}
%    \begin{macro}{\BIC@Pow}
%    |#1|: basis\\
%    |#2|: power
%    \begin{macrocode}
\def\BIC@Pow#1!#2{%
  \expandafter\expandafter\expandafter\BIC@PowSwitch
  \bigintcalcNum{#2}!#1!%
}
%    \end{macrocode}
%    \end{macro}
%    \begin{macro}{\BIC@PowSwitch}
%    |#1#2|: power $y$\\
%    |#3#4|: basis $x$\\
%    Decision table for \cs{BIC@PowSwitch}.
%    \begin{quote}
%    \def\M#1{\multicolumn{#1}{l}{}}%
%    \begin{tabular}[t]{@{}|l|l|l|l|@{}}
%    \hline
%    $y=0$ & \M{2}                        & $1$\\
%    \hline
%    $y=1$ & \M{2}                        & $x$\\
%    \hline
%    $y=2$ & $x<0$          & \M{1}       & $\opMul(-x,-x)$\\
%    \cline{2-4}
%          & else           & \M{1}       & $\opMul(x,x)$\\
%    \hline
%    $y<0$ & $x=0$          & \M{1}       & DivisionByZero\\
%    \cline{2-4}
%          & $x=1$          & \M{1}       & $1$\\
%    \cline{2-4}
%          & $x=-1$         & $\opODD(y)$ & $-1$\\
%    \cline{3-4}
%          &                & else        & $1$\\
%    \cline{2-4}
%          & else ($\string|x\string|>1$) & \M{1}       & $0$\\
%    \hline
%    $y>2$ & $x=0$          & \M{1}       & $0$\\
%    \cline{2-4}
%          & $x=1$          & \M{1}       & $1$\\
%    \cline{2-4}
%          & $x=-1$         & $\opODD(y)$ & $-1$\\
%    \cline{3-4}
%          &                & else        & $1$\\
%    \cline{2-4}
%          & $x<-1$ ($x<0$) & $\opODD(y)$ & $-\opPow(-x,y)$\\
%    \cline{3-4}
%          &                & else        & $\opPow(-x,y)$\\
%    \cline{2-4}
%          & else ($x>1$)   & \M{1}       & $\opPow(x,y)$\\
%    \hline
%    \end{tabular}
%    \end{quote}
%    \begin{macrocode}
\def\BIC@PowSwitch#1#2!#3#4!{%
  \ifcase\ifx\\#2\\%
           \ifx#100 % y = 0
           \else\ifx#111 % y = 1
           \else\ifx#122 % y = 2
           \else4 % y > 2
           \fi\fi\fi
         \else
           \ifx#1-3 % y < 0
           \else4 % y > 2
           \fi
         \fi
    \BIC@AfterFi{ 1}% y = 0
  \or % y = 1
    \BIC@AfterFi{ #3#4}%
  \or % y = 2
    \ifx#3-% x < 0
      \BIC@AfterFiFi{%
        \BIC@ProcessMul0!#4!#4!%
      }%
    \else % x >= 0
      \BIC@AfterFiFi{%
        \BIC@ProcessMul0!#3#4!#3#4!%
      }%
    \fi
  \or % y < 0
    \ifcase\ifx\\#4\\%
             \ifx#300 % x = 0
             \else\ifx#311 % x = 1
             \else3 % x > 1
             \fi\fi
           \else
             \ifcase\BIC@MinusOne#3#4! %
               3 % |x| > 1
             \or
               2 % x = -1
?            \else\BigIntCalcError:ThisCannotHappen%
             \fi
           \fi
      \BIC@AfterFiFi{ 0\BigIntCalcError:DivisionByZero}% x = 0
    \or % x = 1
      \BIC@AfterFiFi{ 1}% x = 1
    \or % x = -1
      \ifcase\BIC@ModTwo#2! % even(y)
        \BIC@AfterFiFiFi{ 1}%
      \or % odd(y)
        \BIC@AfterFiFiFi{ -1}%
?     \else\BigIntCalcError:ThisCannotHappen%
      \fi
    \or % |x| > 1
      \BIC@AfterFiFi{ 0}%
?   \else\BigIntCalcError:ThisCannotHappen%
    \fi
  \or % y > 2
    \ifcase\ifx\\#4\\%
             \ifx#300 % x = 0
             \else\ifx#311 % x = 1
             \else4 % x > 1
             \fi\fi
           \else
             \ifx#3-%
               \ifcase\BIC@MinusOne#3#4! %
                 3 % x < -1
               \else
                 2 % x = -1
               \fi
             \else
               4 % x > 1
             \fi
           \fi
      \BIC@AfterFiFi{ 0}% x = 0
    \or % x = 1
      \BIC@AfterFiFi{ 1}% x = 1
    \or % x = -1
      \ifcase\BIC@ModTwo#1#2! % even(y)
        \BIC@AfterFiFiFi{ 1}%
      \or % odd(y)
        \BIC@AfterFiFiFi{ -1}%
?     \else\BigIntCalcError:ThisCannotHappen%
      \fi
    \or % x < -1
      \ifcase\BIC@ModTwo#1#2! % even(y)
        \BIC@AfterFiFiFi{%
          \BIC@PowRec#4!#1#2!1!%
        }%
      \or % odd(y)
        \expandafter-\romannumeral0%
        \BIC@AfterFiFiFi{%
          \BIC@PowRec#4!#1#2!1!%
        }%
?     \else\BigIntCalcError:ThisCannotHappen%
      \fi
    \or % x > 1
      \BIC@AfterFiFi{%
        \BIC@PowRec#3#4!#1#2!1!%
      }%
?   \else\BigIntCalcError:ThisCannotHappen%
    \fi
? \else\BigIntCalcError:ThisCannotHappen%
  \BIC@Fi
}
%    \end{macrocode}
%    \end{macro}
%
% \subsubsection{Help macros}
%
%    \begin{macro}{\BIC@ModTwo}
%    Macro \cs{BIC@ModTwo} expects a number without sign
%    and returns digit |1| or |0| if the number is odd or even.
%    \begin{macrocode}
\def\BIC@ModTwo#1#2!{%
  \ifx\\#2\\%
    \ifodd#1 %
      \BIC@AfterFiFi1%
    \else
      \BIC@AfterFiFi0%
    \fi
  \else
    \BIC@AfterFi{%
      \BIC@ModTwo#2!%
    }%
  \BIC@Fi
}
%    \end{macrocode}
%    \end{macro}
%
%    \begin{macro}{\BIC@MinusOne}
%    Macro \cs{BIC@MinusOne} expects a number and returns
%    digit |1| if the number equals minus one and returns |0| otherwise.
%    \begin{macrocode}
\def\BIC@MinusOne#1#2!{%
  \ifx#1-%
    \BIC@@MinusOne#2!%
  \else
    0%
  \fi
}
%    \end{macrocode}
%    \end{macro}
%    \begin{macro}{\BIC@@MinusOne}
%    \begin{macrocode}
\def\BIC@@MinusOne#1#2!{%
  \ifx#11%
    \ifx\\#2\\%
      1%
    \else
      0%
    \fi
  \else
    0%
  \fi
}
%    \end{macrocode}
%    \end{macro}
%
% \subsubsection{Recursive calculation}
%
%    \begin{macro}{\BIC@PowRec}
%\begin{quote}
%\begin{verbatim}
%Pow(x, y) {
%  PowRec(x, y, 1)
%}
%PowRec(x, y, r) {
%  if y == 1 then
%    return r
%  else
%    ifodd y then
%      return PowRec(x*x, y div 2, r*x) % y div 2 = (y-1)/2
%    else
%      return PowRec(x*x, y div 2, r)
%    fi
%  fi
%}
%\end{verbatim}
%\end{quote}
%    |#1|: $x$ (basis)\\
%    |#2#3|: $y$ (power)\\
%    |#4|: $r$ (result)
%    \begin{macrocode}
\def\BIC@PowRec#1!#2#3!#4!{%
  \ifcase\ifx#21\ifx\\#3\\0 \else1 \fi\else1 \fi % y = 1
    \ifnum\BIC@PosCmp#1!#4!=1 % x > r
      \BIC@AfterFiFi{%
        \BIC@ProcessMul0!#1!#4!%
      }%
    \else
      \BIC@AfterFiFi{%
        \BIC@ProcessMul0!#4!#1!%
      }%
    \fi
  \or
    \ifcase\BIC@ModTwo#2#3! % even(y)
      \BIC@AfterFiFi{%
        \expandafter\BIC@@PowRec\romannumeral0%
        \BIC@@Shr#2#3!%
        !#1!#4!%
      }%
    \or % odd(y)
      \ifnum\BIC@PosCmp#1!#4!=1 % x > r
        \BIC@AfterFiFiFi{%
          \expandafter\BIC@@@PowRec\romannumeral0%
          \BIC@ProcessMul0!#1!#4!%
          !#1!#2#3!%
        }%
      \else
        \BIC@AfterFiFiFi{%
          \expandafter\BIC@@@PowRec\romannumeral0%
          \BIC@ProcessMul0!#1!#4!%
          !#1!#2#3!%
        }%
      \fi
?   \else\BigIntCalcError:ThisCannotHappen%
    \fi
? \else\BigIntCalcError:ThisCannotHappen%
  \BIC@Fi
}
%    \end{macrocode}
%    \end{macro}
%    \begin{macro}{\BIC@@PowRec}
%    |#1|: $y/2$\\
%    |#2|: $x$\\
%    |#3|: new $r$ ($r$ or $r*x$)
%    \begin{macrocode}
\def\BIC@@PowRec#1!#2!#3!{%
  \expandafter\BIC@PowRec\romannumeral0%
  \BIC@ProcessMul0!#2!#2!%
  !#1!#3!%
}
%    \end{macrocode}
%    \end{macro}
%    \begin{macro}{\BIC@@@PowRec}
%    |#1|: $r*x$
%    |#2|: $x$
%    |#3|: $y$
%    \begin{macrocode}
\def\BIC@@@PowRec#1!#2!#3!{%
  \expandafter\BIC@@PowRec\romannumeral0%
  \BIC@@Shr#3!%
  !#2!#1!%
}
%    \end{macrocode}
%    \end{macro}
%
% \subsection{\op{Div}}
%
%    \begin{macro}{\bigintcalcDiv}
%    |#1|: $x$\\
%    |#2|: $y$ (divisor)
%    \begin{macrocode}
\def\bigintcalcDiv#1{%
  \romannumeral0%
  \expandafter\expandafter\expandafter\BIC@Div
  \bigintcalcNum{#1}!%
}
%    \end{macrocode}
%    \end{macro}
%    \begin{macro}{\BIC@Div}
%    |#1|: $x$\\
%    |#2|: $y$
%    \begin{macrocode}
\def\BIC@Div#1!#2{%
  \expandafter\expandafter\expandafter\BIC@DivSwitchSign
  \bigintcalcNum{#2}!#1!%
}
%    \end{macrocode}
%    \end{macro}
%    \begin{macro}{\BigIntCalcDiv}
%    \begin{macrocode}
\def\BigIntCalcDiv#1!#2!{%
  \romannumeral0%
  \BIC@DivSwitchSign#2!#1!%
}
%    \end{macrocode}
%    \end{macro}
%    \begin{macro}{\BIC@DivSwitchSign}
%    Decision table for \cs{BIC@DivSwitchSign}.
%    \begin{quote}
%    \begin{tabular}{@{}|l|l|l|@{}}
%    \hline
%    $y=0$ & \multicolumn{1}{l}{} & DivisionByZero\\
%    \hline
%    $y>0$ & $x=0$ & $0$\\
%    \cline{2-3}
%          & $x>0$ & DivSwitch$(+,x,y)$\\
%    \cline{2-3}
%          & $x<0$ & DivSwitch$(-,-x,y)$\\
%    \hline
%    $y<0$ & $x=0$ & $0$\\
%    \cline{2-3}
%          & $x>0$ & DivSwitch$(-,x,-y)$\\
%    \cline{2-3}
%          & $x<0$ & DivSwitch$(+,-x,-y)$\\
%    \hline
%    \end{tabular}
%    \end{quote}
%    |#1|: $y$ (divisor)\\
%    |#2|: $x$
%    \begin{macrocode}
\def\BIC@DivSwitchSign#1#2!#3#4!{%
  \ifcase\BIC@Sgn#1#2! % y = 0
    \BIC@AfterFi{ 0\BigIntCalcError:DivisionByZero}%
  \or % y > 0
    \ifcase\BIC@Sgn#3#4! % x = 0
      \BIC@AfterFiFi{ 0}%
    \or % x > 0
      \BIC@AfterFiFi{%
        \BIC@DivSwitch{}#3#4!#1#2!%
      }%
    \else % x < 0
      \BIC@AfterFiFi{%
        \BIC@DivSwitch-#4!#1#2!%
      }%
    \fi
  \else % y < 0
    \ifcase\BIC@Sgn#3#4! % x = 0
      \BIC@AfterFiFi{ 0}%
    \or % x > 0
      \BIC@AfterFiFi{%
        \BIC@DivSwitch-#3#4!#2!%
      }%
    \else % x < 0
      \BIC@AfterFiFi{%
        \BIC@DivSwitch{}#4!#2!%
      }%
    \fi
  \BIC@Fi
}
%    \end{macrocode}
%    \end{macro}
%    \begin{macro}{\BIC@DivSwitch}
%    Decision table for \cs{BIC@DivSwitch}.
%    \begin{quote}
%    \begin{tabular}{@{}|l|l|l|@{}}
%    \hline
%    $y=x$ & \multicolumn{1}{l}{} & sign $1$\\
%    \hline
%    $y>x$ & \multicolumn{1}{l}{} & $0$\\
%    \hline
%    $y<x$ & $y=1$                & sign $x$\\
%    \cline{2-3}
%          & $y=2$                & sign Shr$(x)$\\
%    \cline{2-3}
%          & $y=4$                & sign Shr(Shr$(x)$)\\
%    \cline{2-3}
%          & else                 & sign ProcessDiv$(x,y)$\\
%    \hline
%    \end{tabular}
%    \end{quote}
%    |#1|: sign\\
%    |#2|: $x$\\
%    |#3#4|: $y$ ($y\ne 0$)
%    \begin{macrocode}
\def\BIC@DivSwitch#1#2!#3#4!{%
  \ifcase\BIC@PosCmp#3#4!#2!% y = x
    \BIC@AfterFi{ #11}%
  \or % y > x
    \BIC@AfterFi{ 0}%
  \else % y < x
    \ifx\\#1\\%
    \else
      \expandafter-\romannumeral0%
    \fi
    \ifcase\ifx\\#4\\%
             \ifx#310 % y = 1
             \else\ifx#321 % y = 2
             \else\ifx#342 % y = 4
             \else3 % y > 2
             \fi\fi\fi
           \else
             3 % y > 2
           \fi
      \BIC@AfterFiFi{ #2}% y = 1
    \or % y = 2
      \BIC@AfterFiFi{%
        \BIC@@Shr#2!%
      }%
    \or % y = 4
      \BIC@AfterFiFi{%
        \expandafter\BIC@@Shr\romannumeral0%
          \BIC@@Shr#2!!%
      }%
    \or % y > 2
      \BIC@AfterFiFi{%
        \BIC@DivStartX#2!#3#4!!!%
      }%
?   \else\BigIntCalcError:ThisCannotHappen%
    \fi
  \BIC@Fi
}
%    \end{macrocode}
%    \end{macro}
%    \begin{macro}{\BIC@ProcessDiv}
%    |#1#2|: $x$\\
%    |#3#4|: $y$\\
%    |#5|: collect first digits of $x$\\
%    |#6|: corresponding digits of $y$
%    \begin{macrocode}
\def\BIC@DivStartX#1#2!#3#4!#5!#6!{%
  \ifx\\#4\\%
    \BIC@AfterFi{%
      \BIC@DivStartYii#6#3#4!{#5#1}#2=!%
    }%
  \else
    \BIC@AfterFi{%
      \BIC@DivStartX#2!#4!#5#1!#6#3!%
    }%
  \BIC@Fi
}
%    \end{macrocode}
%    \end{macro}
%    \begin{macro}{\BIC@DivStartYii}
%    |#1|: $y$\\
%    |#2|: $x$, |=|
%    \begin{macrocode}
\def\BIC@DivStartYii#1!{%
  \expandafter\BIC@DivStartYiv\romannumeral0%
  \BIC@Shl#1!%
  !#1!%
}
%    \end{macrocode}
%    \end{macro}
%    \begin{macro}{\BIC@DivStartYiv}
%    |#1|: $2y$\\
%    |#2|: $y$\\
%    |#3|: $x$, |=|
%    \begin{macrocode}
\def\BIC@DivStartYiv#1!{%
  \expandafter\BIC@DivStartYvi\romannumeral0%
  \BIC@Shl#1!%
  !#1!%
}
%    \end{macrocode}
%    \end{macro}
%    \begin{macro}{\BIC@DivStartYvi}
%    |#1|: $4y$\\
%    |#2|: $2y$\\
%    |#3|: $y$\\
%    |#4|: $x$, |=|
%    \begin{macrocode}
\def\BIC@DivStartYvi#1!#2!{%
  \expandafter\BIC@DivStartYviii\romannumeral0%
  \BIC@AddXY#1!#2!!!%
  !#1!#2!%
}
%    \end{macrocode}
%    \end{macro}
%    \begin{macro}{\BIC@DivStartYviii}
%    |#1|: $6y$\\
%    |#2|: $4y$\\
%    |#3|: $2y$\\
%    |#4|: $y$\\
%    |#5|: $x$, |=|
%    \begin{macrocode}
\def\BIC@DivStartYviii#1!#2!{%
  \expandafter\BIC@DivStart\romannumeral0%
  \BIC@Shl#2!%
  !#1!#2!%
}
%    \end{macrocode}
%    \end{macro}
%    \begin{macro}{\BIC@DivStart}
%    |#1|: $8y$\\
%    |#2|: $6y$\\
%    |#3|: $4y$\\
%    |#4|: $2y$\\
%    |#5|: $y$\\
%    |#6|: $x$, |=|
%    \begin{macrocode}
\def\BIC@DivStart#1!#2!#3!#4!#5!#6!{%
  \BIC@ProcessDiv#6!!#5!#4!#3!#2!#1!=%
}
%    \end{macrocode}
%    \end{macro}
%    \begin{macro}{\BIC@ProcessDiv}
%    |#1#2#3|: $x$, |=|\\
%    |#4|: result\\
%    |#5|: $y$\\
%    |#6|: $2y$\\
%    |#7|: $4y$\\
%    |#8|: $6y$\\
%    |#9|: $8y$
%    \begin{macrocode}
\def\BIC@ProcessDiv#1#2#3!#4!#5!{%
  \ifcase\BIC@PosCmp#5!#1!% y = #1
    \ifx#2=%
      \BIC@AfterFiFi{\BIC@DivCleanup{#41}}%
    \else
      \BIC@AfterFiFi{%
        \BIC@ProcessDiv#2#3!#41!#5!%
      }%
    \fi
  \or % y > #1
    \ifx#2=%
      \BIC@AfterFiFi{\BIC@DivCleanup{#40}}%
    \else
      \ifx\\#4\\%
        \BIC@AfterFiFiFi{%
          \BIC@ProcessDiv{#1#2}#3!!#5!%
        }%
      \else
        \BIC@AfterFiFiFi{%
          \BIC@ProcessDiv{#1#2}#3!#40!#5!%
        }%
      \fi
    \fi
  \else % y < #1
    \BIC@AfterFi{%
      \BIC@@ProcessDiv{#1}#2#3!#4!#5!%
    }%
  \BIC@Fi
}
%    \end{macrocode}
%    \end{macro}
%    \begin{macro}{\BIC@DivCleanup}
%    |#1|: result\\
%    |#2|: garbage
%    \begin{macrocode}
\def\BIC@DivCleanup#1#2={ #1}%
%    \end{macrocode}
%    \end{macro}
%    \begin{macro}{\BIC@@ProcessDiv}
%    \begin{macrocode}
\def\BIC@@ProcessDiv#1#2#3!#4!#5!#6!#7!{%
  \ifcase\BIC@PosCmp#7!#1!% 4y = #1
    \ifx#2=%
      \BIC@AfterFiFi{\BIC@DivCleanup{#44}}%
    \else
     \BIC@AfterFiFi{%
       \BIC@ProcessDiv#2#3!#44!#5!#6!#7!%
     }%
    \fi
  \or % 4y > #1
    \ifcase\BIC@PosCmp#6!#1!% 2y = #1
      \ifx#2=%
        \BIC@AfterFiFiFi{\BIC@DivCleanup{#42}}%
      \else
        \BIC@AfterFiFiFi{%
          \BIC@ProcessDiv#2#3!#42!#5!#6!#7!%
        }%
      \fi
    \or % 2y > #1
      \ifx#2=%
        \BIC@AfterFiFiFi{\BIC@DivCleanup{#41}}%
      \else
        \BIC@AfterFiFiFi{%
          \BIC@DivSub#1!#5!#2#3!#41!#5!#6!#7!%
        }%
      \fi
    \else % 2y < #1
      \BIC@AfterFiFi{%
        \expandafter\BIC@ProcessDivII\romannumeral0%
        \BIC@SubXY#1!#6!!!%
        !#2#3!#4!#5!23%
        #6!#7!%
      }%
    \fi
  \else % 4y < #1
    \BIC@AfterFi{%
      \BIC@@@ProcessDiv{#1}#2#3!#4!#5!#6!#7!%
    }%
  \BIC@Fi
}
%    \end{macrocode}
%    \end{macro}
%    \begin{macro}{\BIC@DivSub}
%    Next token group: |#1|-|#2| and next digit |#3|.
%    \begin{macrocode}
\def\BIC@DivSub#1!#2!#3{%
  \expandafter\BIC@ProcessDiv\expandafter{%
    \romannumeral0%
    \BIC@SubXY#1!#2!!!%
    #3%
  }%
}
%    \end{macrocode}
%    \end{macro}
%    \begin{macro}{\BIC@ProcessDivII}
%    |#1|: $x'-2y$\\
%    |#2#3|: remaining $x$, |=|\\
%    |#4|: result\\
%    |#5|: $y$\\
%    |#6|: first possible result digit\\
%    |#7|: second possible result digit
%    \begin{macrocode}
\def\BIC@ProcessDivII#1!#2#3!#4!#5!#6#7{%
  \ifcase\BIC@PosCmp#5!#1!% y = #1
    \ifx#2=%
      \BIC@AfterFiFi{\BIC@DivCleanup{#4#7}}%
    \else
      \BIC@AfterFiFi{%
        \BIC@ProcessDiv#2#3!#4#7!#5!%
      }%
    \fi
  \or % y > #1
    \ifx#2=%
      \BIC@AfterFiFi{\BIC@DivCleanup{#4#6}}%
    \else
      \BIC@AfterFiFi{%
        \BIC@ProcessDiv{#1#2}#3!#4#6!#5!%
      }%
    \fi
  \else % y < #1
    \ifx#2=%
      \BIC@AfterFiFi{\BIC@DivCleanup{#4#7}}%
    \else
      \BIC@AfterFiFi{%
        \BIC@DivSub#1!#5!#2#3!#4#7!#5!%
      }%
    \fi
  \BIC@Fi
}
%    \end{macrocode}
%    \end{macro}
%    \begin{macro}{\BIC@ProcessDivIV}
%    |#1#2#3|: $x$, |=|, $x>4y$\\
%    |#4|: result\\
%    |#5|: $y$\\
%    |#6|: $2y$\\
%    |#7|: $4y$\\
%    |#8|: $6y$\\
%    |#9|: $8y$
%    \begin{macrocode}
\def\BIC@@@ProcessDiv#1#2#3!#4!#5!#6!#7!#8!#9!{%
  \ifcase\BIC@PosCmp#8!#1!% 6y = #1
    \ifx#2=%
      \BIC@AfterFiFi{\BIC@DivCleanup{#46}}%
    \else
      \BIC@AfterFiFi{%
        \BIC@ProcessDiv#2#3!#46!#5!#6!#7!#8!#9!%
      }%
    \fi
  \or % 6y > #1
    \BIC@AfterFi{%
      \expandafter\BIC@ProcessDivII\romannumeral0%
      \BIC@SubXY#1!#7!!!%
      !#2#3!#4!#5!45%
      #6!#7!#8!#9!%
    }%
  \else % 6y < #1
    \ifcase\BIC@PosCmp#9!#1!% 8y = #1
      \ifx#2=%
        \BIC@AfterFiFiFi{\BIC@DivCleanup{#48}}%
      \else
        \BIC@AfterFiFiFi{%
          \BIC@ProcessDiv#2#3!#48!#5!#6!#7!#8!#9!%
        }%
      \fi
    \or % 8y > #1
      \BIC@AfterFiFi{%
        \expandafter\BIC@ProcessDivII\romannumeral0%
        \BIC@SubXY#1!#8!!!%
        !#2#3!#4!#5!67%
        #6!#7!#8!#9!%
      }%
    \else % 8y < #1
      \BIC@AfterFiFi{%
        \expandafter\BIC@ProcessDivII\romannumeral0%
        \BIC@SubXY#1!#9!!!%
        !#2#3!#4!#5!89%
        #6!#7!#8!#9!%
      }%
    \fi
  \BIC@Fi
}
%    \end{macrocode}
%    \end{macro}
%
% \subsection{\op{Mod}}
%
%    \begin{macro}{\bigintcalcMod}
%    |#1|: $x$\\
%    |#2|: $y$
%    \begin{macrocode}
\def\bigintcalcMod#1{%
  \romannumeral0%
  \expandafter\expandafter\expandafter\BIC@Mod
  \bigintcalcNum{#1}!%
}
%    \end{macrocode}
%    \end{macro}
%    \begin{macro}{\BIC@Mod}
%    |#1|: $x$\\
%    |#2|: $y$
%    \begin{macrocode}
\def\BIC@Mod#1!#2{%
  \expandafter\expandafter\expandafter\BIC@ModSwitchSign
  \bigintcalcNum{#2}!#1!%
}
%    \end{macrocode}
%    \end{macro}
%    \begin{macro}{\BigIntCalcMod}
%    \begin{macrocode}
\def\BigIntCalcMod#1!#2!{%
  \romannumeral0%
  \BIC@ModSwitchSign#2!#1!%
}
%    \end{macrocode}
%    \end{macro}
%    \begin{macro}{\BIC@ModSwitchSign}
%    Decision table for \cs{BIC@ModSwitchSign}.
%    \begin{quote}
%    \begin{tabular}{@{}|l|l|l|@{}}
%    \hline
%    $y=0$ & \multicolumn{1}{l}{} & DivisionByZero\\
%    \hline
%    $y>0$ & $x=0$ & $0$\\
%    \cline{2-3}
%          & else  & ModSwitch$(+,x,y)$\\
%    \hline
%    $y<0$ & \multicolumn{1}{l}{} & ModSwitch$(-,-x,-y)$\\
%    \hline
%    \end{tabular}
%    \end{quote}
%    |#1#2|: $y$\\
%    |#3#4|: $x$
%    \begin{macrocode}
\def\BIC@ModSwitchSign#1#2!#3#4!{%
  \ifcase\ifx\\#2\\%
           \ifx#100 % y = 0
           \else1 % y > 0
           \fi
          \else
            \ifx#1-2 % y < 0
            \else1 % y > 0
            \fi
          \fi
    \BIC@AfterFi{ 0\BigIntCalcError:DivisionByZero}%
  \or % y > 0
    \ifcase\ifx\\#4\\\ifx#300 \else1 \fi\else1 \fi % x = 0
      \BIC@AfterFiFi{ 0}%
    \else
      \BIC@AfterFiFi{%
        \BIC@ModSwitch{}#3#4!#1#2!%
      }%
    \fi
  \else % y < 0
    \ifcase\ifx\\#4\\%
             \ifx#300 % x = 0
             \else1 % x > 0
             \fi
           \else
             \ifx#3-2 % x < 0
             \else1 % x > 0
             \fi
           \fi
      \BIC@AfterFiFi{ 0}%
    \or % x > 0
      \BIC@AfterFiFi{%
        \BIC@ModSwitch--#3#4!#2!%
      }%
    \else % x < 0
      \BIC@AfterFiFi{%
        \BIC@ModSwitch-#4!#2!%
      }%
    \fi
  \BIC@Fi
}
%    \end{macrocode}
%    \end{macro}
%    \begin{macro}{\BIC@ModSwitch}
%    Decision table for \cs{BIC@ModSwitch}.
%    \begin{quote}
%    \begin{tabular}{@{}|l|l|l|@{}}
%    \hline
%    $y=1$ & \multicolumn{1}{l}{} & $0$\\
%    \hline
%    $y=2$ & ifodd$(x)$ & sign $1$\\
%    \cline{2-3}
%          & else       & $0$\\
%    \hline
%    $y>2$ & $x<0$      &
%        $z\leftarrow x-(x/y)*y;\quad (z<0)\mathbin{?}z+y\mathbin{:}z$\\
%    \cline{2-3}
%          & $x>0$      & $x - (x/y) * y$\\
%    \hline
%    \end{tabular}
%    \end{quote}
%    |#1|: sign\\
%    |#2#3|: $x$\\
%    |#4#5|: $y$
%    \begin{macrocode}
\def\BIC@ModSwitch#1#2#3!#4#5!{%
  \ifcase\ifx\\#5\\%
           \ifx#410 % y = 1
           \else\ifx#421 % y = 2
           \else2 % y > 2
           \fi\fi
         \else2 % y > 2
         \fi
    \BIC@AfterFi{ 0}% y = 1
  \or % y = 2
    \ifcase\BIC@ModTwo#2#3! % even(x)
      \BIC@AfterFiFi{ 0}%
    \or % odd(x)
      \BIC@AfterFiFi{ #11}%
?   \else\BigIntCalcError:ThisCannotHappen%
    \fi
  \or % y > 2
    \ifx\\#1\\%
    \else
      \expandafter\BIC@Space\romannumeral0%
      \expandafter\BIC@ModMinus\romannumeral0%
    \fi
    \ifx#2-% x < 0
      \BIC@AfterFiFi{%
        \expandafter\expandafter\expandafter\BIC@ModX
        \bigintcalcSub{#2#3}{%
          \bigintcalcMul{#4#5}{\bigintcalcDiv{#2#3}{#4#5}}%
        }!#4#5!%
      }%
    \else % x > 0
      \BIC@AfterFiFi{%
        \expandafter\expandafter\expandafter\BIC@Space
        \bigintcalcSub{#2#3}{%
          \bigintcalcMul{#4#5}{\bigintcalcDiv{#2#3}{#4#5}}%
        }%
      }%
    \fi
? \else\BigIntCalcError:ThisCannotHappen%
  \BIC@Fi
}
%    \end{macrocode}
%    \end{macro}
%    \begin{macro}{\BIC@ModMinus}
%    \begin{macrocode}
\def\BIC@ModMinus#1{%
  \ifx#10%
    \BIC@AfterFi{ 0}%
  \else
    \BIC@AfterFi{ -#1}%
  \BIC@Fi
}
%    \end{macrocode}
%    \end{macro}
%    \begin{macro}{\BIC@ModX}
%    |#1#2|: $z$\\
%    |#3|: $x$
%    \begin{macrocode}
\def\BIC@ModX#1#2!#3!{%
  \ifx#1-% z < 0
    \BIC@AfterFi{%
      \expandafter\BIC@Space\romannumeral0%
      \BIC@SubXY#3!#2!!!%
    }%
  \else % z >= 0
    \BIC@AfterFi{ #1#2}%
  \BIC@Fi
}
%    \end{macrocode}
%    \end{macro}
%
%    \begin{macrocode}
\BIC@AtEnd%
%    \end{macrocode}
%
%    \begin{macrocode}
%</package>
%    \end{macrocode}
%
% \section{Test}
%
% \subsection{Catcode checks for loading}
%
%    \begin{macrocode}
%<*test1>
%    \end{macrocode}
%    \begin{macrocode}
\catcode`\{=1 %
\catcode`\}=2 %
\catcode`\#=6 %
\catcode`\@=11 %
\expandafter\ifx\csname count@\endcsname\relax
  \countdef\count@=255 %
\fi
\expandafter\ifx\csname @gobble\endcsname\relax
  \long\def\@gobble#1{}%
\fi
\expandafter\ifx\csname @firstofone\endcsname\relax
  \long\def\@firstofone#1{#1}%
\fi
\expandafter\ifx\csname loop\endcsname\relax
  \expandafter\@firstofone
\else
  \expandafter\@gobble
\fi
{%
  \def\loop#1\repeat{%
    \def\body{#1}%
    \iterate
  }%
  \def\iterate{%
    \body
      \let\next\iterate
    \else
      \let\next\relax
    \fi
    \next
  }%
  \let\repeat=\fi
}%
\def\RestoreCatcodes{}
\count@=0 %
\loop
  \edef\RestoreCatcodes{%
    \RestoreCatcodes
    \catcode\the\count@=\the\catcode\count@\relax
  }%
\ifnum\count@<255 %
  \advance\count@ 1 %
\repeat

\def\RangeCatcodeInvalid#1#2{%
  \count@=#1\relax
  \loop
    \catcode\count@=15 %
  \ifnum\count@<#2\relax
    \advance\count@ 1 %
  \repeat
}
\def\RangeCatcodeCheck#1#2#3{%
  \count@=#1\relax
  \loop
    \ifnum#3=\catcode\count@
    \else
      \errmessage{%
        Character \the\count@\space
        with wrong catcode \the\catcode\count@\space
        instead of \number#3%
      }%
    \fi
  \ifnum\count@<#2\relax
    \advance\count@ 1 %
  \repeat
}
\def\space{ }
\expandafter\ifx\csname LoadCommand\endcsname\relax
  \def\LoadCommand{\input bigintcalc.sty\relax}%
\fi
\def\Test{%
  \RangeCatcodeInvalid{0}{47}%
  \RangeCatcodeInvalid{58}{64}%
  \RangeCatcodeInvalid{91}{96}%
  \RangeCatcodeInvalid{123}{255}%
  \catcode`\@=12 %
  \catcode`\\=0 %
  \catcode`\%=14 %
  \LoadCommand
  \RangeCatcodeCheck{0}{36}{15}%
  \RangeCatcodeCheck{37}{37}{14}%
  \RangeCatcodeCheck{38}{47}{15}%
  \RangeCatcodeCheck{48}{57}{12}%
  \RangeCatcodeCheck{58}{63}{15}%
  \RangeCatcodeCheck{64}{64}{12}%
  \RangeCatcodeCheck{65}{90}{11}%
  \RangeCatcodeCheck{91}{91}{15}%
  \RangeCatcodeCheck{92}{92}{0}%
  \RangeCatcodeCheck{93}{96}{15}%
  \RangeCatcodeCheck{97}{122}{11}%
  \RangeCatcodeCheck{123}{255}{15}%
  \RestoreCatcodes
}
\Test
\csname @@end\endcsname
\end
%    \end{macrocode}
%    \begin{macrocode}
%</test1>
%    \end{macrocode}
%
% \subsection{Macro tests}
%
% \subsubsection{Preamble with test macro definitions}
%
%    \begin{macrocode}
%<*test2>
\NeedsTeXFormat{LaTeX2e}
\nofiles
\documentclass{article}
%<noetex>\let\SavedNumexpr\numexpr
%<noetex>\let\numexpr\UNDEFINED
\makeatletter
\chardef\BIC@TestMode=1 %
\makeatother
\usepackage{bigintcalc}[2016/05/16]
%<noetex>\let\numexpr\SavedNumexpr
\usepackage{qstest}
\IncludeTests{*}
\LogTests{log}{*}{*}
\newcommand*{\TestSpaceAtEnd}[1]{%
%<noetex>  \let\SavedNumexpr\numexpr
%<noetex>  \let\numexpr\UNDEFINED
  \edef\resultA{#1}%
  \edef\resultB{#1 }%
%<noetex>  \let\numexpr\SavedNumexpr
  \Expect*{\resultA\space}*{\resultB}%
}
\newcommand*{\TestResult}[2]{%
%<noetex>  \let\SavedNumexpr\numexpr
%<noetex>  \let\numexpr\UNDEFINED
  \edef\result{#1}%
%<noetex>  \let\numexpr\SavedNumexpr
  \Expect*{\result}{#2}%
}
\newcommand*{\TestResultTwoExpansions}[2]{%
%<*noetex>
  \begingroup
    \let\numexpr\UNDEFINED
    \expandafter\expandafter\expandafter
  \endgroup
%</noetex>
  \expandafter\expandafter\expandafter\Expect
  \expandafter\expandafter\expandafter{#1}{#2}%
}
\newcount\TestCount
%<etex>\newcommand*{\TestArg}[1]{\numexpr#1\relax}
%<noetex>\newcommand*{\TestArg}[1]{#1}
\newcommand*{\TestTeXDivide}[2]{%
  \TestCount=\TestArg{#1}\relax
  \divide\TestCount by \TestArg{#2}\relax
  \Expect*{\bigintcalcDiv{#1}{#2}}*{\the\TestCount}%
}
\newcommand*{\Test}[2]{%
  \TestResult{#1}{#2}%
  \TestResultTwoExpansions{#1}{#2}%
  \TestSpaceAtEnd{#1}%
}
\newcommand*{\TestExch}[2]{\Test{#2}{#1}}
\newcommand*{\TestInv}[2]{%
  \Test{\bigintcalcInv{#1}}{#2}%
}
\newcommand*{\TestAbs}[2]{%
  \Test{\bigintcalcAbs{#1}}{#2}%
}
\newcommand*{\TestSgn}[2]{%
  \Test{\bigintcalcSgn{#1}}{#2}%
}
\newcommand*{\TestMin}[3]{%
  \Test{\bigintcalcMin{#1}{#2}}{#3}%
}
\newcommand*{\TestMax}[3]{%
  \Test{\bigintcalcMax{#1}{#2}}{#3}%
}
\newcommand*{\TestCmp}[3]{%
  \Test{\bigintcalcCmp{#1}{#2}}{#3}%
}
\newcommand*{\TestOdd}[2]{%
  \Test{\bigintcalcOdd{#1}}{#2}%
  \edef\x{%
    \noexpand\Test{%
      \noexpand\BigIntCalcOdd
      \bigintcalcAbs{#1}!%
    }{#2}%
  }%
  \x
}
\newcommand*{\TestInc}[2]{%
  \Test{\bigintcalcInc{#1}}{#2}%
  \ifnum\bigintcalcSgn{#1}>-1 %
    \edef\x{%
      \noexpand\Test{%
        \noexpand\BigIntCalcInc\bigintcalcNum{#1}!%
      }{#2}%
    }%
    \x
  \fi
}
\newcommand*{\TestDec}[2]{%
  \Test{\bigintcalcDec{#1}}{#2}%
  \ifnum\bigintcalcSgn{#1}>0 %
    \edef\x{%
      \noexpand\Test{%
        \noexpand\BigIntCalcDec\bigintcalcNum{#1}!%
      }{#2}%
    }%
    \x
  \fi
}
\newcommand*{\TestAdd}[3]{%
  \Test{\bigintcalcAdd{#1}{#2}}{#3}%
  \ifnum\bigintcalcSgn{#1}>0 %
    \ifnum\bigintcalcSgn{#2}> 0 %
      \ifnum\bigintcalcCmp{#1}{#2}>0 %
        \edef\x{%
          \noexpand\Test{%
            \noexpand\BigIntCalcAdd
            \bigintcalcNum{#1}!\bigintcalcNum{#2}!%
          }{#3}%
        }%
        \x
      \else
        \edef\x{%
          \noexpand\Test{%
            \noexpand\BigIntCalcAdd
            \bigintcalcNum{#2}!\bigintcalcNum{#1}!%
          }{#3}%
        }%
        \x
      \fi
    \fi
  \fi
}
\newcommand*{\TestSub}[3]{%
  \Test{\bigintcalcSub{#1}{#2}}{#3}%
  \ifnum\bigintcalcSgn{#1}>0 %
    \ifnum\bigintcalcSgn{#2}> 0 %
      \ifnum\bigintcalcCmp{#1}{#2}>0 %
        \edef\x{%
          \noexpand\Test{%
            \noexpand\BigIntCalcSub
            \bigintcalcNum{#1}!\bigintcalcNum{#2}!%
          }{#3}%
        }%
        \x
      \fi
    \fi
  \fi
}
\newcommand*{\TestShl}[2]{%
  \Test{\bigintcalcShl{#1}}{#2}%
  \edef\x{%
    \noexpand\Test{%
      \noexpand\BigIntCalcShl\bigintcalcAbs{#1}!%
    }{\bigintcalcAbs{#2}}%
  }%
  \x
}
\newcommand*{\TestShr}[2]{%
  \Test{\bigintcalcShr{#1}}{#2}%
  \edef\x{%
    \noexpand\Test{%
      \noexpand\BigIntCalcShr\bigintcalcAbs{#1}!%
    }{\bigintcalcAbs{#2}}%
  }%
  \x
}
\newcommand*{\TestMul}[3]{%
  \Test{\bigintcalcMul{#1}{#2}}{#3}%
  \edef\x{%
    \noexpand\Test{%
      \noexpand\BigIntCalcMul
      \bigintcalcAbs{#1}!\bigintcalcAbs{#2}!%
    }{\bigintcalcAbs{#3}}%
  }%
  \x
}
\newcommand*{\TestSqr}[2]{%
  \Test{\bigintcalcSqr{#1}}{#2}%
}
\newcommand*{\TestFac}[2]{%
  \expandafter\TestExch\expandafter{%
    \the\numexpr#2%
  }{\bigintcalcFac{#1}}%
}
\newcommand*{\TestFacBig}[2]{%
  \Test{\bigintcalcFac{#1}}{#2}%
}
\newcommand*{\TestPow}[3]{%
  \Test{\bigintcalcPow{#1}{#2}}{#3}%
}
\newcommand*{\TestDiv}[3]{%
  \Test{\bigintcalcDiv{#1}{#2}}{#3}%
  \TestTeXDivide{#1}{#2}%
}
\newcommand*{\TestDivBig}[3]{%
  \Test{\bigintcalcDiv{#1}{#2}}{#3}%
  \edef\x{%
    \noexpand\Test{%
      \noexpand\BigIntCalcDiv\bigintcalcAbs{#1}!\bigintcalcAbs{#2}!%
    }{\bigintcalcAbs{#3}}%
  }%
}
\newcommand*{\TestMod}[3]{%
  \Test{\bigintcalcMod{#1}{#2}}{#3}%
  \ifcase\ifcase\bigintcalcSgn{#1} 0%
         \or
           \ifcase\bigintcalcSgn{#2} 1%
           \or 0%
           \else 1%
           \fi
         \else
           \ifcase\bigintcalcSgn{#2} 1%
           \or 1%
           \else 0%
           \fi
         \fi\relax
    \edef\x{%
      \noexpand\Test{%
        \noexpand\BigIntCalcMod
        \bigintcalcAbs{#1}!\bigintcalcAbs{#2}!%
      }{\bigintcalcAbs{#3}}%
    }%
    \x
  \fi
}
%    \end{macrocode}
%
% \subsubsection{Time}
%
%    \begin{macrocode}
\begingroup\expandafter\expandafter\expandafter\endgroup
\expandafter\ifx\csname pdfresettimer\endcsname\relax
\else
  \makeatletter
  \newcount\SummaryTime
  \newcount\TestTime
  \SummaryTime=\z@
  \newcommand*{\PrintTime}[2]{%
    \typeout{%
      [Time #1: \strip@pt\dimexpr\number#2sp\relax\space s]%
    }%
  }%
  \newcommand*{\StartTime}[1]{%
    \renewcommand*{\TimeDescription}{#1}%
    \pdfresettimer
  }%
  \newcommand*{\TimeDescription}{}%
  \newcommand*{\StopTime}{%
    \TestTime=\pdfelapsedtime
    \global\advance\SummaryTime\TestTime
    \PrintTime\TimeDescription\TestTime
  }%
  \let\saved@qstest\qstest
  \let\saved@endqstest\endqstest
  \def\qstest#1#2{%
    \saved@qstest{#1}{#2}%
    \StartTime{#1}%
  }%
  \def\endqstest{%
    \StopTime
    \saved@endqstest
  }%
  \AtEndDocument{%
    \PrintTime{summary}\SummaryTime
  }%
  \makeatother
\fi
%    \end{macrocode}
%
% \subsubsection{Test sets}
%
%    \begin{macrocode}
\makeatletter

\begin{qstest}{inv}{inv}%
  \TestInv{0}{0}%
  \TestInv{1}{-1}%
  \TestInv{-1}{1}%
  \TestInv{10}{-10}%
  \TestInv{-10}{10}%
  \TestInv{2147483647}{-2147483647}%
  \TestInv{-2147483647}{2147483647}%
  \TestInv{12345678901234567890}{-12345678901234567890}%
  \TestInv{-12345678901234567890}{12345678901234567890}%
  \TestInv{ 0 }{0}%
  \TestInv{ 1 }{-1}%
  \TestInv{--1}{-1}%
  \TestInv{\number\z@}{0}%
  \TestInv{\ifx\relax\relax1\fi}{-1}%
  \TestInv{\ifx\relax\relax-\fi\ifx234\else1\fi}{1}%
\end{qstest}

\begin{qstest}{abs}{abs}%
  \TestAbs{0}{0}%
  \TestAbs{1}{1}%
  \TestAbs{-1}{1}%
  \TestAbs{10}{10}%
  \TestAbs{-10}{10}%
  \TestAbs{2147483647}{2147483647}%
  \TestAbs{-2147483647}{2147483647}%
  \TestAbs{12345678901234567890}{12345678901234567890}%
  \TestAbs{-12345678901234567890}{12345678901234567890}%
  \TestAbs{ 0 }{0}%
  \TestAbs{ 1 }{1}%
  \TestAbs{--1}{1}%
  \TestAbs{-+-+1}{1}%
  \TestAbs{00000000000}{0}%
  \TestAbs{00000001000}{1000}%
  \TestAbs{\ifx\relax\relax 0\else 1\fi}{0}%
\end{qstest}

\begin{qstest}{sign}{sign}%
  \TestSgn{0}{0}%
  \TestSgn{1}{1}%
  \TestSgn{-1}{-1}%
  \TestSgn{10}{1}%
  \TestSgn{-10}{-1}%
  \TestSgn{2147483647}{1}%
  \TestSgn{-2147483647}{-1}%
  \TestSgn{12345678901234567890}{1}%
  \TestSgn{-12345678901234567890}{-1}%
  \TestSgn{ 0 }{0}%
  \TestSgn{ 2 }{1}%
  \TestSgn{ -2 }{-1}%
  \TestSgn{--2}{1}%
  \TestSgn{\number\z@}{0}%
  \TestSgn{\number\@ne}{1}%
  \TestSgn{\number\m@ne}{-1}%
  \TestSgn{%
    -+-+\number\z@\number\z@
    \iftrue1\fi\iftrue2\fi\iftrue3\fi
  }{1}%
\end{qstest}

\begin{qstest}{min}{min}%
  \TestMin{0}{1}{0}%
  \TestMin{1}{0}{0}%
  \TestMin{-10}{-20}{-20}%
  \TestMin{ 1 }{ 2 }{1}%
  \TestMin{ 2 }{ 1 }{1}%
  \TestMin{1}{1}{1}%
  \TestMin{\number\z@}{\number\@ne}{0}%
  \TestMin{\number\@ne}{\number\m@ne}{-1}%
\end{qstest}

\begin{qstest}{max}{max}%
  \TestMax{0}{1}{1}%
  \TestMax{1}{0}{1}%
  \TestMax{-10}{-20}{-10}%
  \TestMax{ 1 }{ 2 }{2}%
  \TestMax{ 2 }{ 1 }{2}%
  \TestMax{1}{1}{1}%
  \TestMax{\number\z@}{\number\@ne}{1}%
  \TestMax{\number\@ne}{\number\m@ne}{1}%
\end{qstest}

\begin{qstest}{cmp}{cmp}%
  \TestCmp{0}{0}{0}%
  \TestCmp{-21}{17}{-1}%
  \TestCmp{3}{4}{-1}%
  \TestCmp{-10}{-10}{0}%
  \TestCmp{-10}{-11}{1}%
  \TestCmp{100}{5}{1}%
  \TestCmp{9}{10}{-1}%
  \TestCmp{10}{9}{1}%
  \TestCmp{ 3 }{ 3 }{0}%
  \TestCmp{-9}{-10}{1}%
  \TestCmp{-10}{-9}{-1}%
  \TestCmp{-3}{-3}{0}%
  \TestCmp{0}{-2}{1}%
  \TestCmp{0}{2}{-1}%
  \TestCmp{2}{0}{1}%
  \TestCmp{-2}{0}{-1}%
  \TestCmp{12}{11}{1}%
  \TestCmp{11}{12}{-1}%
  \TestCmp{2147483647}{-2147483647}{1}%
  \TestCmp{-2147483647}{2147483647}{-1}%
  \TestCmp{2147483647}{2147483647}{0}%
  \TestCmp{\number\z@}{\number\@ne}{-1}%
  \TestCmp{\number\@ne}{\number\m@ne}{1}%
  \TestCmp{ 4 }{ 5 }{-1}%
  \TestCmp{ -3 }{ -7 }{1}%
\end{qstest}

\begin{qstest}{odd}{odd}
\tracingmacros=1
  \TestOdd{0}{0}%
  \TestOdd{1}{1}%
  \TestOdd{2}{0}%
  \TestOdd{3}{1}%
  \TestOdd{14}{0}%
  \TestOdd{15}{1}%
  \TestOdd{12345678901234567896}{0}%
  \TestOdd{12345678901234567897}{1}%
\end{qstest}

\begin{qstest}{inc}{inc}%
  \TestInc{0}{1}%
  \TestInc{1}{2}%
  \TestInc{-1}{0}%
  \TestInc{10}{11}%
  \TestInc{-10}{-9}%
  \TestInc{ 3 }{4}%
  \TestInc{999}{1000}%
  \TestInc{-1000}{-999}%
  \TestInc{129}{130}%
  \TestInc{2147483646}{2147483647}%
  \TestInc{-2147483647}{-2147483646}%
  \TestInc{12345678901234567890}{12345678901234567891}%
  \TestInc{99999999999999999999}{100000000000000000000}%
  \TestInc{-12345678901234567891}{-12345678901234567890}%
  \TestInc{-100000000000000000000}{-99999999999999999999}%
\end{qstest}

\begin{qstest}{dec}{dec}%
  \TestDec{0}{-1}%
  \TestDec{1}{0}%
  \TestDec{-1}{-2}%
  \TestDec{10}{9}%
  \TestDec{-10}{-11}%
  \TestDec{1000}{999}%
  \TestDec{-999}{-1000}%
  \TestDec{130}{129}%
  \TestDec{2147483647}{2147483646}%
  \TestDec{-2147483646}{-2147483647}%
  \TestDec{12345678901234567891}{12345678901234567890}%
  \TestDec{100000000000000000000}{99999999999999999999}%
  \TestDec{-12345678901234567890}{-12345678901234567891}%
  \TestDec{-99999999999999999999}{-100000000000000000000}%
\end{qstest}

\begin{qstest}{add}{add}%
  \TestAdd{0}{0}{0}%
  \TestAdd{1}{0}{1}%
  \TestAdd{0}{1}{1}%
  \TestAdd{1}{2}{3}%
  \TestAdd{-1}{-1}{-2}%
  \TestAdd{2147483646}{1}{2147483647}%
  \TestAdd{-2147483647}{2147483647}{0}%
  \TestAdd{20}{-5}{15}%
  \TestAdd{-4}{-1}{-5}%
  \TestAdd{-1}{-4}{-5}%
  \TestAdd{-4}{1}{-3}%
  \TestAdd{-1}{4}{3}%
  \TestAdd{4}{-1}{3}%
  \TestAdd{1}{-4}{-3}%
  \TestAdd{-4}{-1}{-5}%
  \TestAdd{-1}{-4}{-5}%
  \TestAdd{ -4 }{ -1 }{-5}%
  \TestAdd{ -1 }{ -4 }{-5}%
  \TestAdd{ -4 }{ 1 }{-3}%
  \TestAdd{ -1 }{ 4 }{3}%
  \TestAdd{ 4 }{ -1 }{3}%
  \TestAdd{ 1 }{ -4 }{-3}%
  \TestAdd{ -4 }{ -1 }{-5}%
  \TestAdd{ -1 }{ -4 }{-5}%
  \TestAdd{876543210}{111111111}{987654321}%
  \TestAdd{999999999}{2}{1000000001}%
\end{qstest}

\begin{qstest}{sub}{sub}
  \TestSub{0}{0}{0}%
  \TestSub{1}{0}{1}%
  \TestSub{1}{2}{-1}%
  \TestSub{-1}{-1}{0}%
  \TestSub{2147483646}{-1}{2147483647}%
  \TestSub{-2147483647}{-2147483647}{0}%
  \TestSub{-4}{-1}{-3}%
  \TestSub{-1}{-4}{3}%
  \TestSub{-4}{1}{-5}%
  \TestSub{-1}{4}{-5}%
  \TestSub{4}{-1}{5}%
  \TestSub{1}{-4}{5}%
  \TestSub{-4}{-1}{-3}%
  \TestSub{-1}{-4}{3}%
  \TestSub{ -4 }{ -1 }{-3}%
  \TestSub{ -1 }{ -4 }{3}%
  \TestSub{ -4 }{ 1 }{-5}%
  \TestSub{ -1 }{ 4 }{-5}%
  \TestSub{ 4 }{ -1 }{5}%
  \TestSub{ 1 }{ -4 }{5}%
  \TestSub{ -4 }{ -1 }{-3}%
  \TestSub{ -1 }{ -4 }{3}%
  \TestSub{1000000000}{2}{999999998}%
  \TestSub{987654321}{111111111}{876543210}%
\end{qstest}

\begin{qstest}{shl}{shl}
  \TestShl{0}{0}%
  \TestShl{1}{2}%
  \TestShl{2}{4}%
  \TestShl{5621}{11242}%
  \TestShl{1073741823}{2147483646}%
\end{qstest}

\begin{qstest}{shr}{shr}
  \TestShr{0}{0}%
  \TestShr{1}{0}%
  \TestShr{2}{1}%
  \TestShr{3}{1}%
  \TestShr{4}{2}%
  \TestShr{5}{2}%
  \TestShr{6}{3}%
  \TestShr{7}{3}%
  \TestShr{8}{4}%
  \TestShr{9}{4}%
  \TestShr{10}{5}%
  \TestShr{11}{5}%
  \TestShr{12}{6}%
  \TestShr{13}{6}%
  \TestShr{14}{7}%
  \TestShr{15}{7}%
  \TestShr{16}{8}%
  \TestShr{17}{8}%
  \TestShr{18}{9}%
  \TestShr{19}{9}%
  \TestShr{20}{10}%
  \TestShr{21}{10}%
  \TestShr{22}{11}%
  \TestShr{11241}{5620}%
  \TestShr{73054202}{36527101}%
  \TestShr{2147483646}{1073741823}%
\end{qstest}

\begin{qstest}{mul}{mul}
  \TestMul{0}{0}{0}%
  \TestMul{1}{0}{0}%
  \TestMul{0}{1}{0}%
  \TestMul{1}{1}{1}%
  \TestMul{3}{1}{3}%
  \TestMul{1}{-3}{-3}%
  \TestMul{-4}{-5}{20}%
  \TestMul{3}{7}{21}%
  \TestMul{7}{3}{21}%
  \TestMul{3}{-7}{-21}%
  \TestMul{7}{-3}{-21}%
  \TestMul{-3}{7}{-21}%
  \TestMul{-7}{3}{-21}%
  \TestMul{-3}{-7}{21}%
  \TestMul{-7}{-3}{21}%
  \TestMul{12}{11}{132}%
  \TestMul{999}{333}{332667}%
  \TestMul{1000}{4321}{4321000}%
  \TestMul{12345}{173955}{2147474475}%
  \TestMul{1073741823}{2}{2147483646}%
  \TestMul{2}{1073741823}{2147483646}%
  \TestMul{-1073741823}{2}{-2147483646}%
  \TestMul{2}{-1073741823}{-2147483646}%
  \TestMul{6706022400}{13}{87178291200}%
\end{qstest}

\begin{qstest}{sqr}{sqr}
  \TestSqr{0}{0}%
  \TestSqr{1}{1}%
  \TestSqr{2}{4}%
  \TestSqr{3}{9}%
  \TestSqr{4}{16}%
  \TestSqr{9}{81}%
  \TestSqr{10}{100}%
  \TestSqr{46340}{2147395600}%
  \TestSqr{-1}{1}%
  \TestSqr{-2}{4}%
  \TestSqr{-46340}{2147395600}%
\end{qstest}

\begin{qstest}{fac}{fac}
  \TestFac{0}{1}%
  \TestFac{1}{1}%
  \TestFac{2}{2}%
  \TestFac{3}{2*3}%
  \TestFac{4}{2*3*4}%
  \TestFac{5}{2*3*4*5}%
  \TestFac{6}{2*3*4*5*6}%
  \TestFac{7}{2*3*4*5*6*7}%
  \TestFac{8}{2*3*4*5*6*7*8}%
  \TestFac{9}{2*3*4*5*6*7*8*9}%
  \TestFac{10}{2*3*4*5*6*7*8*9*10}%
  \TestFac{11}{2*3*4*5*6*7*8*9*10*11}%
  \TestFac{12}{2*3*4*5*6*7*8*9*10*11*12}%
  \TestFacBig{13}{6227020800}%
  \TestFacBig{14}{87178291200}%
  \TestFacBig{15}{1307674368000}%
  \TestFacBig{16}{20922789888000}%
  \TestFacBig{17}{355687428096000}%
  \TestFacBig{18}{6402373705728000}%
  \TestFacBig{19}{121645100408832000}%
  \TestFacBig{20}{2432902008176640000}%
  \TestFacBig{21}{51090942171709440000}%
  \TestFacBig{22}{1124000727777607680000}%
\end{qstest}

\begin{qstest}{pow}{pow}
  \TestPow{-2}{0}{1}%
  \TestPow{-1}{0}{1}%
  \TestPow{0}{0}{1}%
  \TestPow{1}{0}{1}%
  \TestPow{2}{0}{1}%
  \TestPow{3}{0}{1}%
  \TestPow{-2}{1}{-2}%
  \TestPow{-1}{1}{-1}%
  \TestPow{1}{1}{1}%
  \TestPow{2}{1}{2}%
  \TestPow{3}{1}{3}%
  \TestPow{-2}{2}{4}%
  \TestPow{-1}{2}{1}%
  \TestPow{0}{2}{0}%
  \TestPow{1}{2}{1}%
  \TestPow{2}{2}{4}%
  \TestPow{3}{2}{9}%
  \TestPow{0}{1}{0}%
  \TestPow{1}{-2}{1}%
  \TestPow{1}{-1}{1}%
  \TestPow{-1}{-2}{1}%
  \TestPow{-1}{-1}{-1}%
  \TestPow{-1}{3}{-1}%
  \TestPow{-1}{4}{1}%
  \TestPow{-2}{-1}{0}%
  \TestPow{-2}{-2}{0}%
  \TestPow{2}{3}{8}%
  \TestPow{2}{4}{16}%
  \TestPow{2}{5}{32}%
  \TestPow{2}{6}{64}%
  \TestPow{2}{7}{128}%
  \TestPow{2}{8}{256}%
  \TestPow{2}{9}{512}%
  \TestPow{2}{10}{1024}%
  \TestPow{-2}{3}{-8}%
  \TestPow{-2}{4}{16}%
  \TestPow{-2}{5}{-32}%
  \TestPow{-2}{6}{64}%
  \TestPow{-2}{7}{-128}%
  \TestPow{-2}{8}{256}%
  \TestPow{-2}{9}{-512}%
  \TestPow{-2}{10}{1024}%
  \TestPow{3}{3}{27}%
  \TestPow{3}{4}{81}%
  \TestPow{3}{5}{243}%
  \TestPow{-3}{3}{-27}%
  \TestPow{-3}{4}{81}%
  \TestPow{-3}{5}{-243}%
  \TestPow{2}{30}{1073741824}%
  \TestPow{-3}{19}{-1162261467}%
  \TestPow{5}{13}{1220703125}%
  \TestPow{-7}{11}{-1977326743}%
\end{qstest}

\begin{qstest}{div}{div}
  \TestDiv{1}{1}{1}%
  \TestDiv{2}{1}{2}%
  \TestDiv{-2}{1}{-2}%
  \TestDiv{2}{-1}{-2}%
  \TestDiv{-2}{-1}{2}%
  \TestDiv{15}{2}{7}%
  \TestDiv{-16}{2}{-8}%
  \TestDiv{1}{2}{0}%
  \TestDiv{1}{3}{0}%
  \TestDiv{2}{3}{0}%
  \TestDiv{-2}{3}{0}%
  \TestDiv{2}{-3}{0}%
  \TestDiv{-2}{-3}{0}%
  \TestDiv{13}{3}{4}%
  \TestDiv{-13}{-3}{4}%
  \TestDiv{-13}{3}{-4}%
  \TestDiv{-6}{5}{-1}%
  \TestDiv{-5}{5}{-1}%
  \TestDiv{-4}{5}{0}%
  \TestDiv{-3}{5}{0}%
  \TestDiv{-2}{5}{0}%
  \TestDiv{-1}{5}{0}%
  \TestDiv{0}{5}{0}%
  \TestDiv{1}{5}{0}%
  \TestDiv{2}{5}{0}%
  \TestDiv{3}{5}{0}%
  \TestDiv{4}{5}{0}%
  \TestDiv{5}{5}{1}%
  \TestDiv{6}{5}{1}%
  \TestDiv{-5}{4}{-1}%
  \TestDiv{-4}{4}{-1}%
  \TestDiv{-3}{4}{0}%
  \TestDiv{-2}{4}{0}%
  \TestDiv{-1}{4}{0}%
  \TestDiv{0}{4}{0}%
  \TestDiv{1}{4}{0}%
  \TestDiv{2}{4}{0}%
  \TestDiv{3}{4}{0}%
  \TestDiv{4}{4}{1}%
  \TestDiv{5}{4}{1}%
  \TestDiv{12345}{678}{18}%
  \TestDiv{32372}{5952}{5}%
  \TestDiv{284271294}{18162}{15651}%
  \TestDiv{217652429}{12561}{17327}%
  \TestDiv{462028434}{5439}{84947}%
  \TestDiv{2147483647}{1000}{2147483}%
  \TestDiv{2147483647}{-1000}{-2147483}%
  \TestDiv{-2147483647}{1000}{-2147483}%
  \TestDiv{-2147483647}{-1000}{2147483}%
  \TestDiv{0}{3}{0}%
  \TestDiv{1}{3}{0}%
  \TestDiv{2}{3}{0}%
  \TestDiv{3}{3}{1}%
  \TestDiv{4}{3}{1}%
  \TestDiv{5}{3}{1}%
  \TestDiv{6}{3}{2}%
  \TestDiv{7}{3}{2}%
  \TestDiv{8}{3}{2}%
  \TestDiv{9}{3}{3}%
  \TestDiv{10}{3}{3}%
  \TestDiv{11}{3}{3}%
  \TestDiv{12}{3}{4}%
  \TestDiv{13}{3}{4}%
  \TestDiv{14}{3}{4}%
  \TestDiv{15}{3}{5}%
  \TestDiv{16}{3}{5}%
  \TestDiv{17}{3}{5}%
  \TestDiv{18}{3}{6}%
  \TestDiv{19}{3}{6}%
  \TestDiv{20}{3}{6}%
  \TestDiv{21}{3}{7}%
  \TestDiv{22}{3}{7}%
  \TestDiv{23}{3}{7}%
  \TestDiv{24}{3}{8}%
  \TestDiv{25}{3}{8}%
  \TestDiv{26}{3}{8}%
  \TestDiv{27}{3}{9}%
  \TestDiv{28}{3}{9}%
  \TestDiv{29}{3}{9}%
  \TestDiv{30}{3}{10}%
  \TestDiv{31}{3}{10}%
  \TestDivBig{17363436332507}{24702}{702916214}%
\end{qstest}

\begin{qstest}{mod}{mod}
  \TestMod{-6}{5}{4}%
  \TestMod{-5}{5}{0}%
  \TestMod{-4}{5}{1}%
  \TestMod{-3}{5}{2}%
  \TestMod{-2}{5}{3}%
  \TestMod{-1}{5}{4}%
  \TestMod{0}{5}{0}%
  \TestMod{1}{5}{1}%
  \TestMod{2}{5}{2}%
  \TestMod{3}{5}{3}%
  \TestMod{4}{5}{4}%
  \TestMod{5}{5}{0}%
  \TestMod{6}{5}{1}%
  \TestMod{-5}{4}{3}%
  \TestMod{-4}{4}{0}%
  \TestMod{-3}{4}{1}%
  \TestMod{-2}{4}{2}%
  \TestMod{-1}{4}{3}%
  \TestMod{0}{4}{0}%
  \TestMod{1}{4}{1}%
  \TestMod{2}{4}{2}%
  \TestMod{3}{4}{3}%
  \TestMod{4}{4}{0}%
  \TestMod{5}{4}{1}%
  \TestMod{-6}{-5}{-1}%
  \TestMod{-5}{-5}{0}%
  \TestMod{-4}{-5}{-4}%
  \TestMod{-3}{-5}{-3}%
  \TestMod{-2}{-5}{-2}%
  \TestMod{-1}{-5}{-1}%
  \TestMod{0}{-5}{0}%
  \TestMod{1}{-5}{-4}%
  \TestMod{2}{-5}{-3}%
  \TestMod{3}{-5}{-2}%
  \TestMod{4}{-5}{-1}%
  \TestMod{5}{-5}{0}%
  \TestMod{6}{-5}{-4}%
  \TestMod{-5}{-4}{-1}%
  \TestMod{-4}{-4}{0}%
  \TestMod{-3}{-4}{-3}%
  \TestMod{-2}{-4}{-2}%
  \TestMod{-1}{-4}{-1}%
  \TestMod{0}{-4}{0}%
  \TestMod{1}{-4}{-3}%
  \TestMod{2}{-4}{-2}%
  \TestMod{3}{-4}{-1}%
  \TestMod{4}{-4}{0}%
  \TestMod{5}{-4}{-3}%
  \TestMod{2147483647}{1000}{647}%
  \TestMod{2147483647}{-1000}{-353}%
  \TestMod{-2147483647}{1000}{353}%
  \TestMod{-2147483647}{-1000}{-647}%
  \TestMod{ 0 }{ 4 }{0}%
  \TestMod{ 1 }{ 4 }{1}%
  \TestMod{ -1 }{ 4 }{3}%
  \TestMod{ 0 }{ -4 }{0}%
  \TestMod{ 1 }{ -4 }{-3}%
  \TestMod{ -1 }{ -4 }{-1}%
  \TestMod{18362}{25}{12}%
\end{qstest}

\newcommand*{\TestError}[2]{%
  \begingroup
    \expandafter\def\csname BigIntCalcError:#1\endcsname{}%
    \Expect*{#2}{0}%
    \expandafter\def\csname BigIntCalcError:#1\endcsname{ERROR}%
    \Expect*{#2}{0ERROR}%
  \endgroup
}
\begin{qstest}{error}{error}
  \TestError{FacNegative}{\bigintcalcFac{-1}}%
  \TestError{FacNegative}{\bigintcalcFac{-2147483647}}%
  \TestError{DivisionByZero}{\bigintcalcPow{0}{-1}}%
  \TestError{DivisionByZero}{\bigintcalcDiv{1}{0}}%
  \TestError{DivisionByZero}{\bigintcalcMod{1}{0}}%
\end{qstest}

\begin{document}
\end{document}
%</test2>
%    \end{macrocode}
%
% \section{Installation}
%
% \subsection{Download}
%
% \paragraph{Package.} This package is available on
% CTAN\footnote{\url{http://ctan.org/pkg/bigintcalc}}:
% \begin{description}
% \item[\CTAN{macros/latex/contrib/oberdiek/bigintcalc.dtx}] The source file.
% \item[\CTAN{macros/latex/contrib/oberdiek/bigintcalc.pdf}] Documentation.
% \end{description}
%
%
% \paragraph{Bundle.} All the packages of the bundle `oberdiek'
% are also available in a TDS compliant ZIP archive. There
% the packages are already unpacked and the documentation files
% are generated. The files and directories obey the TDS standard.
% \begin{description}
% \item[\CTAN{install/macros/latex/contrib/oberdiek.tds.zip}]
% \end{description}
% \emph{TDS} refers to the standard ``A Directory Structure
% for \TeX\ Files'' (\CTAN{tds/tds.pdf}). Directories
% with \xfile{texmf} in their name are usually organized this way.
%
% \subsection{Bundle installation}
%
% \paragraph{Unpacking.} Unpack the \xfile{oberdiek.tds.zip} in the
% TDS tree (also known as \xfile{texmf} tree) of your choice.
% Example (linux):
% \begin{quote}
%   |unzip oberdiek.tds.zip -d ~/texmf|
% \end{quote}
%
% \paragraph{Script installation.}
% Check the directory \xfile{TDS:scripts/oberdiek/} for
% scripts that need further installation steps.
% Package \xpackage{attachfile2} comes with the Perl script
% \xfile{pdfatfi.pl} that should be installed in such a way
% that it can be called as \texttt{pdfatfi}.
% Example (linux):
% \begin{quote}
%   |chmod +x scripts/oberdiek/pdfatfi.pl|\\
%   |cp scripts/oberdiek/pdfatfi.pl /usr/local/bin/|
% \end{quote}
%
% \subsection{Package installation}
%
% \paragraph{Unpacking.} The \xfile{.dtx} file is a self-extracting
% \docstrip\ archive. The files are extracted by running the
% \xfile{.dtx} through \plainTeX:
% \begin{quote}
%   \verb|tex bigintcalc.dtx|
% \end{quote}
%
% \paragraph{TDS.} Now the different files must be moved into
% the different directories in your installation TDS tree
% (also known as \xfile{texmf} tree):
% \begin{quote}
% \def\t{^^A
% \begin{tabular}{@{}>{\ttfamily}l@{ $\rightarrow$ }>{\ttfamily}l@{}}
%   bigintcalc.sty & tex/generic/oberdiek/bigintcalc.sty\\
%   bigintcalc.pdf & doc/latex/oberdiek/bigintcalc.pdf\\
%   test/bigintcalc-test1.tex & doc/latex/oberdiek/test/bigintcalc-test1.tex\\
%   test/bigintcalc-test2.tex & doc/latex/oberdiek/test/bigintcalc-test2.tex\\
%   test/bigintcalc-test3.tex & doc/latex/oberdiek/test/bigintcalc-test3.tex\\
%   bigintcalc.dtx & source/latex/oberdiek/bigintcalc.dtx\\
% \end{tabular}^^A
% }^^A
% \sbox0{\t}^^A
% \ifdim\wd0>\linewidth
%   \begingroup
%     \advance\linewidth by\leftmargin
%     \advance\linewidth by\rightmargin
%   \edef\x{\endgroup
%     \def\noexpand\lw{\the\linewidth}^^A
%   }\x
%   \def\lwbox{^^A
%     \leavevmode
%     \hbox to \linewidth{^^A
%       \kern-\leftmargin\relax
%       \hss
%       \usebox0
%       \hss
%       \kern-\rightmargin\relax
%     }^^A
%   }^^A
%   \ifdim\wd0>\lw
%     \sbox0{\small\t}^^A
%     \ifdim\wd0>\linewidth
%       \ifdim\wd0>\lw
%         \sbox0{\footnotesize\t}^^A
%         \ifdim\wd0>\linewidth
%           \ifdim\wd0>\lw
%             \sbox0{\scriptsize\t}^^A
%             \ifdim\wd0>\linewidth
%               \ifdim\wd0>\lw
%                 \sbox0{\tiny\t}^^A
%                 \ifdim\wd0>\linewidth
%                   \lwbox
%                 \else
%                   \usebox0
%                 \fi
%               \else
%                 \lwbox
%               \fi
%             \else
%               \usebox0
%             \fi
%           \else
%             \lwbox
%           \fi
%         \else
%           \usebox0
%         \fi
%       \else
%         \lwbox
%       \fi
%     \else
%       \usebox0
%     \fi
%   \else
%     \lwbox
%   \fi
% \else
%   \usebox0
% \fi
% \end{quote}
% If you have a \xfile{docstrip.cfg} that configures and enables \docstrip's
% TDS installing feature, then some files can already be in the right
% place, see the documentation of \docstrip.
%
% \subsection{Refresh file name databases}
%
% If your \TeX~distribution
% (\teTeX, \mikTeX, \dots) relies on file name databases, you must refresh
% these. For example, \teTeX\ users run \verb|texhash| or
% \verb|mktexlsr|.
%
% \subsection{Some details for the interested}
%
% \paragraph{Attached source.}
%
% The PDF documentation on CTAN also includes the
% \xfile{.dtx} source file. It can be extracted by
% AcrobatReader 6 or higher. Another option is \textsf{pdftk},
% e.g. unpack the file into the current directory:
% \begin{quote}
%   \verb|pdftk bigintcalc.pdf unpack_files output .|
% \end{quote}
%
% \paragraph{Unpacking with \LaTeX.}
% The \xfile{.dtx} chooses its action depending on the format:
% \begin{description}
% \item[\plainTeX:] Run \docstrip\ and extract the files.
% \item[\LaTeX:] Generate the documentation.
% \end{description}
% If you insist on using \LaTeX\ for \docstrip\ (really,
% \docstrip\ does not need \LaTeX), then inform the autodetect routine
% about your intention:
% \begin{quote}
%   \verb|latex \let\install=y\input{bigintcalc.dtx}|
% \end{quote}
% Do not forget to quote the argument according to the demands
% of your shell.
%
% \paragraph{Generating the documentation.}
% You can use both the \xfile{.dtx} or the \xfile{.drv} to generate
% the documentation. The process can be configured by the
% configuration file \xfile{ltxdoc.cfg}. For instance, put this
% line into this file, if you want to have A4 as paper format:
% \begin{quote}
%   \verb|\PassOptionsToClass{a4paper}{article}|
% \end{quote}
% An example follows how to generate the
% documentation with pdf\LaTeX:
% \begin{quote}
%\begin{verbatim}
%pdflatex bigintcalc.dtx
%makeindex -s gind.ist bigintcalc.idx
%pdflatex bigintcalc.dtx
%makeindex -s gind.ist bigintcalc.idx
%pdflatex bigintcalc.dtx
%\end{verbatim}
% \end{quote}
%
% \section{Catalogue}
%
% The following XML file can be used as source for the
% \href{http://mirror.ctan.org/help/Catalogue/catalogue.html}{\TeX\ Catalogue}.
% The elements \texttt{caption} and \texttt{description} are imported
% from the original XML file from the Catalogue.
% The name of the XML file in the Catalogue is \xfile{bigintcalc.xml}.
%    \begin{macrocode}
%<*catalogue>
<?xml version='1.0' encoding='us-ascii'?>
<!DOCTYPE entry SYSTEM 'catalogue.dtd'>
<entry datestamp='$Date$' modifier='$Author$' id='bigintcalc'>
  <name>bigintcalc</name>
  <caption>Integer calculations on very large numbers.</caption>
  <authorref id='auth:oberdiek'/>
  <copyright owner='Heiko Oberdiek' year='2007,2011,2012'/>
  <license type='lppl1.3'/>
  <version number='1.4'/>
  <description>
    This package provides expandable arithmetic operations
    with big integers that can exceed TeX's number limits.
    <p/>
    The package is part of the <xref refid='oberdiek'>oberdiek</xref> bundle.
  </description>
  <documentation details='Package documentation'
      href='ctan:/macros/latex/contrib/oberdiek/bigintcalc.pdf'/>
  <ctan file='true' path='/macros/latex/contrib/oberdiek/bigintcalc.dtx'/>
  <miktex location='oberdiek'/>
  <texlive location='oberdiek'/>
  <install path='/macros/latex/contrib/oberdiek/oberdiek.tds.zip'/>
</entry>
%</catalogue>
%    \end{macrocode}
%
% \begin{History}
%   \begin{Version}{2007/09/27 v1.0}
%   \item
%     First version.
%   \end{Version}
%   \begin{Version}{2007/11/11 v1.1}
%   \item
%     Use of package \xpackage{pdftexcmds} for \LuaTeX\ support.
%   \end{Version}
%   \begin{Version}{2011/01/30 v1.2}
%   \item
%     Already loaded package files are not input in \hologo{plainTeX}.
%   \end{Version}
%   \begin{Version}{2012/04/08 v1.3}
%   \item
%     Fix: \xpackage{pdftexcmds} wasn't loaded in case of \hologo{LaTeX}.
%   \end{Version}
%   \begin{Version}{2016/05/16 v1.4}
%   \item
%     Documentation updates.
%   \end{Version}
% \end{History}
%
% \PrintIndex
%
% \Finale
\endinput
|
% \end{quote}
% Do not forget to quote the argument according to the demands
% of your shell.
%
% \paragraph{Generating the documentation.}
% You can use both the \xfile{.dtx} or the \xfile{.drv} to generate
% the documentation. The process can be configured by the
% configuration file \xfile{ltxdoc.cfg}. For instance, put this
% line into this file, if you want to have A4 as paper format:
% \begin{quote}
%   \verb|\PassOptionsToClass{a4paper}{article}|
% \end{quote}
% An example follows how to generate the
% documentation with pdf\LaTeX:
% \begin{quote}
%\begin{verbatim}
%pdflatex bigintcalc.dtx
%makeindex -s gind.ist bigintcalc.idx
%pdflatex bigintcalc.dtx
%makeindex -s gind.ist bigintcalc.idx
%pdflatex bigintcalc.dtx
%\end{verbatim}
% \end{quote}
%
% \section{Catalogue}
%
% The following XML file can be used as source for the
% \href{http://mirror.ctan.org/help/Catalogue/catalogue.html}{\TeX\ Catalogue}.
% The elements \texttt{caption} and \texttt{description} are imported
% from the original XML file from the Catalogue.
% The name of the XML file in the Catalogue is \xfile{bigintcalc.xml}.
%    \begin{macrocode}
%<*catalogue>
<?xml version='1.0' encoding='us-ascii'?>
<!DOCTYPE entry SYSTEM 'catalogue.dtd'>
<entry datestamp='$Date$' modifier='$Author$' id='bigintcalc'>
  <name>bigintcalc</name>
  <caption>Integer calculations on very large numbers.</caption>
  <authorref id='auth:oberdiek'/>
  <copyright owner='Heiko Oberdiek' year='2007,2011,2012'/>
  <license type='lppl1.3'/>
  <version number='1.4'/>
  <description>
    This package provides expandable arithmetic operations
    with big integers that can exceed TeX's number limits.
    <p/>
    The package is part of the <xref refid='oberdiek'>oberdiek</xref> bundle.
  </description>
  <documentation details='Package documentation'
      href='ctan:/macros/latex/contrib/oberdiek/bigintcalc.pdf'/>
  <ctan file='true' path='/macros/latex/contrib/oberdiek/bigintcalc.dtx'/>
  <miktex location='oberdiek'/>
  <texlive location='oberdiek'/>
  <install path='/macros/latex/contrib/oberdiek/oberdiek.tds.zip'/>
</entry>
%</catalogue>
%    \end{macrocode}
%
% \begin{History}
%   \begin{Version}{2007/09/27 v1.0}
%   \item
%     First version.
%   \end{Version}
%   \begin{Version}{2007/11/11 v1.1}
%   \item
%     Use of package \xpackage{pdftexcmds} for \LuaTeX\ support.
%   \end{Version}
%   \begin{Version}{2011/01/30 v1.2}
%   \item
%     Already loaded package files are not input in \hologo{plainTeX}.
%   \end{Version}
%   \begin{Version}{2012/04/08 v1.3}
%   \item
%     Fix: \xpackage{pdftexcmds} wasn't loaded in case of \hologo{LaTeX}.
%   \end{Version}
%   \begin{Version}{2016/05/16 v1.4}
%   \item
%     Documentation updates.
%   \end{Version}
% \end{History}
%
% \PrintIndex
%
% \Finale
\endinput

%        (quote the arguments according to the demands of your shell)
%
% Documentation:
%    (a) If bigintcalc.drv is present:
%           latex bigintcalc.drv
%    (b) Without bigintcalc.drv:
%           latex bigintcalc.dtx; ...
%    The class ltxdoc loads the configuration file ltxdoc.cfg
%    if available. Here you can specify further options, e.g.
%    use A4 as paper format:
%       \PassOptionsToClass{a4paper}{article}
%
%    Programm calls to get the documentation (example):
%       pdflatex bigintcalc.dtx
%       makeindex -s gind.ist bigintcalc.idx
%       pdflatex bigintcalc.dtx
%       makeindex -s gind.ist bigintcalc.idx
%       pdflatex bigintcalc.dtx
%
% Installation:
%    TDS:tex/generic/oberdiek/bigintcalc.sty
%    TDS:doc/latex/oberdiek/bigintcalc.pdf
%    TDS:doc/latex/oberdiek/test/bigintcalc-test1.tex
%    TDS:doc/latex/oberdiek/test/bigintcalc-test2.tex
%    TDS:doc/latex/oberdiek/test/bigintcalc-test3.tex
%    TDS:source/latex/oberdiek/bigintcalc.dtx
%
%<*ignore>
\begingroup
  \catcode123=1 %
  \catcode125=2 %
  \def\x{LaTeX2e}%
\expandafter\endgroup
\ifcase 0\ifx\install y1\fi\expandafter
         \ifx\csname processbatchFile\endcsname\relax\else1\fi
         \ifx\fmtname\x\else 1\fi\relax
\else\csname fi\endcsname
%</ignore>
%<*install>
\input docstrip.tex
\Msg{************************************************************************}
\Msg{* Installation}
\Msg{* Package: bigintcalc 2016/05/16 v1.4 Expandable calculations on big integers (HO)}
\Msg{************************************************************************}

\keepsilent
\askforoverwritefalse

\let\MetaPrefix\relax
\preamble

This is a generated file.

Project: bigintcalc
Version: 2016/05/16 v1.4

Copyright (C) 2007, 2011, 2012 by
   Heiko Oberdiek <heiko.oberdiek at googlemail.com>

This work may be distributed and/or modified under the
conditions of the LaTeX Project Public License, either
version 1.3c of this license or (at your option) any later
version. This version of this license is in
   http://www.latex-project.org/lppl/lppl-1-3c.txt
and the latest version of this license is in
   http://www.latex-project.org/lppl.txt
and version 1.3 or later is part of all distributions of
LaTeX version 2005/12/01 or later.

This work has the LPPL maintenance status "maintained".

This Current Maintainer of this work is Heiko Oberdiek.

The Base Interpreter refers to any `TeX-Format',
because some files are installed in TDS:tex/generic//.

This work consists of the main source file bigintcalc.dtx
and the derived files
   bigintcalc.sty, bigintcalc.pdf, bigintcalc.ins, bigintcalc.drv,
   bigintcalc-test1.tex, bigintcalc-test2.tex,
   bigintcalc-test3.tex.

\endpreamble
\let\MetaPrefix\DoubleperCent

\generate{%
  \file{bigintcalc.ins}{\from{bigintcalc.dtx}{install}}%
  \file{bigintcalc.drv}{\from{bigintcalc.dtx}{driver}}%
  \usedir{tex/generic/oberdiek}%
  \file{bigintcalc.sty}{\from{bigintcalc.dtx}{package}}%
%  \usedir{doc/latex/oberdiek/test}%
%  \file{bigintcalc-test1.tex}{\from{bigintcalc.dtx}{test1}}%
%  \file{bigintcalc-test2.tex}{\from{bigintcalc.dtx}{test2,etex}}%
%  \file{bigintcalc-test3.tex}{\from{bigintcalc.dtx}{test2,noetex}}%
  \nopreamble
  \nopostamble
  \usedir{source/latex/oberdiek/catalogue}%
  \file{bigintcalc.xml}{\from{bigintcalc.dtx}{catalogue}}%
}

\catcode32=13\relax% active space
\let =\space%
\Msg{************************************************************************}
\Msg{*}
\Msg{* To finish the installation you have to move the following}
\Msg{* file into a directory searched by TeX:}
\Msg{*}
\Msg{*     bigintcalc.sty}
\Msg{*}
\Msg{* To produce the documentation run the file `bigintcalc.drv'}
\Msg{* through LaTeX.}
\Msg{*}
\Msg{* Happy TeXing!}
\Msg{*}
\Msg{************************************************************************}

\endbatchfile
%</install>
%<*ignore>
\fi
%</ignore>
%<*driver>
\NeedsTeXFormat{LaTeX2e}
\ProvidesFile{bigintcalc.drv}%
  [2016/05/16 v1.4 Expandable calculations on big integers (HO)]%
\documentclass{ltxdoc}
\usepackage{wasysym}
\let\iint\relax
\let\iiint\relax
\usepackage[fleqn]{amsmath}

\DeclareMathOperator{\opInv}{Inv}
\DeclareMathOperator{\opAbs}{Abs}
\DeclareMathOperator{\opSgn}{Sgn}
\DeclareMathOperator{\opMin}{Min}
\DeclareMathOperator{\opMax}{Max}
\DeclareMathOperator{\opCmp}{Cmp}
\DeclareMathOperator{\opOdd}{Odd}
\DeclareMathOperator{\opInc}{Inc}
\DeclareMathOperator{\opDec}{Dec}
\DeclareMathOperator{\opAdd}{Add}
\DeclareMathOperator{\opSub}{Sub}
\DeclareMathOperator{\opShl}{Shl}
\DeclareMathOperator{\opShr}{Shr}
\DeclareMathOperator{\opMul}{Mul}
\DeclareMathOperator{\opSqr}{Sqr}
\DeclareMathOperator{\opFac}{Fac}
\DeclareMathOperator{\opPow}{Pow}
\DeclareMathOperator{\opDiv}{Div}
\DeclareMathOperator{\opMod}{Mod}
\DeclareMathOperator{\opInt}{Int}
\DeclareMathOperator{\opODD}{ifodd}

\newcommand*{\Def}{%
  \ensuremath{%
    \mathrel{\mathop{:}}=%
  }%
}
\newcommand*{\op}[1]{%
  \textsf{#1}%
}
\usepackage{holtxdoc}[2011/11/22]
\begin{document}
  \DocInput{bigintcalc.dtx}%
\end{document}
%</driver>
% \fi
%
%
% \CharacterTable
%  {Upper-case    \A\B\C\D\E\F\G\H\I\J\K\L\M\N\O\P\Q\R\S\T\U\V\W\X\Y\Z
%   Lower-case    \a\b\c\d\e\f\g\h\i\j\k\l\m\n\o\p\q\r\s\t\u\v\w\x\y\z
%   Digits        \0\1\2\3\4\5\6\7\8\9
%   Exclamation   \!     Double quote  \"     Hash (number) \#
%   Dollar        \$     Percent       \%     Ampersand     \&
%   Acute accent  \'     Left paren    \(     Right paren   \)
%   Asterisk      \*     Plus          \+     Comma         \,
%   Minus         \-     Point         \.     Solidus       \/
%   Colon         \:     Semicolon     \;     Less than     \<
%   Equals        \=     Greater than  \>     Question mark \?
%   Commercial at \@     Left bracket  \[     Backslash     \\
%   Right bracket \]     Circumflex    \^     Underscore    \_
%   Grave accent  \`     Left brace    \{     Vertical bar  \|
%   Right brace   \}     Tilde         \~}
%
% \GetFileInfo{bigintcalc.drv}
%
% \title{The \xpackage{bigintcalc} package}
% \date{2016/05/16 v1.4}
% \author{Heiko Oberdiek\thanks
% {Please report any issues at https://github.com/ho-tex/oberdiek/issues}\\
% \xemail{heiko.oberdiek at googlemail.com}}
%
% \maketitle
%
% \begin{abstract}
% This package provides expandable arithmetic operations
% with big integers that can exceed \TeX's number limits.
% \end{abstract}
%
% \tableofcontents
%
% \section{Documentation}
%
% \subsection{Introduction}
%
% Package \xpackage{bigintcalc} defines arithmetic operations
% that deal with big integers. Big integers can be given
% either as explicit integer number or as macro code
% that expands to an explicit number. \emph{Big} means that
% there is no limit on the size of the number. Big integers
% may exceed \TeX's range limitation of -2147483647 and 2147483647.
% Only memory issues will limit the usable range.
%
% In opposite to package \xpackage{intcalc}
% unexpandable command tokens are not supported, even if
% they are valid \TeX\ numbers like count registers or
% commands created by \cs{chardef}. Nevertheless they may
% be used, if they are prefixed by \cs{number}.
%
% Also \eTeX's \cs{numexpr} expressions are not supported
% directly in the manner of package \xpackage{intcalc}.
% However they can be given if \cs{the}\cs{numexpr}
% or \cs{number}\cs{numexpr} are used.
%
% The operations have the form of macros that take one or
% two integers as parameter and return the integer result.
% The macro name is a three letter operation name prefixed
% by the package name, e.g. \cs{bigintcalcAdd}|{10}{43}| returns
% |53|.
%
% The macros are fully expandable, exactly two expansion
% steps generate the result. Therefore the operations
% may be used nearly everywhere in \TeX, even inside
% \cs{csname}, file names,  or other
% expandable contexts.
%
% \subsection{Conditions}
%
% \subsubsection{Preconditions}
%
% \begin{itemize}
% \item
%   Arguments can be anything that expands to a number that consists
%   of optional signs and digits.
% \item
%   The arguments and return values must be sound.
%   Zero as divisor or factorials of negative numbers will cause errors.
% \end{itemize}
%
% \subsubsection{Postconditions}
%
% Additional properties of the macros apart from calculating
% a correct result (of course \smiley):
% \begin{itemize}
% \item
%   The macros are fully expandable. Thus they can
%   be used inside \cs{edef}, \cs{csname},
%   for example.
% \item
%   Furthermore exactly two expansion steps calculate the result.
% \item
%   The number consists of one optional minus sign and one or more
%   digits. The first digit is larger than zero for
%   numbers that consists of more than one digit.
%
%   In short, the number format is exactly the same as
%   \cs{number} generates, but without its range limitation.
%   And the tokens (minus sign, digits)
%   have catcode 12 (other).
% \item
%   Call by value is simulated. First the arguments are
%   converted to numbers. Then these numbers are used
%   in the calculations.
%
%   Remember that arguments
%   may contain expensive macros or \eTeX\ expressions.
%   This strategy avoids multiple evaluations of such
%   arguments.
% \end{itemize}
%
% \subsection{Error handling}
%
% Some errors are detected by the macros, example: division by zero.
% In this cases an undefined control sequence is called and causes
% a TeX error message, example: \cs{BigIntCalcError:DivisionByZero}.
% The name of the control sequence contains
% the reason for the error. The \TeX\ error may be ignored.
% Then the operation returns zero as result.
% Because the macros are supposed to work in expandible contexts.
% An traditional error message, however, is not expandable and
% would break these contexts.
%
% \subsection{Operations}
%
% Some definition equations below use the function $\opInt$
% that converts a real number to an integer. The number
% is truncated that means rounding to zero:
% \begin{gather*}
%   \opInt(x) \Def
%   \begin{cases}
%     \lfloor x\rfloor & \text{if $x\geq0$}\\
%     \lceil x\rceil & \text{otherwise}
%   \end{cases}
% \end{gather*}
%
% \subsubsection{\op{Num}}
%
% \begin{declcs}{bigintcalcNum} \M{x}
% \end{declcs}
% Macro \cs{bigintcalcNum} converts its argument to a normalized integer
% number without unnecessary leading zeros or signs.
% The result matches the regular expression:
%\begin{quote}
%\begin{verbatim}
%0|-?[1-9][0-9]*
%\end{verbatim}
%\end{quote}
%
% \subsubsection{\op{Inv}, \op{Abs}, \op{Sgn}}
%
% \begin{declcs}{bigintcalcInv} \M{x}
% \end{declcs}
% Macro \cs{bigintcalcInv} switches the sign.
% \begin{gather*}
%   \opInv(x) \Def -x
% \end{gather*}
%
% \begin{declcs}{bigintcalcAbs} \M{x}
% \end{declcs}
% Macro \cs{bigintcalcAbs} returns the absolute value
% of integer \meta{x}.
% \begin{gather*}
%   \opAbs(x) \Def \vert x\vert
% \end{gather*}
%
% \begin{declcs}{bigintcalcSgn} \M{x}
% \end{declcs}
% Macro \cs{bigintcalcSgn} encodes the sign of \meta{x} as number.
% \begin{gather*}
%   \opSgn(x) \Def
%   \begin{cases}
%     -1& \text{if $x<0$}\\
%      0& \text{if $x=0$}\\
%      1& \text{if $x>0$}
%   \end{cases}
% \end{gather*}
% These return values can easily be distinguished by \cs{ifcase}:
%\begin{quote}
%\begin{verbatim}
%\ifcase\bigintcalcSgn{<x>}
%  $x=0$
%\or
%  $x>0$
%\else
%  $x<0$
%\fi
%\end{verbatim}
%\end{quote}
%
% \subsubsection{\op{Min}, \op{Max}, \op{Cmp}}
%
% \begin{declcs}{bigintcalcMin} \M{x} \M{y}
% \end{declcs}
% Macro \cs{bigintcalcMin} returns the smaller of the two integers.
% \begin{gather*}
%   \opMin(x,y) \Def
%   \begin{cases}
%     x & \text{if $x<y$}\\
%     y & \text{otherwise}
%   \end{cases}
% \end{gather*}
%
% \begin{declcs}{bigintcalcMax} \M{x} \M{y}
% \end{declcs}
% Macro \cs{bigintcalcMax} returns the larger of the two integers.
% \begin{gather*}
%   \opMax(x,y) \Def
%   \begin{cases}
%     x & \text{if $x>y$}\\
%     y & \text{otherwise}
%   \end{cases}
% \end{gather*}
%
% \begin{declcs}{bigintcalcCmp} \M{x} \M{y}
% \end{declcs}
% Macro \cs{bigintcalcCmp} encodes the comparison result as number:
% \begin{gather*}
%   \opCmp(x,y) \Def
%   \begin{cases}
%     -1 & \text{if $x<y$}\\
%      0 & \text{if $x=y$}\\
%      1 & \text{if $x>y$}
%   \end{cases}
% \end{gather*}
% These values can be distinguished by \cs{ifcase}:
%\begin{quote}
%\begin{verbatim}
%\ifcase\bigintcalcCmp{<x>}{<y>}
%  $x=y$
%\or
%  $x>y$
%\else
%  $x<y$
%\fi
%\end{verbatim}
%\end{quote}
%
% \subsubsection{\op{Odd}}
%
% \begin{declcs}{bigintcalcOdd} \M{x}
% \end{declcs}
% \begin{gather*}
%   \opOdd(x) \Def
%   \begin{cases}
%     1 & \text{if $x$ is odd}\\
%     0 & \text{if $x$ is even}
%   \end{cases}
% \end{gather*}
%
% \subsubsection{\op{Inc}, \op{Dec}, \op{Add}, \op{Sub}}
%
% \begin{declcs}{bigintcalcInc} \M{x}
% \end{declcs}
% Macro \cs{bigintcalcInc} increments \meta{x} by one.
% \begin{gather*}
%   \opInc(x) \Def x + 1
% \end{gather*}
%
% \begin{declcs}{bigintcalcDec} \M{x}
% \end{declcs}
% Macro \cs{bigintcalcDec} decrements \meta{x} by one.
% \begin{gather*}
%   \opDec(x) \Def x - 1
% \end{gather*}
%
% \begin{declcs}{bigintcalcAdd} \M{x} \M{y}
% \end{declcs}
% Macro \cs{bigintcalcAdd} adds the two numbers.
% \begin{gather*}
%   \opAdd(x, y) \Def x + y
% \end{gather*}
%
% \begin{declcs}{bigintcalcSub} \M{x} \M{y}
% \end{declcs}
% Macro \cs{bigintcalcSub} calculates the difference.
% \begin{gather*}
%   \opSub(x, y) \Def x - y
% \end{gather*}
%
% \subsubsection{\op{Shl}, \op{Shr}}
%
% \begin{declcs}{bigintcalcShl} \M{x}
% \end{declcs}
% Macro \cs{bigintcalcShl} implements shifting to the left that
% means the number is multiplied by two.
% The sign is preserved.
% \begin{gather*}
%   \opShl(x) \Def x*2
% \end{gather*}
%
% \begin{declcs}{bigintcalcShr} \M{x}
% \end{declcs}
% Macro \cs{bigintcalcShr} implements shifting to the right.
% That is equivalent to an integer division by two.
% The sign is preserved.
% \begin{gather*}
%   \opShr(x) \Def \opInt(x/2)
% \end{gather*}
%
% \subsubsection{\op{Mul}, \op{Sqr}, \op{Fac}, \op{Pow}}
%
% \begin{declcs}{bigintcalcMul} \M{x} \M{y}
% \end{declcs}
% Macro \cs{bigintcalcMul} calculates the product of
% \meta{x} and \meta{y}.
% \begin{gather*}
%   \opMul(x,y) \Def x*y
% \end{gather*}
%
% \begin{declcs}{bigintcalcSqr} \M{x}
% \end{declcs}
% Macro \cs{bigintcalcSqr} returns the square product.
% \begin{gather*}
%   \opSqr(x) \Def x^2
% \end{gather*}
%
% \begin{declcs}{bigintcalcFac} \M{x}
% \end{declcs}
% Macro \cs{bigintcalcFac} returns the factorial of \meta{x}.
% Negative numbers are not permitted.
% \begin{gather*}
%   \opFac(x) \Def x!\qquad\text{for $x\geq0$}
% \end{gather*}
% ($0! = 1$)
%
% \begin{declcs}{bigintcalcPow} M{x} M{y}
% \end{declcs}
% Macro \cs{bigintcalcPow} calculates the value of \meta{x} to the
% power of \meta{y}. The error ``division by zero'' is thrown
% if \meta{x} is zero and \meta{y} is negative.
% permitted:
% \begin{gather*}
%   \opPow(x,y) \Def
%   \opInt(x^y)\qquad\text{for $x\neq0$ or $y\geq0$}
% \end{gather*}
% ($0^0 = 1$)
%
% \subsubsection{\op{Div}, \op{Mul}}
%
% \begin{declcs}{bigintcalcDiv} \M{x} \M{y}
% \end{declcs}
% Macro \cs{bigintcalcDiv} performs an integer division.
% Argument \meta{y} must not be zero.
% \begin{gather*}
%   \opDiv(x,y) \Def \opInt(x/y)\qquad\text{for $y\neq0$}
% \end{gather*}
%
% \begin{declcs}{bigintcalcMod} \M{x} \M{y}
% \end{declcs}
% Macro \cs{bigintcalcMod} gets the remainder of the integer
% division. The sign follows the divisor \meta{y}.
% Argument \meta{y} must not be zero.
% \begin{gather*}
%   \opMod(x,y) \Def x\mathrel{\%}y\qquad\text{for $y\neq0$}
% \end{gather*}
% The result ranges:
% \begin{gather*}
%   -\vert y\vert < \opMod(x,y) \leq 0\qquad\text{for $y<0$}\\
%   0 \leq \opMod(x,y) < y\qquad\text{for $y\geq0$}
% \end{gather*}
%
% \subsection{Interface for programmers}
%
% If the programmer can ensure some more properties about
% the arguments of the operations, then the following
% macros are a little more efficient.
%
% In general numbers must obey the following constraints:
% \begin{itemize}
% \item Plain number: digit tokens only, no command tokens.
% \item Non-negative. Signs are forbidden.
% \item Delimited by exclamation mark. Curly braces
%       around the number are not allowed and will
%       break the code.
% \end{itemize}
%
% \begin{declcs}{BigIntCalcOdd} \meta{number} |!|
% \end{declcs}
% |1|/|0| is returned if \meta{number} is odd/even.
%
% \begin{declcs}{BigIntCalcInc} \meta{number} |!|
% \end{declcs}
% Incrementation.
%
% \begin{declcs}{BigIntCalcDec} \meta{number} |!|
% \end{declcs}
% Decrementation, positive number without zero.
%
% \begin{declcs}{BigIntCalcAdd} \meta{number A} |!| \meta{number B} |!|
% \end{declcs}
% Addition, $A\geq B$.
%
% \begin{declcs}{BigIntCalcSub} \meta{number A} |!| \meta{number B} |!|
% \end{declcs}
% Subtraction, $A\geq B$.
%
% \begin{declcs}{BigIntCalcShl} \meta{number} |!|
% \end{declcs}
% Left shift (multiplication with two).
%
% \begin{declcs}{BigIntCalcShr} \meta{number} |!|
% \end{declcs}
% Right shift (integer division by two).
%
% \begin{declcs}{BigIntCalcMul} \meta{number A} |!| \meta{number B} |!|
% \end{declcs}
% Multiplication, $A\geq B$.
%
% \begin{declcs}{BigIntCalcDiv} \meta{number A} |!| \meta{number B} |!|
% \end{declcs}
% Division operation.
%
% \begin{declcs}{BigIntCalcMod} \meta{number A} |!| \meta{number B} |!|
% \end{declcs}
% Modulo operation.
%
%
% \StopEventually{
% }
%
% \section{Implementation}
%
%    \begin{macrocode}
%<*package>
%    \end{macrocode}
%
% \subsection{Reload check and package identification}
%    Reload check, especially if the package is not used with \LaTeX.
%    \begin{macrocode}
\begingroup\catcode61\catcode48\catcode32=10\relax%
  \catcode13=5 % ^^M
  \endlinechar=13 %
  \catcode35=6 % #
  \catcode39=12 % '
  \catcode44=12 % ,
  \catcode45=12 % -
  \catcode46=12 % .
  \catcode58=12 % :
  \catcode64=11 % @
  \catcode123=1 % {
  \catcode125=2 % }
  \expandafter\let\expandafter\x\csname ver@bigintcalc.sty\endcsname
  \ifx\x\relax % plain-TeX, first loading
  \else
    \def\empty{}%
    \ifx\x\empty % LaTeX, first loading,
      % variable is initialized, but \ProvidesPackage not yet seen
    \else
      \expandafter\ifx\csname PackageInfo\endcsname\relax
        \def\x#1#2{%
          \immediate\write-1{Package #1 Info: #2.}%
        }%
      \else
        \def\x#1#2{\PackageInfo{#1}{#2, stopped}}%
      \fi
      \x{bigintcalc}{The package is already loaded}%
      \aftergroup\endinput
    \fi
  \fi
\endgroup%
%    \end{macrocode}
%    Package identification:
%    \begin{macrocode}
\begingroup\catcode61\catcode48\catcode32=10\relax%
  \catcode13=5 % ^^M
  \endlinechar=13 %
  \catcode35=6 % #
  \catcode39=12 % '
  \catcode40=12 % (
  \catcode41=12 % )
  \catcode44=12 % ,
  \catcode45=12 % -
  \catcode46=12 % .
  \catcode47=12 % /
  \catcode58=12 % :
  \catcode64=11 % @
  \catcode91=12 % [
  \catcode93=12 % ]
  \catcode123=1 % {
  \catcode125=2 % }
  \expandafter\ifx\csname ProvidesPackage\endcsname\relax
    \def\x#1#2#3[#4]{\endgroup
      \immediate\write-1{Package: #3 #4}%
      \xdef#1{#4}%
    }%
  \else
    \def\x#1#2[#3]{\endgroup
      #2[{#3}]%
      \ifx#1\@undefined
        \xdef#1{#3}%
      \fi
      \ifx#1\relax
        \xdef#1{#3}%
      \fi
    }%
  \fi
\expandafter\x\csname ver@bigintcalc.sty\endcsname
\ProvidesPackage{bigintcalc}%
  [2016/05/16 v1.4 Expandable calculations on big integers (HO)]%
%    \end{macrocode}
%
% \subsection{Catcodes}
%
%    \begin{macrocode}
\begingroup\catcode61\catcode48\catcode32=10\relax%
  \catcode13=5 % ^^M
  \endlinechar=13 %
  \catcode123=1 % {
  \catcode125=2 % }
  \catcode64=11 % @
  \def\x{\endgroup
    \expandafter\edef\csname BIC@AtEnd\endcsname{%
      \endlinechar=\the\endlinechar\relax
      \catcode13=\the\catcode13\relax
      \catcode32=\the\catcode32\relax
      \catcode35=\the\catcode35\relax
      \catcode61=\the\catcode61\relax
      \catcode64=\the\catcode64\relax
      \catcode123=\the\catcode123\relax
      \catcode125=\the\catcode125\relax
    }%
  }%
\x\catcode61\catcode48\catcode32=10\relax%
\catcode13=5 % ^^M
\endlinechar=13 %
\catcode35=6 % #
\catcode64=11 % @
\catcode123=1 % {
\catcode125=2 % }
\def\TMP@EnsureCode#1#2{%
  \edef\BIC@AtEnd{%
    \BIC@AtEnd
    \catcode#1=\the\catcode#1\relax
  }%
  \catcode#1=#2\relax
}
\TMP@EnsureCode{33}{12}% !
\TMP@EnsureCode{36}{14}% $ (comment!)
\TMP@EnsureCode{38}{14}% & (comment!)
\TMP@EnsureCode{40}{12}% (
\TMP@EnsureCode{41}{12}% )
\TMP@EnsureCode{42}{12}% *
\TMP@EnsureCode{43}{12}% +
\TMP@EnsureCode{45}{12}% -
\TMP@EnsureCode{46}{12}% .
\TMP@EnsureCode{47}{12}% /
\TMP@EnsureCode{58}{11}% : (letter!)
\TMP@EnsureCode{60}{12}% <
\TMP@EnsureCode{62}{12}% >
\TMP@EnsureCode{63}{14}% ? (comment!)
\TMP@EnsureCode{91}{12}% [
\TMP@EnsureCode{93}{12}% ]
\edef\BIC@AtEnd{\BIC@AtEnd\noexpand\endinput}
\begingroup\expandafter\expandafter\expandafter\endgroup
\expandafter\ifx\csname BIC@TestMode\endcsname\relax
\else
  \catcode63=9 % ? (ignore)
\fi
? \let\BIC@@TestMode\BIC@TestMode
%    \end{macrocode}
%
% \subsection{\eTeX\ detection}
%
%    \begin{macrocode}
\begingroup\expandafter\expandafter\expandafter\endgroup
\expandafter\ifx\csname numexpr\endcsname\relax
  \catcode36=9 % $ (ignore)
\else
  \catcode38=9 % & (ignore)
\fi
%    \end{macrocode}
%
% \subsection{Help macros}
%
%    \begin{macro}{\BIC@Fi}
%    \begin{macrocode}
\let\BIC@Fi\fi
%    \end{macrocode}
%    \end{macro}
%    \begin{macro}{\BIC@AfterFi}
%    \begin{macrocode}
\def\BIC@AfterFi#1#2\BIC@Fi{\fi#1}%
%    \end{macrocode}
%    \end{macro}
%    \begin{macro}{\BIC@AfterFiFi}
%    \begin{macrocode}
\def\BIC@AfterFiFi#1#2\BIC@Fi{\fi\fi#1}%
%    \end{macrocode}
%    \end{macro}
%    \begin{macro}{\BIC@AfterFiFiFi}
%    \begin{macrocode}
\def\BIC@AfterFiFiFi#1#2\BIC@Fi{\fi\fi\fi#1}%
%    \end{macrocode}
%    \end{macro}
%
%    \begin{macro}{\BIC@Space}
%    \begin{macrocode}
\begingroup
  \def\x#1{\endgroup
    \let\BIC@Space= #1%
  }%
\x{ }
%    \end{macrocode}
%    \end{macro}
%
% \subsection{Expand number}
%
%    \begin{macrocode}
\begingroup\expandafter\expandafter\expandafter\endgroup
\expandafter\ifx\csname RequirePackage\endcsname\relax
  \def\TMP@RequirePackage#1[#2]{%
    \begingroup\expandafter\expandafter\expandafter\endgroup
    \expandafter\ifx\csname ver@#1.sty\endcsname\relax
      \input #1.sty\relax
    \fi
  }%
  \TMP@RequirePackage{pdftexcmds}[2007/11/11]%
\else
  \RequirePackage{pdftexcmds}[2007/11/11]%
\fi
%    \end{macrocode}
%
%    \begin{macrocode}
\begingroup\expandafter\expandafter\expandafter\endgroup
\expandafter\ifx\csname pdf@escapehex\endcsname\relax
%    \end{macrocode}
%
%    \begin{macro}{\BIC@Expand}
%    \begin{macrocode}
  \def\BIC@Expand#1{%
    \romannumeral0%
    \BIC@@Expand#1!\@nil{}%
  }%
%    \end{macrocode}
%    \end{macro}
%    \begin{macro}{\BIC@@Expand}
%    \begin{macrocode}
  \def\BIC@@Expand#1#2\@nil#3{%
    \expandafter\ifcat\noexpand#1\relax
      \expandafter\@firstoftwo
    \else
      \expandafter\@secondoftwo
    \fi
    {%
      \expandafter\BIC@@Expand#1#2\@nil{#3}%
    }{%
      \ifx#1!%
        \expandafter\@firstoftwo
      \else
        \expandafter\@secondoftwo
      \fi
      { #3}{%
        \BIC@@Expand#2\@nil{#3#1}%
      }%
    }%
  }%
%    \end{macrocode}
%    \end{macro}
%    \begin{macro}{\@firstoftwo}
%    \begin{macrocode}
  \expandafter\ifx\csname @firstoftwo\endcsname\relax
    \long\def\@firstoftwo#1#2{#1}%
  \fi
%    \end{macrocode}
%    \end{macro}
%    \begin{macro}{\@secondoftwo}
%    \begin{macrocode}
  \expandafter\ifx\csname @secondoftwo\endcsname\relax
    \long\def\@secondoftwo#1#2{#2}%
  \fi
%    \end{macrocode}
%    \end{macro}
%
%    \begin{macrocode}
\else
%    \end{macrocode}
%    \begin{macro}{\BIC@Expand}
%    \begin{macrocode}
  \def\BIC@Expand#1{%
    \romannumeral0\expandafter\expandafter\expandafter\BIC@Space
    \pdf@unescapehex{%
      \expandafter\expandafter\expandafter
      \BIC@StripHexSpace\pdf@escapehex{#1}20\@nil
    }%
  }%
%    \end{macrocode}
%    \end{macro}
%    \begin{macro}{\BIC@StripHexSpace}
%    \begin{macrocode}
  \def\BIC@StripHexSpace#120#2\@nil{%
    #1%
    \ifx\\#2\\%
    \else
      \BIC@AfterFi{%
        \BIC@StripHexSpace#2\@nil
      }%
    \BIC@Fi
  }%
%    \end{macrocode}
%    \end{macro}
%    \begin{macrocode}
\fi
%    \end{macrocode}
%
% \subsection{Normalize expanded number}
%
%    \begin{macro}{\BIC@Normalize}
%    |#1|: result sign\\
%    |#2|: first token of number
%    \begin{macrocode}
\def\BIC@Normalize#1#2{%
  \ifx#2-%
    \ifx\\#1\\%
      \BIC@AfterFiFi{%
        \BIC@Normalize-%
      }%
    \else
      \BIC@AfterFiFi{%
        \BIC@Normalize{}%
      }%
    \fi
  \else
    \ifx#2+%
      \BIC@AfterFiFi{%
        \BIC@Normalize{#1}%
      }%
    \else
      \ifx#20%
        \BIC@AfterFiFiFi{%
          \BIC@NormalizeZero{#1}%
        }%
      \else
        \BIC@AfterFiFiFi{%
          \BIC@NormalizeDigits#1#2%
        }%
      \fi
    \fi
  \BIC@Fi
}
%    \end{macrocode}
%    \end{macro}
%    \begin{macro}{\BIC@NormalizeZero}
%    \begin{macrocode}
\def\BIC@NormalizeZero#1#2{%
  \ifx#2!%
    \BIC@AfterFi{ 0}%
  \else
    \ifx#20%
      \BIC@AfterFiFi{%
        \BIC@NormalizeZero{#1}%
      }%
    \else
      \BIC@AfterFiFi{%
        \BIC@NormalizeDigits#1#2%
      }%
    \fi
  \BIC@Fi
}
%    \end{macrocode}
%    \end{macro}
%    \begin{macro}{\BIC@NormalizeDigits}
%    \begin{macrocode}
\def\BIC@NormalizeDigits#1!{ #1}
%    \end{macrocode}
%    \end{macro}
%
% \subsection{\op{Num}}
%
%    \begin{macro}{\bigintcalcNum}
%    \begin{macrocode}
\def\bigintcalcNum#1{%
  \romannumeral0%
  \expandafter\expandafter\expandafter\BIC@Normalize
  \expandafter\expandafter\expandafter{%
  \expandafter\expandafter\expandafter}%
  \BIC@Expand{#1}!%
}
%    \end{macrocode}
%    \end{macro}
%
% \subsection{\op{Inv}, \op{Abs}, \op{Sgn}}
%
%
%    \begin{macro}{\bigintcalcInv}
%    \begin{macrocode}
\def\bigintcalcInv#1{%
  \romannumeral0\expandafter\expandafter\expandafter\BIC@Space
  \bigintcalcNum{-#1}%
}
%    \end{macrocode}
%    \end{macro}
%
%    \begin{macro}{\bigintcalcAbs}
%    \begin{macrocode}
\def\bigintcalcAbs#1{%
  \romannumeral0%
  \expandafter\expandafter\expandafter\BIC@Abs
  \bigintcalcNum{#1}%
}
%    \end{macrocode}
%    \end{macro}
%    \begin{macro}{\BIC@Abs}
%    \begin{macrocode}
\def\BIC@Abs#1{%
  \ifx#1-%
    \expandafter\BIC@Space
  \else
    \expandafter\BIC@Space
    \expandafter#1%
  \fi
}
%    \end{macrocode}
%    \end{macro}
%
%    \begin{macro}{\bigintcalcSgn}
%    \begin{macrocode}
\def\bigintcalcSgn#1{%
  \number
  \expandafter\expandafter\expandafter\BIC@Sgn
  \bigintcalcNum{#1}! %
}
%    \end{macrocode}
%    \end{macro}
%    \begin{macro}{\BIC@Sgn}
%    \begin{macrocode}
\def\BIC@Sgn#1#2!{%
  \ifx#1-%
    -1%
  \else
    \ifx#10%
      0%
    \else
      1%
    \fi
  \fi
}
%    \end{macrocode}
%    \end{macro}
%
% \subsection{\op{Cmp}, \op{Min}, \op{Max}}
%
%    \begin{macro}{\bigintcalcCmp}
%    \begin{macrocode}
\def\bigintcalcCmp#1#2{%
  \number
  \expandafter\expandafter\expandafter\BIC@Cmp
  \bigintcalcNum{#2}!{#1}%
}
%    \end{macrocode}
%    \end{macro}
%    \begin{macro}{\BIC@Cmp}
%    \begin{macrocode}
\def\BIC@Cmp#1!#2{%
  \expandafter\expandafter\expandafter\BIC@@Cmp
  \bigintcalcNum{#2}!#1!%
}
%    \end{macrocode}
%    \end{macro}
%    \begin{macro}{\BIC@@Cmp}
%    \begin{macrocode}
\def\BIC@@Cmp#1#2!#3#4!{%
  \ifx#1-%
    \ifx#3-%
      \BIC@AfterFiFi{%
        \BIC@@Cmp#4!#2!%
      }%
    \else
      \BIC@AfterFiFi{%
        -1 %
      }%
    \fi
  \else
    \ifx#3-%
      \BIC@AfterFiFi{%
        1 %
      }%
    \else
      \BIC@AfterFiFi{%
        \BIC@CmpLength#1#2!#3#4!#1#2!#3#4!%
      }%
    \fi
  \BIC@Fi
}
%    \end{macrocode}
%    \end{macro}
%    \begin{macro}{\BIC@PosCmp}
%    \begin{macrocode}
\def\BIC@PosCmp#1!#2!{%
  \BIC@CmpLength#1!#2!#1!#2!%
}
%    \end{macrocode}
%    \end{macro}
%    \begin{macro}{\BIC@CmpLength}
%    \begin{macrocode}
\def\BIC@CmpLength#1#2!#3#4!{%
  \ifx\\#2\\%
    \ifx\\#4\\%
      \BIC@AfterFiFi\BIC@CmpDiff
    \else
      \BIC@AfterFiFi{%
        \BIC@CmpResult{-1}%
      }%
    \fi
  \else
    \ifx\\#4\\%
      \BIC@AfterFiFi{%
        \BIC@CmpResult1%
      }%
    \else
      \BIC@AfterFiFi{%
        \BIC@CmpLength#2!#4!%
      }%
    \fi
  \BIC@Fi
}
%    \end{macrocode}
%    \end{macro}
%    \begin{macro}{\BIC@CmpResult}
%    \begin{macrocode}
\def\BIC@CmpResult#1#2!#3!{#1 }
%    \end{macrocode}
%    \end{macro}
%    \begin{macro}{\BIC@CmpDiff}
%    \begin{macrocode}
\def\BIC@CmpDiff#1#2!#3#4!{%
  \ifnum#1<#3 %
    \BIC@AfterFi{%
      -1 %
    }%
  \else
    \ifnum#1>#3 %
      \BIC@AfterFiFi{%
        1 %
      }%
    \else
      \ifx\\#2\\%
        \BIC@AfterFiFiFi{%
          0 %
        }%
      \else
        \BIC@AfterFiFiFi{%
          \BIC@CmpDiff#2!#4!%
        }%
      \fi
    \fi
  \BIC@Fi
}
%    \end{macrocode}
%    \end{macro}
%
%    \begin{macro}{\bigintcalcMin}
%    \begin{macrocode}
\def\bigintcalcMin#1{%
  \romannumeral0%
  \expandafter\expandafter\expandafter\BIC@MinMax
  \bigintcalcNum{#1}!-!%
}
%    \end{macrocode}
%    \end{macro}
%    \begin{macro}{\bigintcalcMax}
%    \begin{macrocode}
\def\bigintcalcMax#1{%
  \romannumeral0%
  \expandafter\expandafter\expandafter\BIC@MinMax
  \bigintcalcNum{#1}!!%
}
%    \end{macrocode}
%    \end{macro}
%    \begin{macro}{\BIC@MinMax}
%    |#1|: $x$\\
%    |#2|: sign for comparison\\
%    |#3|: $y$
%    \begin{macrocode}
\def\BIC@MinMax#1!#2!#3{%
  \expandafter\expandafter\expandafter\BIC@@MinMax
  \bigintcalcNum{#3}!#1!#2!%
}
%    \end{macrocode}
%    \end{macro}
%    \begin{macro}{\BIC@@MinMax}
%    |#1|: $y$\\
%    |#2|: $x$\\
%    |#3|: sign for comparison
%    \begin{macrocode}
\def\BIC@@MinMax#1!#2!#3!{%
  \ifnum\BIC@@Cmp#1!#2!=#31 %
    \BIC@AfterFi{ #1}%
  \else
    \BIC@AfterFi{ #2}%
  \BIC@Fi
}
%    \end{macrocode}
%    \end{macro}
%
% \subsection{\op{Odd}}
%
%    \begin{macro}{\bigintcalcOdd}
%    \begin{macrocode}
\def\bigintcalcOdd#1{%
  \romannumeral0%
  \expandafter\expandafter\expandafter\BIC@Odd
  \bigintcalcAbs{#1}!%
}
%    \end{macrocode}
%    \end{macro}
%    \begin{macro}{\BigIntCalcOdd}
%    \begin{macrocode}
\def\BigIntCalcOdd#1!{%
  \romannumeral0%
  \BIC@Odd#1!%
}
%    \end{macrocode}
%    \end{macro}
%    \begin{macro}{\BIC@Odd}
%    |#1|: $x$
%    \begin{macrocode}
\def\BIC@Odd#1#2{%
  \ifx#2!%
    \ifodd#1 %
      \BIC@AfterFiFi{ 1}%
    \else
      \BIC@AfterFiFi{ 0}%
    \fi
  \else
    \expandafter\BIC@Odd\expandafter#2%
  \BIC@Fi
}
%    \end{macrocode}
%    \end{macro}
%
% \subsection{\op{Inc}, \op{Dec}}
%
%    \begin{macro}{\bigintcalcInc}
%    \begin{macrocode}
\def\bigintcalcInc#1{%
  \romannumeral0%
  \expandafter\expandafter\expandafter\BIC@IncSwitch
  \bigintcalcNum{#1}!%
}
%    \end{macrocode}
%    \end{macro}
%    \begin{macro}{\BIC@IncSwitch}
%    \begin{macrocode}
\def\BIC@IncSwitch#1#2!{%
  \ifcase\BIC@@Cmp#1#2!-1!%
    \BIC@AfterFi{ 0}%
  \or
    \BIC@AfterFi{%
      \BIC@Inc#1#2!{}%
    }%
  \else
    \BIC@AfterFi{%
      \expandafter-\romannumeral0%
      \BIC@Dec#2!{}%
    }%
  \BIC@Fi
}
%    \end{macrocode}
%    \end{macro}
%
%    \begin{macro}{\bigintcalcDec}
%    \begin{macrocode}
\def\bigintcalcDec#1{%
  \romannumeral0%
  \expandafter\expandafter\expandafter\BIC@DecSwitch
  \bigintcalcNum{#1}!%
}
%    \end{macrocode}
%    \end{macro}
%    \begin{macro}{\BIC@DecSwitch}
%    \begin{macrocode}
\def\BIC@DecSwitch#1#2!{%
  \ifcase\BIC@Sgn#1#2! %
    \BIC@AfterFi{ -1}%
  \or
    \BIC@AfterFi{%
      \BIC@Dec#1#2!{}%
    }%
  \else
    \BIC@AfterFi{%
      \expandafter-\romannumeral0%
      \BIC@Inc#2!{}%
    }%
  \BIC@Fi
}
%    \end{macrocode}
%    \end{macro}
%
%    \begin{macro}{\BigIntCalcInc}
%    \begin{macrocode}
\def\BigIntCalcInc#1!{%
  \romannumeral0\BIC@Inc#1!{}%
}
%    \end{macrocode}
%    \end{macro}
%    \begin{macro}{\BigIntCalcDec}
%    \begin{macrocode}
\def\BigIntCalcDec#1!{%
  \romannumeral0\BIC@Dec#1!{}%
}
%    \end{macrocode}
%    \end{macro}
%
%    \begin{macro}{\BIC@Inc}
%    \begin{macrocode}
\def\BIC@Inc#1#2!#3{%
  \ifx\\#2\\%
    \BIC@AfterFi{%
      \BIC@@Inc1#1#3!{}%
    }%
  \else
    \BIC@AfterFi{%
      \BIC@Inc#2!{#1#3}%
    }%
  \BIC@Fi
}
%    \end{macrocode}
%    \end{macro}
%    \begin{macro}{\BIC@@Inc}
%    \begin{macrocode}
\def\BIC@@Inc#1#2#3!#4{%
  \ifcase#1 %
    \ifx\\#3\\%
      \BIC@AfterFiFi{ #2#4}%
    \else
      \BIC@AfterFiFi{%
        \BIC@@Inc0#3!{#2#4}%
      }%
    \fi
  \else
    \ifnum#2<9 %
      \BIC@AfterFiFi{%
&       \expandafter\BIC@@@Inc\the\numexpr#2+1\relax
$       \expandafter\expandafter\expandafter\BIC@@@Inc
$       \ifcase#2 \expandafter1%
$       \or\expandafter2%
$       \or\expandafter3%
$       \or\expandafter4%
$       \or\expandafter5%
$       \or\expandafter6%
$       \or\expandafter7%
$       \or\expandafter8%
$       \or\expandafter9%
$?      \else\BigIntCalcError:ThisCannotHappen%
$       \fi
        0#3!{#4}%
      }%
    \else
      \BIC@AfterFiFi{%
        \BIC@@@Inc01#3!{#4}%
      }%
    \fi
  \BIC@Fi
}
%    \end{macrocode}
%    \end{macro}
%    \begin{macro}{\BIC@@@Inc}
%    \begin{macrocode}
\def\BIC@@@Inc#1#2#3!#4{%
  \ifx\\#3\\%
    \ifnum#2=1 %
      \BIC@AfterFiFi{ 1#1#4}%
    \else
      \BIC@AfterFiFi{ #1#4}%
    \fi
  \else
    \BIC@AfterFi{%
      \BIC@@Inc#2#3!{#1#4}%
    }%
  \BIC@Fi
}
%    \end{macrocode}
%    \end{macro}
%
%    \begin{macro}{\BIC@Dec}
%    \begin{macrocode}
\def\BIC@Dec#1#2!#3{%
  \ifx\\#2\\%
    \BIC@AfterFi{%
      \BIC@@Dec1#1#3!{}%
    }%
  \else
    \BIC@AfterFi{%
      \BIC@Dec#2!{#1#3}%
    }%
  \BIC@Fi
}
%    \end{macrocode}
%    \end{macro}
%    \begin{macro}{\BIC@@Dec}
%    \begin{macrocode}
\def\BIC@@Dec#1#2#3!#4{%
  \ifcase#1 %
    \ifx\\#3\\%
      \BIC@AfterFiFi{ #2#4}%
    \else
      \BIC@AfterFiFi{%
        \BIC@@Dec0#3!{#2#4}%
      }%
    \fi
  \else
    \ifnum#2>0 %
      \BIC@AfterFiFi{%
&       \expandafter\BIC@@@Dec\the\numexpr#2-1\relax
$       \expandafter\expandafter\expandafter\BIC@@@Dec
$       \ifcase#2
$?        \BigIntCalcError:ThisCannotHappen%
$       \or\expandafter0%
$       \or\expandafter1%
$       \or\expandafter2%
$       \or\expandafter3%
$       \or\expandafter4%
$       \or\expandafter5%
$       \or\expandafter6%
$       \or\expandafter7%
$       \or\expandafter8%
$?      \else\BigIntCalcError:ThisCannotHappen%
$       \fi
        0#3!{#4}%
      }%
    \else
      \BIC@AfterFiFi{%
        \BIC@@@Dec91#3!{#4}%
      }%
    \fi
  \BIC@Fi
}
%    \end{macrocode}
%    \end{macro}
%    \begin{macro}{\BIC@@@Dec}
%    \begin{macrocode}
\def\BIC@@@Dec#1#2#3!#4{%
  \ifx\\#3\\%
    \ifcase#1 %
      \ifx\\#4\\%
        \BIC@AfterFiFiFi{ 0}%
      \else
        \BIC@AfterFiFiFi{ #4}%
      \fi
    \else
      \BIC@AfterFiFi{ #1#4}%
    \fi
  \else
    \BIC@AfterFi{%
      \BIC@@Dec#2#3!{#1#4}%
    }%
  \BIC@Fi
}
%    \end{macrocode}
%    \end{macro}
%
% \subsection{\op{Add}, \op{Sub}}
%
%    \begin{macro}{\bigintcalcAdd}
%    \begin{macrocode}
\def\bigintcalcAdd#1{%
  \romannumeral0%
  \expandafter\expandafter\expandafter\BIC@Add
  \bigintcalcNum{#1}!%
}
%    \end{macrocode}
%    \end{macro}
%    \begin{macro}{\BIC@Add}
%    \begin{macrocode}
\def\BIC@Add#1!#2{%
  \expandafter\expandafter\expandafter
  \BIC@AddSwitch\bigintcalcNum{#2}!#1!%
}
%    \end{macrocode}
%    \end{macro}
%    \begin{macro}{\bigintcalcSub}
%    \begin{macrocode}
\def\bigintcalcSub#1#2{%
  \romannumeral0%
  \expandafter\expandafter\expandafter\BIC@Add
  \bigintcalcNum{-#2}!{#1}%
}
%    \end{macrocode}
%    \end{macro}
%
%    \begin{macro}{\BIC@AddSwitch}
%    Decision table for \cs{BIC@AddSwitch}.
%    \begin{quote}
%    \begin{tabular}[t]{@{}|l|l|l|l|l|@{}}
%      \hline
%      $x<0$ & $y<0$ & $-x>-y$ & $-$ & $\opAdd(-x,-y)$\\
%      \cline{3-3}\cline{5-5}
%            &       & else    &     & $\opAdd(-y,-x)$\\
%      \cline{2-5}
%            & else  & $-x> y$ & $-$ & $\opSub(-x, y)$\\
%      \cline{3-5}
%            &       & $-x= y$ &     & $0$\\
%      \cline{3-5}
%            &       & else    & $+$ & $\opSub( y,-x)$\\
%      \hline
%      else  & $y<0$ & $ x>-y$ & $+$ & $\opSub( x,-y)$\\
%      \cline{3-5}
%            &       & $ x=-y$ &     & $0$\\
%      \cline{3-5}
%            &       & else    & $-$ & $\opSub(-y, x)$\\
%      \cline{2-5}
%            & else  & $ x> y$ & $+$ & $\opAdd( x, y)$\\
%      \cline{3-3}\cline{5-5}
%            &       & else    &     & $\opAdd( y, x)$\\
%      \hline
%    \end{tabular}
%    \end{quote}
%    \begin{macrocode}
\def\BIC@AddSwitch#1#2!#3#4!{%
  \ifx#1-% x < 0
    \ifx#3-% y < 0
      \expandafter-\romannumeral0%
      \ifnum\BIC@PosCmp#2!#4!=1 % -x > -y
        \BIC@AfterFiFiFi{%
          \BIC@AddXY#2!#4!!!%
        }%
      \else % -x <= -y
        \BIC@AfterFiFiFi{%
          \BIC@AddXY#4!#2!!!%
        }%
      \fi
    \else % y >= 0
      \ifcase\BIC@PosCmp#2!#3#4!% -x = y
        \BIC@AfterFiFiFi{ 0}%
      \or % -x > y
        \expandafter-\romannumeral0%
        \BIC@AfterFiFiFi{%
          \BIC@SubXY#2!#3#4!!!%
        }%
      \else % -x <= y
        \BIC@AfterFiFiFi{%
          \BIC@SubXY#3#4!#2!!!%
        }%
      \fi
    \fi
  \else % x >= 0
    \ifx#3-% y < 0
      \ifcase\BIC@PosCmp#1#2!#4!% x = -y
        \BIC@AfterFiFiFi{ 0}%
      \or % x > -y
        \BIC@AfterFiFiFi{%
          \BIC@SubXY#1#2!#4!!!%
        }%
      \else % x <= -y
        \expandafter-\romannumeral0%
        \BIC@AfterFiFiFi{%
          \BIC@SubXY#4!#1#2!!!%
        }%
      \fi
    \else % y >= 0
      \ifnum\BIC@PosCmp#1#2!#3#4!=1 % x > y
        \BIC@AfterFiFiFi{%
          \BIC@AddXY#1#2!#3#4!!!%
        }%
      \else % x <= y
        \BIC@AfterFiFiFi{%
          \BIC@AddXY#3#4!#1#2!!!%
        }%
      \fi
    \fi
  \BIC@Fi
}
%    \end{macrocode}
%    \end{macro}
%
%    \begin{macro}{\BigIntCalcAdd}
%    \begin{macrocode}
\def\BigIntCalcAdd#1!#2!{%
  \romannumeral0\BIC@AddXY#1!#2!!!%
}
%    \end{macrocode}
%    \end{macro}
%    \begin{macro}{\BigIntCalcSub}
%    \begin{macrocode}
\def\BigIntCalcSub#1!#2!{%
  \romannumeral0\BIC@SubXY#1!#2!!!%
}
%    \end{macrocode}
%    \end{macro}
%
%    \begin{macro}{\BIC@AddXY}
%    \begin{macrocode}
\def\BIC@AddXY#1#2!#3#4!#5!#6!{%
  \ifx\\#2\\%
    \ifx\\#3\\%
      \BIC@AfterFiFi{%
        \BIC@DoAdd0!#1#5!#60!%
      }%
    \else
      \BIC@AfterFiFi{%
        \BIC@DoAdd0!#1#5!#3#6!%
      }%
    \fi
  \else
    \ifx\\#4\\%
      \ifx\\#3\\%
        \BIC@AfterFiFiFi{%
          \BIC@AddXY#2!{}!#1#5!#60!%
        }%
      \else
        \BIC@AfterFiFiFi{%
          \BIC@AddXY#2!{}!#1#5!#3#6!%
        }%
      \fi
    \else
      \BIC@AfterFiFi{%
        \BIC@AddXY#2!#4!#1#5!#3#6!%
      }%
    \fi
  \BIC@Fi
}
%    \end{macrocode}
%    \end{macro}
%    \begin{macro}{\BIC@DoAdd}
%    |#1|: carry\\
%    |#2|: reverted result\\
%    |#3#4|: reverted $x$\\
%    |#5#6|: reverted $y$
%    \begin{macrocode}
\def\BIC@DoAdd#1#2!#3#4!#5#6!{%
  \ifx\\#4\\%
    \BIC@AfterFi{%
&     \expandafter\BIC@Space
&     \the\numexpr#1+#3+#5\relax#2%
$     \expandafter\expandafter\expandafter\BIC@AddResult
$     \BIC@AddDigit#1#3#5#2%
    }%
  \else
    \BIC@AfterFi{%
      \expandafter\expandafter\expandafter\BIC@DoAdd
      \BIC@AddDigit#1#3#5#2!#4!#6!%
    }%
  \BIC@Fi
}
%    \end{macrocode}
%    \end{macro}
%    \begin{macro}{\BIC@AddResult}
%    \begin{macrocode}
$ \def\BIC@AddResult#1{%
$   \ifx#10%
$     \expandafter\BIC@Space
$   \else
$     \expandafter\BIC@Space\expandafter#1%
$   \fi
$ }%
%    \end{macrocode}
%    \end{macro}
%    \begin{macro}{\BIC@AddDigit}
%    |#1|: carry\\
%    |#2|: digit of $x$\\
%    |#3|: digit of $y$
%    \begin{macrocode}
\def\BIC@AddDigit#1#2#3{%
  \romannumeral0%
& \expandafter\BIC@@AddDigit\the\numexpr#1+#2+#3!%
$ \expandafter\BIC@@AddDigit\number%
$ \csname
$   BIC@AddCarry%
$   \ifcase#1 %
$     #2%
$   \else
$     \ifcase#2 1\or2\or3\or4\or5\or6\or7\or8\or9\or10\fi
$   \fi
$ \endcsname#3!%
}
%    \end{macrocode}
%    \end{macro}
%    \begin{macro}{\BIC@@AddDigit}
%    \begin{macrocode}
\def\BIC@@AddDigit#1!{%
  \ifnum#1<10 %
    \BIC@AfterFi{ 0#1}%
  \else
    \BIC@AfterFi{ #1}%
  \BIC@Fi
}
%    \end{macrocode}
%    \end{macro}
%    \begin{macro}{\BIC@AddCarry0}
%    \begin{macrocode}
$ \expandafter\def\csname BIC@AddCarry0\endcsname#1{#1}%
%    \end{macrocode}
%    \end{macro}
%    \begin{macro}{\BIC@AddCarry10}
%    \begin{macrocode}
$ \expandafter\def\csname BIC@AddCarry10\endcsname#1{1#1}%
%    \end{macrocode}
%    \end{macro}
%    \begin{macro}{\BIC@AddCarry[1-9]}
%    \begin{macrocode}
$ \def\BIC@Temp#1#2{%
$   \expandafter\def\csname BIC@AddCarry#1\endcsname##1{%
$     \ifcase##1 #1\or
$     #2%
$?    \else\BigIntCalcError:ThisCannotHappen%
$     \fi
$   }%
$ }%
$ \BIC@Temp 0{1\or2\or3\or4\or5\or6\or7\or8\or9}%
$ \BIC@Temp 1{2\or3\or4\or5\or6\or7\or8\or9\or10}%
$ \BIC@Temp 2{3\or4\or5\or6\or7\or8\or9\or10\or11}%
$ \BIC@Temp 3{4\or5\or6\or7\or8\or9\or10\or11\or12}%
$ \BIC@Temp 4{5\or6\or7\or8\or9\or10\or11\or12\or13}%
$ \BIC@Temp 5{6\or7\or8\or9\or10\or11\or12\or13\or14}%
$ \BIC@Temp 6{7\or8\or9\or10\or11\or12\or13\or14\or15}%
$ \BIC@Temp 7{8\or9\or10\or11\or12\or13\or14\or15\or16}%
$ \BIC@Temp 8{9\or10\or11\or12\or13\or14\or15\or16\or17}%
$ \BIC@Temp 9{10\or11\or12\or13\or14\or15\or16\or17\or18}%
%    \end{macrocode}
%    \end{macro}
%
%    \begin{macro}{\BIC@SubXY}
%    Preconditions:
%    \begin{itemize}
%    \item $x > y$, $x \geq 0$, and $y >= 0$
%    \item digits($x$) = digits($y$)
%    \end{itemize}
%    \begin{macrocode}
\def\BIC@SubXY#1#2!#3#4!#5!#6!{%
  \ifx\\#2\\%
    \ifx\\#3\\%
      \BIC@AfterFiFi{%
        \BIC@DoSub0!#1#5!#60!%
      }%
    \else
      \BIC@AfterFiFi{%
        \BIC@DoSub0!#1#5!#3#6!%
      }%
    \fi
  \else
    \ifx\\#4\\%
      \ifx\\#3\\%
        \BIC@AfterFiFiFi{%
          \BIC@SubXY#2!{}!#1#5!#60!%
        }%
      \else
        \BIC@AfterFiFiFi{%
          \BIC@SubXY#2!{}!#1#5!#3#6!%
        }%
      \fi
    \else
      \BIC@AfterFiFi{%
        \BIC@SubXY#2!#4!#1#5!#3#6!%
      }%
    \fi
  \BIC@Fi
}
%    \end{macrocode}
%    \end{macro}
%    \begin{macro}{\BIC@DoSub}
%    |#1|: carry\\
%    |#2|: reverted result\\
%    |#3#4|: reverted x\\
%    |#5#6|: reverted y
%    \begin{macrocode}
\def\BIC@DoSub#1#2!#3#4!#5#6!{%
  \ifx\\#4\\%
    \BIC@AfterFi{%
      \expandafter\expandafter\expandafter\BIC@SubResult
      \BIC@SubDigit#1#3#5#2%
    }%
  \else
    \BIC@AfterFi{%
      \expandafter\expandafter\expandafter\BIC@DoSub
      \BIC@SubDigit#1#3#5#2!#4!#6!%
    }%
  \BIC@Fi
}
%    \end{macrocode}
%    \end{macro}
%    \begin{macro}{\BIC@SubResult}
%    \begin{macrocode}
\def\BIC@SubResult#1{%
  \ifx#10%
    \expandafter\BIC@SubResult
  \else
    \expandafter\BIC@Space\expandafter#1%
  \fi
}
%    \end{macrocode}
%    \end{macro}
%    \begin{macro}{\BIC@SubDigit}
%    |#1|: carry\\
%    |#2|: digit of $x$\\
%    |#3|: digit of $y$
%    \begin{macrocode}
\def\BIC@SubDigit#1#2#3{%
  \romannumeral0%
& \expandafter\BIC@@SubDigit\the\numexpr#2-#3-#1!%
$ \expandafter\BIC@@AddDigit\number
$   \csname
$     BIC@SubCarry%
$     \ifcase#1 %
$       #3%
$     \else
$       \ifcase#3 1\or2\or3\or4\or5\or6\or7\or8\or9\or10\fi
$     \fi
$   \endcsname#2!%
}
%    \end{macrocode}
%    \end{macro}
%    \begin{macro}{\BIC@@SubDigit}
%    \begin{macrocode}
& \def\BIC@@SubDigit#1!{%
&   \ifnum#1<0 %
&     \BIC@AfterFi{%
&       \expandafter\BIC@Space
&       \expandafter1\the\numexpr#1+10\relax
&     }%
&   \else
&     \BIC@AfterFi{ 0#1}%
&   \BIC@Fi
& }%
%    \end{macrocode}
%    \end{macro}
%    \begin{macro}{\BIC@SubCarry0}
%    \begin{macrocode}
$ \expandafter\def\csname BIC@SubCarry0\endcsname#1{#1}%
%    \end{macrocode}
%    \end{macro}
%    \begin{macro}{\BIC@SubCarry10}%
%    \begin{macrocode}
$ \expandafter\def\csname BIC@SubCarry10\endcsname#1{1#1}%
%    \end{macrocode}
%    \end{macro}
%    \begin{macro}{\BIC@SubCarry[1-9]}
%    \begin{macrocode}
$ \def\BIC@Temp#1#2{%
$   \expandafter\def\csname BIC@SubCarry#1\endcsname##1{%
$     \ifcase##1 #2%
$?    \else\BigIntCalcError:ThisCannotHappen%
$     \fi
$   }%
$ }%
$ \BIC@Temp 1{19\or0\or1\or2\or3\or4\or5\or6\or7\or8}%
$ \BIC@Temp 2{18\or19\or0\or1\or2\or3\or4\or5\or6\or7}%
$ \BIC@Temp 3{17\or18\or19\or0\or1\or2\or3\or4\or5\or6}%
$ \BIC@Temp 4{16\or17\or18\or19\or0\or1\or2\or3\or4\or5}%
$ \BIC@Temp 5{15\or16\or17\or18\or19\or0\or1\or2\or3\or4}%
$ \BIC@Temp 6{14\or15\or16\or17\or18\or19\or0\or1\or2\or3}%
$ \BIC@Temp 7{13\or14\or15\or16\or17\or18\or19\or0\or1\or2}%
$ \BIC@Temp 8{12\or13\or14\or15\or16\or17\or18\or19\or0\or1}%
$ \BIC@Temp 9{11\or12\or13\or14\or15\or16\or17\or18\or19\or0}%
%    \end{macrocode}
%    \end{macro}
%
% \subsection{\op{Shl}, \op{Shr}}
%
%    \begin{macro}{\bigintcalcShl}
%    \begin{macrocode}
\def\bigintcalcShl#1{%
  \romannumeral0%
  \expandafter\expandafter\expandafter\BIC@Shl
  \bigintcalcNum{#1}!%
}
%    \end{macrocode}
%    \end{macro}
%    \begin{macro}{\BIC@Shl}
%    \begin{macrocode}
\def\BIC@Shl#1#2!{%
  \ifx#1-%
    \BIC@AfterFi{%
      \expandafter-\romannumeral0%
&     \BIC@@Shl#2!!%
$     \BIC@AddXY#2!#2!!!%
    }%
  \else
    \BIC@AfterFi{%
&     \BIC@@Shl#1#2!!%
$     \BIC@AddXY#1#2!#1#2!!!%
    }%
  \BIC@Fi
}
%    \end{macrocode}
%    \end{macro}
%    \begin{macro}{\BigIntCalcShl}
%    \begin{macrocode}
\def\BigIntCalcShl#1!{%
  \romannumeral0%
& \BIC@@Shl#1!!%
$ \BIC@AddXY#1!#1!!!%
}
%    \end{macrocode}
%    \end{macro}
%    \begin{macro}{\BIC@@Shl}
%    \begin{macrocode}
& \def\BIC@@Shl#1#2!{%
&   \ifx\\#2\\%
&     \BIC@AfterFi{%
&       \BIC@@@Shl0!#1%
&     }%
&   \else
&     \BIC@AfterFi{%
&       \BIC@@Shl#2!#1%
&     }%
&   \BIC@Fi
& }%
%    \end{macrocode}
%    \end{macro}
%    \begin{macro}{\BIC@@@Shl}
%    |#1|: carry\\
%    |#2|: result\\
%    |#3#4|: reverted number
%    \begin{macrocode}
& \def\BIC@@@Shl#1#2!#3#4!{%
&   \ifx\\#4\\%
&     \BIC@AfterFi{%
&       \expandafter\BIC@Space
&       \the\numexpr#3*2+#1\relax#2%
&     }%
&   \else
&     \BIC@AfterFi{%
&       \expandafter\BIC@@@@Shl\the\numexpr#3*2+#1!#2!#4!%
&     }%
&   \BIC@Fi
& }%
%    \end{macrocode}
%    \end{macro}
%    \begin{macro}{\BIC@@@@Shl}
%    \begin{macrocode}
& \def\BIC@@@@Shl#1!{%
&   \ifnum#1<10 %
&     \BIC@AfterFi{%
&       \BIC@@@Shl0#1%
&     }%
&   \else
&     \BIC@AfterFi{%
&       \BIC@@@Shl#1%
&     }%
&   \BIC@Fi
& }%
%    \end{macrocode}
%    \end{macro}
%
%    \begin{macro}{\bigintcalcShr}
%    \begin{macrocode}
\def\bigintcalcShr#1{%
  \romannumeral0%
  \expandafter\expandafter\expandafter\BIC@Shr
  \bigintcalcNum{#1}!%
}
%    \end{macrocode}
%    \end{macro}
%    \begin{macro}{\BIC@Shr}
%    \begin{macrocode}
\def\BIC@Shr#1#2!{%
  \ifx#1-%
    \expandafter-\romannumeral0%
    \BIC@AfterFi{%
      \BIC@@Shr#2!%
    }%
  \else
    \BIC@AfterFi{%
      \BIC@@Shr#1#2!%
    }%
  \BIC@Fi
}
%    \end{macrocode}
%    \end{macro}
%    \begin{macro}{\BigIntCalcShr}
%    \begin{macrocode}
\def\BigIntCalcShr#1!{%
  \romannumeral0%
  \BIC@@Shr#1!%
}
%    \end{macrocode}
%    \end{macro}
%    \begin{macro}{\BIC@@Shr}
%    \begin{macrocode}
\def\BIC@@Shr#1#2!{%
  \ifcase#1 %
    \BIC@AfterFi{ 0}%
  \or
    \ifx\\#2\\%
      \BIC@AfterFiFi{ 0}%
    \else
      \BIC@AfterFiFi{%
        \BIC@@@Shr#1#2!!%
      }%
    \fi
  \else
    \BIC@AfterFi{%
      \BIC@@@Shr0#1#2!!%
    }%
  \BIC@Fi
}
%    \end{macrocode}
%    \end{macro}
%    \begin{macro}{\BIC@@@Shr}
%    |#1|: carry\\
%    |#2#3|: number\\
%    |#4|: result
%    \begin{macrocode}
\def\BIC@@@Shr#1#2#3!#4!{%
  \ifx\\#3\\%
    \ifodd#1#2 %
      \BIC@AfterFiFi{%
&       \expandafter\BIC@ShrResult\the\numexpr(#1#2-1)/2\relax
$       \expandafter\expandafter\expandafter\BIC@ShrResult
$       \csname BIC@ShrDigit#1#2\endcsname
        #4!%
      }%
    \else
      \BIC@AfterFiFi{%
&       \expandafter\BIC@ShrResult\the\numexpr#1#2/2\relax
$       \expandafter\expandafter\expandafter\BIC@ShrResult
$       \csname BIC@ShrDigit#1#2\endcsname
        #4!%
      }%
    \fi
  \else
    \ifodd#1#2 %
      \BIC@AfterFiFi{%
&       \expandafter\BIC@@@@Shr\the\numexpr(#1#2-1)/2\relax1%
$       \expandafter\expandafter\expandafter\BIC@@@@Shr
$       \csname BIC@ShrDigit#1#2\endcsname
        #3!#4!%
      }%
    \else
      \BIC@AfterFiFi{%
&       \expandafter\BIC@@@@Shr\the\numexpr#1#2/2\relax0%
$       \expandafter\expandafter\expandafter\BIC@@@@Shr
$       \csname BIC@ShrDigit#1#2\endcsname
        #3!#4!%
      }%
    \fi
  \BIC@Fi
}
%    \end{macrocode}
%    \end{macro}
%    \begin{macro}{\BIC@ShrResult}
%    \begin{macrocode}
& \def\BIC@ShrResult#1#2!{ #2#1}%
$ \def\BIC@ShrResult#1#2#3!{ #3#1}%
%    \end{macrocode}
%    \end{macro}
%    \begin{macro}{\BIC@@@@Shr}
%    |#1|: new digit\\
%    |#2|: carry\\
%    |#3|: remaining number\\
%    |#4|: result
%    \begin{macrocode}
\def\BIC@@@@Shr#1#2#3!#4!{%
  \BIC@@@Shr#2#3!#4#1!%
}
%    \end{macrocode}
%    \end{macro}
%    \begin{macro}{\BIC@ShrDigit[00-19]}
%    \begin{macrocode}
$ \def\BIC@Temp#1#2#3#4{%
$   \expandafter\def\csname BIC@ShrDigit#1#2\endcsname{#3#4}%
$ }%
$ \BIC@Temp 0000%
$ \BIC@Temp 0101%
$ \BIC@Temp 0210%
$ \BIC@Temp 0311%
$ \BIC@Temp 0420%
$ \BIC@Temp 0521%
$ \BIC@Temp 0630%
$ \BIC@Temp 0731%
$ \BIC@Temp 0840%
$ \BIC@Temp 0941%
$ \BIC@Temp 1050%
$ \BIC@Temp 1151%
$ \BIC@Temp 1260%
$ \BIC@Temp 1361%
$ \BIC@Temp 1470%
$ \BIC@Temp 1571%
$ \BIC@Temp 1680%
$ \BIC@Temp 1781%
$ \BIC@Temp 1890%
$ \BIC@Temp 1991%
%    \end{macrocode}
%    \end{macro}
%
% \subsection{\cs{BIC@Tim}}
%
%    \begin{macro}{\BIC@Tim}
%    Macro \cs{BIC@Tim} implements ``Number \emph{tim}es digit''.\\
%    |#1|: plain number without sign\\
%    |#2|: digit
\def\BIC@Tim#1!#2{%
  \romannumeral0%
  \ifcase#2 % 0
    \BIC@AfterFi{ 0}%
  \or % 1
    \BIC@AfterFi{ #1}%
  \or % 2
    \BIC@AfterFi{%
      \BIC@Shl#1!%
    }%
  \else % 3-9
    \BIC@AfterFi{%
      \BIC@@Tim#1!!#2%
    }%
  \BIC@Fi
}
%    \begin{macrocode}
%    \end{macrocode}
%    \end{macro}
%    \begin{macro}{\BIC@@Tim}
%    |#1#2|: number\\
%    |#3|: reverted number
%    \begin{macrocode}
\def\BIC@@Tim#1#2!{%
  \ifx\\#2\\%
    \BIC@AfterFi{%
      \BIC@ProcessTim0!#1%
    }%
  \else
    \BIC@AfterFi{%
      \BIC@@Tim#2!#1%
    }%
  \BIC@Fi
}
%    \end{macrocode}
%    \end{macro}
%    \begin{macro}{\BIC@ProcessTim}
%    |#1|: carry\\
%    |#2|: result\\
%    |#3#4|: reverted number\\
%    |#5|: digit
%    \begin{macrocode}
\def\BIC@ProcessTim#1#2!#3#4!#5{%
  \ifx\\#4\\%
    \BIC@AfterFi{%
      \expandafter\BIC@Space
&     \the\numexpr#3*#5+#1\relax
$     \romannumeral0\BIC@TimDigit#3#5#1%
      #2%
    }%
  \else
    \BIC@AfterFi{%
      \expandafter\BIC@@ProcessTim
&     \the\numexpr#3*#5+#1%
$     \romannumeral0\BIC@TimDigit#3#5#1%
      !#2!#4!#5%
    }%
  \BIC@Fi
}
%    \end{macrocode}
%    \end{macro}
%    \begin{macro}{\BIC@@ProcessTim}
%    |#1#2|: carry?, new digit\\
%    |#3|: new number\\
%    |#4|: old number\\
%    |#5|: digit
%    \begin{macrocode}
\def\BIC@@ProcessTim#1#2!{%
  \ifx\\#2\\%
    \BIC@AfterFi{%
      \BIC@ProcessTim0#1%
    }%
  \else
    \BIC@AfterFi{%
      \BIC@ProcessTim#1#2%
    }%
  \BIC@Fi
}
%    \end{macrocode}
%    \end{macro}
%    \begin{macro}{\BIC@TimDigit}
%    |#1|: digit 0--9\\
%    |#2|: digit 3--9\\
%    |#3|: carry 0--9
%    \begin{macrocode}
$ \def\BIC@TimDigit#1#2#3{%
$   \ifcase#1 % 0
$     \BIC@AfterFi{ #3}%
$   \or % 1
$     \BIC@AfterFi{%
$       \expandafter\BIC@Space
$       \number\csname BIC@AddCarry#2\endcsname#3 %
$     }%
$   \else
$     \ifcase#3 %
$       \BIC@AfterFiFi{%
$         \expandafter\BIC@Space
$         \number\csname BIC@MulDigit#2\endcsname#1 %
$       }%
$     \else
$       \BIC@AfterFiFi{%
$         \expandafter\BIC@Space
$         \romannumeral0%
$         \expandafter\BIC@AddXY
$         \number\csname BIC@MulDigit#2\endcsname#1!%
$         #3!!!%
$       }%
$     \fi
$   \BIC@Fi
$ }%
%    \end{macrocode}
%    \end{macro}
%    \begin{macro}{\BIC@MulDigit[3-9]}
%    \begin{macrocode}
$ \def\BIC@Temp#1#2{%
$   \expandafter\def\csname BIC@MulDigit#1\endcsname##1{%
$     \ifcase##1 0%
$     \or ##1%
$     \or #2%
$?    \else\BigIntCalcError:ThisCannotHappen%
$     \fi
$   }%
$ }%
$ \BIC@Temp 3{6\or9\or12\or15\or18\or21\or24\or27}%
$ \BIC@Temp 4{8\or12\or16\or20\or24\or28\or32\or36}%
$ \BIC@Temp 5{10\or15\or20\or25\or30\or35\or40\or45}%
$ \BIC@Temp 6{12\or18\or24\or30\or36\or42\or48\or54}%
$ \BIC@Temp 7{14\or21\or28\or35\or42\or49\or56\or63}%
$ \BIC@Temp 8{16\or24\or32\or40\or48\or56\or64\or72}%
$ \BIC@Temp 9{18\or27\or36\or45\or54\or63\or72\or81}%
%    \end{macrocode}
%    \end{macro}
%
% \subsection{\op{Mul}}
%
%    \begin{macro}{\bigintcalcMul}
%    \begin{macrocode}
\def\bigintcalcMul#1#2{%
  \romannumeral0%
  \expandafter\expandafter\expandafter\BIC@Mul
  \bigintcalcNum{#1}!{#2}%
}
%    \end{macrocode}
%    \end{macro}
%    \begin{macro}{\BIC@Mul}
%    \begin{macrocode}
\def\BIC@Mul#1!#2{%
  \expandafter\expandafter\expandafter\BIC@MulSwitch
  \bigintcalcNum{#2}!#1!%
}
%    \end{macrocode}
%    \end{macro}
%    \begin{macro}{\BIC@MulSwitch}
%    Decision table for \cs{BIC@MulSwitch}.
%    \begin{quote}
%    \begin{tabular}[t]{@{}|l|l|l|l|l|@{}}
%      \hline
%      $x=0$ &\multicolumn{3}{l}{}    &$0$\\
%      \hline
%      $x>0$ & $y=0$ &\multicolumn2l{}&$0$\\
%      \cline{2-5}
%            & $y>0$ & $ x> y$ & $+$ & $\opMul( x, y)$\\
%      \cline{3-3}\cline{5-5}
%            &       & else    &     & $\opMul( y, x)$\\
%      \cline{2-5}
%            & $y<0$ & $ x>-y$ & $-$ & $\opMul( x,-y)$\\
%      \cline{3-3}\cline{5-5}
%            &       & else    &     & $\opMul(-y, x)$\\
%      \hline
%      $x<0$ & $y=0$ &\multicolumn2l{}&$0$\\
%      \cline{2-5}
%            & $y>0$ & $-x> y$ & $-$ & $\opMul(-x, y)$\\
%      \cline{3-3}\cline{5-5}
%            &       & else    &     & $\opMul( y,-x)$\\
%      \cline{2-5}
%            & $y<0$ & $-x>-y$ & $+$ & $\opMul(-x,-y)$\\
%      \cline{3-3}\cline{5-5}
%            &       & else    &     & $\opMul(-y,-x)$\\
%      \hline
%    \end{tabular}
%    \end{quote}
%    \begin{macrocode}
\def\BIC@MulSwitch#1#2!#3#4!{%
  \ifcase\BIC@Sgn#1#2! % x = 0
    \BIC@AfterFi{ 0}%
  \or % x > 0
    \ifcase\BIC@Sgn#3#4! % y = 0
      \BIC@AfterFiFi{ 0}%
    \or % y > 0
      \ifnum\BIC@PosCmp#1#2!#3#4!=1 % x > y
        \BIC@AfterFiFiFi{%
          \BIC@ProcessMul0!#1#2!#3#4!%
        }%
      \else % x <= y
        \BIC@AfterFiFiFi{%
          \BIC@ProcessMul0!#3#4!#1#2!%
        }%
      \fi
    \else % y < 0
      \expandafter-\romannumeral0%
      \ifnum\BIC@PosCmp#1#2!#4!=1 % x > -y
        \BIC@AfterFiFiFi{%
          \BIC@ProcessMul0!#1#2!#4!%
        }%
      \else % x <= -y
        \BIC@AfterFiFiFi{%
          \BIC@ProcessMul0!#4!#1#2!%
        }%
      \fi
    \fi
  \else % x < 0
    \ifcase\BIC@Sgn#3#4! % y = 0
      \BIC@AfterFiFi{ 0}%
    \or % y > 0
      \expandafter-\romannumeral0%
      \ifnum\BIC@PosCmp#2!#3#4!=1 % -x > y
        \BIC@AfterFiFiFi{%
          \BIC@ProcessMul0!#2!#3#4!%
        }%
      \else % -x <= y
        \BIC@AfterFiFiFi{%
          \BIC@ProcessMul0!#3#4!#2!%
        }%
      \fi
    \else % y < 0
      \ifnum\BIC@PosCmp#2!#4!=1 % -x > -y
        \BIC@AfterFiFiFi{%
          \BIC@ProcessMul0!#2!#4!%
        }%
      \else % -x <= -y
        \BIC@AfterFiFiFi{%
          \BIC@ProcessMul0!#4!#2!%
        }%
      \fi
    \fi
  \BIC@Fi
}
%    \end{macrocode}
%    \end{macro}
%    \begin{macro}{\BigIntCalcMul}
%    \begin{macrocode}
\def\BigIntCalcMul#1!#2!{%
  \romannumeral0%
  \BIC@ProcessMul0!#1!#2!%
}
%    \end{macrocode}
%    \end{macro}
%    \begin{macro}{\BIC@ProcessMul}
%    |#1|: result\\
%    |#2|: number $x$\\
%    |#3#4|: number $y$
%    \begin{macrocode}
\def\BIC@ProcessMul#1!#2!#3#4!{%
  \ifx\\#4\\%
    \BIC@AfterFi{%
      \expandafter\expandafter\expandafter\BIC@Space
      \bigintcalcAdd{\BIC@Tim#2!#3}{#10}%
    }%
  \else
    \BIC@AfterFi{%
      \expandafter\expandafter\expandafter\BIC@ProcessMul
      \bigintcalcAdd{\BIC@Tim#2!#3}{#10}!#2!#4!%
    }%
  \BIC@Fi
}
%    \end{macrocode}
%    \end{macro}
%
% \subsection{\op{Sqr}}
%
%    \begin{macro}{\bigintcalcSqr}
%    \begin{macrocode}
\def\bigintcalcSqr#1{%
  \romannumeral0%
  \expandafter\expandafter\expandafter\BIC@Sqr
  \bigintcalcNum{#1}!%
}
%    \end{macrocode}
%    \end{macro}
%    \begin{macro}{\BIC@Sqr}
%    \begin{macrocode}
\def\BIC@Sqr#1{%
   \ifx#1-%
     \expandafter\BIC@@Sqr
   \else
     \expandafter\BIC@@Sqr\expandafter#1%
   \fi
}
%    \end{macrocode}
%    \end{macro}
%    \begin{macro}{\BIC@@Sqr}
%    \begin{macrocode}
\def\BIC@@Sqr#1!{%
  \BIC@ProcessMul0!#1!#1!%
}
%    \end{macrocode}
%    \end{macro}
%
% \subsection{\op{Fac}}
%
%    \begin{macro}{\bigintcalcFac}
%    \begin{macrocode}
\def\bigintcalcFac#1{%
  \romannumeral0%
  \expandafter\expandafter\expandafter\BIC@Fac
  \bigintcalcNum{#1}!%
}
%    \end{macrocode}
%    \end{macro}
%    \begin{macro}{\BIC@Fac}
%    \begin{macrocode}
\def\BIC@Fac#1#2!{%
  \ifx#1-%
    \BIC@AfterFi{ 0\BigIntCalcError:FacNegative}%
  \else
    \ifnum\BIC@PosCmp#1#2!13!<0 %
      \ifcase#1#2 %
         \BIC@AfterFiFiFi{ 1}% 0!
      \or\BIC@AfterFiFiFi{ 1}% 1!
      \or\BIC@AfterFiFiFi{ 2}% 2!
      \or\BIC@AfterFiFiFi{ 6}% 3!
      \or\BIC@AfterFiFiFi{ 24}% 4!
      \or\BIC@AfterFiFiFi{ 120}% 5!
      \or\BIC@AfterFiFiFi{ 720}% 6!
      \or\BIC@AfterFiFiFi{ 5040}% 7!
      \or\BIC@AfterFiFiFi{ 40320}% 8!
      \or\BIC@AfterFiFiFi{ 362880}% 9!
      \or\BIC@AfterFiFiFi{ 3628800}% 10!
      \or\BIC@AfterFiFiFi{ 39916800}% 11!
      \or\BIC@AfterFiFiFi{ 479001600}% 12!
?     \else\BigIntCalcError:ThisCannotHappen%
      \fi
    \else
      \BIC@AfterFiFi{%
        \BIC@ProcessFac#1#2!479001600!%
      }%
    \fi
  \BIC@Fi
}
%    \end{macrocode}
%    \end{macro}
%    \begin{macro}{\BIC@ProcessFac}
%    |#1|: $n$\\
%    |#2|: result
%    \begin{macrocode}
\def\BIC@ProcessFac#1!#2!{%
  \ifnum\BIC@PosCmp#1!12!=0 %
    \BIC@AfterFi{ #2}%
  \else
    \BIC@AfterFi{%
      \expandafter\BIC@@ProcessFac
      \romannumeral0\BIC@ProcessMul0!#2!#1!%
      !#1!%
    }%
  \BIC@Fi
}
%    \end{macrocode}
%    \end{macro}
%    \begin{macro}{\BIC@@ProcessFac}
%    |#1|: result\\
%    |#2|: $n$
%    \begin{macrocode}
\def\BIC@@ProcessFac#1!#2!{%
  \expandafter\BIC@ProcessFac
  \romannumeral0\BIC@Dec#2!{}%
  !#1!%
}
%    \end{macrocode}
%    \end{macro}
%
% \subsection{\op{Pow}}
%
%    \begin{macro}{\bigintcalcPow}
%    |#1|: basis\\
%    |#2|: power
%    \begin{macrocode}
\def\bigintcalcPow#1{%
  \romannumeral0%
  \expandafter\expandafter\expandafter\BIC@Pow
  \bigintcalcNum{#1}!%
}
%    \end{macrocode}
%    \end{macro}
%    \begin{macro}{\BIC@Pow}
%    |#1|: basis\\
%    |#2|: power
%    \begin{macrocode}
\def\BIC@Pow#1!#2{%
  \expandafter\expandafter\expandafter\BIC@PowSwitch
  \bigintcalcNum{#2}!#1!%
}
%    \end{macrocode}
%    \end{macro}
%    \begin{macro}{\BIC@PowSwitch}
%    |#1#2|: power $y$\\
%    |#3#4|: basis $x$\\
%    Decision table for \cs{BIC@PowSwitch}.
%    \begin{quote}
%    \def\M#1{\multicolumn{#1}{l}{}}%
%    \begin{tabular}[t]{@{}|l|l|l|l|@{}}
%    \hline
%    $y=0$ & \M{2}                        & $1$\\
%    \hline
%    $y=1$ & \M{2}                        & $x$\\
%    \hline
%    $y=2$ & $x<0$          & \M{1}       & $\opMul(-x,-x)$\\
%    \cline{2-4}
%          & else           & \M{1}       & $\opMul(x,x)$\\
%    \hline
%    $y<0$ & $x=0$          & \M{1}       & DivisionByZero\\
%    \cline{2-4}
%          & $x=1$          & \M{1}       & $1$\\
%    \cline{2-4}
%          & $x=-1$         & $\opODD(y)$ & $-1$\\
%    \cline{3-4}
%          &                & else        & $1$\\
%    \cline{2-4}
%          & else ($\string|x\string|>1$) & \M{1}       & $0$\\
%    \hline
%    $y>2$ & $x=0$          & \M{1}       & $0$\\
%    \cline{2-4}
%          & $x=1$          & \M{1}       & $1$\\
%    \cline{2-4}
%          & $x=-1$         & $\opODD(y)$ & $-1$\\
%    \cline{3-4}
%          &                & else        & $1$\\
%    \cline{2-4}
%          & $x<-1$ ($x<0$) & $\opODD(y)$ & $-\opPow(-x,y)$\\
%    \cline{3-4}
%          &                & else        & $\opPow(-x,y)$\\
%    \cline{2-4}
%          & else ($x>1$)   & \M{1}       & $\opPow(x,y)$\\
%    \hline
%    \end{tabular}
%    \end{quote}
%    \begin{macrocode}
\def\BIC@PowSwitch#1#2!#3#4!{%
  \ifcase\ifx\\#2\\%
           \ifx#100 % y = 0
           \else\ifx#111 % y = 1
           \else\ifx#122 % y = 2
           \else4 % y > 2
           \fi\fi\fi
         \else
           \ifx#1-3 % y < 0
           \else4 % y > 2
           \fi
         \fi
    \BIC@AfterFi{ 1}% y = 0
  \or % y = 1
    \BIC@AfterFi{ #3#4}%
  \or % y = 2
    \ifx#3-% x < 0
      \BIC@AfterFiFi{%
        \BIC@ProcessMul0!#4!#4!%
      }%
    \else % x >= 0
      \BIC@AfterFiFi{%
        \BIC@ProcessMul0!#3#4!#3#4!%
      }%
    \fi
  \or % y < 0
    \ifcase\ifx\\#4\\%
             \ifx#300 % x = 0
             \else\ifx#311 % x = 1
             \else3 % x > 1
             \fi\fi
           \else
             \ifcase\BIC@MinusOne#3#4! %
               3 % |x| > 1
             \or
               2 % x = -1
?            \else\BigIntCalcError:ThisCannotHappen%
             \fi
           \fi
      \BIC@AfterFiFi{ 0\BigIntCalcError:DivisionByZero}% x = 0
    \or % x = 1
      \BIC@AfterFiFi{ 1}% x = 1
    \or % x = -1
      \ifcase\BIC@ModTwo#2! % even(y)
        \BIC@AfterFiFiFi{ 1}%
      \or % odd(y)
        \BIC@AfterFiFiFi{ -1}%
?     \else\BigIntCalcError:ThisCannotHappen%
      \fi
    \or % |x| > 1
      \BIC@AfterFiFi{ 0}%
?   \else\BigIntCalcError:ThisCannotHappen%
    \fi
  \or % y > 2
    \ifcase\ifx\\#4\\%
             \ifx#300 % x = 0
             \else\ifx#311 % x = 1
             \else4 % x > 1
             \fi\fi
           \else
             \ifx#3-%
               \ifcase\BIC@MinusOne#3#4! %
                 3 % x < -1
               \else
                 2 % x = -1
               \fi
             \else
               4 % x > 1
             \fi
           \fi
      \BIC@AfterFiFi{ 0}% x = 0
    \or % x = 1
      \BIC@AfterFiFi{ 1}% x = 1
    \or % x = -1
      \ifcase\BIC@ModTwo#1#2! % even(y)
        \BIC@AfterFiFiFi{ 1}%
      \or % odd(y)
        \BIC@AfterFiFiFi{ -1}%
?     \else\BigIntCalcError:ThisCannotHappen%
      \fi
    \or % x < -1
      \ifcase\BIC@ModTwo#1#2! % even(y)
        \BIC@AfterFiFiFi{%
          \BIC@PowRec#4!#1#2!1!%
        }%
      \or % odd(y)
        \expandafter-\romannumeral0%
        \BIC@AfterFiFiFi{%
          \BIC@PowRec#4!#1#2!1!%
        }%
?     \else\BigIntCalcError:ThisCannotHappen%
      \fi
    \or % x > 1
      \BIC@AfterFiFi{%
        \BIC@PowRec#3#4!#1#2!1!%
      }%
?   \else\BigIntCalcError:ThisCannotHappen%
    \fi
? \else\BigIntCalcError:ThisCannotHappen%
  \BIC@Fi
}
%    \end{macrocode}
%    \end{macro}
%
% \subsubsection{Help macros}
%
%    \begin{macro}{\BIC@ModTwo}
%    Macro \cs{BIC@ModTwo} expects a number without sign
%    and returns digit |1| or |0| if the number is odd or even.
%    \begin{macrocode}
\def\BIC@ModTwo#1#2!{%
  \ifx\\#2\\%
    \ifodd#1 %
      \BIC@AfterFiFi1%
    \else
      \BIC@AfterFiFi0%
    \fi
  \else
    \BIC@AfterFi{%
      \BIC@ModTwo#2!%
    }%
  \BIC@Fi
}
%    \end{macrocode}
%    \end{macro}
%
%    \begin{macro}{\BIC@MinusOne}
%    Macro \cs{BIC@MinusOne} expects a number and returns
%    digit |1| if the number equals minus one and returns |0| otherwise.
%    \begin{macrocode}
\def\BIC@MinusOne#1#2!{%
  \ifx#1-%
    \BIC@@MinusOne#2!%
  \else
    0%
  \fi
}
%    \end{macrocode}
%    \end{macro}
%    \begin{macro}{\BIC@@MinusOne}
%    \begin{macrocode}
\def\BIC@@MinusOne#1#2!{%
  \ifx#11%
    \ifx\\#2\\%
      1%
    \else
      0%
    \fi
  \else
    0%
  \fi
}
%    \end{macrocode}
%    \end{macro}
%
% \subsubsection{Recursive calculation}
%
%    \begin{macro}{\BIC@PowRec}
%\begin{quote}
%\begin{verbatim}
%Pow(x, y) {
%  PowRec(x, y, 1)
%}
%PowRec(x, y, r) {
%  if y == 1 then
%    return r
%  else
%    ifodd y then
%      return PowRec(x*x, y div 2, r*x) % y div 2 = (y-1)/2
%    else
%      return PowRec(x*x, y div 2, r)
%    fi
%  fi
%}
%\end{verbatim}
%\end{quote}
%    |#1|: $x$ (basis)\\
%    |#2#3|: $y$ (power)\\
%    |#4|: $r$ (result)
%    \begin{macrocode}
\def\BIC@PowRec#1!#2#3!#4!{%
  \ifcase\ifx#21\ifx\\#3\\0 \else1 \fi\else1 \fi % y = 1
    \ifnum\BIC@PosCmp#1!#4!=1 % x > r
      \BIC@AfterFiFi{%
        \BIC@ProcessMul0!#1!#4!%
      }%
    \else
      \BIC@AfterFiFi{%
        \BIC@ProcessMul0!#4!#1!%
      }%
    \fi
  \or
    \ifcase\BIC@ModTwo#2#3! % even(y)
      \BIC@AfterFiFi{%
        \expandafter\BIC@@PowRec\romannumeral0%
        \BIC@@Shr#2#3!%
        !#1!#4!%
      }%
    \or % odd(y)
      \ifnum\BIC@PosCmp#1!#4!=1 % x > r
        \BIC@AfterFiFiFi{%
          \expandafter\BIC@@@PowRec\romannumeral0%
          \BIC@ProcessMul0!#1!#4!%
          !#1!#2#3!%
        }%
      \else
        \BIC@AfterFiFiFi{%
          \expandafter\BIC@@@PowRec\romannumeral0%
          \BIC@ProcessMul0!#1!#4!%
          !#1!#2#3!%
        }%
      \fi
?   \else\BigIntCalcError:ThisCannotHappen%
    \fi
? \else\BigIntCalcError:ThisCannotHappen%
  \BIC@Fi
}
%    \end{macrocode}
%    \end{macro}
%    \begin{macro}{\BIC@@PowRec}
%    |#1|: $y/2$\\
%    |#2|: $x$\\
%    |#3|: new $r$ ($r$ or $r*x$)
%    \begin{macrocode}
\def\BIC@@PowRec#1!#2!#3!{%
  \expandafter\BIC@PowRec\romannumeral0%
  \BIC@ProcessMul0!#2!#2!%
  !#1!#3!%
}
%    \end{macrocode}
%    \end{macro}
%    \begin{macro}{\BIC@@@PowRec}
%    |#1|: $r*x$
%    |#2|: $x$
%    |#3|: $y$
%    \begin{macrocode}
\def\BIC@@@PowRec#1!#2!#3!{%
  \expandafter\BIC@@PowRec\romannumeral0%
  \BIC@@Shr#3!%
  !#2!#1!%
}
%    \end{macrocode}
%    \end{macro}
%
% \subsection{\op{Div}}
%
%    \begin{macro}{\bigintcalcDiv}
%    |#1|: $x$\\
%    |#2|: $y$ (divisor)
%    \begin{macrocode}
\def\bigintcalcDiv#1{%
  \romannumeral0%
  \expandafter\expandafter\expandafter\BIC@Div
  \bigintcalcNum{#1}!%
}
%    \end{macrocode}
%    \end{macro}
%    \begin{macro}{\BIC@Div}
%    |#1|: $x$\\
%    |#2|: $y$
%    \begin{macrocode}
\def\BIC@Div#1!#2{%
  \expandafter\expandafter\expandafter\BIC@DivSwitchSign
  \bigintcalcNum{#2}!#1!%
}
%    \end{macrocode}
%    \end{macro}
%    \begin{macro}{\BigIntCalcDiv}
%    \begin{macrocode}
\def\BigIntCalcDiv#1!#2!{%
  \romannumeral0%
  \BIC@DivSwitchSign#2!#1!%
}
%    \end{macrocode}
%    \end{macro}
%    \begin{macro}{\BIC@DivSwitchSign}
%    Decision table for \cs{BIC@DivSwitchSign}.
%    \begin{quote}
%    \begin{tabular}{@{}|l|l|l|@{}}
%    \hline
%    $y=0$ & \multicolumn{1}{l}{} & DivisionByZero\\
%    \hline
%    $y>0$ & $x=0$ & $0$\\
%    \cline{2-3}
%          & $x>0$ & DivSwitch$(+,x,y)$\\
%    \cline{2-3}
%          & $x<0$ & DivSwitch$(-,-x,y)$\\
%    \hline
%    $y<0$ & $x=0$ & $0$\\
%    \cline{2-3}
%          & $x>0$ & DivSwitch$(-,x,-y)$\\
%    \cline{2-3}
%          & $x<0$ & DivSwitch$(+,-x,-y)$\\
%    \hline
%    \end{tabular}
%    \end{quote}
%    |#1|: $y$ (divisor)\\
%    |#2|: $x$
%    \begin{macrocode}
\def\BIC@DivSwitchSign#1#2!#3#4!{%
  \ifcase\BIC@Sgn#1#2! % y = 0
    \BIC@AfterFi{ 0\BigIntCalcError:DivisionByZero}%
  \or % y > 0
    \ifcase\BIC@Sgn#3#4! % x = 0
      \BIC@AfterFiFi{ 0}%
    \or % x > 0
      \BIC@AfterFiFi{%
        \BIC@DivSwitch{}#3#4!#1#2!%
      }%
    \else % x < 0
      \BIC@AfterFiFi{%
        \BIC@DivSwitch-#4!#1#2!%
      }%
    \fi
  \else % y < 0
    \ifcase\BIC@Sgn#3#4! % x = 0
      \BIC@AfterFiFi{ 0}%
    \or % x > 0
      \BIC@AfterFiFi{%
        \BIC@DivSwitch-#3#4!#2!%
      }%
    \else % x < 0
      \BIC@AfterFiFi{%
        \BIC@DivSwitch{}#4!#2!%
      }%
    \fi
  \BIC@Fi
}
%    \end{macrocode}
%    \end{macro}
%    \begin{macro}{\BIC@DivSwitch}
%    Decision table for \cs{BIC@DivSwitch}.
%    \begin{quote}
%    \begin{tabular}{@{}|l|l|l|@{}}
%    \hline
%    $y=x$ & \multicolumn{1}{l}{} & sign $1$\\
%    \hline
%    $y>x$ & \multicolumn{1}{l}{} & $0$\\
%    \hline
%    $y<x$ & $y=1$                & sign $x$\\
%    \cline{2-3}
%          & $y=2$                & sign Shr$(x)$\\
%    \cline{2-3}
%          & $y=4$                & sign Shr(Shr$(x)$)\\
%    \cline{2-3}
%          & else                 & sign ProcessDiv$(x,y)$\\
%    \hline
%    \end{tabular}
%    \end{quote}
%    |#1|: sign\\
%    |#2|: $x$\\
%    |#3#4|: $y$ ($y\ne 0$)
%    \begin{macrocode}
\def\BIC@DivSwitch#1#2!#3#4!{%
  \ifcase\BIC@PosCmp#3#4!#2!% y = x
    \BIC@AfterFi{ #11}%
  \or % y > x
    \BIC@AfterFi{ 0}%
  \else % y < x
    \ifx\\#1\\%
    \else
      \expandafter-\romannumeral0%
    \fi
    \ifcase\ifx\\#4\\%
             \ifx#310 % y = 1
             \else\ifx#321 % y = 2
             \else\ifx#342 % y = 4
             \else3 % y > 2
             \fi\fi\fi
           \else
             3 % y > 2
           \fi
      \BIC@AfterFiFi{ #2}% y = 1
    \or % y = 2
      \BIC@AfterFiFi{%
        \BIC@@Shr#2!%
      }%
    \or % y = 4
      \BIC@AfterFiFi{%
        \expandafter\BIC@@Shr\romannumeral0%
          \BIC@@Shr#2!!%
      }%
    \or % y > 2
      \BIC@AfterFiFi{%
        \BIC@DivStartX#2!#3#4!!!%
      }%
?   \else\BigIntCalcError:ThisCannotHappen%
    \fi
  \BIC@Fi
}
%    \end{macrocode}
%    \end{macro}
%    \begin{macro}{\BIC@ProcessDiv}
%    |#1#2|: $x$\\
%    |#3#4|: $y$\\
%    |#5|: collect first digits of $x$\\
%    |#6|: corresponding digits of $y$
%    \begin{macrocode}
\def\BIC@DivStartX#1#2!#3#4!#5!#6!{%
  \ifx\\#4\\%
    \BIC@AfterFi{%
      \BIC@DivStartYii#6#3#4!{#5#1}#2=!%
    }%
  \else
    \BIC@AfterFi{%
      \BIC@DivStartX#2!#4!#5#1!#6#3!%
    }%
  \BIC@Fi
}
%    \end{macrocode}
%    \end{macro}
%    \begin{macro}{\BIC@DivStartYii}
%    |#1|: $y$\\
%    |#2|: $x$, |=|
%    \begin{macrocode}
\def\BIC@DivStartYii#1!{%
  \expandafter\BIC@DivStartYiv\romannumeral0%
  \BIC@Shl#1!%
  !#1!%
}
%    \end{macrocode}
%    \end{macro}
%    \begin{macro}{\BIC@DivStartYiv}
%    |#1|: $2y$\\
%    |#2|: $y$\\
%    |#3|: $x$, |=|
%    \begin{macrocode}
\def\BIC@DivStartYiv#1!{%
  \expandafter\BIC@DivStartYvi\romannumeral0%
  \BIC@Shl#1!%
  !#1!%
}
%    \end{macrocode}
%    \end{macro}
%    \begin{macro}{\BIC@DivStartYvi}
%    |#1|: $4y$\\
%    |#2|: $2y$\\
%    |#3|: $y$\\
%    |#4|: $x$, |=|
%    \begin{macrocode}
\def\BIC@DivStartYvi#1!#2!{%
  \expandafter\BIC@DivStartYviii\romannumeral0%
  \BIC@AddXY#1!#2!!!%
  !#1!#2!%
}
%    \end{macrocode}
%    \end{macro}
%    \begin{macro}{\BIC@DivStartYviii}
%    |#1|: $6y$\\
%    |#2|: $4y$\\
%    |#3|: $2y$\\
%    |#4|: $y$\\
%    |#5|: $x$, |=|
%    \begin{macrocode}
\def\BIC@DivStartYviii#1!#2!{%
  \expandafter\BIC@DivStart\romannumeral0%
  \BIC@Shl#2!%
  !#1!#2!%
}
%    \end{macrocode}
%    \end{macro}
%    \begin{macro}{\BIC@DivStart}
%    |#1|: $8y$\\
%    |#2|: $6y$\\
%    |#3|: $4y$\\
%    |#4|: $2y$\\
%    |#5|: $y$\\
%    |#6|: $x$, |=|
%    \begin{macrocode}
\def\BIC@DivStart#1!#2!#3!#4!#5!#6!{%
  \BIC@ProcessDiv#6!!#5!#4!#3!#2!#1!=%
}
%    \end{macrocode}
%    \end{macro}
%    \begin{macro}{\BIC@ProcessDiv}
%    |#1#2#3|: $x$, |=|\\
%    |#4|: result\\
%    |#5|: $y$\\
%    |#6|: $2y$\\
%    |#7|: $4y$\\
%    |#8|: $6y$\\
%    |#9|: $8y$
%    \begin{macrocode}
\def\BIC@ProcessDiv#1#2#3!#4!#5!{%
  \ifcase\BIC@PosCmp#5!#1!% y = #1
    \ifx#2=%
      \BIC@AfterFiFi{\BIC@DivCleanup{#41}}%
    \else
      \BIC@AfterFiFi{%
        \BIC@ProcessDiv#2#3!#41!#5!%
      }%
    \fi
  \or % y > #1
    \ifx#2=%
      \BIC@AfterFiFi{\BIC@DivCleanup{#40}}%
    \else
      \ifx\\#4\\%
        \BIC@AfterFiFiFi{%
          \BIC@ProcessDiv{#1#2}#3!!#5!%
        }%
      \else
        \BIC@AfterFiFiFi{%
          \BIC@ProcessDiv{#1#2}#3!#40!#5!%
        }%
      \fi
    \fi
  \else % y < #1
    \BIC@AfterFi{%
      \BIC@@ProcessDiv{#1}#2#3!#4!#5!%
    }%
  \BIC@Fi
}
%    \end{macrocode}
%    \end{macro}
%    \begin{macro}{\BIC@DivCleanup}
%    |#1|: result\\
%    |#2|: garbage
%    \begin{macrocode}
\def\BIC@DivCleanup#1#2={ #1}%
%    \end{macrocode}
%    \end{macro}
%    \begin{macro}{\BIC@@ProcessDiv}
%    \begin{macrocode}
\def\BIC@@ProcessDiv#1#2#3!#4!#5!#6!#7!{%
  \ifcase\BIC@PosCmp#7!#1!% 4y = #1
    \ifx#2=%
      \BIC@AfterFiFi{\BIC@DivCleanup{#44}}%
    \else
     \BIC@AfterFiFi{%
       \BIC@ProcessDiv#2#3!#44!#5!#6!#7!%
     }%
    \fi
  \or % 4y > #1
    \ifcase\BIC@PosCmp#6!#1!% 2y = #1
      \ifx#2=%
        \BIC@AfterFiFiFi{\BIC@DivCleanup{#42}}%
      \else
        \BIC@AfterFiFiFi{%
          \BIC@ProcessDiv#2#3!#42!#5!#6!#7!%
        }%
      \fi
    \or % 2y > #1
      \ifx#2=%
        \BIC@AfterFiFiFi{\BIC@DivCleanup{#41}}%
      \else
        \BIC@AfterFiFiFi{%
          \BIC@DivSub#1!#5!#2#3!#41!#5!#6!#7!%
        }%
      \fi
    \else % 2y < #1
      \BIC@AfterFiFi{%
        \expandafter\BIC@ProcessDivII\romannumeral0%
        \BIC@SubXY#1!#6!!!%
        !#2#3!#4!#5!23%
        #6!#7!%
      }%
    \fi
  \else % 4y < #1
    \BIC@AfterFi{%
      \BIC@@@ProcessDiv{#1}#2#3!#4!#5!#6!#7!%
    }%
  \BIC@Fi
}
%    \end{macrocode}
%    \end{macro}
%    \begin{macro}{\BIC@DivSub}
%    Next token group: |#1|-|#2| and next digit |#3|.
%    \begin{macrocode}
\def\BIC@DivSub#1!#2!#3{%
  \expandafter\BIC@ProcessDiv\expandafter{%
    \romannumeral0%
    \BIC@SubXY#1!#2!!!%
    #3%
  }%
}
%    \end{macrocode}
%    \end{macro}
%    \begin{macro}{\BIC@ProcessDivII}
%    |#1|: $x'-2y$\\
%    |#2#3|: remaining $x$, |=|\\
%    |#4|: result\\
%    |#5|: $y$\\
%    |#6|: first possible result digit\\
%    |#7|: second possible result digit
%    \begin{macrocode}
\def\BIC@ProcessDivII#1!#2#3!#4!#5!#6#7{%
  \ifcase\BIC@PosCmp#5!#1!% y = #1
    \ifx#2=%
      \BIC@AfterFiFi{\BIC@DivCleanup{#4#7}}%
    \else
      \BIC@AfterFiFi{%
        \BIC@ProcessDiv#2#3!#4#7!#5!%
      }%
    \fi
  \or % y > #1
    \ifx#2=%
      \BIC@AfterFiFi{\BIC@DivCleanup{#4#6}}%
    \else
      \BIC@AfterFiFi{%
        \BIC@ProcessDiv{#1#2}#3!#4#6!#5!%
      }%
    \fi
  \else % y < #1
    \ifx#2=%
      \BIC@AfterFiFi{\BIC@DivCleanup{#4#7}}%
    \else
      \BIC@AfterFiFi{%
        \BIC@DivSub#1!#5!#2#3!#4#7!#5!%
      }%
    \fi
  \BIC@Fi
}
%    \end{macrocode}
%    \end{macro}
%    \begin{macro}{\BIC@ProcessDivIV}
%    |#1#2#3|: $x$, |=|, $x>4y$\\
%    |#4|: result\\
%    |#5|: $y$\\
%    |#6|: $2y$\\
%    |#7|: $4y$\\
%    |#8|: $6y$\\
%    |#9|: $8y$
%    \begin{macrocode}
\def\BIC@@@ProcessDiv#1#2#3!#4!#5!#6!#7!#8!#9!{%
  \ifcase\BIC@PosCmp#8!#1!% 6y = #1
    \ifx#2=%
      \BIC@AfterFiFi{\BIC@DivCleanup{#46}}%
    \else
      \BIC@AfterFiFi{%
        \BIC@ProcessDiv#2#3!#46!#5!#6!#7!#8!#9!%
      }%
    \fi
  \or % 6y > #1
    \BIC@AfterFi{%
      \expandafter\BIC@ProcessDivII\romannumeral0%
      \BIC@SubXY#1!#7!!!%
      !#2#3!#4!#5!45%
      #6!#7!#8!#9!%
    }%
  \else % 6y < #1
    \ifcase\BIC@PosCmp#9!#1!% 8y = #1
      \ifx#2=%
        \BIC@AfterFiFiFi{\BIC@DivCleanup{#48}}%
      \else
        \BIC@AfterFiFiFi{%
          \BIC@ProcessDiv#2#3!#48!#5!#6!#7!#8!#9!%
        }%
      \fi
    \or % 8y > #1
      \BIC@AfterFiFi{%
        \expandafter\BIC@ProcessDivII\romannumeral0%
        \BIC@SubXY#1!#8!!!%
        !#2#3!#4!#5!67%
        #6!#7!#8!#9!%
      }%
    \else % 8y < #1
      \BIC@AfterFiFi{%
        \expandafter\BIC@ProcessDivII\romannumeral0%
        \BIC@SubXY#1!#9!!!%
        !#2#3!#4!#5!89%
        #6!#7!#8!#9!%
      }%
    \fi
  \BIC@Fi
}
%    \end{macrocode}
%    \end{macro}
%
% \subsection{\op{Mod}}
%
%    \begin{macro}{\bigintcalcMod}
%    |#1|: $x$\\
%    |#2|: $y$
%    \begin{macrocode}
\def\bigintcalcMod#1{%
  \romannumeral0%
  \expandafter\expandafter\expandafter\BIC@Mod
  \bigintcalcNum{#1}!%
}
%    \end{macrocode}
%    \end{macro}
%    \begin{macro}{\BIC@Mod}
%    |#1|: $x$\\
%    |#2|: $y$
%    \begin{macrocode}
\def\BIC@Mod#1!#2{%
  \expandafter\expandafter\expandafter\BIC@ModSwitchSign
  \bigintcalcNum{#2}!#1!%
}
%    \end{macrocode}
%    \end{macro}
%    \begin{macro}{\BigIntCalcMod}
%    \begin{macrocode}
\def\BigIntCalcMod#1!#2!{%
  \romannumeral0%
  \BIC@ModSwitchSign#2!#1!%
}
%    \end{macrocode}
%    \end{macro}
%    \begin{macro}{\BIC@ModSwitchSign}
%    Decision table for \cs{BIC@ModSwitchSign}.
%    \begin{quote}
%    \begin{tabular}{@{}|l|l|l|@{}}
%    \hline
%    $y=0$ & \multicolumn{1}{l}{} & DivisionByZero\\
%    \hline
%    $y>0$ & $x=0$ & $0$\\
%    \cline{2-3}
%          & else  & ModSwitch$(+,x,y)$\\
%    \hline
%    $y<0$ & \multicolumn{1}{l}{} & ModSwitch$(-,-x,-y)$\\
%    \hline
%    \end{tabular}
%    \end{quote}
%    |#1#2|: $y$\\
%    |#3#4|: $x$
%    \begin{macrocode}
\def\BIC@ModSwitchSign#1#2!#3#4!{%
  \ifcase\ifx\\#2\\%
           \ifx#100 % y = 0
           \else1 % y > 0
           \fi
          \else
            \ifx#1-2 % y < 0
            \else1 % y > 0
            \fi
          \fi
    \BIC@AfterFi{ 0\BigIntCalcError:DivisionByZero}%
  \or % y > 0
    \ifcase\ifx\\#4\\\ifx#300 \else1 \fi\else1 \fi % x = 0
      \BIC@AfterFiFi{ 0}%
    \else
      \BIC@AfterFiFi{%
        \BIC@ModSwitch{}#3#4!#1#2!%
      }%
    \fi
  \else % y < 0
    \ifcase\ifx\\#4\\%
             \ifx#300 % x = 0
             \else1 % x > 0
             \fi
           \else
             \ifx#3-2 % x < 0
             \else1 % x > 0
             \fi
           \fi
      \BIC@AfterFiFi{ 0}%
    \or % x > 0
      \BIC@AfterFiFi{%
        \BIC@ModSwitch--#3#4!#2!%
      }%
    \else % x < 0
      \BIC@AfterFiFi{%
        \BIC@ModSwitch-#4!#2!%
      }%
    \fi
  \BIC@Fi
}
%    \end{macrocode}
%    \end{macro}
%    \begin{macro}{\BIC@ModSwitch}
%    Decision table for \cs{BIC@ModSwitch}.
%    \begin{quote}
%    \begin{tabular}{@{}|l|l|l|@{}}
%    \hline
%    $y=1$ & \multicolumn{1}{l}{} & $0$\\
%    \hline
%    $y=2$ & ifodd$(x)$ & sign $1$\\
%    \cline{2-3}
%          & else       & $0$\\
%    \hline
%    $y>2$ & $x<0$      &
%        $z\leftarrow x-(x/y)*y;\quad (z<0)\mathbin{?}z+y\mathbin{:}z$\\
%    \cline{2-3}
%          & $x>0$      & $x - (x/y) * y$\\
%    \hline
%    \end{tabular}
%    \end{quote}
%    |#1|: sign\\
%    |#2#3|: $x$\\
%    |#4#5|: $y$
%    \begin{macrocode}
\def\BIC@ModSwitch#1#2#3!#4#5!{%
  \ifcase\ifx\\#5\\%
           \ifx#410 % y = 1
           \else\ifx#421 % y = 2
           \else2 % y > 2
           \fi\fi
         \else2 % y > 2
         \fi
    \BIC@AfterFi{ 0}% y = 1
  \or % y = 2
    \ifcase\BIC@ModTwo#2#3! % even(x)
      \BIC@AfterFiFi{ 0}%
    \or % odd(x)
      \BIC@AfterFiFi{ #11}%
?   \else\BigIntCalcError:ThisCannotHappen%
    \fi
  \or % y > 2
    \ifx\\#1\\%
    \else
      \expandafter\BIC@Space\romannumeral0%
      \expandafter\BIC@ModMinus\romannumeral0%
    \fi
    \ifx#2-% x < 0
      \BIC@AfterFiFi{%
        \expandafter\expandafter\expandafter\BIC@ModX
        \bigintcalcSub{#2#3}{%
          \bigintcalcMul{#4#5}{\bigintcalcDiv{#2#3}{#4#5}}%
        }!#4#5!%
      }%
    \else % x > 0
      \BIC@AfterFiFi{%
        \expandafter\expandafter\expandafter\BIC@Space
        \bigintcalcSub{#2#3}{%
          \bigintcalcMul{#4#5}{\bigintcalcDiv{#2#3}{#4#5}}%
        }%
      }%
    \fi
? \else\BigIntCalcError:ThisCannotHappen%
  \BIC@Fi
}
%    \end{macrocode}
%    \end{macro}
%    \begin{macro}{\BIC@ModMinus}
%    \begin{macrocode}
\def\BIC@ModMinus#1{%
  \ifx#10%
    \BIC@AfterFi{ 0}%
  \else
    \BIC@AfterFi{ -#1}%
  \BIC@Fi
}
%    \end{macrocode}
%    \end{macro}
%    \begin{macro}{\BIC@ModX}
%    |#1#2|: $z$\\
%    |#3|: $x$
%    \begin{macrocode}
\def\BIC@ModX#1#2!#3!{%
  \ifx#1-% z < 0
    \BIC@AfterFi{%
      \expandafter\BIC@Space\romannumeral0%
      \BIC@SubXY#3!#2!!!%
    }%
  \else % z >= 0
    \BIC@AfterFi{ #1#2}%
  \BIC@Fi
}
%    \end{macrocode}
%    \end{macro}
%
%    \begin{macrocode}
\BIC@AtEnd%
%    \end{macrocode}
%
%    \begin{macrocode}
%</package>
%    \end{macrocode}
%
% \section{Test}
%
% \subsection{Catcode checks for loading}
%
%    \begin{macrocode}
%<*test1>
%    \end{macrocode}
%    \begin{macrocode}
\catcode`\{=1 %
\catcode`\}=2 %
\catcode`\#=6 %
\catcode`\@=11 %
\expandafter\ifx\csname count@\endcsname\relax
  \countdef\count@=255 %
\fi
\expandafter\ifx\csname @gobble\endcsname\relax
  \long\def\@gobble#1{}%
\fi
\expandafter\ifx\csname @firstofone\endcsname\relax
  \long\def\@firstofone#1{#1}%
\fi
\expandafter\ifx\csname loop\endcsname\relax
  \expandafter\@firstofone
\else
  \expandafter\@gobble
\fi
{%
  \def\loop#1\repeat{%
    \def\body{#1}%
    \iterate
  }%
  \def\iterate{%
    \body
      \let\next\iterate
    \else
      \let\next\relax
    \fi
    \next
  }%
  \let\repeat=\fi
}%
\def\RestoreCatcodes{}
\count@=0 %
\loop
  \edef\RestoreCatcodes{%
    \RestoreCatcodes
    \catcode\the\count@=\the\catcode\count@\relax
  }%
\ifnum\count@<255 %
  \advance\count@ 1 %
\repeat

\def\RangeCatcodeInvalid#1#2{%
  \count@=#1\relax
  \loop
    \catcode\count@=15 %
  \ifnum\count@<#2\relax
    \advance\count@ 1 %
  \repeat
}
\def\RangeCatcodeCheck#1#2#3{%
  \count@=#1\relax
  \loop
    \ifnum#3=\catcode\count@
    \else
      \errmessage{%
        Character \the\count@\space
        with wrong catcode \the\catcode\count@\space
        instead of \number#3%
      }%
    \fi
  \ifnum\count@<#2\relax
    \advance\count@ 1 %
  \repeat
}
\def\space{ }
\expandafter\ifx\csname LoadCommand\endcsname\relax
  \def\LoadCommand{\input bigintcalc.sty\relax}%
\fi
\def\Test{%
  \RangeCatcodeInvalid{0}{47}%
  \RangeCatcodeInvalid{58}{64}%
  \RangeCatcodeInvalid{91}{96}%
  \RangeCatcodeInvalid{123}{255}%
  \catcode`\@=12 %
  \catcode`\\=0 %
  \catcode`\%=14 %
  \LoadCommand
  \RangeCatcodeCheck{0}{36}{15}%
  \RangeCatcodeCheck{37}{37}{14}%
  \RangeCatcodeCheck{38}{47}{15}%
  \RangeCatcodeCheck{48}{57}{12}%
  \RangeCatcodeCheck{58}{63}{15}%
  \RangeCatcodeCheck{64}{64}{12}%
  \RangeCatcodeCheck{65}{90}{11}%
  \RangeCatcodeCheck{91}{91}{15}%
  \RangeCatcodeCheck{92}{92}{0}%
  \RangeCatcodeCheck{93}{96}{15}%
  \RangeCatcodeCheck{97}{122}{11}%
  \RangeCatcodeCheck{123}{255}{15}%
  \RestoreCatcodes
}
\Test
\csname @@end\endcsname
\end
%    \end{macrocode}
%    \begin{macrocode}
%</test1>
%    \end{macrocode}
%
% \subsection{Macro tests}
%
% \subsubsection{Preamble with test macro definitions}
%
%    \begin{macrocode}
%<*test2>
\NeedsTeXFormat{LaTeX2e}
\nofiles
\documentclass{article}
%<noetex>\let\SavedNumexpr\numexpr
%<noetex>\let\numexpr\UNDEFINED
\makeatletter
\chardef\BIC@TestMode=1 %
\makeatother
\usepackage{bigintcalc}[2016/05/16]
%<noetex>\let\numexpr\SavedNumexpr
\usepackage{qstest}
\IncludeTests{*}
\LogTests{log}{*}{*}
\newcommand*{\TestSpaceAtEnd}[1]{%
%<noetex>  \let\SavedNumexpr\numexpr
%<noetex>  \let\numexpr\UNDEFINED
  \edef\resultA{#1}%
  \edef\resultB{#1 }%
%<noetex>  \let\numexpr\SavedNumexpr
  \Expect*{\resultA\space}*{\resultB}%
}
\newcommand*{\TestResult}[2]{%
%<noetex>  \let\SavedNumexpr\numexpr
%<noetex>  \let\numexpr\UNDEFINED
  \edef\result{#1}%
%<noetex>  \let\numexpr\SavedNumexpr
  \Expect*{\result}{#2}%
}
\newcommand*{\TestResultTwoExpansions}[2]{%
%<*noetex>
  \begingroup
    \let\numexpr\UNDEFINED
    \expandafter\expandafter\expandafter
  \endgroup
%</noetex>
  \expandafter\expandafter\expandafter\Expect
  \expandafter\expandafter\expandafter{#1}{#2}%
}
\newcount\TestCount
%<etex>\newcommand*{\TestArg}[1]{\numexpr#1\relax}
%<noetex>\newcommand*{\TestArg}[1]{#1}
\newcommand*{\TestTeXDivide}[2]{%
  \TestCount=\TestArg{#1}\relax
  \divide\TestCount by \TestArg{#2}\relax
  \Expect*{\bigintcalcDiv{#1}{#2}}*{\the\TestCount}%
}
\newcommand*{\Test}[2]{%
  \TestResult{#1}{#2}%
  \TestResultTwoExpansions{#1}{#2}%
  \TestSpaceAtEnd{#1}%
}
\newcommand*{\TestExch}[2]{\Test{#2}{#1}}
\newcommand*{\TestInv}[2]{%
  \Test{\bigintcalcInv{#1}}{#2}%
}
\newcommand*{\TestAbs}[2]{%
  \Test{\bigintcalcAbs{#1}}{#2}%
}
\newcommand*{\TestSgn}[2]{%
  \Test{\bigintcalcSgn{#1}}{#2}%
}
\newcommand*{\TestMin}[3]{%
  \Test{\bigintcalcMin{#1}{#2}}{#3}%
}
\newcommand*{\TestMax}[3]{%
  \Test{\bigintcalcMax{#1}{#2}}{#3}%
}
\newcommand*{\TestCmp}[3]{%
  \Test{\bigintcalcCmp{#1}{#2}}{#3}%
}
\newcommand*{\TestOdd}[2]{%
  \Test{\bigintcalcOdd{#1}}{#2}%
  \edef\x{%
    \noexpand\Test{%
      \noexpand\BigIntCalcOdd
      \bigintcalcAbs{#1}!%
    }{#2}%
  }%
  \x
}
\newcommand*{\TestInc}[2]{%
  \Test{\bigintcalcInc{#1}}{#2}%
  \ifnum\bigintcalcSgn{#1}>-1 %
    \edef\x{%
      \noexpand\Test{%
        \noexpand\BigIntCalcInc\bigintcalcNum{#1}!%
      }{#2}%
    }%
    \x
  \fi
}
\newcommand*{\TestDec}[2]{%
  \Test{\bigintcalcDec{#1}}{#2}%
  \ifnum\bigintcalcSgn{#1}>0 %
    \edef\x{%
      \noexpand\Test{%
        \noexpand\BigIntCalcDec\bigintcalcNum{#1}!%
      }{#2}%
    }%
    \x
  \fi
}
\newcommand*{\TestAdd}[3]{%
  \Test{\bigintcalcAdd{#1}{#2}}{#3}%
  \ifnum\bigintcalcSgn{#1}>0 %
    \ifnum\bigintcalcSgn{#2}> 0 %
      \ifnum\bigintcalcCmp{#1}{#2}>0 %
        \edef\x{%
          \noexpand\Test{%
            \noexpand\BigIntCalcAdd
            \bigintcalcNum{#1}!\bigintcalcNum{#2}!%
          }{#3}%
        }%
        \x
      \else
        \edef\x{%
          \noexpand\Test{%
            \noexpand\BigIntCalcAdd
            \bigintcalcNum{#2}!\bigintcalcNum{#1}!%
          }{#3}%
        }%
        \x
      \fi
    \fi
  \fi
}
\newcommand*{\TestSub}[3]{%
  \Test{\bigintcalcSub{#1}{#2}}{#3}%
  \ifnum\bigintcalcSgn{#1}>0 %
    \ifnum\bigintcalcSgn{#2}> 0 %
      \ifnum\bigintcalcCmp{#1}{#2}>0 %
        \edef\x{%
          \noexpand\Test{%
            \noexpand\BigIntCalcSub
            \bigintcalcNum{#1}!\bigintcalcNum{#2}!%
          }{#3}%
        }%
        \x
      \fi
    \fi
  \fi
}
\newcommand*{\TestShl}[2]{%
  \Test{\bigintcalcShl{#1}}{#2}%
  \edef\x{%
    \noexpand\Test{%
      \noexpand\BigIntCalcShl\bigintcalcAbs{#1}!%
    }{\bigintcalcAbs{#2}}%
  }%
  \x
}
\newcommand*{\TestShr}[2]{%
  \Test{\bigintcalcShr{#1}}{#2}%
  \edef\x{%
    \noexpand\Test{%
      \noexpand\BigIntCalcShr\bigintcalcAbs{#1}!%
    }{\bigintcalcAbs{#2}}%
  }%
  \x
}
\newcommand*{\TestMul}[3]{%
  \Test{\bigintcalcMul{#1}{#2}}{#3}%
  \edef\x{%
    \noexpand\Test{%
      \noexpand\BigIntCalcMul
      \bigintcalcAbs{#1}!\bigintcalcAbs{#2}!%
    }{\bigintcalcAbs{#3}}%
  }%
  \x
}
\newcommand*{\TestSqr}[2]{%
  \Test{\bigintcalcSqr{#1}}{#2}%
}
\newcommand*{\TestFac}[2]{%
  \expandafter\TestExch\expandafter{%
    \the\numexpr#2%
  }{\bigintcalcFac{#1}}%
}
\newcommand*{\TestFacBig}[2]{%
  \Test{\bigintcalcFac{#1}}{#2}%
}
\newcommand*{\TestPow}[3]{%
  \Test{\bigintcalcPow{#1}{#2}}{#3}%
}
\newcommand*{\TestDiv}[3]{%
  \Test{\bigintcalcDiv{#1}{#2}}{#3}%
  \TestTeXDivide{#1}{#2}%
}
\newcommand*{\TestDivBig}[3]{%
  \Test{\bigintcalcDiv{#1}{#2}}{#3}%
  \edef\x{%
    \noexpand\Test{%
      \noexpand\BigIntCalcDiv\bigintcalcAbs{#1}!\bigintcalcAbs{#2}!%
    }{\bigintcalcAbs{#3}}%
  }%
}
\newcommand*{\TestMod}[3]{%
  \Test{\bigintcalcMod{#1}{#2}}{#3}%
  \ifcase\ifcase\bigintcalcSgn{#1} 0%
         \or
           \ifcase\bigintcalcSgn{#2} 1%
           \or 0%
           \else 1%
           \fi
         \else
           \ifcase\bigintcalcSgn{#2} 1%
           \or 1%
           \else 0%
           \fi
         \fi\relax
    \edef\x{%
      \noexpand\Test{%
        \noexpand\BigIntCalcMod
        \bigintcalcAbs{#1}!\bigintcalcAbs{#2}!%
      }{\bigintcalcAbs{#3}}%
    }%
    \x
  \fi
}
%    \end{macrocode}
%
% \subsubsection{Time}
%
%    \begin{macrocode}
\begingroup\expandafter\expandafter\expandafter\endgroup
\expandafter\ifx\csname pdfresettimer\endcsname\relax
\else
  \makeatletter
  \newcount\SummaryTime
  \newcount\TestTime
  \SummaryTime=\z@
  \newcommand*{\PrintTime}[2]{%
    \typeout{%
      [Time #1: \strip@pt\dimexpr\number#2sp\relax\space s]%
    }%
  }%
  \newcommand*{\StartTime}[1]{%
    \renewcommand*{\TimeDescription}{#1}%
    \pdfresettimer
  }%
  \newcommand*{\TimeDescription}{}%
  \newcommand*{\StopTime}{%
    \TestTime=\pdfelapsedtime
    \global\advance\SummaryTime\TestTime
    \PrintTime\TimeDescription\TestTime
  }%
  \let\saved@qstest\qstest
  \let\saved@endqstest\endqstest
  \def\qstest#1#2{%
    \saved@qstest{#1}{#2}%
    \StartTime{#1}%
  }%
  \def\endqstest{%
    \StopTime
    \saved@endqstest
  }%
  \AtEndDocument{%
    \PrintTime{summary}\SummaryTime
  }%
  \makeatother
\fi
%    \end{macrocode}
%
% \subsubsection{Test sets}
%
%    \begin{macrocode}
\makeatletter

\begin{qstest}{inv}{inv}%
  \TestInv{0}{0}%
  \TestInv{1}{-1}%
  \TestInv{-1}{1}%
  \TestInv{10}{-10}%
  \TestInv{-10}{10}%
  \TestInv{2147483647}{-2147483647}%
  \TestInv{-2147483647}{2147483647}%
  \TestInv{12345678901234567890}{-12345678901234567890}%
  \TestInv{-12345678901234567890}{12345678901234567890}%
  \TestInv{ 0 }{0}%
  \TestInv{ 1 }{-1}%
  \TestInv{--1}{-1}%
  \TestInv{\number\z@}{0}%
  \TestInv{\ifx\relax\relax1\fi}{-1}%
  \TestInv{\ifx\relax\relax-\fi\ifx234\else1\fi}{1}%
\end{qstest}

\begin{qstest}{abs}{abs}%
  \TestAbs{0}{0}%
  \TestAbs{1}{1}%
  \TestAbs{-1}{1}%
  \TestAbs{10}{10}%
  \TestAbs{-10}{10}%
  \TestAbs{2147483647}{2147483647}%
  \TestAbs{-2147483647}{2147483647}%
  \TestAbs{12345678901234567890}{12345678901234567890}%
  \TestAbs{-12345678901234567890}{12345678901234567890}%
  \TestAbs{ 0 }{0}%
  \TestAbs{ 1 }{1}%
  \TestAbs{--1}{1}%
  \TestAbs{-+-+1}{1}%
  \TestAbs{00000000000}{0}%
  \TestAbs{00000001000}{1000}%
  \TestAbs{\ifx\relax\relax 0\else 1\fi}{0}%
\end{qstest}

\begin{qstest}{sign}{sign}%
  \TestSgn{0}{0}%
  \TestSgn{1}{1}%
  \TestSgn{-1}{-1}%
  \TestSgn{10}{1}%
  \TestSgn{-10}{-1}%
  \TestSgn{2147483647}{1}%
  \TestSgn{-2147483647}{-1}%
  \TestSgn{12345678901234567890}{1}%
  \TestSgn{-12345678901234567890}{-1}%
  \TestSgn{ 0 }{0}%
  \TestSgn{ 2 }{1}%
  \TestSgn{ -2 }{-1}%
  \TestSgn{--2}{1}%
  \TestSgn{\number\z@}{0}%
  \TestSgn{\number\@ne}{1}%
  \TestSgn{\number\m@ne}{-1}%
  \TestSgn{%
    -+-+\number\z@\number\z@
    \iftrue1\fi\iftrue2\fi\iftrue3\fi
  }{1}%
\end{qstest}

\begin{qstest}{min}{min}%
  \TestMin{0}{1}{0}%
  \TestMin{1}{0}{0}%
  \TestMin{-10}{-20}{-20}%
  \TestMin{ 1 }{ 2 }{1}%
  \TestMin{ 2 }{ 1 }{1}%
  \TestMin{1}{1}{1}%
  \TestMin{\number\z@}{\number\@ne}{0}%
  \TestMin{\number\@ne}{\number\m@ne}{-1}%
\end{qstest}

\begin{qstest}{max}{max}%
  \TestMax{0}{1}{1}%
  \TestMax{1}{0}{1}%
  \TestMax{-10}{-20}{-10}%
  \TestMax{ 1 }{ 2 }{2}%
  \TestMax{ 2 }{ 1 }{2}%
  \TestMax{1}{1}{1}%
  \TestMax{\number\z@}{\number\@ne}{1}%
  \TestMax{\number\@ne}{\number\m@ne}{1}%
\end{qstest}

\begin{qstest}{cmp}{cmp}%
  \TestCmp{0}{0}{0}%
  \TestCmp{-21}{17}{-1}%
  \TestCmp{3}{4}{-1}%
  \TestCmp{-10}{-10}{0}%
  \TestCmp{-10}{-11}{1}%
  \TestCmp{100}{5}{1}%
  \TestCmp{9}{10}{-1}%
  \TestCmp{10}{9}{1}%
  \TestCmp{ 3 }{ 3 }{0}%
  \TestCmp{-9}{-10}{1}%
  \TestCmp{-10}{-9}{-1}%
  \TestCmp{-3}{-3}{0}%
  \TestCmp{0}{-2}{1}%
  \TestCmp{0}{2}{-1}%
  \TestCmp{2}{0}{1}%
  \TestCmp{-2}{0}{-1}%
  \TestCmp{12}{11}{1}%
  \TestCmp{11}{12}{-1}%
  \TestCmp{2147483647}{-2147483647}{1}%
  \TestCmp{-2147483647}{2147483647}{-1}%
  \TestCmp{2147483647}{2147483647}{0}%
  \TestCmp{\number\z@}{\number\@ne}{-1}%
  \TestCmp{\number\@ne}{\number\m@ne}{1}%
  \TestCmp{ 4 }{ 5 }{-1}%
  \TestCmp{ -3 }{ -7 }{1}%
\end{qstest}

\begin{qstest}{odd}{odd}
\tracingmacros=1
  \TestOdd{0}{0}%
  \TestOdd{1}{1}%
  \TestOdd{2}{0}%
  \TestOdd{3}{1}%
  \TestOdd{14}{0}%
  \TestOdd{15}{1}%
  \TestOdd{12345678901234567896}{0}%
  \TestOdd{12345678901234567897}{1}%
\end{qstest}

\begin{qstest}{inc}{inc}%
  \TestInc{0}{1}%
  \TestInc{1}{2}%
  \TestInc{-1}{0}%
  \TestInc{10}{11}%
  \TestInc{-10}{-9}%
  \TestInc{ 3 }{4}%
  \TestInc{999}{1000}%
  \TestInc{-1000}{-999}%
  \TestInc{129}{130}%
  \TestInc{2147483646}{2147483647}%
  \TestInc{-2147483647}{-2147483646}%
  \TestInc{12345678901234567890}{12345678901234567891}%
  \TestInc{99999999999999999999}{100000000000000000000}%
  \TestInc{-12345678901234567891}{-12345678901234567890}%
  \TestInc{-100000000000000000000}{-99999999999999999999}%
\end{qstest}

\begin{qstest}{dec}{dec}%
  \TestDec{0}{-1}%
  \TestDec{1}{0}%
  \TestDec{-1}{-2}%
  \TestDec{10}{9}%
  \TestDec{-10}{-11}%
  \TestDec{1000}{999}%
  \TestDec{-999}{-1000}%
  \TestDec{130}{129}%
  \TestDec{2147483647}{2147483646}%
  \TestDec{-2147483646}{-2147483647}%
  \TestDec{12345678901234567891}{12345678901234567890}%
  \TestDec{100000000000000000000}{99999999999999999999}%
  \TestDec{-12345678901234567890}{-12345678901234567891}%
  \TestDec{-99999999999999999999}{-100000000000000000000}%
\end{qstest}

\begin{qstest}{add}{add}%
  \TestAdd{0}{0}{0}%
  \TestAdd{1}{0}{1}%
  \TestAdd{0}{1}{1}%
  \TestAdd{1}{2}{3}%
  \TestAdd{-1}{-1}{-2}%
  \TestAdd{2147483646}{1}{2147483647}%
  \TestAdd{-2147483647}{2147483647}{0}%
  \TestAdd{20}{-5}{15}%
  \TestAdd{-4}{-1}{-5}%
  \TestAdd{-1}{-4}{-5}%
  \TestAdd{-4}{1}{-3}%
  \TestAdd{-1}{4}{3}%
  \TestAdd{4}{-1}{3}%
  \TestAdd{1}{-4}{-3}%
  \TestAdd{-4}{-1}{-5}%
  \TestAdd{-1}{-4}{-5}%
  \TestAdd{ -4 }{ -1 }{-5}%
  \TestAdd{ -1 }{ -4 }{-5}%
  \TestAdd{ -4 }{ 1 }{-3}%
  \TestAdd{ -1 }{ 4 }{3}%
  \TestAdd{ 4 }{ -1 }{3}%
  \TestAdd{ 1 }{ -4 }{-3}%
  \TestAdd{ -4 }{ -1 }{-5}%
  \TestAdd{ -1 }{ -4 }{-5}%
  \TestAdd{876543210}{111111111}{987654321}%
  \TestAdd{999999999}{2}{1000000001}%
\end{qstest}

\begin{qstest}{sub}{sub}
  \TestSub{0}{0}{0}%
  \TestSub{1}{0}{1}%
  \TestSub{1}{2}{-1}%
  \TestSub{-1}{-1}{0}%
  \TestSub{2147483646}{-1}{2147483647}%
  \TestSub{-2147483647}{-2147483647}{0}%
  \TestSub{-4}{-1}{-3}%
  \TestSub{-1}{-4}{3}%
  \TestSub{-4}{1}{-5}%
  \TestSub{-1}{4}{-5}%
  \TestSub{4}{-1}{5}%
  \TestSub{1}{-4}{5}%
  \TestSub{-4}{-1}{-3}%
  \TestSub{-1}{-4}{3}%
  \TestSub{ -4 }{ -1 }{-3}%
  \TestSub{ -1 }{ -4 }{3}%
  \TestSub{ -4 }{ 1 }{-5}%
  \TestSub{ -1 }{ 4 }{-5}%
  \TestSub{ 4 }{ -1 }{5}%
  \TestSub{ 1 }{ -4 }{5}%
  \TestSub{ -4 }{ -1 }{-3}%
  \TestSub{ -1 }{ -4 }{3}%
  \TestSub{1000000000}{2}{999999998}%
  \TestSub{987654321}{111111111}{876543210}%
\end{qstest}

\begin{qstest}{shl}{shl}
  \TestShl{0}{0}%
  \TestShl{1}{2}%
  \TestShl{2}{4}%
  \TestShl{5621}{11242}%
  \TestShl{1073741823}{2147483646}%
\end{qstest}

\begin{qstest}{shr}{shr}
  \TestShr{0}{0}%
  \TestShr{1}{0}%
  \TestShr{2}{1}%
  \TestShr{3}{1}%
  \TestShr{4}{2}%
  \TestShr{5}{2}%
  \TestShr{6}{3}%
  \TestShr{7}{3}%
  \TestShr{8}{4}%
  \TestShr{9}{4}%
  \TestShr{10}{5}%
  \TestShr{11}{5}%
  \TestShr{12}{6}%
  \TestShr{13}{6}%
  \TestShr{14}{7}%
  \TestShr{15}{7}%
  \TestShr{16}{8}%
  \TestShr{17}{8}%
  \TestShr{18}{9}%
  \TestShr{19}{9}%
  \TestShr{20}{10}%
  \TestShr{21}{10}%
  \TestShr{22}{11}%
  \TestShr{11241}{5620}%
  \TestShr{73054202}{36527101}%
  \TestShr{2147483646}{1073741823}%
\end{qstest}

\begin{qstest}{mul}{mul}
  \TestMul{0}{0}{0}%
  \TestMul{1}{0}{0}%
  \TestMul{0}{1}{0}%
  \TestMul{1}{1}{1}%
  \TestMul{3}{1}{3}%
  \TestMul{1}{-3}{-3}%
  \TestMul{-4}{-5}{20}%
  \TestMul{3}{7}{21}%
  \TestMul{7}{3}{21}%
  \TestMul{3}{-7}{-21}%
  \TestMul{7}{-3}{-21}%
  \TestMul{-3}{7}{-21}%
  \TestMul{-7}{3}{-21}%
  \TestMul{-3}{-7}{21}%
  \TestMul{-7}{-3}{21}%
  \TestMul{12}{11}{132}%
  \TestMul{999}{333}{332667}%
  \TestMul{1000}{4321}{4321000}%
  \TestMul{12345}{173955}{2147474475}%
  \TestMul{1073741823}{2}{2147483646}%
  \TestMul{2}{1073741823}{2147483646}%
  \TestMul{-1073741823}{2}{-2147483646}%
  \TestMul{2}{-1073741823}{-2147483646}%
  \TestMul{6706022400}{13}{87178291200}%
\end{qstest}

\begin{qstest}{sqr}{sqr}
  \TestSqr{0}{0}%
  \TestSqr{1}{1}%
  \TestSqr{2}{4}%
  \TestSqr{3}{9}%
  \TestSqr{4}{16}%
  \TestSqr{9}{81}%
  \TestSqr{10}{100}%
  \TestSqr{46340}{2147395600}%
  \TestSqr{-1}{1}%
  \TestSqr{-2}{4}%
  \TestSqr{-46340}{2147395600}%
\end{qstest}

\begin{qstest}{fac}{fac}
  \TestFac{0}{1}%
  \TestFac{1}{1}%
  \TestFac{2}{2}%
  \TestFac{3}{2*3}%
  \TestFac{4}{2*3*4}%
  \TestFac{5}{2*3*4*5}%
  \TestFac{6}{2*3*4*5*6}%
  \TestFac{7}{2*3*4*5*6*7}%
  \TestFac{8}{2*3*4*5*6*7*8}%
  \TestFac{9}{2*3*4*5*6*7*8*9}%
  \TestFac{10}{2*3*4*5*6*7*8*9*10}%
  \TestFac{11}{2*3*4*5*6*7*8*9*10*11}%
  \TestFac{12}{2*3*4*5*6*7*8*9*10*11*12}%
  \TestFacBig{13}{6227020800}%
  \TestFacBig{14}{87178291200}%
  \TestFacBig{15}{1307674368000}%
  \TestFacBig{16}{20922789888000}%
  \TestFacBig{17}{355687428096000}%
  \TestFacBig{18}{6402373705728000}%
  \TestFacBig{19}{121645100408832000}%
  \TestFacBig{20}{2432902008176640000}%
  \TestFacBig{21}{51090942171709440000}%
  \TestFacBig{22}{1124000727777607680000}%
\end{qstest}

\begin{qstest}{pow}{pow}
  \TestPow{-2}{0}{1}%
  \TestPow{-1}{0}{1}%
  \TestPow{0}{0}{1}%
  \TestPow{1}{0}{1}%
  \TestPow{2}{0}{1}%
  \TestPow{3}{0}{1}%
  \TestPow{-2}{1}{-2}%
  \TestPow{-1}{1}{-1}%
  \TestPow{1}{1}{1}%
  \TestPow{2}{1}{2}%
  \TestPow{3}{1}{3}%
  \TestPow{-2}{2}{4}%
  \TestPow{-1}{2}{1}%
  \TestPow{0}{2}{0}%
  \TestPow{1}{2}{1}%
  \TestPow{2}{2}{4}%
  \TestPow{3}{2}{9}%
  \TestPow{0}{1}{0}%
  \TestPow{1}{-2}{1}%
  \TestPow{1}{-1}{1}%
  \TestPow{-1}{-2}{1}%
  \TestPow{-1}{-1}{-1}%
  \TestPow{-1}{3}{-1}%
  \TestPow{-1}{4}{1}%
  \TestPow{-2}{-1}{0}%
  \TestPow{-2}{-2}{0}%
  \TestPow{2}{3}{8}%
  \TestPow{2}{4}{16}%
  \TestPow{2}{5}{32}%
  \TestPow{2}{6}{64}%
  \TestPow{2}{7}{128}%
  \TestPow{2}{8}{256}%
  \TestPow{2}{9}{512}%
  \TestPow{2}{10}{1024}%
  \TestPow{-2}{3}{-8}%
  \TestPow{-2}{4}{16}%
  \TestPow{-2}{5}{-32}%
  \TestPow{-2}{6}{64}%
  \TestPow{-2}{7}{-128}%
  \TestPow{-2}{8}{256}%
  \TestPow{-2}{9}{-512}%
  \TestPow{-2}{10}{1024}%
  \TestPow{3}{3}{27}%
  \TestPow{3}{4}{81}%
  \TestPow{3}{5}{243}%
  \TestPow{-3}{3}{-27}%
  \TestPow{-3}{4}{81}%
  \TestPow{-3}{5}{-243}%
  \TestPow{2}{30}{1073741824}%
  \TestPow{-3}{19}{-1162261467}%
  \TestPow{5}{13}{1220703125}%
  \TestPow{-7}{11}{-1977326743}%
\end{qstest}

\begin{qstest}{div}{div}
  \TestDiv{1}{1}{1}%
  \TestDiv{2}{1}{2}%
  \TestDiv{-2}{1}{-2}%
  \TestDiv{2}{-1}{-2}%
  \TestDiv{-2}{-1}{2}%
  \TestDiv{15}{2}{7}%
  \TestDiv{-16}{2}{-8}%
  \TestDiv{1}{2}{0}%
  \TestDiv{1}{3}{0}%
  \TestDiv{2}{3}{0}%
  \TestDiv{-2}{3}{0}%
  \TestDiv{2}{-3}{0}%
  \TestDiv{-2}{-3}{0}%
  \TestDiv{13}{3}{4}%
  \TestDiv{-13}{-3}{4}%
  \TestDiv{-13}{3}{-4}%
  \TestDiv{-6}{5}{-1}%
  \TestDiv{-5}{5}{-1}%
  \TestDiv{-4}{5}{0}%
  \TestDiv{-3}{5}{0}%
  \TestDiv{-2}{5}{0}%
  \TestDiv{-1}{5}{0}%
  \TestDiv{0}{5}{0}%
  \TestDiv{1}{5}{0}%
  \TestDiv{2}{5}{0}%
  \TestDiv{3}{5}{0}%
  \TestDiv{4}{5}{0}%
  \TestDiv{5}{5}{1}%
  \TestDiv{6}{5}{1}%
  \TestDiv{-5}{4}{-1}%
  \TestDiv{-4}{4}{-1}%
  \TestDiv{-3}{4}{0}%
  \TestDiv{-2}{4}{0}%
  \TestDiv{-1}{4}{0}%
  \TestDiv{0}{4}{0}%
  \TestDiv{1}{4}{0}%
  \TestDiv{2}{4}{0}%
  \TestDiv{3}{4}{0}%
  \TestDiv{4}{4}{1}%
  \TestDiv{5}{4}{1}%
  \TestDiv{12345}{678}{18}%
  \TestDiv{32372}{5952}{5}%
  \TestDiv{284271294}{18162}{15651}%
  \TestDiv{217652429}{12561}{17327}%
  \TestDiv{462028434}{5439}{84947}%
  \TestDiv{2147483647}{1000}{2147483}%
  \TestDiv{2147483647}{-1000}{-2147483}%
  \TestDiv{-2147483647}{1000}{-2147483}%
  \TestDiv{-2147483647}{-1000}{2147483}%
  \TestDiv{0}{3}{0}%
  \TestDiv{1}{3}{0}%
  \TestDiv{2}{3}{0}%
  \TestDiv{3}{3}{1}%
  \TestDiv{4}{3}{1}%
  \TestDiv{5}{3}{1}%
  \TestDiv{6}{3}{2}%
  \TestDiv{7}{3}{2}%
  \TestDiv{8}{3}{2}%
  \TestDiv{9}{3}{3}%
  \TestDiv{10}{3}{3}%
  \TestDiv{11}{3}{3}%
  \TestDiv{12}{3}{4}%
  \TestDiv{13}{3}{4}%
  \TestDiv{14}{3}{4}%
  \TestDiv{15}{3}{5}%
  \TestDiv{16}{3}{5}%
  \TestDiv{17}{3}{5}%
  \TestDiv{18}{3}{6}%
  \TestDiv{19}{3}{6}%
  \TestDiv{20}{3}{6}%
  \TestDiv{21}{3}{7}%
  \TestDiv{22}{3}{7}%
  \TestDiv{23}{3}{7}%
  \TestDiv{24}{3}{8}%
  \TestDiv{25}{3}{8}%
  \TestDiv{26}{3}{8}%
  \TestDiv{27}{3}{9}%
  \TestDiv{28}{3}{9}%
  \TestDiv{29}{3}{9}%
  \TestDiv{30}{3}{10}%
  \TestDiv{31}{3}{10}%
  \TestDivBig{17363436332507}{24702}{702916214}%
\end{qstest}

\begin{qstest}{mod}{mod}
  \TestMod{-6}{5}{4}%
  \TestMod{-5}{5}{0}%
  \TestMod{-4}{5}{1}%
  \TestMod{-3}{5}{2}%
  \TestMod{-2}{5}{3}%
  \TestMod{-1}{5}{4}%
  \TestMod{0}{5}{0}%
  \TestMod{1}{5}{1}%
  \TestMod{2}{5}{2}%
  \TestMod{3}{5}{3}%
  \TestMod{4}{5}{4}%
  \TestMod{5}{5}{0}%
  \TestMod{6}{5}{1}%
  \TestMod{-5}{4}{3}%
  \TestMod{-4}{4}{0}%
  \TestMod{-3}{4}{1}%
  \TestMod{-2}{4}{2}%
  \TestMod{-1}{4}{3}%
  \TestMod{0}{4}{0}%
  \TestMod{1}{4}{1}%
  \TestMod{2}{4}{2}%
  \TestMod{3}{4}{3}%
  \TestMod{4}{4}{0}%
  \TestMod{5}{4}{1}%
  \TestMod{-6}{-5}{-1}%
  \TestMod{-5}{-5}{0}%
  \TestMod{-4}{-5}{-4}%
  \TestMod{-3}{-5}{-3}%
  \TestMod{-2}{-5}{-2}%
  \TestMod{-1}{-5}{-1}%
  \TestMod{0}{-5}{0}%
  \TestMod{1}{-5}{-4}%
  \TestMod{2}{-5}{-3}%
  \TestMod{3}{-5}{-2}%
  \TestMod{4}{-5}{-1}%
  \TestMod{5}{-5}{0}%
  \TestMod{6}{-5}{-4}%
  \TestMod{-5}{-4}{-1}%
  \TestMod{-4}{-4}{0}%
  \TestMod{-3}{-4}{-3}%
  \TestMod{-2}{-4}{-2}%
  \TestMod{-1}{-4}{-1}%
  \TestMod{0}{-4}{0}%
  \TestMod{1}{-4}{-3}%
  \TestMod{2}{-4}{-2}%
  \TestMod{3}{-4}{-1}%
  \TestMod{4}{-4}{0}%
  \TestMod{5}{-4}{-3}%
  \TestMod{2147483647}{1000}{647}%
  \TestMod{2147483647}{-1000}{-353}%
  \TestMod{-2147483647}{1000}{353}%
  \TestMod{-2147483647}{-1000}{-647}%
  \TestMod{ 0 }{ 4 }{0}%
  \TestMod{ 1 }{ 4 }{1}%
  \TestMod{ -1 }{ 4 }{3}%
  \TestMod{ 0 }{ -4 }{0}%
  \TestMod{ 1 }{ -4 }{-3}%
  \TestMod{ -1 }{ -4 }{-1}%
  \TestMod{18362}{25}{12}%
\end{qstest}

\newcommand*{\TestError}[2]{%
  \begingroup
    \expandafter\def\csname BigIntCalcError:#1\endcsname{}%
    \Expect*{#2}{0}%
    \expandafter\def\csname BigIntCalcError:#1\endcsname{ERROR}%
    \Expect*{#2}{0ERROR}%
  \endgroup
}
\begin{qstest}{error}{error}
  \TestError{FacNegative}{\bigintcalcFac{-1}}%
  \TestError{FacNegative}{\bigintcalcFac{-2147483647}}%
  \TestError{DivisionByZero}{\bigintcalcPow{0}{-1}}%
  \TestError{DivisionByZero}{\bigintcalcDiv{1}{0}}%
  \TestError{DivisionByZero}{\bigintcalcMod{1}{0}}%
\end{qstest}

\begin{document}
\end{document}
%</test2>
%    \end{macrocode}
%
% \section{Installation}
%
% \subsection{Download}
%
% \paragraph{Package.} This package is available on
% CTAN\footnote{\url{http://ctan.org/pkg/bigintcalc}}:
% \begin{description}
% \item[\CTAN{macros/latex/contrib/oberdiek/bigintcalc.dtx}] The source file.
% \item[\CTAN{macros/latex/contrib/oberdiek/bigintcalc.pdf}] Documentation.
% \end{description}
%
%
% \paragraph{Bundle.} All the packages of the bundle `oberdiek'
% are also available in a TDS compliant ZIP archive. There
% the packages are already unpacked and the documentation files
% are generated. The files and directories obey the TDS standard.
% \begin{description}
% \item[\CTAN{install/macros/latex/contrib/oberdiek.tds.zip}]
% \end{description}
% \emph{TDS} refers to the standard ``A Directory Structure
% for \TeX\ Files'' (\CTAN{tds/tds.pdf}). Directories
% with \xfile{texmf} in their name are usually organized this way.
%
% \subsection{Bundle installation}
%
% \paragraph{Unpacking.} Unpack the \xfile{oberdiek.tds.zip} in the
% TDS tree (also known as \xfile{texmf} tree) of your choice.
% Example (linux):
% \begin{quote}
%   |unzip oberdiek.tds.zip -d ~/texmf|
% \end{quote}
%
% \paragraph{Script installation.}
% Check the directory \xfile{TDS:scripts/oberdiek/} for
% scripts that need further installation steps.
% Package \xpackage{attachfile2} comes with the Perl script
% \xfile{pdfatfi.pl} that should be installed in such a way
% that it can be called as \texttt{pdfatfi}.
% Example (linux):
% \begin{quote}
%   |chmod +x scripts/oberdiek/pdfatfi.pl|\\
%   |cp scripts/oberdiek/pdfatfi.pl /usr/local/bin/|
% \end{quote}
%
% \subsection{Package installation}
%
% \paragraph{Unpacking.} The \xfile{.dtx} file is a self-extracting
% \docstrip\ archive. The files are extracted by running the
% \xfile{.dtx} through \plainTeX:
% \begin{quote}
%   \verb|tex bigintcalc.dtx|
% \end{quote}
%
% \paragraph{TDS.} Now the different files must be moved into
% the different directories in your installation TDS tree
% (also known as \xfile{texmf} tree):
% \begin{quote}
% \def\t{^^A
% \begin{tabular}{@{}>{\ttfamily}l@{ $\rightarrow$ }>{\ttfamily}l@{}}
%   bigintcalc.sty & tex/generic/oberdiek/bigintcalc.sty\\
%   bigintcalc.pdf & doc/latex/oberdiek/bigintcalc.pdf\\
%   test/bigintcalc-test1.tex & doc/latex/oberdiek/test/bigintcalc-test1.tex\\
%   test/bigintcalc-test2.tex & doc/latex/oberdiek/test/bigintcalc-test2.tex\\
%   test/bigintcalc-test3.tex & doc/latex/oberdiek/test/bigintcalc-test3.tex\\
%   bigintcalc.dtx & source/latex/oberdiek/bigintcalc.dtx\\
% \end{tabular}^^A
% }^^A
% \sbox0{\t}^^A
% \ifdim\wd0>\linewidth
%   \begingroup
%     \advance\linewidth by\leftmargin
%     \advance\linewidth by\rightmargin
%   \edef\x{\endgroup
%     \def\noexpand\lw{\the\linewidth}^^A
%   }\x
%   \def\lwbox{^^A
%     \leavevmode
%     \hbox to \linewidth{^^A
%       \kern-\leftmargin\relax
%       \hss
%       \usebox0
%       \hss
%       \kern-\rightmargin\relax
%     }^^A
%   }^^A
%   \ifdim\wd0>\lw
%     \sbox0{\small\t}^^A
%     \ifdim\wd0>\linewidth
%       \ifdim\wd0>\lw
%         \sbox0{\footnotesize\t}^^A
%         \ifdim\wd0>\linewidth
%           \ifdim\wd0>\lw
%             \sbox0{\scriptsize\t}^^A
%             \ifdim\wd0>\linewidth
%               \ifdim\wd0>\lw
%                 \sbox0{\tiny\t}^^A
%                 \ifdim\wd0>\linewidth
%                   \lwbox
%                 \else
%                   \usebox0
%                 \fi
%               \else
%                 \lwbox
%               \fi
%             \else
%               \usebox0
%             \fi
%           \else
%             \lwbox
%           \fi
%         \else
%           \usebox0
%         \fi
%       \else
%         \lwbox
%       \fi
%     \else
%       \usebox0
%     \fi
%   \else
%     \lwbox
%   \fi
% \else
%   \usebox0
% \fi
% \end{quote}
% If you have a \xfile{docstrip.cfg} that configures and enables \docstrip's
% TDS installing feature, then some files can already be in the right
% place, see the documentation of \docstrip.
%
% \subsection{Refresh file name databases}
%
% If your \TeX~distribution
% (\teTeX, \mikTeX, \dots) relies on file name databases, you must refresh
% these. For example, \teTeX\ users run \verb|texhash| or
% \verb|mktexlsr|.
%
% \subsection{Some details for the interested}
%
% \paragraph{Attached source.}
%
% The PDF documentation on CTAN also includes the
% \xfile{.dtx} source file. It can be extracted by
% AcrobatReader 6 or higher. Another option is \textsf{pdftk},
% e.g. unpack the file into the current directory:
% \begin{quote}
%   \verb|pdftk bigintcalc.pdf unpack_files output .|
% \end{quote}
%
% \paragraph{Unpacking with \LaTeX.}
% The \xfile{.dtx} chooses its action depending on the format:
% \begin{description}
% \item[\plainTeX:] Run \docstrip\ and extract the files.
% \item[\LaTeX:] Generate the documentation.
% \end{description}
% If you insist on using \LaTeX\ for \docstrip\ (really,
% \docstrip\ does not need \LaTeX), then inform the autodetect routine
% about your intention:
% \begin{quote}
%   \verb|latex \let\install=y% \iffalse meta-comment
%
% File: bigintcalc.dtx
% Version: 2016/05/16 v1.4
% Info: Expandable calculations on big integers
%
% Copyright (C) 2007, 2011, 2012 by
%    Heiko Oberdiek <heiko.oberdiek at googlemail.com>
%    2016
%    https://github.com/ho-tex/oberdiek/issues
%
% This work may be distributed and/or modified under the
% conditions of the LaTeX Project Public License, either
% version 1.3c of this license or (at your option) any later
% version. This version of this license is in
%    http://www.latex-project.org/lppl/lppl-1-3c.txt
% and the latest version of this license is in
%    http://www.latex-project.org/lppl.txt
% and version 1.3 or later is part of all distributions of
% LaTeX version 2005/12/01 or later.
%
% This work has the LPPL maintenance status "maintained".
%
% This Current Maintainer of this work is Heiko Oberdiek.
%
% The Base Interpreter refers to any `TeX-Format',
% because some files are installed in TDS:tex/generic//.
%
% This work consists of the main source file bigintcalc.dtx
% and the derived files
%    bigintcalc.sty, bigintcalc.pdf, bigintcalc.ins, bigintcalc.drv,
%    bigintcalc-test1.tex, bigintcalc-test2.tex,
%    bigintcalc-test3.tex.
%
% Distribution:
%    CTAN:macros/latex/contrib/oberdiek/bigintcalc.dtx
%    CTAN:macros/latex/contrib/oberdiek/bigintcalc.pdf
%
% Unpacking:
%    (a) If bigintcalc.ins is present:
%           tex bigintcalc.ins
%    (b) Without bigintcalc.ins:
%           tex bigintcalc.dtx
%    (c) If you insist on using LaTeX
%           latex \let\install=y% \iffalse meta-comment
%
% File: bigintcalc.dtx
% Version: 2016/05/16 v1.4
% Info: Expandable calculations on big integers
%
% Copyright (C) 2007, 2011, 2012 by
%    Heiko Oberdiek <heiko.oberdiek at googlemail.com>
%    2016
%    https://github.com/ho-tex/oberdiek/issues
%
% This work may be distributed and/or modified under the
% conditions of the LaTeX Project Public License, either
% version 1.3c of this license or (at your option) any later
% version. This version of this license is in
%    http://www.latex-project.org/lppl/lppl-1-3c.txt
% and the latest version of this license is in
%    http://www.latex-project.org/lppl.txt
% and version 1.3 or later is part of all distributions of
% LaTeX version 2005/12/01 or later.
%
% This work has the LPPL maintenance status "maintained".
%
% This Current Maintainer of this work is Heiko Oberdiek.
%
% The Base Interpreter refers to any `TeX-Format',
% because some files are installed in TDS:tex/generic//.
%
% This work consists of the main source file bigintcalc.dtx
% and the derived files
%    bigintcalc.sty, bigintcalc.pdf, bigintcalc.ins, bigintcalc.drv,
%    bigintcalc-test1.tex, bigintcalc-test2.tex,
%    bigintcalc-test3.tex.
%
% Distribution:
%    CTAN:macros/latex/contrib/oberdiek/bigintcalc.dtx
%    CTAN:macros/latex/contrib/oberdiek/bigintcalc.pdf
%
% Unpacking:
%    (a) If bigintcalc.ins is present:
%           tex bigintcalc.ins
%    (b) Without bigintcalc.ins:
%           tex bigintcalc.dtx
%    (c) If you insist on using LaTeX
%           latex \let\install=y\input{bigintcalc.dtx}
%        (quote the arguments according to the demands of your shell)
%
% Documentation:
%    (a) If bigintcalc.drv is present:
%           latex bigintcalc.drv
%    (b) Without bigintcalc.drv:
%           latex bigintcalc.dtx; ...
%    The class ltxdoc loads the configuration file ltxdoc.cfg
%    if available. Here you can specify further options, e.g.
%    use A4 as paper format:
%       \PassOptionsToClass{a4paper}{article}
%
%    Programm calls to get the documentation (example):
%       pdflatex bigintcalc.dtx
%       makeindex -s gind.ist bigintcalc.idx
%       pdflatex bigintcalc.dtx
%       makeindex -s gind.ist bigintcalc.idx
%       pdflatex bigintcalc.dtx
%
% Installation:
%    TDS:tex/generic/oberdiek/bigintcalc.sty
%    TDS:doc/latex/oberdiek/bigintcalc.pdf
%    TDS:doc/latex/oberdiek/test/bigintcalc-test1.tex
%    TDS:doc/latex/oberdiek/test/bigintcalc-test2.tex
%    TDS:doc/latex/oberdiek/test/bigintcalc-test3.tex
%    TDS:source/latex/oberdiek/bigintcalc.dtx
%
%<*ignore>
\begingroup
  \catcode123=1 %
  \catcode125=2 %
  \def\x{LaTeX2e}%
\expandafter\endgroup
\ifcase 0\ifx\install y1\fi\expandafter
         \ifx\csname processbatchFile\endcsname\relax\else1\fi
         \ifx\fmtname\x\else 1\fi\relax
\else\csname fi\endcsname
%</ignore>
%<*install>
\input docstrip.tex
\Msg{************************************************************************}
\Msg{* Installation}
\Msg{* Package: bigintcalc 2016/05/16 v1.4 Expandable calculations on big integers (HO)}
\Msg{************************************************************************}

\keepsilent
\askforoverwritefalse

\let\MetaPrefix\relax
\preamble

This is a generated file.

Project: bigintcalc
Version: 2016/05/16 v1.4

Copyright (C) 2007, 2011, 2012 by
   Heiko Oberdiek <heiko.oberdiek at googlemail.com>

This work may be distributed and/or modified under the
conditions of the LaTeX Project Public License, either
version 1.3c of this license or (at your option) any later
version. This version of this license is in
   http://www.latex-project.org/lppl/lppl-1-3c.txt
and the latest version of this license is in
   http://www.latex-project.org/lppl.txt
and version 1.3 or later is part of all distributions of
LaTeX version 2005/12/01 or later.

This work has the LPPL maintenance status "maintained".

This Current Maintainer of this work is Heiko Oberdiek.

The Base Interpreter refers to any `TeX-Format',
because some files are installed in TDS:tex/generic//.

This work consists of the main source file bigintcalc.dtx
and the derived files
   bigintcalc.sty, bigintcalc.pdf, bigintcalc.ins, bigintcalc.drv,
   bigintcalc-test1.tex, bigintcalc-test2.tex,
   bigintcalc-test3.tex.

\endpreamble
\let\MetaPrefix\DoubleperCent

\generate{%
  \file{bigintcalc.ins}{\from{bigintcalc.dtx}{install}}%
  \file{bigintcalc.drv}{\from{bigintcalc.dtx}{driver}}%
  \usedir{tex/generic/oberdiek}%
  \file{bigintcalc.sty}{\from{bigintcalc.dtx}{package}}%
%  \usedir{doc/latex/oberdiek/test}%
%  \file{bigintcalc-test1.tex}{\from{bigintcalc.dtx}{test1}}%
%  \file{bigintcalc-test2.tex}{\from{bigintcalc.dtx}{test2,etex}}%
%  \file{bigintcalc-test3.tex}{\from{bigintcalc.dtx}{test2,noetex}}%
  \nopreamble
  \nopostamble
  \usedir{source/latex/oberdiek/catalogue}%
  \file{bigintcalc.xml}{\from{bigintcalc.dtx}{catalogue}}%
}

\catcode32=13\relax% active space
\let =\space%
\Msg{************************************************************************}
\Msg{*}
\Msg{* To finish the installation you have to move the following}
\Msg{* file into a directory searched by TeX:}
\Msg{*}
\Msg{*     bigintcalc.sty}
\Msg{*}
\Msg{* To produce the documentation run the file `bigintcalc.drv'}
\Msg{* through LaTeX.}
\Msg{*}
\Msg{* Happy TeXing!}
\Msg{*}
\Msg{************************************************************************}

\endbatchfile
%</install>
%<*ignore>
\fi
%</ignore>
%<*driver>
\NeedsTeXFormat{LaTeX2e}
\ProvidesFile{bigintcalc.drv}%
  [2016/05/16 v1.4 Expandable calculations on big integers (HO)]%
\documentclass{ltxdoc}
\usepackage{wasysym}
\let\iint\relax
\let\iiint\relax
\usepackage[fleqn]{amsmath}

\DeclareMathOperator{\opInv}{Inv}
\DeclareMathOperator{\opAbs}{Abs}
\DeclareMathOperator{\opSgn}{Sgn}
\DeclareMathOperator{\opMin}{Min}
\DeclareMathOperator{\opMax}{Max}
\DeclareMathOperator{\opCmp}{Cmp}
\DeclareMathOperator{\opOdd}{Odd}
\DeclareMathOperator{\opInc}{Inc}
\DeclareMathOperator{\opDec}{Dec}
\DeclareMathOperator{\opAdd}{Add}
\DeclareMathOperator{\opSub}{Sub}
\DeclareMathOperator{\opShl}{Shl}
\DeclareMathOperator{\opShr}{Shr}
\DeclareMathOperator{\opMul}{Mul}
\DeclareMathOperator{\opSqr}{Sqr}
\DeclareMathOperator{\opFac}{Fac}
\DeclareMathOperator{\opPow}{Pow}
\DeclareMathOperator{\opDiv}{Div}
\DeclareMathOperator{\opMod}{Mod}
\DeclareMathOperator{\opInt}{Int}
\DeclareMathOperator{\opODD}{ifodd}

\newcommand*{\Def}{%
  \ensuremath{%
    \mathrel{\mathop{:}}=%
  }%
}
\newcommand*{\op}[1]{%
  \textsf{#1}%
}
\usepackage{holtxdoc}[2011/11/22]
\begin{document}
  \DocInput{bigintcalc.dtx}%
\end{document}
%</driver>
% \fi
%
%
% \CharacterTable
%  {Upper-case    \A\B\C\D\E\F\G\H\I\J\K\L\M\N\O\P\Q\R\S\T\U\V\W\X\Y\Z
%   Lower-case    \a\b\c\d\e\f\g\h\i\j\k\l\m\n\o\p\q\r\s\t\u\v\w\x\y\z
%   Digits        \0\1\2\3\4\5\6\7\8\9
%   Exclamation   \!     Double quote  \"     Hash (number) \#
%   Dollar        \$     Percent       \%     Ampersand     \&
%   Acute accent  \'     Left paren    \(     Right paren   \)
%   Asterisk      \*     Plus          \+     Comma         \,
%   Minus         \-     Point         \.     Solidus       \/
%   Colon         \:     Semicolon     \;     Less than     \<
%   Equals        \=     Greater than  \>     Question mark \?
%   Commercial at \@     Left bracket  \[     Backslash     \\
%   Right bracket \]     Circumflex    \^     Underscore    \_
%   Grave accent  \`     Left brace    \{     Vertical bar  \|
%   Right brace   \}     Tilde         \~}
%
% \GetFileInfo{bigintcalc.drv}
%
% \title{The \xpackage{bigintcalc} package}
% \date{2016/05/16 v1.4}
% \author{Heiko Oberdiek\thanks
% {Please report any issues at https://github.com/ho-tex/oberdiek/issues}\\
% \xemail{heiko.oberdiek at googlemail.com}}
%
% \maketitle
%
% \begin{abstract}
% This package provides expandable arithmetic operations
% with big integers that can exceed \TeX's number limits.
% \end{abstract}
%
% \tableofcontents
%
% \section{Documentation}
%
% \subsection{Introduction}
%
% Package \xpackage{bigintcalc} defines arithmetic operations
% that deal with big integers. Big integers can be given
% either as explicit integer number or as macro code
% that expands to an explicit number. \emph{Big} means that
% there is no limit on the size of the number. Big integers
% may exceed \TeX's range limitation of -2147483647 and 2147483647.
% Only memory issues will limit the usable range.
%
% In opposite to package \xpackage{intcalc}
% unexpandable command tokens are not supported, even if
% they are valid \TeX\ numbers like count registers or
% commands created by \cs{chardef}. Nevertheless they may
% be used, if they are prefixed by \cs{number}.
%
% Also \eTeX's \cs{numexpr} expressions are not supported
% directly in the manner of package \xpackage{intcalc}.
% However they can be given if \cs{the}\cs{numexpr}
% or \cs{number}\cs{numexpr} are used.
%
% The operations have the form of macros that take one or
% two integers as parameter and return the integer result.
% The macro name is a three letter operation name prefixed
% by the package name, e.g. \cs{bigintcalcAdd}|{10}{43}| returns
% |53|.
%
% The macros are fully expandable, exactly two expansion
% steps generate the result. Therefore the operations
% may be used nearly everywhere in \TeX, even inside
% \cs{csname}, file names,  or other
% expandable contexts.
%
% \subsection{Conditions}
%
% \subsubsection{Preconditions}
%
% \begin{itemize}
% \item
%   Arguments can be anything that expands to a number that consists
%   of optional signs and digits.
% \item
%   The arguments and return values must be sound.
%   Zero as divisor or factorials of negative numbers will cause errors.
% \end{itemize}
%
% \subsubsection{Postconditions}
%
% Additional properties of the macros apart from calculating
% a correct result (of course \smiley):
% \begin{itemize}
% \item
%   The macros are fully expandable. Thus they can
%   be used inside \cs{edef}, \cs{csname},
%   for example.
% \item
%   Furthermore exactly two expansion steps calculate the result.
% \item
%   The number consists of one optional minus sign and one or more
%   digits. The first digit is larger than zero for
%   numbers that consists of more than one digit.
%
%   In short, the number format is exactly the same as
%   \cs{number} generates, but without its range limitation.
%   And the tokens (minus sign, digits)
%   have catcode 12 (other).
% \item
%   Call by value is simulated. First the arguments are
%   converted to numbers. Then these numbers are used
%   in the calculations.
%
%   Remember that arguments
%   may contain expensive macros or \eTeX\ expressions.
%   This strategy avoids multiple evaluations of such
%   arguments.
% \end{itemize}
%
% \subsection{Error handling}
%
% Some errors are detected by the macros, example: division by zero.
% In this cases an undefined control sequence is called and causes
% a TeX error message, example: \cs{BigIntCalcError:DivisionByZero}.
% The name of the control sequence contains
% the reason for the error. The \TeX\ error may be ignored.
% Then the operation returns zero as result.
% Because the macros are supposed to work in expandible contexts.
% An traditional error message, however, is not expandable and
% would break these contexts.
%
% \subsection{Operations}
%
% Some definition equations below use the function $\opInt$
% that converts a real number to an integer. The number
% is truncated that means rounding to zero:
% \begin{gather*}
%   \opInt(x) \Def
%   \begin{cases}
%     \lfloor x\rfloor & \text{if $x\geq0$}\\
%     \lceil x\rceil & \text{otherwise}
%   \end{cases}
% \end{gather*}
%
% \subsubsection{\op{Num}}
%
% \begin{declcs}{bigintcalcNum} \M{x}
% \end{declcs}
% Macro \cs{bigintcalcNum} converts its argument to a normalized integer
% number without unnecessary leading zeros or signs.
% The result matches the regular expression:
%\begin{quote}
%\begin{verbatim}
%0|-?[1-9][0-9]*
%\end{verbatim}
%\end{quote}
%
% \subsubsection{\op{Inv}, \op{Abs}, \op{Sgn}}
%
% \begin{declcs}{bigintcalcInv} \M{x}
% \end{declcs}
% Macro \cs{bigintcalcInv} switches the sign.
% \begin{gather*}
%   \opInv(x) \Def -x
% \end{gather*}
%
% \begin{declcs}{bigintcalcAbs} \M{x}
% \end{declcs}
% Macro \cs{bigintcalcAbs} returns the absolute value
% of integer \meta{x}.
% \begin{gather*}
%   \opAbs(x) \Def \vert x\vert
% \end{gather*}
%
% \begin{declcs}{bigintcalcSgn} \M{x}
% \end{declcs}
% Macro \cs{bigintcalcSgn} encodes the sign of \meta{x} as number.
% \begin{gather*}
%   \opSgn(x) \Def
%   \begin{cases}
%     -1& \text{if $x<0$}\\
%      0& \text{if $x=0$}\\
%      1& \text{if $x>0$}
%   \end{cases}
% \end{gather*}
% These return values can easily be distinguished by \cs{ifcase}:
%\begin{quote}
%\begin{verbatim}
%\ifcase\bigintcalcSgn{<x>}
%  $x=0$
%\or
%  $x>0$
%\else
%  $x<0$
%\fi
%\end{verbatim}
%\end{quote}
%
% \subsubsection{\op{Min}, \op{Max}, \op{Cmp}}
%
% \begin{declcs}{bigintcalcMin} \M{x} \M{y}
% \end{declcs}
% Macro \cs{bigintcalcMin} returns the smaller of the two integers.
% \begin{gather*}
%   \opMin(x,y) \Def
%   \begin{cases}
%     x & \text{if $x<y$}\\
%     y & \text{otherwise}
%   \end{cases}
% \end{gather*}
%
% \begin{declcs}{bigintcalcMax} \M{x} \M{y}
% \end{declcs}
% Macro \cs{bigintcalcMax} returns the larger of the two integers.
% \begin{gather*}
%   \opMax(x,y) \Def
%   \begin{cases}
%     x & \text{if $x>y$}\\
%     y & \text{otherwise}
%   \end{cases}
% \end{gather*}
%
% \begin{declcs}{bigintcalcCmp} \M{x} \M{y}
% \end{declcs}
% Macro \cs{bigintcalcCmp} encodes the comparison result as number:
% \begin{gather*}
%   \opCmp(x,y) \Def
%   \begin{cases}
%     -1 & \text{if $x<y$}\\
%      0 & \text{if $x=y$}\\
%      1 & \text{if $x>y$}
%   \end{cases}
% \end{gather*}
% These values can be distinguished by \cs{ifcase}:
%\begin{quote}
%\begin{verbatim}
%\ifcase\bigintcalcCmp{<x>}{<y>}
%  $x=y$
%\or
%  $x>y$
%\else
%  $x<y$
%\fi
%\end{verbatim}
%\end{quote}
%
% \subsubsection{\op{Odd}}
%
% \begin{declcs}{bigintcalcOdd} \M{x}
% \end{declcs}
% \begin{gather*}
%   \opOdd(x) \Def
%   \begin{cases}
%     1 & \text{if $x$ is odd}\\
%     0 & \text{if $x$ is even}
%   \end{cases}
% \end{gather*}
%
% \subsubsection{\op{Inc}, \op{Dec}, \op{Add}, \op{Sub}}
%
% \begin{declcs}{bigintcalcInc} \M{x}
% \end{declcs}
% Macro \cs{bigintcalcInc} increments \meta{x} by one.
% \begin{gather*}
%   \opInc(x) \Def x + 1
% \end{gather*}
%
% \begin{declcs}{bigintcalcDec} \M{x}
% \end{declcs}
% Macro \cs{bigintcalcDec} decrements \meta{x} by one.
% \begin{gather*}
%   \opDec(x) \Def x - 1
% \end{gather*}
%
% \begin{declcs}{bigintcalcAdd} \M{x} \M{y}
% \end{declcs}
% Macro \cs{bigintcalcAdd} adds the two numbers.
% \begin{gather*}
%   \opAdd(x, y) \Def x + y
% \end{gather*}
%
% \begin{declcs}{bigintcalcSub} \M{x} \M{y}
% \end{declcs}
% Macro \cs{bigintcalcSub} calculates the difference.
% \begin{gather*}
%   \opSub(x, y) \Def x - y
% \end{gather*}
%
% \subsubsection{\op{Shl}, \op{Shr}}
%
% \begin{declcs}{bigintcalcShl} \M{x}
% \end{declcs}
% Macro \cs{bigintcalcShl} implements shifting to the left that
% means the number is multiplied by two.
% The sign is preserved.
% \begin{gather*}
%   \opShl(x) \Def x*2
% \end{gather*}
%
% \begin{declcs}{bigintcalcShr} \M{x}
% \end{declcs}
% Macro \cs{bigintcalcShr} implements shifting to the right.
% That is equivalent to an integer division by two.
% The sign is preserved.
% \begin{gather*}
%   \opShr(x) \Def \opInt(x/2)
% \end{gather*}
%
% \subsubsection{\op{Mul}, \op{Sqr}, \op{Fac}, \op{Pow}}
%
% \begin{declcs}{bigintcalcMul} \M{x} \M{y}
% \end{declcs}
% Macro \cs{bigintcalcMul} calculates the product of
% \meta{x} and \meta{y}.
% \begin{gather*}
%   \opMul(x,y) \Def x*y
% \end{gather*}
%
% \begin{declcs}{bigintcalcSqr} \M{x}
% \end{declcs}
% Macro \cs{bigintcalcSqr} returns the square product.
% \begin{gather*}
%   \opSqr(x) \Def x^2
% \end{gather*}
%
% \begin{declcs}{bigintcalcFac} \M{x}
% \end{declcs}
% Macro \cs{bigintcalcFac} returns the factorial of \meta{x}.
% Negative numbers are not permitted.
% \begin{gather*}
%   \opFac(x) \Def x!\qquad\text{for $x\geq0$}
% \end{gather*}
% ($0! = 1$)
%
% \begin{declcs}{bigintcalcPow} M{x} M{y}
% \end{declcs}
% Macro \cs{bigintcalcPow} calculates the value of \meta{x} to the
% power of \meta{y}. The error ``division by zero'' is thrown
% if \meta{x} is zero and \meta{y} is negative.
% permitted:
% \begin{gather*}
%   \opPow(x,y) \Def
%   \opInt(x^y)\qquad\text{for $x\neq0$ or $y\geq0$}
% \end{gather*}
% ($0^0 = 1$)
%
% \subsubsection{\op{Div}, \op{Mul}}
%
% \begin{declcs}{bigintcalcDiv} \M{x} \M{y}
% \end{declcs}
% Macro \cs{bigintcalcDiv} performs an integer division.
% Argument \meta{y} must not be zero.
% \begin{gather*}
%   \opDiv(x,y) \Def \opInt(x/y)\qquad\text{for $y\neq0$}
% \end{gather*}
%
% \begin{declcs}{bigintcalcMod} \M{x} \M{y}
% \end{declcs}
% Macro \cs{bigintcalcMod} gets the remainder of the integer
% division. The sign follows the divisor \meta{y}.
% Argument \meta{y} must not be zero.
% \begin{gather*}
%   \opMod(x,y) \Def x\mathrel{\%}y\qquad\text{for $y\neq0$}
% \end{gather*}
% The result ranges:
% \begin{gather*}
%   -\vert y\vert < \opMod(x,y) \leq 0\qquad\text{for $y<0$}\\
%   0 \leq \opMod(x,y) < y\qquad\text{for $y\geq0$}
% \end{gather*}
%
% \subsection{Interface for programmers}
%
% If the programmer can ensure some more properties about
% the arguments of the operations, then the following
% macros are a little more efficient.
%
% In general numbers must obey the following constraints:
% \begin{itemize}
% \item Plain number: digit tokens only, no command tokens.
% \item Non-negative. Signs are forbidden.
% \item Delimited by exclamation mark. Curly braces
%       around the number are not allowed and will
%       break the code.
% \end{itemize}
%
% \begin{declcs}{BigIntCalcOdd} \meta{number} |!|
% \end{declcs}
% |1|/|0| is returned if \meta{number} is odd/even.
%
% \begin{declcs}{BigIntCalcInc} \meta{number} |!|
% \end{declcs}
% Incrementation.
%
% \begin{declcs}{BigIntCalcDec} \meta{number} |!|
% \end{declcs}
% Decrementation, positive number without zero.
%
% \begin{declcs}{BigIntCalcAdd} \meta{number A} |!| \meta{number B} |!|
% \end{declcs}
% Addition, $A\geq B$.
%
% \begin{declcs}{BigIntCalcSub} \meta{number A} |!| \meta{number B} |!|
% \end{declcs}
% Subtraction, $A\geq B$.
%
% \begin{declcs}{BigIntCalcShl} \meta{number} |!|
% \end{declcs}
% Left shift (multiplication with two).
%
% \begin{declcs}{BigIntCalcShr} \meta{number} |!|
% \end{declcs}
% Right shift (integer division by two).
%
% \begin{declcs}{BigIntCalcMul} \meta{number A} |!| \meta{number B} |!|
% \end{declcs}
% Multiplication, $A\geq B$.
%
% \begin{declcs}{BigIntCalcDiv} \meta{number A} |!| \meta{number B} |!|
% \end{declcs}
% Division operation.
%
% \begin{declcs}{BigIntCalcMod} \meta{number A} |!| \meta{number B} |!|
% \end{declcs}
% Modulo operation.
%
%
% \StopEventually{
% }
%
% \section{Implementation}
%
%    \begin{macrocode}
%<*package>
%    \end{macrocode}
%
% \subsection{Reload check and package identification}
%    Reload check, especially if the package is not used with \LaTeX.
%    \begin{macrocode}
\begingroup\catcode61\catcode48\catcode32=10\relax%
  \catcode13=5 % ^^M
  \endlinechar=13 %
  \catcode35=6 % #
  \catcode39=12 % '
  \catcode44=12 % ,
  \catcode45=12 % -
  \catcode46=12 % .
  \catcode58=12 % :
  \catcode64=11 % @
  \catcode123=1 % {
  \catcode125=2 % }
  \expandafter\let\expandafter\x\csname ver@bigintcalc.sty\endcsname
  \ifx\x\relax % plain-TeX, first loading
  \else
    \def\empty{}%
    \ifx\x\empty % LaTeX, first loading,
      % variable is initialized, but \ProvidesPackage not yet seen
    \else
      \expandafter\ifx\csname PackageInfo\endcsname\relax
        \def\x#1#2{%
          \immediate\write-1{Package #1 Info: #2.}%
        }%
      \else
        \def\x#1#2{\PackageInfo{#1}{#2, stopped}}%
      \fi
      \x{bigintcalc}{The package is already loaded}%
      \aftergroup\endinput
    \fi
  \fi
\endgroup%
%    \end{macrocode}
%    Package identification:
%    \begin{macrocode}
\begingroup\catcode61\catcode48\catcode32=10\relax%
  \catcode13=5 % ^^M
  \endlinechar=13 %
  \catcode35=6 % #
  \catcode39=12 % '
  \catcode40=12 % (
  \catcode41=12 % )
  \catcode44=12 % ,
  \catcode45=12 % -
  \catcode46=12 % .
  \catcode47=12 % /
  \catcode58=12 % :
  \catcode64=11 % @
  \catcode91=12 % [
  \catcode93=12 % ]
  \catcode123=1 % {
  \catcode125=2 % }
  \expandafter\ifx\csname ProvidesPackage\endcsname\relax
    \def\x#1#2#3[#4]{\endgroup
      \immediate\write-1{Package: #3 #4}%
      \xdef#1{#4}%
    }%
  \else
    \def\x#1#2[#3]{\endgroup
      #2[{#3}]%
      \ifx#1\@undefined
        \xdef#1{#3}%
      \fi
      \ifx#1\relax
        \xdef#1{#3}%
      \fi
    }%
  \fi
\expandafter\x\csname ver@bigintcalc.sty\endcsname
\ProvidesPackage{bigintcalc}%
  [2016/05/16 v1.4 Expandable calculations on big integers (HO)]%
%    \end{macrocode}
%
% \subsection{Catcodes}
%
%    \begin{macrocode}
\begingroup\catcode61\catcode48\catcode32=10\relax%
  \catcode13=5 % ^^M
  \endlinechar=13 %
  \catcode123=1 % {
  \catcode125=2 % }
  \catcode64=11 % @
  \def\x{\endgroup
    \expandafter\edef\csname BIC@AtEnd\endcsname{%
      \endlinechar=\the\endlinechar\relax
      \catcode13=\the\catcode13\relax
      \catcode32=\the\catcode32\relax
      \catcode35=\the\catcode35\relax
      \catcode61=\the\catcode61\relax
      \catcode64=\the\catcode64\relax
      \catcode123=\the\catcode123\relax
      \catcode125=\the\catcode125\relax
    }%
  }%
\x\catcode61\catcode48\catcode32=10\relax%
\catcode13=5 % ^^M
\endlinechar=13 %
\catcode35=6 % #
\catcode64=11 % @
\catcode123=1 % {
\catcode125=2 % }
\def\TMP@EnsureCode#1#2{%
  \edef\BIC@AtEnd{%
    \BIC@AtEnd
    \catcode#1=\the\catcode#1\relax
  }%
  \catcode#1=#2\relax
}
\TMP@EnsureCode{33}{12}% !
\TMP@EnsureCode{36}{14}% $ (comment!)
\TMP@EnsureCode{38}{14}% & (comment!)
\TMP@EnsureCode{40}{12}% (
\TMP@EnsureCode{41}{12}% )
\TMP@EnsureCode{42}{12}% *
\TMP@EnsureCode{43}{12}% +
\TMP@EnsureCode{45}{12}% -
\TMP@EnsureCode{46}{12}% .
\TMP@EnsureCode{47}{12}% /
\TMP@EnsureCode{58}{11}% : (letter!)
\TMP@EnsureCode{60}{12}% <
\TMP@EnsureCode{62}{12}% >
\TMP@EnsureCode{63}{14}% ? (comment!)
\TMP@EnsureCode{91}{12}% [
\TMP@EnsureCode{93}{12}% ]
\edef\BIC@AtEnd{\BIC@AtEnd\noexpand\endinput}
\begingroup\expandafter\expandafter\expandafter\endgroup
\expandafter\ifx\csname BIC@TestMode\endcsname\relax
\else
  \catcode63=9 % ? (ignore)
\fi
? \let\BIC@@TestMode\BIC@TestMode
%    \end{macrocode}
%
% \subsection{\eTeX\ detection}
%
%    \begin{macrocode}
\begingroup\expandafter\expandafter\expandafter\endgroup
\expandafter\ifx\csname numexpr\endcsname\relax
  \catcode36=9 % $ (ignore)
\else
  \catcode38=9 % & (ignore)
\fi
%    \end{macrocode}
%
% \subsection{Help macros}
%
%    \begin{macro}{\BIC@Fi}
%    \begin{macrocode}
\let\BIC@Fi\fi
%    \end{macrocode}
%    \end{macro}
%    \begin{macro}{\BIC@AfterFi}
%    \begin{macrocode}
\def\BIC@AfterFi#1#2\BIC@Fi{\fi#1}%
%    \end{macrocode}
%    \end{macro}
%    \begin{macro}{\BIC@AfterFiFi}
%    \begin{macrocode}
\def\BIC@AfterFiFi#1#2\BIC@Fi{\fi\fi#1}%
%    \end{macrocode}
%    \end{macro}
%    \begin{macro}{\BIC@AfterFiFiFi}
%    \begin{macrocode}
\def\BIC@AfterFiFiFi#1#2\BIC@Fi{\fi\fi\fi#1}%
%    \end{macrocode}
%    \end{macro}
%
%    \begin{macro}{\BIC@Space}
%    \begin{macrocode}
\begingroup
  \def\x#1{\endgroup
    \let\BIC@Space= #1%
  }%
\x{ }
%    \end{macrocode}
%    \end{macro}
%
% \subsection{Expand number}
%
%    \begin{macrocode}
\begingroup\expandafter\expandafter\expandafter\endgroup
\expandafter\ifx\csname RequirePackage\endcsname\relax
  \def\TMP@RequirePackage#1[#2]{%
    \begingroup\expandafter\expandafter\expandafter\endgroup
    \expandafter\ifx\csname ver@#1.sty\endcsname\relax
      \input #1.sty\relax
    \fi
  }%
  \TMP@RequirePackage{pdftexcmds}[2007/11/11]%
\else
  \RequirePackage{pdftexcmds}[2007/11/11]%
\fi
%    \end{macrocode}
%
%    \begin{macrocode}
\begingroup\expandafter\expandafter\expandafter\endgroup
\expandafter\ifx\csname pdf@escapehex\endcsname\relax
%    \end{macrocode}
%
%    \begin{macro}{\BIC@Expand}
%    \begin{macrocode}
  \def\BIC@Expand#1{%
    \romannumeral0%
    \BIC@@Expand#1!\@nil{}%
  }%
%    \end{macrocode}
%    \end{macro}
%    \begin{macro}{\BIC@@Expand}
%    \begin{macrocode}
  \def\BIC@@Expand#1#2\@nil#3{%
    \expandafter\ifcat\noexpand#1\relax
      \expandafter\@firstoftwo
    \else
      \expandafter\@secondoftwo
    \fi
    {%
      \expandafter\BIC@@Expand#1#2\@nil{#3}%
    }{%
      \ifx#1!%
        \expandafter\@firstoftwo
      \else
        \expandafter\@secondoftwo
      \fi
      { #3}{%
        \BIC@@Expand#2\@nil{#3#1}%
      }%
    }%
  }%
%    \end{macrocode}
%    \end{macro}
%    \begin{macro}{\@firstoftwo}
%    \begin{macrocode}
  \expandafter\ifx\csname @firstoftwo\endcsname\relax
    \long\def\@firstoftwo#1#2{#1}%
  \fi
%    \end{macrocode}
%    \end{macro}
%    \begin{macro}{\@secondoftwo}
%    \begin{macrocode}
  \expandafter\ifx\csname @secondoftwo\endcsname\relax
    \long\def\@secondoftwo#1#2{#2}%
  \fi
%    \end{macrocode}
%    \end{macro}
%
%    \begin{macrocode}
\else
%    \end{macrocode}
%    \begin{macro}{\BIC@Expand}
%    \begin{macrocode}
  \def\BIC@Expand#1{%
    \romannumeral0\expandafter\expandafter\expandafter\BIC@Space
    \pdf@unescapehex{%
      \expandafter\expandafter\expandafter
      \BIC@StripHexSpace\pdf@escapehex{#1}20\@nil
    }%
  }%
%    \end{macrocode}
%    \end{macro}
%    \begin{macro}{\BIC@StripHexSpace}
%    \begin{macrocode}
  \def\BIC@StripHexSpace#120#2\@nil{%
    #1%
    \ifx\\#2\\%
    \else
      \BIC@AfterFi{%
        \BIC@StripHexSpace#2\@nil
      }%
    \BIC@Fi
  }%
%    \end{macrocode}
%    \end{macro}
%    \begin{macrocode}
\fi
%    \end{macrocode}
%
% \subsection{Normalize expanded number}
%
%    \begin{macro}{\BIC@Normalize}
%    |#1|: result sign\\
%    |#2|: first token of number
%    \begin{macrocode}
\def\BIC@Normalize#1#2{%
  \ifx#2-%
    \ifx\\#1\\%
      \BIC@AfterFiFi{%
        \BIC@Normalize-%
      }%
    \else
      \BIC@AfterFiFi{%
        \BIC@Normalize{}%
      }%
    \fi
  \else
    \ifx#2+%
      \BIC@AfterFiFi{%
        \BIC@Normalize{#1}%
      }%
    \else
      \ifx#20%
        \BIC@AfterFiFiFi{%
          \BIC@NormalizeZero{#1}%
        }%
      \else
        \BIC@AfterFiFiFi{%
          \BIC@NormalizeDigits#1#2%
        }%
      \fi
    \fi
  \BIC@Fi
}
%    \end{macrocode}
%    \end{macro}
%    \begin{macro}{\BIC@NormalizeZero}
%    \begin{macrocode}
\def\BIC@NormalizeZero#1#2{%
  \ifx#2!%
    \BIC@AfterFi{ 0}%
  \else
    \ifx#20%
      \BIC@AfterFiFi{%
        \BIC@NormalizeZero{#1}%
      }%
    \else
      \BIC@AfterFiFi{%
        \BIC@NormalizeDigits#1#2%
      }%
    \fi
  \BIC@Fi
}
%    \end{macrocode}
%    \end{macro}
%    \begin{macro}{\BIC@NormalizeDigits}
%    \begin{macrocode}
\def\BIC@NormalizeDigits#1!{ #1}
%    \end{macrocode}
%    \end{macro}
%
% \subsection{\op{Num}}
%
%    \begin{macro}{\bigintcalcNum}
%    \begin{macrocode}
\def\bigintcalcNum#1{%
  \romannumeral0%
  \expandafter\expandafter\expandafter\BIC@Normalize
  \expandafter\expandafter\expandafter{%
  \expandafter\expandafter\expandafter}%
  \BIC@Expand{#1}!%
}
%    \end{macrocode}
%    \end{macro}
%
% \subsection{\op{Inv}, \op{Abs}, \op{Sgn}}
%
%
%    \begin{macro}{\bigintcalcInv}
%    \begin{macrocode}
\def\bigintcalcInv#1{%
  \romannumeral0\expandafter\expandafter\expandafter\BIC@Space
  \bigintcalcNum{-#1}%
}
%    \end{macrocode}
%    \end{macro}
%
%    \begin{macro}{\bigintcalcAbs}
%    \begin{macrocode}
\def\bigintcalcAbs#1{%
  \romannumeral0%
  \expandafter\expandafter\expandafter\BIC@Abs
  \bigintcalcNum{#1}%
}
%    \end{macrocode}
%    \end{macro}
%    \begin{macro}{\BIC@Abs}
%    \begin{macrocode}
\def\BIC@Abs#1{%
  \ifx#1-%
    \expandafter\BIC@Space
  \else
    \expandafter\BIC@Space
    \expandafter#1%
  \fi
}
%    \end{macrocode}
%    \end{macro}
%
%    \begin{macro}{\bigintcalcSgn}
%    \begin{macrocode}
\def\bigintcalcSgn#1{%
  \number
  \expandafter\expandafter\expandafter\BIC@Sgn
  \bigintcalcNum{#1}! %
}
%    \end{macrocode}
%    \end{macro}
%    \begin{macro}{\BIC@Sgn}
%    \begin{macrocode}
\def\BIC@Sgn#1#2!{%
  \ifx#1-%
    -1%
  \else
    \ifx#10%
      0%
    \else
      1%
    \fi
  \fi
}
%    \end{macrocode}
%    \end{macro}
%
% \subsection{\op{Cmp}, \op{Min}, \op{Max}}
%
%    \begin{macro}{\bigintcalcCmp}
%    \begin{macrocode}
\def\bigintcalcCmp#1#2{%
  \number
  \expandafter\expandafter\expandafter\BIC@Cmp
  \bigintcalcNum{#2}!{#1}%
}
%    \end{macrocode}
%    \end{macro}
%    \begin{macro}{\BIC@Cmp}
%    \begin{macrocode}
\def\BIC@Cmp#1!#2{%
  \expandafter\expandafter\expandafter\BIC@@Cmp
  \bigintcalcNum{#2}!#1!%
}
%    \end{macrocode}
%    \end{macro}
%    \begin{macro}{\BIC@@Cmp}
%    \begin{macrocode}
\def\BIC@@Cmp#1#2!#3#4!{%
  \ifx#1-%
    \ifx#3-%
      \BIC@AfterFiFi{%
        \BIC@@Cmp#4!#2!%
      }%
    \else
      \BIC@AfterFiFi{%
        -1 %
      }%
    \fi
  \else
    \ifx#3-%
      \BIC@AfterFiFi{%
        1 %
      }%
    \else
      \BIC@AfterFiFi{%
        \BIC@CmpLength#1#2!#3#4!#1#2!#3#4!%
      }%
    \fi
  \BIC@Fi
}
%    \end{macrocode}
%    \end{macro}
%    \begin{macro}{\BIC@PosCmp}
%    \begin{macrocode}
\def\BIC@PosCmp#1!#2!{%
  \BIC@CmpLength#1!#2!#1!#2!%
}
%    \end{macrocode}
%    \end{macro}
%    \begin{macro}{\BIC@CmpLength}
%    \begin{macrocode}
\def\BIC@CmpLength#1#2!#3#4!{%
  \ifx\\#2\\%
    \ifx\\#4\\%
      \BIC@AfterFiFi\BIC@CmpDiff
    \else
      \BIC@AfterFiFi{%
        \BIC@CmpResult{-1}%
      }%
    \fi
  \else
    \ifx\\#4\\%
      \BIC@AfterFiFi{%
        \BIC@CmpResult1%
      }%
    \else
      \BIC@AfterFiFi{%
        \BIC@CmpLength#2!#4!%
      }%
    \fi
  \BIC@Fi
}
%    \end{macrocode}
%    \end{macro}
%    \begin{macro}{\BIC@CmpResult}
%    \begin{macrocode}
\def\BIC@CmpResult#1#2!#3!{#1 }
%    \end{macrocode}
%    \end{macro}
%    \begin{macro}{\BIC@CmpDiff}
%    \begin{macrocode}
\def\BIC@CmpDiff#1#2!#3#4!{%
  \ifnum#1<#3 %
    \BIC@AfterFi{%
      -1 %
    }%
  \else
    \ifnum#1>#3 %
      \BIC@AfterFiFi{%
        1 %
      }%
    \else
      \ifx\\#2\\%
        \BIC@AfterFiFiFi{%
          0 %
        }%
      \else
        \BIC@AfterFiFiFi{%
          \BIC@CmpDiff#2!#4!%
        }%
      \fi
    \fi
  \BIC@Fi
}
%    \end{macrocode}
%    \end{macro}
%
%    \begin{macro}{\bigintcalcMin}
%    \begin{macrocode}
\def\bigintcalcMin#1{%
  \romannumeral0%
  \expandafter\expandafter\expandafter\BIC@MinMax
  \bigintcalcNum{#1}!-!%
}
%    \end{macrocode}
%    \end{macro}
%    \begin{macro}{\bigintcalcMax}
%    \begin{macrocode}
\def\bigintcalcMax#1{%
  \romannumeral0%
  \expandafter\expandafter\expandafter\BIC@MinMax
  \bigintcalcNum{#1}!!%
}
%    \end{macrocode}
%    \end{macro}
%    \begin{macro}{\BIC@MinMax}
%    |#1|: $x$\\
%    |#2|: sign for comparison\\
%    |#3|: $y$
%    \begin{macrocode}
\def\BIC@MinMax#1!#2!#3{%
  \expandafter\expandafter\expandafter\BIC@@MinMax
  \bigintcalcNum{#3}!#1!#2!%
}
%    \end{macrocode}
%    \end{macro}
%    \begin{macro}{\BIC@@MinMax}
%    |#1|: $y$\\
%    |#2|: $x$\\
%    |#3|: sign for comparison
%    \begin{macrocode}
\def\BIC@@MinMax#1!#2!#3!{%
  \ifnum\BIC@@Cmp#1!#2!=#31 %
    \BIC@AfterFi{ #1}%
  \else
    \BIC@AfterFi{ #2}%
  \BIC@Fi
}
%    \end{macrocode}
%    \end{macro}
%
% \subsection{\op{Odd}}
%
%    \begin{macro}{\bigintcalcOdd}
%    \begin{macrocode}
\def\bigintcalcOdd#1{%
  \romannumeral0%
  \expandafter\expandafter\expandafter\BIC@Odd
  \bigintcalcAbs{#1}!%
}
%    \end{macrocode}
%    \end{macro}
%    \begin{macro}{\BigIntCalcOdd}
%    \begin{macrocode}
\def\BigIntCalcOdd#1!{%
  \romannumeral0%
  \BIC@Odd#1!%
}
%    \end{macrocode}
%    \end{macro}
%    \begin{macro}{\BIC@Odd}
%    |#1|: $x$
%    \begin{macrocode}
\def\BIC@Odd#1#2{%
  \ifx#2!%
    \ifodd#1 %
      \BIC@AfterFiFi{ 1}%
    \else
      \BIC@AfterFiFi{ 0}%
    \fi
  \else
    \expandafter\BIC@Odd\expandafter#2%
  \BIC@Fi
}
%    \end{macrocode}
%    \end{macro}
%
% \subsection{\op{Inc}, \op{Dec}}
%
%    \begin{macro}{\bigintcalcInc}
%    \begin{macrocode}
\def\bigintcalcInc#1{%
  \romannumeral0%
  \expandafter\expandafter\expandafter\BIC@IncSwitch
  \bigintcalcNum{#1}!%
}
%    \end{macrocode}
%    \end{macro}
%    \begin{macro}{\BIC@IncSwitch}
%    \begin{macrocode}
\def\BIC@IncSwitch#1#2!{%
  \ifcase\BIC@@Cmp#1#2!-1!%
    \BIC@AfterFi{ 0}%
  \or
    \BIC@AfterFi{%
      \BIC@Inc#1#2!{}%
    }%
  \else
    \BIC@AfterFi{%
      \expandafter-\romannumeral0%
      \BIC@Dec#2!{}%
    }%
  \BIC@Fi
}
%    \end{macrocode}
%    \end{macro}
%
%    \begin{macro}{\bigintcalcDec}
%    \begin{macrocode}
\def\bigintcalcDec#1{%
  \romannumeral0%
  \expandafter\expandafter\expandafter\BIC@DecSwitch
  \bigintcalcNum{#1}!%
}
%    \end{macrocode}
%    \end{macro}
%    \begin{macro}{\BIC@DecSwitch}
%    \begin{macrocode}
\def\BIC@DecSwitch#1#2!{%
  \ifcase\BIC@Sgn#1#2! %
    \BIC@AfterFi{ -1}%
  \or
    \BIC@AfterFi{%
      \BIC@Dec#1#2!{}%
    }%
  \else
    \BIC@AfterFi{%
      \expandafter-\romannumeral0%
      \BIC@Inc#2!{}%
    }%
  \BIC@Fi
}
%    \end{macrocode}
%    \end{macro}
%
%    \begin{macro}{\BigIntCalcInc}
%    \begin{macrocode}
\def\BigIntCalcInc#1!{%
  \romannumeral0\BIC@Inc#1!{}%
}
%    \end{macrocode}
%    \end{macro}
%    \begin{macro}{\BigIntCalcDec}
%    \begin{macrocode}
\def\BigIntCalcDec#1!{%
  \romannumeral0\BIC@Dec#1!{}%
}
%    \end{macrocode}
%    \end{macro}
%
%    \begin{macro}{\BIC@Inc}
%    \begin{macrocode}
\def\BIC@Inc#1#2!#3{%
  \ifx\\#2\\%
    \BIC@AfterFi{%
      \BIC@@Inc1#1#3!{}%
    }%
  \else
    \BIC@AfterFi{%
      \BIC@Inc#2!{#1#3}%
    }%
  \BIC@Fi
}
%    \end{macrocode}
%    \end{macro}
%    \begin{macro}{\BIC@@Inc}
%    \begin{macrocode}
\def\BIC@@Inc#1#2#3!#4{%
  \ifcase#1 %
    \ifx\\#3\\%
      \BIC@AfterFiFi{ #2#4}%
    \else
      \BIC@AfterFiFi{%
        \BIC@@Inc0#3!{#2#4}%
      }%
    \fi
  \else
    \ifnum#2<9 %
      \BIC@AfterFiFi{%
&       \expandafter\BIC@@@Inc\the\numexpr#2+1\relax
$       \expandafter\expandafter\expandafter\BIC@@@Inc
$       \ifcase#2 \expandafter1%
$       \or\expandafter2%
$       \or\expandafter3%
$       \or\expandafter4%
$       \or\expandafter5%
$       \or\expandafter6%
$       \or\expandafter7%
$       \or\expandafter8%
$       \or\expandafter9%
$?      \else\BigIntCalcError:ThisCannotHappen%
$       \fi
        0#3!{#4}%
      }%
    \else
      \BIC@AfterFiFi{%
        \BIC@@@Inc01#3!{#4}%
      }%
    \fi
  \BIC@Fi
}
%    \end{macrocode}
%    \end{macro}
%    \begin{macro}{\BIC@@@Inc}
%    \begin{macrocode}
\def\BIC@@@Inc#1#2#3!#4{%
  \ifx\\#3\\%
    \ifnum#2=1 %
      \BIC@AfterFiFi{ 1#1#4}%
    \else
      \BIC@AfterFiFi{ #1#4}%
    \fi
  \else
    \BIC@AfterFi{%
      \BIC@@Inc#2#3!{#1#4}%
    }%
  \BIC@Fi
}
%    \end{macrocode}
%    \end{macro}
%
%    \begin{macro}{\BIC@Dec}
%    \begin{macrocode}
\def\BIC@Dec#1#2!#3{%
  \ifx\\#2\\%
    \BIC@AfterFi{%
      \BIC@@Dec1#1#3!{}%
    }%
  \else
    \BIC@AfterFi{%
      \BIC@Dec#2!{#1#3}%
    }%
  \BIC@Fi
}
%    \end{macrocode}
%    \end{macro}
%    \begin{macro}{\BIC@@Dec}
%    \begin{macrocode}
\def\BIC@@Dec#1#2#3!#4{%
  \ifcase#1 %
    \ifx\\#3\\%
      \BIC@AfterFiFi{ #2#4}%
    \else
      \BIC@AfterFiFi{%
        \BIC@@Dec0#3!{#2#4}%
      }%
    \fi
  \else
    \ifnum#2>0 %
      \BIC@AfterFiFi{%
&       \expandafter\BIC@@@Dec\the\numexpr#2-1\relax
$       \expandafter\expandafter\expandafter\BIC@@@Dec
$       \ifcase#2
$?        \BigIntCalcError:ThisCannotHappen%
$       \or\expandafter0%
$       \or\expandafter1%
$       \or\expandafter2%
$       \or\expandafter3%
$       \or\expandafter4%
$       \or\expandafter5%
$       \or\expandafter6%
$       \or\expandafter7%
$       \or\expandafter8%
$?      \else\BigIntCalcError:ThisCannotHappen%
$       \fi
        0#3!{#4}%
      }%
    \else
      \BIC@AfterFiFi{%
        \BIC@@@Dec91#3!{#4}%
      }%
    \fi
  \BIC@Fi
}
%    \end{macrocode}
%    \end{macro}
%    \begin{macro}{\BIC@@@Dec}
%    \begin{macrocode}
\def\BIC@@@Dec#1#2#3!#4{%
  \ifx\\#3\\%
    \ifcase#1 %
      \ifx\\#4\\%
        \BIC@AfterFiFiFi{ 0}%
      \else
        \BIC@AfterFiFiFi{ #4}%
      \fi
    \else
      \BIC@AfterFiFi{ #1#4}%
    \fi
  \else
    \BIC@AfterFi{%
      \BIC@@Dec#2#3!{#1#4}%
    }%
  \BIC@Fi
}
%    \end{macrocode}
%    \end{macro}
%
% \subsection{\op{Add}, \op{Sub}}
%
%    \begin{macro}{\bigintcalcAdd}
%    \begin{macrocode}
\def\bigintcalcAdd#1{%
  \romannumeral0%
  \expandafter\expandafter\expandafter\BIC@Add
  \bigintcalcNum{#1}!%
}
%    \end{macrocode}
%    \end{macro}
%    \begin{macro}{\BIC@Add}
%    \begin{macrocode}
\def\BIC@Add#1!#2{%
  \expandafter\expandafter\expandafter
  \BIC@AddSwitch\bigintcalcNum{#2}!#1!%
}
%    \end{macrocode}
%    \end{macro}
%    \begin{macro}{\bigintcalcSub}
%    \begin{macrocode}
\def\bigintcalcSub#1#2{%
  \romannumeral0%
  \expandafter\expandafter\expandafter\BIC@Add
  \bigintcalcNum{-#2}!{#1}%
}
%    \end{macrocode}
%    \end{macro}
%
%    \begin{macro}{\BIC@AddSwitch}
%    Decision table for \cs{BIC@AddSwitch}.
%    \begin{quote}
%    \begin{tabular}[t]{@{}|l|l|l|l|l|@{}}
%      \hline
%      $x<0$ & $y<0$ & $-x>-y$ & $-$ & $\opAdd(-x,-y)$\\
%      \cline{3-3}\cline{5-5}
%            &       & else    &     & $\opAdd(-y,-x)$\\
%      \cline{2-5}
%            & else  & $-x> y$ & $-$ & $\opSub(-x, y)$\\
%      \cline{3-5}
%            &       & $-x= y$ &     & $0$\\
%      \cline{3-5}
%            &       & else    & $+$ & $\opSub( y,-x)$\\
%      \hline
%      else  & $y<0$ & $ x>-y$ & $+$ & $\opSub( x,-y)$\\
%      \cline{3-5}
%            &       & $ x=-y$ &     & $0$\\
%      \cline{3-5}
%            &       & else    & $-$ & $\opSub(-y, x)$\\
%      \cline{2-5}
%            & else  & $ x> y$ & $+$ & $\opAdd( x, y)$\\
%      \cline{3-3}\cline{5-5}
%            &       & else    &     & $\opAdd( y, x)$\\
%      \hline
%    \end{tabular}
%    \end{quote}
%    \begin{macrocode}
\def\BIC@AddSwitch#1#2!#3#4!{%
  \ifx#1-% x < 0
    \ifx#3-% y < 0
      \expandafter-\romannumeral0%
      \ifnum\BIC@PosCmp#2!#4!=1 % -x > -y
        \BIC@AfterFiFiFi{%
          \BIC@AddXY#2!#4!!!%
        }%
      \else % -x <= -y
        \BIC@AfterFiFiFi{%
          \BIC@AddXY#4!#2!!!%
        }%
      \fi
    \else % y >= 0
      \ifcase\BIC@PosCmp#2!#3#4!% -x = y
        \BIC@AfterFiFiFi{ 0}%
      \or % -x > y
        \expandafter-\romannumeral0%
        \BIC@AfterFiFiFi{%
          \BIC@SubXY#2!#3#4!!!%
        }%
      \else % -x <= y
        \BIC@AfterFiFiFi{%
          \BIC@SubXY#3#4!#2!!!%
        }%
      \fi
    \fi
  \else % x >= 0
    \ifx#3-% y < 0
      \ifcase\BIC@PosCmp#1#2!#4!% x = -y
        \BIC@AfterFiFiFi{ 0}%
      \or % x > -y
        \BIC@AfterFiFiFi{%
          \BIC@SubXY#1#2!#4!!!%
        }%
      \else % x <= -y
        \expandafter-\romannumeral0%
        \BIC@AfterFiFiFi{%
          \BIC@SubXY#4!#1#2!!!%
        }%
      \fi
    \else % y >= 0
      \ifnum\BIC@PosCmp#1#2!#3#4!=1 % x > y
        \BIC@AfterFiFiFi{%
          \BIC@AddXY#1#2!#3#4!!!%
        }%
      \else % x <= y
        \BIC@AfterFiFiFi{%
          \BIC@AddXY#3#4!#1#2!!!%
        }%
      \fi
    \fi
  \BIC@Fi
}
%    \end{macrocode}
%    \end{macro}
%
%    \begin{macro}{\BigIntCalcAdd}
%    \begin{macrocode}
\def\BigIntCalcAdd#1!#2!{%
  \romannumeral0\BIC@AddXY#1!#2!!!%
}
%    \end{macrocode}
%    \end{macro}
%    \begin{macro}{\BigIntCalcSub}
%    \begin{macrocode}
\def\BigIntCalcSub#1!#2!{%
  \romannumeral0\BIC@SubXY#1!#2!!!%
}
%    \end{macrocode}
%    \end{macro}
%
%    \begin{macro}{\BIC@AddXY}
%    \begin{macrocode}
\def\BIC@AddXY#1#2!#3#4!#5!#6!{%
  \ifx\\#2\\%
    \ifx\\#3\\%
      \BIC@AfterFiFi{%
        \BIC@DoAdd0!#1#5!#60!%
      }%
    \else
      \BIC@AfterFiFi{%
        \BIC@DoAdd0!#1#5!#3#6!%
      }%
    \fi
  \else
    \ifx\\#4\\%
      \ifx\\#3\\%
        \BIC@AfterFiFiFi{%
          \BIC@AddXY#2!{}!#1#5!#60!%
        }%
      \else
        \BIC@AfterFiFiFi{%
          \BIC@AddXY#2!{}!#1#5!#3#6!%
        }%
      \fi
    \else
      \BIC@AfterFiFi{%
        \BIC@AddXY#2!#4!#1#5!#3#6!%
      }%
    \fi
  \BIC@Fi
}
%    \end{macrocode}
%    \end{macro}
%    \begin{macro}{\BIC@DoAdd}
%    |#1|: carry\\
%    |#2|: reverted result\\
%    |#3#4|: reverted $x$\\
%    |#5#6|: reverted $y$
%    \begin{macrocode}
\def\BIC@DoAdd#1#2!#3#4!#5#6!{%
  \ifx\\#4\\%
    \BIC@AfterFi{%
&     \expandafter\BIC@Space
&     \the\numexpr#1+#3+#5\relax#2%
$     \expandafter\expandafter\expandafter\BIC@AddResult
$     \BIC@AddDigit#1#3#5#2%
    }%
  \else
    \BIC@AfterFi{%
      \expandafter\expandafter\expandafter\BIC@DoAdd
      \BIC@AddDigit#1#3#5#2!#4!#6!%
    }%
  \BIC@Fi
}
%    \end{macrocode}
%    \end{macro}
%    \begin{macro}{\BIC@AddResult}
%    \begin{macrocode}
$ \def\BIC@AddResult#1{%
$   \ifx#10%
$     \expandafter\BIC@Space
$   \else
$     \expandafter\BIC@Space\expandafter#1%
$   \fi
$ }%
%    \end{macrocode}
%    \end{macro}
%    \begin{macro}{\BIC@AddDigit}
%    |#1|: carry\\
%    |#2|: digit of $x$\\
%    |#3|: digit of $y$
%    \begin{macrocode}
\def\BIC@AddDigit#1#2#3{%
  \romannumeral0%
& \expandafter\BIC@@AddDigit\the\numexpr#1+#2+#3!%
$ \expandafter\BIC@@AddDigit\number%
$ \csname
$   BIC@AddCarry%
$   \ifcase#1 %
$     #2%
$   \else
$     \ifcase#2 1\or2\or3\or4\or5\or6\or7\or8\or9\or10\fi
$   \fi
$ \endcsname#3!%
}
%    \end{macrocode}
%    \end{macro}
%    \begin{macro}{\BIC@@AddDigit}
%    \begin{macrocode}
\def\BIC@@AddDigit#1!{%
  \ifnum#1<10 %
    \BIC@AfterFi{ 0#1}%
  \else
    \BIC@AfterFi{ #1}%
  \BIC@Fi
}
%    \end{macrocode}
%    \end{macro}
%    \begin{macro}{\BIC@AddCarry0}
%    \begin{macrocode}
$ \expandafter\def\csname BIC@AddCarry0\endcsname#1{#1}%
%    \end{macrocode}
%    \end{macro}
%    \begin{macro}{\BIC@AddCarry10}
%    \begin{macrocode}
$ \expandafter\def\csname BIC@AddCarry10\endcsname#1{1#1}%
%    \end{macrocode}
%    \end{macro}
%    \begin{macro}{\BIC@AddCarry[1-9]}
%    \begin{macrocode}
$ \def\BIC@Temp#1#2{%
$   \expandafter\def\csname BIC@AddCarry#1\endcsname##1{%
$     \ifcase##1 #1\or
$     #2%
$?    \else\BigIntCalcError:ThisCannotHappen%
$     \fi
$   }%
$ }%
$ \BIC@Temp 0{1\or2\or3\or4\or5\or6\or7\or8\or9}%
$ \BIC@Temp 1{2\or3\or4\or5\or6\or7\or8\or9\or10}%
$ \BIC@Temp 2{3\or4\or5\or6\or7\or8\or9\or10\or11}%
$ \BIC@Temp 3{4\or5\or6\or7\or8\or9\or10\or11\or12}%
$ \BIC@Temp 4{5\or6\or7\or8\or9\or10\or11\or12\or13}%
$ \BIC@Temp 5{6\or7\or8\or9\or10\or11\or12\or13\or14}%
$ \BIC@Temp 6{7\or8\or9\or10\or11\or12\or13\or14\or15}%
$ \BIC@Temp 7{8\or9\or10\or11\or12\or13\or14\or15\or16}%
$ \BIC@Temp 8{9\or10\or11\or12\or13\or14\or15\or16\or17}%
$ \BIC@Temp 9{10\or11\or12\or13\or14\or15\or16\or17\or18}%
%    \end{macrocode}
%    \end{macro}
%
%    \begin{macro}{\BIC@SubXY}
%    Preconditions:
%    \begin{itemize}
%    \item $x > y$, $x \geq 0$, and $y >= 0$
%    \item digits($x$) = digits($y$)
%    \end{itemize}
%    \begin{macrocode}
\def\BIC@SubXY#1#2!#3#4!#5!#6!{%
  \ifx\\#2\\%
    \ifx\\#3\\%
      \BIC@AfterFiFi{%
        \BIC@DoSub0!#1#5!#60!%
      }%
    \else
      \BIC@AfterFiFi{%
        \BIC@DoSub0!#1#5!#3#6!%
      }%
    \fi
  \else
    \ifx\\#4\\%
      \ifx\\#3\\%
        \BIC@AfterFiFiFi{%
          \BIC@SubXY#2!{}!#1#5!#60!%
        }%
      \else
        \BIC@AfterFiFiFi{%
          \BIC@SubXY#2!{}!#1#5!#3#6!%
        }%
      \fi
    \else
      \BIC@AfterFiFi{%
        \BIC@SubXY#2!#4!#1#5!#3#6!%
      }%
    \fi
  \BIC@Fi
}
%    \end{macrocode}
%    \end{macro}
%    \begin{macro}{\BIC@DoSub}
%    |#1|: carry\\
%    |#2|: reverted result\\
%    |#3#4|: reverted x\\
%    |#5#6|: reverted y
%    \begin{macrocode}
\def\BIC@DoSub#1#2!#3#4!#5#6!{%
  \ifx\\#4\\%
    \BIC@AfterFi{%
      \expandafter\expandafter\expandafter\BIC@SubResult
      \BIC@SubDigit#1#3#5#2%
    }%
  \else
    \BIC@AfterFi{%
      \expandafter\expandafter\expandafter\BIC@DoSub
      \BIC@SubDigit#1#3#5#2!#4!#6!%
    }%
  \BIC@Fi
}
%    \end{macrocode}
%    \end{macro}
%    \begin{macro}{\BIC@SubResult}
%    \begin{macrocode}
\def\BIC@SubResult#1{%
  \ifx#10%
    \expandafter\BIC@SubResult
  \else
    \expandafter\BIC@Space\expandafter#1%
  \fi
}
%    \end{macrocode}
%    \end{macro}
%    \begin{macro}{\BIC@SubDigit}
%    |#1|: carry\\
%    |#2|: digit of $x$\\
%    |#3|: digit of $y$
%    \begin{macrocode}
\def\BIC@SubDigit#1#2#3{%
  \romannumeral0%
& \expandafter\BIC@@SubDigit\the\numexpr#2-#3-#1!%
$ \expandafter\BIC@@AddDigit\number
$   \csname
$     BIC@SubCarry%
$     \ifcase#1 %
$       #3%
$     \else
$       \ifcase#3 1\or2\or3\or4\or5\or6\or7\or8\or9\or10\fi
$     \fi
$   \endcsname#2!%
}
%    \end{macrocode}
%    \end{macro}
%    \begin{macro}{\BIC@@SubDigit}
%    \begin{macrocode}
& \def\BIC@@SubDigit#1!{%
&   \ifnum#1<0 %
&     \BIC@AfterFi{%
&       \expandafter\BIC@Space
&       \expandafter1\the\numexpr#1+10\relax
&     }%
&   \else
&     \BIC@AfterFi{ 0#1}%
&   \BIC@Fi
& }%
%    \end{macrocode}
%    \end{macro}
%    \begin{macro}{\BIC@SubCarry0}
%    \begin{macrocode}
$ \expandafter\def\csname BIC@SubCarry0\endcsname#1{#1}%
%    \end{macrocode}
%    \end{macro}
%    \begin{macro}{\BIC@SubCarry10}%
%    \begin{macrocode}
$ \expandafter\def\csname BIC@SubCarry10\endcsname#1{1#1}%
%    \end{macrocode}
%    \end{macro}
%    \begin{macro}{\BIC@SubCarry[1-9]}
%    \begin{macrocode}
$ \def\BIC@Temp#1#2{%
$   \expandafter\def\csname BIC@SubCarry#1\endcsname##1{%
$     \ifcase##1 #2%
$?    \else\BigIntCalcError:ThisCannotHappen%
$     \fi
$   }%
$ }%
$ \BIC@Temp 1{19\or0\or1\or2\or3\or4\or5\or6\or7\or8}%
$ \BIC@Temp 2{18\or19\or0\or1\or2\or3\or4\or5\or6\or7}%
$ \BIC@Temp 3{17\or18\or19\or0\or1\or2\or3\or4\or5\or6}%
$ \BIC@Temp 4{16\or17\or18\or19\or0\or1\or2\or3\or4\or5}%
$ \BIC@Temp 5{15\or16\or17\or18\or19\or0\or1\or2\or3\or4}%
$ \BIC@Temp 6{14\or15\or16\or17\or18\or19\or0\or1\or2\or3}%
$ \BIC@Temp 7{13\or14\or15\or16\or17\or18\or19\or0\or1\or2}%
$ \BIC@Temp 8{12\or13\or14\or15\or16\or17\or18\or19\or0\or1}%
$ \BIC@Temp 9{11\or12\or13\or14\or15\or16\or17\or18\or19\or0}%
%    \end{macrocode}
%    \end{macro}
%
% \subsection{\op{Shl}, \op{Shr}}
%
%    \begin{macro}{\bigintcalcShl}
%    \begin{macrocode}
\def\bigintcalcShl#1{%
  \romannumeral0%
  \expandafter\expandafter\expandafter\BIC@Shl
  \bigintcalcNum{#1}!%
}
%    \end{macrocode}
%    \end{macro}
%    \begin{macro}{\BIC@Shl}
%    \begin{macrocode}
\def\BIC@Shl#1#2!{%
  \ifx#1-%
    \BIC@AfterFi{%
      \expandafter-\romannumeral0%
&     \BIC@@Shl#2!!%
$     \BIC@AddXY#2!#2!!!%
    }%
  \else
    \BIC@AfterFi{%
&     \BIC@@Shl#1#2!!%
$     \BIC@AddXY#1#2!#1#2!!!%
    }%
  \BIC@Fi
}
%    \end{macrocode}
%    \end{macro}
%    \begin{macro}{\BigIntCalcShl}
%    \begin{macrocode}
\def\BigIntCalcShl#1!{%
  \romannumeral0%
& \BIC@@Shl#1!!%
$ \BIC@AddXY#1!#1!!!%
}
%    \end{macrocode}
%    \end{macro}
%    \begin{macro}{\BIC@@Shl}
%    \begin{macrocode}
& \def\BIC@@Shl#1#2!{%
&   \ifx\\#2\\%
&     \BIC@AfterFi{%
&       \BIC@@@Shl0!#1%
&     }%
&   \else
&     \BIC@AfterFi{%
&       \BIC@@Shl#2!#1%
&     }%
&   \BIC@Fi
& }%
%    \end{macrocode}
%    \end{macro}
%    \begin{macro}{\BIC@@@Shl}
%    |#1|: carry\\
%    |#2|: result\\
%    |#3#4|: reverted number
%    \begin{macrocode}
& \def\BIC@@@Shl#1#2!#3#4!{%
&   \ifx\\#4\\%
&     \BIC@AfterFi{%
&       \expandafter\BIC@Space
&       \the\numexpr#3*2+#1\relax#2%
&     }%
&   \else
&     \BIC@AfterFi{%
&       \expandafter\BIC@@@@Shl\the\numexpr#3*2+#1!#2!#4!%
&     }%
&   \BIC@Fi
& }%
%    \end{macrocode}
%    \end{macro}
%    \begin{macro}{\BIC@@@@Shl}
%    \begin{macrocode}
& \def\BIC@@@@Shl#1!{%
&   \ifnum#1<10 %
&     \BIC@AfterFi{%
&       \BIC@@@Shl0#1%
&     }%
&   \else
&     \BIC@AfterFi{%
&       \BIC@@@Shl#1%
&     }%
&   \BIC@Fi
& }%
%    \end{macrocode}
%    \end{macro}
%
%    \begin{macro}{\bigintcalcShr}
%    \begin{macrocode}
\def\bigintcalcShr#1{%
  \romannumeral0%
  \expandafter\expandafter\expandafter\BIC@Shr
  \bigintcalcNum{#1}!%
}
%    \end{macrocode}
%    \end{macro}
%    \begin{macro}{\BIC@Shr}
%    \begin{macrocode}
\def\BIC@Shr#1#2!{%
  \ifx#1-%
    \expandafter-\romannumeral0%
    \BIC@AfterFi{%
      \BIC@@Shr#2!%
    }%
  \else
    \BIC@AfterFi{%
      \BIC@@Shr#1#2!%
    }%
  \BIC@Fi
}
%    \end{macrocode}
%    \end{macro}
%    \begin{macro}{\BigIntCalcShr}
%    \begin{macrocode}
\def\BigIntCalcShr#1!{%
  \romannumeral0%
  \BIC@@Shr#1!%
}
%    \end{macrocode}
%    \end{macro}
%    \begin{macro}{\BIC@@Shr}
%    \begin{macrocode}
\def\BIC@@Shr#1#2!{%
  \ifcase#1 %
    \BIC@AfterFi{ 0}%
  \or
    \ifx\\#2\\%
      \BIC@AfterFiFi{ 0}%
    \else
      \BIC@AfterFiFi{%
        \BIC@@@Shr#1#2!!%
      }%
    \fi
  \else
    \BIC@AfterFi{%
      \BIC@@@Shr0#1#2!!%
    }%
  \BIC@Fi
}
%    \end{macrocode}
%    \end{macro}
%    \begin{macro}{\BIC@@@Shr}
%    |#1|: carry\\
%    |#2#3|: number\\
%    |#4|: result
%    \begin{macrocode}
\def\BIC@@@Shr#1#2#3!#4!{%
  \ifx\\#3\\%
    \ifodd#1#2 %
      \BIC@AfterFiFi{%
&       \expandafter\BIC@ShrResult\the\numexpr(#1#2-1)/2\relax
$       \expandafter\expandafter\expandafter\BIC@ShrResult
$       \csname BIC@ShrDigit#1#2\endcsname
        #4!%
      }%
    \else
      \BIC@AfterFiFi{%
&       \expandafter\BIC@ShrResult\the\numexpr#1#2/2\relax
$       \expandafter\expandafter\expandafter\BIC@ShrResult
$       \csname BIC@ShrDigit#1#2\endcsname
        #4!%
      }%
    \fi
  \else
    \ifodd#1#2 %
      \BIC@AfterFiFi{%
&       \expandafter\BIC@@@@Shr\the\numexpr(#1#2-1)/2\relax1%
$       \expandafter\expandafter\expandafter\BIC@@@@Shr
$       \csname BIC@ShrDigit#1#2\endcsname
        #3!#4!%
      }%
    \else
      \BIC@AfterFiFi{%
&       \expandafter\BIC@@@@Shr\the\numexpr#1#2/2\relax0%
$       \expandafter\expandafter\expandafter\BIC@@@@Shr
$       \csname BIC@ShrDigit#1#2\endcsname
        #3!#4!%
      }%
    \fi
  \BIC@Fi
}
%    \end{macrocode}
%    \end{macro}
%    \begin{macro}{\BIC@ShrResult}
%    \begin{macrocode}
& \def\BIC@ShrResult#1#2!{ #2#1}%
$ \def\BIC@ShrResult#1#2#3!{ #3#1}%
%    \end{macrocode}
%    \end{macro}
%    \begin{macro}{\BIC@@@@Shr}
%    |#1|: new digit\\
%    |#2|: carry\\
%    |#3|: remaining number\\
%    |#4|: result
%    \begin{macrocode}
\def\BIC@@@@Shr#1#2#3!#4!{%
  \BIC@@@Shr#2#3!#4#1!%
}
%    \end{macrocode}
%    \end{macro}
%    \begin{macro}{\BIC@ShrDigit[00-19]}
%    \begin{macrocode}
$ \def\BIC@Temp#1#2#3#4{%
$   \expandafter\def\csname BIC@ShrDigit#1#2\endcsname{#3#4}%
$ }%
$ \BIC@Temp 0000%
$ \BIC@Temp 0101%
$ \BIC@Temp 0210%
$ \BIC@Temp 0311%
$ \BIC@Temp 0420%
$ \BIC@Temp 0521%
$ \BIC@Temp 0630%
$ \BIC@Temp 0731%
$ \BIC@Temp 0840%
$ \BIC@Temp 0941%
$ \BIC@Temp 1050%
$ \BIC@Temp 1151%
$ \BIC@Temp 1260%
$ \BIC@Temp 1361%
$ \BIC@Temp 1470%
$ \BIC@Temp 1571%
$ \BIC@Temp 1680%
$ \BIC@Temp 1781%
$ \BIC@Temp 1890%
$ \BIC@Temp 1991%
%    \end{macrocode}
%    \end{macro}
%
% \subsection{\cs{BIC@Tim}}
%
%    \begin{macro}{\BIC@Tim}
%    Macro \cs{BIC@Tim} implements ``Number \emph{tim}es digit''.\\
%    |#1|: plain number without sign\\
%    |#2|: digit
\def\BIC@Tim#1!#2{%
  \romannumeral0%
  \ifcase#2 % 0
    \BIC@AfterFi{ 0}%
  \or % 1
    \BIC@AfterFi{ #1}%
  \or % 2
    \BIC@AfterFi{%
      \BIC@Shl#1!%
    }%
  \else % 3-9
    \BIC@AfterFi{%
      \BIC@@Tim#1!!#2%
    }%
  \BIC@Fi
}
%    \begin{macrocode}
%    \end{macrocode}
%    \end{macro}
%    \begin{macro}{\BIC@@Tim}
%    |#1#2|: number\\
%    |#3|: reverted number
%    \begin{macrocode}
\def\BIC@@Tim#1#2!{%
  \ifx\\#2\\%
    \BIC@AfterFi{%
      \BIC@ProcessTim0!#1%
    }%
  \else
    \BIC@AfterFi{%
      \BIC@@Tim#2!#1%
    }%
  \BIC@Fi
}
%    \end{macrocode}
%    \end{macro}
%    \begin{macro}{\BIC@ProcessTim}
%    |#1|: carry\\
%    |#2|: result\\
%    |#3#4|: reverted number\\
%    |#5|: digit
%    \begin{macrocode}
\def\BIC@ProcessTim#1#2!#3#4!#5{%
  \ifx\\#4\\%
    \BIC@AfterFi{%
      \expandafter\BIC@Space
&     \the\numexpr#3*#5+#1\relax
$     \romannumeral0\BIC@TimDigit#3#5#1%
      #2%
    }%
  \else
    \BIC@AfterFi{%
      \expandafter\BIC@@ProcessTim
&     \the\numexpr#3*#5+#1%
$     \romannumeral0\BIC@TimDigit#3#5#1%
      !#2!#4!#5%
    }%
  \BIC@Fi
}
%    \end{macrocode}
%    \end{macro}
%    \begin{macro}{\BIC@@ProcessTim}
%    |#1#2|: carry?, new digit\\
%    |#3|: new number\\
%    |#4|: old number\\
%    |#5|: digit
%    \begin{macrocode}
\def\BIC@@ProcessTim#1#2!{%
  \ifx\\#2\\%
    \BIC@AfterFi{%
      \BIC@ProcessTim0#1%
    }%
  \else
    \BIC@AfterFi{%
      \BIC@ProcessTim#1#2%
    }%
  \BIC@Fi
}
%    \end{macrocode}
%    \end{macro}
%    \begin{macro}{\BIC@TimDigit}
%    |#1|: digit 0--9\\
%    |#2|: digit 3--9\\
%    |#3|: carry 0--9
%    \begin{macrocode}
$ \def\BIC@TimDigit#1#2#3{%
$   \ifcase#1 % 0
$     \BIC@AfterFi{ #3}%
$   \or % 1
$     \BIC@AfterFi{%
$       \expandafter\BIC@Space
$       \number\csname BIC@AddCarry#2\endcsname#3 %
$     }%
$   \else
$     \ifcase#3 %
$       \BIC@AfterFiFi{%
$         \expandafter\BIC@Space
$         \number\csname BIC@MulDigit#2\endcsname#1 %
$       }%
$     \else
$       \BIC@AfterFiFi{%
$         \expandafter\BIC@Space
$         \romannumeral0%
$         \expandafter\BIC@AddXY
$         \number\csname BIC@MulDigit#2\endcsname#1!%
$         #3!!!%
$       }%
$     \fi
$   \BIC@Fi
$ }%
%    \end{macrocode}
%    \end{macro}
%    \begin{macro}{\BIC@MulDigit[3-9]}
%    \begin{macrocode}
$ \def\BIC@Temp#1#2{%
$   \expandafter\def\csname BIC@MulDigit#1\endcsname##1{%
$     \ifcase##1 0%
$     \or ##1%
$     \or #2%
$?    \else\BigIntCalcError:ThisCannotHappen%
$     \fi
$   }%
$ }%
$ \BIC@Temp 3{6\or9\or12\or15\or18\or21\or24\or27}%
$ \BIC@Temp 4{8\or12\or16\or20\or24\or28\or32\or36}%
$ \BIC@Temp 5{10\or15\or20\or25\or30\or35\or40\or45}%
$ \BIC@Temp 6{12\or18\or24\or30\or36\or42\or48\or54}%
$ \BIC@Temp 7{14\or21\or28\or35\or42\or49\or56\or63}%
$ \BIC@Temp 8{16\or24\or32\or40\or48\or56\or64\or72}%
$ \BIC@Temp 9{18\or27\or36\or45\or54\or63\or72\or81}%
%    \end{macrocode}
%    \end{macro}
%
% \subsection{\op{Mul}}
%
%    \begin{macro}{\bigintcalcMul}
%    \begin{macrocode}
\def\bigintcalcMul#1#2{%
  \romannumeral0%
  \expandafter\expandafter\expandafter\BIC@Mul
  \bigintcalcNum{#1}!{#2}%
}
%    \end{macrocode}
%    \end{macro}
%    \begin{macro}{\BIC@Mul}
%    \begin{macrocode}
\def\BIC@Mul#1!#2{%
  \expandafter\expandafter\expandafter\BIC@MulSwitch
  \bigintcalcNum{#2}!#1!%
}
%    \end{macrocode}
%    \end{macro}
%    \begin{macro}{\BIC@MulSwitch}
%    Decision table for \cs{BIC@MulSwitch}.
%    \begin{quote}
%    \begin{tabular}[t]{@{}|l|l|l|l|l|@{}}
%      \hline
%      $x=0$ &\multicolumn{3}{l}{}    &$0$\\
%      \hline
%      $x>0$ & $y=0$ &\multicolumn2l{}&$0$\\
%      \cline{2-5}
%            & $y>0$ & $ x> y$ & $+$ & $\opMul( x, y)$\\
%      \cline{3-3}\cline{5-5}
%            &       & else    &     & $\opMul( y, x)$\\
%      \cline{2-5}
%            & $y<0$ & $ x>-y$ & $-$ & $\opMul( x,-y)$\\
%      \cline{3-3}\cline{5-5}
%            &       & else    &     & $\opMul(-y, x)$\\
%      \hline
%      $x<0$ & $y=0$ &\multicolumn2l{}&$0$\\
%      \cline{2-5}
%            & $y>0$ & $-x> y$ & $-$ & $\opMul(-x, y)$\\
%      \cline{3-3}\cline{5-5}
%            &       & else    &     & $\opMul( y,-x)$\\
%      \cline{2-5}
%            & $y<0$ & $-x>-y$ & $+$ & $\opMul(-x,-y)$\\
%      \cline{3-3}\cline{5-5}
%            &       & else    &     & $\opMul(-y,-x)$\\
%      \hline
%    \end{tabular}
%    \end{quote}
%    \begin{macrocode}
\def\BIC@MulSwitch#1#2!#3#4!{%
  \ifcase\BIC@Sgn#1#2! % x = 0
    \BIC@AfterFi{ 0}%
  \or % x > 0
    \ifcase\BIC@Sgn#3#4! % y = 0
      \BIC@AfterFiFi{ 0}%
    \or % y > 0
      \ifnum\BIC@PosCmp#1#2!#3#4!=1 % x > y
        \BIC@AfterFiFiFi{%
          \BIC@ProcessMul0!#1#2!#3#4!%
        }%
      \else % x <= y
        \BIC@AfterFiFiFi{%
          \BIC@ProcessMul0!#3#4!#1#2!%
        }%
      \fi
    \else % y < 0
      \expandafter-\romannumeral0%
      \ifnum\BIC@PosCmp#1#2!#4!=1 % x > -y
        \BIC@AfterFiFiFi{%
          \BIC@ProcessMul0!#1#2!#4!%
        }%
      \else % x <= -y
        \BIC@AfterFiFiFi{%
          \BIC@ProcessMul0!#4!#1#2!%
        }%
      \fi
    \fi
  \else % x < 0
    \ifcase\BIC@Sgn#3#4! % y = 0
      \BIC@AfterFiFi{ 0}%
    \or % y > 0
      \expandafter-\romannumeral0%
      \ifnum\BIC@PosCmp#2!#3#4!=1 % -x > y
        \BIC@AfterFiFiFi{%
          \BIC@ProcessMul0!#2!#3#4!%
        }%
      \else % -x <= y
        \BIC@AfterFiFiFi{%
          \BIC@ProcessMul0!#3#4!#2!%
        }%
      \fi
    \else % y < 0
      \ifnum\BIC@PosCmp#2!#4!=1 % -x > -y
        \BIC@AfterFiFiFi{%
          \BIC@ProcessMul0!#2!#4!%
        }%
      \else % -x <= -y
        \BIC@AfterFiFiFi{%
          \BIC@ProcessMul0!#4!#2!%
        }%
      \fi
    \fi
  \BIC@Fi
}
%    \end{macrocode}
%    \end{macro}
%    \begin{macro}{\BigIntCalcMul}
%    \begin{macrocode}
\def\BigIntCalcMul#1!#2!{%
  \romannumeral0%
  \BIC@ProcessMul0!#1!#2!%
}
%    \end{macrocode}
%    \end{macro}
%    \begin{macro}{\BIC@ProcessMul}
%    |#1|: result\\
%    |#2|: number $x$\\
%    |#3#4|: number $y$
%    \begin{macrocode}
\def\BIC@ProcessMul#1!#2!#3#4!{%
  \ifx\\#4\\%
    \BIC@AfterFi{%
      \expandafter\expandafter\expandafter\BIC@Space
      \bigintcalcAdd{\BIC@Tim#2!#3}{#10}%
    }%
  \else
    \BIC@AfterFi{%
      \expandafter\expandafter\expandafter\BIC@ProcessMul
      \bigintcalcAdd{\BIC@Tim#2!#3}{#10}!#2!#4!%
    }%
  \BIC@Fi
}
%    \end{macrocode}
%    \end{macro}
%
% \subsection{\op{Sqr}}
%
%    \begin{macro}{\bigintcalcSqr}
%    \begin{macrocode}
\def\bigintcalcSqr#1{%
  \romannumeral0%
  \expandafter\expandafter\expandafter\BIC@Sqr
  \bigintcalcNum{#1}!%
}
%    \end{macrocode}
%    \end{macro}
%    \begin{macro}{\BIC@Sqr}
%    \begin{macrocode}
\def\BIC@Sqr#1{%
   \ifx#1-%
     \expandafter\BIC@@Sqr
   \else
     \expandafter\BIC@@Sqr\expandafter#1%
   \fi
}
%    \end{macrocode}
%    \end{macro}
%    \begin{macro}{\BIC@@Sqr}
%    \begin{macrocode}
\def\BIC@@Sqr#1!{%
  \BIC@ProcessMul0!#1!#1!%
}
%    \end{macrocode}
%    \end{macro}
%
% \subsection{\op{Fac}}
%
%    \begin{macro}{\bigintcalcFac}
%    \begin{macrocode}
\def\bigintcalcFac#1{%
  \romannumeral0%
  \expandafter\expandafter\expandafter\BIC@Fac
  \bigintcalcNum{#1}!%
}
%    \end{macrocode}
%    \end{macro}
%    \begin{macro}{\BIC@Fac}
%    \begin{macrocode}
\def\BIC@Fac#1#2!{%
  \ifx#1-%
    \BIC@AfterFi{ 0\BigIntCalcError:FacNegative}%
  \else
    \ifnum\BIC@PosCmp#1#2!13!<0 %
      \ifcase#1#2 %
         \BIC@AfterFiFiFi{ 1}% 0!
      \or\BIC@AfterFiFiFi{ 1}% 1!
      \or\BIC@AfterFiFiFi{ 2}% 2!
      \or\BIC@AfterFiFiFi{ 6}% 3!
      \or\BIC@AfterFiFiFi{ 24}% 4!
      \or\BIC@AfterFiFiFi{ 120}% 5!
      \or\BIC@AfterFiFiFi{ 720}% 6!
      \or\BIC@AfterFiFiFi{ 5040}% 7!
      \or\BIC@AfterFiFiFi{ 40320}% 8!
      \or\BIC@AfterFiFiFi{ 362880}% 9!
      \or\BIC@AfterFiFiFi{ 3628800}% 10!
      \or\BIC@AfterFiFiFi{ 39916800}% 11!
      \or\BIC@AfterFiFiFi{ 479001600}% 12!
?     \else\BigIntCalcError:ThisCannotHappen%
      \fi
    \else
      \BIC@AfterFiFi{%
        \BIC@ProcessFac#1#2!479001600!%
      }%
    \fi
  \BIC@Fi
}
%    \end{macrocode}
%    \end{macro}
%    \begin{macro}{\BIC@ProcessFac}
%    |#1|: $n$\\
%    |#2|: result
%    \begin{macrocode}
\def\BIC@ProcessFac#1!#2!{%
  \ifnum\BIC@PosCmp#1!12!=0 %
    \BIC@AfterFi{ #2}%
  \else
    \BIC@AfterFi{%
      \expandafter\BIC@@ProcessFac
      \romannumeral0\BIC@ProcessMul0!#2!#1!%
      !#1!%
    }%
  \BIC@Fi
}
%    \end{macrocode}
%    \end{macro}
%    \begin{macro}{\BIC@@ProcessFac}
%    |#1|: result\\
%    |#2|: $n$
%    \begin{macrocode}
\def\BIC@@ProcessFac#1!#2!{%
  \expandafter\BIC@ProcessFac
  \romannumeral0\BIC@Dec#2!{}%
  !#1!%
}
%    \end{macrocode}
%    \end{macro}
%
% \subsection{\op{Pow}}
%
%    \begin{macro}{\bigintcalcPow}
%    |#1|: basis\\
%    |#2|: power
%    \begin{macrocode}
\def\bigintcalcPow#1{%
  \romannumeral0%
  \expandafter\expandafter\expandafter\BIC@Pow
  \bigintcalcNum{#1}!%
}
%    \end{macrocode}
%    \end{macro}
%    \begin{macro}{\BIC@Pow}
%    |#1|: basis\\
%    |#2|: power
%    \begin{macrocode}
\def\BIC@Pow#1!#2{%
  \expandafter\expandafter\expandafter\BIC@PowSwitch
  \bigintcalcNum{#2}!#1!%
}
%    \end{macrocode}
%    \end{macro}
%    \begin{macro}{\BIC@PowSwitch}
%    |#1#2|: power $y$\\
%    |#3#4|: basis $x$\\
%    Decision table for \cs{BIC@PowSwitch}.
%    \begin{quote}
%    \def\M#1{\multicolumn{#1}{l}{}}%
%    \begin{tabular}[t]{@{}|l|l|l|l|@{}}
%    \hline
%    $y=0$ & \M{2}                        & $1$\\
%    \hline
%    $y=1$ & \M{2}                        & $x$\\
%    \hline
%    $y=2$ & $x<0$          & \M{1}       & $\opMul(-x,-x)$\\
%    \cline{2-4}
%          & else           & \M{1}       & $\opMul(x,x)$\\
%    \hline
%    $y<0$ & $x=0$          & \M{1}       & DivisionByZero\\
%    \cline{2-4}
%          & $x=1$          & \M{1}       & $1$\\
%    \cline{2-4}
%          & $x=-1$         & $\opODD(y)$ & $-1$\\
%    \cline{3-4}
%          &                & else        & $1$\\
%    \cline{2-4}
%          & else ($\string|x\string|>1$) & \M{1}       & $0$\\
%    \hline
%    $y>2$ & $x=0$          & \M{1}       & $0$\\
%    \cline{2-4}
%          & $x=1$          & \M{1}       & $1$\\
%    \cline{2-4}
%          & $x=-1$         & $\opODD(y)$ & $-1$\\
%    \cline{3-4}
%          &                & else        & $1$\\
%    \cline{2-4}
%          & $x<-1$ ($x<0$) & $\opODD(y)$ & $-\opPow(-x,y)$\\
%    \cline{3-4}
%          &                & else        & $\opPow(-x,y)$\\
%    \cline{2-4}
%          & else ($x>1$)   & \M{1}       & $\opPow(x,y)$\\
%    \hline
%    \end{tabular}
%    \end{quote}
%    \begin{macrocode}
\def\BIC@PowSwitch#1#2!#3#4!{%
  \ifcase\ifx\\#2\\%
           \ifx#100 % y = 0
           \else\ifx#111 % y = 1
           \else\ifx#122 % y = 2
           \else4 % y > 2
           \fi\fi\fi
         \else
           \ifx#1-3 % y < 0
           \else4 % y > 2
           \fi
         \fi
    \BIC@AfterFi{ 1}% y = 0
  \or % y = 1
    \BIC@AfterFi{ #3#4}%
  \or % y = 2
    \ifx#3-% x < 0
      \BIC@AfterFiFi{%
        \BIC@ProcessMul0!#4!#4!%
      }%
    \else % x >= 0
      \BIC@AfterFiFi{%
        \BIC@ProcessMul0!#3#4!#3#4!%
      }%
    \fi
  \or % y < 0
    \ifcase\ifx\\#4\\%
             \ifx#300 % x = 0
             \else\ifx#311 % x = 1
             \else3 % x > 1
             \fi\fi
           \else
             \ifcase\BIC@MinusOne#3#4! %
               3 % |x| > 1
             \or
               2 % x = -1
?            \else\BigIntCalcError:ThisCannotHappen%
             \fi
           \fi
      \BIC@AfterFiFi{ 0\BigIntCalcError:DivisionByZero}% x = 0
    \or % x = 1
      \BIC@AfterFiFi{ 1}% x = 1
    \or % x = -1
      \ifcase\BIC@ModTwo#2! % even(y)
        \BIC@AfterFiFiFi{ 1}%
      \or % odd(y)
        \BIC@AfterFiFiFi{ -1}%
?     \else\BigIntCalcError:ThisCannotHappen%
      \fi
    \or % |x| > 1
      \BIC@AfterFiFi{ 0}%
?   \else\BigIntCalcError:ThisCannotHappen%
    \fi
  \or % y > 2
    \ifcase\ifx\\#4\\%
             \ifx#300 % x = 0
             \else\ifx#311 % x = 1
             \else4 % x > 1
             \fi\fi
           \else
             \ifx#3-%
               \ifcase\BIC@MinusOne#3#4! %
                 3 % x < -1
               \else
                 2 % x = -1
               \fi
             \else
               4 % x > 1
             \fi
           \fi
      \BIC@AfterFiFi{ 0}% x = 0
    \or % x = 1
      \BIC@AfterFiFi{ 1}% x = 1
    \or % x = -1
      \ifcase\BIC@ModTwo#1#2! % even(y)
        \BIC@AfterFiFiFi{ 1}%
      \or % odd(y)
        \BIC@AfterFiFiFi{ -1}%
?     \else\BigIntCalcError:ThisCannotHappen%
      \fi
    \or % x < -1
      \ifcase\BIC@ModTwo#1#2! % even(y)
        \BIC@AfterFiFiFi{%
          \BIC@PowRec#4!#1#2!1!%
        }%
      \or % odd(y)
        \expandafter-\romannumeral0%
        \BIC@AfterFiFiFi{%
          \BIC@PowRec#4!#1#2!1!%
        }%
?     \else\BigIntCalcError:ThisCannotHappen%
      \fi
    \or % x > 1
      \BIC@AfterFiFi{%
        \BIC@PowRec#3#4!#1#2!1!%
      }%
?   \else\BigIntCalcError:ThisCannotHappen%
    \fi
? \else\BigIntCalcError:ThisCannotHappen%
  \BIC@Fi
}
%    \end{macrocode}
%    \end{macro}
%
% \subsubsection{Help macros}
%
%    \begin{macro}{\BIC@ModTwo}
%    Macro \cs{BIC@ModTwo} expects a number without sign
%    and returns digit |1| or |0| if the number is odd or even.
%    \begin{macrocode}
\def\BIC@ModTwo#1#2!{%
  \ifx\\#2\\%
    \ifodd#1 %
      \BIC@AfterFiFi1%
    \else
      \BIC@AfterFiFi0%
    \fi
  \else
    \BIC@AfterFi{%
      \BIC@ModTwo#2!%
    }%
  \BIC@Fi
}
%    \end{macrocode}
%    \end{macro}
%
%    \begin{macro}{\BIC@MinusOne}
%    Macro \cs{BIC@MinusOne} expects a number and returns
%    digit |1| if the number equals minus one and returns |0| otherwise.
%    \begin{macrocode}
\def\BIC@MinusOne#1#2!{%
  \ifx#1-%
    \BIC@@MinusOne#2!%
  \else
    0%
  \fi
}
%    \end{macrocode}
%    \end{macro}
%    \begin{macro}{\BIC@@MinusOne}
%    \begin{macrocode}
\def\BIC@@MinusOne#1#2!{%
  \ifx#11%
    \ifx\\#2\\%
      1%
    \else
      0%
    \fi
  \else
    0%
  \fi
}
%    \end{macrocode}
%    \end{macro}
%
% \subsubsection{Recursive calculation}
%
%    \begin{macro}{\BIC@PowRec}
%\begin{quote}
%\begin{verbatim}
%Pow(x, y) {
%  PowRec(x, y, 1)
%}
%PowRec(x, y, r) {
%  if y == 1 then
%    return r
%  else
%    ifodd y then
%      return PowRec(x*x, y div 2, r*x) % y div 2 = (y-1)/2
%    else
%      return PowRec(x*x, y div 2, r)
%    fi
%  fi
%}
%\end{verbatim}
%\end{quote}
%    |#1|: $x$ (basis)\\
%    |#2#3|: $y$ (power)\\
%    |#4|: $r$ (result)
%    \begin{macrocode}
\def\BIC@PowRec#1!#2#3!#4!{%
  \ifcase\ifx#21\ifx\\#3\\0 \else1 \fi\else1 \fi % y = 1
    \ifnum\BIC@PosCmp#1!#4!=1 % x > r
      \BIC@AfterFiFi{%
        \BIC@ProcessMul0!#1!#4!%
      }%
    \else
      \BIC@AfterFiFi{%
        \BIC@ProcessMul0!#4!#1!%
      }%
    \fi
  \or
    \ifcase\BIC@ModTwo#2#3! % even(y)
      \BIC@AfterFiFi{%
        \expandafter\BIC@@PowRec\romannumeral0%
        \BIC@@Shr#2#3!%
        !#1!#4!%
      }%
    \or % odd(y)
      \ifnum\BIC@PosCmp#1!#4!=1 % x > r
        \BIC@AfterFiFiFi{%
          \expandafter\BIC@@@PowRec\romannumeral0%
          \BIC@ProcessMul0!#1!#4!%
          !#1!#2#3!%
        }%
      \else
        \BIC@AfterFiFiFi{%
          \expandafter\BIC@@@PowRec\romannumeral0%
          \BIC@ProcessMul0!#1!#4!%
          !#1!#2#3!%
        }%
      \fi
?   \else\BigIntCalcError:ThisCannotHappen%
    \fi
? \else\BigIntCalcError:ThisCannotHappen%
  \BIC@Fi
}
%    \end{macrocode}
%    \end{macro}
%    \begin{macro}{\BIC@@PowRec}
%    |#1|: $y/2$\\
%    |#2|: $x$\\
%    |#3|: new $r$ ($r$ or $r*x$)
%    \begin{macrocode}
\def\BIC@@PowRec#1!#2!#3!{%
  \expandafter\BIC@PowRec\romannumeral0%
  \BIC@ProcessMul0!#2!#2!%
  !#1!#3!%
}
%    \end{macrocode}
%    \end{macro}
%    \begin{macro}{\BIC@@@PowRec}
%    |#1|: $r*x$
%    |#2|: $x$
%    |#3|: $y$
%    \begin{macrocode}
\def\BIC@@@PowRec#1!#2!#3!{%
  \expandafter\BIC@@PowRec\romannumeral0%
  \BIC@@Shr#3!%
  !#2!#1!%
}
%    \end{macrocode}
%    \end{macro}
%
% \subsection{\op{Div}}
%
%    \begin{macro}{\bigintcalcDiv}
%    |#1|: $x$\\
%    |#2|: $y$ (divisor)
%    \begin{macrocode}
\def\bigintcalcDiv#1{%
  \romannumeral0%
  \expandafter\expandafter\expandafter\BIC@Div
  \bigintcalcNum{#1}!%
}
%    \end{macrocode}
%    \end{macro}
%    \begin{macro}{\BIC@Div}
%    |#1|: $x$\\
%    |#2|: $y$
%    \begin{macrocode}
\def\BIC@Div#1!#2{%
  \expandafter\expandafter\expandafter\BIC@DivSwitchSign
  \bigintcalcNum{#2}!#1!%
}
%    \end{macrocode}
%    \end{macro}
%    \begin{macro}{\BigIntCalcDiv}
%    \begin{macrocode}
\def\BigIntCalcDiv#1!#2!{%
  \romannumeral0%
  \BIC@DivSwitchSign#2!#1!%
}
%    \end{macrocode}
%    \end{macro}
%    \begin{macro}{\BIC@DivSwitchSign}
%    Decision table for \cs{BIC@DivSwitchSign}.
%    \begin{quote}
%    \begin{tabular}{@{}|l|l|l|@{}}
%    \hline
%    $y=0$ & \multicolumn{1}{l}{} & DivisionByZero\\
%    \hline
%    $y>0$ & $x=0$ & $0$\\
%    \cline{2-3}
%          & $x>0$ & DivSwitch$(+,x,y)$\\
%    \cline{2-3}
%          & $x<0$ & DivSwitch$(-,-x,y)$\\
%    \hline
%    $y<0$ & $x=0$ & $0$\\
%    \cline{2-3}
%          & $x>0$ & DivSwitch$(-,x,-y)$\\
%    \cline{2-3}
%          & $x<0$ & DivSwitch$(+,-x,-y)$\\
%    \hline
%    \end{tabular}
%    \end{quote}
%    |#1|: $y$ (divisor)\\
%    |#2|: $x$
%    \begin{macrocode}
\def\BIC@DivSwitchSign#1#2!#3#4!{%
  \ifcase\BIC@Sgn#1#2! % y = 0
    \BIC@AfterFi{ 0\BigIntCalcError:DivisionByZero}%
  \or % y > 0
    \ifcase\BIC@Sgn#3#4! % x = 0
      \BIC@AfterFiFi{ 0}%
    \or % x > 0
      \BIC@AfterFiFi{%
        \BIC@DivSwitch{}#3#4!#1#2!%
      }%
    \else % x < 0
      \BIC@AfterFiFi{%
        \BIC@DivSwitch-#4!#1#2!%
      }%
    \fi
  \else % y < 0
    \ifcase\BIC@Sgn#3#4! % x = 0
      \BIC@AfterFiFi{ 0}%
    \or % x > 0
      \BIC@AfterFiFi{%
        \BIC@DivSwitch-#3#4!#2!%
      }%
    \else % x < 0
      \BIC@AfterFiFi{%
        \BIC@DivSwitch{}#4!#2!%
      }%
    \fi
  \BIC@Fi
}
%    \end{macrocode}
%    \end{macro}
%    \begin{macro}{\BIC@DivSwitch}
%    Decision table for \cs{BIC@DivSwitch}.
%    \begin{quote}
%    \begin{tabular}{@{}|l|l|l|@{}}
%    \hline
%    $y=x$ & \multicolumn{1}{l}{} & sign $1$\\
%    \hline
%    $y>x$ & \multicolumn{1}{l}{} & $0$\\
%    \hline
%    $y<x$ & $y=1$                & sign $x$\\
%    \cline{2-3}
%          & $y=2$                & sign Shr$(x)$\\
%    \cline{2-3}
%          & $y=4$                & sign Shr(Shr$(x)$)\\
%    \cline{2-3}
%          & else                 & sign ProcessDiv$(x,y)$\\
%    \hline
%    \end{tabular}
%    \end{quote}
%    |#1|: sign\\
%    |#2|: $x$\\
%    |#3#4|: $y$ ($y\ne 0$)
%    \begin{macrocode}
\def\BIC@DivSwitch#1#2!#3#4!{%
  \ifcase\BIC@PosCmp#3#4!#2!% y = x
    \BIC@AfterFi{ #11}%
  \or % y > x
    \BIC@AfterFi{ 0}%
  \else % y < x
    \ifx\\#1\\%
    \else
      \expandafter-\romannumeral0%
    \fi
    \ifcase\ifx\\#4\\%
             \ifx#310 % y = 1
             \else\ifx#321 % y = 2
             \else\ifx#342 % y = 4
             \else3 % y > 2
             \fi\fi\fi
           \else
             3 % y > 2
           \fi
      \BIC@AfterFiFi{ #2}% y = 1
    \or % y = 2
      \BIC@AfterFiFi{%
        \BIC@@Shr#2!%
      }%
    \or % y = 4
      \BIC@AfterFiFi{%
        \expandafter\BIC@@Shr\romannumeral0%
          \BIC@@Shr#2!!%
      }%
    \or % y > 2
      \BIC@AfterFiFi{%
        \BIC@DivStartX#2!#3#4!!!%
      }%
?   \else\BigIntCalcError:ThisCannotHappen%
    \fi
  \BIC@Fi
}
%    \end{macrocode}
%    \end{macro}
%    \begin{macro}{\BIC@ProcessDiv}
%    |#1#2|: $x$\\
%    |#3#4|: $y$\\
%    |#5|: collect first digits of $x$\\
%    |#6|: corresponding digits of $y$
%    \begin{macrocode}
\def\BIC@DivStartX#1#2!#3#4!#5!#6!{%
  \ifx\\#4\\%
    \BIC@AfterFi{%
      \BIC@DivStartYii#6#3#4!{#5#1}#2=!%
    }%
  \else
    \BIC@AfterFi{%
      \BIC@DivStartX#2!#4!#5#1!#6#3!%
    }%
  \BIC@Fi
}
%    \end{macrocode}
%    \end{macro}
%    \begin{macro}{\BIC@DivStartYii}
%    |#1|: $y$\\
%    |#2|: $x$, |=|
%    \begin{macrocode}
\def\BIC@DivStartYii#1!{%
  \expandafter\BIC@DivStartYiv\romannumeral0%
  \BIC@Shl#1!%
  !#1!%
}
%    \end{macrocode}
%    \end{macro}
%    \begin{macro}{\BIC@DivStartYiv}
%    |#1|: $2y$\\
%    |#2|: $y$\\
%    |#3|: $x$, |=|
%    \begin{macrocode}
\def\BIC@DivStartYiv#1!{%
  \expandafter\BIC@DivStartYvi\romannumeral0%
  \BIC@Shl#1!%
  !#1!%
}
%    \end{macrocode}
%    \end{macro}
%    \begin{macro}{\BIC@DivStartYvi}
%    |#1|: $4y$\\
%    |#2|: $2y$\\
%    |#3|: $y$\\
%    |#4|: $x$, |=|
%    \begin{macrocode}
\def\BIC@DivStartYvi#1!#2!{%
  \expandafter\BIC@DivStartYviii\romannumeral0%
  \BIC@AddXY#1!#2!!!%
  !#1!#2!%
}
%    \end{macrocode}
%    \end{macro}
%    \begin{macro}{\BIC@DivStartYviii}
%    |#1|: $6y$\\
%    |#2|: $4y$\\
%    |#3|: $2y$\\
%    |#4|: $y$\\
%    |#5|: $x$, |=|
%    \begin{macrocode}
\def\BIC@DivStartYviii#1!#2!{%
  \expandafter\BIC@DivStart\romannumeral0%
  \BIC@Shl#2!%
  !#1!#2!%
}
%    \end{macrocode}
%    \end{macro}
%    \begin{macro}{\BIC@DivStart}
%    |#1|: $8y$\\
%    |#2|: $6y$\\
%    |#3|: $4y$\\
%    |#4|: $2y$\\
%    |#5|: $y$\\
%    |#6|: $x$, |=|
%    \begin{macrocode}
\def\BIC@DivStart#1!#2!#3!#4!#5!#6!{%
  \BIC@ProcessDiv#6!!#5!#4!#3!#2!#1!=%
}
%    \end{macrocode}
%    \end{macro}
%    \begin{macro}{\BIC@ProcessDiv}
%    |#1#2#3|: $x$, |=|\\
%    |#4|: result\\
%    |#5|: $y$\\
%    |#6|: $2y$\\
%    |#7|: $4y$\\
%    |#8|: $6y$\\
%    |#9|: $8y$
%    \begin{macrocode}
\def\BIC@ProcessDiv#1#2#3!#4!#5!{%
  \ifcase\BIC@PosCmp#5!#1!% y = #1
    \ifx#2=%
      \BIC@AfterFiFi{\BIC@DivCleanup{#41}}%
    \else
      \BIC@AfterFiFi{%
        \BIC@ProcessDiv#2#3!#41!#5!%
      }%
    \fi
  \or % y > #1
    \ifx#2=%
      \BIC@AfterFiFi{\BIC@DivCleanup{#40}}%
    \else
      \ifx\\#4\\%
        \BIC@AfterFiFiFi{%
          \BIC@ProcessDiv{#1#2}#3!!#5!%
        }%
      \else
        \BIC@AfterFiFiFi{%
          \BIC@ProcessDiv{#1#2}#3!#40!#5!%
        }%
      \fi
    \fi
  \else % y < #1
    \BIC@AfterFi{%
      \BIC@@ProcessDiv{#1}#2#3!#4!#5!%
    }%
  \BIC@Fi
}
%    \end{macrocode}
%    \end{macro}
%    \begin{macro}{\BIC@DivCleanup}
%    |#1|: result\\
%    |#2|: garbage
%    \begin{macrocode}
\def\BIC@DivCleanup#1#2={ #1}%
%    \end{macrocode}
%    \end{macro}
%    \begin{macro}{\BIC@@ProcessDiv}
%    \begin{macrocode}
\def\BIC@@ProcessDiv#1#2#3!#4!#5!#6!#7!{%
  \ifcase\BIC@PosCmp#7!#1!% 4y = #1
    \ifx#2=%
      \BIC@AfterFiFi{\BIC@DivCleanup{#44}}%
    \else
     \BIC@AfterFiFi{%
       \BIC@ProcessDiv#2#3!#44!#5!#6!#7!%
     }%
    \fi
  \or % 4y > #1
    \ifcase\BIC@PosCmp#6!#1!% 2y = #1
      \ifx#2=%
        \BIC@AfterFiFiFi{\BIC@DivCleanup{#42}}%
      \else
        \BIC@AfterFiFiFi{%
          \BIC@ProcessDiv#2#3!#42!#5!#6!#7!%
        }%
      \fi
    \or % 2y > #1
      \ifx#2=%
        \BIC@AfterFiFiFi{\BIC@DivCleanup{#41}}%
      \else
        \BIC@AfterFiFiFi{%
          \BIC@DivSub#1!#5!#2#3!#41!#5!#6!#7!%
        }%
      \fi
    \else % 2y < #1
      \BIC@AfterFiFi{%
        \expandafter\BIC@ProcessDivII\romannumeral0%
        \BIC@SubXY#1!#6!!!%
        !#2#3!#4!#5!23%
        #6!#7!%
      }%
    \fi
  \else % 4y < #1
    \BIC@AfterFi{%
      \BIC@@@ProcessDiv{#1}#2#3!#4!#5!#6!#7!%
    }%
  \BIC@Fi
}
%    \end{macrocode}
%    \end{macro}
%    \begin{macro}{\BIC@DivSub}
%    Next token group: |#1|-|#2| and next digit |#3|.
%    \begin{macrocode}
\def\BIC@DivSub#1!#2!#3{%
  \expandafter\BIC@ProcessDiv\expandafter{%
    \romannumeral0%
    \BIC@SubXY#1!#2!!!%
    #3%
  }%
}
%    \end{macrocode}
%    \end{macro}
%    \begin{macro}{\BIC@ProcessDivII}
%    |#1|: $x'-2y$\\
%    |#2#3|: remaining $x$, |=|\\
%    |#4|: result\\
%    |#5|: $y$\\
%    |#6|: first possible result digit\\
%    |#7|: second possible result digit
%    \begin{macrocode}
\def\BIC@ProcessDivII#1!#2#3!#4!#5!#6#7{%
  \ifcase\BIC@PosCmp#5!#1!% y = #1
    \ifx#2=%
      \BIC@AfterFiFi{\BIC@DivCleanup{#4#7}}%
    \else
      \BIC@AfterFiFi{%
        \BIC@ProcessDiv#2#3!#4#7!#5!%
      }%
    \fi
  \or % y > #1
    \ifx#2=%
      \BIC@AfterFiFi{\BIC@DivCleanup{#4#6}}%
    \else
      \BIC@AfterFiFi{%
        \BIC@ProcessDiv{#1#2}#3!#4#6!#5!%
      }%
    \fi
  \else % y < #1
    \ifx#2=%
      \BIC@AfterFiFi{\BIC@DivCleanup{#4#7}}%
    \else
      \BIC@AfterFiFi{%
        \BIC@DivSub#1!#5!#2#3!#4#7!#5!%
      }%
    \fi
  \BIC@Fi
}
%    \end{macrocode}
%    \end{macro}
%    \begin{macro}{\BIC@ProcessDivIV}
%    |#1#2#3|: $x$, |=|, $x>4y$\\
%    |#4|: result\\
%    |#5|: $y$\\
%    |#6|: $2y$\\
%    |#7|: $4y$\\
%    |#8|: $6y$\\
%    |#9|: $8y$
%    \begin{macrocode}
\def\BIC@@@ProcessDiv#1#2#3!#4!#5!#6!#7!#8!#9!{%
  \ifcase\BIC@PosCmp#8!#1!% 6y = #1
    \ifx#2=%
      \BIC@AfterFiFi{\BIC@DivCleanup{#46}}%
    \else
      \BIC@AfterFiFi{%
        \BIC@ProcessDiv#2#3!#46!#5!#6!#7!#8!#9!%
      }%
    \fi
  \or % 6y > #1
    \BIC@AfterFi{%
      \expandafter\BIC@ProcessDivII\romannumeral0%
      \BIC@SubXY#1!#7!!!%
      !#2#3!#4!#5!45%
      #6!#7!#8!#9!%
    }%
  \else % 6y < #1
    \ifcase\BIC@PosCmp#9!#1!% 8y = #1
      \ifx#2=%
        \BIC@AfterFiFiFi{\BIC@DivCleanup{#48}}%
      \else
        \BIC@AfterFiFiFi{%
          \BIC@ProcessDiv#2#3!#48!#5!#6!#7!#8!#9!%
        }%
      \fi
    \or % 8y > #1
      \BIC@AfterFiFi{%
        \expandafter\BIC@ProcessDivII\romannumeral0%
        \BIC@SubXY#1!#8!!!%
        !#2#3!#4!#5!67%
        #6!#7!#8!#9!%
      }%
    \else % 8y < #1
      \BIC@AfterFiFi{%
        \expandafter\BIC@ProcessDivII\romannumeral0%
        \BIC@SubXY#1!#9!!!%
        !#2#3!#4!#5!89%
        #6!#7!#8!#9!%
      }%
    \fi
  \BIC@Fi
}
%    \end{macrocode}
%    \end{macro}
%
% \subsection{\op{Mod}}
%
%    \begin{macro}{\bigintcalcMod}
%    |#1|: $x$\\
%    |#2|: $y$
%    \begin{macrocode}
\def\bigintcalcMod#1{%
  \romannumeral0%
  \expandafter\expandafter\expandafter\BIC@Mod
  \bigintcalcNum{#1}!%
}
%    \end{macrocode}
%    \end{macro}
%    \begin{macro}{\BIC@Mod}
%    |#1|: $x$\\
%    |#2|: $y$
%    \begin{macrocode}
\def\BIC@Mod#1!#2{%
  \expandafter\expandafter\expandafter\BIC@ModSwitchSign
  \bigintcalcNum{#2}!#1!%
}
%    \end{macrocode}
%    \end{macro}
%    \begin{macro}{\BigIntCalcMod}
%    \begin{macrocode}
\def\BigIntCalcMod#1!#2!{%
  \romannumeral0%
  \BIC@ModSwitchSign#2!#1!%
}
%    \end{macrocode}
%    \end{macro}
%    \begin{macro}{\BIC@ModSwitchSign}
%    Decision table for \cs{BIC@ModSwitchSign}.
%    \begin{quote}
%    \begin{tabular}{@{}|l|l|l|@{}}
%    \hline
%    $y=0$ & \multicolumn{1}{l}{} & DivisionByZero\\
%    \hline
%    $y>0$ & $x=0$ & $0$\\
%    \cline{2-3}
%          & else  & ModSwitch$(+,x,y)$\\
%    \hline
%    $y<0$ & \multicolumn{1}{l}{} & ModSwitch$(-,-x,-y)$\\
%    \hline
%    \end{tabular}
%    \end{quote}
%    |#1#2|: $y$\\
%    |#3#4|: $x$
%    \begin{macrocode}
\def\BIC@ModSwitchSign#1#2!#3#4!{%
  \ifcase\ifx\\#2\\%
           \ifx#100 % y = 0
           \else1 % y > 0
           \fi
          \else
            \ifx#1-2 % y < 0
            \else1 % y > 0
            \fi
          \fi
    \BIC@AfterFi{ 0\BigIntCalcError:DivisionByZero}%
  \or % y > 0
    \ifcase\ifx\\#4\\\ifx#300 \else1 \fi\else1 \fi % x = 0
      \BIC@AfterFiFi{ 0}%
    \else
      \BIC@AfterFiFi{%
        \BIC@ModSwitch{}#3#4!#1#2!%
      }%
    \fi
  \else % y < 0
    \ifcase\ifx\\#4\\%
             \ifx#300 % x = 0
             \else1 % x > 0
             \fi
           \else
             \ifx#3-2 % x < 0
             \else1 % x > 0
             \fi
           \fi
      \BIC@AfterFiFi{ 0}%
    \or % x > 0
      \BIC@AfterFiFi{%
        \BIC@ModSwitch--#3#4!#2!%
      }%
    \else % x < 0
      \BIC@AfterFiFi{%
        \BIC@ModSwitch-#4!#2!%
      }%
    \fi
  \BIC@Fi
}
%    \end{macrocode}
%    \end{macro}
%    \begin{macro}{\BIC@ModSwitch}
%    Decision table for \cs{BIC@ModSwitch}.
%    \begin{quote}
%    \begin{tabular}{@{}|l|l|l|@{}}
%    \hline
%    $y=1$ & \multicolumn{1}{l}{} & $0$\\
%    \hline
%    $y=2$ & ifodd$(x)$ & sign $1$\\
%    \cline{2-3}
%          & else       & $0$\\
%    \hline
%    $y>2$ & $x<0$      &
%        $z\leftarrow x-(x/y)*y;\quad (z<0)\mathbin{?}z+y\mathbin{:}z$\\
%    \cline{2-3}
%          & $x>0$      & $x - (x/y) * y$\\
%    \hline
%    \end{tabular}
%    \end{quote}
%    |#1|: sign\\
%    |#2#3|: $x$\\
%    |#4#5|: $y$
%    \begin{macrocode}
\def\BIC@ModSwitch#1#2#3!#4#5!{%
  \ifcase\ifx\\#5\\%
           \ifx#410 % y = 1
           \else\ifx#421 % y = 2
           \else2 % y > 2
           \fi\fi
         \else2 % y > 2
         \fi
    \BIC@AfterFi{ 0}% y = 1
  \or % y = 2
    \ifcase\BIC@ModTwo#2#3! % even(x)
      \BIC@AfterFiFi{ 0}%
    \or % odd(x)
      \BIC@AfterFiFi{ #11}%
?   \else\BigIntCalcError:ThisCannotHappen%
    \fi
  \or % y > 2
    \ifx\\#1\\%
    \else
      \expandafter\BIC@Space\romannumeral0%
      \expandafter\BIC@ModMinus\romannumeral0%
    \fi
    \ifx#2-% x < 0
      \BIC@AfterFiFi{%
        \expandafter\expandafter\expandafter\BIC@ModX
        \bigintcalcSub{#2#3}{%
          \bigintcalcMul{#4#5}{\bigintcalcDiv{#2#3}{#4#5}}%
        }!#4#5!%
      }%
    \else % x > 0
      \BIC@AfterFiFi{%
        \expandafter\expandafter\expandafter\BIC@Space
        \bigintcalcSub{#2#3}{%
          \bigintcalcMul{#4#5}{\bigintcalcDiv{#2#3}{#4#5}}%
        }%
      }%
    \fi
? \else\BigIntCalcError:ThisCannotHappen%
  \BIC@Fi
}
%    \end{macrocode}
%    \end{macro}
%    \begin{macro}{\BIC@ModMinus}
%    \begin{macrocode}
\def\BIC@ModMinus#1{%
  \ifx#10%
    \BIC@AfterFi{ 0}%
  \else
    \BIC@AfterFi{ -#1}%
  \BIC@Fi
}
%    \end{macrocode}
%    \end{macro}
%    \begin{macro}{\BIC@ModX}
%    |#1#2|: $z$\\
%    |#3|: $x$
%    \begin{macrocode}
\def\BIC@ModX#1#2!#3!{%
  \ifx#1-% z < 0
    \BIC@AfterFi{%
      \expandafter\BIC@Space\romannumeral0%
      \BIC@SubXY#3!#2!!!%
    }%
  \else % z >= 0
    \BIC@AfterFi{ #1#2}%
  \BIC@Fi
}
%    \end{macrocode}
%    \end{macro}
%
%    \begin{macrocode}
\BIC@AtEnd%
%    \end{macrocode}
%
%    \begin{macrocode}
%</package>
%    \end{macrocode}
%
% \section{Test}
%
% \subsection{Catcode checks for loading}
%
%    \begin{macrocode}
%<*test1>
%    \end{macrocode}
%    \begin{macrocode}
\catcode`\{=1 %
\catcode`\}=2 %
\catcode`\#=6 %
\catcode`\@=11 %
\expandafter\ifx\csname count@\endcsname\relax
  \countdef\count@=255 %
\fi
\expandafter\ifx\csname @gobble\endcsname\relax
  \long\def\@gobble#1{}%
\fi
\expandafter\ifx\csname @firstofone\endcsname\relax
  \long\def\@firstofone#1{#1}%
\fi
\expandafter\ifx\csname loop\endcsname\relax
  \expandafter\@firstofone
\else
  \expandafter\@gobble
\fi
{%
  \def\loop#1\repeat{%
    \def\body{#1}%
    \iterate
  }%
  \def\iterate{%
    \body
      \let\next\iterate
    \else
      \let\next\relax
    \fi
    \next
  }%
  \let\repeat=\fi
}%
\def\RestoreCatcodes{}
\count@=0 %
\loop
  \edef\RestoreCatcodes{%
    \RestoreCatcodes
    \catcode\the\count@=\the\catcode\count@\relax
  }%
\ifnum\count@<255 %
  \advance\count@ 1 %
\repeat

\def\RangeCatcodeInvalid#1#2{%
  \count@=#1\relax
  \loop
    \catcode\count@=15 %
  \ifnum\count@<#2\relax
    \advance\count@ 1 %
  \repeat
}
\def\RangeCatcodeCheck#1#2#3{%
  \count@=#1\relax
  \loop
    \ifnum#3=\catcode\count@
    \else
      \errmessage{%
        Character \the\count@\space
        with wrong catcode \the\catcode\count@\space
        instead of \number#3%
      }%
    \fi
  \ifnum\count@<#2\relax
    \advance\count@ 1 %
  \repeat
}
\def\space{ }
\expandafter\ifx\csname LoadCommand\endcsname\relax
  \def\LoadCommand{\input bigintcalc.sty\relax}%
\fi
\def\Test{%
  \RangeCatcodeInvalid{0}{47}%
  \RangeCatcodeInvalid{58}{64}%
  \RangeCatcodeInvalid{91}{96}%
  \RangeCatcodeInvalid{123}{255}%
  \catcode`\@=12 %
  \catcode`\\=0 %
  \catcode`\%=14 %
  \LoadCommand
  \RangeCatcodeCheck{0}{36}{15}%
  \RangeCatcodeCheck{37}{37}{14}%
  \RangeCatcodeCheck{38}{47}{15}%
  \RangeCatcodeCheck{48}{57}{12}%
  \RangeCatcodeCheck{58}{63}{15}%
  \RangeCatcodeCheck{64}{64}{12}%
  \RangeCatcodeCheck{65}{90}{11}%
  \RangeCatcodeCheck{91}{91}{15}%
  \RangeCatcodeCheck{92}{92}{0}%
  \RangeCatcodeCheck{93}{96}{15}%
  \RangeCatcodeCheck{97}{122}{11}%
  \RangeCatcodeCheck{123}{255}{15}%
  \RestoreCatcodes
}
\Test
\csname @@end\endcsname
\end
%    \end{macrocode}
%    \begin{macrocode}
%</test1>
%    \end{macrocode}
%
% \subsection{Macro tests}
%
% \subsubsection{Preamble with test macro definitions}
%
%    \begin{macrocode}
%<*test2>
\NeedsTeXFormat{LaTeX2e}
\nofiles
\documentclass{article}
%<noetex>\let\SavedNumexpr\numexpr
%<noetex>\let\numexpr\UNDEFINED
\makeatletter
\chardef\BIC@TestMode=1 %
\makeatother
\usepackage{bigintcalc}[2016/05/16]
%<noetex>\let\numexpr\SavedNumexpr
\usepackage{qstest}
\IncludeTests{*}
\LogTests{log}{*}{*}
\newcommand*{\TestSpaceAtEnd}[1]{%
%<noetex>  \let\SavedNumexpr\numexpr
%<noetex>  \let\numexpr\UNDEFINED
  \edef\resultA{#1}%
  \edef\resultB{#1 }%
%<noetex>  \let\numexpr\SavedNumexpr
  \Expect*{\resultA\space}*{\resultB}%
}
\newcommand*{\TestResult}[2]{%
%<noetex>  \let\SavedNumexpr\numexpr
%<noetex>  \let\numexpr\UNDEFINED
  \edef\result{#1}%
%<noetex>  \let\numexpr\SavedNumexpr
  \Expect*{\result}{#2}%
}
\newcommand*{\TestResultTwoExpansions}[2]{%
%<*noetex>
  \begingroup
    \let\numexpr\UNDEFINED
    \expandafter\expandafter\expandafter
  \endgroup
%</noetex>
  \expandafter\expandafter\expandafter\Expect
  \expandafter\expandafter\expandafter{#1}{#2}%
}
\newcount\TestCount
%<etex>\newcommand*{\TestArg}[1]{\numexpr#1\relax}
%<noetex>\newcommand*{\TestArg}[1]{#1}
\newcommand*{\TestTeXDivide}[2]{%
  \TestCount=\TestArg{#1}\relax
  \divide\TestCount by \TestArg{#2}\relax
  \Expect*{\bigintcalcDiv{#1}{#2}}*{\the\TestCount}%
}
\newcommand*{\Test}[2]{%
  \TestResult{#1}{#2}%
  \TestResultTwoExpansions{#1}{#2}%
  \TestSpaceAtEnd{#1}%
}
\newcommand*{\TestExch}[2]{\Test{#2}{#1}}
\newcommand*{\TestInv}[2]{%
  \Test{\bigintcalcInv{#1}}{#2}%
}
\newcommand*{\TestAbs}[2]{%
  \Test{\bigintcalcAbs{#1}}{#2}%
}
\newcommand*{\TestSgn}[2]{%
  \Test{\bigintcalcSgn{#1}}{#2}%
}
\newcommand*{\TestMin}[3]{%
  \Test{\bigintcalcMin{#1}{#2}}{#3}%
}
\newcommand*{\TestMax}[3]{%
  \Test{\bigintcalcMax{#1}{#2}}{#3}%
}
\newcommand*{\TestCmp}[3]{%
  \Test{\bigintcalcCmp{#1}{#2}}{#3}%
}
\newcommand*{\TestOdd}[2]{%
  \Test{\bigintcalcOdd{#1}}{#2}%
  \edef\x{%
    \noexpand\Test{%
      \noexpand\BigIntCalcOdd
      \bigintcalcAbs{#1}!%
    }{#2}%
  }%
  \x
}
\newcommand*{\TestInc}[2]{%
  \Test{\bigintcalcInc{#1}}{#2}%
  \ifnum\bigintcalcSgn{#1}>-1 %
    \edef\x{%
      \noexpand\Test{%
        \noexpand\BigIntCalcInc\bigintcalcNum{#1}!%
      }{#2}%
    }%
    \x
  \fi
}
\newcommand*{\TestDec}[2]{%
  \Test{\bigintcalcDec{#1}}{#2}%
  \ifnum\bigintcalcSgn{#1}>0 %
    \edef\x{%
      \noexpand\Test{%
        \noexpand\BigIntCalcDec\bigintcalcNum{#1}!%
      }{#2}%
    }%
    \x
  \fi
}
\newcommand*{\TestAdd}[3]{%
  \Test{\bigintcalcAdd{#1}{#2}}{#3}%
  \ifnum\bigintcalcSgn{#1}>0 %
    \ifnum\bigintcalcSgn{#2}> 0 %
      \ifnum\bigintcalcCmp{#1}{#2}>0 %
        \edef\x{%
          \noexpand\Test{%
            \noexpand\BigIntCalcAdd
            \bigintcalcNum{#1}!\bigintcalcNum{#2}!%
          }{#3}%
        }%
        \x
      \else
        \edef\x{%
          \noexpand\Test{%
            \noexpand\BigIntCalcAdd
            \bigintcalcNum{#2}!\bigintcalcNum{#1}!%
          }{#3}%
        }%
        \x
      \fi
    \fi
  \fi
}
\newcommand*{\TestSub}[3]{%
  \Test{\bigintcalcSub{#1}{#2}}{#3}%
  \ifnum\bigintcalcSgn{#1}>0 %
    \ifnum\bigintcalcSgn{#2}> 0 %
      \ifnum\bigintcalcCmp{#1}{#2}>0 %
        \edef\x{%
          \noexpand\Test{%
            \noexpand\BigIntCalcSub
            \bigintcalcNum{#1}!\bigintcalcNum{#2}!%
          }{#3}%
        }%
        \x
      \fi
    \fi
  \fi
}
\newcommand*{\TestShl}[2]{%
  \Test{\bigintcalcShl{#1}}{#2}%
  \edef\x{%
    \noexpand\Test{%
      \noexpand\BigIntCalcShl\bigintcalcAbs{#1}!%
    }{\bigintcalcAbs{#2}}%
  }%
  \x
}
\newcommand*{\TestShr}[2]{%
  \Test{\bigintcalcShr{#1}}{#2}%
  \edef\x{%
    \noexpand\Test{%
      \noexpand\BigIntCalcShr\bigintcalcAbs{#1}!%
    }{\bigintcalcAbs{#2}}%
  }%
  \x
}
\newcommand*{\TestMul}[3]{%
  \Test{\bigintcalcMul{#1}{#2}}{#3}%
  \edef\x{%
    \noexpand\Test{%
      \noexpand\BigIntCalcMul
      \bigintcalcAbs{#1}!\bigintcalcAbs{#2}!%
    }{\bigintcalcAbs{#3}}%
  }%
  \x
}
\newcommand*{\TestSqr}[2]{%
  \Test{\bigintcalcSqr{#1}}{#2}%
}
\newcommand*{\TestFac}[2]{%
  \expandafter\TestExch\expandafter{%
    \the\numexpr#2%
  }{\bigintcalcFac{#1}}%
}
\newcommand*{\TestFacBig}[2]{%
  \Test{\bigintcalcFac{#1}}{#2}%
}
\newcommand*{\TestPow}[3]{%
  \Test{\bigintcalcPow{#1}{#2}}{#3}%
}
\newcommand*{\TestDiv}[3]{%
  \Test{\bigintcalcDiv{#1}{#2}}{#3}%
  \TestTeXDivide{#1}{#2}%
}
\newcommand*{\TestDivBig}[3]{%
  \Test{\bigintcalcDiv{#1}{#2}}{#3}%
  \edef\x{%
    \noexpand\Test{%
      \noexpand\BigIntCalcDiv\bigintcalcAbs{#1}!\bigintcalcAbs{#2}!%
    }{\bigintcalcAbs{#3}}%
  }%
}
\newcommand*{\TestMod}[3]{%
  \Test{\bigintcalcMod{#1}{#2}}{#3}%
  \ifcase\ifcase\bigintcalcSgn{#1} 0%
         \or
           \ifcase\bigintcalcSgn{#2} 1%
           \or 0%
           \else 1%
           \fi
         \else
           \ifcase\bigintcalcSgn{#2} 1%
           \or 1%
           \else 0%
           \fi
         \fi\relax
    \edef\x{%
      \noexpand\Test{%
        \noexpand\BigIntCalcMod
        \bigintcalcAbs{#1}!\bigintcalcAbs{#2}!%
      }{\bigintcalcAbs{#3}}%
    }%
    \x
  \fi
}
%    \end{macrocode}
%
% \subsubsection{Time}
%
%    \begin{macrocode}
\begingroup\expandafter\expandafter\expandafter\endgroup
\expandafter\ifx\csname pdfresettimer\endcsname\relax
\else
  \makeatletter
  \newcount\SummaryTime
  \newcount\TestTime
  \SummaryTime=\z@
  \newcommand*{\PrintTime}[2]{%
    \typeout{%
      [Time #1: \strip@pt\dimexpr\number#2sp\relax\space s]%
    }%
  }%
  \newcommand*{\StartTime}[1]{%
    \renewcommand*{\TimeDescription}{#1}%
    \pdfresettimer
  }%
  \newcommand*{\TimeDescription}{}%
  \newcommand*{\StopTime}{%
    \TestTime=\pdfelapsedtime
    \global\advance\SummaryTime\TestTime
    \PrintTime\TimeDescription\TestTime
  }%
  \let\saved@qstest\qstest
  \let\saved@endqstest\endqstest
  \def\qstest#1#2{%
    \saved@qstest{#1}{#2}%
    \StartTime{#1}%
  }%
  \def\endqstest{%
    \StopTime
    \saved@endqstest
  }%
  \AtEndDocument{%
    \PrintTime{summary}\SummaryTime
  }%
  \makeatother
\fi
%    \end{macrocode}
%
% \subsubsection{Test sets}
%
%    \begin{macrocode}
\makeatletter

\begin{qstest}{inv}{inv}%
  \TestInv{0}{0}%
  \TestInv{1}{-1}%
  \TestInv{-1}{1}%
  \TestInv{10}{-10}%
  \TestInv{-10}{10}%
  \TestInv{2147483647}{-2147483647}%
  \TestInv{-2147483647}{2147483647}%
  \TestInv{12345678901234567890}{-12345678901234567890}%
  \TestInv{-12345678901234567890}{12345678901234567890}%
  \TestInv{ 0 }{0}%
  \TestInv{ 1 }{-1}%
  \TestInv{--1}{-1}%
  \TestInv{\number\z@}{0}%
  \TestInv{\ifx\relax\relax1\fi}{-1}%
  \TestInv{\ifx\relax\relax-\fi\ifx234\else1\fi}{1}%
\end{qstest}

\begin{qstest}{abs}{abs}%
  \TestAbs{0}{0}%
  \TestAbs{1}{1}%
  \TestAbs{-1}{1}%
  \TestAbs{10}{10}%
  \TestAbs{-10}{10}%
  \TestAbs{2147483647}{2147483647}%
  \TestAbs{-2147483647}{2147483647}%
  \TestAbs{12345678901234567890}{12345678901234567890}%
  \TestAbs{-12345678901234567890}{12345678901234567890}%
  \TestAbs{ 0 }{0}%
  \TestAbs{ 1 }{1}%
  \TestAbs{--1}{1}%
  \TestAbs{-+-+1}{1}%
  \TestAbs{00000000000}{0}%
  \TestAbs{00000001000}{1000}%
  \TestAbs{\ifx\relax\relax 0\else 1\fi}{0}%
\end{qstest}

\begin{qstest}{sign}{sign}%
  \TestSgn{0}{0}%
  \TestSgn{1}{1}%
  \TestSgn{-1}{-1}%
  \TestSgn{10}{1}%
  \TestSgn{-10}{-1}%
  \TestSgn{2147483647}{1}%
  \TestSgn{-2147483647}{-1}%
  \TestSgn{12345678901234567890}{1}%
  \TestSgn{-12345678901234567890}{-1}%
  \TestSgn{ 0 }{0}%
  \TestSgn{ 2 }{1}%
  \TestSgn{ -2 }{-1}%
  \TestSgn{--2}{1}%
  \TestSgn{\number\z@}{0}%
  \TestSgn{\number\@ne}{1}%
  \TestSgn{\number\m@ne}{-1}%
  \TestSgn{%
    -+-+\number\z@\number\z@
    \iftrue1\fi\iftrue2\fi\iftrue3\fi
  }{1}%
\end{qstest}

\begin{qstest}{min}{min}%
  \TestMin{0}{1}{0}%
  \TestMin{1}{0}{0}%
  \TestMin{-10}{-20}{-20}%
  \TestMin{ 1 }{ 2 }{1}%
  \TestMin{ 2 }{ 1 }{1}%
  \TestMin{1}{1}{1}%
  \TestMin{\number\z@}{\number\@ne}{0}%
  \TestMin{\number\@ne}{\number\m@ne}{-1}%
\end{qstest}

\begin{qstest}{max}{max}%
  \TestMax{0}{1}{1}%
  \TestMax{1}{0}{1}%
  \TestMax{-10}{-20}{-10}%
  \TestMax{ 1 }{ 2 }{2}%
  \TestMax{ 2 }{ 1 }{2}%
  \TestMax{1}{1}{1}%
  \TestMax{\number\z@}{\number\@ne}{1}%
  \TestMax{\number\@ne}{\number\m@ne}{1}%
\end{qstest}

\begin{qstest}{cmp}{cmp}%
  \TestCmp{0}{0}{0}%
  \TestCmp{-21}{17}{-1}%
  \TestCmp{3}{4}{-1}%
  \TestCmp{-10}{-10}{0}%
  \TestCmp{-10}{-11}{1}%
  \TestCmp{100}{5}{1}%
  \TestCmp{9}{10}{-1}%
  \TestCmp{10}{9}{1}%
  \TestCmp{ 3 }{ 3 }{0}%
  \TestCmp{-9}{-10}{1}%
  \TestCmp{-10}{-9}{-1}%
  \TestCmp{-3}{-3}{0}%
  \TestCmp{0}{-2}{1}%
  \TestCmp{0}{2}{-1}%
  \TestCmp{2}{0}{1}%
  \TestCmp{-2}{0}{-1}%
  \TestCmp{12}{11}{1}%
  \TestCmp{11}{12}{-1}%
  \TestCmp{2147483647}{-2147483647}{1}%
  \TestCmp{-2147483647}{2147483647}{-1}%
  \TestCmp{2147483647}{2147483647}{0}%
  \TestCmp{\number\z@}{\number\@ne}{-1}%
  \TestCmp{\number\@ne}{\number\m@ne}{1}%
  \TestCmp{ 4 }{ 5 }{-1}%
  \TestCmp{ -3 }{ -7 }{1}%
\end{qstest}

\begin{qstest}{odd}{odd}
\tracingmacros=1
  \TestOdd{0}{0}%
  \TestOdd{1}{1}%
  \TestOdd{2}{0}%
  \TestOdd{3}{1}%
  \TestOdd{14}{0}%
  \TestOdd{15}{1}%
  \TestOdd{12345678901234567896}{0}%
  \TestOdd{12345678901234567897}{1}%
\end{qstest}

\begin{qstest}{inc}{inc}%
  \TestInc{0}{1}%
  \TestInc{1}{2}%
  \TestInc{-1}{0}%
  \TestInc{10}{11}%
  \TestInc{-10}{-9}%
  \TestInc{ 3 }{4}%
  \TestInc{999}{1000}%
  \TestInc{-1000}{-999}%
  \TestInc{129}{130}%
  \TestInc{2147483646}{2147483647}%
  \TestInc{-2147483647}{-2147483646}%
  \TestInc{12345678901234567890}{12345678901234567891}%
  \TestInc{99999999999999999999}{100000000000000000000}%
  \TestInc{-12345678901234567891}{-12345678901234567890}%
  \TestInc{-100000000000000000000}{-99999999999999999999}%
\end{qstest}

\begin{qstest}{dec}{dec}%
  \TestDec{0}{-1}%
  \TestDec{1}{0}%
  \TestDec{-1}{-2}%
  \TestDec{10}{9}%
  \TestDec{-10}{-11}%
  \TestDec{1000}{999}%
  \TestDec{-999}{-1000}%
  \TestDec{130}{129}%
  \TestDec{2147483647}{2147483646}%
  \TestDec{-2147483646}{-2147483647}%
  \TestDec{12345678901234567891}{12345678901234567890}%
  \TestDec{100000000000000000000}{99999999999999999999}%
  \TestDec{-12345678901234567890}{-12345678901234567891}%
  \TestDec{-99999999999999999999}{-100000000000000000000}%
\end{qstest}

\begin{qstest}{add}{add}%
  \TestAdd{0}{0}{0}%
  \TestAdd{1}{0}{1}%
  \TestAdd{0}{1}{1}%
  \TestAdd{1}{2}{3}%
  \TestAdd{-1}{-1}{-2}%
  \TestAdd{2147483646}{1}{2147483647}%
  \TestAdd{-2147483647}{2147483647}{0}%
  \TestAdd{20}{-5}{15}%
  \TestAdd{-4}{-1}{-5}%
  \TestAdd{-1}{-4}{-5}%
  \TestAdd{-4}{1}{-3}%
  \TestAdd{-1}{4}{3}%
  \TestAdd{4}{-1}{3}%
  \TestAdd{1}{-4}{-3}%
  \TestAdd{-4}{-1}{-5}%
  \TestAdd{-1}{-4}{-5}%
  \TestAdd{ -4 }{ -1 }{-5}%
  \TestAdd{ -1 }{ -4 }{-5}%
  \TestAdd{ -4 }{ 1 }{-3}%
  \TestAdd{ -1 }{ 4 }{3}%
  \TestAdd{ 4 }{ -1 }{3}%
  \TestAdd{ 1 }{ -4 }{-3}%
  \TestAdd{ -4 }{ -1 }{-5}%
  \TestAdd{ -1 }{ -4 }{-5}%
  \TestAdd{876543210}{111111111}{987654321}%
  \TestAdd{999999999}{2}{1000000001}%
\end{qstest}

\begin{qstest}{sub}{sub}
  \TestSub{0}{0}{0}%
  \TestSub{1}{0}{1}%
  \TestSub{1}{2}{-1}%
  \TestSub{-1}{-1}{0}%
  \TestSub{2147483646}{-1}{2147483647}%
  \TestSub{-2147483647}{-2147483647}{0}%
  \TestSub{-4}{-1}{-3}%
  \TestSub{-1}{-4}{3}%
  \TestSub{-4}{1}{-5}%
  \TestSub{-1}{4}{-5}%
  \TestSub{4}{-1}{5}%
  \TestSub{1}{-4}{5}%
  \TestSub{-4}{-1}{-3}%
  \TestSub{-1}{-4}{3}%
  \TestSub{ -4 }{ -1 }{-3}%
  \TestSub{ -1 }{ -4 }{3}%
  \TestSub{ -4 }{ 1 }{-5}%
  \TestSub{ -1 }{ 4 }{-5}%
  \TestSub{ 4 }{ -1 }{5}%
  \TestSub{ 1 }{ -4 }{5}%
  \TestSub{ -4 }{ -1 }{-3}%
  \TestSub{ -1 }{ -4 }{3}%
  \TestSub{1000000000}{2}{999999998}%
  \TestSub{987654321}{111111111}{876543210}%
\end{qstest}

\begin{qstest}{shl}{shl}
  \TestShl{0}{0}%
  \TestShl{1}{2}%
  \TestShl{2}{4}%
  \TestShl{5621}{11242}%
  \TestShl{1073741823}{2147483646}%
\end{qstest}

\begin{qstest}{shr}{shr}
  \TestShr{0}{0}%
  \TestShr{1}{0}%
  \TestShr{2}{1}%
  \TestShr{3}{1}%
  \TestShr{4}{2}%
  \TestShr{5}{2}%
  \TestShr{6}{3}%
  \TestShr{7}{3}%
  \TestShr{8}{4}%
  \TestShr{9}{4}%
  \TestShr{10}{5}%
  \TestShr{11}{5}%
  \TestShr{12}{6}%
  \TestShr{13}{6}%
  \TestShr{14}{7}%
  \TestShr{15}{7}%
  \TestShr{16}{8}%
  \TestShr{17}{8}%
  \TestShr{18}{9}%
  \TestShr{19}{9}%
  \TestShr{20}{10}%
  \TestShr{21}{10}%
  \TestShr{22}{11}%
  \TestShr{11241}{5620}%
  \TestShr{73054202}{36527101}%
  \TestShr{2147483646}{1073741823}%
\end{qstest}

\begin{qstest}{mul}{mul}
  \TestMul{0}{0}{0}%
  \TestMul{1}{0}{0}%
  \TestMul{0}{1}{0}%
  \TestMul{1}{1}{1}%
  \TestMul{3}{1}{3}%
  \TestMul{1}{-3}{-3}%
  \TestMul{-4}{-5}{20}%
  \TestMul{3}{7}{21}%
  \TestMul{7}{3}{21}%
  \TestMul{3}{-7}{-21}%
  \TestMul{7}{-3}{-21}%
  \TestMul{-3}{7}{-21}%
  \TestMul{-7}{3}{-21}%
  \TestMul{-3}{-7}{21}%
  \TestMul{-7}{-3}{21}%
  \TestMul{12}{11}{132}%
  \TestMul{999}{333}{332667}%
  \TestMul{1000}{4321}{4321000}%
  \TestMul{12345}{173955}{2147474475}%
  \TestMul{1073741823}{2}{2147483646}%
  \TestMul{2}{1073741823}{2147483646}%
  \TestMul{-1073741823}{2}{-2147483646}%
  \TestMul{2}{-1073741823}{-2147483646}%
  \TestMul{6706022400}{13}{87178291200}%
\end{qstest}

\begin{qstest}{sqr}{sqr}
  \TestSqr{0}{0}%
  \TestSqr{1}{1}%
  \TestSqr{2}{4}%
  \TestSqr{3}{9}%
  \TestSqr{4}{16}%
  \TestSqr{9}{81}%
  \TestSqr{10}{100}%
  \TestSqr{46340}{2147395600}%
  \TestSqr{-1}{1}%
  \TestSqr{-2}{4}%
  \TestSqr{-46340}{2147395600}%
\end{qstest}

\begin{qstest}{fac}{fac}
  \TestFac{0}{1}%
  \TestFac{1}{1}%
  \TestFac{2}{2}%
  \TestFac{3}{2*3}%
  \TestFac{4}{2*3*4}%
  \TestFac{5}{2*3*4*5}%
  \TestFac{6}{2*3*4*5*6}%
  \TestFac{7}{2*3*4*5*6*7}%
  \TestFac{8}{2*3*4*5*6*7*8}%
  \TestFac{9}{2*3*4*5*6*7*8*9}%
  \TestFac{10}{2*3*4*5*6*7*8*9*10}%
  \TestFac{11}{2*3*4*5*6*7*8*9*10*11}%
  \TestFac{12}{2*3*4*5*6*7*8*9*10*11*12}%
  \TestFacBig{13}{6227020800}%
  \TestFacBig{14}{87178291200}%
  \TestFacBig{15}{1307674368000}%
  \TestFacBig{16}{20922789888000}%
  \TestFacBig{17}{355687428096000}%
  \TestFacBig{18}{6402373705728000}%
  \TestFacBig{19}{121645100408832000}%
  \TestFacBig{20}{2432902008176640000}%
  \TestFacBig{21}{51090942171709440000}%
  \TestFacBig{22}{1124000727777607680000}%
\end{qstest}

\begin{qstest}{pow}{pow}
  \TestPow{-2}{0}{1}%
  \TestPow{-1}{0}{1}%
  \TestPow{0}{0}{1}%
  \TestPow{1}{0}{1}%
  \TestPow{2}{0}{1}%
  \TestPow{3}{0}{1}%
  \TestPow{-2}{1}{-2}%
  \TestPow{-1}{1}{-1}%
  \TestPow{1}{1}{1}%
  \TestPow{2}{1}{2}%
  \TestPow{3}{1}{3}%
  \TestPow{-2}{2}{4}%
  \TestPow{-1}{2}{1}%
  \TestPow{0}{2}{0}%
  \TestPow{1}{2}{1}%
  \TestPow{2}{2}{4}%
  \TestPow{3}{2}{9}%
  \TestPow{0}{1}{0}%
  \TestPow{1}{-2}{1}%
  \TestPow{1}{-1}{1}%
  \TestPow{-1}{-2}{1}%
  \TestPow{-1}{-1}{-1}%
  \TestPow{-1}{3}{-1}%
  \TestPow{-1}{4}{1}%
  \TestPow{-2}{-1}{0}%
  \TestPow{-2}{-2}{0}%
  \TestPow{2}{3}{8}%
  \TestPow{2}{4}{16}%
  \TestPow{2}{5}{32}%
  \TestPow{2}{6}{64}%
  \TestPow{2}{7}{128}%
  \TestPow{2}{8}{256}%
  \TestPow{2}{9}{512}%
  \TestPow{2}{10}{1024}%
  \TestPow{-2}{3}{-8}%
  \TestPow{-2}{4}{16}%
  \TestPow{-2}{5}{-32}%
  \TestPow{-2}{6}{64}%
  \TestPow{-2}{7}{-128}%
  \TestPow{-2}{8}{256}%
  \TestPow{-2}{9}{-512}%
  \TestPow{-2}{10}{1024}%
  \TestPow{3}{3}{27}%
  \TestPow{3}{4}{81}%
  \TestPow{3}{5}{243}%
  \TestPow{-3}{3}{-27}%
  \TestPow{-3}{4}{81}%
  \TestPow{-3}{5}{-243}%
  \TestPow{2}{30}{1073741824}%
  \TestPow{-3}{19}{-1162261467}%
  \TestPow{5}{13}{1220703125}%
  \TestPow{-7}{11}{-1977326743}%
\end{qstest}

\begin{qstest}{div}{div}
  \TestDiv{1}{1}{1}%
  \TestDiv{2}{1}{2}%
  \TestDiv{-2}{1}{-2}%
  \TestDiv{2}{-1}{-2}%
  \TestDiv{-2}{-1}{2}%
  \TestDiv{15}{2}{7}%
  \TestDiv{-16}{2}{-8}%
  \TestDiv{1}{2}{0}%
  \TestDiv{1}{3}{0}%
  \TestDiv{2}{3}{0}%
  \TestDiv{-2}{3}{0}%
  \TestDiv{2}{-3}{0}%
  \TestDiv{-2}{-3}{0}%
  \TestDiv{13}{3}{4}%
  \TestDiv{-13}{-3}{4}%
  \TestDiv{-13}{3}{-4}%
  \TestDiv{-6}{5}{-1}%
  \TestDiv{-5}{5}{-1}%
  \TestDiv{-4}{5}{0}%
  \TestDiv{-3}{5}{0}%
  \TestDiv{-2}{5}{0}%
  \TestDiv{-1}{5}{0}%
  \TestDiv{0}{5}{0}%
  \TestDiv{1}{5}{0}%
  \TestDiv{2}{5}{0}%
  \TestDiv{3}{5}{0}%
  \TestDiv{4}{5}{0}%
  \TestDiv{5}{5}{1}%
  \TestDiv{6}{5}{1}%
  \TestDiv{-5}{4}{-1}%
  \TestDiv{-4}{4}{-1}%
  \TestDiv{-3}{4}{0}%
  \TestDiv{-2}{4}{0}%
  \TestDiv{-1}{4}{0}%
  \TestDiv{0}{4}{0}%
  \TestDiv{1}{4}{0}%
  \TestDiv{2}{4}{0}%
  \TestDiv{3}{4}{0}%
  \TestDiv{4}{4}{1}%
  \TestDiv{5}{4}{1}%
  \TestDiv{12345}{678}{18}%
  \TestDiv{32372}{5952}{5}%
  \TestDiv{284271294}{18162}{15651}%
  \TestDiv{217652429}{12561}{17327}%
  \TestDiv{462028434}{5439}{84947}%
  \TestDiv{2147483647}{1000}{2147483}%
  \TestDiv{2147483647}{-1000}{-2147483}%
  \TestDiv{-2147483647}{1000}{-2147483}%
  \TestDiv{-2147483647}{-1000}{2147483}%
  \TestDiv{0}{3}{0}%
  \TestDiv{1}{3}{0}%
  \TestDiv{2}{3}{0}%
  \TestDiv{3}{3}{1}%
  \TestDiv{4}{3}{1}%
  \TestDiv{5}{3}{1}%
  \TestDiv{6}{3}{2}%
  \TestDiv{7}{3}{2}%
  \TestDiv{8}{3}{2}%
  \TestDiv{9}{3}{3}%
  \TestDiv{10}{3}{3}%
  \TestDiv{11}{3}{3}%
  \TestDiv{12}{3}{4}%
  \TestDiv{13}{3}{4}%
  \TestDiv{14}{3}{4}%
  \TestDiv{15}{3}{5}%
  \TestDiv{16}{3}{5}%
  \TestDiv{17}{3}{5}%
  \TestDiv{18}{3}{6}%
  \TestDiv{19}{3}{6}%
  \TestDiv{20}{3}{6}%
  \TestDiv{21}{3}{7}%
  \TestDiv{22}{3}{7}%
  \TestDiv{23}{3}{7}%
  \TestDiv{24}{3}{8}%
  \TestDiv{25}{3}{8}%
  \TestDiv{26}{3}{8}%
  \TestDiv{27}{3}{9}%
  \TestDiv{28}{3}{9}%
  \TestDiv{29}{3}{9}%
  \TestDiv{30}{3}{10}%
  \TestDiv{31}{3}{10}%
  \TestDivBig{17363436332507}{24702}{702916214}%
\end{qstest}

\begin{qstest}{mod}{mod}
  \TestMod{-6}{5}{4}%
  \TestMod{-5}{5}{0}%
  \TestMod{-4}{5}{1}%
  \TestMod{-3}{5}{2}%
  \TestMod{-2}{5}{3}%
  \TestMod{-1}{5}{4}%
  \TestMod{0}{5}{0}%
  \TestMod{1}{5}{1}%
  \TestMod{2}{5}{2}%
  \TestMod{3}{5}{3}%
  \TestMod{4}{5}{4}%
  \TestMod{5}{5}{0}%
  \TestMod{6}{5}{1}%
  \TestMod{-5}{4}{3}%
  \TestMod{-4}{4}{0}%
  \TestMod{-3}{4}{1}%
  \TestMod{-2}{4}{2}%
  \TestMod{-1}{4}{3}%
  \TestMod{0}{4}{0}%
  \TestMod{1}{4}{1}%
  \TestMod{2}{4}{2}%
  \TestMod{3}{4}{3}%
  \TestMod{4}{4}{0}%
  \TestMod{5}{4}{1}%
  \TestMod{-6}{-5}{-1}%
  \TestMod{-5}{-5}{0}%
  \TestMod{-4}{-5}{-4}%
  \TestMod{-3}{-5}{-3}%
  \TestMod{-2}{-5}{-2}%
  \TestMod{-1}{-5}{-1}%
  \TestMod{0}{-5}{0}%
  \TestMod{1}{-5}{-4}%
  \TestMod{2}{-5}{-3}%
  \TestMod{3}{-5}{-2}%
  \TestMod{4}{-5}{-1}%
  \TestMod{5}{-5}{0}%
  \TestMod{6}{-5}{-4}%
  \TestMod{-5}{-4}{-1}%
  \TestMod{-4}{-4}{0}%
  \TestMod{-3}{-4}{-3}%
  \TestMod{-2}{-4}{-2}%
  \TestMod{-1}{-4}{-1}%
  \TestMod{0}{-4}{0}%
  \TestMod{1}{-4}{-3}%
  \TestMod{2}{-4}{-2}%
  \TestMod{3}{-4}{-1}%
  \TestMod{4}{-4}{0}%
  \TestMod{5}{-4}{-3}%
  \TestMod{2147483647}{1000}{647}%
  \TestMod{2147483647}{-1000}{-353}%
  \TestMod{-2147483647}{1000}{353}%
  \TestMod{-2147483647}{-1000}{-647}%
  \TestMod{ 0 }{ 4 }{0}%
  \TestMod{ 1 }{ 4 }{1}%
  \TestMod{ -1 }{ 4 }{3}%
  \TestMod{ 0 }{ -4 }{0}%
  \TestMod{ 1 }{ -4 }{-3}%
  \TestMod{ -1 }{ -4 }{-1}%
  \TestMod{18362}{25}{12}%
\end{qstest}

\newcommand*{\TestError}[2]{%
  \begingroup
    \expandafter\def\csname BigIntCalcError:#1\endcsname{}%
    \Expect*{#2}{0}%
    \expandafter\def\csname BigIntCalcError:#1\endcsname{ERROR}%
    \Expect*{#2}{0ERROR}%
  \endgroup
}
\begin{qstest}{error}{error}
  \TestError{FacNegative}{\bigintcalcFac{-1}}%
  \TestError{FacNegative}{\bigintcalcFac{-2147483647}}%
  \TestError{DivisionByZero}{\bigintcalcPow{0}{-1}}%
  \TestError{DivisionByZero}{\bigintcalcDiv{1}{0}}%
  \TestError{DivisionByZero}{\bigintcalcMod{1}{0}}%
\end{qstest}

\begin{document}
\end{document}
%</test2>
%    \end{macrocode}
%
% \section{Installation}
%
% \subsection{Download}
%
% \paragraph{Package.} This package is available on
% CTAN\footnote{\url{http://ctan.org/pkg/bigintcalc}}:
% \begin{description}
% \item[\CTAN{macros/latex/contrib/oberdiek/bigintcalc.dtx}] The source file.
% \item[\CTAN{macros/latex/contrib/oberdiek/bigintcalc.pdf}] Documentation.
% \end{description}
%
%
% \paragraph{Bundle.} All the packages of the bundle `oberdiek'
% are also available in a TDS compliant ZIP archive. There
% the packages are already unpacked and the documentation files
% are generated. The files and directories obey the TDS standard.
% \begin{description}
% \item[\CTAN{install/macros/latex/contrib/oberdiek.tds.zip}]
% \end{description}
% \emph{TDS} refers to the standard ``A Directory Structure
% for \TeX\ Files'' (\CTAN{tds/tds.pdf}). Directories
% with \xfile{texmf} in their name are usually organized this way.
%
% \subsection{Bundle installation}
%
% \paragraph{Unpacking.} Unpack the \xfile{oberdiek.tds.zip} in the
% TDS tree (also known as \xfile{texmf} tree) of your choice.
% Example (linux):
% \begin{quote}
%   |unzip oberdiek.tds.zip -d ~/texmf|
% \end{quote}
%
% \paragraph{Script installation.}
% Check the directory \xfile{TDS:scripts/oberdiek/} for
% scripts that need further installation steps.
% Package \xpackage{attachfile2} comes with the Perl script
% \xfile{pdfatfi.pl} that should be installed in such a way
% that it can be called as \texttt{pdfatfi}.
% Example (linux):
% \begin{quote}
%   |chmod +x scripts/oberdiek/pdfatfi.pl|\\
%   |cp scripts/oberdiek/pdfatfi.pl /usr/local/bin/|
% \end{quote}
%
% \subsection{Package installation}
%
% \paragraph{Unpacking.} The \xfile{.dtx} file is a self-extracting
% \docstrip\ archive. The files are extracted by running the
% \xfile{.dtx} through \plainTeX:
% \begin{quote}
%   \verb|tex bigintcalc.dtx|
% \end{quote}
%
% \paragraph{TDS.} Now the different files must be moved into
% the different directories in your installation TDS tree
% (also known as \xfile{texmf} tree):
% \begin{quote}
% \def\t{^^A
% \begin{tabular}{@{}>{\ttfamily}l@{ $\rightarrow$ }>{\ttfamily}l@{}}
%   bigintcalc.sty & tex/generic/oberdiek/bigintcalc.sty\\
%   bigintcalc.pdf & doc/latex/oberdiek/bigintcalc.pdf\\
%   test/bigintcalc-test1.tex & doc/latex/oberdiek/test/bigintcalc-test1.tex\\
%   test/bigintcalc-test2.tex & doc/latex/oberdiek/test/bigintcalc-test2.tex\\
%   test/bigintcalc-test3.tex & doc/latex/oberdiek/test/bigintcalc-test3.tex\\
%   bigintcalc.dtx & source/latex/oberdiek/bigintcalc.dtx\\
% \end{tabular}^^A
% }^^A
% \sbox0{\t}^^A
% \ifdim\wd0>\linewidth
%   \begingroup
%     \advance\linewidth by\leftmargin
%     \advance\linewidth by\rightmargin
%   \edef\x{\endgroup
%     \def\noexpand\lw{\the\linewidth}^^A
%   }\x
%   \def\lwbox{^^A
%     \leavevmode
%     \hbox to \linewidth{^^A
%       \kern-\leftmargin\relax
%       \hss
%       \usebox0
%       \hss
%       \kern-\rightmargin\relax
%     }^^A
%   }^^A
%   \ifdim\wd0>\lw
%     \sbox0{\small\t}^^A
%     \ifdim\wd0>\linewidth
%       \ifdim\wd0>\lw
%         \sbox0{\footnotesize\t}^^A
%         \ifdim\wd0>\linewidth
%           \ifdim\wd0>\lw
%             \sbox0{\scriptsize\t}^^A
%             \ifdim\wd0>\linewidth
%               \ifdim\wd0>\lw
%                 \sbox0{\tiny\t}^^A
%                 \ifdim\wd0>\linewidth
%                   \lwbox
%                 \else
%                   \usebox0
%                 \fi
%               \else
%                 \lwbox
%               \fi
%             \else
%               \usebox0
%             \fi
%           \else
%             \lwbox
%           \fi
%         \else
%           \usebox0
%         \fi
%       \else
%         \lwbox
%       \fi
%     \else
%       \usebox0
%     \fi
%   \else
%     \lwbox
%   \fi
% \else
%   \usebox0
% \fi
% \end{quote}
% If you have a \xfile{docstrip.cfg} that configures and enables \docstrip's
% TDS installing feature, then some files can already be in the right
% place, see the documentation of \docstrip.
%
% \subsection{Refresh file name databases}
%
% If your \TeX~distribution
% (\teTeX, \mikTeX, \dots) relies on file name databases, you must refresh
% these. For example, \teTeX\ users run \verb|texhash| or
% \verb|mktexlsr|.
%
% \subsection{Some details for the interested}
%
% \paragraph{Attached source.}
%
% The PDF documentation on CTAN also includes the
% \xfile{.dtx} source file. It can be extracted by
% AcrobatReader 6 or higher. Another option is \textsf{pdftk},
% e.g. unpack the file into the current directory:
% \begin{quote}
%   \verb|pdftk bigintcalc.pdf unpack_files output .|
% \end{quote}
%
% \paragraph{Unpacking with \LaTeX.}
% The \xfile{.dtx} chooses its action depending on the format:
% \begin{description}
% \item[\plainTeX:] Run \docstrip\ and extract the files.
% \item[\LaTeX:] Generate the documentation.
% \end{description}
% If you insist on using \LaTeX\ for \docstrip\ (really,
% \docstrip\ does not need \LaTeX), then inform the autodetect routine
% about your intention:
% \begin{quote}
%   \verb|latex \let\install=y\input{bigintcalc.dtx}|
% \end{quote}
% Do not forget to quote the argument according to the demands
% of your shell.
%
% \paragraph{Generating the documentation.}
% You can use both the \xfile{.dtx} or the \xfile{.drv} to generate
% the documentation. The process can be configured by the
% configuration file \xfile{ltxdoc.cfg}. For instance, put this
% line into this file, if you want to have A4 as paper format:
% \begin{quote}
%   \verb|\PassOptionsToClass{a4paper}{article}|
% \end{quote}
% An example follows how to generate the
% documentation with pdf\LaTeX:
% \begin{quote}
%\begin{verbatim}
%pdflatex bigintcalc.dtx
%makeindex -s gind.ist bigintcalc.idx
%pdflatex bigintcalc.dtx
%makeindex -s gind.ist bigintcalc.idx
%pdflatex bigintcalc.dtx
%\end{verbatim}
% \end{quote}
%
% \section{Catalogue}
%
% The following XML file can be used as source for the
% \href{http://mirror.ctan.org/help/Catalogue/catalogue.html}{\TeX\ Catalogue}.
% The elements \texttt{caption} and \texttt{description} are imported
% from the original XML file from the Catalogue.
% The name of the XML file in the Catalogue is \xfile{bigintcalc.xml}.
%    \begin{macrocode}
%<*catalogue>
<?xml version='1.0' encoding='us-ascii'?>
<!DOCTYPE entry SYSTEM 'catalogue.dtd'>
<entry datestamp='$Date$' modifier='$Author$' id='bigintcalc'>
  <name>bigintcalc</name>
  <caption>Integer calculations on very large numbers.</caption>
  <authorref id='auth:oberdiek'/>
  <copyright owner='Heiko Oberdiek' year='2007,2011,2012'/>
  <license type='lppl1.3'/>
  <version number='1.4'/>
  <description>
    This package provides expandable arithmetic operations
    with big integers that can exceed TeX's number limits.
    <p/>
    The package is part of the <xref refid='oberdiek'>oberdiek</xref> bundle.
  </description>
  <documentation details='Package documentation'
      href='ctan:/macros/latex/contrib/oberdiek/bigintcalc.pdf'/>
  <ctan file='true' path='/macros/latex/contrib/oberdiek/bigintcalc.dtx'/>
  <miktex location='oberdiek'/>
  <texlive location='oberdiek'/>
  <install path='/macros/latex/contrib/oberdiek/oberdiek.tds.zip'/>
</entry>
%</catalogue>
%    \end{macrocode}
%
% \begin{History}
%   \begin{Version}{2007/09/27 v1.0}
%   \item
%     First version.
%   \end{Version}
%   \begin{Version}{2007/11/11 v1.1}
%   \item
%     Use of package \xpackage{pdftexcmds} for \LuaTeX\ support.
%   \end{Version}
%   \begin{Version}{2011/01/30 v1.2}
%   \item
%     Already loaded package files are not input in \hologo{plainTeX}.
%   \end{Version}
%   \begin{Version}{2012/04/08 v1.3}
%   \item
%     Fix: \xpackage{pdftexcmds} wasn't loaded in case of \hologo{LaTeX}.
%   \end{Version}
%   \begin{Version}{2016/05/16 v1.4}
%   \item
%     Documentation updates.
%   \end{Version}
% \end{History}
%
% \PrintIndex
%
% \Finale
\endinput

%        (quote the arguments according to the demands of your shell)
%
% Documentation:
%    (a) If bigintcalc.drv is present:
%           latex bigintcalc.drv
%    (b) Without bigintcalc.drv:
%           latex bigintcalc.dtx; ...
%    The class ltxdoc loads the configuration file ltxdoc.cfg
%    if available. Here you can specify further options, e.g.
%    use A4 as paper format:
%       \PassOptionsToClass{a4paper}{article}
%
%    Programm calls to get the documentation (example):
%       pdflatex bigintcalc.dtx
%       makeindex -s gind.ist bigintcalc.idx
%       pdflatex bigintcalc.dtx
%       makeindex -s gind.ist bigintcalc.idx
%       pdflatex bigintcalc.dtx
%
% Installation:
%    TDS:tex/generic/oberdiek/bigintcalc.sty
%    TDS:doc/latex/oberdiek/bigintcalc.pdf
%    TDS:doc/latex/oberdiek/test/bigintcalc-test1.tex
%    TDS:doc/latex/oberdiek/test/bigintcalc-test2.tex
%    TDS:doc/latex/oberdiek/test/bigintcalc-test3.tex
%    TDS:source/latex/oberdiek/bigintcalc.dtx
%
%<*ignore>
\begingroup
  \catcode123=1 %
  \catcode125=2 %
  \def\x{LaTeX2e}%
\expandafter\endgroup
\ifcase 0\ifx\install y1\fi\expandafter
         \ifx\csname processbatchFile\endcsname\relax\else1\fi
         \ifx\fmtname\x\else 1\fi\relax
\else\csname fi\endcsname
%</ignore>
%<*install>
\input docstrip.tex
\Msg{************************************************************************}
\Msg{* Installation}
\Msg{* Package: bigintcalc 2016/05/16 v1.4 Expandable calculations on big integers (HO)}
\Msg{************************************************************************}

\keepsilent
\askforoverwritefalse

\let\MetaPrefix\relax
\preamble

This is a generated file.

Project: bigintcalc
Version: 2016/05/16 v1.4

Copyright (C) 2007, 2011, 2012 by
   Heiko Oberdiek <heiko.oberdiek at googlemail.com>

This work may be distributed and/or modified under the
conditions of the LaTeX Project Public License, either
version 1.3c of this license or (at your option) any later
version. This version of this license is in
   http://www.latex-project.org/lppl/lppl-1-3c.txt
and the latest version of this license is in
   http://www.latex-project.org/lppl.txt
and version 1.3 or later is part of all distributions of
LaTeX version 2005/12/01 or later.

This work has the LPPL maintenance status "maintained".

This Current Maintainer of this work is Heiko Oberdiek.

The Base Interpreter refers to any `TeX-Format',
because some files are installed in TDS:tex/generic//.

This work consists of the main source file bigintcalc.dtx
and the derived files
   bigintcalc.sty, bigintcalc.pdf, bigintcalc.ins, bigintcalc.drv,
   bigintcalc-test1.tex, bigintcalc-test2.tex,
   bigintcalc-test3.tex.

\endpreamble
\let\MetaPrefix\DoubleperCent

\generate{%
  \file{bigintcalc.ins}{\from{bigintcalc.dtx}{install}}%
  \file{bigintcalc.drv}{\from{bigintcalc.dtx}{driver}}%
  \usedir{tex/generic/oberdiek}%
  \file{bigintcalc.sty}{\from{bigintcalc.dtx}{package}}%
%  \usedir{doc/latex/oberdiek/test}%
%  \file{bigintcalc-test1.tex}{\from{bigintcalc.dtx}{test1}}%
%  \file{bigintcalc-test2.tex}{\from{bigintcalc.dtx}{test2,etex}}%
%  \file{bigintcalc-test3.tex}{\from{bigintcalc.dtx}{test2,noetex}}%
  \nopreamble
  \nopostamble
  \usedir{source/latex/oberdiek/catalogue}%
  \file{bigintcalc.xml}{\from{bigintcalc.dtx}{catalogue}}%
}

\catcode32=13\relax% active space
\let =\space%
\Msg{************************************************************************}
\Msg{*}
\Msg{* To finish the installation you have to move the following}
\Msg{* file into a directory searched by TeX:}
\Msg{*}
\Msg{*     bigintcalc.sty}
\Msg{*}
\Msg{* To produce the documentation run the file `bigintcalc.drv'}
\Msg{* through LaTeX.}
\Msg{*}
\Msg{* Happy TeXing!}
\Msg{*}
\Msg{************************************************************************}

\endbatchfile
%</install>
%<*ignore>
\fi
%</ignore>
%<*driver>
\NeedsTeXFormat{LaTeX2e}
\ProvidesFile{bigintcalc.drv}%
  [2016/05/16 v1.4 Expandable calculations on big integers (HO)]%
\documentclass{ltxdoc}
\usepackage{wasysym}
\let\iint\relax
\let\iiint\relax
\usepackage[fleqn]{amsmath}

\DeclareMathOperator{\opInv}{Inv}
\DeclareMathOperator{\opAbs}{Abs}
\DeclareMathOperator{\opSgn}{Sgn}
\DeclareMathOperator{\opMin}{Min}
\DeclareMathOperator{\opMax}{Max}
\DeclareMathOperator{\opCmp}{Cmp}
\DeclareMathOperator{\opOdd}{Odd}
\DeclareMathOperator{\opInc}{Inc}
\DeclareMathOperator{\opDec}{Dec}
\DeclareMathOperator{\opAdd}{Add}
\DeclareMathOperator{\opSub}{Sub}
\DeclareMathOperator{\opShl}{Shl}
\DeclareMathOperator{\opShr}{Shr}
\DeclareMathOperator{\opMul}{Mul}
\DeclareMathOperator{\opSqr}{Sqr}
\DeclareMathOperator{\opFac}{Fac}
\DeclareMathOperator{\opPow}{Pow}
\DeclareMathOperator{\opDiv}{Div}
\DeclareMathOperator{\opMod}{Mod}
\DeclareMathOperator{\opInt}{Int}
\DeclareMathOperator{\opODD}{ifodd}

\newcommand*{\Def}{%
  \ensuremath{%
    \mathrel{\mathop{:}}=%
  }%
}
\newcommand*{\op}[1]{%
  \textsf{#1}%
}
\usepackage{holtxdoc}[2011/11/22]
\begin{document}
  \DocInput{bigintcalc.dtx}%
\end{document}
%</driver>
% \fi
%
%
% \CharacterTable
%  {Upper-case    \A\B\C\D\E\F\G\H\I\J\K\L\M\N\O\P\Q\R\S\T\U\V\W\X\Y\Z
%   Lower-case    \a\b\c\d\e\f\g\h\i\j\k\l\m\n\o\p\q\r\s\t\u\v\w\x\y\z
%   Digits        \0\1\2\3\4\5\6\7\8\9
%   Exclamation   \!     Double quote  \"     Hash (number) \#
%   Dollar        \$     Percent       \%     Ampersand     \&
%   Acute accent  \'     Left paren    \(     Right paren   \)
%   Asterisk      \*     Plus          \+     Comma         \,
%   Minus         \-     Point         \.     Solidus       \/
%   Colon         \:     Semicolon     \;     Less than     \<
%   Equals        \=     Greater than  \>     Question mark \?
%   Commercial at \@     Left bracket  \[     Backslash     \\
%   Right bracket \]     Circumflex    \^     Underscore    \_
%   Grave accent  \`     Left brace    \{     Vertical bar  \|
%   Right brace   \}     Tilde         \~}
%
% \GetFileInfo{bigintcalc.drv}
%
% \title{The \xpackage{bigintcalc} package}
% \date{2016/05/16 v1.4}
% \author{Heiko Oberdiek\thanks
% {Please report any issues at https://github.com/ho-tex/oberdiek/issues}\\
% \xemail{heiko.oberdiek at googlemail.com}}
%
% \maketitle
%
% \begin{abstract}
% This package provides expandable arithmetic operations
% with big integers that can exceed \TeX's number limits.
% \end{abstract}
%
% \tableofcontents
%
% \section{Documentation}
%
% \subsection{Introduction}
%
% Package \xpackage{bigintcalc} defines arithmetic operations
% that deal with big integers. Big integers can be given
% either as explicit integer number or as macro code
% that expands to an explicit number. \emph{Big} means that
% there is no limit on the size of the number. Big integers
% may exceed \TeX's range limitation of -2147483647 and 2147483647.
% Only memory issues will limit the usable range.
%
% In opposite to package \xpackage{intcalc}
% unexpandable command tokens are not supported, even if
% they are valid \TeX\ numbers like count registers or
% commands created by \cs{chardef}. Nevertheless they may
% be used, if they are prefixed by \cs{number}.
%
% Also \eTeX's \cs{numexpr} expressions are not supported
% directly in the manner of package \xpackage{intcalc}.
% However they can be given if \cs{the}\cs{numexpr}
% or \cs{number}\cs{numexpr} are used.
%
% The operations have the form of macros that take one or
% two integers as parameter and return the integer result.
% The macro name is a three letter operation name prefixed
% by the package name, e.g. \cs{bigintcalcAdd}|{10}{43}| returns
% |53|.
%
% The macros are fully expandable, exactly two expansion
% steps generate the result. Therefore the operations
% may be used nearly everywhere in \TeX, even inside
% \cs{csname}, file names,  or other
% expandable contexts.
%
% \subsection{Conditions}
%
% \subsubsection{Preconditions}
%
% \begin{itemize}
% \item
%   Arguments can be anything that expands to a number that consists
%   of optional signs and digits.
% \item
%   The arguments and return values must be sound.
%   Zero as divisor or factorials of negative numbers will cause errors.
% \end{itemize}
%
% \subsubsection{Postconditions}
%
% Additional properties of the macros apart from calculating
% a correct result (of course \smiley):
% \begin{itemize}
% \item
%   The macros are fully expandable. Thus they can
%   be used inside \cs{edef}, \cs{csname},
%   for example.
% \item
%   Furthermore exactly two expansion steps calculate the result.
% \item
%   The number consists of one optional minus sign and one or more
%   digits. The first digit is larger than zero for
%   numbers that consists of more than one digit.
%
%   In short, the number format is exactly the same as
%   \cs{number} generates, but without its range limitation.
%   And the tokens (minus sign, digits)
%   have catcode 12 (other).
% \item
%   Call by value is simulated. First the arguments are
%   converted to numbers. Then these numbers are used
%   in the calculations.
%
%   Remember that arguments
%   may contain expensive macros or \eTeX\ expressions.
%   This strategy avoids multiple evaluations of such
%   arguments.
% \end{itemize}
%
% \subsection{Error handling}
%
% Some errors are detected by the macros, example: division by zero.
% In this cases an undefined control sequence is called and causes
% a TeX error message, example: \cs{BigIntCalcError:DivisionByZero}.
% The name of the control sequence contains
% the reason for the error. The \TeX\ error may be ignored.
% Then the operation returns zero as result.
% Because the macros are supposed to work in expandible contexts.
% An traditional error message, however, is not expandable and
% would break these contexts.
%
% \subsection{Operations}
%
% Some definition equations below use the function $\opInt$
% that converts a real number to an integer. The number
% is truncated that means rounding to zero:
% \begin{gather*}
%   \opInt(x) \Def
%   \begin{cases}
%     \lfloor x\rfloor & \text{if $x\geq0$}\\
%     \lceil x\rceil & \text{otherwise}
%   \end{cases}
% \end{gather*}
%
% \subsubsection{\op{Num}}
%
% \begin{declcs}{bigintcalcNum} \M{x}
% \end{declcs}
% Macro \cs{bigintcalcNum} converts its argument to a normalized integer
% number without unnecessary leading zeros or signs.
% The result matches the regular expression:
%\begin{quote}
%\begin{verbatim}
%0|-?[1-9][0-9]*
%\end{verbatim}
%\end{quote}
%
% \subsubsection{\op{Inv}, \op{Abs}, \op{Sgn}}
%
% \begin{declcs}{bigintcalcInv} \M{x}
% \end{declcs}
% Macro \cs{bigintcalcInv} switches the sign.
% \begin{gather*}
%   \opInv(x) \Def -x
% \end{gather*}
%
% \begin{declcs}{bigintcalcAbs} \M{x}
% \end{declcs}
% Macro \cs{bigintcalcAbs} returns the absolute value
% of integer \meta{x}.
% \begin{gather*}
%   \opAbs(x) \Def \vert x\vert
% \end{gather*}
%
% \begin{declcs}{bigintcalcSgn} \M{x}
% \end{declcs}
% Macro \cs{bigintcalcSgn} encodes the sign of \meta{x} as number.
% \begin{gather*}
%   \opSgn(x) \Def
%   \begin{cases}
%     -1& \text{if $x<0$}\\
%      0& \text{if $x=0$}\\
%      1& \text{if $x>0$}
%   \end{cases}
% \end{gather*}
% These return values can easily be distinguished by \cs{ifcase}:
%\begin{quote}
%\begin{verbatim}
%\ifcase\bigintcalcSgn{<x>}
%  $x=0$
%\or
%  $x>0$
%\else
%  $x<0$
%\fi
%\end{verbatim}
%\end{quote}
%
% \subsubsection{\op{Min}, \op{Max}, \op{Cmp}}
%
% \begin{declcs}{bigintcalcMin} \M{x} \M{y}
% \end{declcs}
% Macro \cs{bigintcalcMin} returns the smaller of the two integers.
% \begin{gather*}
%   \opMin(x,y) \Def
%   \begin{cases}
%     x & \text{if $x<y$}\\
%     y & \text{otherwise}
%   \end{cases}
% \end{gather*}
%
% \begin{declcs}{bigintcalcMax} \M{x} \M{y}
% \end{declcs}
% Macro \cs{bigintcalcMax} returns the larger of the two integers.
% \begin{gather*}
%   \opMax(x,y) \Def
%   \begin{cases}
%     x & \text{if $x>y$}\\
%     y & \text{otherwise}
%   \end{cases}
% \end{gather*}
%
% \begin{declcs}{bigintcalcCmp} \M{x} \M{y}
% \end{declcs}
% Macro \cs{bigintcalcCmp} encodes the comparison result as number:
% \begin{gather*}
%   \opCmp(x,y) \Def
%   \begin{cases}
%     -1 & \text{if $x<y$}\\
%      0 & \text{if $x=y$}\\
%      1 & \text{if $x>y$}
%   \end{cases}
% \end{gather*}
% These values can be distinguished by \cs{ifcase}:
%\begin{quote}
%\begin{verbatim}
%\ifcase\bigintcalcCmp{<x>}{<y>}
%  $x=y$
%\or
%  $x>y$
%\else
%  $x<y$
%\fi
%\end{verbatim}
%\end{quote}
%
% \subsubsection{\op{Odd}}
%
% \begin{declcs}{bigintcalcOdd} \M{x}
% \end{declcs}
% \begin{gather*}
%   \opOdd(x) \Def
%   \begin{cases}
%     1 & \text{if $x$ is odd}\\
%     0 & \text{if $x$ is even}
%   \end{cases}
% \end{gather*}
%
% \subsubsection{\op{Inc}, \op{Dec}, \op{Add}, \op{Sub}}
%
% \begin{declcs}{bigintcalcInc} \M{x}
% \end{declcs}
% Macro \cs{bigintcalcInc} increments \meta{x} by one.
% \begin{gather*}
%   \opInc(x) \Def x + 1
% \end{gather*}
%
% \begin{declcs}{bigintcalcDec} \M{x}
% \end{declcs}
% Macro \cs{bigintcalcDec} decrements \meta{x} by one.
% \begin{gather*}
%   \opDec(x) \Def x - 1
% \end{gather*}
%
% \begin{declcs}{bigintcalcAdd} \M{x} \M{y}
% \end{declcs}
% Macro \cs{bigintcalcAdd} adds the two numbers.
% \begin{gather*}
%   \opAdd(x, y) \Def x + y
% \end{gather*}
%
% \begin{declcs}{bigintcalcSub} \M{x} \M{y}
% \end{declcs}
% Macro \cs{bigintcalcSub} calculates the difference.
% \begin{gather*}
%   \opSub(x, y) \Def x - y
% \end{gather*}
%
% \subsubsection{\op{Shl}, \op{Shr}}
%
% \begin{declcs}{bigintcalcShl} \M{x}
% \end{declcs}
% Macro \cs{bigintcalcShl} implements shifting to the left that
% means the number is multiplied by two.
% The sign is preserved.
% \begin{gather*}
%   \opShl(x) \Def x*2
% \end{gather*}
%
% \begin{declcs}{bigintcalcShr} \M{x}
% \end{declcs}
% Macro \cs{bigintcalcShr} implements shifting to the right.
% That is equivalent to an integer division by two.
% The sign is preserved.
% \begin{gather*}
%   \opShr(x) \Def \opInt(x/2)
% \end{gather*}
%
% \subsubsection{\op{Mul}, \op{Sqr}, \op{Fac}, \op{Pow}}
%
% \begin{declcs}{bigintcalcMul} \M{x} \M{y}
% \end{declcs}
% Macro \cs{bigintcalcMul} calculates the product of
% \meta{x} and \meta{y}.
% \begin{gather*}
%   \opMul(x,y) \Def x*y
% \end{gather*}
%
% \begin{declcs}{bigintcalcSqr} \M{x}
% \end{declcs}
% Macro \cs{bigintcalcSqr} returns the square product.
% \begin{gather*}
%   \opSqr(x) \Def x^2
% \end{gather*}
%
% \begin{declcs}{bigintcalcFac} \M{x}
% \end{declcs}
% Macro \cs{bigintcalcFac} returns the factorial of \meta{x}.
% Negative numbers are not permitted.
% \begin{gather*}
%   \opFac(x) \Def x!\qquad\text{for $x\geq0$}
% \end{gather*}
% ($0! = 1$)
%
% \begin{declcs}{bigintcalcPow} M{x} M{y}
% \end{declcs}
% Macro \cs{bigintcalcPow} calculates the value of \meta{x} to the
% power of \meta{y}. The error ``division by zero'' is thrown
% if \meta{x} is zero and \meta{y} is negative.
% permitted:
% \begin{gather*}
%   \opPow(x,y) \Def
%   \opInt(x^y)\qquad\text{for $x\neq0$ or $y\geq0$}
% \end{gather*}
% ($0^0 = 1$)
%
% \subsubsection{\op{Div}, \op{Mul}}
%
% \begin{declcs}{bigintcalcDiv} \M{x} \M{y}
% \end{declcs}
% Macro \cs{bigintcalcDiv} performs an integer division.
% Argument \meta{y} must not be zero.
% \begin{gather*}
%   \opDiv(x,y) \Def \opInt(x/y)\qquad\text{for $y\neq0$}
% \end{gather*}
%
% \begin{declcs}{bigintcalcMod} \M{x} \M{y}
% \end{declcs}
% Macro \cs{bigintcalcMod} gets the remainder of the integer
% division. The sign follows the divisor \meta{y}.
% Argument \meta{y} must not be zero.
% \begin{gather*}
%   \opMod(x,y) \Def x\mathrel{\%}y\qquad\text{for $y\neq0$}
% \end{gather*}
% The result ranges:
% \begin{gather*}
%   -\vert y\vert < \opMod(x,y) \leq 0\qquad\text{for $y<0$}\\
%   0 \leq \opMod(x,y) < y\qquad\text{for $y\geq0$}
% \end{gather*}
%
% \subsection{Interface for programmers}
%
% If the programmer can ensure some more properties about
% the arguments of the operations, then the following
% macros are a little more efficient.
%
% In general numbers must obey the following constraints:
% \begin{itemize}
% \item Plain number: digit tokens only, no command tokens.
% \item Non-negative. Signs are forbidden.
% \item Delimited by exclamation mark. Curly braces
%       around the number are not allowed and will
%       break the code.
% \end{itemize}
%
% \begin{declcs}{BigIntCalcOdd} \meta{number} |!|
% \end{declcs}
% |1|/|0| is returned if \meta{number} is odd/even.
%
% \begin{declcs}{BigIntCalcInc} \meta{number} |!|
% \end{declcs}
% Incrementation.
%
% \begin{declcs}{BigIntCalcDec} \meta{number} |!|
% \end{declcs}
% Decrementation, positive number without zero.
%
% \begin{declcs}{BigIntCalcAdd} \meta{number A} |!| \meta{number B} |!|
% \end{declcs}
% Addition, $A\geq B$.
%
% \begin{declcs}{BigIntCalcSub} \meta{number A} |!| \meta{number B} |!|
% \end{declcs}
% Subtraction, $A\geq B$.
%
% \begin{declcs}{BigIntCalcShl} \meta{number} |!|
% \end{declcs}
% Left shift (multiplication with two).
%
% \begin{declcs}{BigIntCalcShr} \meta{number} |!|
% \end{declcs}
% Right shift (integer division by two).
%
% \begin{declcs}{BigIntCalcMul} \meta{number A} |!| \meta{number B} |!|
% \end{declcs}
% Multiplication, $A\geq B$.
%
% \begin{declcs}{BigIntCalcDiv} \meta{number A} |!| \meta{number B} |!|
% \end{declcs}
% Division operation.
%
% \begin{declcs}{BigIntCalcMod} \meta{number A} |!| \meta{number B} |!|
% \end{declcs}
% Modulo operation.
%
%
% \StopEventually{
% }
%
% \section{Implementation}
%
%    \begin{macrocode}
%<*package>
%    \end{macrocode}
%
% \subsection{Reload check and package identification}
%    Reload check, especially if the package is not used with \LaTeX.
%    \begin{macrocode}
\begingroup\catcode61\catcode48\catcode32=10\relax%
  \catcode13=5 % ^^M
  \endlinechar=13 %
  \catcode35=6 % #
  \catcode39=12 % '
  \catcode44=12 % ,
  \catcode45=12 % -
  \catcode46=12 % .
  \catcode58=12 % :
  \catcode64=11 % @
  \catcode123=1 % {
  \catcode125=2 % }
  \expandafter\let\expandafter\x\csname ver@bigintcalc.sty\endcsname
  \ifx\x\relax % plain-TeX, first loading
  \else
    \def\empty{}%
    \ifx\x\empty % LaTeX, first loading,
      % variable is initialized, but \ProvidesPackage not yet seen
    \else
      \expandafter\ifx\csname PackageInfo\endcsname\relax
        \def\x#1#2{%
          \immediate\write-1{Package #1 Info: #2.}%
        }%
      \else
        \def\x#1#2{\PackageInfo{#1}{#2, stopped}}%
      \fi
      \x{bigintcalc}{The package is already loaded}%
      \aftergroup\endinput
    \fi
  \fi
\endgroup%
%    \end{macrocode}
%    Package identification:
%    \begin{macrocode}
\begingroup\catcode61\catcode48\catcode32=10\relax%
  \catcode13=5 % ^^M
  \endlinechar=13 %
  \catcode35=6 % #
  \catcode39=12 % '
  \catcode40=12 % (
  \catcode41=12 % )
  \catcode44=12 % ,
  \catcode45=12 % -
  \catcode46=12 % .
  \catcode47=12 % /
  \catcode58=12 % :
  \catcode64=11 % @
  \catcode91=12 % [
  \catcode93=12 % ]
  \catcode123=1 % {
  \catcode125=2 % }
  \expandafter\ifx\csname ProvidesPackage\endcsname\relax
    \def\x#1#2#3[#4]{\endgroup
      \immediate\write-1{Package: #3 #4}%
      \xdef#1{#4}%
    }%
  \else
    \def\x#1#2[#3]{\endgroup
      #2[{#3}]%
      \ifx#1\@undefined
        \xdef#1{#3}%
      \fi
      \ifx#1\relax
        \xdef#1{#3}%
      \fi
    }%
  \fi
\expandafter\x\csname ver@bigintcalc.sty\endcsname
\ProvidesPackage{bigintcalc}%
  [2016/05/16 v1.4 Expandable calculations on big integers (HO)]%
%    \end{macrocode}
%
% \subsection{Catcodes}
%
%    \begin{macrocode}
\begingroup\catcode61\catcode48\catcode32=10\relax%
  \catcode13=5 % ^^M
  \endlinechar=13 %
  \catcode123=1 % {
  \catcode125=2 % }
  \catcode64=11 % @
  \def\x{\endgroup
    \expandafter\edef\csname BIC@AtEnd\endcsname{%
      \endlinechar=\the\endlinechar\relax
      \catcode13=\the\catcode13\relax
      \catcode32=\the\catcode32\relax
      \catcode35=\the\catcode35\relax
      \catcode61=\the\catcode61\relax
      \catcode64=\the\catcode64\relax
      \catcode123=\the\catcode123\relax
      \catcode125=\the\catcode125\relax
    }%
  }%
\x\catcode61\catcode48\catcode32=10\relax%
\catcode13=5 % ^^M
\endlinechar=13 %
\catcode35=6 % #
\catcode64=11 % @
\catcode123=1 % {
\catcode125=2 % }
\def\TMP@EnsureCode#1#2{%
  \edef\BIC@AtEnd{%
    \BIC@AtEnd
    \catcode#1=\the\catcode#1\relax
  }%
  \catcode#1=#2\relax
}
\TMP@EnsureCode{33}{12}% !
\TMP@EnsureCode{36}{14}% $ (comment!)
\TMP@EnsureCode{38}{14}% & (comment!)
\TMP@EnsureCode{40}{12}% (
\TMP@EnsureCode{41}{12}% )
\TMP@EnsureCode{42}{12}% *
\TMP@EnsureCode{43}{12}% +
\TMP@EnsureCode{45}{12}% -
\TMP@EnsureCode{46}{12}% .
\TMP@EnsureCode{47}{12}% /
\TMP@EnsureCode{58}{11}% : (letter!)
\TMP@EnsureCode{60}{12}% <
\TMP@EnsureCode{62}{12}% >
\TMP@EnsureCode{63}{14}% ? (comment!)
\TMP@EnsureCode{91}{12}% [
\TMP@EnsureCode{93}{12}% ]
\edef\BIC@AtEnd{\BIC@AtEnd\noexpand\endinput}
\begingroup\expandafter\expandafter\expandafter\endgroup
\expandafter\ifx\csname BIC@TestMode\endcsname\relax
\else
  \catcode63=9 % ? (ignore)
\fi
? \let\BIC@@TestMode\BIC@TestMode
%    \end{macrocode}
%
% \subsection{\eTeX\ detection}
%
%    \begin{macrocode}
\begingroup\expandafter\expandafter\expandafter\endgroup
\expandafter\ifx\csname numexpr\endcsname\relax
  \catcode36=9 % $ (ignore)
\else
  \catcode38=9 % & (ignore)
\fi
%    \end{macrocode}
%
% \subsection{Help macros}
%
%    \begin{macro}{\BIC@Fi}
%    \begin{macrocode}
\let\BIC@Fi\fi
%    \end{macrocode}
%    \end{macro}
%    \begin{macro}{\BIC@AfterFi}
%    \begin{macrocode}
\def\BIC@AfterFi#1#2\BIC@Fi{\fi#1}%
%    \end{macrocode}
%    \end{macro}
%    \begin{macro}{\BIC@AfterFiFi}
%    \begin{macrocode}
\def\BIC@AfterFiFi#1#2\BIC@Fi{\fi\fi#1}%
%    \end{macrocode}
%    \end{macro}
%    \begin{macro}{\BIC@AfterFiFiFi}
%    \begin{macrocode}
\def\BIC@AfterFiFiFi#1#2\BIC@Fi{\fi\fi\fi#1}%
%    \end{macrocode}
%    \end{macro}
%
%    \begin{macro}{\BIC@Space}
%    \begin{macrocode}
\begingroup
  \def\x#1{\endgroup
    \let\BIC@Space= #1%
  }%
\x{ }
%    \end{macrocode}
%    \end{macro}
%
% \subsection{Expand number}
%
%    \begin{macrocode}
\begingroup\expandafter\expandafter\expandafter\endgroup
\expandafter\ifx\csname RequirePackage\endcsname\relax
  \def\TMP@RequirePackage#1[#2]{%
    \begingroup\expandafter\expandafter\expandafter\endgroup
    \expandafter\ifx\csname ver@#1.sty\endcsname\relax
      \input #1.sty\relax
    \fi
  }%
  \TMP@RequirePackage{pdftexcmds}[2007/11/11]%
\else
  \RequirePackage{pdftexcmds}[2007/11/11]%
\fi
%    \end{macrocode}
%
%    \begin{macrocode}
\begingroup\expandafter\expandafter\expandafter\endgroup
\expandafter\ifx\csname pdf@escapehex\endcsname\relax
%    \end{macrocode}
%
%    \begin{macro}{\BIC@Expand}
%    \begin{macrocode}
  \def\BIC@Expand#1{%
    \romannumeral0%
    \BIC@@Expand#1!\@nil{}%
  }%
%    \end{macrocode}
%    \end{macro}
%    \begin{macro}{\BIC@@Expand}
%    \begin{macrocode}
  \def\BIC@@Expand#1#2\@nil#3{%
    \expandafter\ifcat\noexpand#1\relax
      \expandafter\@firstoftwo
    \else
      \expandafter\@secondoftwo
    \fi
    {%
      \expandafter\BIC@@Expand#1#2\@nil{#3}%
    }{%
      \ifx#1!%
        \expandafter\@firstoftwo
      \else
        \expandafter\@secondoftwo
      \fi
      { #3}{%
        \BIC@@Expand#2\@nil{#3#1}%
      }%
    }%
  }%
%    \end{macrocode}
%    \end{macro}
%    \begin{macro}{\@firstoftwo}
%    \begin{macrocode}
  \expandafter\ifx\csname @firstoftwo\endcsname\relax
    \long\def\@firstoftwo#1#2{#1}%
  \fi
%    \end{macrocode}
%    \end{macro}
%    \begin{macro}{\@secondoftwo}
%    \begin{macrocode}
  \expandafter\ifx\csname @secondoftwo\endcsname\relax
    \long\def\@secondoftwo#1#2{#2}%
  \fi
%    \end{macrocode}
%    \end{macro}
%
%    \begin{macrocode}
\else
%    \end{macrocode}
%    \begin{macro}{\BIC@Expand}
%    \begin{macrocode}
  \def\BIC@Expand#1{%
    \romannumeral0\expandafter\expandafter\expandafter\BIC@Space
    \pdf@unescapehex{%
      \expandafter\expandafter\expandafter
      \BIC@StripHexSpace\pdf@escapehex{#1}20\@nil
    }%
  }%
%    \end{macrocode}
%    \end{macro}
%    \begin{macro}{\BIC@StripHexSpace}
%    \begin{macrocode}
  \def\BIC@StripHexSpace#120#2\@nil{%
    #1%
    \ifx\\#2\\%
    \else
      \BIC@AfterFi{%
        \BIC@StripHexSpace#2\@nil
      }%
    \BIC@Fi
  }%
%    \end{macrocode}
%    \end{macro}
%    \begin{macrocode}
\fi
%    \end{macrocode}
%
% \subsection{Normalize expanded number}
%
%    \begin{macro}{\BIC@Normalize}
%    |#1|: result sign\\
%    |#2|: first token of number
%    \begin{macrocode}
\def\BIC@Normalize#1#2{%
  \ifx#2-%
    \ifx\\#1\\%
      \BIC@AfterFiFi{%
        \BIC@Normalize-%
      }%
    \else
      \BIC@AfterFiFi{%
        \BIC@Normalize{}%
      }%
    \fi
  \else
    \ifx#2+%
      \BIC@AfterFiFi{%
        \BIC@Normalize{#1}%
      }%
    \else
      \ifx#20%
        \BIC@AfterFiFiFi{%
          \BIC@NormalizeZero{#1}%
        }%
      \else
        \BIC@AfterFiFiFi{%
          \BIC@NormalizeDigits#1#2%
        }%
      \fi
    \fi
  \BIC@Fi
}
%    \end{macrocode}
%    \end{macro}
%    \begin{macro}{\BIC@NormalizeZero}
%    \begin{macrocode}
\def\BIC@NormalizeZero#1#2{%
  \ifx#2!%
    \BIC@AfterFi{ 0}%
  \else
    \ifx#20%
      \BIC@AfterFiFi{%
        \BIC@NormalizeZero{#1}%
      }%
    \else
      \BIC@AfterFiFi{%
        \BIC@NormalizeDigits#1#2%
      }%
    \fi
  \BIC@Fi
}
%    \end{macrocode}
%    \end{macro}
%    \begin{macro}{\BIC@NormalizeDigits}
%    \begin{macrocode}
\def\BIC@NormalizeDigits#1!{ #1}
%    \end{macrocode}
%    \end{macro}
%
% \subsection{\op{Num}}
%
%    \begin{macro}{\bigintcalcNum}
%    \begin{macrocode}
\def\bigintcalcNum#1{%
  \romannumeral0%
  \expandafter\expandafter\expandafter\BIC@Normalize
  \expandafter\expandafter\expandafter{%
  \expandafter\expandafter\expandafter}%
  \BIC@Expand{#1}!%
}
%    \end{macrocode}
%    \end{macro}
%
% \subsection{\op{Inv}, \op{Abs}, \op{Sgn}}
%
%
%    \begin{macro}{\bigintcalcInv}
%    \begin{macrocode}
\def\bigintcalcInv#1{%
  \romannumeral0\expandafter\expandafter\expandafter\BIC@Space
  \bigintcalcNum{-#1}%
}
%    \end{macrocode}
%    \end{macro}
%
%    \begin{macro}{\bigintcalcAbs}
%    \begin{macrocode}
\def\bigintcalcAbs#1{%
  \romannumeral0%
  \expandafter\expandafter\expandafter\BIC@Abs
  \bigintcalcNum{#1}%
}
%    \end{macrocode}
%    \end{macro}
%    \begin{macro}{\BIC@Abs}
%    \begin{macrocode}
\def\BIC@Abs#1{%
  \ifx#1-%
    \expandafter\BIC@Space
  \else
    \expandafter\BIC@Space
    \expandafter#1%
  \fi
}
%    \end{macrocode}
%    \end{macro}
%
%    \begin{macro}{\bigintcalcSgn}
%    \begin{macrocode}
\def\bigintcalcSgn#1{%
  \number
  \expandafter\expandafter\expandafter\BIC@Sgn
  \bigintcalcNum{#1}! %
}
%    \end{macrocode}
%    \end{macro}
%    \begin{macro}{\BIC@Sgn}
%    \begin{macrocode}
\def\BIC@Sgn#1#2!{%
  \ifx#1-%
    -1%
  \else
    \ifx#10%
      0%
    \else
      1%
    \fi
  \fi
}
%    \end{macrocode}
%    \end{macro}
%
% \subsection{\op{Cmp}, \op{Min}, \op{Max}}
%
%    \begin{macro}{\bigintcalcCmp}
%    \begin{macrocode}
\def\bigintcalcCmp#1#2{%
  \number
  \expandafter\expandafter\expandafter\BIC@Cmp
  \bigintcalcNum{#2}!{#1}%
}
%    \end{macrocode}
%    \end{macro}
%    \begin{macro}{\BIC@Cmp}
%    \begin{macrocode}
\def\BIC@Cmp#1!#2{%
  \expandafter\expandafter\expandafter\BIC@@Cmp
  \bigintcalcNum{#2}!#1!%
}
%    \end{macrocode}
%    \end{macro}
%    \begin{macro}{\BIC@@Cmp}
%    \begin{macrocode}
\def\BIC@@Cmp#1#2!#3#4!{%
  \ifx#1-%
    \ifx#3-%
      \BIC@AfterFiFi{%
        \BIC@@Cmp#4!#2!%
      }%
    \else
      \BIC@AfterFiFi{%
        -1 %
      }%
    \fi
  \else
    \ifx#3-%
      \BIC@AfterFiFi{%
        1 %
      }%
    \else
      \BIC@AfterFiFi{%
        \BIC@CmpLength#1#2!#3#4!#1#2!#3#4!%
      }%
    \fi
  \BIC@Fi
}
%    \end{macrocode}
%    \end{macro}
%    \begin{macro}{\BIC@PosCmp}
%    \begin{macrocode}
\def\BIC@PosCmp#1!#2!{%
  \BIC@CmpLength#1!#2!#1!#2!%
}
%    \end{macrocode}
%    \end{macro}
%    \begin{macro}{\BIC@CmpLength}
%    \begin{macrocode}
\def\BIC@CmpLength#1#2!#3#4!{%
  \ifx\\#2\\%
    \ifx\\#4\\%
      \BIC@AfterFiFi\BIC@CmpDiff
    \else
      \BIC@AfterFiFi{%
        \BIC@CmpResult{-1}%
      }%
    \fi
  \else
    \ifx\\#4\\%
      \BIC@AfterFiFi{%
        \BIC@CmpResult1%
      }%
    \else
      \BIC@AfterFiFi{%
        \BIC@CmpLength#2!#4!%
      }%
    \fi
  \BIC@Fi
}
%    \end{macrocode}
%    \end{macro}
%    \begin{macro}{\BIC@CmpResult}
%    \begin{macrocode}
\def\BIC@CmpResult#1#2!#3!{#1 }
%    \end{macrocode}
%    \end{macro}
%    \begin{macro}{\BIC@CmpDiff}
%    \begin{macrocode}
\def\BIC@CmpDiff#1#2!#3#4!{%
  \ifnum#1<#3 %
    \BIC@AfterFi{%
      -1 %
    }%
  \else
    \ifnum#1>#3 %
      \BIC@AfterFiFi{%
        1 %
      }%
    \else
      \ifx\\#2\\%
        \BIC@AfterFiFiFi{%
          0 %
        }%
      \else
        \BIC@AfterFiFiFi{%
          \BIC@CmpDiff#2!#4!%
        }%
      \fi
    \fi
  \BIC@Fi
}
%    \end{macrocode}
%    \end{macro}
%
%    \begin{macro}{\bigintcalcMin}
%    \begin{macrocode}
\def\bigintcalcMin#1{%
  \romannumeral0%
  \expandafter\expandafter\expandafter\BIC@MinMax
  \bigintcalcNum{#1}!-!%
}
%    \end{macrocode}
%    \end{macro}
%    \begin{macro}{\bigintcalcMax}
%    \begin{macrocode}
\def\bigintcalcMax#1{%
  \romannumeral0%
  \expandafter\expandafter\expandafter\BIC@MinMax
  \bigintcalcNum{#1}!!%
}
%    \end{macrocode}
%    \end{macro}
%    \begin{macro}{\BIC@MinMax}
%    |#1|: $x$\\
%    |#2|: sign for comparison\\
%    |#3|: $y$
%    \begin{macrocode}
\def\BIC@MinMax#1!#2!#3{%
  \expandafter\expandafter\expandafter\BIC@@MinMax
  \bigintcalcNum{#3}!#1!#2!%
}
%    \end{macrocode}
%    \end{macro}
%    \begin{macro}{\BIC@@MinMax}
%    |#1|: $y$\\
%    |#2|: $x$\\
%    |#3|: sign for comparison
%    \begin{macrocode}
\def\BIC@@MinMax#1!#2!#3!{%
  \ifnum\BIC@@Cmp#1!#2!=#31 %
    \BIC@AfterFi{ #1}%
  \else
    \BIC@AfterFi{ #2}%
  \BIC@Fi
}
%    \end{macrocode}
%    \end{macro}
%
% \subsection{\op{Odd}}
%
%    \begin{macro}{\bigintcalcOdd}
%    \begin{macrocode}
\def\bigintcalcOdd#1{%
  \romannumeral0%
  \expandafter\expandafter\expandafter\BIC@Odd
  \bigintcalcAbs{#1}!%
}
%    \end{macrocode}
%    \end{macro}
%    \begin{macro}{\BigIntCalcOdd}
%    \begin{macrocode}
\def\BigIntCalcOdd#1!{%
  \romannumeral0%
  \BIC@Odd#1!%
}
%    \end{macrocode}
%    \end{macro}
%    \begin{macro}{\BIC@Odd}
%    |#1|: $x$
%    \begin{macrocode}
\def\BIC@Odd#1#2{%
  \ifx#2!%
    \ifodd#1 %
      \BIC@AfterFiFi{ 1}%
    \else
      \BIC@AfterFiFi{ 0}%
    \fi
  \else
    \expandafter\BIC@Odd\expandafter#2%
  \BIC@Fi
}
%    \end{macrocode}
%    \end{macro}
%
% \subsection{\op{Inc}, \op{Dec}}
%
%    \begin{macro}{\bigintcalcInc}
%    \begin{macrocode}
\def\bigintcalcInc#1{%
  \romannumeral0%
  \expandafter\expandafter\expandafter\BIC@IncSwitch
  \bigintcalcNum{#1}!%
}
%    \end{macrocode}
%    \end{macro}
%    \begin{macro}{\BIC@IncSwitch}
%    \begin{macrocode}
\def\BIC@IncSwitch#1#2!{%
  \ifcase\BIC@@Cmp#1#2!-1!%
    \BIC@AfterFi{ 0}%
  \or
    \BIC@AfterFi{%
      \BIC@Inc#1#2!{}%
    }%
  \else
    \BIC@AfterFi{%
      \expandafter-\romannumeral0%
      \BIC@Dec#2!{}%
    }%
  \BIC@Fi
}
%    \end{macrocode}
%    \end{macro}
%
%    \begin{macro}{\bigintcalcDec}
%    \begin{macrocode}
\def\bigintcalcDec#1{%
  \romannumeral0%
  \expandafter\expandafter\expandafter\BIC@DecSwitch
  \bigintcalcNum{#1}!%
}
%    \end{macrocode}
%    \end{macro}
%    \begin{macro}{\BIC@DecSwitch}
%    \begin{macrocode}
\def\BIC@DecSwitch#1#2!{%
  \ifcase\BIC@Sgn#1#2! %
    \BIC@AfterFi{ -1}%
  \or
    \BIC@AfterFi{%
      \BIC@Dec#1#2!{}%
    }%
  \else
    \BIC@AfterFi{%
      \expandafter-\romannumeral0%
      \BIC@Inc#2!{}%
    }%
  \BIC@Fi
}
%    \end{macrocode}
%    \end{macro}
%
%    \begin{macro}{\BigIntCalcInc}
%    \begin{macrocode}
\def\BigIntCalcInc#1!{%
  \romannumeral0\BIC@Inc#1!{}%
}
%    \end{macrocode}
%    \end{macro}
%    \begin{macro}{\BigIntCalcDec}
%    \begin{macrocode}
\def\BigIntCalcDec#1!{%
  \romannumeral0\BIC@Dec#1!{}%
}
%    \end{macrocode}
%    \end{macro}
%
%    \begin{macro}{\BIC@Inc}
%    \begin{macrocode}
\def\BIC@Inc#1#2!#3{%
  \ifx\\#2\\%
    \BIC@AfterFi{%
      \BIC@@Inc1#1#3!{}%
    }%
  \else
    \BIC@AfterFi{%
      \BIC@Inc#2!{#1#3}%
    }%
  \BIC@Fi
}
%    \end{macrocode}
%    \end{macro}
%    \begin{macro}{\BIC@@Inc}
%    \begin{macrocode}
\def\BIC@@Inc#1#2#3!#4{%
  \ifcase#1 %
    \ifx\\#3\\%
      \BIC@AfterFiFi{ #2#4}%
    \else
      \BIC@AfterFiFi{%
        \BIC@@Inc0#3!{#2#4}%
      }%
    \fi
  \else
    \ifnum#2<9 %
      \BIC@AfterFiFi{%
&       \expandafter\BIC@@@Inc\the\numexpr#2+1\relax
$       \expandafter\expandafter\expandafter\BIC@@@Inc
$       \ifcase#2 \expandafter1%
$       \or\expandafter2%
$       \or\expandafter3%
$       \or\expandafter4%
$       \or\expandafter5%
$       \or\expandafter6%
$       \or\expandafter7%
$       \or\expandafter8%
$       \or\expandafter9%
$?      \else\BigIntCalcError:ThisCannotHappen%
$       \fi
        0#3!{#4}%
      }%
    \else
      \BIC@AfterFiFi{%
        \BIC@@@Inc01#3!{#4}%
      }%
    \fi
  \BIC@Fi
}
%    \end{macrocode}
%    \end{macro}
%    \begin{macro}{\BIC@@@Inc}
%    \begin{macrocode}
\def\BIC@@@Inc#1#2#3!#4{%
  \ifx\\#3\\%
    \ifnum#2=1 %
      \BIC@AfterFiFi{ 1#1#4}%
    \else
      \BIC@AfterFiFi{ #1#4}%
    \fi
  \else
    \BIC@AfterFi{%
      \BIC@@Inc#2#3!{#1#4}%
    }%
  \BIC@Fi
}
%    \end{macrocode}
%    \end{macro}
%
%    \begin{macro}{\BIC@Dec}
%    \begin{macrocode}
\def\BIC@Dec#1#2!#3{%
  \ifx\\#2\\%
    \BIC@AfterFi{%
      \BIC@@Dec1#1#3!{}%
    }%
  \else
    \BIC@AfterFi{%
      \BIC@Dec#2!{#1#3}%
    }%
  \BIC@Fi
}
%    \end{macrocode}
%    \end{macro}
%    \begin{macro}{\BIC@@Dec}
%    \begin{macrocode}
\def\BIC@@Dec#1#2#3!#4{%
  \ifcase#1 %
    \ifx\\#3\\%
      \BIC@AfterFiFi{ #2#4}%
    \else
      \BIC@AfterFiFi{%
        \BIC@@Dec0#3!{#2#4}%
      }%
    \fi
  \else
    \ifnum#2>0 %
      \BIC@AfterFiFi{%
&       \expandafter\BIC@@@Dec\the\numexpr#2-1\relax
$       \expandafter\expandafter\expandafter\BIC@@@Dec
$       \ifcase#2
$?        \BigIntCalcError:ThisCannotHappen%
$       \or\expandafter0%
$       \or\expandafter1%
$       \or\expandafter2%
$       \or\expandafter3%
$       \or\expandafter4%
$       \or\expandafter5%
$       \or\expandafter6%
$       \or\expandafter7%
$       \or\expandafter8%
$?      \else\BigIntCalcError:ThisCannotHappen%
$       \fi
        0#3!{#4}%
      }%
    \else
      \BIC@AfterFiFi{%
        \BIC@@@Dec91#3!{#4}%
      }%
    \fi
  \BIC@Fi
}
%    \end{macrocode}
%    \end{macro}
%    \begin{macro}{\BIC@@@Dec}
%    \begin{macrocode}
\def\BIC@@@Dec#1#2#3!#4{%
  \ifx\\#3\\%
    \ifcase#1 %
      \ifx\\#4\\%
        \BIC@AfterFiFiFi{ 0}%
      \else
        \BIC@AfterFiFiFi{ #4}%
      \fi
    \else
      \BIC@AfterFiFi{ #1#4}%
    \fi
  \else
    \BIC@AfterFi{%
      \BIC@@Dec#2#3!{#1#4}%
    }%
  \BIC@Fi
}
%    \end{macrocode}
%    \end{macro}
%
% \subsection{\op{Add}, \op{Sub}}
%
%    \begin{macro}{\bigintcalcAdd}
%    \begin{macrocode}
\def\bigintcalcAdd#1{%
  \romannumeral0%
  \expandafter\expandafter\expandafter\BIC@Add
  \bigintcalcNum{#1}!%
}
%    \end{macrocode}
%    \end{macro}
%    \begin{macro}{\BIC@Add}
%    \begin{macrocode}
\def\BIC@Add#1!#2{%
  \expandafter\expandafter\expandafter
  \BIC@AddSwitch\bigintcalcNum{#2}!#1!%
}
%    \end{macrocode}
%    \end{macro}
%    \begin{macro}{\bigintcalcSub}
%    \begin{macrocode}
\def\bigintcalcSub#1#2{%
  \romannumeral0%
  \expandafter\expandafter\expandafter\BIC@Add
  \bigintcalcNum{-#2}!{#1}%
}
%    \end{macrocode}
%    \end{macro}
%
%    \begin{macro}{\BIC@AddSwitch}
%    Decision table for \cs{BIC@AddSwitch}.
%    \begin{quote}
%    \begin{tabular}[t]{@{}|l|l|l|l|l|@{}}
%      \hline
%      $x<0$ & $y<0$ & $-x>-y$ & $-$ & $\opAdd(-x,-y)$\\
%      \cline{3-3}\cline{5-5}
%            &       & else    &     & $\opAdd(-y,-x)$\\
%      \cline{2-5}
%            & else  & $-x> y$ & $-$ & $\opSub(-x, y)$\\
%      \cline{3-5}
%            &       & $-x= y$ &     & $0$\\
%      \cline{3-5}
%            &       & else    & $+$ & $\opSub( y,-x)$\\
%      \hline
%      else  & $y<0$ & $ x>-y$ & $+$ & $\opSub( x,-y)$\\
%      \cline{3-5}
%            &       & $ x=-y$ &     & $0$\\
%      \cline{3-5}
%            &       & else    & $-$ & $\opSub(-y, x)$\\
%      \cline{2-5}
%            & else  & $ x> y$ & $+$ & $\opAdd( x, y)$\\
%      \cline{3-3}\cline{5-5}
%            &       & else    &     & $\opAdd( y, x)$\\
%      \hline
%    \end{tabular}
%    \end{quote}
%    \begin{macrocode}
\def\BIC@AddSwitch#1#2!#3#4!{%
  \ifx#1-% x < 0
    \ifx#3-% y < 0
      \expandafter-\romannumeral0%
      \ifnum\BIC@PosCmp#2!#4!=1 % -x > -y
        \BIC@AfterFiFiFi{%
          \BIC@AddXY#2!#4!!!%
        }%
      \else % -x <= -y
        \BIC@AfterFiFiFi{%
          \BIC@AddXY#4!#2!!!%
        }%
      \fi
    \else % y >= 0
      \ifcase\BIC@PosCmp#2!#3#4!% -x = y
        \BIC@AfterFiFiFi{ 0}%
      \or % -x > y
        \expandafter-\romannumeral0%
        \BIC@AfterFiFiFi{%
          \BIC@SubXY#2!#3#4!!!%
        }%
      \else % -x <= y
        \BIC@AfterFiFiFi{%
          \BIC@SubXY#3#4!#2!!!%
        }%
      \fi
    \fi
  \else % x >= 0
    \ifx#3-% y < 0
      \ifcase\BIC@PosCmp#1#2!#4!% x = -y
        \BIC@AfterFiFiFi{ 0}%
      \or % x > -y
        \BIC@AfterFiFiFi{%
          \BIC@SubXY#1#2!#4!!!%
        }%
      \else % x <= -y
        \expandafter-\romannumeral0%
        \BIC@AfterFiFiFi{%
          \BIC@SubXY#4!#1#2!!!%
        }%
      \fi
    \else % y >= 0
      \ifnum\BIC@PosCmp#1#2!#3#4!=1 % x > y
        \BIC@AfterFiFiFi{%
          \BIC@AddXY#1#2!#3#4!!!%
        }%
      \else % x <= y
        \BIC@AfterFiFiFi{%
          \BIC@AddXY#3#4!#1#2!!!%
        }%
      \fi
    \fi
  \BIC@Fi
}
%    \end{macrocode}
%    \end{macro}
%
%    \begin{macro}{\BigIntCalcAdd}
%    \begin{macrocode}
\def\BigIntCalcAdd#1!#2!{%
  \romannumeral0\BIC@AddXY#1!#2!!!%
}
%    \end{macrocode}
%    \end{macro}
%    \begin{macro}{\BigIntCalcSub}
%    \begin{macrocode}
\def\BigIntCalcSub#1!#2!{%
  \romannumeral0\BIC@SubXY#1!#2!!!%
}
%    \end{macrocode}
%    \end{macro}
%
%    \begin{macro}{\BIC@AddXY}
%    \begin{macrocode}
\def\BIC@AddXY#1#2!#3#4!#5!#6!{%
  \ifx\\#2\\%
    \ifx\\#3\\%
      \BIC@AfterFiFi{%
        \BIC@DoAdd0!#1#5!#60!%
      }%
    \else
      \BIC@AfterFiFi{%
        \BIC@DoAdd0!#1#5!#3#6!%
      }%
    \fi
  \else
    \ifx\\#4\\%
      \ifx\\#3\\%
        \BIC@AfterFiFiFi{%
          \BIC@AddXY#2!{}!#1#5!#60!%
        }%
      \else
        \BIC@AfterFiFiFi{%
          \BIC@AddXY#2!{}!#1#5!#3#6!%
        }%
      \fi
    \else
      \BIC@AfterFiFi{%
        \BIC@AddXY#2!#4!#1#5!#3#6!%
      }%
    \fi
  \BIC@Fi
}
%    \end{macrocode}
%    \end{macro}
%    \begin{macro}{\BIC@DoAdd}
%    |#1|: carry\\
%    |#2|: reverted result\\
%    |#3#4|: reverted $x$\\
%    |#5#6|: reverted $y$
%    \begin{macrocode}
\def\BIC@DoAdd#1#2!#3#4!#5#6!{%
  \ifx\\#4\\%
    \BIC@AfterFi{%
&     \expandafter\BIC@Space
&     \the\numexpr#1+#3+#5\relax#2%
$     \expandafter\expandafter\expandafter\BIC@AddResult
$     \BIC@AddDigit#1#3#5#2%
    }%
  \else
    \BIC@AfterFi{%
      \expandafter\expandafter\expandafter\BIC@DoAdd
      \BIC@AddDigit#1#3#5#2!#4!#6!%
    }%
  \BIC@Fi
}
%    \end{macrocode}
%    \end{macro}
%    \begin{macro}{\BIC@AddResult}
%    \begin{macrocode}
$ \def\BIC@AddResult#1{%
$   \ifx#10%
$     \expandafter\BIC@Space
$   \else
$     \expandafter\BIC@Space\expandafter#1%
$   \fi
$ }%
%    \end{macrocode}
%    \end{macro}
%    \begin{macro}{\BIC@AddDigit}
%    |#1|: carry\\
%    |#2|: digit of $x$\\
%    |#3|: digit of $y$
%    \begin{macrocode}
\def\BIC@AddDigit#1#2#3{%
  \romannumeral0%
& \expandafter\BIC@@AddDigit\the\numexpr#1+#2+#3!%
$ \expandafter\BIC@@AddDigit\number%
$ \csname
$   BIC@AddCarry%
$   \ifcase#1 %
$     #2%
$   \else
$     \ifcase#2 1\or2\or3\or4\or5\or6\or7\or8\or9\or10\fi
$   \fi
$ \endcsname#3!%
}
%    \end{macrocode}
%    \end{macro}
%    \begin{macro}{\BIC@@AddDigit}
%    \begin{macrocode}
\def\BIC@@AddDigit#1!{%
  \ifnum#1<10 %
    \BIC@AfterFi{ 0#1}%
  \else
    \BIC@AfterFi{ #1}%
  \BIC@Fi
}
%    \end{macrocode}
%    \end{macro}
%    \begin{macro}{\BIC@AddCarry0}
%    \begin{macrocode}
$ \expandafter\def\csname BIC@AddCarry0\endcsname#1{#1}%
%    \end{macrocode}
%    \end{macro}
%    \begin{macro}{\BIC@AddCarry10}
%    \begin{macrocode}
$ \expandafter\def\csname BIC@AddCarry10\endcsname#1{1#1}%
%    \end{macrocode}
%    \end{macro}
%    \begin{macro}{\BIC@AddCarry[1-9]}
%    \begin{macrocode}
$ \def\BIC@Temp#1#2{%
$   \expandafter\def\csname BIC@AddCarry#1\endcsname##1{%
$     \ifcase##1 #1\or
$     #2%
$?    \else\BigIntCalcError:ThisCannotHappen%
$     \fi
$   }%
$ }%
$ \BIC@Temp 0{1\or2\or3\or4\or5\or6\or7\or8\or9}%
$ \BIC@Temp 1{2\or3\or4\or5\or6\or7\or8\or9\or10}%
$ \BIC@Temp 2{3\or4\or5\or6\or7\or8\or9\or10\or11}%
$ \BIC@Temp 3{4\or5\or6\or7\or8\or9\or10\or11\or12}%
$ \BIC@Temp 4{5\or6\or7\or8\or9\or10\or11\or12\or13}%
$ \BIC@Temp 5{6\or7\or8\or9\or10\or11\or12\or13\or14}%
$ \BIC@Temp 6{7\or8\or9\or10\or11\or12\or13\or14\or15}%
$ \BIC@Temp 7{8\or9\or10\or11\or12\or13\or14\or15\or16}%
$ \BIC@Temp 8{9\or10\or11\or12\or13\or14\or15\or16\or17}%
$ \BIC@Temp 9{10\or11\or12\or13\or14\or15\or16\or17\or18}%
%    \end{macrocode}
%    \end{macro}
%
%    \begin{macro}{\BIC@SubXY}
%    Preconditions:
%    \begin{itemize}
%    \item $x > y$, $x \geq 0$, and $y >= 0$
%    \item digits($x$) = digits($y$)
%    \end{itemize}
%    \begin{macrocode}
\def\BIC@SubXY#1#2!#3#4!#5!#6!{%
  \ifx\\#2\\%
    \ifx\\#3\\%
      \BIC@AfterFiFi{%
        \BIC@DoSub0!#1#5!#60!%
      }%
    \else
      \BIC@AfterFiFi{%
        \BIC@DoSub0!#1#5!#3#6!%
      }%
    \fi
  \else
    \ifx\\#4\\%
      \ifx\\#3\\%
        \BIC@AfterFiFiFi{%
          \BIC@SubXY#2!{}!#1#5!#60!%
        }%
      \else
        \BIC@AfterFiFiFi{%
          \BIC@SubXY#2!{}!#1#5!#3#6!%
        }%
      \fi
    \else
      \BIC@AfterFiFi{%
        \BIC@SubXY#2!#4!#1#5!#3#6!%
      }%
    \fi
  \BIC@Fi
}
%    \end{macrocode}
%    \end{macro}
%    \begin{macro}{\BIC@DoSub}
%    |#1|: carry\\
%    |#2|: reverted result\\
%    |#3#4|: reverted x\\
%    |#5#6|: reverted y
%    \begin{macrocode}
\def\BIC@DoSub#1#2!#3#4!#5#6!{%
  \ifx\\#4\\%
    \BIC@AfterFi{%
      \expandafter\expandafter\expandafter\BIC@SubResult
      \BIC@SubDigit#1#3#5#2%
    }%
  \else
    \BIC@AfterFi{%
      \expandafter\expandafter\expandafter\BIC@DoSub
      \BIC@SubDigit#1#3#5#2!#4!#6!%
    }%
  \BIC@Fi
}
%    \end{macrocode}
%    \end{macro}
%    \begin{macro}{\BIC@SubResult}
%    \begin{macrocode}
\def\BIC@SubResult#1{%
  \ifx#10%
    \expandafter\BIC@SubResult
  \else
    \expandafter\BIC@Space\expandafter#1%
  \fi
}
%    \end{macrocode}
%    \end{macro}
%    \begin{macro}{\BIC@SubDigit}
%    |#1|: carry\\
%    |#2|: digit of $x$\\
%    |#3|: digit of $y$
%    \begin{macrocode}
\def\BIC@SubDigit#1#2#3{%
  \romannumeral0%
& \expandafter\BIC@@SubDigit\the\numexpr#2-#3-#1!%
$ \expandafter\BIC@@AddDigit\number
$   \csname
$     BIC@SubCarry%
$     \ifcase#1 %
$       #3%
$     \else
$       \ifcase#3 1\or2\or3\or4\or5\or6\or7\or8\or9\or10\fi
$     \fi
$   \endcsname#2!%
}
%    \end{macrocode}
%    \end{macro}
%    \begin{macro}{\BIC@@SubDigit}
%    \begin{macrocode}
& \def\BIC@@SubDigit#1!{%
&   \ifnum#1<0 %
&     \BIC@AfterFi{%
&       \expandafter\BIC@Space
&       \expandafter1\the\numexpr#1+10\relax
&     }%
&   \else
&     \BIC@AfterFi{ 0#1}%
&   \BIC@Fi
& }%
%    \end{macrocode}
%    \end{macro}
%    \begin{macro}{\BIC@SubCarry0}
%    \begin{macrocode}
$ \expandafter\def\csname BIC@SubCarry0\endcsname#1{#1}%
%    \end{macrocode}
%    \end{macro}
%    \begin{macro}{\BIC@SubCarry10}%
%    \begin{macrocode}
$ \expandafter\def\csname BIC@SubCarry10\endcsname#1{1#1}%
%    \end{macrocode}
%    \end{macro}
%    \begin{macro}{\BIC@SubCarry[1-9]}
%    \begin{macrocode}
$ \def\BIC@Temp#1#2{%
$   \expandafter\def\csname BIC@SubCarry#1\endcsname##1{%
$     \ifcase##1 #2%
$?    \else\BigIntCalcError:ThisCannotHappen%
$     \fi
$   }%
$ }%
$ \BIC@Temp 1{19\or0\or1\or2\or3\or4\or5\or6\or7\or8}%
$ \BIC@Temp 2{18\or19\or0\or1\or2\or3\or4\or5\or6\or7}%
$ \BIC@Temp 3{17\or18\or19\or0\or1\or2\or3\or4\or5\or6}%
$ \BIC@Temp 4{16\or17\or18\or19\or0\or1\or2\or3\or4\or5}%
$ \BIC@Temp 5{15\or16\or17\or18\or19\or0\or1\or2\or3\or4}%
$ \BIC@Temp 6{14\or15\or16\or17\or18\or19\or0\or1\or2\or3}%
$ \BIC@Temp 7{13\or14\or15\or16\or17\or18\or19\or0\or1\or2}%
$ \BIC@Temp 8{12\or13\or14\or15\or16\or17\or18\or19\or0\or1}%
$ \BIC@Temp 9{11\or12\or13\or14\or15\or16\or17\or18\or19\or0}%
%    \end{macrocode}
%    \end{macro}
%
% \subsection{\op{Shl}, \op{Shr}}
%
%    \begin{macro}{\bigintcalcShl}
%    \begin{macrocode}
\def\bigintcalcShl#1{%
  \romannumeral0%
  \expandafter\expandafter\expandafter\BIC@Shl
  \bigintcalcNum{#1}!%
}
%    \end{macrocode}
%    \end{macro}
%    \begin{macro}{\BIC@Shl}
%    \begin{macrocode}
\def\BIC@Shl#1#2!{%
  \ifx#1-%
    \BIC@AfterFi{%
      \expandafter-\romannumeral0%
&     \BIC@@Shl#2!!%
$     \BIC@AddXY#2!#2!!!%
    }%
  \else
    \BIC@AfterFi{%
&     \BIC@@Shl#1#2!!%
$     \BIC@AddXY#1#2!#1#2!!!%
    }%
  \BIC@Fi
}
%    \end{macrocode}
%    \end{macro}
%    \begin{macro}{\BigIntCalcShl}
%    \begin{macrocode}
\def\BigIntCalcShl#1!{%
  \romannumeral0%
& \BIC@@Shl#1!!%
$ \BIC@AddXY#1!#1!!!%
}
%    \end{macrocode}
%    \end{macro}
%    \begin{macro}{\BIC@@Shl}
%    \begin{macrocode}
& \def\BIC@@Shl#1#2!{%
&   \ifx\\#2\\%
&     \BIC@AfterFi{%
&       \BIC@@@Shl0!#1%
&     }%
&   \else
&     \BIC@AfterFi{%
&       \BIC@@Shl#2!#1%
&     }%
&   \BIC@Fi
& }%
%    \end{macrocode}
%    \end{macro}
%    \begin{macro}{\BIC@@@Shl}
%    |#1|: carry\\
%    |#2|: result\\
%    |#3#4|: reverted number
%    \begin{macrocode}
& \def\BIC@@@Shl#1#2!#3#4!{%
&   \ifx\\#4\\%
&     \BIC@AfterFi{%
&       \expandafter\BIC@Space
&       \the\numexpr#3*2+#1\relax#2%
&     }%
&   \else
&     \BIC@AfterFi{%
&       \expandafter\BIC@@@@Shl\the\numexpr#3*2+#1!#2!#4!%
&     }%
&   \BIC@Fi
& }%
%    \end{macrocode}
%    \end{macro}
%    \begin{macro}{\BIC@@@@Shl}
%    \begin{macrocode}
& \def\BIC@@@@Shl#1!{%
&   \ifnum#1<10 %
&     \BIC@AfterFi{%
&       \BIC@@@Shl0#1%
&     }%
&   \else
&     \BIC@AfterFi{%
&       \BIC@@@Shl#1%
&     }%
&   \BIC@Fi
& }%
%    \end{macrocode}
%    \end{macro}
%
%    \begin{macro}{\bigintcalcShr}
%    \begin{macrocode}
\def\bigintcalcShr#1{%
  \romannumeral0%
  \expandafter\expandafter\expandafter\BIC@Shr
  \bigintcalcNum{#1}!%
}
%    \end{macrocode}
%    \end{macro}
%    \begin{macro}{\BIC@Shr}
%    \begin{macrocode}
\def\BIC@Shr#1#2!{%
  \ifx#1-%
    \expandafter-\romannumeral0%
    \BIC@AfterFi{%
      \BIC@@Shr#2!%
    }%
  \else
    \BIC@AfterFi{%
      \BIC@@Shr#1#2!%
    }%
  \BIC@Fi
}
%    \end{macrocode}
%    \end{macro}
%    \begin{macro}{\BigIntCalcShr}
%    \begin{macrocode}
\def\BigIntCalcShr#1!{%
  \romannumeral0%
  \BIC@@Shr#1!%
}
%    \end{macrocode}
%    \end{macro}
%    \begin{macro}{\BIC@@Shr}
%    \begin{macrocode}
\def\BIC@@Shr#1#2!{%
  \ifcase#1 %
    \BIC@AfterFi{ 0}%
  \or
    \ifx\\#2\\%
      \BIC@AfterFiFi{ 0}%
    \else
      \BIC@AfterFiFi{%
        \BIC@@@Shr#1#2!!%
      }%
    \fi
  \else
    \BIC@AfterFi{%
      \BIC@@@Shr0#1#2!!%
    }%
  \BIC@Fi
}
%    \end{macrocode}
%    \end{macro}
%    \begin{macro}{\BIC@@@Shr}
%    |#1|: carry\\
%    |#2#3|: number\\
%    |#4|: result
%    \begin{macrocode}
\def\BIC@@@Shr#1#2#3!#4!{%
  \ifx\\#3\\%
    \ifodd#1#2 %
      \BIC@AfterFiFi{%
&       \expandafter\BIC@ShrResult\the\numexpr(#1#2-1)/2\relax
$       \expandafter\expandafter\expandafter\BIC@ShrResult
$       \csname BIC@ShrDigit#1#2\endcsname
        #4!%
      }%
    \else
      \BIC@AfterFiFi{%
&       \expandafter\BIC@ShrResult\the\numexpr#1#2/2\relax
$       \expandafter\expandafter\expandafter\BIC@ShrResult
$       \csname BIC@ShrDigit#1#2\endcsname
        #4!%
      }%
    \fi
  \else
    \ifodd#1#2 %
      \BIC@AfterFiFi{%
&       \expandafter\BIC@@@@Shr\the\numexpr(#1#2-1)/2\relax1%
$       \expandafter\expandafter\expandafter\BIC@@@@Shr
$       \csname BIC@ShrDigit#1#2\endcsname
        #3!#4!%
      }%
    \else
      \BIC@AfterFiFi{%
&       \expandafter\BIC@@@@Shr\the\numexpr#1#2/2\relax0%
$       \expandafter\expandafter\expandafter\BIC@@@@Shr
$       \csname BIC@ShrDigit#1#2\endcsname
        #3!#4!%
      }%
    \fi
  \BIC@Fi
}
%    \end{macrocode}
%    \end{macro}
%    \begin{macro}{\BIC@ShrResult}
%    \begin{macrocode}
& \def\BIC@ShrResult#1#2!{ #2#1}%
$ \def\BIC@ShrResult#1#2#3!{ #3#1}%
%    \end{macrocode}
%    \end{macro}
%    \begin{macro}{\BIC@@@@Shr}
%    |#1|: new digit\\
%    |#2|: carry\\
%    |#3|: remaining number\\
%    |#4|: result
%    \begin{macrocode}
\def\BIC@@@@Shr#1#2#3!#4!{%
  \BIC@@@Shr#2#3!#4#1!%
}
%    \end{macrocode}
%    \end{macro}
%    \begin{macro}{\BIC@ShrDigit[00-19]}
%    \begin{macrocode}
$ \def\BIC@Temp#1#2#3#4{%
$   \expandafter\def\csname BIC@ShrDigit#1#2\endcsname{#3#4}%
$ }%
$ \BIC@Temp 0000%
$ \BIC@Temp 0101%
$ \BIC@Temp 0210%
$ \BIC@Temp 0311%
$ \BIC@Temp 0420%
$ \BIC@Temp 0521%
$ \BIC@Temp 0630%
$ \BIC@Temp 0731%
$ \BIC@Temp 0840%
$ \BIC@Temp 0941%
$ \BIC@Temp 1050%
$ \BIC@Temp 1151%
$ \BIC@Temp 1260%
$ \BIC@Temp 1361%
$ \BIC@Temp 1470%
$ \BIC@Temp 1571%
$ \BIC@Temp 1680%
$ \BIC@Temp 1781%
$ \BIC@Temp 1890%
$ \BIC@Temp 1991%
%    \end{macrocode}
%    \end{macro}
%
% \subsection{\cs{BIC@Tim}}
%
%    \begin{macro}{\BIC@Tim}
%    Macro \cs{BIC@Tim} implements ``Number \emph{tim}es digit''.\\
%    |#1|: plain number without sign\\
%    |#2|: digit
\def\BIC@Tim#1!#2{%
  \romannumeral0%
  \ifcase#2 % 0
    \BIC@AfterFi{ 0}%
  \or % 1
    \BIC@AfterFi{ #1}%
  \or % 2
    \BIC@AfterFi{%
      \BIC@Shl#1!%
    }%
  \else % 3-9
    \BIC@AfterFi{%
      \BIC@@Tim#1!!#2%
    }%
  \BIC@Fi
}
%    \begin{macrocode}
%    \end{macrocode}
%    \end{macro}
%    \begin{macro}{\BIC@@Tim}
%    |#1#2|: number\\
%    |#3|: reverted number
%    \begin{macrocode}
\def\BIC@@Tim#1#2!{%
  \ifx\\#2\\%
    \BIC@AfterFi{%
      \BIC@ProcessTim0!#1%
    }%
  \else
    \BIC@AfterFi{%
      \BIC@@Tim#2!#1%
    }%
  \BIC@Fi
}
%    \end{macrocode}
%    \end{macro}
%    \begin{macro}{\BIC@ProcessTim}
%    |#1|: carry\\
%    |#2|: result\\
%    |#3#4|: reverted number\\
%    |#5|: digit
%    \begin{macrocode}
\def\BIC@ProcessTim#1#2!#3#4!#5{%
  \ifx\\#4\\%
    \BIC@AfterFi{%
      \expandafter\BIC@Space
&     \the\numexpr#3*#5+#1\relax
$     \romannumeral0\BIC@TimDigit#3#5#1%
      #2%
    }%
  \else
    \BIC@AfterFi{%
      \expandafter\BIC@@ProcessTim
&     \the\numexpr#3*#5+#1%
$     \romannumeral0\BIC@TimDigit#3#5#1%
      !#2!#4!#5%
    }%
  \BIC@Fi
}
%    \end{macrocode}
%    \end{macro}
%    \begin{macro}{\BIC@@ProcessTim}
%    |#1#2|: carry?, new digit\\
%    |#3|: new number\\
%    |#4|: old number\\
%    |#5|: digit
%    \begin{macrocode}
\def\BIC@@ProcessTim#1#2!{%
  \ifx\\#2\\%
    \BIC@AfterFi{%
      \BIC@ProcessTim0#1%
    }%
  \else
    \BIC@AfterFi{%
      \BIC@ProcessTim#1#2%
    }%
  \BIC@Fi
}
%    \end{macrocode}
%    \end{macro}
%    \begin{macro}{\BIC@TimDigit}
%    |#1|: digit 0--9\\
%    |#2|: digit 3--9\\
%    |#3|: carry 0--9
%    \begin{macrocode}
$ \def\BIC@TimDigit#1#2#3{%
$   \ifcase#1 % 0
$     \BIC@AfterFi{ #3}%
$   \or % 1
$     \BIC@AfterFi{%
$       \expandafter\BIC@Space
$       \number\csname BIC@AddCarry#2\endcsname#3 %
$     }%
$   \else
$     \ifcase#3 %
$       \BIC@AfterFiFi{%
$         \expandafter\BIC@Space
$         \number\csname BIC@MulDigit#2\endcsname#1 %
$       }%
$     \else
$       \BIC@AfterFiFi{%
$         \expandafter\BIC@Space
$         \romannumeral0%
$         \expandafter\BIC@AddXY
$         \number\csname BIC@MulDigit#2\endcsname#1!%
$         #3!!!%
$       }%
$     \fi
$   \BIC@Fi
$ }%
%    \end{macrocode}
%    \end{macro}
%    \begin{macro}{\BIC@MulDigit[3-9]}
%    \begin{macrocode}
$ \def\BIC@Temp#1#2{%
$   \expandafter\def\csname BIC@MulDigit#1\endcsname##1{%
$     \ifcase##1 0%
$     \or ##1%
$     \or #2%
$?    \else\BigIntCalcError:ThisCannotHappen%
$     \fi
$   }%
$ }%
$ \BIC@Temp 3{6\or9\or12\or15\or18\or21\or24\or27}%
$ \BIC@Temp 4{8\or12\or16\or20\or24\or28\or32\or36}%
$ \BIC@Temp 5{10\or15\or20\or25\or30\or35\or40\or45}%
$ \BIC@Temp 6{12\or18\or24\or30\or36\or42\or48\or54}%
$ \BIC@Temp 7{14\or21\or28\or35\or42\or49\or56\or63}%
$ \BIC@Temp 8{16\or24\or32\or40\or48\or56\or64\or72}%
$ \BIC@Temp 9{18\or27\or36\or45\or54\or63\or72\or81}%
%    \end{macrocode}
%    \end{macro}
%
% \subsection{\op{Mul}}
%
%    \begin{macro}{\bigintcalcMul}
%    \begin{macrocode}
\def\bigintcalcMul#1#2{%
  \romannumeral0%
  \expandafter\expandafter\expandafter\BIC@Mul
  \bigintcalcNum{#1}!{#2}%
}
%    \end{macrocode}
%    \end{macro}
%    \begin{macro}{\BIC@Mul}
%    \begin{macrocode}
\def\BIC@Mul#1!#2{%
  \expandafter\expandafter\expandafter\BIC@MulSwitch
  \bigintcalcNum{#2}!#1!%
}
%    \end{macrocode}
%    \end{macro}
%    \begin{macro}{\BIC@MulSwitch}
%    Decision table for \cs{BIC@MulSwitch}.
%    \begin{quote}
%    \begin{tabular}[t]{@{}|l|l|l|l|l|@{}}
%      \hline
%      $x=0$ &\multicolumn{3}{l}{}    &$0$\\
%      \hline
%      $x>0$ & $y=0$ &\multicolumn2l{}&$0$\\
%      \cline{2-5}
%            & $y>0$ & $ x> y$ & $+$ & $\opMul( x, y)$\\
%      \cline{3-3}\cline{5-5}
%            &       & else    &     & $\opMul( y, x)$\\
%      \cline{2-5}
%            & $y<0$ & $ x>-y$ & $-$ & $\opMul( x,-y)$\\
%      \cline{3-3}\cline{5-5}
%            &       & else    &     & $\opMul(-y, x)$\\
%      \hline
%      $x<0$ & $y=0$ &\multicolumn2l{}&$0$\\
%      \cline{2-5}
%            & $y>0$ & $-x> y$ & $-$ & $\opMul(-x, y)$\\
%      \cline{3-3}\cline{5-5}
%            &       & else    &     & $\opMul( y,-x)$\\
%      \cline{2-5}
%            & $y<0$ & $-x>-y$ & $+$ & $\opMul(-x,-y)$\\
%      \cline{3-3}\cline{5-5}
%            &       & else    &     & $\opMul(-y,-x)$\\
%      \hline
%    \end{tabular}
%    \end{quote}
%    \begin{macrocode}
\def\BIC@MulSwitch#1#2!#3#4!{%
  \ifcase\BIC@Sgn#1#2! % x = 0
    \BIC@AfterFi{ 0}%
  \or % x > 0
    \ifcase\BIC@Sgn#3#4! % y = 0
      \BIC@AfterFiFi{ 0}%
    \or % y > 0
      \ifnum\BIC@PosCmp#1#2!#3#4!=1 % x > y
        \BIC@AfterFiFiFi{%
          \BIC@ProcessMul0!#1#2!#3#4!%
        }%
      \else % x <= y
        \BIC@AfterFiFiFi{%
          \BIC@ProcessMul0!#3#4!#1#2!%
        }%
      \fi
    \else % y < 0
      \expandafter-\romannumeral0%
      \ifnum\BIC@PosCmp#1#2!#4!=1 % x > -y
        \BIC@AfterFiFiFi{%
          \BIC@ProcessMul0!#1#2!#4!%
        }%
      \else % x <= -y
        \BIC@AfterFiFiFi{%
          \BIC@ProcessMul0!#4!#1#2!%
        }%
      \fi
    \fi
  \else % x < 0
    \ifcase\BIC@Sgn#3#4! % y = 0
      \BIC@AfterFiFi{ 0}%
    \or % y > 0
      \expandafter-\romannumeral0%
      \ifnum\BIC@PosCmp#2!#3#4!=1 % -x > y
        \BIC@AfterFiFiFi{%
          \BIC@ProcessMul0!#2!#3#4!%
        }%
      \else % -x <= y
        \BIC@AfterFiFiFi{%
          \BIC@ProcessMul0!#3#4!#2!%
        }%
      \fi
    \else % y < 0
      \ifnum\BIC@PosCmp#2!#4!=1 % -x > -y
        \BIC@AfterFiFiFi{%
          \BIC@ProcessMul0!#2!#4!%
        }%
      \else % -x <= -y
        \BIC@AfterFiFiFi{%
          \BIC@ProcessMul0!#4!#2!%
        }%
      \fi
    \fi
  \BIC@Fi
}
%    \end{macrocode}
%    \end{macro}
%    \begin{macro}{\BigIntCalcMul}
%    \begin{macrocode}
\def\BigIntCalcMul#1!#2!{%
  \romannumeral0%
  \BIC@ProcessMul0!#1!#2!%
}
%    \end{macrocode}
%    \end{macro}
%    \begin{macro}{\BIC@ProcessMul}
%    |#1|: result\\
%    |#2|: number $x$\\
%    |#3#4|: number $y$
%    \begin{macrocode}
\def\BIC@ProcessMul#1!#2!#3#4!{%
  \ifx\\#4\\%
    \BIC@AfterFi{%
      \expandafter\expandafter\expandafter\BIC@Space
      \bigintcalcAdd{\BIC@Tim#2!#3}{#10}%
    }%
  \else
    \BIC@AfterFi{%
      \expandafter\expandafter\expandafter\BIC@ProcessMul
      \bigintcalcAdd{\BIC@Tim#2!#3}{#10}!#2!#4!%
    }%
  \BIC@Fi
}
%    \end{macrocode}
%    \end{macro}
%
% \subsection{\op{Sqr}}
%
%    \begin{macro}{\bigintcalcSqr}
%    \begin{macrocode}
\def\bigintcalcSqr#1{%
  \romannumeral0%
  \expandafter\expandafter\expandafter\BIC@Sqr
  \bigintcalcNum{#1}!%
}
%    \end{macrocode}
%    \end{macro}
%    \begin{macro}{\BIC@Sqr}
%    \begin{macrocode}
\def\BIC@Sqr#1{%
   \ifx#1-%
     \expandafter\BIC@@Sqr
   \else
     \expandafter\BIC@@Sqr\expandafter#1%
   \fi
}
%    \end{macrocode}
%    \end{macro}
%    \begin{macro}{\BIC@@Sqr}
%    \begin{macrocode}
\def\BIC@@Sqr#1!{%
  \BIC@ProcessMul0!#1!#1!%
}
%    \end{macrocode}
%    \end{macro}
%
% \subsection{\op{Fac}}
%
%    \begin{macro}{\bigintcalcFac}
%    \begin{macrocode}
\def\bigintcalcFac#1{%
  \romannumeral0%
  \expandafter\expandafter\expandafter\BIC@Fac
  \bigintcalcNum{#1}!%
}
%    \end{macrocode}
%    \end{macro}
%    \begin{macro}{\BIC@Fac}
%    \begin{macrocode}
\def\BIC@Fac#1#2!{%
  \ifx#1-%
    \BIC@AfterFi{ 0\BigIntCalcError:FacNegative}%
  \else
    \ifnum\BIC@PosCmp#1#2!13!<0 %
      \ifcase#1#2 %
         \BIC@AfterFiFiFi{ 1}% 0!
      \or\BIC@AfterFiFiFi{ 1}% 1!
      \or\BIC@AfterFiFiFi{ 2}% 2!
      \or\BIC@AfterFiFiFi{ 6}% 3!
      \or\BIC@AfterFiFiFi{ 24}% 4!
      \or\BIC@AfterFiFiFi{ 120}% 5!
      \or\BIC@AfterFiFiFi{ 720}% 6!
      \or\BIC@AfterFiFiFi{ 5040}% 7!
      \or\BIC@AfterFiFiFi{ 40320}% 8!
      \or\BIC@AfterFiFiFi{ 362880}% 9!
      \or\BIC@AfterFiFiFi{ 3628800}% 10!
      \or\BIC@AfterFiFiFi{ 39916800}% 11!
      \or\BIC@AfterFiFiFi{ 479001600}% 12!
?     \else\BigIntCalcError:ThisCannotHappen%
      \fi
    \else
      \BIC@AfterFiFi{%
        \BIC@ProcessFac#1#2!479001600!%
      }%
    \fi
  \BIC@Fi
}
%    \end{macrocode}
%    \end{macro}
%    \begin{macro}{\BIC@ProcessFac}
%    |#1|: $n$\\
%    |#2|: result
%    \begin{macrocode}
\def\BIC@ProcessFac#1!#2!{%
  \ifnum\BIC@PosCmp#1!12!=0 %
    \BIC@AfterFi{ #2}%
  \else
    \BIC@AfterFi{%
      \expandafter\BIC@@ProcessFac
      \romannumeral0\BIC@ProcessMul0!#2!#1!%
      !#1!%
    }%
  \BIC@Fi
}
%    \end{macrocode}
%    \end{macro}
%    \begin{macro}{\BIC@@ProcessFac}
%    |#1|: result\\
%    |#2|: $n$
%    \begin{macrocode}
\def\BIC@@ProcessFac#1!#2!{%
  \expandafter\BIC@ProcessFac
  \romannumeral0\BIC@Dec#2!{}%
  !#1!%
}
%    \end{macrocode}
%    \end{macro}
%
% \subsection{\op{Pow}}
%
%    \begin{macro}{\bigintcalcPow}
%    |#1|: basis\\
%    |#2|: power
%    \begin{macrocode}
\def\bigintcalcPow#1{%
  \romannumeral0%
  \expandafter\expandafter\expandafter\BIC@Pow
  \bigintcalcNum{#1}!%
}
%    \end{macrocode}
%    \end{macro}
%    \begin{macro}{\BIC@Pow}
%    |#1|: basis\\
%    |#2|: power
%    \begin{macrocode}
\def\BIC@Pow#1!#2{%
  \expandafter\expandafter\expandafter\BIC@PowSwitch
  \bigintcalcNum{#2}!#1!%
}
%    \end{macrocode}
%    \end{macro}
%    \begin{macro}{\BIC@PowSwitch}
%    |#1#2|: power $y$\\
%    |#3#4|: basis $x$\\
%    Decision table for \cs{BIC@PowSwitch}.
%    \begin{quote}
%    \def\M#1{\multicolumn{#1}{l}{}}%
%    \begin{tabular}[t]{@{}|l|l|l|l|@{}}
%    \hline
%    $y=0$ & \M{2}                        & $1$\\
%    \hline
%    $y=1$ & \M{2}                        & $x$\\
%    \hline
%    $y=2$ & $x<0$          & \M{1}       & $\opMul(-x,-x)$\\
%    \cline{2-4}
%          & else           & \M{1}       & $\opMul(x,x)$\\
%    \hline
%    $y<0$ & $x=0$          & \M{1}       & DivisionByZero\\
%    \cline{2-4}
%          & $x=1$          & \M{1}       & $1$\\
%    \cline{2-4}
%          & $x=-1$         & $\opODD(y)$ & $-1$\\
%    \cline{3-4}
%          &                & else        & $1$\\
%    \cline{2-4}
%          & else ($\string|x\string|>1$) & \M{1}       & $0$\\
%    \hline
%    $y>2$ & $x=0$          & \M{1}       & $0$\\
%    \cline{2-4}
%          & $x=1$          & \M{1}       & $1$\\
%    \cline{2-4}
%          & $x=-1$         & $\opODD(y)$ & $-1$\\
%    \cline{3-4}
%          &                & else        & $1$\\
%    \cline{2-4}
%          & $x<-1$ ($x<0$) & $\opODD(y)$ & $-\opPow(-x,y)$\\
%    \cline{3-4}
%          &                & else        & $\opPow(-x,y)$\\
%    \cline{2-4}
%          & else ($x>1$)   & \M{1}       & $\opPow(x,y)$\\
%    \hline
%    \end{tabular}
%    \end{quote}
%    \begin{macrocode}
\def\BIC@PowSwitch#1#2!#3#4!{%
  \ifcase\ifx\\#2\\%
           \ifx#100 % y = 0
           \else\ifx#111 % y = 1
           \else\ifx#122 % y = 2
           \else4 % y > 2
           \fi\fi\fi
         \else
           \ifx#1-3 % y < 0
           \else4 % y > 2
           \fi
         \fi
    \BIC@AfterFi{ 1}% y = 0
  \or % y = 1
    \BIC@AfterFi{ #3#4}%
  \or % y = 2
    \ifx#3-% x < 0
      \BIC@AfterFiFi{%
        \BIC@ProcessMul0!#4!#4!%
      }%
    \else % x >= 0
      \BIC@AfterFiFi{%
        \BIC@ProcessMul0!#3#4!#3#4!%
      }%
    \fi
  \or % y < 0
    \ifcase\ifx\\#4\\%
             \ifx#300 % x = 0
             \else\ifx#311 % x = 1
             \else3 % x > 1
             \fi\fi
           \else
             \ifcase\BIC@MinusOne#3#4! %
               3 % |x| > 1
             \or
               2 % x = -1
?            \else\BigIntCalcError:ThisCannotHappen%
             \fi
           \fi
      \BIC@AfterFiFi{ 0\BigIntCalcError:DivisionByZero}% x = 0
    \or % x = 1
      \BIC@AfterFiFi{ 1}% x = 1
    \or % x = -1
      \ifcase\BIC@ModTwo#2! % even(y)
        \BIC@AfterFiFiFi{ 1}%
      \or % odd(y)
        \BIC@AfterFiFiFi{ -1}%
?     \else\BigIntCalcError:ThisCannotHappen%
      \fi
    \or % |x| > 1
      \BIC@AfterFiFi{ 0}%
?   \else\BigIntCalcError:ThisCannotHappen%
    \fi
  \or % y > 2
    \ifcase\ifx\\#4\\%
             \ifx#300 % x = 0
             \else\ifx#311 % x = 1
             \else4 % x > 1
             \fi\fi
           \else
             \ifx#3-%
               \ifcase\BIC@MinusOne#3#4! %
                 3 % x < -1
               \else
                 2 % x = -1
               \fi
             \else
               4 % x > 1
             \fi
           \fi
      \BIC@AfterFiFi{ 0}% x = 0
    \or % x = 1
      \BIC@AfterFiFi{ 1}% x = 1
    \or % x = -1
      \ifcase\BIC@ModTwo#1#2! % even(y)
        \BIC@AfterFiFiFi{ 1}%
      \or % odd(y)
        \BIC@AfterFiFiFi{ -1}%
?     \else\BigIntCalcError:ThisCannotHappen%
      \fi
    \or % x < -1
      \ifcase\BIC@ModTwo#1#2! % even(y)
        \BIC@AfterFiFiFi{%
          \BIC@PowRec#4!#1#2!1!%
        }%
      \or % odd(y)
        \expandafter-\romannumeral0%
        \BIC@AfterFiFiFi{%
          \BIC@PowRec#4!#1#2!1!%
        }%
?     \else\BigIntCalcError:ThisCannotHappen%
      \fi
    \or % x > 1
      \BIC@AfterFiFi{%
        \BIC@PowRec#3#4!#1#2!1!%
      }%
?   \else\BigIntCalcError:ThisCannotHappen%
    \fi
? \else\BigIntCalcError:ThisCannotHappen%
  \BIC@Fi
}
%    \end{macrocode}
%    \end{macro}
%
% \subsubsection{Help macros}
%
%    \begin{macro}{\BIC@ModTwo}
%    Macro \cs{BIC@ModTwo} expects a number without sign
%    and returns digit |1| or |0| if the number is odd or even.
%    \begin{macrocode}
\def\BIC@ModTwo#1#2!{%
  \ifx\\#2\\%
    \ifodd#1 %
      \BIC@AfterFiFi1%
    \else
      \BIC@AfterFiFi0%
    \fi
  \else
    \BIC@AfterFi{%
      \BIC@ModTwo#2!%
    }%
  \BIC@Fi
}
%    \end{macrocode}
%    \end{macro}
%
%    \begin{macro}{\BIC@MinusOne}
%    Macro \cs{BIC@MinusOne} expects a number and returns
%    digit |1| if the number equals minus one and returns |0| otherwise.
%    \begin{macrocode}
\def\BIC@MinusOne#1#2!{%
  \ifx#1-%
    \BIC@@MinusOne#2!%
  \else
    0%
  \fi
}
%    \end{macrocode}
%    \end{macro}
%    \begin{macro}{\BIC@@MinusOne}
%    \begin{macrocode}
\def\BIC@@MinusOne#1#2!{%
  \ifx#11%
    \ifx\\#2\\%
      1%
    \else
      0%
    \fi
  \else
    0%
  \fi
}
%    \end{macrocode}
%    \end{macro}
%
% \subsubsection{Recursive calculation}
%
%    \begin{macro}{\BIC@PowRec}
%\begin{quote}
%\begin{verbatim}
%Pow(x, y) {
%  PowRec(x, y, 1)
%}
%PowRec(x, y, r) {
%  if y == 1 then
%    return r
%  else
%    ifodd y then
%      return PowRec(x*x, y div 2, r*x) % y div 2 = (y-1)/2
%    else
%      return PowRec(x*x, y div 2, r)
%    fi
%  fi
%}
%\end{verbatim}
%\end{quote}
%    |#1|: $x$ (basis)\\
%    |#2#3|: $y$ (power)\\
%    |#4|: $r$ (result)
%    \begin{macrocode}
\def\BIC@PowRec#1!#2#3!#4!{%
  \ifcase\ifx#21\ifx\\#3\\0 \else1 \fi\else1 \fi % y = 1
    \ifnum\BIC@PosCmp#1!#4!=1 % x > r
      \BIC@AfterFiFi{%
        \BIC@ProcessMul0!#1!#4!%
      }%
    \else
      \BIC@AfterFiFi{%
        \BIC@ProcessMul0!#4!#1!%
      }%
    \fi
  \or
    \ifcase\BIC@ModTwo#2#3! % even(y)
      \BIC@AfterFiFi{%
        \expandafter\BIC@@PowRec\romannumeral0%
        \BIC@@Shr#2#3!%
        !#1!#4!%
      }%
    \or % odd(y)
      \ifnum\BIC@PosCmp#1!#4!=1 % x > r
        \BIC@AfterFiFiFi{%
          \expandafter\BIC@@@PowRec\romannumeral0%
          \BIC@ProcessMul0!#1!#4!%
          !#1!#2#3!%
        }%
      \else
        \BIC@AfterFiFiFi{%
          \expandafter\BIC@@@PowRec\romannumeral0%
          \BIC@ProcessMul0!#1!#4!%
          !#1!#2#3!%
        }%
      \fi
?   \else\BigIntCalcError:ThisCannotHappen%
    \fi
? \else\BigIntCalcError:ThisCannotHappen%
  \BIC@Fi
}
%    \end{macrocode}
%    \end{macro}
%    \begin{macro}{\BIC@@PowRec}
%    |#1|: $y/2$\\
%    |#2|: $x$\\
%    |#3|: new $r$ ($r$ or $r*x$)
%    \begin{macrocode}
\def\BIC@@PowRec#1!#2!#3!{%
  \expandafter\BIC@PowRec\romannumeral0%
  \BIC@ProcessMul0!#2!#2!%
  !#1!#3!%
}
%    \end{macrocode}
%    \end{macro}
%    \begin{macro}{\BIC@@@PowRec}
%    |#1|: $r*x$
%    |#2|: $x$
%    |#3|: $y$
%    \begin{macrocode}
\def\BIC@@@PowRec#1!#2!#3!{%
  \expandafter\BIC@@PowRec\romannumeral0%
  \BIC@@Shr#3!%
  !#2!#1!%
}
%    \end{macrocode}
%    \end{macro}
%
% \subsection{\op{Div}}
%
%    \begin{macro}{\bigintcalcDiv}
%    |#1|: $x$\\
%    |#2|: $y$ (divisor)
%    \begin{macrocode}
\def\bigintcalcDiv#1{%
  \romannumeral0%
  \expandafter\expandafter\expandafter\BIC@Div
  \bigintcalcNum{#1}!%
}
%    \end{macrocode}
%    \end{macro}
%    \begin{macro}{\BIC@Div}
%    |#1|: $x$\\
%    |#2|: $y$
%    \begin{macrocode}
\def\BIC@Div#1!#2{%
  \expandafter\expandafter\expandafter\BIC@DivSwitchSign
  \bigintcalcNum{#2}!#1!%
}
%    \end{macrocode}
%    \end{macro}
%    \begin{macro}{\BigIntCalcDiv}
%    \begin{macrocode}
\def\BigIntCalcDiv#1!#2!{%
  \romannumeral0%
  \BIC@DivSwitchSign#2!#1!%
}
%    \end{macrocode}
%    \end{macro}
%    \begin{macro}{\BIC@DivSwitchSign}
%    Decision table for \cs{BIC@DivSwitchSign}.
%    \begin{quote}
%    \begin{tabular}{@{}|l|l|l|@{}}
%    \hline
%    $y=0$ & \multicolumn{1}{l}{} & DivisionByZero\\
%    \hline
%    $y>0$ & $x=0$ & $0$\\
%    \cline{2-3}
%          & $x>0$ & DivSwitch$(+,x,y)$\\
%    \cline{2-3}
%          & $x<0$ & DivSwitch$(-,-x,y)$\\
%    \hline
%    $y<0$ & $x=0$ & $0$\\
%    \cline{2-3}
%          & $x>0$ & DivSwitch$(-,x,-y)$\\
%    \cline{2-3}
%          & $x<0$ & DivSwitch$(+,-x,-y)$\\
%    \hline
%    \end{tabular}
%    \end{quote}
%    |#1|: $y$ (divisor)\\
%    |#2|: $x$
%    \begin{macrocode}
\def\BIC@DivSwitchSign#1#2!#3#4!{%
  \ifcase\BIC@Sgn#1#2! % y = 0
    \BIC@AfterFi{ 0\BigIntCalcError:DivisionByZero}%
  \or % y > 0
    \ifcase\BIC@Sgn#3#4! % x = 0
      \BIC@AfterFiFi{ 0}%
    \or % x > 0
      \BIC@AfterFiFi{%
        \BIC@DivSwitch{}#3#4!#1#2!%
      }%
    \else % x < 0
      \BIC@AfterFiFi{%
        \BIC@DivSwitch-#4!#1#2!%
      }%
    \fi
  \else % y < 0
    \ifcase\BIC@Sgn#3#4! % x = 0
      \BIC@AfterFiFi{ 0}%
    \or % x > 0
      \BIC@AfterFiFi{%
        \BIC@DivSwitch-#3#4!#2!%
      }%
    \else % x < 0
      \BIC@AfterFiFi{%
        \BIC@DivSwitch{}#4!#2!%
      }%
    \fi
  \BIC@Fi
}
%    \end{macrocode}
%    \end{macro}
%    \begin{macro}{\BIC@DivSwitch}
%    Decision table for \cs{BIC@DivSwitch}.
%    \begin{quote}
%    \begin{tabular}{@{}|l|l|l|@{}}
%    \hline
%    $y=x$ & \multicolumn{1}{l}{} & sign $1$\\
%    \hline
%    $y>x$ & \multicolumn{1}{l}{} & $0$\\
%    \hline
%    $y<x$ & $y=1$                & sign $x$\\
%    \cline{2-3}
%          & $y=2$                & sign Shr$(x)$\\
%    \cline{2-3}
%          & $y=4$                & sign Shr(Shr$(x)$)\\
%    \cline{2-3}
%          & else                 & sign ProcessDiv$(x,y)$\\
%    \hline
%    \end{tabular}
%    \end{quote}
%    |#1|: sign\\
%    |#2|: $x$\\
%    |#3#4|: $y$ ($y\ne 0$)
%    \begin{macrocode}
\def\BIC@DivSwitch#1#2!#3#4!{%
  \ifcase\BIC@PosCmp#3#4!#2!% y = x
    \BIC@AfterFi{ #11}%
  \or % y > x
    \BIC@AfterFi{ 0}%
  \else % y < x
    \ifx\\#1\\%
    \else
      \expandafter-\romannumeral0%
    \fi
    \ifcase\ifx\\#4\\%
             \ifx#310 % y = 1
             \else\ifx#321 % y = 2
             \else\ifx#342 % y = 4
             \else3 % y > 2
             \fi\fi\fi
           \else
             3 % y > 2
           \fi
      \BIC@AfterFiFi{ #2}% y = 1
    \or % y = 2
      \BIC@AfterFiFi{%
        \BIC@@Shr#2!%
      }%
    \or % y = 4
      \BIC@AfterFiFi{%
        \expandafter\BIC@@Shr\romannumeral0%
          \BIC@@Shr#2!!%
      }%
    \or % y > 2
      \BIC@AfterFiFi{%
        \BIC@DivStartX#2!#3#4!!!%
      }%
?   \else\BigIntCalcError:ThisCannotHappen%
    \fi
  \BIC@Fi
}
%    \end{macrocode}
%    \end{macro}
%    \begin{macro}{\BIC@ProcessDiv}
%    |#1#2|: $x$\\
%    |#3#4|: $y$\\
%    |#5|: collect first digits of $x$\\
%    |#6|: corresponding digits of $y$
%    \begin{macrocode}
\def\BIC@DivStartX#1#2!#3#4!#5!#6!{%
  \ifx\\#4\\%
    \BIC@AfterFi{%
      \BIC@DivStartYii#6#3#4!{#5#1}#2=!%
    }%
  \else
    \BIC@AfterFi{%
      \BIC@DivStartX#2!#4!#5#1!#6#3!%
    }%
  \BIC@Fi
}
%    \end{macrocode}
%    \end{macro}
%    \begin{macro}{\BIC@DivStartYii}
%    |#1|: $y$\\
%    |#2|: $x$, |=|
%    \begin{macrocode}
\def\BIC@DivStartYii#1!{%
  \expandafter\BIC@DivStartYiv\romannumeral0%
  \BIC@Shl#1!%
  !#1!%
}
%    \end{macrocode}
%    \end{macro}
%    \begin{macro}{\BIC@DivStartYiv}
%    |#1|: $2y$\\
%    |#2|: $y$\\
%    |#3|: $x$, |=|
%    \begin{macrocode}
\def\BIC@DivStartYiv#1!{%
  \expandafter\BIC@DivStartYvi\romannumeral0%
  \BIC@Shl#1!%
  !#1!%
}
%    \end{macrocode}
%    \end{macro}
%    \begin{macro}{\BIC@DivStartYvi}
%    |#1|: $4y$\\
%    |#2|: $2y$\\
%    |#3|: $y$\\
%    |#4|: $x$, |=|
%    \begin{macrocode}
\def\BIC@DivStartYvi#1!#2!{%
  \expandafter\BIC@DivStartYviii\romannumeral0%
  \BIC@AddXY#1!#2!!!%
  !#1!#2!%
}
%    \end{macrocode}
%    \end{macro}
%    \begin{macro}{\BIC@DivStartYviii}
%    |#1|: $6y$\\
%    |#2|: $4y$\\
%    |#3|: $2y$\\
%    |#4|: $y$\\
%    |#5|: $x$, |=|
%    \begin{macrocode}
\def\BIC@DivStartYviii#1!#2!{%
  \expandafter\BIC@DivStart\romannumeral0%
  \BIC@Shl#2!%
  !#1!#2!%
}
%    \end{macrocode}
%    \end{macro}
%    \begin{macro}{\BIC@DivStart}
%    |#1|: $8y$\\
%    |#2|: $6y$\\
%    |#3|: $4y$\\
%    |#4|: $2y$\\
%    |#5|: $y$\\
%    |#6|: $x$, |=|
%    \begin{macrocode}
\def\BIC@DivStart#1!#2!#3!#4!#5!#6!{%
  \BIC@ProcessDiv#6!!#5!#4!#3!#2!#1!=%
}
%    \end{macrocode}
%    \end{macro}
%    \begin{macro}{\BIC@ProcessDiv}
%    |#1#2#3|: $x$, |=|\\
%    |#4|: result\\
%    |#5|: $y$\\
%    |#6|: $2y$\\
%    |#7|: $4y$\\
%    |#8|: $6y$\\
%    |#9|: $8y$
%    \begin{macrocode}
\def\BIC@ProcessDiv#1#2#3!#4!#5!{%
  \ifcase\BIC@PosCmp#5!#1!% y = #1
    \ifx#2=%
      \BIC@AfterFiFi{\BIC@DivCleanup{#41}}%
    \else
      \BIC@AfterFiFi{%
        \BIC@ProcessDiv#2#3!#41!#5!%
      }%
    \fi
  \or % y > #1
    \ifx#2=%
      \BIC@AfterFiFi{\BIC@DivCleanup{#40}}%
    \else
      \ifx\\#4\\%
        \BIC@AfterFiFiFi{%
          \BIC@ProcessDiv{#1#2}#3!!#5!%
        }%
      \else
        \BIC@AfterFiFiFi{%
          \BIC@ProcessDiv{#1#2}#3!#40!#5!%
        }%
      \fi
    \fi
  \else % y < #1
    \BIC@AfterFi{%
      \BIC@@ProcessDiv{#1}#2#3!#4!#5!%
    }%
  \BIC@Fi
}
%    \end{macrocode}
%    \end{macro}
%    \begin{macro}{\BIC@DivCleanup}
%    |#1|: result\\
%    |#2|: garbage
%    \begin{macrocode}
\def\BIC@DivCleanup#1#2={ #1}%
%    \end{macrocode}
%    \end{macro}
%    \begin{macro}{\BIC@@ProcessDiv}
%    \begin{macrocode}
\def\BIC@@ProcessDiv#1#2#3!#4!#5!#6!#7!{%
  \ifcase\BIC@PosCmp#7!#1!% 4y = #1
    \ifx#2=%
      \BIC@AfterFiFi{\BIC@DivCleanup{#44}}%
    \else
     \BIC@AfterFiFi{%
       \BIC@ProcessDiv#2#3!#44!#5!#6!#7!%
     }%
    \fi
  \or % 4y > #1
    \ifcase\BIC@PosCmp#6!#1!% 2y = #1
      \ifx#2=%
        \BIC@AfterFiFiFi{\BIC@DivCleanup{#42}}%
      \else
        \BIC@AfterFiFiFi{%
          \BIC@ProcessDiv#2#3!#42!#5!#6!#7!%
        }%
      \fi
    \or % 2y > #1
      \ifx#2=%
        \BIC@AfterFiFiFi{\BIC@DivCleanup{#41}}%
      \else
        \BIC@AfterFiFiFi{%
          \BIC@DivSub#1!#5!#2#3!#41!#5!#6!#7!%
        }%
      \fi
    \else % 2y < #1
      \BIC@AfterFiFi{%
        \expandafter\BIC@ProcessDivII\romannumeral0%
        \BIC@SubXY#1!#6!!!%
        !#2#3!#4!#5!23%
        #6!#7!%
      }%
    \fi
  \else % 4y < #1
    \BIC@AfterFi{%
      \BIC@@@ProcessDiv{#1}#2#3!#4!#5!#6!#7!%
    }%
  \BIC@Fi
}
%    \end{macrocode}
%    \end{macro}
%    \begin{macro}{\BIC@DivSub}
%    Next token group: |#1|-|#2| and next digit |#3|.
%    \begin{macrocode}
\def\BIC@DivSub#1!#2!#3{%
  \expandafter\BIC@ProcessDiv\expandafter{%
    \romannumeral0%
    \BIC@SubXY#1!#2!!!%
    #3%
  }%
}
%    \end{macrocode}
%    \end{macro}
%    \begin{macro}{\BIC@ProcessDivII}
%    |#1|: $x'-2y$\\
%    |#2#3|: remaining $x$, |=|\\
%    |#4|: result\\
%    |#5|: $y$\\
%    |#6|: first possible result digit\\
%    |#7|: second possible result digit
%    \begin{macrocode}
\def\BIC@ProcessDivII#1!#2#3!#4!#5!#6#7{%
  \ifcase\BIC@PosCmp#5!#1!% y = #1
    \ifx#2=%
      \BIC@AfterFiFi{\BIC@DivCleanup{#4#7}}%
    \else
      \BIC@AfterFiFi{%
        \BIC@ProcessDiv#2#3!#4#7!#5!%
      }%
    \fi
  \or % y > #1
    \ifx#2=%
      \BIC@AfterFiFi{\BIC@DivCleanup{#4#6}}%
    \else
      \BIC@AfterFiFi{%
        \BIC@ProcessDiv{#1#2}#3!#4#6!#5!%
      }%
    \fi
  \else % y < #1
    \ifx#2=%
      \BIC@AfterFiFi{\BIC@DivCleanup{#4#7}}%
    \else
      \BIC@AfterFiFi{%
        \BIC@DivSub#1!#5!#2#3!#4#7!#5!%
      }%
    \fi
  \BIC@Fi
}
%    \end{macrocode}
%    \end{macro}
%    \begin{macro}{\BIC@ProcessDivIV}
%    |#1#2#3|: $x$, |=|, $x>4y$\\
%    |#4|: result\\
%    |#5|: $y$\\
%    |#6|: $2y$\\
%    |#7|: $4y$\\
%    |#8|: $6y$\\
%    |#9|: $8y$
%    \begin{macrocode}
\def\BIC@@@ProcessDiv#1#2#3!#4!#5!#6!#7!#8!#9!{%
  \ifcase\BIC@PosCmp#8!#1!% 6y = #1
    \ifx#2=%
      \BIC@AfterFiFi{\BIC@DivCleanup{#46}}%
    \else
      \BIC@AfterFiFi{%
        \BIC@ProcessDiv#2#3!#46!#5!#6!#7!#8!#9!%
      }%
    \fi
  \or % 6y > #1
    \BIC@AfterFi{%
      \expandafter\BIC@ProcessDivII\romannumeral0%
      \BIC@SubXY#1!#7!!!%
      !#2#3!#4!#5!45%
      #6!#7!#8!#9!%
    }%
  \else % 6y < #1
    \ifcase\BIC@PosCmp#9!#1!% 8y = #1
      \ifx#2=%
        \BIC@AfterFiFiFi{\BIC@DivCleanup{#48}}%
      \else
        \BIC@AfterFiFiFi{%
          \BIC@ProcessDiv#2#3!#48!#5!#6!#7!#8!#9!%
        }%
      \fi
    \or % 8y > #1
      \BIC@AfterFiFi{%
        \expandafter\BIC@ProcessDivII\romannumeral0%
        \BIC@SubXY#1!#8!!!%
        !#2#3!#4!#5!67%
        #6!#7!#8!#9!%
      }%
    \else % 8y < #1
      \BIC@AfterFiFi{%
        \expandafter\BIC@ProcessDivII\romannumeral0%
        \BIC@SubXY#1!#9!!!%
        !#2#3!#4!#5!89%
        #6!#7!#8!#9!%
      }%
    \fi
  \BIC@Fi
}
%    \end{macrocode}
%    \end{macro}
%
% \subsection{\op{Mod}}
%
%    \begin{macro}{\bigintcalcMod}
%    |#1|: $x$\\
%    |#2|: $y$
%    \begin{macrocode}
\def\bigintcalcMod#1{%
  \romannumeral0%
  \expandafter\expandafter\expandafter\BIC@Mod
  \bigintcalcNum{#1}!%
}
%    \end{macrocode}
%    \end{macro}
%    \begin{macro}{\BIC@Mod}
%    |#1|: $x$\\
%    |#2|: $y$
%    \begin{macrocode}
\def\BIC@Mod#1!#2{%
  \expandafter\expandafter\expandafter\BIC@ModSwitchSign
  \bigintcalcNum{#2}!#1!%
}
%    \end{macrocode}
%    \end{macro}
%    \begin{macro}{\BigIntCalcMod}
%    \begin{macrocode}
\def\BigIntCalcMod#1!#2!{%
  \romannumeral0%
  \BIC@ModSwitchSign#2!#1!%
}
%    \end{macrocode}
%    \end{macro}
%    \begin{macro}{\BIC@ModSwitchSign}
%    Decision table for \cs{BIC@ModSwitchSign}.
%    \begin{quote}
%    \begin{tabular}{@{}|l|l|l|@{}}
%    \hline
%    $y=0$ & \multicolumn{1}{l}{} & DivisionByZero\\
%    \hline
%    $y>0$ & $x=0$ & $0$\\
%    \cline{2-3}
%          & else  & ModSwitch$(+,x,y)$\\
%    \hline
%    $y<0$ & \multicolumn{1}{l}{} & ModSwitch$(-,-x,-y)$\\
%    \hline
%    \end{tabular}
%    \end{quote}
%    |#1#2|: $y$\\
%    |#3#4|: $x$
%    \begin{macrocode}
\def\BIC@ModSwitchSign#1#2!#3#4!{%
  \ifcase\ifx\\#2\\%
           \ifx#100 % y = 0
           \else1 % y > 0
           \fi
          \else
            \ifx#1-2 % y < 0
            \else1 % y > 0
            \fi
          \fi
    \BIC@AfterFi{ 0\BigIntCalcError:DivisionByZero}%
  \or % y > 0
    \ifcase\ifx\\#4\\\ifx#300 \else1 \fi\else1 \fi % x = 0
      \BIC@AfterFiFi{ 0}%
    \else
      \BIC@AfterFiFi{%
        \BIC@ModSwitch{}#3#4!#1#2!%
      }%
    \fi
  \else % y < 0
    \ifcase\ifx\\#4\\%
             \ifx#300 % x = 0
             \else1 % x > 0
             \fi
           \else
             \ifx#3-2 % x < 0
             \else1 % x > 0
             \fi
           \fi
      \BIC@AfterFiFi{ 0}%
    \or % x > 0
      \BIC@AfterFiFi{%
        \BIC@ModSwitch--#3#4!#2!%
      }%
    \else % x < 0
      \BIC@AfterFiFi{%
        \BIC@ModSwitch-#4!#2!%
      }%
    \fi
  \BIC@Fi
}
%    \end{macrocode}
%    \end{macro}
%    \begin{macro}{\BIC@ModSwitch}
%    Decision table for \cs{BIC@ModSwitch}.
%    \begin{quote}
%    \begin{tabular}{@{}|l|l|l|@{}}
%    \hline
%    $y=1$ & \multicolumn{1}{l}{} & $0$\\
%    \hline
%    $y=2$ & ifodd$(x)$ & sign $1$\\
%    \cline{2-3}
%          & else       & $0$\\
%    \hline
%    $y>2$ & $x<0$      &
%        $z\leftarrow x-(x/y)*y;\quad (z<0)\mathbin{?}z+y\mathbin{:}z$\\
%    \cline{2-3}
%          & $x>0$      & $x - (x/y) * y$\\
%    \hline
%    \end{tabular}
%    \end{quote}
%    |#1|: sign\\
%    |#2#3|: $x$\\
%    |#4#5|: $y$
%    \begin{macrocode}
\def\BIC@ModSwitch#1#2#3!#4#5!{%
  \ifcase\ifx\\#5\\%
           \ifx#410 % y = 1
           \else\ifx#421 % y = 2
           \else2 % y > 2
           \fi\fi
         \else2 % y > 2
         \fi
    \BIC@AfterFi{ 0}% y = 1
  \or % y = 2
    \ifcase\BIC@ModTwo#2#3! % even(x)
      \BIC@AfterFiFi{ 0}%
    \or % odd(x)
      \BIC@AfterFiFi{ #11}%
?   \else\BigIntCalcError:ThisCannotHappen%
    \fi
  \or % y > 2
    \ifx\\#1\\%
    \else
      \expandafter\BIC@Space\romannumeral0%
      \expandafter\BIC@ModMinus\romannumeral0%
    \fi
    \ifx#2-% x < 0
      \BIC@AfterFiFi{%
        \expandafter\expandafter\expandafter\BIC@ModX
        \bigintcalcSub{#2#3}{%
          \bigintcalcMul{#4#5}{\bigintcalcDiv{#2#3}{#4#5}}%
        }!#4#5!%
      }%
    \else % x > 0
      \BIC@AfterFiFi{%
        \expandafter\expandafter\expandafter\BIC@Space
        \bigintcalcSub{#2#3}{%
          \bigintcalcMul{#4#5}{\bigintcalcDiv{#2#3}{#4#5}}%
        }%
      }%
    \fi
? \else\BigIntCalcError:ThisCannotHappen%
  \BIC@Fi
}
%    \end{macrocode}
%    \end{macro}
%    \begin{macro}{\BIC@ModMinus}
%    \begin{macrocode}
\def\BIC@ModMinus#1{%
  \ifx#10%
    \BIC@AfterFi{ 0}%
  \else
    \BIC@AfterFi{ -#1}%
  \BIC@Fi
}
%    \end{macrocode}
%    \end{macro}
%    \begin{macro}{\BIC@ModX}
%    |#1#2|: $z$\\
%    |#3|: $x$
%    \begin{macrocode}
\def\BIC@ModX#1#2!#3!{%
  \ifx#1-% z < 0
    \BIC@AfterFi{%
      \expandafter\BIC@Space\romannumeral0%
      \BIC@SubXY#3!#2!!!%
    }%
  \else % z >= 0
    \BIC@AfterFi{ #1#2}%
  \BIC@Fi
}
%    \end{macrocode}
%    \end{macro}
%
%    \begin{macrocode}
\BIC@AtEnd%
%    \end{macrocode}
%
%    \begin{macrocode}
%</package>
%    \end{macrocode}
%
% \section{Test}
%
% \subsection{Catcode checks for loading}
%
%    \begin{macrocode}
%<*test1>
%    \end{macrocode}
%    \begin{macrocode}
\catcode`\{=1 %
\catcode`\}=2 %
\catcode`\#=6 %
\catcode`\@=11 %
\expandafter\ifx\csname count@\endcsname\relax
  \countdef\count@=255 %
\fi
\expandafter\ifx\csname @gobble\endcsname\relax
  \long\def\@gobble#1{}%
\fi
\expandafter\ifx\csname @firstofone\endcsname\relax
  \long\def\@firstofone#1{#1}%
\fi
\expandafter\ifx\csname loop\endcsname\relax
  \expandafter\@firstofone
\else
  \expandafter\@gobble
\fi
{%
  \def\loop#1\repeat{%
    \def\body{#1}%
    \iterate
  }%
  \def\iterate{%
    \body
      \let\next\iterate
    \else
      \let\next\relax
    \fi
    \next
  }%
  \let\repeat=\fi
}%
\def\RestoreCatcodes{}
\count@=0 %
\loop
  \edef\RestoreCatcodes{%
    \RestoreCatcodes
    \catcode\the\count@=\the\catcode\count@\relax
  }%
\ifnum\count@<255 %
  \advance\count@ 1 %
\repeat

\def\RangeCatcodeInvalid#1#2{%
  \count@=#1\relax
  \loop
    \catcode\count@=15 %
  \ifnum\count@<#2\relax
    \advance\count@ 1 %
  \repeat
}
\def\RangeCatcodeCheck#1#2#3{%
  \count@=#1\relax
  \loop
    \ifnum#3=\catcode\count@
    \else
      \errmessage{%
        Character \the\count@\space
        with wrong catcode \the\catcode\count@\space
        instead of \number#3%
      }%
    \fi
  \ifnum\count@<#2\relax
    \advance\count@ 1 %
  \repeat
}
\def\space{ }
\expandafter\ifx\csname LoadCommand\endcsname\relax
  \def\LoadCommand{\input bigintcalc.sty\relax}%
\fi
\def\Test{%
  \RangeCatcodeInvalid{0}{47}%
  \RangeCatcodeInvalid{58}{64}%
  \RangeCatcodeInvalid{91}{96}%
  \RangeCatcodeInvalid{123}{255}%
  \catcode`\@=12 %
  \catcode`\\=0 %
  \catcode`\%=14 %
  \LoadCommand
  \RangeCatcodeCheck{0}{36}{15}%
  \RangeCatcodeCheck{37}{37}{14}%
  \RangeCatcodeCheck{38}{47}{15}%
  \RangeCatcodeCheck{48}{57}{12}%
  \RangeCatcodeCheck{58}{63}{15}%
  \RangeCatcodeCheck{64}{64}{12}%
  \RangeCatcodeCheck{65}{90}{11}%
  \RangeCatcodeCheck{91}{91}{15}%
  \RangeCatcodeCheck{92}{92}{0}%
  \RangeCatcodeCheck{93}{96}{15}%
  \RangeCatcodeCheck{97}{122}{11}%
  \RangeCatcodeCheck{123}{255}{15}%
  \RestoreCatcodes
}
\Test
\csname @@end\endcsname
\end
%    \end{macrocode}
%    \begin{macrocode}
%</test1>
%    \end{macrocode}
%
% \subsection{Macro tests}
%
% \subsubsection{Preamble with test macro definitions}
%
%    \begin{macrocode}
%<*test2>
\NeedsTeXFormat{LaTeX2e}
\nofiles
\documentclass{article}
%<noetex>\let\SavedNumexpr\numexpr
%<noetex>\let\numexpr\UNDEFINED
\makeatletter
\chardef\BIC@TestMode=1 %
\makeatother
\usepackage{bigintcalc}[2016/05/16]
%<noetex>\let\numexpr\SavedNumexpr
\usepackage{qstest}
\IncludeTests{*}
\LogTests{log}{*}{*}
\newcommand*{\TestSpaceAtEnd}[1]{%
%<noetex>  \let\SavedNumexpr\numexpr
%<noetex>  \let\numexpr\UNDEFINED
  \edef\resultA{#1}%
  \edef\resultB{#1 }%
%<noetex>  \let\numexpr\SavedNumexpr
  \Expect*{\resultA\space}*{\resultB}%
}
\newcommand*{\TestResult}[2]{%
%<noetex>  \let\SavedNumexpr\numexpr
%<noetex>  \let\numexpr\UNDEFINED
  \edef\result{#1}%
%<noetex>  \let\numexpr\SavedNumexpr
  \Expect*{\result}{#2}%
}
\newcommand*{\TestResultTwoExpansions}[2]{%
%<*noetex>
  \begingroup
    \let\numexpr\UNDEFINED
    \expandafter\expandafter\expandafter
  \endgroup
%</noetex>
  \expandafter\expandafter\expandafter\Expect
  \expandafter\expandafter\expandafter{#1}{#2}%
}
\newcount\TestCount
%<etex>\newcommand*{\TestArg}[1]{\numexpr#1\relax}
%<noetex>\newcommand*{\TestArg}[1]{#1}
\newcommand*{\TestTeXDivide}[2]{%
  \TestCount=\TestArg{#1}\relax
  \divide\TestCount by \TestArg{#2}\relax
  \Expect*{\bigintcalcDiv{#1}{#2}}*{\the\TestCount}%
}
\newcommand*{\Test}[2]{%
  \TestResult{#1}{#2}%
  \TestResultTwoExpansions{#1}{#2}%
  \TestSpaceAtEnd{#1}%
}
\newcommand*{\TestExch}[2]{\Test{#2}{#1}}
\newcommand*{\TestInv}[2]{%
  \Test{\bigintcalcInv{#1}}{#2}%
}
\newcommand*{\TestAbs}[2]{%
  \Test{\bigintcalcAbs{#1}}{#2}%
}
\newcommand*{\TestSgn}[2]{%
  \Test{\bigintcalcSgn{#1}}{#2}%
}
\newcommand*{\TestMin}[3]{%
  \Test{\bigintcalcMin{#1}{#2}}{#3}%
}
\newcommand*{\TestMax}[3]{%
  \Test{\bigintcalcMax{#1}{#2}}{#3}%
}
\newcommand*{\TestCmp}[3]{%
  \Test{\bigintcalcCmp{#1}{#2}}{#3}%
}
\newcommand*{\TestOdd}[2]{%
  \Test{\bigintcalcOdd{#1}}{#2}%
  \edef\x{%
    \noexpand\Test{%
      \noexpand\BigIntCalcOdd
      \bigintcalcAbs{#1}!%
    }{#2}%
  }%
  \x
}
\newcommand*{\TestInc}[2]{%
  \Test{\bigintcalcInc{#1}}{#2}%
  \ifnum\bigintcalcSgn{#1}>-1 %
    \edef\x{%
      \noexpand\Test{%
        \noexpand\BigIntCalcInc\bigintcalcNum{#1}!%
      }{#2}%
    }%
    \x
  \fi
}
\newcommand*{\TestDec}[2]{%
  \Test{\bigintcalcDec{#1}}{#2}%
  \ifnum\bigintcalcSgn{#1}>0 %
    \edef\x{%
      \noexpand\Test{%
        \noexpand\BigIntCalcDec\bigintcalcNum{#1}!%
      }{#2}%
    }%
    \x
  \fi
}
\newcommand*{\TestAdd}[3]{%
  \Test{\bigintcalcAdd{#1}{#2}}{#3}%
  \ifnum\bigintcalcSgn{#1}>0 %
    \ifnum\bigintcalcSgn{#2}> 0 %
      \ifnum\bigintcalcCmp{#1}{#2}>0 %
        \edef\x{%
          \noexpand\Test{%
            \noexpand\BigIntCalcAdd
            \bigintcalcNum{#1}!\bigintcalcNum{#2}!%
          }{#3}%
        }%
        \x
      \else
        \edef\x{%
          \noexpand\Test{%
            \noexpand\BigIntCalcAdd
            \bigintcalcNum{#2}!\bigintcalcNum{#1}!%
          }{#3}%
        }%
        \x
      \fi
    \fi
  \fi
}
\newcommand*{\TestSub}[3]{%
  \Test{\bigintcalcSub{#1}{#2}}{#3}%
  \ifnum\bigintcalcSgn{#1}>0 %
    \ifnum\bigintcalcSgn{#2}> 0 %
      \ifnum\bigintcalcCmp{#1}{#2}>0 %
        \edef\x{%
          \noexpand\Test{%
            \noexpand\BigIntCalcSub
            \bigintcalcNum{#1}!\bigintcalcNum{#2}!%
          }{#3}%
        }%
        \x
      \fi
    \fi
  \fi
}
\newcommand*{\TestShl}[2]{%
  \Test{\bigintcalcShl{#1}}{#2}%
  \edef\x{%
    \noexpand\Test{%
      \noexpand\BigIntCalcShl\bigintcalcAbs{#1}!%
    }{\bigintcalcAbs{#2}}%
  }%
  \x
}
\newcommand*{\TestShr}[2]{%
  \Test{\bigintcalcShr{#1}}{#2}%
  \edef\x{%
    \noexpand\Test{%
      \noexpand\BigIntCalcShr\bigintcalcAbs{#1}!%
    }{\bigintcalcAbs{#2}}%
  }%
  \x
}
\newcommand*{\TestMul}[3]{%
  \Test{\bigintcalcMul{#1}{#2}}{#3}%
  \edef\x{%
    \noexpand\Test{%
      \noexpand\BigIntCalcMul
      \bigintcalcAbs{#1}!\bigintcalcAbs{#2}!%
    }{\bigintcalcAbs{#3}}%
  }%
  \x
}
\newcommand*{\TestSqr}[2]{%
  \Test{\bigintcalcSqr{#1}}{#2}%
}
\newcommand*{\TestFac}[2]{%
  \expandafter\TestExch\expandafter{%
    \the\numexpr#2%
  }{\bigintcalcFac{#1}}%
}
\newcommand*{\TestFacBig}[2]{%
  \Test{\bigintcalcFac{#1}}{#2}%
}
\newcommand*{\TestPow}[3]{%
  \Test{\bigintcalcPow{#1}{#2}}{#3}%
}
\newcommand*{\TestDiv}[3]{%
  \Test{\bigintcalcDiv{#1}{#2}}{#3}%
  \TestTeXDivide{#1}{#2}%
}
\newcommand*{\TestDivBig}[3]{%
  \Test{\bigintcalcDiv{#1}{#2}}{#3}%
  \edef\x{%
    \noexpand\Test{%
      \noexpand\BigIntCalcDiv\bigintcalcAbs{#1}!\bigintcalcAbs{#2}!%
    }{\bigintcalcAbs{#3}}%
  }%
}
\newcommand*{\TestMod}[3]{%
  \Test{\bigintcalcMod{#1}{#2}}{#3}%
  \ifcase\ifcase\bigintcalcSgn{#1} 0%
         \or
           \ifcase\bigintcalcSgn{#2} 1%
           \or 0%
           \else 1%
           \fi
         \else
           \ifcase\bigintcalcSgn{#2} 1%
           \or 1%
           \else 0%
           \fi
         \fi\relax
    \edef\x{%
      \noexpand\Test{%
        \noexpand\BigIntCalcMod
        \bigintcalcAbs{#1}!\bigintcalcAbs{#2}!%
      }{\bigintcalcAbs{#3}}%
    }%
    \x
  \fi
}
%    \end{macrocode}
%
% \subsubsection{Time}
%
%    \begin{macrocode}
\begingroup\expandafter\expandafter\expandafter\endgroup
\expandafter\ifx\csname pdfresettimer\endcsname\relax
\else
  \makeatletter
  \newcount\SummaryTime
  \newcount\TestTime
  \SummaryTime=\z@
  \newcommand*{\PrintTime}[2]{%
    \typeout{%
      [Time #1: \strip@pt\dimexpr\number#2sp\relax\space s]%
    }%
  }%
  \newcommand*{\StartTime}[1]{%
    \renewcommand*{\TimeDescription}{#1}%
    \pdfresettimer
  }%
  \newcommand*{\TimeDescription}{}%
  \newcommand*{\StopTime}{%
    \TestTime=\pdfelapsedtime
    \global\advance\SummaryTime\TestTime
    \PrintTime\TimeDescription\TestTime
  }%
  \let\saved@qstest\qstest
  \let\saved@endqstest\endqstest
  \def\qstest#1#2{%
    \saved@qstest{#1}{#2}%
    \StartTime{#1}%
  }%
  \def\endqstest{%
    \StopTime
    \saved@endqstest
  }%
  \AtEndDocument{%
    \PrintTime{summary}\SummaryTime
  }%
  \makeatother
\fi
%    \end{macrocode}
%
% \subsubsection{Test sets}
%
%    \begin{macrocode}
\makeatletter

\begin{qstest}{inv}{inv}%
  \TestInv{0}{0}%
  \TestInv{1}{-1}%
  \TestInv{-1}{1}%
  \TestInv{10}{-10}%
  \TestInv{-10}{10}%
  \TestInv{2147483647}{-2147483647}%
  \TestInv{-2147483647}{2147483647}%
  \TestInv{12345678901234567890}{-12345678901234567890}%
  \TestInv{-12345678901234567890}{12345678901234567890}%
  \TestInv{ 0 }{0}%
  \TestInv{ 1 }{-1}%
  \TestInv{--1}{-1}%
  \TestInv{\number\z@}{0}%
  \TestInv{\ifx\relax\relax1\fi}{-1}%
  \TestInv{\ifx\relax\relax-\fi\ifx234\else1\fi}{1}%
\end{qstest}

\begin{qstest}{abs}{abs}%
  \TestAbs{0}{0}%
  \TestAbs{1}{1}%
  \TestAbs{-1}{1}%
  \TestAbs{10}{10}%
  \TestAbs{-10}{10}%
  \TestAbs{2147483647}{2147483647}%
  \TestAbs{-2147483647}{2147483647}%
  \TestAbs{12345678901234567890}{12345678901234567890}%
  \TestAbs{-12345678901234567890}{12345678901234567890}%
  \TestAbs{ 0 }{0}%
  \TestAbs{ 1 }{1}%
  \TestAbs{--1}{1}%
  \TestAbs{-+-+1}{1}%
  \TestAbs{00000000000}{0}%
  \TestAbs{00000001000}{1000}%
  \TestAbs{\ifx\relax\relax 0\else 1\fi}{0}%
\end{qstest}

\begin{qstest}{sign}{sign}%
  \TestSgn{0}{0}%
  \TestSgn{1}{1}%
  \TestSgn{-1}{-1}%
  \TestSgn{10}{1}%
  \TestSgn{-10}{-1}%
  \TestSgn{2147483647}{1}%
  \TestSgn{-2147483647}{-1}%
  \TestSgn{12345678901234567890}{1}%
  \TestSgn{-12345678901234567890}{-1}%
  \TestSgn{ 0 }{0}%
  \TestSgn{ 2 }{1}%
  \TestSgn{ -2 }{-1}%
  \TestSgn{--2}{1}%
  \TestSgn{\number\z@}{0}%
  \TestSgn{\number\@ne}{1}%
  \TestSgn{\number\m@ne}{-1}%
  \TestSgn{%
    -+-+\number\z@\number\z@
    \iftrue1\fi\iftrue2\fi\iftrue3\fi
  }{1}%
\end{qstest}

\begin{qstest}{min}{min}%
  \TestMin{0}{1}{0}%
  \TestMin{1}{0}{0}%
  \TestMin{-10}{-20}{-20}%
  \TestMin{ 1 }{ 2 }{1}%
  \TestMin{ 2 }{ 1 }{1}%
  \TestMin{1}{1}{1}%
  \TestMin{\number\z@}{\number\@ne}{0}%
  \TestMin{\number\@ne}{\number\m@ne}{-1}%
\end{qstest}

\begin{qstest}{max}{max}%
  \TestMax{0}{1}{1}%
  \TestMax{1}{0}{1}%
  \TestMax{-10}{-20}{-10}%
  \TestMax{ 1 }{ 2 }{2}%
  \TestMax{ 2 }{ 1 }{2}%
  \TestMax{1}{1}{1}%
  \TestMax{\number\z@}{\number\@ne}{1}%
  \TestMax{\number\@ne}{\number\m@ne}{1}%
\end{qstest}

\begin{qstest}{cmp}{cmp}%
  \TestCmp{0}{0}{0}%
  \TestCmp{-21}{17}{-1}%
  \TestCmp{3}{4}{-1}%
  \TestCmp{-10}{-10}{0}%
  \TestCmp{-10}{-11}{1}%
  \TestCmp{100}{5}{1}%
  \TestCmp{9}{10}{-1}%
  \TestCmp{10}{9}{1}%
  \TestCmp{ 3 }{ 3 }{0}%
  \TestCmp{-9}{-10}{1}%
  \TestCmp{-10}{-9}{-1}%
  \TestCmp{-3}{-3}{0}%
  \TestCmp{0}{-2}{1}%
  \TestCmp{0}{2}{-1}%
  \TestCmp{2}{0}{1}%
  \TestCmp{-2}{0}{-1}%
  \TestCmp{12}{11}{1}%
  \TestCmp{11}{12}{-1}%
  \TestCmp{2147483647}{-2147483647}{1}%
  \TestCmp{-2147483647}{2147483647}{-1}%
  \TestCmp{2147483647}{2147483647}{0}%
  \TestCmp{\number\z@}{\number\@ne}{-1}%
  \TestCmp{\number\@ne}{\number\m@ne}{1}%
  \TestCmp{ 4 }{ 5 }{-1}%
  \TestCmp{ -3 }{ -7 }{1}%
\end{qstest}

\begin{qstest}{odd}{odd}
\tracingmacros=1
  \TestOdd{0}{0}%
  \TestOdd{1}{1}%
  \TestOdd{2}{0}%
  \TestOdd{3}{1}%
  \TestOdd{14}{0}%
  \TestOdd{15}{1}%
  \TestOdd{12345678901234567896}{0}%
  \TestOdd{12345678901234567897}{1}%
\end{qstest}

\begin{qstest}{inc}{inc}%
  \TestInc{0}{1}%
  \TestInc{1}{2}%
  \TestInc{-1}{0}%
  \TestInc{10}{11}%
  \TestInc{-10}{-9}%
  \TestInc{ 3 }{4}%
  \TestInc{999}{1000}%
  \TestInc{-1000}{-999}%
  \TestInc{129}{130}%
  \TestInc{2147483646}{2147483647}%
  \TestInc{-2147483647}{-2147483646}%
  \TestInc{12345678901234567890}{12345678901234567891}%
  \TestInc{99999999999999999999}{100000000000000000000}%
  \TestInc{-12345678901234567891}{-12345678901234567890}%
  \TestInc{-100000000000000000000}{-99999999999999999999}%
\end{qstest}

\begin{qstest}{dec}{dec}%
  \TestDec{0}{-1}%
  \TestDec{1}{0}%
  \TestDec{-1}{-2}%
  \TestDec{10}{9}%
  \TestDec{-10}{-11}%
  \TestDec{1000}{999}%
  \TestDec{-999}{-1000}%
  \TestDec{130}{129}%
  \TestDec{2147483647}{2147483646}%
  \TestDec{-2147483646}{-2147483647}%
  \TestDec{12345678901234567891}{12345678901234567890}%
  \TestDec{100000000000000000000}{99999999999999999999}%
  \TestDec{-12345678901234567890}{-12345678901234567891}%
  \TestDec{-99999999999999999999}{-100000000000000000000}%
\end{qstest}

\begin{qstest}{add}{add}%
  \TestAdd{0}{0}{0}%
  \TestAdd{1}{0}{1}%
  \TestAdd{0}{1}{1}%
  \TestAdd{1}{2}{3}%
  \TestAdd{-1}{-1}{-2}%
  \TestAdd{2147483646}{1}{2147483647}%
  \TestAdd{-2147483647}{2147483647}{0}%
  \TestAdd{20}{-5}{15}%
  \TestAdd{-4}{-1}{-5}%
  \TestAdd{-1}{-4}{-5}%
  \TestAdd{-4}{1}{-3}%
  \TestAdd{-1}{4}{3}%
  \TestAdd{4}{-1}{3}%
  \TestAdd{1}{-4}{-3}%
  \TestAdd{-4}{-1}{-5}%
  \TestAdd{-1}{-4}{-5}%
  \TestAdd{ -4 }{ -1 }{-5}%
  \TestAdd{ -1 }{ -4 }{-5}%
  \TestAdd{ -4 }{ 1 }{-3}%
  \TestAdd{ -1 }{ 4 }{3}%
  \TestAdd{ 4 }{ -1 }{3}%
  \TestAdd{ 1 }{ -4 }{-3}%
  \TestAdd{ -4 }{ -1 }{-5}%
  \TestAdd{ -1 }{ -4 }{-5}%
  \TestAdd{876543210}{111111111}{987654321}%
  \TestAdd{999999999}{2}{1000000001}%
\end{qstest}

\begin{qstest}{sub}{sub}
  \TestSub{0}{0}{0}%
  \TestSub{1}{0}{1}%
  \TestSub{1}{2}{-1}%
  \TestSub{-1}{-1}{0}%
  \TestSub{2147483646}{-1}{2147483647}%
  \TestSub{-2147483647}{-2147483647}{0}%
  \TestSub{-4}{-1}{-3}%
  \TestSub{-1}{-4}{3}%
  \TestSub{-4}{1}{-5}%
  \TestSub{-1}{4}{-5}%
  \TestSub{4}{-1}{5}%
  \TestSub{1}{-4}{5}%
  \TestSub{-4}{-1}{-3}%
  \TestSub{-1}{-4}{3}%
  \TestSub{ -4 }{ -1 }{-3}%
  \TestSub{ -1 }{ -4 }{3}%
  \TestSub{ -4 }{ 1 }{-5}%
  \TestSub{ -1 }{ 4 }{-5}%
  \TestSub{ 4 }{ -1 }{5}%
  \TestSub{ 1 }{ -4 }{5}%
  \TestSub{ -4 }{ -1 }{-3}%
  \TestSub{ -1 }{ -4 }{3}%
  \TestSub{1000000000}{2}{999999998}%
  \TestSub{987654321}{111111111}{876543210}%
\end{qstest}

\begin{qstest}{shl}{shl}
  \TestShl{0}{0}%
  \TestShl{1}{2}%
  \TestShl{2}{4}%
  \TestShl{5621}{11242}%
  \TestShl{1073741823}{2147483646}%
\end{qstest}

\begin{qstest}{shr}{shr}
  \TestShr{0}{0}%
  \TestShr{1}{0}%
  \TestShr{2}{1}%
  \TestShr{3}{1}%
  \TestShr{4}{2}%
  \TestShr{5}{2}%
  \TestShr{6}{3}%
  \TestShr{7}{3}%
  \TestShr{8}{4}%
  \TestShr{9}{4}%
  \TestShr{10}{5}%
  \TestShr{11}{5}%
  \TestShr{12}{6}%
  \TestShr{13}{6}%
  \TestShr{14}{7}%
  \TestShr{15}{7}%
  \TestShr{16}{8}%
  \TestShr{17}{8}%
  \TestShr{18}{9}%
  \TestShr{19}{9}%
  \TestShr{20}{10}%
  \TestShr{21}{10}%
  \TestShr{22}{11}%
  \TestShr{11241}{5620}%
  \TestShr{73054202}{36527101}%
  \TestShr{2147483646}{1073741823}%
\end{qstest}

\begin{qstest}{mul}{mul}
  \TestMul{0}{0}{0}%
  \TestMul{1}{0}{0}%
  \TestMul{0}{1}{0}%
  \TestMul{1}{1}{1}%
  \TestMul{3}{1}{3}%
  \TestMul{1}{-3}{-3}%
  \TestMul{-4}{-5}{20}%
  \TestMul{3}{7}{21}%
  \TestMul{7}{3}{21}%
  \TestMul{3}{-7}{-21}%
  \TestMul{7}{-3}{-21}%
  \TestMul{-3}{7}{-21}%
  \TestMul{-7}{3}{-21}%
  \TestMul{-3}{-7}{21}%
  \TestMul{-7}{-3}{21}%
  \TestMul{12}{11}{132}%
  \TestMul{999}{333}{332667}%
  \TestMul{1000}{4321}{4321000}%
  \TestMul{12345}{173955}{2147474475}%
  \TestMul{1073741823}{2}{2147483646}%
  \TestMul{2}{1073741823}{2147483646}%
  \TestMul{-1073741823}{2}{-2147483646}%
  \TestMul{2}{-1073741823}{-2147483646}%
  \TestMul{6706022400}{13}{87178291200}%
\end{qstest}

\begin{qstest}{sqr}{sqr}
  \TestSqr{0}{0}%
  \TestSqr{1}{1}%
  \TestSqr{2}{4}%
  \TestSqr{3}{9}%
  \TestSqr{4}{16}%
  \TestSqr{9}{81}%
  \TestSqr{10}{100}%
  \TestSqr{46340}{2147395600}%
  \TestSqr{-1}{1}%
  \TestSqr{-2}{4}%
  \TestSqr{-46340}{2147395600}%
\end{qstest}

\begin{qstest}{fac}{fac}
  \TestFac{0}{1}%
  \TestFac{1}{1}%
  \TestFac{2}{2}%
  \TestFac{3}{2*3}%
  \TestFac{4}{2*3*4}%
  \TestFac{5}{2*3*4*5}%
  \TestFac{6}{2*3*4*5*6}%
  \TestFac{7}{2*3*4*5*6*7}%
  \TestFac{8}{2*3*4*5*6*7*8}%
  \TestFac{9}{2*3*4*5*6*7*8*9}%
  \TestFac{10}{2*3*4*5*6*7*8*9*10}%
  \TestFac{11}{2*3*4*5*6*7*8*9*10*11}%
  \TestFac{12}{2*3*4*5*6*7*8*9*10*11*12}%
  \TestFacBig{13}{6227020800}%
  \TestFacBig{14}{87178291200}%
  \TestFacBig{15}{1307674368000}%
  \TestFacBig{16}{20922789888000}%
  \TestFacBig{17}{355687428096000}%
  \TestFacBig{18}{6402373705728000}%
  \TestFacBig{19}{121645100408832000}%
  \TestFacBig{20}{2432902008176640000}%
  \TestFacBig{21}{51090942171709440000}%
  \TestFacBig{22}{1124000727777607680000}%
\end{qstest}

\begin{qstest}{pow}{pow}
  \TestPow{-2}{0}{1}%
  \TestPow{-1}{0}{1}%
  \TestPow{0}{0}{1}%
  \TestPow{1}{0}{1}%
  \TestPow{2}{0}{1}%
  \TestPow{3}{0}{1}%
  \TestPow{-2}{1}{-2}%
  \TestPow{-1}{1}{-1}%
  \TestPow{1}{1}{1}%
  \TestPow{2}{1}{2}%
  \TestPow{3}{1}{3}%
  \TestPow{-2}{2}{4}%
  \TestPow{-1}{2}{1}%
  \TestPow{0}{2}{0}%
  \TestPow{1}{2}{1}%
  \TestPow{2}{2}{4}%
  \TestPow{3}{2}{9}%
  \TestPow{0}{1}{0}%
  \TestPow{1}{-2}{1}%
  \TestPow{1}{-1}{1}%
  \TestPow{-1}{-2}{1}%
  \TestPow{-1}{-1}{-1}%
  \TestPow{-1}{3}{-1}%
  \TestPow{-1}{4}{1}%
  \TestPow{-2}{-1}{0}%
  \TestPow{-2}{-2}{0}%
  \TestPow{2}{3}{8}%
  \TestPow{2}{4}{16}%
  \TestPow{2}{5}{32}%
  \TestPow{2}{6}{64}%
  \TestPow{2}{7}{128}%
  \TestPow{2}{8}{256}%
  \TestPow{2}{9}{512}%
  \TestPow{2}{10}{1024}%
  \TestPow{-2}{3}{-8}%
  \TestPow{-2}{4}{16}%
  \TestPow{-2}{5}{-32}%
  \TestPow{-2}{6}{64}%
  \TestPow{-2}{7}{-128}%
  \TestPow{-2}{8}{256}%
  \TestPow{-2}{9}{-512}%
  \TestPow{-2}{10}{1024}%
  \TestPow{3}{3}{27}%
  \TestPow{3}{4}{81}%
  \TestPow{3}{5}{243}%
  \TestPow{-3}{3}{-27}%
  \TestPow{-3}{4}{81}%
  \TestPow{-3}{5}{-243}%
  \TestPow{2}{30}{1073741824}%
  \TestPow{-3}{19}{-1162261467}%
  \TestPow{5}{13}{1220703125}%
  \TestPow{-7}{11}{-1977326743}%
\end{qstest}

\begin{qstest}{div}{div}
  \TestDiv{1}{1}{1}%
  \TestDiv{2}{1}{2}%
  \TestDiv{-2}{1}{-2}%
  \TestDiv{2}{-1}{-2}%
  \TestDiv{-2}{-1}{2}%
  \TestDiv{15}{2}{7}%
  \TestDiv{-16}{2}{-8}%
  \TestDiv{1}{2}{0}%
  \TestDiv{1}{3}{0}%
  \TestDiv{2}{3}{0}%
  \TestDiv{-2}{3}{0}%
  \TestDiv{2}{-3}{0}%
  \TestDiv{-2}{-3}{0}%
  \TestDiv{13}{3}{4}%
  \TestDiv{-13}{-3}{4}%
  \TestDiv{-13}{3}{-4}%
  \TestDiv{-6}{5}{-1}%
  \TestDiv{-5}{5}{-1}%
  \TestDiv{-4}{5}{0}%
  \TestDiv{-3}{5}{0}%
  \TestDiv{-2}{5}{0}%
  \TestDiv{-1}{5}{0}%
  \TestDiv{0}{5}{0}%
  \TestDiv{1}{5}{0}%
  \TestDiv{2}{5}{0}%
  \TestDiv{3}{5}{0}%
  \TestDiv{4}{5}{0}%
  \TestDiv{5}{5}{1}%
  \TestDiv{6}{5}{1}%
  \TestDiv{-5}{4}{-1}%
  \TestDiv{-4}{4}{-1}%
  \TestDiv{-3}{4}{0}%
  \TestDiv{-2}{4}{0}%
  \TestDiv{-1}{4}{0}%
  \TestDiv{0}{4}{0}%
  \TestDiv{1}{4}{0}%
  \TestDiv{2}{4}{0}%
  \TestDiv{3}{4}{0}%
  \TestDiv{4}{4}{1}%
  \TestDiv{5}{4}{1}%
  \TestDiv{12345}{678}{18}%
  \TestDiv{32372}{5952}{5}%
  \TestDiv{284271294}{18162}{15651}%
  \TestDiv{217652429}{12561}{17327}%
  \TestDiv{462028434}{5439}{84947}%
  \TestDiv{2147483647}{1000}{2147483}%
  \TestDiv{2147483647}{-1000}{-2147483}%
  \TestDiv{-2147483647}{1000}{-2147483}%
  \TestDiv{-2147483647}{-1000}{2147483}%
  \TestDiv{0}{3}{0}%
  \TestDiv{1}{3}{0}%
  \TestDiv{2}{3}{0}%
  \TestDiv{3}{3}{1}%
  \TestDiv{4}{3}{1}%
  \TestDiv{5}{3}{1}%
  \TestDiv{6}{3}{2}%
  \TestDiv{7}{3}{2}%
  \TestDiv{8}{3}{2}%
  \TestDiv{9}{3}{3}%
  \TestDiv{10}{3}{3}%
  \TestDiv{11}{3}{3}%
  \TestDiv{12}{3}{4}%
  \TestDiv{13}{3}{4}%
  \TestDiv{14}{3}{4}%
  \TestDiv{15}{3}{5}%
  \TestDiv{16}{3}{5}%
  \TestDiv{17}{3}{5}%
  \TestDiv{18}{3}{6}%
  \TestDiv{19}{3}{6}%
  \TestDiv{20}{3}{6}%
  \TestDiv{21}{3}{7}%
  \TestDiv{22}{3}{7}%
  \TestDiv{23}{3}{7}%
  \TestDiv{24}{3}{8}%
  \TestDiv{25}{3}{8}%
  \TestDiv{26}{3}{8}%
  \TestDiv{27}{3}{9}%
  \TestDiv{28}{3}{9}%
  \TestDiv{29}{3}{9}%
  \TestDiv{30}{3}{10}%
  \TestDiv{31}{3}{10}%
  \TestDivBig{17363436332507}{24702}{702916214}%
\end{qstest}

\begin{qstest}{mod}{mod}
  \TestMod{-6}{5}{4}%
  \TestMod{-5}{5}{0}%
  \TestMod{-4}{5}{1}%
  \TestMod{-3}{5}{2}%
  \TestMod{-2}{5}{3}%
  \TestMod{-1}{5}{4}%
  \TestMod{0}{5}{0}%
  \TestMod{1}{5}{1}%
  \TestMod{2}{5}{2}%
  \TestMod{3}{5}{3}%
  \TestMod{4}{5}{4}%
  \TestMod{5}{5}{0}%
  \TestMod{6}{5}{1}%
  \TestMod{-5}{4}{3}%
  \TestMod{-4}{4}{0}%
  \TestMod{-3}{4}{1}%
  \TestMod{-2}{4}{2}%
  \TestMod{-1}{4}{3}%
  \TestMod{0}{4}{0}%
  \TestMod{1}{4}{1}%
  \TestMod{2}{4}{2}%
  \TestMod{3}{4}{3}%
  \TestMod{4}{4}{0}%
  \TestMod{5}{4}{1}%
  \TestMod{-6}{-5}{-1}%
  \TestMod{-5}{-5}{0}%
  \TestMod{-4}{-5}{-4}%
  \TestMod{-3}{-5}{-3}%
  \TestMod{-2}{-5}{-2}%
  \TestMod{-1}{-5}{-1}%
  \TestMod{0}{-5}{0}%
  \TestMod{1}{-5}{-4}%
  \TestMod{2}{-5}{-3}%
  \TestMod{3}{-5}{-2}%
  \TestMod{4}{-5}{-1}%
  \TestMod{5}{-5}{0}%
  \TestMod{6}{-5}{-4}%
  \TestMod{-5}{-4}{-1}%
  \TestMod{-4}{-4}{0}%
  \TestMod{-3}{-4}{-3}%
  \TestMod{-2}{-4}{-2}%
  \TestMod{-1}{-4}{-1}%
  \TestMod{0}{-4}{0}%
  \TestMod{1}{-4}{-3}%
  \TestMod{2}{-4}{-2}%
  \TestMod{3}{-4}{-1}%
  \TestMod{4}{-4}{0}%
  \TestMod{5}{-4}{-3}%
  \TestMod{2147483647}{1000}{647}%
  \TestMod{2147483647}{-1000}{-353}%
  \TestMod{-2147483647}{1000}{353}%
  \TestMod{-2147483647}{-1000}{-647}%
  \TestMod{ 0 }{ 4 }{0}%
  \TestMod{ 1 }{ 4 }{1}%
  \TestMod{ -1 }{ 4 }{3}%
  \TestMod{ 0 }{ -4 }{0}%
  \TestMod{ 1 }{ -4 }{-3}%
  \TestMod{ -1 }{ -4 }{-1}%
  \TestMod{18362}{25}{12}%
\end{qstest}

\newcommand*{\TestError}[2]{%
  \begingroup
    \expandafter\def\csname BigIntCalcError:#1\endcsname{}%
    \Expect*{#2}{0}%
    \expandafter\def\csname BigIntCalcError:#1\endcsname{ERROR}%
    \Expect*{#2}{0ERROR}%
  \endgroup
}
\begin{qstest}{error}{error}
  \TestError{FacNegative}{\bigintcalcFac{-1}}%
  \TestError{FacNegative}{\bigintcalcFac{-2147483647}}%
  \TestError{DivisionByZero}{\bigintcalcPow{0}{-1}}%
  \TestError{DivisionByZero}{\bigintcalcDiv{1}{0}}%
  \TestError{DivisionByZero}{\bigintcalcMod{1}{0}}%
\end{qstest}

\begin{document}
\end{document}
%</test2>
%    \end{macrocode}
%
% \section{Installation}
%
% \subsection{Download}
%
% \paragraph{Package.} This package is available on
% CTAN\footnote{\url{http://ctan.org/pkg/bigintcalc}}:
% \begin{description}
% \item[\CTAN{macros/latex/contrib/oberdiek/bigintcalc.dtx}] The source file.
% \item[\CTAN{macros/latex/contrib/oberdiek/bigintcalc.pdf}] Documentation.
% \end{description}
%
%
% \paragraph{Bundle.} All the packages of the bundle `oberdiek'
% are also available in a TDS compliant ZIP archive. There
% the packages are already unpacked and the documentation files
% are generated. The files and directories obey the TDS standard.
% \begin{description}
% \item[\CTAN{install/macros/latex/contrib/oberdiek.tds.zip}]
% \end{description}
% \emph{TDS} refers to the standard ``A Directory Structure
% for \TeX\ Files'' (\CTAN{tds/tds.pdf}). Directories
% with \xfile{texmf} in their name are usually organized this way.
%
% \subsection{Bundle installation}
%
% \paragraph{Unpacking.} Unpack the \xfile{oberdiek.tds.zip} in the
% TDS tree (also known as \xfile{texmf} tree) of your choice.
% Example (linux):
% \begin{quote}
%   |unzip oberdiek.tds.zip -d ~/texmf|
% \end{quote}
%
% \paragraph{Script installation.}
% Check the directory \xfile{TDS:scripts/oberdiek/} for
% scripts that need further installation steps.
% Package \xpackage{attachfile2} comes with the Perl script
% \xfile{pdfatfi.pl} that should be installed in such a way
% that it can be called as \texttt{pdfatfi}.
% Example (linux):
% \begin{quote}
%   |chmod +x scripts/oberdiek/pdfatfi.pl|\\
%   |cp scripts/oberdiek/pdfatfi.pl /usr/local/bin/|
% \end{quote}
%
% \subsection{Package installation}
%
% \paragraph{Unpacking.} The \xfile{.dtx} file is a self-extracting
% \docstrip\ archive. The files are extracted by running the
% \xfile{.dtx} through \plainTeX:
% \begin{quote}
%   \verb|tex bigintcalc.dtx|
% \end{quote}
%
% \paragraph{TDS.} Now the different files must be moved into
% the different directories in your installation TDS tree
% (also known as \xfile{texmf} tree):
% \begin{quote}
% \def\t{^^A
% \begin{tabular}{@{}>{\ttfamily}l@{ $\rightarrow$ }>{\ttfamily}l@{}}
%   bigintcalc.sty & tex/generic/oberdiek/bigintcalc.sty\\
%   bigintcalc.pdf & doc/latex/oberdiek/bigintcalc.pdf\\
%   test/bigintcalc-test1.tex & doc/latex/oberdiek/test/bigintcalc-test1.tex\\
%   test/bigintcalc-test2.tex & doc/latex/oberdiek/test/bigintcalc-test2.tex\\
%   test/bigintcalc-test3.tex & doc/latex/oberdiek/test/bigintcalc-test3.tex\\
%   bigintcalc.dtx & source/latex/oberdiek/bigintcalc.dtx\\
% \end{tabular}^^A
% }^^A
% \sbox0{\t}^^A
% \ifdim\wd0>\linewidth
%   \begingroup
%     \advance\linewidth by\leftmargin
%     \advance\linewidth by\rightmargin
%   \edef\x{\endgroup
%     \def\noexpand\lw{\the\linewidth}^^A
%   }\x
%   \def\lwbox{^^A
%     \leavevmode
%     \hbox to \linewidth{^^A
%       \kern-\leftmargin\relax
%       \hss
%       \usebox0
%       \hss
%       \kern-\rightmargin\relax
%     }^^A
%   }^^A
%   \ifdim\wd0>\lw
%     \sbox0{\small\t}^^A
%     \ifdim\wd0>\linewidth
%       \ifdim\wd0>\lw
%         \sbox0{\footnotesize\t}^^A
%         \ifdim\wd0>\linewidth
%           \ifdim\wd0>\lw
%             \sbox0{\scriptsize\t}^^A
%             \ifdim\wd0>\linewidth
%               \ifdim\wd0>\lw
%                 \sbox0{\tiny\t}^^A
%                 \ifdim\wd0>\linewidth
%                   \lwbox
%                 \else
%                   \usebox0
%                 \fi
%               \else
%                 \lwbox
%               \fi
%             \else
%               \usebox0
%             \fi
%           \else
%             \lwbox
%           \fi
%         \else
%           \usebox0
%         \fi
%       \else
%         \lwbox
%       \fi
%     \else
%       \usebox0
%     \fi
%   \else
%     \lwbox
%   \fi
% \else
%   \usebox0
% \fi
% \end{quote}
% If you have a \xfile{docstrip.cfg} that configures and enables \docstrip's
% TDS installing feature, then some files can already be in the right
% place, see the documentation of \docstrip.
%
% \subsection{Refresh file name databases}
%
% If your \TeX~distribution
% (\teTeX, \mikTeX, \dots) relies on file name databases, you must refresh
% these. For example, \teTeX\ users run \verb|texhash| or
% \verb|mktexlsr|.
%
% \subsection{Some details for the interested}
%
% \paragraph{Attached source.}
%
% The PDF documentation on CTAN also includes the
% \xfile{.dtx} source file. It can be extracted by
% AcrobatReader 6 or higher. Another option is \textsf{pdftk},
% e.g. unpack the file into the current directory:
% \begin{quote}
%   \verb|pdftk bigintcalc.pdf unpack_files output .|
% \end{quote}
%
% \paragraph{Unpacking with \LaTeX.}
% The \xfile{.dtx} chooses its action depending on the format:
% \begin{description}
% \item[\plainTeX:] Run \docstrip\ and extract the files.
% \item[\LaTeX:] Generate the documentation.
% \end{description}
% If you insist on using \LaTeX\ for \docstrip\ (really,
% \docstrip\ does not need \LaTeX), then inform the autodetect routine
% about your intention:
% \begin{quote}
%   \verb|latex \let\install=y% \iffalse meta-comment
%
% File: bigintcalc.dtx
% Version: 2016/05/16 v1.4
% Info: Expandable calculations on big integers
%
% Copyright (C) 2007, 2011, 2012 by
%    Heiko Oberdiek <heiko.oberdiek at googlemail.com>
%    2016
%    https://github.com/ho-tex/oberdiek/issues
%
% This work may be distributed and/or modified under the
% conditions of the LaTeX Project Public License, either
% version 1.3c of this license or (at your option) any later
% version. This version of this license is in
%    http://www.latex-project.org/lppl/lppl-1-3c.txt
% and the latest version of this license is in
%    http://www.latex-project.org/lppl.txt
% and version 1.3 or later is part of all distributions of
% LaTeX version 2005/12/01 or later.
%
% This work has the LPPL maintenance status "maintained".
%
% This Current Maintainer of this work is Heiko Oberdiek.
%
% The Base Interpreter refers to any `TeX-Format',
% because some files are installed in TDS:tex/generic//.
%
% This work consists of the main source file bigintcalc.dtx
% and the derived files
%    bigintcalc.sty, bigintcalc.pdf, bigintcalc.ins, bigintcalc.drv,
%    bigintcalc-test1.tex, bigintcalc-test2.tex,
%    bigintcalc-test3.tex.
%
% Distribution:
%    CTAN:macros/latex/contrib/oberdiek/bigintcalc.dtx
%    CTAN:macros/latex/contrib/oberdiek/bigintcalc.pdf
%
% Unpacking:
%    (a) If bigintcalc.ins is present:
%           tex bigintcalc.ins
%    (b) Without bigintcalc.ins:
%           tex bigintcalc.dtx
%    (c) If you insist on using LaTeX
%           latex \let\install=y\input{bigintcalc.dtx}
%        (quote the arguments according to the demands of your shell)
%
% Documentation:
%    (a) If bigintcalc.drv is present:
%           latex bigintcalc.drv
%    (b) Without bigintcalc.drv:
%           latex bigintcalc.dtx; ...
%    The class ltxdoc loads the configuration file ltxdoc.cfg
%    if available. Here you can specify further options, e.g.
%    use A4 as paper format:
%       \PassOptionsToClass{a4paper}{article}
%
%    Programm calls to get the documentation (example):
%       pdflatex bigintcalc.dtx
%       makeindex -s gind.ist bigintcalc.idx
%       pdflatex bigintcalc.dtx
%       makeindex -s gind.ist bigintcalc.idx
%       pdflatex bigintcalc.dtx
%
% Installation:
%    TDS:tex/generic/oberdiek/bigintcalc.sty
%    TDS:doc/latex/oberdiek/bigintcalc.pdf
%    TDS:doc/latex/oberdiek/test/bigintcalc-test1.tex
%    TDS:doc/latex/oberdiek/test/bigintcalc-test2.tex
%    TDS:doc/latex/oberdiek/test/bigintcalc-test3.tex
%    TDS:source/latex/oberdiek/bigintcalc.dtx
%
%<*ignore>
\begingroup
  \catcode123=1 %
  \catcode125=2 %
  \def\x{LaTeX2e}%
\expandafter\endgroup
\ifcase 0\ifx\install y1\fi\expandafter
         \ifx\csname processbatchFile\endcsname\relax\else1\fi
         \ifx\fmtname\x\else 1\fi\relax
\else\csname fi\endcsname
%</ignore>
%<*install>
\input docstrip.tex
\Msg{************************************************************************}
\Msg{* Installation}
\Msg{* Package: bigintcalc 2016/05/16 v1.4 Expandable calculations on big integers (HO)}
\Msg{************************************************************************}

\keepsilent
\askforoverwritefalse

\let\MetaPrefix\relax
\preamble

This is a generated file.

Project: bigintcalc
Version: 2016/05/16 v1.4

Copyright (C) 2007, 2011, 2012 by
   Heiko Oberdiek <heiko.oberdiek at googlemail.com>

This work may be distributed and/or modified under the
conditions of the LaTeX Project Public License, either
version 1.3c of this license or (at your option) any later
version. This version of this license is in
   http://www.latex-project.org/lppl/lppl-1-3c.txt
and the latest version of this license is in
   http://www.latex-project.org/lppl.txt
and version 1.3 or later is part of all distributions of
LaTeX version 2005/12/01 or later.

This work has the LPPL maintenance status "maintained".

This Current Maintainer of this work is Heiko Oberdiek.

The Base Interpreter refers to any `TeX-Format',
because some files are installed in TDS:tex/generic//.

This work consists of the main source file bigintcalc.dtx
and the derived files
   bigintcalc.sty, bigintcalc.pdf, bigintcalc.ins, bigintcalc.drv,
   bigintcalc-test1.tex, bigintcalc-test2.tex,
   bigintcalc-test3.tex.

\endpreamble
\let\MetaPrefix\DoubleperCent

\generate{%
  \file{bigintcalc.ins}{\from{bigintcalc.dtx}{install}}%
  \file{bigintcalc.drv}{\from{bigintcalc.dtx}{driver}}%
  \usedir{tex/generic/oberdiek}%
  \file{bigintcalc.sty}{\from{bigintcalc.dtx}{package}}%
%  \usedir{doc/latex/oberdiek/test}%
%  \file{bigintcalc-test1.tex}{\from{bigintcalc.dtx}{test1}}%
%  \file{bigintcalc-test2.tex}{\from{bigintcalc.dtx}{test2,etex}}%
%  \file{bigintcalc-test3.tex}{\from{bigintcalc.dtx}{test2,noetex}}%
  \nopreamble
  \nopostamble
  \usedir{source/latex/oberdiek/catalogue}%
  \file{bigintcalc.xml}{\from{bigintcalc.dtx}{catalogue}}%
}

\catcode32=13\relax% active space
\let =\space%
\Msg{************************************************************************}
\Msg{*}
\Msg{* To finish the installation you have to move the following}
\Msg{* file into a directory searched by TeX:}
\Msg{*}
\Msg{*     bigintcalc.sty}
\Msg{*}
\Msg{* To produce the documentation run the file `bigintcalc.drv'}
\Msg{* through LaTeX.}
\Msg{*}
\Msg{* Happy TeXing!}
\Msg{*}
\Msg{************************************************************************}

\endbatchfile
%</install>
%<*ignore>
\fi
%</ignore>
%<*driver>
\NeedsTeXFormat{LaTeX2e}
\ProvidesFile{bigintcalc.drv}%
  [2016/05/16 v1.4 Expandable calculations on big integers (HO)]%
\documentclass{ltxdoc}
\usepackage{wasysym}
\let\iint\relax
\let\iiint\relax
\usepackage[fleqn]{amsmath}

\DeclareMathOperator{\opInv}{Inv}
\DeclareMathOperator{\opAbs}{Abs}
\DeclareMathOperator{\opSgn}{Sgn}
\DeclareMathOperator{\opMin}{Min}
\DeclareMathOperator{\opMax}{Max}
\DeclareMathOperator{\opCmp}{Cmp}
\DeclareMathOperator{\opOdd}{Odd}
\DeclareMathOperator{\opInc}{Inc}
\DeclareMathOperator{\opDec}{Dec}
\DeclareMathOperator{\opAdd}{Add}
\DeclareMathOperator{\opSub}{Sub}
\DeclareMathOperator{\opShl}{Shl}
\DeclareMathOperator{\opShr}{Shr}
\DeclareMathOperator{\opMul}{Mul}
\DeclareMathOperator{\opSqr}{Sqr}
\DeclareMathOperator{\opFac}{Fac}
\DeclareMathOperator{\opPow}{Pow}
\DeclareMathOperator{\opDiv}{Div}
\DeclareMathOperator{\opMod}{Mod}
\DeclareMathOperator{\opInt}{Int}
\DeclareMathOperator{\opODD}{ifodd}

\newcommand*{\Def}{%
  \ensuremath{%
    \mathrel{\mathop{:}}=%
  }%
}
\newcommand*{\op}[1]{%
  \textsf{#1}%
}
\usepackage{holtxdoc}[2011/11/22]
\begin{document}
  \DocInput{bigintcalc.dtx}%
\end{document}
%</driver>
% \fi
%
%
% \CharacterTable
%  {Upper-case    \A\B\C\D\E\F\G\H\I\J\K\L\M\N\O\P\Q\R\S\T\U\V\W\X\Y\Z
%   Lower-case    \a\b\c\d\e\f\g\h\i\j\k\l\m\n\o\p\q\r\s\t\u\v\w\x\y\z
%   Digits        \0\1\2\3\4\5\6\7\8\9
%   Exclamation   \!     Double quote  \"     Hash (number) \#
%   Dollar        \$     Percent       \%     Ampersand     \&
%   Acute accent  \'     Left paren    \(     Right paren   \)
%   Asterisk      \*     Plus          \+     Comma         \,
%   Minus         \-     Point         \.     Solidus       \/
%   Colon         \:     Semicolon     \;     Less than     \<
%   Equals        \=     Greater than  \>     Question mark \?
%   Commercial at \@     Left bracket  \[     Backslash     \\
%   Right bracket \]     Circumflex    \^     Underscore    \_
%   Grave accent  \`     Left brace    \{     Vertical bar  \|
%   Right brace   \}     Tilde         \~}
%
% \GetFileInfo{bigintcalc.drv}
%
% \title{The \xpackage{bigintcalc} package}
% \date{2016/05/16 v1.4}
% \author{Heiko Oberdiek\thanks
% {Please report any issues at https://github.com/ho-tex/oberdiek/issues}\\
% \xemail{heiko.oberdiek at googlemail.com}}
%
% \maketitle
%
% \begin{abstract}
% This package provides expandable arithmetic operations
% with big integers that can exceed \TeX's number limits.
% \end{abstract}
%
% \tableofcontents
%
% \section{Documentation}
%
% \subsection{Introduction}
%
% Package \xpackage{bigintcalc} defines arithmetic operations
% that deal with big integers. Big integers can be given
% either as explicit integer number or as macro code
% that expands to an explicit number. \emph{Big} means that
% there is no limit on the size of the number. Big integers
% may exceed \TeX's range limitation of -2147483647 and 2147483647.
% Only memory issues will limit the usable range.
%
% In opposite to package \xpackage{intcalc}
% unexpandable command tokens are not supported, even if
% they are valid \TeX\ numbers like count registers or
% commands created by \cs{chardef}. Nevertheless they may
% be used, if they are prefixed by \cs{number}.
%
% Also \eTeX's \cs{numexpr} expressions are not supported
% directly in the manner of package \xpackage{intcalc}.
% However they can be given if \cs{the}\cs{numexpr}
% or \cs{number}\cs{numexpr} are used.
%
% The operations have the form of macros that take one or
% two integers as parameter and return the integer result.
% The macro name is a three letter operation name prefixed
% by the package name, e.g. \cs{bigintcalcAdd}|{10}{43}| returns
% |53|.
%
% The macros are fully expandable, exactly two expansion
% steps generate the result. Therefore the operations
% may be used nearly everywhere in \TeX, even inside
% \cs{csname}, file names,  or other
% expandable contexts.
%
% \subsection{Conditions}
%
% \subsubsection{Preconditions}
%
% \begin{itemize}
% \item
%   Arguments can be anything that expands to a number that consists
%   of optional signs and digits.
% \item
%   The arguments and return values must be sound.
%   Zero as divisor or factorials of negative numbers will cause errors.
% \end{itemize}
%
% \subsubsection{Postconditions}
%
% Additional properties of the macros apart from calculating
% a correct result (of course \smiley):
% \begin{itemize}
% \item
%   The macros are fully expandable. Thus they can
%   be used inside \cs{edef}, \cs{csname},
%   for example.
% \item
%   Furthermore exactly two expansion steps calculate the result.
% \item
%   The number consists of one optional minus sign and one or more
%   digits. The first digit is larger than zero for
%   numbers that consists of more than one digit.
%
%   In short, the number format is exactly the same as
%   \cs{number} generates, but without its range limitation.
%   And the tokens (minus sign, digits)
%   have catcode 12 (other).
% \item
%   Call by value is simulated. First the arguments are
%   converted to numbers. Then these numbers are used
%   in the calculations.
%
%   Remember that arguments
%   may contain expensive macros or \eTeX\ expressions.
%   This strategy avoids multiple evaluations of such
%   arguments.
% \end{itemize}
%
% \subsection{Error handling}
%
% Some errors are detected by the macros, example: division by zero.
% In this cases an undefined control sequence is called and causes
% a TeX error message, example: \cs{BigIntCalcError:DivisionByZero}.
% The name of the control sequence contains
% the reason for the error. The \TeX\ error may be ignored.
% Then the operation returns zero as result.
% Because the macros are supposed to work in expandible contexts.
% An traditional error message, however, is not expandable and
% would break these contexts.
%
% \subsection{Operations}
%
% Some definition equations below use the function $\opInt$
% that converts a real number to an integer. The number
% is truncated that means rounding to zero:
% \begin{gather*}
%   \opInt(x) \Def
%   \begin{cases}
%     \lfloor x\rfloor & \text{if $x\geq0$}\\
%     \lceil x\rceil & \text{otherwise}
%   \end{cases}
% \end{gather*}
%
% \subsubsection{\op{Num}}
%
% \begin{declcs}{bigintcalcNum} \M{x}
% \end{declcs}
% Macro \cs{bigintcalcNum} converts its argument to a normalized integer
% number without unnecessary leading zeros or signs.
% The result matches the regular expression:
%\begin{quote}
%\begin{verbatim}
%0|-?[1-9][0-9]*
%\end{verbatim}
%\end{quote}
%
% \subsubsection{\op{Inv}, \op{Abs}, \op{Sgn}}
%
% \begin{declcs}{bigintcalcInv} \M{x}
% \end{declcs}
% Macro \cs{bigintcalcInv} switches the sign.
% \begin{gather*}
%   \opInv(x) \Def -x
% \end{gather*}
%
% \begin{declcs}{bigintcalcAbs} \M{x}
% \end{declcs}
% Macro \cs{bigintcalcAbs} returns the absolute value
% of integer \meta{x}.
% \begin{gather*}
%   \opAbs(x) \Def \vert x\vert
% \end{gather*}
%
% \begin{declcs}{bigintcalcSgn} \M{x}
% \end{declcs}
% Macro \cs{bigintcalcSgn} encodes the sign of \meta{x} as number.
% \begin{gather*}
%   \opSgn(x) \Def
%   \begin{cases}
%     -1& \text{if $x<0$}\\
%      0& \text{if $x=0$}\\
%      1& \text{if $x>0$}
%   \end{cases}
% \end{gather*}
% These return values can easily be distinguished by \cs{ifcase}:
%\begin{quote}
%\begin{verbatim}
%\ifcase\bigintcalcSgn{<x>}
%  $x=0$
%\or
%  $x>0$
%\else
%  $x<0$
%\fi
%\end{verbatim}
%\end{quote}
%
% \subsubsection{\op{Min}, \op{Max}, \op{Cmp}}
%
% \begin{declcs}{bigintcalcMin} \M{x} \M{y}
% \end{declcs}
% Macro \cs{bigintcalcMin} returns the smaller of the two integers.
% \begin{gather*}
%   \opMin(x,y) \Def
%   \begin{cases}
%     x & \text{if $x<y$}\\
%     y & \text{otherwise}
%   \end{cases}
% \end{gather*}
%
% \begin{declcs}{bigintcalcMax} \M{x} \M{y}
% \end{declcs}
% Macro \cs{bigintcalcMax} returns the larger of the two integers.
% \begin{gather*}
%   \opMax(x,y) \Def
%   \begin{cases}
%     x & \text{if $x>y$}\\
%     y & \text{otherwise}
%   \end{cases}
% \end{gather*}
%
% \begin{declcs}{bigintcalcCmp} \M{x} \M{y}
% \end{declcs}
% Macro \cs{bigintcalcCmp} encodes the comparison result as number:
% \begin{gather*}
%   \opCmp(x,y) \Def
%   \begin{cases}
%     -1 & \text{if $x<y$}\\
%      0 & \text{if $x=y$}\\
%      1 & \text{if $x>y$}
%   \end{cases}
% \end{gather*}
% These values can be distinguished by \cs{ifcase}:
%\begin{quote}
%\begin{verbatim}
%\ifcase\bigintcalcCmp{<x>}{<y>}
%  $x=y$
%\or
%  $x>y$
%\else
%  $x<y$
%\fi
%\end{verbatim}
%\end{quote}
%
% \subsubsection{\op{Odd}}
%
% \begin{declcs}{bigintcalcOdd} \M{x}
% \end{declcs}
% \begin{gather*}
%   \opOdd(x) \Def
%   \begin{cases}
%     1 & \text{if $x$ is odd}\\
%     0 & \text{if $x$ is even}
%   \end{cases}
% \end{gather*}
%
% \subsubsection{\op{Inc}, \op{Dec}, \op{Add}, \op{Sub}}
%
% \begin{declcs}{bigintcalcInc} \M{x}
% \end{declcs}
% Macro \cs{bigintcalcInc} increments \meta{x} by one.
% \begin{gather*}
%   \opInc(x) \Def x + 1
% \end{gather*}
%
% \begin{declcs}{bigintcalcDec} \M{x}
% \end{declcs}
% Macro \cs{bigintcalcDec} decrements \meta{x} by one.
% \begin{gather*}
%   \opDec(x) \Def x - 1
% \end{gather*}
%
% \begin{declcs}{bigintcalcAdd} \M{x} \M{y}
% \end{declcs}
% Macro \cs{bigintcalcAdd} adds the two numbers.
% \begin{gather*}
%   \opAdd(x, y) \Def x + y
% \end{gather*}
%
% \begin{declcs}{bigintcalcSub} \M{x} \M{y}
% \end{declcs}
% Macro \cs{bigintcalcSub} calculates the difference.
% \begin{gather*}
%   \opSub(x, y) \Def x - y
% \end{gather*}
%
% \subsubsection{\op{Shl}, \op{Shr}}
%
% \begin{declcs}{bigintcalcShl} \M{x}
% \end{declcs}
% Macro \cs{bigintcalcShl} implements shifting to the left that
% means the number is multiplied by two.
% The sign is preserved.
% \begin{gather*}
%   \opShl(x) \Def x*2
% \end{gather*}
%
% \begin{declcs}{bigintcalcShr} \M{x}
% \end{declcs}
% Macro \cs{bigintcalcShr} implements shifting to the right.
% That is equivalent to an integer division by two.
% The sign is preserved.
% \begin{gather*}
%   \opShr(x) \Def \opInt(x/2)
% \end{gather*}
%
% \subsubsection{\op{Mul}, \op{Sqr}, \op{Fac}, \op{Pow}}
%
% \begin{declcs}{bigintcalcMul} \M{x} \M{y}
% \end{declcs}
% Macro \cs{bigintcalcMul} calculates the product of
% \meta{x} and \meta{y}.
% \begin{gather*}
%   \opMul(x,y) \Def x*y
% \end{gather*}
%
% \begin{declcs}{bigintcalcSqr} \M{x}
% \end{declcs}
% Macro \cs{bigintcalcSqr} returns the square product.
% \begin{gather*}
%   \opSqr(x) \Def x^2
% \end{gather*}
%
% \begin{declcs}{bigintcalcFac} \M{x}
% \end{declcs}
% Macro \cs{bigintcalcFac} returns the factorial of \meta{x}.
% Negative numbers are not permitted.
% \begin{gather*}
%   \opFac(x) \Def x!\qquad\text{for $x\geq0$}
% \end{gather*}
% ($0! = 1$)
%
% \begin{declcs}{bigintcalcPow} M{x} M{y}
% \end{declcs}
% Macro \cs{bigintcalcPow} calculates the value of \meta{x} to the
% power of \meta{y}. The error ``division by zero'' is thrown
% if \meta{x} is zero and \meta{y} is negative.
% permitted:
% \begin{gather*}
%   \opPow(x,y) \Def
%   \opInt(x^y)\qquad\text{for $x\neq0$ or $y\geq0$}
% \end{gather*}
% ($0^0 = 1$)
%
% \subsubsection{\op{Div}, \op{Mul}}
%
% \begin{declcs}{bigintcalcDiv} \M{x} \M{y}
% \end{declcs}
% Macro \cs{bigintcalcDiv} performs an integer division.
% Argument \meta{y} must not be zero.
% \begin{gather*}
%   \opDiv(x,y) \Def \opInt(x/y)\qquad\text{for $y\neq0$}
% \end{gather*}
%
% \begin{declcs}{bigintcalcMod} \M{x} \M{y}
% \end{declcs}
% Macro \cs{bigintcalcMod} gets the remainder of the integer
% division. The sign follows the divisor \meta{y}.
% Argument \meta{y} must not be zero.
% \begin{gather*}
%   \opMod(x,y) \Def x\mathrel{\%}y\qquad\text{for $y\neq0$}
% \end{gather*}
% The result ranges:
% \begin{gather*}
%   -\vert y\vert < \opMod(x,y) \leq 0\qquad\text{for $y<0$}\\
%   0 \leq \opMod(x,y) < y\qquad\text{for $y\geq0$}
% \end{gather*}
%
% \subsection{Interface for programmers}
%
% If the programmer can ensure some more properties about
% the arguments of the operations, then the following
% macros are a little more efficient.
%
% In general numbers must obey the following constraints:
% \begin{itemize}
% \item Plain number: digit tokens only, no command tokens.
% \item Non-negative. Signs are forbidden.
% \item Delimited by exclamation mark. Curly braces
%       around the number are not allowed and will
%       break the code.
% \end{itemize}
%
% \begin{declcs}{BigIntCalcOdd} \meta{number} |!|
% \end{declcs}
% |1|/|0| is returned if \meta{number} is odd/even.
%
% \begin{declcs}{BigIntCalcInc} \meta{number} |!|
% \end{declcs}
% Incrementation.
%
% \begin{declcs}{BigIntCalcDec} \meta{number} |!|
% \end{declcs}
% Decrementation, positive number without zero.
%
% \begin{declcs}{BigIntCalcAdd} \meta{number A} |!| \meta{number B} |!|
% \end{declcs}
% Addition, $A\geq B$.
%
% \begin{declcs}{BigIntCalcSub} \meta{number A} |!| \meta{number B} |!|
% \end{declcs}
% Subtraction, $A\geq B$.
%
% \begin{declcs}{BigIntCalcShl} \meta{number} |!|
% \end{declcs}
% Left shift (multiplication with two).
%
% \begin{declcs}{BigIntCalcShr} \meta{number} |!|
% \end{declcs}
% Right shift (integer division by two).
%
% \begin{declcs}{BigIntCalcMul} \meta{number A} |!| \meta{number B} |!|
% \end{declcs}
% Multiplication, $A\geq B$.
%
% \begin{declcs}{BigIntCalcDiv} \meta{number A} |!| \meta{number B} |!|
% \end{declcs}
% Division operation.
%
% \begin{declcs}{BigIntCalcMod} \meta{number A} |!| \meta{number B} |!|
% \end{declcs}
% Modulo operation.
%
%
% \StopEventually{
% }
%
% \section{Implementation}
%
%    \begin{macrocode}
%<*package>
%    \end{macrocode}
%
% \subsection{Reload check and package identification}
%    Reload check, especially if the package is not used with \LaTeX.
%    \begin{macrocode}
\begingroup\catcode61\catcode48\catcode32=10\relax%
  \catcode13=5 % ^^M
  \endlinechar=13 %
  \catcode35=6 % #
  \catcode39=12 % '
  \catcode44=12 % ,
  \catcode45=12 % -
  \catcode46=12 % .
  \catcode58=12 % :
  \catcode64=11 % @
  \catcode123=1 % {
  \catcode125=2 % }
  \expandafter\let\expandafter\x\csname ver@bigintcalc.sty\endcsname
  \ifx\x\relax % plain-TeX, first loading
  \else
    \def\empty{}%
    \ifx\x\empty % LaTeX, first loading,
      % variable is initialized, but \ProvidesPackage not yet seen
    \else
      \expandafter\ifx\csname PackageInfo\endcsname\relax
        \def\x#1#2{%
          \immediate\write-1{Package #1 Info: #2.}%
        }%
      \else
        \def\x#1#2{\PackageInfo{#1}{#2, stopped}}%
      \fi
      \x{bigintcalc}{The package is already loaded}%
      \aftergroup\endinput
    \fi
  \fi
\endgroup%
%    \end{macrocode}
%    Package identification:
%    \begin{macrocode}
\begingroup\catcode61\catcode48\catcode32=10\relax%
  \catcode13=5 % ^^M
  \endlinechar=13 %
  \catcode35=6 % #
  \catcode39=12 % '
  \catcode40=12 % (
  \catcode41=12 % )
  \catcode44=12 % ,
  \catcode45=12 % -
  \catcode46=12 % .
  \catcode47=12 % /
  \catcode58=12 % :
  \catcode64=11 % @
  \catcode91=12 % [
  \catcode93=12 % ]
  \catcode123=1 % {
  \catcode125=2 % }
  \expandafter\ifx\csname ProvidesPackage\endcsname\relax
    \def\x#1#2#3[#4]{\endgroup
      \immediate\write-1{Package: #3 #4}%
      \xdef#1{#4}%
    }%
  \else
    \def\x#1#2[#3]{\endgroup
      #2[{#3}]%
      \ifx#1\@undefined
        \xdef#1{#3}%
      \fi
      \ifx#1\relax
        \xdef#1{#3}%
      \fi
    }%
  \fi
\expandafter\x\csname ver@bigintcalc.sty\endcsname
\ProvidesPackage{bigintcalc}%
  [2016/05/16 v1.4 Expandable calculations on big integers (HO)]%
%    \end{macrocode}
%
% \subsection{Catcodes}
%
%    \begin{macrocode}
\begingroup\catcode61\catcode48\catcode32=10\relax%
  \catcode13=5 % ^^M
  \endlinechar=13 %
  \catcode123=1 % {
  \catcode125=2 % }
  \catcode64=11 % @
  \def\x{\endgroup
    \expandafter\edef\csname BIC@AtEnd\endcsname{%
      \endlinechar=\the\endlinechar\relax
      \catcode13=\the\catcode13\relax
      \catcode32=\the\catcode32\relax
      \catcode35=\the\catcode35\relax
      \catcode61=\the\catcode61\relax
      \catcode64=\the\catcode64\relax
      \catcode123=\the\catcode123\relax
      \catcode125=\the\catcode125\relax
    }%
  }%
\x\catcode61\catcode48\catcode32=10\relax%
\catcode13=5 % ^^M
\endlinechar=13 %
\catcode35=6 % #
\catcode64=11 % @
\catcode123=1 % {
\catcode125=2 % }
\def\TMP@EnsureCode#1#2{%
  \edef\BIC@AtEnd{%
    \BIC@AtEnd
    \catcode#1=\the\catcode#1\relax
  }%
  \catcode#1=#2\relax
}
\TMP@EnsureCode{33}{12}% !
\TMP@EnsureCode{36}{14}% $ (comment!)
\TMP@EnsureCode{38}{14}% & (comment!)
\TMP@EnsureCode{40}{12}% (
\TMP@EnsureCode{41}{12}% )
\TMP@EnsureCode{42}{12}% *
\TMP@EnsureCode{43}{12}% +
\TMP@EnsureCode{45}{12}% -
\TMP@EnsureCode{46}{12}% .
\TMP@EnsureCode{47}{12}% /
\TMP@EnsureCode{58}{11}% : (letter!)
\TMP@EnsureCode{60}{12}% <
\TMP@EnsureCode{62}{12}% >
\TMP@EnsureCode{63}{14}% ? (comment!)
\TMP@EnsureCode{91}{12}% [
\TMP@EnsureCode{93}{12}% ]
\edef\BIC@AtEnd{\BIC@AtEnd\noexpand\endinput}
\begingroup\expandafter\expandafter\expandafter\endgroup
\expandafter\ifx\csname BIC@TestMode\endcsname\relax
\else
  \catcode63=9 % ? (ignore)
\fi
? \let\BIC@@TestMode\BIC@TestMode
%    \end{macrocode}
%
% \subsection{\eTeX\ detection}
%
%    \begin{macrocode}
\begingroup\expandafter\expandafter\expandafter\endgroup
\expandafter\ifx\csname numexpr\endcsname\relax
  \catcode36=9 % $ (ignore)
\else
  \catcode38=9 % & (ignore)
\fi
%    \end{macrocode}
%
% \subsection{Help macros}
%
%    \begin{macro}{\BIC@Fi}
%    \begin{macrocode}
\let\BIC@Fi\fi
%    \end{macrocode}
%    \end{macro}
%    \begin{macro}{\BIC@AfterFi}
%    \begin{macrocode}
\def\BIC@AfterFi#1#2\BIC@Fi{\fi#1}%
%    \end{macrocode}
%    \end{macro}
%    \begin{macro}{\BIC@AfterFiFi}
%    \begin{macrocode}
\def\BIC@AfterFiFi#1#2\BIC@Fi{\fi\fi#1}%
%    \end{macrocode}
%    \end{macro}
%    \begin{macro}{\BIC@AfterFiFiFi}
%    \begin{macrocode}
\def\BIC@AfterFiFiFi#1#2\BIC@Fi{\fi\fi\fi#1}%
%    \end{macrocode}
%    \end{macro}
%
%    \begin{macro}{\BIC@Space}
%    \begin{macrocode}
\begingroup
  \def\x#1{\endgroup
    \let\BIC@Space= #1%
  }%
\x{ }
%    \end{macrocode}
%    \end{macro}
%
% \subsection{Expand number}
%
%    \begin{macrocode}
\begingroup\expandafter\expandafter\expandafter\endgroup
\expandafter\ifx\csname RequirePackage\endcsname\relax
  \def\TMP@RequirePackage#1[#2]{%
    \begingroup\expandafter\expandafter\expandafter\endgroup
    \expandafter\ifx\csname ver@#1.sty\endcsname\relax
      \input #1.sty\relax
    \fi
  }%
  \TMP@RequirePackage{pdftexcmds}[2007/11/11]%
\else
  \RequirePackage{pdftexcmds}[2007/11/11]%
\fi
%    \end{macrocode}
%
%    \begin{macrocode}
\begingroup\expandafter\expandafter\expandafter\endgroup
\expandafter\ifx\csname pdf@escapehex\endcsname\relax
%    \end{macrocode}
%
%    \begin{macro}{\BIC@Expand}
%    \begin{macrocode}
  \def\BIC@Expand#1{%
    \romannumeral0%
    \BIC@@Expand#1!\@nil{}%
  }%
%    \end{macrocode}
%    \end{macro}
%    \begin{macro}{\BIC@@Expand}
%    \begin{macrocode}
  \def\BIC@@Expand#1#2\@nil#3{%
    \expandafter\ifcat\noexpand#1\relax
      \expandafter\@firstoftwo
    \else
      \expandafter\@secondoftwo
    \fi
    {%
      \expandafter\BIC@@Expand#1#2\@nil{#3}%
    }{%
      \ifx#1!%
        \expandafter\@firstoftwo
      \else
        \expandafter\@secondoftwo
      \fi
      { #3}{%
        \BIC@@Expand#2\@nil{#3#1}%
      }%
    }%
  }%
%    \end{macrocode}
%    \end{macro}
%    \begin{macro}{\@firstoftwo}
%    \begin{macrocode}
  \expandafter\ifx\csname @firstoftwo\endcsname\relax
    \long\def\@firstoftwo#1#2{#1}%
  \fi
%    \end{macrocode}
%    \end{macro}
%    \begin{macro}{\@secondoftwo}
%    \begin{macrocode}
  \expandafter\ifx\csname @secondoftwo\endcsname\relax
    \long\def\@secondoftwo#1#2{#2}%
  \fi
%    \end{macrocode}
%    \end{macro}
%
%    \begin{macrocode}
\else
%    \end{macrocode}
%    \begin{macro}{\BIC@Expand}
%    \begin{macrocode}
  \def\BIC@Expand#1{%
    \romannumeral0\expandafter\expandafter\expandafter\BIC@Space
    \pdf@unescapehex{%
      \expandafter\expandafter\expandafter
      \BIC@StripHexSpace\pdf@escapehex{#1}20\@nil
    }%
  }%
%    \end{macrocode}
%    \end{macro}
%    \begin{macro}{\BIC@StripHexSpace}
%    \begin{macrocode}
  \def\BIC@StripHexSpace#120#2\@nil{%
    #1%
    \ifx\\#2\\%
    \else
      \BIC@AfterFi{%
        \BIC@StripHexSpace#2\@nil
      }%
    \BIC@Fi
  }%
%    \end{macrocode}
%    \end{macro}
%    \begin{macrocode}
\fi
%    \end{macrocode}
%
% \subsection{Normalize expanded number}
%
%    \begin{macro}{\BIC@Normalize}
%    |#1|: result sign\\
%    |#2|: first token of number
%    \begin{macrocode}
\def\BIC@Normalize#1#2{%
  \ifx#2-%
    \ifx\\#1\\%
      \BIC@AfterFiFi{%
        \BIC@Normalize-%
      }%
    \else
      \BIC@AfterFiFi{%
        \BIC@Normalize{}%
      }%
    \fi
  \else
    \ifx#2+%
      \BIC@AfterFiFi{%
        \BIC@Normalize{#1}%
      }%
    \else
      \ifx#20%
        \BIC@AfterFiFiFi{%
          \BIC@NormalizeZero{#1}%
        }%
      \else
        \BIC@AfterFiFiFi{%
          \BIC@NormalizeDigits#1#2%
        }%
      \fi
    \fi
  \BIC@Fi
}
%    \end{macrocode}
%    \end{macro}
%    \begin{macro}{\BIC@NormalizeZero}
%    \begin{macrocode}
\def\BIC@NormalizeZero#1#2{%
  \ifx#2!%
    \BIC@AfterFi{ 0}%
  \else
    \ifx#20%
      \BIC@AfterFiFi{%
        \BIC@NormalizeZero{#1}%
      }%
    \else
      \BIC@AfterFiFi{%
        \BIC@NormalizeDigits#1#2%
      }%
    \fi
  \BIC@Fi
}
%    \end{macrocode}
%    \end{macro}
%    \begin{macro}{\BIC@NormalizeDigits}
%    \begin{macrocode}
\def\BIC@NormalizeDigits#1!{ #1}
%    \end{macrocode}
%    \end{macro}
%
% \subsection{\op{Num}}
%
%    \begin{macro}{\bigintcalcNum}
%    \begin{macrocode}
\def\bigintcalcNum#1{%
  \romannumeral0%
  \expandafter\expandafter\expandafter\BIC@Normalize
  \expandafter\expandafter\expandafter{%
  \expandafter\expandafter\expandafter}%
  \BIC@Expand{#1}!%
}
%    \end{macrocode}
%    \end{macro}
%
% \subsection{\op{Inv}, \op{Abs}, \op{Sgn}}
%
%
%    \begin{macro}{\bigintcalcInv}
%    \begin{macrocode}
\def\bigintcalcInv#1{%
  \romannumeral0\expandafter\expandafter\expandafter\BIC@Space
  \bigintcalcNum{-#1}%
}
%    \end{macrocode}
%    \end{macro}
%
%    \begin{macro}{\bigintcalcAbs}
%    \begin{macrocode}
\def\bigintcalcAbs#1{%
  \romannumeral0%
  \expandafter\expandafter\expandafter\BIC@Abs
  \bigintcalcNum{#1}%
}
%    \end{macrocode}
%    \end{macro}
%    \begin{macro}{\BIC@Abs}
%    \begin{macrocode}
\def\BIC@Abs#1{%
  \ifx#1-%
    \expandafter\BIC@Space
  \else
    \expandafter\BIC@Space
    \expandafter#1%
  \fi
}
%    \end{macrocode}
%    \end{macro}
%
%    \begin{macro}{\bigintcalcSgn}
%    \begin{macrocode}
\def\bigintcalcSgn#1{%
  \number
  \expandafter\expandafter\expandafter\BIC@Sgn
  \bigintcalcNum{#1}! %
}
%    \end{macrocode}
%    \end{macro}
%    \begin{macro}{\BIC@Sgn}
%    \begin{macrocode}
\def\BIC@Sgn#1#2!{%
  \ifx#1-%
    -1%
  \else
    \ifx#10%
      0%
    \else
      1%
    \fi
  \fi
}
%    \end{macrocode}
%    \end{macro}
%
% \subsection{\op{Cmp}, \op{Min}, \op{Max}}
%
%    \begin{macro}{\bigintcalcCmp}
%    \begin{macrocode}
\def\bigintcalcCmp#1#2{%
  \number
  \expandafter\expandafter\expandafter\BIC@Cmp
  \bigintcalcNum{#2}!{#1}%
}
%    \end{macrocode}
%    \end{macro}
%    \begin{macro}{\BIC@Cmp}
%    \begin{macrocode}
\def\BIC@Cmp#1!#2{%
  \expandafter\expandafter\expandafter\BIC@@Cmp
  \bigintcalcNum{#2}!#1!%
}
%    \end{macrocode}
%    \end{macro}
%    \begin{macro}{\BIC@@Cmp}
%    \begin{macrocode}
\def\BIC@@Cmp#1#2!#3#4!{%
  \ifx#1-%
    \ifx#3-%
      \BIC@AfterFiFi{%
        \BIC@@Cmp#4!#2!%
      }%
    \else
      \BIC@AfterFiFi{%
        -1 %
      }%
    \fi
  \else
    \ifx#3-%
      \BIC@AfterFiFi{%
        1 %
      }%
    \else
      \BIC@AfterFiFi{%
        \BIC@CmpLength#1#2!#3#4!#1#2!#3#4!%
      }%
    \fi
  \BIC@Fi
}
%    \end{macrocode}
%    \end{macro}
%    \begin{macro}{\BIC@PosCmp}
%    \begin{macrocode}
\def\BIC@PosCmp#1!#2!{%
  \BIC@CmpLength#1!#2!#1!#2!%
}
%    \end{macrocode}
%    \end{macro}
%    \begin{macro}{\BIC@CmpLength}
%    \begin{macrocode}
\def\BIC@CmpLength#1#2!#3#4!{%
  \ifx\\#2\\%
    \ifx\\#4\\%
      \BIC@AfterFiFi\BIC@CmpDiff
    \else
      \BIC@AfterFiFi{%
        \BIC@CmpResult{-1}%
      }%
    \fi
  \else
    \ifx\\#4\\%
      \BIC@AfterFiFi{%
        \BIC@CmpResult1%
      }%
    \else
      \BIC@AfterFiFi{%
        \BIC@CmpLength#2!#4!%
      }%
    \fi
  \BIC@Fi
}
%    \end{macrocode}
%    \end{macro}
%    \begin{macro}{\BIC@CmpResult}
%    \begin{macrocode}
\def\BIC@CmpResult#1#2!#3!{#1 }
%    \end{macrocode}
%    \end{macro}
%    \begin{macro}{\BIC@CmpDiff}
%    \begin{macrocode}
\def\BIC@CmpDiff#1#2!#3#4!{%
  \ifnum#1<#3 %
    \BIC@AfterFi{%
      -1 %
    }%
  \else
    \ifnum#1>#3 %
      \BIC@AfterFiFi{%
        1 %
      }%
    \else
      \ifx\\#2\\%
        \BIC@AfterFiFiFi{%
          0 %
        }%
      \else
        \BIC@AfterFiFiFi{%
          \BIC@CmpDiff#2!#4!%
        }%
      \fi
    \fi
  \BIC@Fi
}
%    \end{macrocode}
%    \end{macro}
%
%    \begin{macro}{\bigintcalcMin}
%    \begin{macrocode}
\def\bigintcalcMin#1{%
  \romannumeral0%
  \expandafter\expandafter\expandafter\BIC@MinMax
  \bigintcalcNum{#1}!-!%
}
%    \end{macrocode}
%    \end{macro}
%    \begin{macro}{\bigintcalcMax}
%    \begin{macrocode}
\def\bigintcalcMax#1{%
  \romannumeral0%
  \expandafter\expandafter\expandafter\BIC@MinMax
  \bigintcalcNum{#1}!!%
}
%    \end{macrocode}
%    \end{macro}
%    \begin{macro}{\BIC@MinMax}
%    |#1|: $x$\\
%    |#2|: sign for comparison\\
%    |#3|: $y$
%    \begin{macrocode}
\def\BIC@MinMax#1!#2!#3{%
  \expandafter\expandafter\expandafter\BIC@@MinMax
  \bigintcalcNum{#3}!#1!#2!%
}
%    \end{macrocode}
%    \end{macro}
%    \begin{macro}{\BIC@@MinMax}
%    |#1|: $y$\\
%    |#2|: $x$\\
%    |#3|: sign for comparison
%    \begin{macrocode}
\def\BIC@@MinMax#1!#2!#3!{%
  \ifnum\BIC@@Cmp#1!#2!=#31 %
    \BIC@AfterFi{ #1}%
  \else
    \BIC@AfterFi{ #2}%
  \BIC@Fi
}
%    \end{macrocode}
%    \end{macro}
%
% \subsection{\op{Odd}}
%
%    \begin{macro}{\bigintcalcOdd}
%    \begin{macrocode}
\def\bigintcalcOdd#1{%
  \romannumeral0%
  \expandafter\expandafter\expandafter\BIC@Odd
  \bigintcalcAbs{#1}!%
}
%    \end{macrocode}
%    \end{macro}
%    \begin{macro}{\BigIntCalcOdd}
%    \begin{macrocode}
\def\BigIntCalcOdd#1!{%
  \romannumeral0%
  \BIC@Odd#1!%
}
%    \end{macrocode}
%    \end{macro}
%    \begin{macro}{\BIC@Odd}
%    |#1|: $x$
%    \begin{macrocode}
\def\BIC@Odd#1#2{%
  \ifx#2!%
    \ifodd#1 %
      \BIC@AfterFiFi{ 1}%
    \else
      \BIC@AfterFiFi{ 0}%
    \fi
  \else
    \expandafter\BIC@Odd\expandafter#2%
  \BIC@Fi
}
%    \end{macrocode}
%    \end{macro}
%
% \subsection{\op{Inc}, \op{Dec}}
%
%    \begin{macro}{\bigintcalcInc}
%    \begin{macrocode}
\def\bigintcalcInc#1{%
  \romannumeral0%
  \expandafter\expandafter\expandafter\BIC@IncSwitch
  \bigintcalcNum{#1}!%
}
%    \end{macrocode}
%    \end{macro}
%    \begin{macro}{\BIC@IncSwitch}
%    \begin{macrocode}
\def\BIC@IncSwitch#1#2!{%
  \ifcase\BIC@@Cmp#1#2!-1!%
    \BIC@AfterFi{ 0}%
  \or
    \BIC@AfterFi{%
      \BIC@Inc#1#2!{}%
    }%
  \else
    \BIC@AfterFi{%
      \expandafter-\romannumeral0%
      \BIC@Dec#2!{}%
    }%
  \BIC@Fi
}
%    \end{macrocode}
%    \end{macro}
%
%    \begin{macro}{\bigintcalcDec}
%    \begin{macrocode}
\def\bigintcalcDec#1{%
  \romannumeral0%
  \expandafter\expandafter\expandafter\BIC@DecSwitch
  \bigintcalcNum{#1}!%
}
%    \end{macrocode}
%    \end{macro}
%    \begin{macro}{\BIC@DecSwitch}
%    \begin{macrocode}
\def\BIC@DecSwitch#1#2!{%
  \ifcase\BIC@Sgn#1#2! %
    \BIC@AfterFi{ -1}%
  \or
    \BIC@AfterFi{%
      \BIC@Dec#1#2!{}%
    }%
  \else
    \BIC@AfterFi{%
      \expandafter-\romannumeral0%
      \BIC@Inc#2!{}%
    }%
  \BIC@Fi
}
%    \end{macrocode}
%    \end{macro}
%
%    \begin{macro}{\BigIntCalcInc}
%    \begin{macrocode}
\def\BigIntCalcInc#1!{%
  \romannumeral0\BIC@Inc#1!{}%
}
%    \end{macrocode}
%    \end{macro}
%    \begin{macro}{\BigIntCalcDec}
%    \begin{macrocode}
\def\BigIntCalcDec#1!{%
  \romannumeral0\BIC@Dec#1!{}%
}
%    \end{macrocode}
%    \end{macro}
%
%    \begin{macro}{\BIC@Inc}
%    \begin{macrocode}
\def\BIC@Inc#1#2!#3{%
  \ifx\\#2\\%
    \BIC@AfterFi{%
      \BIC@@Inc1#1#3!{}%
    }%
  \else
    \BIC@AfterFi{%
      \BIC@Inc#2!{#1#3}%
    }%
  \BIC@Fi
}
%    \end{macrocode}
%    \end{macro}
%    \begin{macro}{\BIC@@Inc}
%    \begin{macrocode}
\def\BIC@@Inc#1#2#3!#4{%
  \ifcase#1 %
    \ifx\\#3\\%
      \BIC@AfterFiFi{ #2#4}%
    \else
      \BIC@AfterFiFi{%
        \BIC@@Inc0#3!{#2#4}%
      }%
    \fi
  \else
    \ifnum#2<9 %
      \BIC@AfterFiFi{%
&       \expandafter\BIC@@@Inc\the\numexpr#2+1\relax
$       \expandafter\expandafter\expandafter\BIC@@@Inc
$       \ifcase#2 \expandafter1%
$       \or\expandafter2%
$       \or\expandafter3%
$       \or\expandafter4%
$       \or\expandafter5%
$       \or\expandafter6%
$       \or\expandafter7%
$       \or\expandafter8%
$       \or\expandafter9%
$?      \else\BigIntCalcError:ThisCannotHappen%
$       \fi
        0#3!{#4}%
      }%
    \else
      \BIC@AfterFiFi{%
        \BIC@@@Inc01#3!{#4}%
      }%
    \fi
  \BIC@Fi
}
%    \end{macrocode}
%    \end{macro}
%    \begin{macro}{\BIC@@@Inc}
%    \begin{macrocode}
\def\BIC@@@Inc#1#2#3!#4{%
  \ifx\\#3\\%
    \ifnum#2=1 %
      \BIC@AfterFiFi{ 1#1#4}%
    \else
      \BIC@AfterFiFi{ #1#4}%
    \fi
  \else
    \BIC@AfterFi{%
      \BIC@@Inc#2#3!{#1#4}%
    }%
  \BIC@Fi
}
%    \end{macrocode}
%    \end{macro}
%
%    \begin{macro}{\BIC@Dec}
%    \begin{macrocode}
\def\BIC@Dec#1#2!#3{%
  \ifx\\#2\\%
    \BIC@AfterFi{%
      \BIC@@Dec1#1#3!{}%
    }%
  \else
    \BIC@AfterFi{%
      \BIC@Dec#2!{#1#3}%
    }%
  \BIC@Fi
}
%    \end{macrocode}
%    \end{macro}
%    \begin{macro}{\BIC@@Dec}
%    \begin{macrocode}
\def\BIC@@Dec#1#2#3!#4{%
  \ifcase#1 %
    \ifx\\#3\\%
      \BIC@AfterFiFi{ #2#4}%
    \else
      \BIC@AfterFiFi{%
        \BIC@@Dec0#3!{#2#4}%
      }%
    \fi
  \else
    \ifnum#2>0 %
      \BIC@AfterFiFi{%
&       \expandafter\BIC@@@Dec\the\numexpr#2-1\relax
$       \expandafter\expandafter\expandafter\BIC@@@Dec
$       \ifcase#2
$?        \BigIntCalcError:ThisCannotHappen%
$       \or\expandafter0%
$       \or\expandafter1%
$       \or\expandafter2%
$       \or\expandafter3%
$       \or\expandafter4%
$       \or\expandafter5%
$       \or\expandafter6%
$       \or\expandafter7%
$       \or\expandafter8%
$?      \else\BigIntCalcError:ThisCannotHappen%
$       \fi
        0#3!{#4}%
      }%
    \else
      \BIC@AfterFiFi{%
        \BIC@@@Dec91#3!{#4}%
      }%
    \fi
  \BIC@Fi
}
%    \end{macrocode}
%    \end{macro}
%    \begin{macro}{\BIC@@@Dec}
%    \begin{macrocode}
\def\BIC@@@Dec#1#2#3!#4{%
  \ifx\\#3\\%
    \ifcase#1 %
      \ifx\\#4\\%
        \BIC@AfterFiFiFi{ 0}%
      \else
        \BIC@AfterFiFiFi{ #4}%
      \fi
    \else
      \BIC@AfterFiFi{ #1#4}%
    \fi
  \else
    \BIC@AfterFi{%
      \BIC@@Dec#2#3!{#1#4}%
    }%
  \BIC@Fi
}
%    \end{macrocode}
%    \end{macro}
%
% \subsection{\op{Add}, \op{Sub}}
%
%    \begin{macro}{\bigintcalcAdd}
%    \begin{macrocode}
\def\bigintcalcAdd#1{%
  \romannumeral0%
  \expandafter\expandafter\expandafter\BIC@Add
  \bigintcalcNum{#1}!%
}
%    \end{macrocode}
%    \end{macro}
%    \begin{macro}{\BIC@Add}
%    \begin{macrocode}
\def\BIC@Add#1!#2{%
  \expandafter\expandafter\expandafter
  \BIC@AddSwitch\bigintcalcNum{#2}!#1!%
}
%    \end{macrocode}
%    \end{macro}
%    \begin{macro}{\bigintcalcSub}
%    \begin{macrocode}
\def\bigintcalcSub#1#2{%
  \romannumeral0%
  \expandafter\expandafter\expandafter\BIC@Add
  \bigintcalcNum{-#2}!{#1}%
}
%    \end{macrocode}
%    \end{macro}
%
%    \begin{macro}{\BIC@AddSwitch}
%    Decision table for \cs{BIC@AddSwitch}.
%    \begin{quote}
%    \begin{tabular}[t]{@{}|l|l|l|l|l|@{}}
%      \hline
%      $x<0$ & $y<0$ & $-x>-y$ & $-$ & $\opAdd(-x,-y)$\\
%      \cline{3-3}\cline{5-5}
%            &       & else    &     & $\opAdd(-y,-x)$\\
%      \cline{2-5}
%            & else  & $-x> y$ & $-$ & $\opSub(-x, y)$\\
%      \cline{3-5}
%            &       & $-x= y$ &     & $0$\\
%      \cline{3-5}
%            &       & else    & $+$ & $\opSub( y,-x)$\\
%      \hline
%      else  & $y<0$ & $ x>-y$ & $+$ & $\opSub( x,-y)$\\
%      \cline{3-5}
%            &       & $ x=-y$ &     & $0$\\
%      \cline{3-5}
%            &       & else    & $-$ & $\opSub(-y, x)$\\
%      \cline{2-5}
%            & else  & $ x> y$ & $+$ & $\opAdd( x, y)$\\
%      \cline{3-3}\cline{5-5}
%            &       & else    &     & $\opAdd( y, x)$\\
%      \hline
%    \end{tabular}
%    \end{quote}
%    \begin{macrocode}
\def\BIC@AddSwitch#1#2!#3#4!{%
  \ifx#1-% x < 0
    \ifx#3-% y < 0
      \expandafter-\romannumeral0%
      \ifnum\BIC@PosCmp#2!#4!=1 % -x > -y
        \BIC@AfterFiFiFi{%
          \BIC@AddXY#2!#4!!!%
        }%
      \else % -x <= -y
        \BIC@AfterFiFiFi{%
          \BIC@AddXY#4!#2!!!%
        }%
      \fi
    \else % y >= 0
      \ifcase\BIC@PosCmp#2!#3#4!% -x = y
        \BIC@AfterFiFiFi{ 0}%
      \or % -x > y
        \expandafter-\romannumeral0%
        \BIC@AfterFiFiFi{%
          \BIC@SubXY#2!#3#4!!!%
        }%
      \else % -x <= y
        \BIC@AfterFiFiFi{%
          \BIC@SubXY#3#4!#2!!!%
        }%
      \fi
    \fi
  \else % x >= 0
    \ifx#3-% y < 0
      \ifcase\BIC@PosCmp#1#2!#4!% x = -y
        \BIC@AfterFiFiFi{ 0}%
      \or % x > -y
        \BIC@AfterFiFiFi{%
          \BIC@SubXY#1#2!#4!!!%
        }%
      \else % x <= -y
        \expandafter-\romannumeral0%
        \BIC@AfterFiFiFi{%
          \BIC@SubXY#4!#1#2!!!%
        }%
      \fi
    \else % y >= 0
      \ifnum\BIC@PosCmp#1#2!#3#4!=1 % x > y
        \BIC@AfterFiFiFi{%
          \BIC@AddXY#1#2!#3#4!!!%
        }%
      \else % x <= y
        \BIC@AfterFiFiFi{%
          \BIC@AddXY#3#4!#1#2!!!%
        }%
      \fi
    \fi
  \BIC@Fi
}
%    \end{macrocode}
%    \end{macro}
%
%    \begin{macro}{\BigIntCalcAdd}
%    \begin{macrocode}
\def\BigIntCalcAdd#1!#2!{%
  \romannumeral0\BIC@AddXY#1!#2!!!%
}
%    \end{macrocode}
%    \end{macro}
%    \begin{macro}{\BigIntCalcSub}
%    \begin{macrocode}
\def\BigIntCalcSub#1!#2!{%
  \romannumeral0\BIC@SubXY#1!#2!!!%
}
%    \end{macrocode}
%    \end{macro}
%
%    \begin{macro}{\BIC@AddXY}
%    \begin{macrocode}
\def\BIC@AddXY#1#2!#3#4!#5!#6!{%
  \ifx\\#2\\%
    \ifx\\#3\\%
      \BIC@AfterFiFi{%
        \BIC@DoAdd0!#1#5!#60!%
      }%
    \else
      \BIC@AfterFiFi{%
        \BIC@DoAdd0!#1#5!#3#6!%
      }%
    \fi
  \else
    \ifx\\#4\\%
      \ifx\\#3\\%
        \BIC@AfterFiFiFi{%
          \BIC@AddXY#2!{}!#1#5!#60!%
        }%
      \else
        \BIC@AfterFiFiFi{%
          \BIC@AddXY#2!{}!#1#5!#3#6!%
        }%
      \fi
    \else
      \BIC@AfterFiFi{%
        \BIC@AddXY#2!#4!#1#5!#3#6!%
      }%
    \fi
  \BIC@Fi
}
%    \end{macrocode}
%    \end{macro}
%    \begin{macro}{\BIC@DoAdd}
%    |#1|: carry\\
%    |#2|: reverted result\\
%    |#3#4|: reverted $x$\\
%    |#5#6|: reverted $y$
%    \begin{macrocode}
\def\BIC@DoAdd#1#2!#3#4!#5#6!{%
  \ifx\\#4\\%
    \BIC@AfterFi{%
&     \expandafter\BIC@Space
&     \the\numexpr#1+#3+#5\relax#2%
$     \expandafter\expandafter\expandafter\BIC@AddResult
$     \BIC@AddDigit#1#3#5#2%
    }%
  \else
    \BIC@AfterFi{%
      \expandafter\expandafter\expandafter\BIC@DoAdd
      \BIC@AddDigit#1#3#5#2!#4!#6!%
    }%
  \BIC@Fi
}
%    \end{macrocode}
%    \end{macro}
%    \begin{macro}{\BIC@AddResult}
%    \begin{macrocode}
$ \def\BIC@AddResult#1{%
$   \ifx#10%
$     \expandafter\BIC@Space
$   \else
$     \expandafter\BIC@Space\expandafter#1%
$   \fi
$ }%
%    \end{macrocode}
%    \end{macro}
%    \begin{macro}{\BIC@AddDigit}
%    |#1|: carry\\
%    |#2|: digit of $x$\\
%    |#3|: digit of $y$
%    \begin{macrocode}
\def\BIC@AddDigit#1#2#3{%
  \romannumeral0%
& \expandafter\BIC@@AddDigit\the\numexpr#1+#2+#3!%
$ \expandafter\BIC@@AddDigit\number%
$ \csname
$   BIC@AddCarry%
$   \ifcase#1 %
$     #2%
$   \else
$     \ifcase#2 1\or2\or3\or4\or5\or6\or7\or8\or9\or10\fi
$   \fi
$ \endcsname#3!%
}
%    \end{macrocode}
%    \end{macro}
%    \begin{macro}{\BIC@@AddDigit}
%    \begin{macrocode}
\def\BIC@@AddDigit#1!{%
  \ifnum#1<10 %
    \BIC@AfterFi{ 0#1}%
  \else
    \BIC@AfterFi{ #1}%
  \BIC@Fi
}
%    \end{macrocode}
%    \end{macro}
%    \begin{macro}{\BIC@AddCarry0}
%    \begin{macrocode}
$ \expandafter\def\csname BIC@AddCarry0\endcsname#1{#1}%
%    \end{macrocode}
%    \end{macro}
%    \begin{macro}{\BIC@AddCarry10}
%    \begin{macrocode}
$ \expandafter\def\csname BIC@AddCarry10\endcsname#1{1#1}%
%    \end{macrocode}
%    \end{macro}
%    \begin{macro}{\BIC@AddCarry[1-9]}
%    \begin{macrocode}
$ \def\BIC@Temp#1#2{%
$   \expandafter\def\csname BIC@AddCarry#1\endcsname##1{%
$     \ifcase##1 #1\or
$     #2%
$?    \else\BigIntCalcError:ThisCannotHappen%
$     \fi
$   }%
$ }%
$ \BIC@Temp 0{1\or2\or3\or4\or5\or6\or7\or8\or9}%
$ \BIC@Temp 1{2\or3\or4\or5\or6\or7\or8\or9\or10}%
$ \BIC@Temp 2{3\or4\or5\or6\or7\or8\or9\or10\or11}%
$ \BIC@Temp 3{4\or5\or6\or7\or8\or9\or10\or11\or12}%
$ \BIC@Temp 4{5\or6\or7\or8\or9\or10\or11\or12\or13}%
$ \BIC@Temp 5{6\or7\or8\or9\or10\or11\or12\or13\or14}%
$ \BIC@Temp 6{7\or8\or9\or10\or11\or12\or13\or14\or15}%
$ \BIC@Temp 7{8\or9\or10\or11\or12\or13\or14\or15\or16}%
$ \BIC@Temp 8{9\or10\or11\or12\or13\or14\or15\or16\or17}%
$ \BIC@Temp 9{10\or11\or12\or13\or14\or15\or16\or17\or18}%
%    \end{macrocode}
%    \end{macro}
%
%    \begin{macro}{\BIC@SubXY}
%    Preconditions:
%    \begin{itemize}
%    \item $x > y$, $x \geq 0$, and $y >= 0$
%    \item digits($x$) = digits($y$)
%    \end{itemize}
%    \begin{macrocode}
\def\BIC@SubXY#1#2!#3#4!#5!#6!{%
  \ifx\\#2\\%
    \ifx\\#3\\%
      \BIC@AfterFiFi{%
        \BIC@DoSub0!#1#5!#60!%
      }%
    \else
      \BIC@AfterFiFi{%
        \BIC@DoSub0!#1#5!#3#6!%
      }%
    \fi
  \else
    \ifx\\#4\\%
      \ifx\\#3\\%
        \BIC@AfterFiFiFi{%
          \BIC@SubXY#2!{}!#1#5!#60!%
        }%
      \else
        \BIC@AfterFiFiFi{%
          \BIC@SubXY#2!{}!#1#5!#3#6!%
        }%
      \fi
    \else
      \BIC@AfterFiFi{%
        \BIC@SubXY#2!#4!#1#5!#3#6!%
      }%
    \fi
  \BIC@Fi
}
%    \end{macrocode}
%    \end{macro}
%    \begin{macro}{\BIC@DoSub}
%    |#1|: carry\\
%    |#2|: reverted result\\
%    |#3#4|: reverted x\\
%    |#5#6|: reverted y
%    \begin{macrocode}
\def\BIC@DoSub#1#2!#3#4!#5#6!{%
  \ifx\\#4\\%
    \BIC@AfterFi{%
      \expandafter\expandafter\expandafter\BIC@SubResult
      \BIC@SubDigit#1#3#5#2%
    }%
  \else
    \BIC@AfterFi{%
      \expandafter\expandafter\expandafter\BIC@DoSub
      \BIC@SubDigit#1#3#5#2!#4!#6!%
    }%
  \BIC@Fi
}
%    \end{macrocode}
%    \end{macro}
%    \begin{macro}{\BIC@SubResult}
%    \begin{macrocode}
\def\BIC@SubResult#1{%
  \ifx#10%
    \expandafter\BIC@SubResult
  \else
    \expandafter\BIC@Space\expandafter#1%
  \fi
}
%    \end{macrocode}
%    \end{macro}
%    \begin{macro}{\BIC@SubDigit}
%    |#1|: carry\\
%    |#2|: digit of $x$\\
%    |#3|: digit of $y$
%    \begin{macrocode}
\def\BIC@SubDigit#1#2#3{%
  \romannumeral0%
& \expandafter\BIC@@SubDigit\the\numexpr#2-#3-#1!%
$ \expandafter\BIC@@AddDigit\number
$   \csname
$     BIC@SubCarry%
$     \ifcase#1 %
$       #3%
$     \else
$       \ifcase#3 1\or2\or3\or4\or5\or6\or7\or8\or9\or10\fi
$     \fi
$   \endcsname#2!%
}
%    \end{macrocode}
%    \end{macro}
%    \begin{macro}{\BIC@@SubDigit}
%    \begin{macrocode}
& \def\BIC@@SubDigit#1!{%
&   \ifnum#1<0 %
&     \BIC@AfterFi{%
&       \expandafter\BIC@Space
&       \expandafter1\the\numexpr#1+10\relax
&     }%
&   \else
&     \BIC@AfterFi{ 0#1}%
&   \BIC@Fi
& }%
%    \end{macrocode}
%    \end{macro}
%    \begin{macro}{\BIC@SubCarry0}
%    \begin{macrocode}
$ \expandafter\def\csname BIC@SubCarry0\endcsname#1{#1}%
%    \end{macrocode}
%    \end{macro}
%    \begin{macro}{\BIC@SubCarry10}%
%    \begin{macrocode}
$ \expandafter\def\csname BIC@SubCarry10\endcsname#1{1#1}%
%    \end{macrocode}
%    \end{macro}
%    \begin{macro}{\BIC@SubCarry[1-9]}
%    \begin{macrocode}
$ \def\BIC@Temp#1#2{%
$   \expandafter\def\csname BIC@SubCarry#1\endcsname##1{%
$     \ifcase##1 #2%
$?    \else\BigIntCalcError:ThisCannotHappen%
$     \fi
$   }%
$ }%
$ \BIC@Temp 1{19\or0\or1\or2\or3\or4\or5\or6\or7\or8}%
$ \BIC@Temp 2{18\or19\or0\or1\or2\or3\or4\or5\or6\or7}%
$ \BIC@Temp 3{17\or18\or19\or0\or1\or2\or3\or4\or5\or6}%
$ \BIC@Temp 4{16\or17\or18\or19\or0\or1\or2\or3\or4\or5}%
$ \BIC@Temp 5{15\or16\or17\or18\or19\or0\or1\or2\or3\or4}%
$ \BIC@Temp 6{14\or15\or16\or17\or18\or19\or0\or1\or2\or3}%
$ \BIC@Temp 7{13\or14\or15\or16\or17\or18\or19\or0\or1\or2}%
$ \BIC@Temp 8{12\or13\or14\or15\or16\or17\or18\or19\or0\or1}%
$ \BIC@Temp 9{11\or12\or13\or14\or15\or16\or17\or18\or19\or0}%
%    \end{macrocode}
%    \end{macro}
%
% \subsection{\op{Shl}, \op{Shr}}
%
%    \begin{macro}{\bigintcalcShl}
%    \begin{macrocode}
\def\bigintcalcShl#1{%
  \romannumeral0%
  \expandafter\expandafter\expandafter\BIC@Shl
  \bigintcalcNum{#1}!%
}
%    \end{macrocode}
%    \end{macro}
%    \begin{macro}{\BIC@Shl}
%    \begin{macrocode}
\def\BIC@Shl#1#2!{%
  \ifx#1-%
    \BIC@AfterFi{%
      \expandafter-\romannumeral0%
&     \BIC@@Shl#2!!%
$     \BIC@AddXY#2!#2!!!%
    }%
  \else
    \BIC@AfterFi{%
&     \BIC@@Shl#1#2!!%
$     \BIC@AddXY#1#2!#1#2!!!%
    }%
  \BIC@Fi
}
%    \end{macrocode}
%    \end{macro}
%    \begin{macro}{\BigIntCalcShl}
%    \begin{macrocode}
\def\BigIntCalcShl#1!{%
  \romannumeral0%
& \BIC@@Shl#1!!%
$ \BIC@AddXY#1!#1!!!%
}
%    \end{macrocode}
%    \end{macro}
%    \begin{macro}{\BIC@@Shl}
%    \begin{macrocode}
& \def\BIC@@Shl#1#2!{%
&   \ifx\\#2\\%
&     \BIC@AfterFi{%
&       \BIC@@@Shl0!#1%
&     }%
&   \else
&     \BIC@AfterFi{%
&       \BIC@@Shl#2!#1%
&     }%
&   \BIC@Fi
& }%
%    \end{macrocode}
%    \end{macro}
%    \begin{macro}{\BIC@@@Shl}
%    |#1|: carry\\
%    |#2|: result\\
%    |#3#4|: reverted number
%    \begin{macrocode}
& \def\BIC@@@Shl#1#2!#3#4!{%
&   \ifx\\#4\\%
&     \BIC@AfterFi{%
&       \expandafter\BIC@Space
&       \the\numexpr#3*2+#1\relax#2%
&     }%
&   \else
&     \BIC@AfterFi{%
&       \expandafter\BIC@@@@Shl\the\numexpr#3*2+#1!#2!#4!%
&     }%
&   \BIC@Fi
& }%
%    \end{macrocode}
%    \end{macro}
%    \begin{macro}{\BIC@@@@Shl}
%    \begin{macrocode}
& \def\BIC@@@@Shl#1!{%
&   \ifnum#1<10 %
&     \BIC@AfterFi{%
&       \BIC@@@Shl0#1%
&     }%
&   \else
&     \BIC@AfterFi{%
&       \BIC@@@Shl#1%
&     }%
&   \BIC@Fi
& }%
%    \end{macrocode}
%    \end{macro}
%
%    \begin{macro}{\bigintcalcShr}
%    \begin{macrocode}
\def\bigintcalcShr#1{%
  \romannumeral0%
  \expandafter\expandafter\expandafter\BIC@Shr
  \bigintcalcNum{#1}!%
}
%    \end{macrocode}
%    \end{macro}
%    \begin{macro}{\BIC@Shr}
%    \begin{macrocode}
\def\BIC@Shr#1#2!{%
  \ifx#1-%
    \expandafter-\romannumeral0%
    \BIC@AfterFi{%
      \BIC@@Shr#2!%
    }%
  \else
    \BIC@AfterFi{%
      \BIC@@Shr#1#2!%
    }%
  \BIC@Fi
}
%    \end{macrocode}
%    \end{macro}
%    \begin{macro}{\BigIntCalcShr}
%    \begin{macrocode}
\def\BigIntCalcShr#1!{%
  \romannumeral0%
  \BIC@@Shr#1!%
}
%    \end{macrocode}
%    \end{macro}
%    \begin{macro}{\BIC@@Shr}
%    \begin{macrocode}
\def\BIC@@Shr#1#2!{%
  \ifcase#1 %
    \BIC@AfterFi{ 0}%
  \or
    \ifx\\#2\\%
      \BIC@AfterFiFi{ 0}%
    \else
      \BIC@AfterFiFi{%
        \BIC@@@Shr#1#2!!%
      }%
    \fi
  \else
    \BIC@AfterFi{%
      \BIC@@@Shr0#1#2!!%
    }%
  \BIC@Fi
}
%    \end{macrocode}
%    \end{macro}
%    \begin{macro}{\BIC@@@Shr}
%    |#1|: carry\\
%    |#2#3|: number\\
%    |#4|: result
%    \begin{macrocode}
\def\BIC@@@Shr#1#2#3!#4!{%
  \ifx\\#3\\%
    \ifodd#1#2 %
      \BIC@AfterFiFi{%
&       \expandafter\BIC@ShrResult\the\numexpr(#1#2-1)/2\relax
$       \expandafter\expandafter\expandafter\BIC@ShrResult
$       \csname BIC@ShrDigit#1#2\endcsname
        #4!%
      }%
    \else
      \BIC@AfterFiFi{%
&       \expandafter\BIC@ShrResult\the\numexpr#1#2/2\relax
$       \expandafter\expandafter\expandafter\BIC@ShrResult
$       \csname BIC@ShrDigit#1#2\endcsname
        #4!%
      }%
    \fi
  \else
    \ifodd#1#2 %
      \BIC@AfterFiFi{%
&       \expandafter\BIC@@@@Shr\the\numexpr(#1#2-1)/2\relax1%
$       \expandafter\expandafter\expandafter\BIC@@@@Shr
$       \csname BIC@ShrDigit#1#2\endcsname
        #3!#4!%
      }%
    \else
      \BIC@AfterFiFi{%
&       \expandafter\BIC@@@@Shr\the\numexpr#1#2/2\relax0%
$       \expandafter\expandafter\expandafter\BIC@@@@Shr
$       \csname BIC@ShrDigit#1#2\endcsname
        #3!#4!%
      }%
    \fi
  \BIC@Fi
}
%    \end{macrocode}
%    \end{macro}
%    \begin{macro}{\BIC@ShrResult}
%    \begin{macrocode}
& \def\BIC@ShrResult#1#2!{ #2#1}%
$ \def\BIC@ShrResult#1#2#3!{ #3#1}%
%    \end{macrocode}
%    \end{macro}
%    \begin{macro}{\BIC@@@@Shr}
%    |#1|: new digit\\
%    |#2|: carry\\
%    |#3|: remaining number\\
%    |#4|: result
%    \begin{macrocode}
\def\BIC@@@@Shr#1#2#3!#4!{%
  \BIC@@@Shr#2#3!#4#1!%
}
%    \end{macrocode}
%    \end{macro}
%    \begin{macro}{\BIC@ShrDigit[00-19]}
%    \begin{macrocode}
$ \def\BIC@Temp#1#2#3#4{%
$   \expandafter\def\csname BIC@ShrDigit#1#2\endcsname{#3#4}%
$ }%
$ \BIC@Temp 0000%
$ \BIC@Temp 0101%
$ \BIC@Temp 0210%
$ \BIC@Temp 0311%
$ \BIC@Temp 0420%
$ \BIC@Temp 0521%
$ \BIC@Temp 0630%
$ \BIC@Temp 0731%
$ \BIC@Temp 0840%
$ \BIC@Temp 0941%
$ \BIC@Temp 1050%
$ \BIC@Temp 1151%
$ \BIC@Temp 1260%
$ \BIC@Temp 1361%
$ \BIC@Temp 1470%
$ \BIC@Temp 1571%
$ \BIC@Temp 1680%
$ \BIC@Temp 1781%
$ \BIC@Temp 1890%
$ \BIC@Temp 1991%
%    \end{macrocode}
%    \end{macro}
%
% \subsection{\cs{BIC@Tim}}
%
%    \begin{macro}{\BIC@Tim}
%    Macro \cs{BIC@Tim} implements ``Number \emph{tim}es digit''.\\
%    |#1|: plain number without sign\\
%    |#2|: digit
\def\BIC@Tim#1!#2{%
  \romannumeral0%
  \ifcase#2 % 0
    \BIC@AfterFi{ 0}%
  \or % 1
    \BIC@AfterFi{ #1}%
  \or % 2
    \BIC@AfterFi{%
      \BIC@Shl#1!%
    }%
  \else % 3-9
    \BIC@AfterFi{%
      \BIC@@Tim#1!!#2%
    }%
  \BIC@Fi
}
%    \begin{macrocode}
%    \end{macrocode}
%    \end{macro}
%    \begin{macro}{\BIC@@Tim}
%    |#1#2|: number\\
%    |#3|: reverted number
%    \begin{macrocode}
\def\BIC@@Tim#1#2!{%
  \ifx\\#2\\%
    \BIC@AfterFi{%
      \BIC@ProcessTim0!#1%
    }%
  \else
    \BIC@AfterFi{%
      \BIC@@Tim#2!#1%
    }%
  \BIC@Fi
}
%    \end{macrocode}
%    \end{macro}
%    \begin{macro}{\BIC@ProcessTim}
%    |#1|: carry\\
%    |#2|: result\\
%    |#3#4|: reverted number\\
%    |#5|: digit
%    \begin{macrocode}
\def\BIC@ProcessTim#1#2!#3#4!#5{%
  \ifx\\#4\\%
    \BIC@AfterFi{%
      \expandafter\BIC@Space
&     \the\numexpr#3*#5+#1\relax
$     \romannumeral0\BIC@TimDigit#3#5#1%
      #2%
    }%
  \else
    \BIC@AfterFi{%
      \expandafter\BIC@@ProcessTim
&     \the\numexpr#3*#5+#1%
$     \romannumeral0\BIC@TimDigit#3#5#1%
      !#2!#4!#5%
    }%
  \BIC@Fi
}
%    \end{macrocode}
%    \end{macro}
%    \begin{macro}{\BIC@@ProcessTim}
%    |#1#2|: carry?, new digit\\
%    |#3|: new number\\
%    |#4|: old number\\
%    |#5|: digit
%    \begin{macrocode}
\def\BIC@@ProcessTim#1#2!{%
  \ifx\\#2\\%
    \BIC@AfterFi{%
      \BIC@ProcessTim0#1%
    }%
  \else
    \BIC@AfterFi{%
      \BIC@ProcessTim#1#2%
    }%
  \BIC@Fi
}
%    \end{macrocode}
%    \end{macro}
%    \begin{macro}{\BIC@TimDigit}
%    |#1|: digit 0--9\\
%    |#2|: digit 3--9\\
%    |#3|: carry 0--9
%    \begin{macrocode}
$ \def\BIC@TimDigit#1#2#3{%
$   \ifcase#1 % 0
$     \BIC@AfterFi{ #3}%
$   \or % 1
$     \BIC@AfterFi{%
$       \expandafter\BIC@Space
$       \number\csname BIC@AddCarry#2\endcsname#3 %
$     }%
$   \else
$     \ifcase#3 %
$       \BIC@AfterFiFi{%
$         \expandafter\BIC@Space
$         \number\csname BIC@MulDigit#2\endcsname#1 %
$       }%
$     \else
$       \BIC@AfterFiFi{%
$         \expandafter\BIC@Space
$         \romannumeral0%
$         \expandafter\BIC@AddXY
$         \number\csname BIC@MulDigit#2\endcsname#1!%
$         #3!!!%
$       }%
$     \fi
$   \BIC@Fi
$ }%
%    \end{macrocode}
%    \end{macro}
%    \begin{macro}{\BIC@MulDigit[3-9]}
%    \begin{macrocode}
$ \def\BIC@Temp#1#2{%
$   \expandafter\def\csname BIC@MulDigit#1\endcsname##1{%
$     \ifcase##1 0%
$     \or ##1%
$     \or #2%
$?    \else\BigIntCalcError:ThisCannotHappen%
$     \fi
$   }%
$ }%
$ \BIC@Temp 3{6\or9\or12\or15\or18\or21\or24\or27}%
$ \BIC@Temp 4{8\or12\or16\or20\or24\or28\or32\or36}%
$ \BIC@Temp 5{10\or15\or20\or25\or30\or35\or40\or45}%
$ \BIC@Temp 6{12\or18\or24\or30\or36\or42\or48\or54}%
$ \BIC@Temp 7{14\or21\or28\or35\or42\or49\or56\or63}%
$ \BIC@Temp 8{16\or24\or32\or40\or48\or56\or64\or72}%
$ \BIC@Temp 9{18\or27\or36\or45\or54\or63\or72\or81}%
%    \end{macrocode}
%    \end{macro}
%
% \subsection{\op{Mul}}
%
%    \begin{macro}{\bigintcalcMul}
%    \begin{macrocode}
\def\bigintcalcMul#1#2{%
  \romannumeral0%
  \expandafter\expandafter\expandafter\BIC@Mul
  \bigintcalcNum{#1}!{#2}%
}
%    \end{macrocode}
%    \end{macro}
%    \begin{macro}{\BIC@Mul}
%    \begin{macrocode}
\def\BIC@Mul#1!#2{%
  \expandafter\expandafter\expandafter\BIC@MulSwitch
  \bigintcalcNum{#2}!#1!%
}
%    \end{macrocode}
%    \end{macro}
%    \begin{macro}{\BIC@MulSwitch}
%    Decision table for \cs{BIC@MulSwitch}.
%    \begin{quote}
%    \begin{tabular}[t]{@{}|l|l|l|l|l|@{}}
%      \hline
%      $x=0$ &\multicolumn{3}{l}{}    &$0$\\
%      \hline
%      $x>0$ & $y=0$ &\multicolumn2l{}&$0$\\
%      \cline{2-5}
%            & $y>0$ & $ x> y$ & $+$ & $\opMul( x, y)$\\
%      \cline{3-3}\cline{5-5}
%            &       & else    &     & $\opMul( y, x)$\\
%      \cline{2-5}
%            & $y<0$ & $ x>-y$ & $-$ & $\opMul( x,-y)$\\
%      \cline{3-3}\cline{5-5}
%            &       & else    &     & $\opMul(-y, x)$\\
%      \hline
%      $x<0$ & $y=0$ &\multicolumn2l{}&$0$\\
%      \cline{2-5}
%            & $y>0$ & $-x> y$ & $-$ & $\opMul(-x, y)$\\
%      \cline{3-3}\cline{5-5}
%            &       & else    &     & $\opMul( y,-x)$\\
%      \cline{2-5}
%            & $y<0$ & $-x>-y$ & $+$ & $\opMul(-x,-y)$\\
%      \cline{3-3}\cline{5-5}
%            &       & else    &     & $\opMul(-y,-x)$\\
%      \hline
%    \end{tabular}
%    \end{quote}
%    \begin{macrocode}
\def\BIC@MulSwitch#1#2!#3#4!{%
  \ifcase\BIC@Sgn#1#2! % x = 0
    \BIC@AfterFi{ 0}%
  \or % x > 0
    \ifcase\BIC@Sgn#3#4! % y = 0
      \BIC@AfterFiFi{ 0}%
    \or % y > 0
      \ifnum\BIC@PosCmp#1#2!#3#4!=1 % x > y
        \BIC@AfterFiFiFi{%
          \BIC@ProcessMul0!#1#2!#3#4!%
        }%
      \else % x <= y
        \BIC@AfterFiFiFi{%
          \BIC@ProcessMul0!#3#4!#1#2!%
        }%
      \fi
    \else % y < 0
      \expandafter-\romannumeral0%
      \ifnum\BIC@PosCmp#1#2!#4!=1 % x > -y
        \BIC@AfterFiFiFi{%
          \BIC@ProcessMul0!#1#2!#4!%
        }%
      \else % x <= -y
        \BIC@AfterFiFiFi{%
          \BIC@ProcessMul0!#4!#1#2!%
        }%
      \fi
    \fi
  \else % x < 0
    \ifcase\BIC@Sgn#3#4! % y = 0
      \BIC@AfterFiFi{ 0}%
    \or % y > 0
      \expandafter-\romannumeral0%
      \ifnum\BIC@PosCmp#2!#3#4!=1 % -x > y
        \BIC@AfterFiFiFi{%
          \BIC@ProcessMul0!#2!#3#4!%
        }%
      \else % -x <= y
        \BIC@AfterFiFiFi{%
          \BIC@ProcessMul0!#3#4!#2!%
        }%
      \fi
    \else % y < 0
      \ifnum\BIC@PosCmp#2!#4!=1 % -x > -y
        \BIC@AfterFiFiFi{%
          \BIC@ProcessMul0!#2!#4!%
        }%
      \else % -x <= -y
        \BIC@AfterFiFiFi{%
          \BIC@ProcessMul0!#4!#2!%
        }%
      \fi
    \fi
  \BIC@Fi
}
%    \end{macrocode}
%    \end{macro}
%    \begin{macro}{\BigIntCalcMul}
%    \begin{macrocode}
\def\BigIntCalcMul#1!#2!{%
  \romannumeral0%
  \BIC@ProcessMul0!#1!#2!%
}
%    \end{macrocode}
%    \end{macro}
%    \begin{macro}{\BIC@ProcessMul}
%    |#1|: result\\
%    |#2|: number $x$\\
%    |#3#4|: number $y$
%    \begin{macrocode}
\def\BIC@ProcessMul#1!#2!#3#4!{%
  \ifx\\#4\\%
    \BIC@AfterFi{%
      \expandafter\expandafter\expandafter\BIC@Space
      \bigintcalcAdd{\BIC@Tim#2!#3}{#10}%
    }%
  \else
    \BIC@AfterFi{%
      \expandafter\expandafter\expandafter\BIC@ProcessMul
      \bigintcalcAdd{\BIC@Tim#2!#3}{#10}!#2!#4!%
    }%
  \BIC@Fi
}
%    \end{macrocode}
%    \end{macro}
%
% \subsection{\op{Sqr}}
%
%    \begin{macro}{\bigintcalcSqr}
%    \begin{macrocode}
\def\bigintcalcSqr#1{%
  \romannumeral0%
  \expandafter\expandafter\expandafter\BIC@Sqr
  \bigintcalcNum{#1}!%
}
%    \end{macrocode}
%    \end{macro}
%    \begin{macro}{\BIC@Sqr}
%    \begin{macrocode}
\def\BIC@Sqr#1{%
   \ifx#1-%
     \expandafter\BIC@@Sqr
   \else
     \expandafter\BIC@@Sqr\expandafter#1%
   \fi
}
%    \end{macrocode}
%    \end{macro}
%    \begin{macro}{\BIC@@Sqr}
%    \begin{macrocode}
\def\BIC@@Sqr#1!{%
  \BIC@ProcessMul0!#1!#1!%
}
%    \end{macrocode}
%    \end{macro}
%
% \subsection{\op{Fac}}
%
%    \begin{macro}{\bigintcalcFac}
%    \begin{macrocode}
\def\bigintcalcFac#1{%
  \romannumeral0%
  \expandafter\expandafter\expandafter\BIC@Fac
  \bigintcalcNum{#1}!%
}
%    \end{macrocode}
%    \end{macro}
%    \begin{macro}{\BIC@Fac}
%    \begin{macrocode}
\def\BIC@Fac#1#2!{%
  \ifx#1-%
    \BIC@AfterFi{ 0\BigIntCalcError:FacNegative}%
  \else
    \ifnum\BIC@PosCmp#1#2!13!<0 %
      \ifcase#1#2 %
         \BIC@AfterFiFiFi{ 1}% 0!
      \or\BIC@AfterFiFiFi{ 1}% 1!
      \or\BIC@AfterFiFiFi{ 2}% 2!
      \or\BIC@AfterFiFiFi{ 6}% 3!
      \or\BIC@AfterFiFiFi{ 24}% 4!
      \or\BIC@AfterFiFiFi{ 120}% 5!
      \or\BIC@AfterFiFiFi{ 720}% 6!
      \or\BIC@AfterFiFiFi{ 5040}% 7!
      \or\BIC@AfterFiFiFi{ 40320}% 8!
      \or\BIC@AfterFiFiFi{ 362880}% 9!
      \or\BIC@AfterFiFiFi{ 3628800}% 10!
      \or\BIC@AfterFiFiFi{ 39916800}% 11!
      \or\BIC@AfterFiFiFi{ 479001600}% 12!
?     \else\BigIntCalcError:ThisCannotHappen%
      \fi
    \else
      \BIC@AfterFiFi{%
        \BIC@ProcessFac#1#2!479001600!%
      }%
    \fi
  \BIC@Fi
}
%    \end{macrocode}
%    \end{macro}
%    \begin{macro}{\BIC@ProcessFac}
%    |#1|: $n$\\
%    |#2|: result
%    \begin{macrocode}
\def\BIC@ProcessFac#1!#2!{%
  \ifnum\BIC@PosCmp#1!12!=0 %
    \BIC@AfterFi{ #2}%
  \else
    \BIC@AfterFi{%
      \expandafter\BIC@@ProcessFac
      \romannumeral0\BIC@ProcessMul0!#2!#1!%
      !#1!%
    }%
  \BIC@Fi
}
%    \end{macrocode}
%    \end{macro}
%    \begin{macro}{\BIC@@ProcessFac}
%    |#1|: result\\
%    |#2|: $n$
%    \begin{macrocode}
\def\BIC@@ProcessFac#1!#2!{%
  \expandafter\BIC@ProcessFac
  \romannumeral0\BIC@Dec#2!{}%
  !#1!%
}
%    \end{macrocode}
%    \end{macro}
%
% \subsection{\op{Pow}}
%
%    \begin{macro}{\bigintcalcPow}
%    |#1|: basis\\
%    |#2|: power
%    \begin{macrocode}
\def\bigintcalcPow#1{%
  \romannumeral0%
  \expandafter\expandafter\expandafter\BIC@Pow
  \bigintcalcNum{#1}!%
}
%    \end{macrocode}
%    \end{macro}
%    \begin{macro}{\BIC@Pow}
%    |#1|: basis\\
%    |#2|: power
%    \begin{macrocode}
\def\BIC@Pow#1!#2{%
  \expandafter\expandafter\expandafter\BIC@PowSwitch
  \bigintcalcNum{#2}!#1!%
}
%    \end{macrocode}
%    \end{macro}
%    \begin{macro}{\BIC@PowSwitch}
%    |#1#2|: power $y$\\
%    |#3#4|: basis $x$\\
%    Decision table for \cs{BIC@PowSwitch}.
%    \begin{quote}
%    \def\M#1{\multicolumn{#1}{l}{}}%
%    \begin{tabular}[t]{@{}|l|l|l|l|@{}}
%    \hline
%    $y=0$ & \M{2}                        & $1$\\
%    \hline
%    $y=1$ & \M{2}                        & $x$\\
%    \hline
%    $y=2$ & $x<0$          & \M{1}       & $\opMul(-x,-x)$\\
%    \cline{2-4}
%          & else           & \M{1}       & $\opMul(x,x)$\\
%    \hline
%    $y<0$ & $x=0$          & \M{1}       & DivisionByZero\\
%    \cline{2-4}
%          & $x=1$          & \M{1}       & $1$\\
%    \cline{2-4}
%          & $x=-1$         & $\opODD(y)$ & $-1$\\
%    \cline{3-4}
%          &                & else        & $1$\\
%    \cline{2-4}
%          & else ($\string|x\string|>1$) & \M{1}       & $0$\\
%    \hline
%    $y>2$ & $x=0$          & \M{1}       & $0$\\
%    \cline{2-4}
%          & $x=1$          & \M{1}       & $1$\\
%    \cline{2-4}
%          & $x=-1$         & $\opODD(y)$ & $-1$\\
%    \cline{3-4}
%          &                & else        & $1$\\
%    \cline{2-4}
%          & $x<-1$ ($x<0$) & $\opODD(y)$ & $-\opPow(-x,y)$\\
%    \cline{3-4}
%          &                & else        & $\opPow(-x,y)$\\
%    \cline{2-4}
%          & else ($x>1$)   & \M{1}       & $\opPow(x,y)$\\
%    \hline
%    \end{tabular}
%    \end{quote}
%    \begin{macrocode}
\def\BIC@PowSwitch#1#2!#3#4!{%
  \ifcase\ifx\\#2\\%
           \ifx#100 % y = 0
           \else\ifx#111 % y = 1
           \else\ifx#122 % y = 2
           \else4 % y > 2
           \fi\fi\fi
         \else
           \ifx#1-3 % y < 0
           \else4 % y > 2
           \fi
         \fi
    \BIC@AfterFi{ 1}% y = 0
  \or % y = 1
    \BIC@AfterFi{ #3#4}%
  \or % y = 2
    \ifx#3-% x < 0
      \BIC@AfterFiFi{%
        \BIC@ProcessMul0!#4!#4!%
      }%
    \else % x >= 0
      \BIC@AfterFiFi{%
        \BIC@ProcessMul0!#3#4!#3#4!%
      }%
    \fi
  \or % y < 0
    \ifcase\ifx\\#4\\%
             \ifx#300 % x = 0
             \else\ifx#311 % x = 1
             \else3 % x > 1
             \fi\fi
           \else
             \ifcase\BIC@MinusOne#3#4! %
               3 % |x| > 1
             \or
               2 % x = -1
?            \else\BigIntCalcError:ThisCannotHappen%
             \fi
           \fi
      \BIC@AfterFiFi{ 0\BigIntCalcError:DivisionByZero}% x = 0
    \or % x = 1
      \BIC@AfterFiFi{ 1}% x = 1
    \or % x = -1
      \ifcase\BIC@ModTwo#2! % even(y)
        \BIC@AfterFiFiFi{ 1}%
      \or % odd(y)
        \BIC@AfterFiFiFi{ -1}%
?     \else\BigIntCalcError:ThisCannotHappen%
      \fi
    \or % |x| > 1
      \BIC@AfterFiFi{ 0}%
?   \else\BigIntCalcError:ThisCannotHappen%
    \fi
  \or % y > 2
    \ifcase\ifx\\#4\\%
             \ifx#300 % x = 0
             \else\ifx#311 % x = 1
             \else4 % x > 1
             \fi\fi
           \else
             \ifx#3-%
               \ifcase\BIC@MinusOne#3#4! %
                 3 % x < -1
               \else
                 2 % x = -1
               \fi
             \else
               4 % x > 1
             \fi
           \fi
      \BIC@AfterFiFi{ 0}% x = 0
    \or % x = 1
      \BIC@AfterFiFi{ 1}% x = 1
    \or % x = -1
      \ifcase\BIC@ModTwo#1#2! % even(y)
        \BIC@AfterFiFiFi{ 1}%
      \or % odd(y)
        \BIC@AfterFiFiFi{ -1}%
?     \else\BigIntCalcError:ThisCannotHappen%
      \fi
    \or % x < -1
      \ifcase\BIC@ModTwo#1#2! % even(y)
        \BIC@AfterFiFiFi{%
          \BIC@PowRec#4!#1#2!1!%
        }%
      \or % odd(y)
        \expandafter-\romannumeral0%
        \BIC@AfterFiFiFi{%
          \BIC@PowRec#4!#1#2!1!%
        }%
?     \else\BigIntCalcError:ThisCannotHappen%
      \fi
    \or % x > 1
      \BIC@AfterFiFi{%
        \BIC@PowRec#3#4!#1#2!1!%
      }%
?   \else\BigIntCalcError:ThisCannotHappen%
    \fi
? \else\BigIntCalcError:ThisCannotHappen%
  \BIC@Fi
}
%    \end{macrocode}
%    \end{macro}
%
% \subsubsection{Help macros}
%
%    \begin{macro}{\BIC@ModTwo}
%    Macro \cs{BIC@ModTwo} expects a number without sign
%    and returns digit |1| or |0| if the number is odd or even.
%    \begin{macrocode}
\def\BIC@ModTwo#1#2!{%
  \ifx\\#2\\%
    \ifodd#1 %
      \BIC@AfterFiFi1%
    \else
      \BIC@AfterFiFi0%
    \fi
  \else
    \BIC@AfterFi{%
      \BIC@ModTwo#2!%
    }%
  \BIC@Fi
}
%    \end{macrocode}
%    \end{macro}
%
%    \begin{macro}{\BIC@MinusOne}
%    Macro \cs{BIC@MinusOne} expects a number and returns
%    digit |1| if the number equals minus one and returns |0| otherwise.
%    \begin{macrocode}
\def\BIC@MinusOne#1#2!{%
  \ifx#1-%
    \BIC@@MinusOne#2!%
  \else
    0%
  \fi
}
%    \end{macrocode}
%    \end{macro}
%    \begin{macro}{\BIC@@MinusOne}
%    \begin{macrocode}
\def\BIC@@MinusOne#1#2!{%
  \ifx#11%
    \ifx\\#2\\%
      1%
    \else
      0%
    \fi
  \else
    0%
  \fi
}
%    \end{macrocode}
%    \end{macro}
%
% \subsubsection{Recursive calculation}
%
%    \begin{macro}{\BIC@PowRec}
%\begin{quote}
%\begin{verbatim}
%Pow(x, y) {
%  PowRec(x, y, 1)
%}
%PowRec(x, y, r) {
%  if y == 1 then
%    return r
%  else
%    ifodd y then
%      return PowRec(x*x, y div 2, r*x) % y div 2 = (y-1)/2
%    else
%      return PowRec(x*x, y div 2, r)
%    fi
%  fi
%}
%\end{verbatim}
%\end{quote}
%    |#1|: $x$ (basis)\\
%    |#2#3|: $y$ (power)\\
%    |#4|: $r$ (result)
%    \begin{macrocode}
\def\BIC@PowRec#1!#2#3!#4!{%
  \ifcase\ifx#21\ifx\\#3\\0 \else1 \fi\else1 \fi % y = 1
    \ifnum\BIC@PosCmp#1!#4!=1 % x > r
      \BIC@AfterFiFi{%
        \BIC@ProcessMul0!#1!#4!%
      }%
    \else
      \BIC@AfterFiFi{%
        \BIC@ProcessMul0!#4!#1!%
      }%
    \fi
  \or
    \ifcase\BIC@ModTwo#2#3! % even(y)
      \BIC@AfterFiFi{%
        \expandafter\BIC@@PowRec\romannumeral0%
        \BIC@@Shr#2#3!%
        !#1!#4!%
      }%
    \or % odd(y)
      \ifnum\BIC@PosCmp#1!#4!=1 % x > r
        \BIC@AfterFiFiFi{%
          \expandafter\BIC@@@PowRec\romannumeral0%
          \BIC@ProcessMul0!#1!#4!%
          !#1!#2#3!%
        }%
      \else
        \BIC@AfterFiFiFi{%
          \expandafter\BIC@@@PowRec\romannumeral0%
          \BIC@ProcessMul0!#1!#4!%
          !#1!#2#3!%
        }%
      \fi
?   \else\BigIntCalcError:ThisCannotHappen%
    \fi
? \else\BigIntCalcError:ThisCannotHappen%
  \BIC@Fi
}
%    \end{macrocode}
%    \end{macro}
%    \begin{macro}{\BIC@@PowRec}
%    |#1|: $y/2$\\
%    |#2|: $x$\\
%    |#3|: new $r$ ($r$ or $r*x$)
%    \begin{macrocode}
\def\BIC@@PowRec#1!#2!#3!{%
  \expandafter\BIC@PowRec\romannumeral0%
  \BIC@ProcessMul0!#2!#2!%
  !#1!#3!%
}
%    \end{macrocode}
%    \end{macro}
%    \begin{macro}{\BIC@@@PowRec}
%    |#1|: $r*x$
%    |#2|: $x$
%    |#3|: $y$
%    \begin{macrocode}
\def\BIC@@@PowRec#1!#2!#3!{%
  \expandafter\BIC@@PowRec\romannumeral0%
  \BIC@@Shr#3!%
  !#2!#1!%
}
%    \end{macrocode}
%    \end{macro}
%
% \subsection{\op{Div}}
%
%    \begin{macro}{\bigintcalcDiv}
%    |#1|: $x$\\
%    |#2|: $y$ (divisor)
%    \begin{macrocode}
\def\bigintcalcDiv#1{%
  \romannumeral0%
  \expandafter\expandafter\expandafter\BIC@Div
  \bigintcalcNum{#1}!%
}
%    \end{macrocode}
%    \end{macro}
%    \begin{macro}{\BIC@Div}
%    |#1|: $x$\\
%    |#2|: $y$
%    \begin{macrocode}
\def\BIC@Div#1!#2{%
  \expandafter\expandafter\expandafter\BIC@DivSwitchSign
  \bigintcalcNum{#2}!#1!%
}
%    \end{macrocode}
%    \end{macro}
%    \begin{macro}{\BigIntCalcDiv}
%    \begin{macrocode}
\def\BigIntCalcDiv#1!#2!{%
  \romannumeral0%
  \BIC@DivSwitchSign#2!#1!%
}
%    \end{macrocode}
%    \end{macro}
%    \begin{macro}{\BIC@DivSwitchSign}
%    Decision table for \cs{BIC@DivSwitchSign}.
%    \begin{quote}
%    \begin{tabular}{@{}|l|l|l|@{}}
%    \hline
%    $y=0$ & \multicolumn{1}{l}{} & DivisionByZero\\
%    \hline
%    $y>0$ & $x=0$ & $0$\\
%    \cline{2-3}
%          & $x>0$ & DivSwitch$(+,x,y)$\\
%    \cline{2-3}
%          & $x<0$ & DivSwitch$(-,-x,y)$\\
%    \hline
%    $y<0$ & $x=0$ & $0$\\
%    \cline{2-3}
%          & $x>0$ & DivSwitch$(-,x,-y)$\\
%    \cline{2-3}
%          & $x<0$ & DivSwitch$(+,-x,-y)$\\
%    \hline
%    \end{tabular}
%    \end{quote}
%    |#1|: $y$ (divisor)\\
%    |#2|: $x$
%    \begin{macrocode}
\def\BIC@DivSwitchSign#1#2!#3#4!{%
  \ifcase\BIC@Sgn#1#2! % y = 0
    \BIC@AfterFi{ 0\BigIntCalcError:DivisionByZero}%
  \or % y > 0
    \ifcase\BIC@Sgn#3#4! % x = 0
      \BIC@AfterFiFi{ 0}%
    \or % x > 0
      \BIC@AfterFiFi{%
        \BIC@DivSwitch{}#3#4!#1#2!%
      }%
    \else % x < 0
      \BIC@AfterFiFi{%
        \BIC@DivSwitch-#4!#1#2!%
      }%
    \fi
  \else % y < 0
    \ifcase\BIC@Sgn#3#4! % x = 0
      \BIC@AfterFiFi{ 0}%
    \or % x > 0
      \BIC@AfterFiFi{%
        \BIC@DivSwitch-#3#4!#2!%
      }%
    \else % x < 0
      \BIC@AfterFiFi{%
        \BIC@DivSwitch{}#4!#2!%
      }%
    \fi
  \BIC@Fi
}
%    \end{macrocode}
%    \end{macro}
%    \begin{macro}{\BIC@DivSwitch}
%    Decision table for \cs{BIC@DivSwitch}.
%    \begin{quote}
%    \begin{tabular}{@{}|l|l|l|@{}}
%    \hline
%    $y=x$ & \multicolumn{1}{l}{} & sign $1$\\
%    \hline
%    $y>x$ & \multicolumn{1}{l}{} & $0$\\
%    \hline
%    $y<x$ & $y=1$                & sign $x$\\
%    \cline{2-3}
%          & $y=2$                & sign Shr$(x)$\\
%    \cline{2-3}
%          & $y=4$                & sign Shr(Shr$(x)$)\\
%    \cline{2-3}
%          & else                 & sign ProcessDiv$(x,y)$\\
%    \hline
%    \end{tabular}
%    \end{quote}
%    |#1|: sign\\
%    |#2|: $x$\\
%    |#3#4|: $y$ ($y\ne 0$)
%    \begin{macrocode}
\def\BIC@DivSwitch#1#2!#3#4!{%
  \ifcase\BIC@PosCmp#3#4!#2!% y = x
    \BIC@AfterFi{ #11}%
  \or % y > x
    \BIC@AfterFi{ 0}%
  \else % y < x
    \ifx\\#1\\%
    \else
      \expandafter-\romannumeral0%
    \fi
    \ifcase\ifx\\#4\\%
             \ifx#310 % y = 1
             \else\ifx#321 % y = 2
             \else\ifx#342 % y = 4
             \else3 % y > 2
             \fi\fi\fi
           \else
             3 % y > 2
           \fi
      \BIC@AfterFiFi{ #2}% y = 1
    \or % y = 2
      \BIC@AfterFiFi{%
        \BIC@@Shr#2!%
      }%
    \or % y = 4
      \BIC@AfterFiFi{%
        \expandafter\BIC@@Shr\romannumeral0%
          \BIC@@Shr#2!!%
      }%
    \or % y > 2
      \BIC@AfterFiFi{%
        \BIC@DivStartX#2!#3#4!!!%
      }%
?   \else\BigIntCalcError:ThisCannotHappen%
    \fi
  \BIC@Fi
}
%    \end{macrocode}
%    \end{macro}
%    \begin{macro}{\BIC@ProcessDiv}
%    |#1#2|: $x$\\
%    |#3#4|: $y$\\
%    |#5|: collect first digits of $x$\\
%    |#6|: corresponding digits of $y$
%    \begin{macrocode}
\def\BIC@DivStartX#1#2!#3#4!#5!#6!{%
  \ifx\\#4\\%
    \BIC@AfterFi{%
      \BIC@DivStartYii#6#3#4!{#5#1}#2=!%
    }%
  \else
    \BIC@AfterFi{%
      \BIC@DivStartX#2!#4!#5#1!#6#3!%
    }%
  \BIC@Fi
}
%    \end{macrocode}
%    \end{macro}
%    \begin{macro}{\BIC@DivStartYii}
%    |#1|: $y$\\
%    |#2|: $x$, |=|
%    \begin{macrocode}
\def\BIC@DivStartYii#1!{%
  \expandafter\BIC@DivStartYiv\romannumeral0%
  \BIC@Shl#1!%
  !#1!%
}
%    \end{macrocode}
%    \end{macro}
%    \begin{macro}{\BIC@DivStartYiv}
%    |#1|: $2y$\\
%    |#2|: $y$\\
%    |#3|: $x$, |=|
%    \begin{macrocode}
\def\BIC@DivStartYiv#1!{%
  \expandafter\BIC@DivStartYvi\romannumeral0%
  \BIC@Shl#1!%
  !#1!%
}
%    \end{macrocode}
%    \end{macro}
%    \begin{macro}{\BIC@DivStartYvi}
%    |#1|: $4y$\\
%    |#2|: $2y$\\
%    |#3|: $y$\\
%    |#4|: $x$, |=|
%    \begin{macrocode}
\def\BIC@DivStartYvi#1!#2!{%
  \expandafter\BIC@DivStartYviii\romannumeral0%
  \BIC@AddXY#1!#2!!!%
  !#1!#2!%
}
%    \end{macrocode}
%    \end{macro}
%    \begin{macro}{\BIC@DivStartYviii}
%    |#1|: $6y$\\
%    |#2|: $4y$\\
%    |#3|: $2y$\\
%    |#4|: $y$\\
%    |#5|: $x$, |=|
%    \begin{macrocode}
\def\BIC@DivStartYviii#1!#2!{%
  \expandafter\BIC@DivStart\romannumeral0%
  \BIC@Shl#2!%
  !#1!#2!%
}
%    \end{macrocode}
%    \end{macro}
%    \begin{macro}{\BIC@DivStart}
%    |#1|: $8y$\\
%    |#2|: $6y$\\
%    |#3|: $4y$\\
%    |#4|: $2y$\\
%    |#5|: $y$\\
%    |#6|: $x$, |=|
%    \begin{macrocode}
\def\BIC@DivStart#1!#2!#3!#4!#5!#6!{%
  \BIC@ProcessDiv#6!!#5!#4!#3!#2!#1!=%
}
%    \end{macrocode}
%    \end{macro}
%    \begin{macro}{\BIC@ProcessDiv}
%    |#1#2#3|: $x$, |=|\\
%    |#4|: result\\
%    |#5|: $y$\\
%    |#6|: $2y$\\
%    |#7|: $4y$\\
%    |#8|: $6y$\\
%    |#9|: $8y$
%    \begin{macrocode}
\def\BIC@ProcessDiv#1#2#3!#4!#5!{%
  \ifcase\BIC@PosCmp#5!#1!% y = #1
    \ifx#2=%
      \BIC@AfterFiFi{\BIC@DivCleanup{#41}}%
    \else
      \BIC@AfterFiFi{%
        \BIC@ProcessDiv#2#3!#41!#5!%
      }%
    \fi
  \or % y > #1
    \ifx#2=%
      \BIC@AfterFiFi{\BIC@DivCleanup{#40}}%
    \else
      \ifx\\#4\\%
        \BIC@AfterFiFiFi{%
          \BIC@ProcessDiv{#1#2}#3!!#5!%
        }%
      \else
        \BIC@AfterFiFiFi{%
          \BIC@ProcessDiv{#1#2}#3!#40!#5!%
        }%
      \fi
    \fi
  \else % y < #1
    \BIC@AfterFi{%
      \BIC@@ProcessDiv{#1}#2#3!#4!#5!%
    }%
  \BIC@Fi
}
%    \end{macrocode}
%    \end{macro}
%    \begin{macro}{\BIC@DivCleanup}
%    |#1|: result\\
%    |#2|: garbage
%    \begin{macrocode}
\def\BIC@DivCleanup#1#2={ #1}%
%    \end{macrocode}
%    \end{macro}
%    \begin{macro}{\BIC@@ProcessDiv}
%    \begin{macrocode}
\def\BIC@@ProcessDiv#1#2#3!#4!#5!#6!#7!{%
  \ifcase\BIC@PosCmp#7!#1!% 4y = #1
    \ifx#2=%
      \BIC@AfterFiFi{\BIC@DivCleanup{#44}}%
    \else
     \BIC@AfterFiFi{%
       \BIC@ProcessDiv#2#3!#44!#5!#6!#7!%
     }%
    \fi
  \or % 4y > #1
    \ifcase\BIC@PosCmp#6!#1!% 2y = #1
      \ifx#2=%
        \BIC@AfterFiFiFi{\BIC@DivCleanup{#42}}%
      \else
        \BIC@AfterFiFiFi{%
          \BIC@ProcessDiv#2#3!#42!#5!#6!#7!%
        }%
      \fi
    \or % 2y > #1
      \ifx#2=%
        \BIC@AfterFiFiFi{\BIC@DivCleanup{#41}}%
      \else
        \BIC@AfterFiFiFi{%
          \BIC@DivSub#1!#5!#2#3!#41!#5!#6!#7!%
        }%
      \fi
    \else % 2y < #1
      \BIC@AfterFiFi{%
        \expandafter\BIC@ProcessDivII\romannumeral0%
        \BIC@SubXY#1!#6!!!%
        !#2#3!#4!#5!23%
        #6!#7!%
      }%
    \fi
  \else % 4y < #1
    \BIC@AfterFi{%
      \BIC@@@ProcessDiv{#1}#2#3!#4!#5!#6!#7!%
    }%
  \BIC@Fi
}
%    \end{macrocode}
%    \end{macro}
%    \begin{macro}{\BIC@DivSub}
%    Next token group: |#1|-|#2| and next digit |#3|.
%    \begin{macrocode}
\def\BIC@DivSub#1!#2!#3{%
  \expandafter\BIC@ProcessDiv\expandafter{%
    \romannumeral0%
    \BIC@SubXY#1!#2!!!%
    #3%
  }%
}
%    \end{macrocode}
%    \end{macro}
%    \begin{macro}{\BIC@ProcessDivII}
%    |#1|: $x'-2y$\\
%    |#2#3|: remaining $x$, |=|\\
%    |#4|: result\\
%    |#5|: $y$\\
%    |#6|: first possible result digit\\
%    |#7|: second possible result digit
%    \begin{macrocode}
\def\BIC@ProcessDivII#1!#2#3!#4!#5!#6#7{%
  \ifcase\BIC@PosCmp#5!#1!% y = #1
    \ifx#2=%
      \BIC@AfterFiFi{\BIC@DivCleanup{#4#7}}%
    \else
      \BIC@AfterFiFi{%
        \BIC@ProcessDiv#2#3!#4#7!#5!%
      }%
    \fi
  \or % y > #1
    \ifx#2=%
      \BIC@AfterFiFi{\BIC@DivCleanup{#4#6}}%
    \else
      \BIC@AfterFiFi{%
        \BIC@ProcessDiv{#1#2}#3!#4#6!#5!%
      }%
    \fi
  \else % y < #1
    \ifx#2=%
      \BIC@AfterFiFi{\BIC@DivCleanup{#4#7}}%
    \else
      \BIC@AfterFiFi{%
        \BIC@DivSub#1!#5!#2#3!#4#7!#5!%
      }%
    \fi
  \BIC@Fi
}
%    \end{macrocode}
%    \end{macro}
%    \begin{macro}{\BIC@ProcessDivIV}
%    |#1#2#3|: $x$, |=|, $x>4y$\\
%    |#4|: result\\
%    |#5|: $y$\\
%    |#6|: $2y$\\
%    |#7|: $4y$\\
%    |#8|: $6y$\\
%    |#9|: $8y$
%    \begin{macrocode}
\def\BIC@@@ProcessDiv#1#2#3!#4!#5!#6!#7!#8!#9!{%
  \ifcase\BIC@PosCmp#8!#1!% 6y = #1
    \ifx#2=%
      \BIC@AfterFiFi{\BIC@DivCleanup{#46}}%
    \else
      \BIC@AfterFiFi{%
        \BIC@ProcessDiv#2#3!#46!#5!#6!#7!#8!#9!%
      }%
    \fi
  \or % 6y > #1
    \BIC@AfterFi{%
      \expandafter\BIC@ProcessDivII\romannumeral0%
      \BIC@SubXY#1!#7!!!%
      !#2#3!#4!#5!45%
      #6!#7!#8!#9!%
    }%
  \else % 6y < #1
    \ifcase\BIC@PosCmp#9!#1!% 8y = #1
      \ifx#2=%
        \BIC@AfterFiFiFi{\BIC@DivCleanup{#48}}%
      \else
        \BIC@AfterFiFiFi{%
          \BIC@ProcessDiv#2#3!#48!#5!#6!#7!#8!#9!%
        }%
      \fi
    \or % 8y > #1
      \BIC@AfterFiFi{%
        \expandafter\BIC@ProcessDivII\romannumeral0%
        \BIC@SubXY#1!#8!!!%
        !#2#3!#4!#5!67%
        #6!#7!#8!#9!%
      }%
    \else % 8y < #1
      \BIC@AfterFiFi{%
        \expandafter\BIC@ProcessDivII\romannumeral0%
        \BIC@SubXY#1!#9!!!%
        !#2#3!#4!#5!89%
        #6!#7!#8!#9!%
      }%
    \fi
  \BIC@Fi
}
%    \end{macrocode}
%    \end{macro}
%
% \subsection{\op{Mod}}
%
%    \begin{macro}{\bigintcalcMod}
%    |#1|: $x$\\
%    |#2|: $y$
%    \begin{macrocode}
\def\bigintcalcMod#1{%
  \romannumeral0%
  \expandafter\expandafter\expandafter\BIC@Mod
  \bigintcalcNum{#1}!%
}
%    \end{macrocode}
%    \end{macro}
%    \begin{macro}{\BIC@Mod}
%    |#1|: $x$\\
%    |#2|: $y$
%    \begin{macrocode}
\def\BIC@Mod#1!#2{%
  \expandafter\expandafter\expandafter\BIC@ModSwitchSign
  \bigintcalcNum{#2}!#1!%
}
%    \end{macrocode}
%    \end{macro}
%    \begin{macro}{\BigIntCalcMod}
%    \begin{macrocode}
\def\BigIntCalcMod#1!#2!{%
  \romannumeral0%
  \BIC@ModSwitchSign#2!#1!%
}
%    \end{macrocode}
%    \end{macro}
%    \begin{macro}{\BIC@ModSwitchSign}
%    Decision table for \cs{BIC@ModSwitchSign}.
%    \begin{quote}
%    \begin{tabular}{@{}|l|l|l|@{}}
%    \hline
%    $y=0$ & \multicolumn{1}{l}{} & DivisionByZero\\
%    \hline
%    $y>0$ & $x=0$ & $0$\\
%    \cline{2-3}
%          & else  & ModSwitch$(+,x,y)$\\
%    \hline
%    $y<0$ & \multicolumn{1}{l}{} & ModSwitch$(-,-x,-y)$\\
%    \hline
%    \end{tabular}
%    \end{quote}
%    |#1#2|: $y$\\
%    |#3#4|: $x$
%    \begin{macrocode}
\def\BIC@ModSwitchSign#1#2!#3#4!{%
  \ifcase\ifx\\#2\\%
           \ifx#100 % y = 0
           \else1 % y > 0
           \fi
          \else
            \ifx#1-2 % y < 0
            \else1 % y > 0
            \fi
          \fi
    \BIC@AfterFi{ 0\BigIntCalcError:DivisionByZero}%
  \or % y > 0
    \ifcase\ifx\\#4\\\ifx#300 \else1 \fi\else1 \fi % x = 0
      \BIC@AfterFiFi{ 0}%
    \else
      \BIC@AfterFiFi{%
        \BIC@ModSwitch{}#3#4!#1#2!%
      }%
    \fi
  \else % y < 0
    \ifcase\ifx\\#4\\%
             \ifx#300 % x = 0
             \else1 % x > 0
             \fi
           \else
             \ifx#3-2 % x < 0
             \else1 % x > 0
             \fi
           \fi
      \BIC@AfterFiFi{ 0}%
    \or % x > 0
      \BIC@AfterFiFi{%
        \BIC@ModSwitch--#3#4!#2!%
      }%
    \else % x < 0
      \BIC@AfterFiFi{%
        \BIC@ModSwitch-#4!#2!%
      }%
    \fi
  \BIC@Fi
}
%    \end{macrocode}
%    \end{macro}
%    \begin{macro}{\BIC@ModSwitch}
%    Decision table for \cs{BIC@ModSwitch}.
%    \begin{quote}
%    \begin{tabular}{@{}|l|l|l|@{}}
%    \hline
%    $y=1$ & \multicolumn{1}{l}{} & $0$\\
%    \hline
%    $y=2$ & ifodd$(x)$ & sign $1$\\
%    \cline{2-3}
%          & else       & $0$\\
%    \hline
%    $y>2$ & $x<0$      &
%        $z\leftarrow x-(x/y)*y;\quad (z<0)\mathbin{?}z+y\mathbin{:}z$\\
%    \cline{2-3}
%          & $x>0$      & $x - (x/y) * y$\\
%    \hline
%    \end{tabular}
%    \end{quote}
%    |#1|: sign\\
%    |#2#3|: $x$\\
%    |#4#5|: $y$
%    \begin{macrocode}
\def\BIC@ModSwitch#1#2#3!#4#5!{%
  \ifcase\ifx\\#5\\%
           \ifx#410 % y = 1
           \else\ifx#421 % y = 2
           \else2 % y > 2
           \fi\fi
         \else2 % y > 2
         \fi
    \BIC@AfterFi{ 0}% y = 1
  \or % y = 2
    \ifcase\BIC@ModTwo#2#3! % even(x)
      \BIC@AfterFiFi{ 0}%
    \or % odd(x)
      \BIC@AfterFiFi{ #11}%
?   \else\BigIntCalcError:ThisCannotHappen%
    \fi
  \or % y > 2
    \ifx\\#1\\%
    \else
      \expandafter\BIC@Space\romannumeral0%
      \expandafter\BIC@ModMinus\romannumeral0%
    \fi
    \ifx#2-% x < 0
      \BIC@AfterFiFi{%
        \expandafter\expandafter\expandafter\BIC@ModX
        \bigintcalcSub{#2#3}{%
          \bigintcalcMul{#4#5}{\bigintcalcDiv{#2#3}{#4#5}}%
        }!#4#5!%
      }%
    \else % x > 0
      \BIC@AfterFiFi{%
        \expandafter\expandafter\expandafter\BIC@Space
        \bigintcalcSub{#2#3}{%
          \bigintcalcMul{#4#5}{\bigintcalcDiv{#2#3}{#4#5}}%
        }%
      }%
    \fi
? \else\BigIntCalcError:ThisCannotHappen%
  \BIC@Fi
}
%    \end{macrocode}
%    \end{macro}
%    \begin{macro}{\BIC@ModMinus}
%    \begin{macrocode}
\def\BIC@ModMinus#1{%
  \ifx#10%
    \BIC@AfterFi{ 0}%
  \else
    \BIC@AfterFi{ -#1}%
  \BIC@Fi
}
%    \end{macrocode}
%    \end{macro}
%    \begin{macro}{\BIC@ModX}
%    |#1#2|: $z$\\
%    |#3|: $x$
%    \begin{macrocode}
\def\BIC@ModX#1#2!#3!{%
  \ifx#1-% z < 0
    \BIC@AfterFi{%
      \expandafter\BIC@Space\romannumeral0%
      \BIC@SubXY#3!#2!!!%
    }%
  \else % z >= 0
    \BIC@AfterFi{ #1#2}%
  \BIC@Fi
}
%    \end{macrocode}
%    \end{macro}
%
%    \begin{macrocode}
\BIC@AtEnd%
%    \end{macrocode}
%
%    \begin{macrocode}
%</package>
%    \end{macrocode}
%
% \section{Test}
%
% \subsection{Catcode checks for loading}
%
%    \begin{macrocode}
%<*test1>
%    \end{macrocode}
%    \begin{macrocode}
\catcode`\{=1 %
\catcode`\}=2 %
\catcode`\#=6 %
\catcode`\@=11 %
\expandafter\ifx\csname count@\endcsname\relax
  \countdef\count@=255 %
\fi
\expandafter\ifx\csname @gobble\endcsname\relax
  \long\def\@gobble#1{}%
\fi
\expandafter\ifx\csname @firstofone\endcsname\relax
  \long\def\@firstofone#1{#1}%
\fi
\expandafter\ifx\csname loop\endcsname\relax
  \expandafter\@firstofone
\else
  \expandafter\@gobble
\fi
{%
  \def\loop#1\repeat{%
    \def\body{#1}%
    \iterate
  }%
  \def\iterate{%
    \body
      \let\next\iterate
    \else
      \let\next\relax
    \fi
    \next
  }%
  \let\repeat=\fi
}%
\def\RestoreCatcodes{}
\count@=0 %
\loop
  \edef\RestoreCatcodes{%
    \RestoreCatcodes
    \catcode\the\count@=\the\catcode\count@\relax
  }%
\ifnum\count@<255 %
  \advance\count@ 1 %
\repeat

\def\RangeCatcodeInvalid#1#2{%
  \count@=#1\relax
  \loop
    \catcode\count@=15 %
  \ifnum\count@<#2\relax
    \advance\count@ 1 %
  \repeat
}
\def\RangeCatcodeCheck#1#2#3{%
  \count@=#1\relax
  \loop
    \ifnum#3=\catcode\count@
    \else
      \errmessage{%
        Character \the\count@\space
        with wrong catcode \the\catcode\count@\space
        instead of \number#3%
      }%
    \fi
  \ifnum\count@<#2\relax
    \advance\count@ 1 %
  \repeat
}
\def\space{ }
\expandafter\ifx\csname LoadCommand\endcsname\relax
  \def\LoadCommand{\input bigintcalc.sty\relax}%
\fi
\def\Test{%
  \RangeCatcodeInvalid{0}{47}%
  \RangeCatcodeInvalid{58}{64}%
  \RangeCatcodeInvalid{91}{96}%
  \RangeCatcodeInvalid{123}{255}%
  \catcode`\@=12 %
  \catcode`\\=0 %
  \catcode`\%=14 %
  \LoadCommand
  \RangeCatcodeCheck{0}{36}{15}%
  \RangeCatcodeCheck{37}{37}{14}%
  \RangeCatcodeCheck{38}{47}{15}%
  \RangeCatcodeCheck{48}{57}{12}%
  \RangeCatcodeCheck{58}{63}{15}%
  \RangeCatcodeCheck{64}{64}{12}%
  \RangeCatcodeCheck{65}{90}{11}%
  \RangeCatcodeCheck{91}{91}{15}%
  \RangeCatcodeCheck{92}{92}{0}%
  \RangeCatcodeCheck{93}{96}{15}%
  \RangeCatcodeCheck{97}{122}{11}%
  \RangeCatcodeCheck{123}{255}{15}%
  \RestoreCatcodes
}
\Test
\csname @@end\endcsname
\end
%    \end{macrocode}
%    \begin{macrocode}
%</test1>
%    \end{macrocode}
%
% \subsection{Macro tests}
%
% \subsubsection{Preamble with test macro definitions}
%
%    \begin{macrocode}
%<*test2>
\NeedsTeXFormat{LaTeX2e}
\nofiles
\documentclass{article}
%<noetex>\let\SavedNumexpr\numexpr
%<noetex>\let\numexpr\UNDEFINED
\makeatletter
\chardef\BIC@TestMode=1 %
\makeatother
\usepackage{bigintcalc}[2016/05/16]
%<noetex>\let\numexpr\SavedNumexpr
\usepackage{qstest}
\IncludeTests{*}
\LogTests{log}{*}{*}
\newcommand*{\TestSpaceAtEnd}[1]{%
%<noetex>  \let\SavedNumexpr\numexpr
%<noetex>  \let\numexpr\UNDEFINED
  \edef\resultA{#1}%
  \edef\resultB{#1 }%
%<noetex>  \let\numexpr\SavedNumexpr
  \Expect*{\resultA\space}*{\resultB}%
}
\newcommand*{\TestResult}[2]{%
%<noetex>  \let\SavedNumexpr\numexpr
%<noetex>  \let\numexpr\UNDEFINED
  \edef\result{#1}%
%<noetex>  \let\numexpr\SavedNumexpr
  \Expect*{\result}{#2}%
}
\newcommand*{\TestResultTwoExpansions}[2]{%
%<*noetex>
  \begingroup
    \let\numexpr\UNDEFINED
    \expandafter\expandafter\expandafter
  \endgroup
%</noetex>
  \expandafter\expandafter\expandafter\Expect
  \expandafter\expandafter\expandafter{#1}{#2}%
}
\newcount\TestCount
%<etex>\newcommand*{\TestArg}[1]{\numexpr#1\relax}
%<noetex>\newcommand*{\TestArg}[1]{#1}
\newcommand*{\TestTeXDivide}[2]{%
  \TestCount=\TestArg{#1}\relax
  \divide\TestCount by \TestArg{#2}\relax
  \Expect*{\bigintcalcDiv{#1}{#2}}*{\the\TestCount}%
}
\newcommand*{\Test}[2]{%
  \TestResult{#1}{#2}%
  \TestResultTwoExpansions{#1}{#2}%
  \TestSpaceAtEnd{#1}%
}
\newcommand*{\TestExch}[2]{\Test{#2}{#1}}
\newcommand*{\TestInv}[2]{%
  \Test{\bigintcalcInv{#1}}{#2}%
}
\newcommand*{\TestAbs}[2]{%
  \Test{\bigintcalcAbs{#1}}{#2}%
}
\newcommand*{\TestSgn}[2]{%
  \Test{\bigintcalcSgn{#1}}{#2}%
}
\newcommand*{\TestMin}[3]{%
  \Test{\bigintcalcMin{#1}{#2}}{#3}%
}
\newcommand*{\TestMax}[3]{%
  \Test{\bigintcalcMax{#1}{#2}}{#3}%
}
\newcommand*{\TestCmp}[3]{%
  \Test{\bigintcalcCmp{#1}{#2}}{#3}%
}
\newcommand*{\TestOdd}[2]{%
  \Test{\bigintcalcOdd{#1}}{#2}%
  \edef\x{%
    \noexpand\Test{%
      \noexpand\BigIntCalcOdd
      \bigintcalcAbs{#1}!%
    }{#2}%
  }%
  \x
}
\newcommand*{\TestInc}[2]{%
  \Test{\bigintcalcInc{#1}}{#2}%
  \ifnum\bigintcalcSgn{#1}>-1 %
    \edef\x{%
      \noexpand\Test{%
        \noexpand\BigIntCalcInc\bigintcalcNum{#1}!%
      }{#2}%
    }%
    \x
  \fi
}
\newcommand*{\TestDec}[2]{%
  \Test{\bigintcalcDec{#1}}{#2}%
  \ifnum\bigintcalcSgn{#1}>0 %
    \edef\x{%
      \noexpand\Test{%
        \noexpand\BigIntCalcDec\bigintcalcNum{#1}!%
      }{#2}%
    }%
    \x
  \fi
}
\newcommand*{\TestAdd}[3]{%
  \Test{\bigintcalcAdd{#1}{#2}}{#3}%
  \ifnum\bigintcalcSgn{#1}>0 %
    \ifnum\bigintcalcSgn{#2}> 0 %
      \ifnum\bigintcalcCmp{#1}{#2}>0 %
        \edef\x{%
          \noexpand\Test{%
            \noexpand\BigIntCalcAdd
            \bigintcalcNum{#1}!\bigintcalcNum{#2}!%
          }{#3}%
        }%
        \x
      \else
        \edef\x{%
          \noexpand\Test{%
            \noexpand\BigIntCalcAdd
            \bigintcalcNum{#2}!\bigintcalcNum{#1}!%
          }{#3}%
        }%
        \x
      \fi
    \fi
  \fi
}
\newcommand*{\TestSub}[3]{%
  \Test{\bigintcalcSub{#1}{#2}}{#3}%
  \ifnum\bigintcalcSgn{#1}>0 %
    \ifnum\bigintcalcSgn{#2}> 0 %
      \ifnum\bigintcalcCmp{#1}{#2}>0 %
        \edef\x{%
          \noexpand\Test{%
            \noexpand\BigIntCalcSub
            \bigintcalcNum{#1}!\bigintcalcNum{#2}!%
          }{#3}%
        }%
        \x
      \fi
    \fi
  \fi
}
\newcommand*{\TestShl}[2]{%
  \Test{\bigintcalcShl{#1}}{#2}%
  \edef\x{%
    \noexpand\Test{%
      \noexpand\BigIntCalcShl\bigintcalcAbs{#1}!%
    }{\bigintcalcAbs{#2}}%
  }%
  \x
}
\newcommand*{\TestShr}[2]{%
  \Test{\bigintcalcShr{#1}}{#2}%
  \edef\x{%
    \noexpand\Test{%
      \noexpand\BigIntCalcShr\bigintcalcAbs{#1}!%
    }{\bigintcalcAbs{#2}}%
  }%
  \x
}
\newcommand*{\TestMul}[3]{%
  \Test{\bigintcalcMul{#1}{#2}}{#3}%
  \edef\x{%
    \noexpand\Test{%
      \noexpand\BigIntCalcMul
      \bigintcalcAbs{#1}!\bigintcalcAbs{#2}!%
    }{\bigintcalcAbs{#3}}%
  }%
  \x
}
\newcommand*{\TestSqr}[2]{%
  \Test{\bigintcalcSqr{#1}}{#2}%
}
\newcommand*{\TestFac}[2]{%
  \expandafter\TestExch\expandafter{%
    \the\numexpr#2%
  }{\bigintcalcFac{#1}}%
}
\newcommand*{\TestFacBig}[2]{%
  \Test{\bigintcalcFac{#1}}{#2}%
}
\newcommand*{\TestPow}[3]{%
  \Test{\bigintcalcPow{#1}{#2}}{#3}%
}
\newcommand*{\TestDiv}[3]{%
  \Test{\bigintcalcDiv{#1}{#2}}{#3}%
  \TestTeXDivide{#1}{#2}%
}
\newcommand*{\TestDivBig}[3]{%
  \Test{\bigintcalcDiv{#1}{#2}}{#3}%
  \edef\x{%
    \noexpand\Test{%
      \noexpand\BigIntCalcDiv\bigintcalcAbs{#1}!\bigintcalcAbs{#2}!%
    }{\bigintcalcAbs{#3}}%
  }%
}
\newcommand*{\TestMod}[3]{%
  \Test{\bigintcalcMod{#1}{#2}}{#3}%
  \ifcase\ifcase\bigintcalcSgn{#1} 0%
         \or
           \ifcase\bigintcalcSgn{#2} 1%
           \or 0%
           \else 1%
           \fi
         \else
           \ifcase\bigintcalcSgn{#2} 1%
           \or 1%
           \else 0%
           \fi
         \fi\relax
    \edef\x{%
      \noexpand\Test{%
        \noexpand\BigIntCalcMod
        \bigintcalcAbs{#1}!\bigintcalcAbs{#2}!%
      }{\bigintcalcAbs{#3}}%
    }%
    \x
  \fi
}
%    \end{macrocode}
%
% \subsubsection{Time}
%
%    \begin{macrocode}
\begingroup\expandafter\expandafter\expandafter\endgroup
\expandafter\ifx\csname pdfresettimer\endcsname\relax
\else
  \makeatletter
  \newcount\SummaryTime
  \newcount\TestTime
  \SummaryTime=\z@
  \newcommand*{\PrintTime}[2]{%
    \typeout{%
      [Time #1: \strip@pt\dimexpr\number#2sp\relax\space s]%
    }%
  }%
  \newcommand*{\StartTime}[1]{%
    \renewcommand*{\TimeDescription}{#1}%
    \pdfresettimer
  }%
  \newcommand*{\TimeDescription}{}%
  \newcommand*{\StopTime}{%
    \TestTime=\pdfelapsedtime
    \global\advance\SummaryTime\TestTime
    \PrintTime\TimeDescription\TestTime
  }%
  \let\saved@qstest\qstest
  \let\saved@endqstest\endqstest
  \def\qstest#1#2{%
    \saved@qstest{#1}{#2}%
    \StartTime{#1}%
  }%
  \def\endqstest{%
    \StopTime
    \saved@endqstest
  }%
  \AtEndDocument{%
    \PrintTime{summary}\SummaryTime
  }%
  \makeatother
\fi
%    \end{macrocode}
%
% \subsubsection{Test sets}
%
%    \begin{macrocode}
\makeatletter

\begin{qstest}{inv}{inv}%
  \TestInv{0}{0}%
  \TestInv{1}{-1}%
  \TestInv{-1}{1}%
  \TestInv{10}{-10}%
  \TestInv{-10}{10}%
  \TestInv{2147483647}{-2147483647}%
  \TestInv{-2147483647}{2147483647}%
  \TestInv{12345678901234567890}{-12345678901234567890}%
  \TestInv{-12345678901234567890}{12345678901234567890}%
  \TestInv{ 0 }{0}%
  \TestInv{ 1 }{-1}%
  \TestInv{--1}{-1}%
  \TestInv{\number\z@}{0}%
  \TestInv{\ifx\relax\relax1\fi}{-1}%
  \TestInv{\ifx\relax\relax-\fi\ifx234\else1\fi}{1}%
\end{qstest}

\begin{qstest}{abs}{abs}%
  \TestAbs{0}{0}%
  \TestAbs{1}{1}%
  \TestAbs{-1}{1}%
  \TestAbs{10}{10}%
  \TestAbs{-10}{10}%
  \TestAbs{2147483647}{2147483647}%
  \TestAbs{-2147483647}{2147483647}%
  \TestAbs{12345678901234567890}{12345678901234567890}%
  \TestAbs{-12345678901234567890}{12345678901234567890}%
  \TestAbs{ 0 }{0}%
  \TestAbs{ 1 }{1}%
  \TestAbs{--1}{1}%
  \TestAbs{-+-+1}{1}%
  \TestAbs{00000000000}{0}%
  \TestAbs{00000001000}{1000}%
  \TestAbs{\ifx\relax\relax 0\else 1\fi}{0}%
\end{qstest}

\begin{qstest}{sign}{sign}%
  \TestSgn{0}{0}%
  \TestSgn{1}{1}%
  \TestSgn{-1}{-1}%
  \TestSgn{10}{1}%
  \TestSgn{-10}{-1}%
  \TestSgn{2147483647}{1}%
  \TestSgn{-2147483647}{-1}%
  \TestSgn{12345678901234567890}{1}%
  \TestSgn{-12345678901234567890}{-1}%
  \TestSgn{ 0 }{0}%
  \TestSgn{ 2 }{1}%
  \TestSgn{ -2 }{-1}%
  \TestSgn{--2}{1}%
  \TestSgn{\number\z@}{0}%
  \TestSgn{\number\@ne}{1}%
  \TestSgn{\number\m@ne}{-1}%
  \TestSgn{%
    -+-+\number\z@\number\z@
    \iftrue1\fi\iftrue2\fi\iftrue3\fi
  }{1}%
\end{qstest}

\begin{qstest}{min}{min}%
  \TestMin{0}{1}{0}%
  \TestMin{1}{0}{0}%
  \TestMin{-10}{-20}{-20}%
  \TestMin{ 1 }{ 2 }{1}%
  \TestMin{ 2 }{ 1 }{1}%
  \TestMin{1}{1}{1}%
  \TestMin{\number\z@}{\number\@ne}{0}%
  \TestMin{\number\@ne}{\number\m@ne}{-1}%
\end{qstest}

\begin{qstest}{max}{max}%
  \TestMax{0}{1}{1}%
  \TestMax{1}{0}{1}%
  \TestMax{-10}{-20}{-10}%
  \TestMax{ 1 }{ 2 }{2}%
  \TestMax{ 2 }{ 1 }{2}%
  \TestMax{1}{1}{1}%
  \TestMax{\number\z@}{\number\@ne}{1}%
  \TestMax{\number\@ne}{\number\m@ne}{1}%
\end{qstest}

\begin{qstest}{cmp}{cmp}%
  \TestCmp{0}{0}{0}%
  \TestCmp{-21}{17}{-1}%
  \TestCmp{3}{4}{-1}%
  \TestCmp{-10}{-10}{0}%
  \TestCmp{-10}{-11}{1}%
  \TestCmp{100}{5}{1}%
  \TestCmp{9}{10}{-1}%
  \TestCmp{10}{9}{1}%
  \TestCmp{ 3 }{ 3 }{0}%
  \TestCmp{-9}{-10}{1}%
  \TestCmp{-10}{-9}{-1}%
  \TestCmp{-3}{-3}{0}%
  \TestCmp{0}{-2}{1}%
  \TestCmp{0}{2}{-1}%
  \TestCmp{2}{0}{1}%
  \TestCmp{-2}{0}{-1}%
  \TestCmp{12}{11}{1}%
  \TestCmp{11}{12}{-1}%
  \TestCmp{2147483647}{-2147483647}{1}%
  \TestCmp{-2147483647}{2147483647}{-1}%
  \TestCmp{2147483647}{2147483647}{0}%
  \TestCmp{\number\z@}{\number\@ne}{-1}%
  \TestCmp{\number\@ne}{\number\m@ne}{1}%
  \TestCmp{ 4 }{ 5 }{-1}%
  \TestCmp{ -3 }{ -7 }{1}%
\end{qstest}

\begin{qstest}{odd}{odd}
\tracingmacros=1
  \TestOdd{0}{0}%
  \TestOdd{1}{1}%
  \TestOdd{2}{0}%
  \TestOdd{3}{1}%
  \TestOdd{14}{0}%
  \TestOdd{15}{1}%
  \TestOdd{12345678901234567896}{0}%
  \TestOdd{12345678901234567897}{1}%
\end{qstest}

\begin{qstest}{inc}{inc}%
  \TestInc{0}{1}%
  \TestInc{1}{2}%
  \TestInc{-1}{0}%
  \TestInc{10}{11}%
  \TestInc{-10}{-9}%
  \TestInc{ 3 }{4}%
  \TestInc{999}{1000}%
  \TestInc{-1000}{-999}%
  \TestInc{129}{130}%
  \TestInc{2147483646}{2147483647}%
  \TestInc{-2147483647}{-2147483646}%
  \TestInc{12345678901234567890}{12345678901234567891}%
  \TestInc{99999999999999999999}{100000000000000000000}%
  \TestInc{-12345678901234567891}{-12345678901234567890}%
  \TestInc{-100000000000000000000}{-99999999999999999999}%
\end{qstest}

\begin{qstest}{dec}{dec}%
  \TestDec{0}{-1}%
  \TestDec{1}{0}%
  \TestDec{-1}{-2}%
  \TestDec{10}{9}%
  \TestDec{-10}{-11}%
  \TestDec{1000}{999}%
  \TestDec{-999}{-1000}%
  \TestDec{130}{129}%
  \TestDec{2147483647}{2147483646}%
  \TestDec{-2147483646}{-2147483647}%
  \TestDec{12345678901234567891}{12345678901234567890}%
  \TestDec{100000000000000000000}{99999999999999999999}%
  \TestDec{-12345678901234567890}{-12345678901234567891}%
  \TestDec{-99999999999999999999}{-100000000000000000000}%
\end{qstest}

\begin{qstest}{add}{add}%
  \TestAdd{0}{0}{0}%
  \TestAdd{1}{0}{1}%
  \TestAdd{0}{1}{1}%
  \TestAdd{1}{2}{3}%
  \TestAdd{-1}{-1}{-2}%
  \TestAdd{2147483646}{1}{2147483647}%
  \TestAdd{-2147483647}{2147483647}{0}%
  \TestAdd{20}{-5}{15}%
  \TestAdd{-4}{-1}{-5}%
  \TestAdd{-1}{-4}{-5}%
  \TestAdd{-4}{1}{-3}%
  \TestAdd{-1}{4}{3}%
  \TestAdd{4}{-1}{3}%
  \TestAdd{1}{-4}{-3}%
  \TestAdd{-4}{-1}{-5}%
  \TestAdd{-1}{-4}{-5}%
  \TestAdd{ -4 }{ -1 }{-5}%
  \TestAdd{ -1 }{ -4 }{-5}%
  \TestAdd{ -4 }{ 1 }{-3}%
  \TestAdd{ -1 }{ 4 }{3}%
  \TestAdd{ 4 }{ -1 }{3}%
  \TestAdd{ 1 }{ -4 }{-3}%
  \TestAdd{ -4 }{ -1 }{-5}%
  \TestAdd{ -1 }{ -4 }{-5}%
  \TestAdd{876543210}{111111111}{987654321}%
  \TestAdd{999999999}{2}{1000000001}%
\end{qstest}

\begin{qstest}{sub}{sub}
  \TestSub{0}{0}{0}%
  \TestSub{1}{0}{1}%
  \TestSub{1}{2}{-1}%
  \TestSub{-1}{-1}{0}%
  \TestSub{2147483646}{-1}{2147483647}%
  \TestSub{-2147483647}{-2147483647}{0}%
  \TestSub{-4}{-1}{-3}%
  \TestSub{-1}{-4}{3}%
  \TestSub{-4}{1}{-5}%
  \TestSub{-1}{4}{-5}%
  \TestSub{4}{-1}{5}%
  \TestSub{1}{-4}{5}%
  \TestSub{-4}{-1}{-3}%
  \TestSub{-1}{-4}{3}%
  \TestSub{ -4 }{ -1 }{-3}%
  \TestSub{ -1 }{ -4 }{3}%
  \TestSub{ -4 }{ 1 }{-5}%
  \TestSub{ -1 }{ 4 }{-5}%
  \TestSub{ 4 }{ -1 }{5}%
  \TestSub{ 1 }{ -4 }{5}%
  \TestSub{ -4 }{ -1 }{-3}%
  \TestSub{ -1 }{ -4 }{3}%
  \TestSub{1000000000}{2}{999999998}%
  \TestSub{987654321}{111111111}{876543210}%
\end{qstest}

\begin{qstest}{shl}{shl}
  \TestShl{0}{0}%
  \TestShl{1}{2}%
  \TestShl{2}{4}%
  \TestShl{5621}{11242}%
  \TestShl{1073741823}{2147483646}%
\end{qstest}

\begin{qstest}{shr}{shr}
  \TestShr{0}{0}%
  \TestShr{1}{0}%
  \TestShr{2}{1}%
  \TestShr{3}{1}%
  \TestShr{4}{2}%
  \TestShr{5}{2}%
  \TestShr{6}{3}%
  \TestShr{7}{3}%
  \TestShr{8}{4}%
  \TestShr{9}{4}%
  \TestShr{10}{5}%
  \TestShr{11}{5}%
  \TestShr{12}{6}%
  \TestShr{13}{6}%
  \TestShr{14}{7}%
  \TestShr{15}{7}%
  \TestShr{16}{8}%
  \TestShr{17}{8}%
  \TestShr{18}{9}%
  \TestShr{19}{9}%
  \TestShr{20}{10}%
  \TestShr{21}{10}%
  \TestShr{22}{11}%
  \TestShr{11241}{5620}%
  \TestShr{73054202}{36527101}%
  \TestShr{2147483646}{1073741823}%
\end{qstest}

\begin{qstest}{mul}{mul}
  \TestMul{0}{0}{0}%
  \TestMul{1}{0}{0}%
  \TestMul{0}{1}{0}%
  \TestMul{1}{1}{1}%
  \TestMul{3}{1}{3}%
  \TestMul{1}{-3}{-3}%
  \TestMul{-4}{-5}{20}%
  \TestMul{3}{7}{21}%
  \TestMul{7}{3}{21}%
  \TestMul{3}{-7}{-21}%
  \TestMul{7}{-3}{-21}%
  \TestMul{-3}{7}{-21}%
  \TestMul{-7}{3}{-21}%
  \TestMul{-3}{-7}{21}%
  \TestMul{-7}{-3}{21}%
  \TestMul{12}{11}{132}%
  \TestMul{999}{333}{332667}%
  \TestMul{1000}{4321}{4321000}%
  \TestMul{12345}{173955}{2147474475}%
  \TestMul{1073741823}{2}{2147483646}%
  \TestMul{2}{1073741823}{2147483646}%
  \TestMul{-1073741823}{2}{-2147483646}%
  \TestMul{2}{-1073741823}{-2147483646}%
  \TestMul{6706022400}{13}{87178291200}%
\end{qstest}

\begin{qstest}{sqr}{sqr}
  \TestSqr{0}{0}%
  \TestSqr{1}{1}%
  \TestSqr{2}{4}%
  \TestSqr{3}{9}%
  \TestSqr{4}{16}%
  \TestSqr{9}{81}%
  \TestSqr{10}{100}%
  \TestSqr{46340}{2147395600}%
  \TestSqr{-1}{1}%
  \TestSqr{-2}{4}%
  \TestSqr{-46340}{2147395600}%
\end{qstest}

\begin{qstest}{fac}{fac}
  \TestFac{0}{1}%
  \TestFac{1}{1}%
  \TestFac{2}{2}%
  \TestFac{3}{2*3}%
  \TestFac{4}{2*3*4}%
  \TestFac{5}{2*3*4*5}%
  \TestFac{6}{2*3*4*5*6}%
  \TestFac{7}{2*3*4*5*6*7}%
  \TestFac{8}{2*3*4*5*6*7*8}%
  \TestFac{9}{2*3*4*5*6*7*8*9}%
  \TestFac{10}{2*3*4*5*6*7*8*9*10}%
  \TestFac{11}{2*3*4*5*6*7*8*9*10*11}%
  \TestFac{12}{2*3*4*5*6*7*8*9*10*11*12}%
  \TestFacBig{13}{6227020800}%
  \TestFacBig{14}{87178291200}%
  \TestFacBig{15}{1307674368000}%
  \TestFacBig{16}{20922789888000}%
  \TestFacBig{17}{355687428096000}%
  \TestFacBig{18}{6402373705728000}%
  \TestFacBig{19}{121645100408832000}%
  \TestFacBig{20}{2432902008176640000}%
  \TestFacBig{21}{51090942171709440000}%
  \TestFacBig{22}{1124000727777607680000}%
\end{qstest}

\begin{qstest}{pow}{pow}
  \TestPow{-2}{0}{1}%
  \TestPow{-1}{0}{1}%
  \TestPow{0}{0}{1}%
  \TestPow{1}{0}{1}%
  \TestPow{2}{0}{1}%
  \TestPow{3}{0}{1}%
  \TestPow{-2}{1}{-2}%
  \TestPow{-1}{1}{-1}%
  \TestPow{1}{1}{1}%
  \TestPow{2}{1}{2}%
  \TestPow{3}{1}{3}%
  \TestPow{-2}{2}{4}%
  \TestPow{-1}{2}{1}%
  \TestPow{0}{2}{0}%
  \TestPow{1}{2}{1}%
  \TestPow{2}{2}{4}%
  \TestPow{3}{2}{9}%
  \TestPow{0}{1}{0}%
  \TestPow{1}{-2}{1}%
  \TestPow{1}{-1}{1}%
  \TestPow{-1}{-2}{1}%
  \TestPow{-1}{-1}{-1}%
  \TestPow{-1}{3}{-1}%
  \TestPow{-1}{4}{1}%
  \TestPow{-2}{-1}{0}%
  \TestPow{-2}{-2}{0}%
  \TestPow{2}{3}{8}%
  \TestPow{2}{4}{16}%
  \TestPow{2}{5}{32}%
  \TestPow{2}{6}{64}%
  \TestPow{2}{7}{128}%
  \TestPow{2}{8}{256}%
  \TestPow{2}{9}{512}%
  \TestPow{2}{10}{1024}%
  \TestPow{-2}{3}{-8}%
  \TestPow{-2}{4}{16}%
  \TestPow{-2}{5}{-32}%
  \TestPow{-2}{6}{64}%
  \TestPow{-2}{7}{-128}%
  \TestPow{-2}{8}{256}%
  \TestPow{-2}{9}{-512}%
  \TestPow{-2}{10}{1024}%
  \TestPow{3}{3}{27}%
  \TestPow{3}{4}{81}%
  \TestPow{3}{5}{243}%
  \TestPow{-3}{3}{-27}%
  \TestPow{-3}{4}{81}%
  \TestPow{-3}{5}{-243}%
  \TestPow{2}{30}{1073741824}%
  \TestPow{-3}{19}{-1162261467}%
  \TestPow{5}{13}{1220703125}%
  \TestPow{-7}{11}{-1977326743}%
\end{qstest}

\begin{qstest}{div}{div}
  \TestDiv{1}{1}{1}%
  \TestDiv{2}{1}{2}%
  \TestDiv{-2}{1}{-2}%
  \TestDiv{2}{-1}{-2}%
  \TestDiv{-2}{-1}{2}%
  \TestDiv{15}{2}{7}%
  \TestDiv{-16}{2}{-8}%
  \TestDiv{1}{2}{0}%
  \TestDiv{1}{3}{0}%
  \TestDiv{2}{3}{0}%
  \TestDiv{-2}{3}{0}%
  \TestDiv{2}{-3}{0}%
  \TestDiv{-2}{-3}{0}%
  \TestDiv{13}{3}{4}%
  \TestDiv{-13}{-3}{4}%
  \TestDiv{-13}{3}{-4}%
  \TestDiv{-6}{5}{-1}%
  \TestDiv{-5}{5}{-1}%
  \TestDiv{-4}{5}{0}%
  \TestDiv{-3}{5}{0}%
  \TestDiv{-2}{5}{0}%
  \TestDiv{-1}{5}{0}%
  \TestDiv{0}{5}{0}%
  \TestDiv{1}{5}{0}%
  \TestDiv{2}{5}{0}%
  \TestDiv{3}{5}{0}%
  \TestDiv{4}{5}{0}%
  \TestDiv{5}{5}{1}%
  \TestDiv{6}{5}{1}%
  \TestDiv{-5}{4}{-1}%
  \TestDiv{-4}{4}{-1}%
  \TestDiv{-3}{4}{0}%
  \TestDiv{-2}{4}{0}%
  \TestDiv{-1}{4}{0}%
  \TestDiv{0}{4}{0}%
  \TestDiv{1}{4}{0}%
  \TestDiv{2}{4}{0}%
  \TestDiv{3}{4}{0}%
  \TestDiv{4}{4}{1}%
  \TestDiv{5}{4}{1}%
  \TestDiv{12345}{678}{18}%
  \TestDiv{32372}{5952}{5}%
  \TestDiv{284271294}{18162}{15651}%
  \TestDiv{217652429}{12561}{17327}%
  \TestDiv{462028434}{5439}{84947}%
  \TestDiv{2147483647}{1000}{2147483}%
  \TestDiv{2147483647}{-1000}{-2147483}%
  \TestDiv{-2147483647}{1000}{-2147483}%
  \TestDiv{-2147483647}{-1000}{2147483}%
  \TestDiv{0}{3}{0}%
  \TestDiv{1}{3}{0}%
  \TestDiv{2}{3}{0}%
  \TestDiv{3}{3}{1}%
  \TestDiv{4}{3}{1}%
  \TestDiv{5}{3}{1}%
  \TestDiv{6}{3}{2}%
  \TestDiv{7}{3}{2}%
  \TestDiv{8}{3}{2}%
  \TestDiv{9}{3}{3}%
  \TestDiv{10}{3}{3}%
  \TestDiv{11}{3}{3}%
  \TestDiv{12}{3}{4}%
  \TestDiv{13}{3}{4}%
  \TestDiv{14}{3}{4}%
  \TestDiv{15}{3}{5}%
  \TestDiv{16}{3}{5}%
  \TestDiv{17}{3}{5}%
  \TestDiv{18}{3}{6}%
  \TestDiv{19}{3}{6}%
  \TestDiv{20}{3}{6}%
  \TestDiv{21}{3}{7}%
  \TestDiv{22}{3}{7}%
  \TestDiv{23}{3}{7}%
  \TestDiv{24}{3}{8}%
  \TestDiv{25}{3}{8}%
  \TestDiv{26}{3}{8}%
  \TestDiv{27}{3}{9}%
  \TestDiv{28}{3}{9}%
  \TestDiv{29}{3}{9}%
  \TestDiv{30}{3}{10}%
  \TestDiv{31}{3}{10}%
  \TestDivBig{17363436332507}{24702}{702916214}%
\end{qstest}

\begin{qstest}{mod}{mod}
  \TestMod{-6}{5}{4}%
  \TestMod{-5}{5}{0}%
  \TestMod{-4}{5}{1}%
  \TestMod{-3}{5}{2}%
  \TestMod{-2}{5}{3}%
  \TestMod{-1}{5}{4}%
  \TestMod{0}{5}{0}%
  \TestMod{1}{5}{1}%
  \TestMod{2}{5}{2}%
  \TestMod{3}{5}{3}%
  \TestMod{4}{5}{4}%
  \TestMod{5}{5}{0}%
  \TestMod{6}{5}{1}%
  \TestMod{-5}{4}{3}%
  \TestMod{-4}{4}{0}%
  \TestMod{-3}{4}{1}%
  \TestMod{-2}{4}{2}%
  \TestMod{-1}{4}{3}%
  \TestMod{0}{4}{0}%
  \TestMod{1}{4}{1}%
  \TestMod{2}{4}{2}%
  \TestMod{3}{4}{3}%
  \TestMod{4}{4}{0}%
  \TestMod{5}{4}{1}%
  \TestMod{-6}{-5}{-1}%
  \TestMod{-5}{-5}{0}%
  \TestMod{-4}{-5}{-4}%
  \TestMod{-3}{-5}{-3}%
  \TestMod{-2}{-5}{-2}%
  \TestMod{-1}{-5}{-1}%
  \TestMod{0}{-5}{0}%
  \TestMod{1}{-5}{-4}%
  \TestMod{2}{-5}{-3}%
  \TestMod{3}{-5}{-2}%
  \TestMod{4}{-5}{-1}%
  \TestMod{5}{-5}{0}%
  \TestMod{6}{-5}{-4}%
  \TestMod{-5}{-4}{-1}%
  \TestMod{-4}{-4}{0}%
  \TestMod{-3}{-4}{-3}%
  \TestMod{-2}{-4}{-2}%
  \TestMod{-1}{-4}{-1}%
  \TestMod{0}{-4}{0}%
  \TestMod{1}{-4}{-3}%
  \TestMod{2}{-4}{-2}%
  \TestMod{3}{-4}{-1}%
  \TestMod{4}{-4}{0}%
  \TestMod{5}{-4}{-3}%
  \TestMod{2147483647}{1000}{647}%
  \TestMod{2147483647}{-1000}{-353}%
  \TestMod{-2147483647}{1000}{353}%
  \TestMod{-2147483647}{-1000}{-647}%
  \TestMod{ 0 }{ 4 }{0}%
  \TestMod{ 1 }{ 4 }{1}%
  \TestMod{ -1 }{ 4 }{3}%
  \TestMod{ 0 }{ -4 }{0}%
  \TestMod{ 1 }{ -4 }{-3}%
  \TestMod{ -1 }{ -4 }{-1}%
  \TestMod{18362}{25}{12}%
\end{qstest}

\newcommand*{\TestError}[2]{%
  \begingroup
    \expandafter\def\csname BigIntCalcError:#1\endcsname{}%
    \Expect*{#2}{0}%
    \expandafter\def\csname BigIntCalcError:#1\endcsname{ERROR}%
    \Expect*{#2}{0ERROR}%
  \endgroup
}
\begin{qstest}{error}{error}
  \TestError{FacNegative}{\bigintcalcFac{-1}}%
  \TestError{FacNegative}{\bigintcalcFac{-2147483647}}%
  \TestError{DivisionByZero}{\bigintcalcPow{0}{-1}}%
  \TestError{DivisionByZero}{\bigintcalcDiv{1}{0}}%
  \TestError{DivisionByZero}{\bigintcalcMod{1}{0}}%
\end{qstest}

\begin{document}
\end{document}
%</test2>
%    \end{macrocode}
%
% \section{Installation}
%
% \subsection{Download}
%
% \paragraph{Package.} This package is available on
% CTAN\footnote{\url{http://ctan.org/pkg/bigintcalc}}:
% \begin{description}
% \item[\CTAN{macros/latex/contrib/oberdiek/bigintcalc.dtx}] The source file.
% \item[\CTAN{macros/latex/contrib/oberdiek/bigintcalc.pdf}] Documentation.
% \end{description}
%
%
% \paragraph{Bundle.} All the packages of the bundle `oberdiek'
% are also available in a TDS compliant ZIP archive. There
% the packages are already unpacked and the documentation files
% are generated. The files and directories obey the TDS standard.
% \begin{description}
% \item[\CTAN{install/macros/latex/contrib/oberdiek.tds.zip}]
% \end{description}
% \emph{TDS} refers to the standard ``A Directory Structure
% for \TeX\ Files'' (\CTAN{tds/tds.pdf}). Directories
% with \xfile{texmf} in their name are usually organized this way.
%
% \subsection{Bundle installation}
%
% \paragraph{Unpacking.} Unpack the \xfile{oberdiek.tds.zip} in the
% TDS tree (also known as \xfile{texmf} tree) of your choice.
% Example (linux):
% \begin{quote}
%   |unzip oberdiek.tds.zip -d ~/texmf|
% \end{quote}
%
% \paragraph{Script installation.}
% Check the directory \xfile{TDS:scripts/oberdiek/} for
% scripts that need further installation steps.
% Package \xpackage{attachfile2} comes with the Perl script
% \xfile{pdfatfi.pl} that should be installed in such a way
% that it can be called as \texttt{pdfatfi}.
% Example (linux):
% \begin{quote}
%   |chmod +x scripts/oberdiek/pdfatfi.pl|\\
%   |cp scripts/oberdiek/pdfatfi.pl /usr/local/bin/|
% \end{quote}
%
% \subsection{Package installation}
%
% \paragraph{Unpacking.} The \xfile{.dtx} file is a self-extracting
% \docstrip\ archive. The files are extracted by running the
% \xfile{.dtx} through \plainTeX:
% \begin{quote}
%   \verb|tex bigintcalc.dtx|
% \end{quote}
%
% \paragraph{TDS.} Now the different files must be moved into
% the different directories in your installation TDS tree
% (also known as \xfile{texmf} tree):
% \begin{quote}
% \def\t{^^A
% \begin{tabular}{@{}>{\ttfamily}l@{ $\rightarrow$ }>{\ttfamily}l@{}}
%   bigintcalc.sty & tex/generic/oberdiek/bigintcalc.sty\\
%   bigintcalc.pdf & doc/latex/oberdiek/bigintcalc.pdf\\
%   test/bigintcalc-test1.tex & doc/latex/oberdiek/test/bigintcalc-test1.tex\\
%   test/bigintcalc-test2.tex & doc/latex/oberdiek/test/bigintcalc-test2.tex\\
%   test/bigintcalc-test3.tex & doc/latex/oberdiek/test/bigintcalc-test3.tex\\
%   bigintcalc.dtx & source/latex/oberdiek/bigintcalc.dtx\\
% \end{tabular}^^A
% }^^A
% \sbox0{\t}^^A
% \ifdim\wd0>\linewidth
%   \begingroup
%     \advance\linewidth by\leftmargin
%     \advance\linewidth by\rightmargin
%   \edef\x{\endgroup
%     \def\noexpand\lw{\the\linewidth}^^A
%   }\x
%   \def\lwbox{^^A
%     \leavevmode
%     \hbox to \linewidth{^^A
%       \kern-\leftmargin\relax
%       \hss
%       \usebox0
%       \hss
%       \kern-\rightmargin\relax
%     }^^A
%   }^^A
%   \ifdim\wd0>\lw
%     \sbox0{\small\t}^^A
%     \ifdim\wd0>\linewidth
%       \ifdim\wd0>\lw
%         \sbox0{\footnotesize\t}^^A
%         \ifdim\wd0>\linewidth
%           \ifdim\wd0>\lw
%             \sbox0{\scriptsize\t}^^A
%             \ifdim\wd0>\linewidth
%               \ifdim\wd0>\lw
%                 \sbox0{\tiny\t}^^A
%                 \ifdim\wd0>\linewidth
%                   \lwbox
%                 \else
%                   \usebox0
%                 \fi
%               \else
%                 \lwbox
%               \fi
%             \else
%               \usebox0
%             \fi
%           \else
%             \lwbox
%           \fi
%         \else
%           \usebox0
%         \fi
%       \else
%         \lwbox
%       \fi
%     \else
%       \usebox0
%     \fi
%   \else
%     \lwbox
%   \fi
% \else
%   \usebox0
% \fi
% \end{quote}
% If you have a \xfile{docstrip.cfg} that configures and enables \docstrip's
% TDS installing feature, then some files can already be in the right
% place, see the documentation of \docstrip.
%
% \subsection{Refresh file name databases}
%
% If your \TeX~distribution
% (\teTeX, \mikTeX, \dots) relies on file name databases, you must refresh
% these. For example, \teTeX\ users run \verb|texhash| or
% \verb|mktexlsr|.
%
% \subsection{Some details for the interested}
%
% \paragraph{Attached source.}
%
% The PDF documentation on CTAN also includes the
% \xfile{.dtx} source file. It can be extracted by
% AcrobatReader 6 or higher. Another option is \textsf{pdftk},
% e.g. unpack the file into the current directory:
% \begin{quote}
%   \verb|pdftk bigintcalc.pdf unpack_files output .|
% \end{quote}
%
% \paragraph{Unpacking with \LaTeX.}
% The \xfile{.dtx} chooses its action depending on the format:
% \begin{description}
% \item[\plainTeX:] Run \docstrip\ and extract the files.
% \item[\LaTeX:] Generate the documentation.
% \end{description}
% If you insist on using \LaTeX\ for \docstrip\ (really,
% \docstrip\ does not need \LaTeX), then inform the autodetect routine
% about your intention:
% \begin{quote}
%   \verb|latex \let\install=y\input{bigintcalc.dtx}|
% \end{quote}
% Do not forget to quote the argument according to the demands
% of your shell.
%
% \paragraph{Generating the documentation.}
% You can use both the \xfile{.dtx} or the \xfile{.drv} to generate
% the documentation. The process can be configured by the
% configuration file \xfile{ltxdoc.cfg}. For instance, put this
% line into this file, if you want to have A4 as paper format:
% \begin{quote}
%   \verb|\PassOptionsToClass{a4paper}{article}|
% \end{quote}
% An example follows how to generate the
% documentation with pdf\LaTeX:
% \begin{quote}
%\begin{verbatim}
%pdflatex bigintcalc.dtx
%makeindex -s gind.ist bigintcalc.idx
%pdflatex bigintcalc.dtx
%makeindex -s gind.ist bigintcalc.idx
%pdflatex bigintcalc.dtx
%\end{verbatim}
% \end{quote}
%
% \section{Catalogue}
%
% The following XML file can be used as source for the
% \href{http://mirror.ctan.org/help/Catalogue/catalogue.html}{\TeX\ Catalogue}.
% The elements \texttt{caption} and \texttt{description} are imported
% from the original XML file from the Catalogue.
% The name of the XML file in the Catalogue is \xfile{bigintcalc.xml}.
%    \begin{macrocode}
%<*catalogue>
<?xml version='1.0' encoding='us-ascii'?>
<!DOCTYPE entry SYSTEM 'catalogue.dtd'>
<entry datestamp='$Date$' modifier='$Author$' id='bigintcalc'>
  <name>bigintcalc</name>
  <caption>Integer calculations on very large numbers.</caption>
  <authorref id='auth:oberdiek'/>
  <copyright owner='Heiko Oberdiek' year='2007,2011,2012'/>
  <license type='lppl1.3'/>
  <version number='1.4'/>
  <description>
    This package provides expandable arithmetic operations
    with big integers that can exceed TeX's number limits.
    <p/>
    The package is part of the <xref refid='oberdiek'>oberdiek</xref> bundle.
  </description>
  <documentation details='Package documentation'
      href='ctan:/macros/latex/contrib/oberdiek/bigintcalc.pdf'/>
  <ctan file='true' path='/macros/latex/contrib/oberdiek/bigintcalc.dtx'/>
  <miktex location='oberdiek'/>
  <texlive location='oberdiek'/>
  <install path='/macros/latex/contrib/oberdiek/oberdiek.tds.zip'/>
</entry>
%</catalogue>
%    \end{macrocode}
%
% \begin{History}
%   \begin{Version}{2007/09/27 v1.0}
%   \item
%     First version.
%   \end{Version}
%   \begin{Version}{2007/11/11 v1.1}
%   \item
%     Use of package \xpackage{pdftexcmds} for \LuaTeX\ support.
%   \end{Version}
%   \begin{Version}{2011/01/30 v1.2}
%   \item
%     Already loaded package files are not input in \hologo{plainTeX}.
%   \end{Version}
%   \begin{Version}{2012/04/08 v1.3}
%   \item
%     Fix: \xpackage{pdftexcmds} wasn't loaded in case of \hologo{LaTeX}.
%   \end{Version}
%   \begin{Version}{2016/05/16 v1.4}
%   \item
%     Documentation updates.
%   \end{Version}
% \end{History}
%
% \PrintIndex
%
% \Finale
\endinput
|
% \end{quote}
% Do not forget to quote the argument according to the demands
% of your shell.
%
% \paragraph{Generating the documentation.}
% You can use both the \xfile{.dtx} or the \xfile{.drv} to generate
% the documentation. The process can be configured by the
% configuration file \xfile{ltxdoc.cfg}. For instance, put this
% line into this file, if you want to have A4 as paper format:
% \begin{quote}
%   \verb|\PassOptionsToClass{a4paper}{article}|
% \end{quote}
% An example follows how to generate the
% documentation with pdf\LaTeX:
% \begin{quote}
%\begin{verbatim}
%pdflatex bigintcalc.dtx
%makeindex -s gind.ist bigintcalc.idx
%pdflatex bigintcalc.dtx
%makeindex -s gind.ist bigintcalc.idx
%pdflatex bigintcalc.dtx
%\end{verbatim}
% \end{quote}
%
% \section{Catalogue}
%
% The following XML file can be used as source for the
% \href{http://mirror.ctan.org/help/Catalogue/catalogue.html}{\TeX\ Catalogue}.
% The elements \texttt{caption} and \texttt{description} are imported
% from the original XML file from the Catalogue.
% The name of the XML file in the Catalogue is \xfile{bigintcalc.xml}.
%    \begin{macrocode}
%<*catalogue>
<?xml version='1.0' encoding='us-ascii'?>
<!DOCTYPE entry SYSTEM 'catalogue.dtd'>
<entry datestamp='$Date$' modifier='$Author$' id='bigintcalc'>
  <name>bigintcalc</name>
  <caption>Integer calculations on very large numbers.</caption>
  <authorref id='auth:oberdiek'/>
  <copyright owner='Heiko Oberdiek' year='2007,2011,2012'/>
  <license type='lppl1.3'/>
  <version number='1.4'/>
  <description>
    This package provides expandable arithmetic operations
    with big integers that can exceed TeX's number limits.
    <p/>
    The package is part of the <xref refid='oberdiek'>oberdiek</xref> bundle.
  </description>
  <documentation details='Package documentation'
      href='ctan:/macros/latex/contrib/oberdiek/bigintcalc.pdf'/>
  <ctan file='true' path='/macros/latex/contrib/oberdiek/bigintcalc.dtx'/>
  <miktex location='oberdiek'/>
  <texlive location='oberdiek'/>
  <install path='/macros/latex/contrib/oberdiek/oberdiek.tds.zip'/>
</entry>
%</catalogue>
%    \end{macrocode}
%
% \begin{History}
%   \begin{Version}{2007/09/27 v1.0}
%   \item
%     First version.
%   \end{Version}
%   \begin{Version}{2007/11/11 v1.1}
%   \item
%     Use of package \xpackage{pdftexcmds} for \LuaTeX\ support.
%   \end{Version}
%   \begin{Version}{2011/01/30 v1.2}
%   \item
%     Already loaded package files are not input in \hologo{plainTeX}.
%   \end{Version}
%   \begin{Version}{2012/04/08 v1.3}
%   \item
%     Fix: \xpackage{pdftexcmds} wasn't loaded in case of \hologo{LaTeX}.
%   \end{Version}
%   \begin{Version}{2016/05/16 v1.4}
%   \item
%     Documentation updates.
%   \end{Version}
% \end{History}
%
% \PrintIndex
%
% \Finale
\endinput
|
% \end{quote}
% Do not forget to quote the argument according to the demands
% of your shell.
%
% \paragraph{Generating the documentation.}
% You can use both the \xfile{.dtx} or the \xfile{.drv} to generate
% the documentation. The process can be configured by the
% configuration file \xfile{ltxdoc.cfg}. For instance, put this
% line into this file, if you want to have A4 as paper format:
% \begin{quote}
%   \verb|\PassOptionsToClass{a4paper}{article}|
% \end{quote}
% An example follows how to generate the
% documentation with pdf\LaTeX:
% \begin{quote}
%\begin{verbatim}
%pdflatex bigintcalc.dtx
%makeindex -s gind.ist bigintcalc.idx
%pdflatex bigintcalc.dtx
%makeindex -s gind.ist bigintcalc.idx
%pdflatex bigintcalc.dtx
%\end{verbatim}
% \end{quote}
%
% \section{Catalogue}
%
% The following XML file can be used as source for the
% \href{http://mirror.ctan.org/help/Catalogue/catalogue.html}{\TeX\ Catalogue}.
% The elements \texttt{caption} and \texttt{description} are imported
% from the original XML file from the Catalogue.
% The name of the XML file in the Catalogue is \xfile{bigintcalc.xml}.
%    \begin{macrocode}
%<*catalogue>
<?xml version='1.0' encoding='us-ascii'?>
<!DOCTYPE entry SYSTEM 'catalogue.dtd'>
<entry datestamp='$Date$' modifier='$Author$' id='bigintcalc'>
  <name>bigintcalc</name>
  <caption>Integer calculations on very large numbers.</caption>
  <authorref id='auth:oberdiek'/>
  <copyright owner='Heiko Oberdiek' year='2007,2011,2012'/>
  <license type='lppl1.3'/>
  <version number='1.4'/>
  <description>
    This package provides expandable arithmetic operations
    with big integers that can exceed TeX's number limits.
    <p/>
    The package is part of the <xref refid='oberdiek'>oberdiek</xref> bundle.
  </description>
  <documentation details='Package documentation'
      href='ctan:/macros/latex/contrib/oberdiek/bigintcalc.pdf'/>
  <ctan file='true' path='/macros/latex/contrib/oberdiek/bigintcalc.dtx'/>
  <miktex location='oberdiek'/>
  <texlive location='oberdiek'/>
  <install path='/macros/latex/contrib/oberdiek/oberdiek.tds.zip'/>
</entry>
%</catalogue>
%    \end{macrocode}
%
% \begin{History}
%   \begin{Version}{2007/09/27 v1.0}
%   \item
%     First version.
%   \end{Version}
%   \begin{Version}{2007/11/11 v1.1}
%   \item
%     Use of package \xpackage{pdftexcmds} for \LuaTeX\ support.
%   \end{Version}
%   \begin{Version}{2011/01/30 v1.2}
%   \item
%     Already loaded package files are not input in \hologo{plainTeX}.
%   \end{Version}
%   \begin{Version}{2012/04/08 v1.3}
%   \item
%     Fix: \xpackage{pdftexcmds} wasn't loaded in case of \hologo{LaTeX}.
%   \end{Version}
%   \begin{Version}{2016/05/16 v1.4}
%   \item
%     Documentation updates.
%   \end{Version}
% \end{History}
%
% \PrintIndex
%
% \Finale
\endinput

%        (quote the arguments according to the demands of your shell)
%
% Documentation:
%    (a) If bigintcalc.drv is present:
%           latex bigintcalc.drv
%    (b) Without bigintcalc.drv:
%           latex bigintcalc.dtx; ...
%    The class ltxdoc loads the configuration file ltxdoc.cfg
%    if available. Here you can specify further options, e.g.
%    use A4 as paper format:
%       \PassOptionsToClass{a4paper}{article}
%
%    Programm calls to get the documentation (example):
%       pdflatex bigintcalc.dtx
%       makeindex -s gind.ist bigintcalc.idx
%       pdflatex bigintcalc.dtx
%       makeindex -s gind.ist bigintcalc.idx
%       pdflatex bigintcalc.dtx
%
% Installation:
%    TDS:tex/generic/oberdiek/bigintcalc.sty
%    TDS:doc/latex/oberdiek/bigintcalc.pdf
%    TDS:doc/latex/oberdiek/test/bigintcalc-test1.tex
%    TDS:doc/latex/oberdiek/test/bigintcalc-test2.tex
%    TDS:doc/latex/oberdiek/test/bigintcalc-test3.tex
%    TDS:source/latex/oberdiek/bigintcalc.dtx
%
%<*ignore>
\begingroup
  \catcode123=1 %
  \catcode125=2 %
  \def\x{LaTeX2e}%
\expandafter\endgroup
\ifcase 0\ifx\install y1\fi\expandafter
         \ifx\csname processbatchFile\endcsname\relax\else1\fi
         \ifx\fmtname\x\else 1\fi\relax
\else\csname fi\endcsname
%</ignore>
%<*install>
\input docstrip.tex
\Msg{************************************************************************}
\Msg{* Installation}
\Msg{* Package: bigintcalc 2012/04/08 v1.3 Expandable calculations on big integers (HO)}
\Msg{************************************************************************}

\keepsilent
\askforoverwritefalse

\let\MetaPrefix\relax
\preamble

This is a generated file.

Project: bigintcalc
Version: 2012/04/08 v1.3

Copyright (C) 2007, 2011, 2012 by
   Heiko Oberdiek <heiko.oberdiek at googlemail.com>

This work may be distributed and/or modified under the
conditions of the LaTeX Project Public License, either
version 1.3c of this license or (at your option) any later
version. This version of this license is in
   http://www.latex-project.org/lppl/lppl-1-3c.txt
and the latest version of this license is in
   http://www.latex-project.org/lppl.txt
and version 1.3 or later is part of all distributions of
LaTeX version 2005/12/01 or later.

This work has the LPPL maintenance status "maintained".

This Current Maintainer of this work is Heiko Oberdiek.

The Base Interpreter refers to any `TeX-Format',
because some files are installed in TDS:tex/generic//.

This work consists of the main source file bigintcalc.dtx
and the derived files
   bigintcalc.sty, bigintcalc.pdf, bigintcalc.ins, bigintcalc.drv,
   bigintcalc-test1.tex, bigintcalc-test2.tex,
   bigintcalc-test3.tex.

\endpreamble
\let\MetaPrefix\DoubleperCent

\generate{%
  \file{bigintcalc.ins}{\from{bigintcalc.dtx}{install}}%
  \file{bigintcalc.drv}{\from{bigintcalc.dtx}{driver}}%
  \usedir{tex/generic/oberdiek}%
  \file{bigintcalc.sty}{\from{bigintcalc.dtx}{package}}%
  \usedir{doc/latex/oberdiek/test}%
  \file{bigintcalc-test1.tex}{\from{bigintcalc.dtx}{test1}}%
  \file{bigintcalc-test2.tex}{\from{bigintcalc.dtx}{test2,etex}}%
  \file{bigintcalc-test3.tex}{\from{bigintcalc.dtx}{test2,noetex}}%
  \nopreamble
  \nopostamble
  \usedir{source/latex/oberdiek/catalogue}%
  \file{bigintcalc.xml}{\from{bigintcalc.dtx}{catalogue}}%
}

\catcode32=13\relax% active space
\let =\space%
\Msg{************************************************************************}
\Msg{*}
\Msg{* To finish the installation you have to move the following}
\Msg{* file into a directory searched by TeX:}
\Msg{*}
\Msg{*     bigintcalc.sty}
\Msg{*}
\Msg{* To produce the documentation run the file `bigintcalc.drv'}
\Msg{* through LaTeX.}
\Msg{*}
\Msg{* Happy TeXing!}
\Msg{*}
\Msg{************************************************************************}

\endbatchfile
%</install>
%<*ignore>
\fi
%</ignore>
%<*driver>
\NeedsTeXFormat{LaTeX2e}
\ProvidesFile{bigintcalc.drv}%
  [2012/04/08 v1.3 Expandable calculations on big integers (HO)]%
\documentclass{ltxdoc}
\usepackage{wasysym}
\let\iint\relax
\let\iiint\relax
\usepackage[fleqn]{amsmath}

\DeclareMathOperator{\opInv}{Inv}
\DeclareMathOperator{\opAbs}{Abs}
\DeclareMathOperator{\opSgn}{Sgn}
\DeclareMathOperator{\opMin}{Min}
\DeclareMathOperator{\opMax}{Max}
\DeclareMathOperator{\opCmp}{Cmp}
\DeclareMathOperator{\opOdd}{Odd}
\DeclareMathOperator{\opInc}{Inc}
\DeclareMathOperator{\opDec}{Dec}
\DeclareMathOperator{\opAdd}{Add}
\DeclareMathOperator{\opSub}{Sub}
\DeclareMathOperator{\opShl}{Shl}
\DeclareMathOperator{\opShr}{Shr}
\DeclareMathOperator{\opMul}{Mul}
\DeclareMathOperator{\opSqr}{Sqr}
\DeclareMathOperator{\opFac}{Fac}
\DeclareMathOperator{\opPow}{Pow}
\DeclareMathOperator{\opDiv}{Div}
\DeclareMathOperator{\opMod}{Mod}
\DeclareMathOperator{\opInt}{Int}
\DeclareMathOperator{\opODD}{ifodd}

\newcommand*{\Def}{%
  \ensuremath{%
    \mathrel{\mathop{:}}=%
  }%
}
\newcommand*{\op}[1]{%
  \textsf{#1}%
}
\usepackage{holtxdoc}[2011/11/22]
\begin{document}
  \DocInput{bigintcalc.dtx}%
\end{document}
%</driver>
% \fi
%
% \CheckSum{3723}
%
% \CharacterTable
%  {Upper-case    \A\B\C\D\E\F\G\H\I\J\K\L\M\N\O\P\Q\R\S\T\U\V\W\X\Y\Z
%   Lower-case    \a\b\c\d\e\f\g\h\i\j\k\l\m\n\o\p\q\r\s\t\u\v\w\x\y\z
%   Digits        \0\1\2\3\4\5\6\7\8\9
%   Exclamation   \!     Double quote  \"     Hash (number) \#
%   Dollar        \$     Percent       \%     Ampersand     \&
%   Acute accent  \'     Left paren    \(     Right paren   \)
%   Asterisk      \*     Plus          \+     Comma         \,
%   Minus         \-     Point         \.     Solidus       \/
%   Colon         \:     Semicolon     \;     Less than     \<
%   Equals        \=     Greater than  \>     Question mark \?
%   Commercial at \@     Left bracket  \[     Backslash     \\
%   Right bracket \]     Circumflex    \^     Underscore    \_
%   Grave accent  \`     Left brace    \{     Vertical bar  \|
%   Right brace   \}     Tilde         \~}
%
% \GetFileInfo{bigintcalc.drv}
%
% \title{The \xpackage{bigintcalc} package}
% \date{2012/04/08 v1.3}
% \author{Heiko Oberdiek\\\xemail{heiko.oberdiek at googlemail.com}}
%
% \maketitle
%
% \begin{abstract}
% This package provides expandable arithmetic operations
% with big integers that can exceed \TeX's number limits.
% \end{abstract}
%
% \tableofcontents
%
% \section{Documentation}
%
% \subsection{Introduction}
%
% Package \xpackage{bigintcalc} defines arithmetic operations
% that deal with big integers. Big integers can be given
% either as explicit integer number or as macro code
% that expands to an explicit number. \emph{Big} means that
% there is no limit on the size of the number. Big integers
% may exceed \TeX's range limitation of -2147483647 and 2147483647.
% Only memory issues will limit the usable range.
%
% In opposite to package \xpackage{intcalc}
% unexpandable command tokens are not supported, even if
% they are valid \TeX\ numbers like count registers or
% commands created by \cs{chardef}. Nevertheless they may
% be used, if they are prefixed by \cs{number}.
%
% Also \eTeX's \cs{numexpr} expressions are not supported
% directly in the manner of package \xpackage{intcalc}.
% However they can be given if \cs{the}\cs{numexpr}
% or \cs{number}\cs{numexpr} are used.
%
% The operations have the form of macros that take one or
% two integers as parameter and return the integer result.
% The macro name is a three letter operation name prefixed
% by the package name, e.g. \cs{bigintcalcAdd}|{10}{43}| returns
% |53|.
%
% The macros are fully expandable, exactly two expansion
% steps generate the result. Therefore the operations
% may be used nearly everywhere in \TeX, even inside
% \cs{csname}, file names,  or other
% expandable contexts.
%
% \subsection{Conditions}
%
% \subsubsection{Preconditions}
%
% \begin{itemize}
% \item
%   Arguments can be anything that expands to a number that consists
%   of optional signs and digits.
% \item
%   The arguments and return values must be sound.
%   Zero as divisor or factorials of negative numbers will cause errors.
% \end{itemize}
%
% \subsubsection{Postconditions}
%
% Additional properties of the macros apart from calculating
% a correct result (of course \smiley):
% \begin{itemize}
% \item
%   The macros are fully expandable. Thus they can
%   be used inside \cs{edef}, \cs{csname},
%   for example.
% \item
%   Furthermore exactly two expansion steps calculate the result.
% \item
%   The number consists of one optional minus sign and one or more
%   digits. The first digit is larger than zero for
%   numbers that consists of more than one digit.
%
%   In short, the number format is exactly the same as
%   \cs{number} generates, but without its range limitation.
%   And the tokens (minus sign, digits)
%   have catcode 12 (other).
% \item
%   Call by value is simulated. First the arguments are
%   converted to numbers. Then these numbers are used
%   in the calculations.
%
%   Remember that arguments
%   may contain expensive macros or \eTeX\ expressions.
%   This strategy avoids multiple evaluations of such
%   arguments.
% \end{itemize}
%
% \subsection{Error handling}
%
% Some errors are detected by the macros, example: division by zero.
% In this cases an undefined control sequence is called and causes
% a TeX error message, example: \cs{BigIntCalcError:DivisionByZero}.
% The name of the control sequence contains
% the reason for the error. The \TeX\ error may be ignored.
% Then the operation returns zero as result.
% Because the macros are supposed to work in expandible contexts.
% An traditional error message, however, is not expandable and
% would break these contexts.
%
% \subsection{Operations}
%
% Some definition equations below use the function $\opInt$
% that converts a real number to an integer. The number
% is truncated that means rounding to zero:
% \begin{gather*}
%   \opInt(x) \Def
%   \begin{cases}
%     \lfloor x\rfloor & \text{if $x\geq0$}\\
%     \lceil x\rceil & \text{otherwise}
%   \end{cases}
% \end{gather*}
%
% \subsubsection{\op{Num}}
%
% \begin{declcs}{bigintcalcNum} \M{x}
% \end{declcs}
% Macro \cs{bigintcalcNum} converts its argument to a normalized integer
% number without unnecessary leading zeros or signs.
% The result matches the regular expression:
%\begin{quote}
%\begin{verbatim}
%0|-?[1-9][0-9]*
%\end{verbatim}
%\end{quote}
%
% \subsubsection{\op{Inv}, \op{Abs}, \op{Sgn}}
%
% \begin{declcs}{bigintcalcInv} \M{x}
% \end{declcs}
% Macro \cs{bigintcalcInv} switches the sign.
% \begin{gather*}
%   \opInv(x) \Def -x
% \end{gather*}
%
% \begin{declcs}{bigintcalcAbs} \M{x}
% \end{declcs}
% Macro \cs{bigintcalcAbs} returns the absolute value
% of integer \meta{x}.
% \begin{gather*}
%   \opAbs(x) \Def \vert x\vert
% \end{gather*}
%
% \begin{declcs}{bigintcalcSgn} \M{x}
% \end{declcs}
% Macro \cs{bigintcalcSgn} encodes the sign of \meta{x} as number.
% \begin{gather*}
%   \opSgn(x) \Def
%   \begin{cases}
%     -1& \text{if $x<0$}\\
%      0& \text{if $x=0$}\\
%      1& \text{if $x>0$}
%   \end{cases}
% \end{gather*}
% These return values can easily be distinguished by \cs{ifcase}:
%\begin{quote}
%\begin{verbatim}
%\ifcase\bigintcalcSgn{<x>}
%  $x=0$
%\or
%  $x>0$
%\else
%  $x<0$
%\fi
%\end{verbatim}
%\end{quote}
%
% \subsubsection{\op{Min}, \op{Max}, \op{Cmp}}
%
% \begin{declcs}{bigintcalcMin} \M{x} \M{y}
% \end{declcs}
% Macro \cs{bigintcalcMin} returns the smaller of the two integers.
% \begin{gather*}
%   \opMin(x,y) \Def
%   \begin{cases}
%     x & \text{if $x<y$}\\
%     y & \text{otherwise}
%   \end{cases}
% \end{gather*}
%
% \begin{declcs}{bigintcalcMax} \M{x} \M{y}
% \end{declcs}
% Macro \cs{bigintcalcMax} returns the larger of the two integers.
% \begin{gather*}
%   \opMax(x,y) \Def
%   \begin{cases}
%     x & \text{if $x>y$}\\
%     y & \text{otherwise}
%   \end{cases}
% \end{gather*}
%
% \begin{declcs}{bigintcalcCmp} \M{x} \M{y}
% \end{declcs}
% Macro \cs{bigintcalcCmp} encodes the comparison result as number:
% \begin{gather*}
%   \opCmp(x,y) \Def
%   \begin{cases}
%     -1 & \text{if $x<y$}\\
%      0 & \text{if $x=y$}\\
%      1 & \text{if $x>y$}
%   \end{cases}
% \end{gather*}
% These values can be distinguished by \cs{ifcase}:
%\begin{quote}
%\begin{verbatim}
%\ifcase\bigintcalcCmp{<x>}{<y>}
%  $x=y$
%\or
%  $x>y$
%\else
%  $x<y$
%\fi
%\end{verbatim}
%\end{quote}
%
% \subsubsection{\op{Odd}}
%
% \begin{declcs}{bigintcalcOdd} \M{x}
% \end{declcs}
% \begin{gather*}
%   \opOdd(x) \Def
%   \begin{cases}
%     1 & \text{if $x$ is odd}\\
%     0 & \text{if $x$ is even}
%   \end{cases}
% \end{gather*}
%
% \subsubsection{\op{Inc}, \op{Dec}, \op{Add}, \op{Sub}}
%
% \begin{declcs}{bigintcalcInc} \M{x}
% \end{declcs}
% Macro \cs{bigintcalcInc} increments \meta{x} by one.
% \begin{gather*}
%   \opInc(x) \Def x + 1
% \end{gather*}
%
% \begin{declcs}{bigintcalcDec} \M{x}
% \end{declcs}
% Macro \cs{bigintcalcDec} decrements \meta{x} by one.
% \begin{gather*}
%   \opDec(x) \Def x - 1
% \end{gather*}
%
% \begin{declcs}{bigintcalcAdd} \M{x} \M{y}
% \end{declcs}
% Macro \cs{bigintcalcAdd} adds the two numbers.
% \begin{gather*}
%   \opAdd(x, y) \Def x + y
% \end{gather*}
%
% \begin{declcs}{bigintcalcSub} \M{x} \M{y}
% \end{declcs}
% Macro \cs{bigintcalcSub} calculates the difference.
% \begin{gather*}
%   \opSub(x, y) \Def x - y
% \end{gather*}
%
% \subsubsection{\op{Shl}, \op{Shr}}
%
% \begin{declcs}{bigintcalcShl} \M{x}
% \end{declcs}
% Macro \cs{bigintcalcShl} implements shifting to the left that
% means the number is multiplied by two.
% The sign is preserved.
% \begin{gather*}
%   \opShl(x) \Def x*2
% \end{gather*}
%
% \begin{declcs}{bigintcalcShr} \M{x}
% \end{declcs}
% Macro \cs{bigintcalcShr} implements shifting to the right.
% That is equivalent to an integer division by two.
% The sign is preserved.
% \begin{gather*}
%   \opShr(x) \Def \opInt(x/2)
% \end{gather*}
%
% \subsubsection{\op{Mul}, \op{Sqr}, \op{Fac}, \op{Pow}}
%
% \begin{declcs}{bigintcalcMul} \M{x} \M{y}
% \end{declcs}
% Macro \cs{bigintcalcMul} calculates the product of
% \meta{x} and \meta{y}.
% \begin{gather*}
%   \opMul(x,y) \Def x*y
% \end{gather*}
%
% \begin{declcs}{bigintcalcSqr} \M{x}
% \end{declcs}
% Macro \cs{bigintcalcSqr} returns the square product.
% \begin{gather*}
%   \opSqr(x) \Def x^2
% \end{gather*}
%
% \begin{declcs}{bigintcalcFac} \M{x}
% \end{declcs}
% Macro \cs{bigintcalcFac} returns the factorial of \meta{x}.
% Negative numbers are not permitted.
% \begin{gather*}
%   \opFac(x) \Def x!\qquad\text{for $x\geq0$}
% \end{gather*}
% ($0! = 1$)
%
% \begin{declcs}{bigintcalcPow} M{x} M{y}
% \end{declcs}
% Macro \cs{bigintcalcPow} calculates the value of \meta{x} to the
% power of \meta{y}. The error ``division by zero'' is thrown
% if \meta{x} is zero and \meta{y} is negative.
% permitted:
% \begin{gather*}
%   \opPow(x,y) \Def
%   \opInt(x^y)\qquad\text{for $x\neq0$ or $y\geq0$}
% \end{gather*}
% ($0^0 = 1$)
%
% \subsubsection{\op{Div}, \op{Mul}}
%
% \begin{declcs}{bigintcalcDiv} \M{x} \M{y}
% \end{declcs}
% Macro \cs{bigintcalcDiv} performs an integer division.
% Argument \meta{y} must not be zero.
% \begin{gather*}
%   \opDiv(x,y) \Def \opInt(x/y)\qquad\text{for $y\neq0$}
% \end{gather*}
%
% \begin{declcs}{bigintcalcMod} \M{x} \M{y}
% \end{declcs}
% Macro \cs{bigintcalcMod} gets the remainder of the integer
% division. The sign follows the divisor \meta{y}.
% Argument \meta{y} must not be zero.
% \begin{gather*}
%   \opMod(x,y) \Def x\mathrel{\%}y\qquad\text{for $y\neq0$}
% \end{gather*}
% The result ranges:
% \begin{gather*}
%   -\vert y\vert < \opMod(x,y) \leq 0\qquad\text{for $y<0$}\\
%   0 \leq \opMod(x,y) < y\qquad\text{for $y\geq0$}
% \end{gather*}
%
% \subsection{Interface for programmers}
%
% If the programmer can ensure some more properties about
% the arguments of the operations, then the following
% macros are a little more efficient.
%
% In general numbers must obey the following constraints:
% \begin{itemize}
% \item Plain number: digit tokens only, no command tokens.
% \item Non-negative. Signs are forbidden.
% \item Delimited by exclamation mark. Curly braces
%       around the number are not allowed and will
%       break the code.
% \end{itemize}
%
% \begin{declcs}{BigIntCalcOdd} \meta{number} |!|
% \end{declcs}
% |1|/|0| is returned if \meta{number} is odd/even.
%
% \begin{declcs}{BigIntCalcInc} \meta{number} |!|
% \end{declcs}
% Incrementation.
%
% \begin{declcs}{BigIntCalcDec} \meta{number} |!|
% \end{declcs}
% Decrementation, positive number without zero.
%
% \begin{declcs}{BigIntCalcAdd} \meta{number A} |!| \meta{number B} |!|
% \end{declcs}
% Addition, $A\geq B$.
%
% \begin{declcs}{BigIntCalcSub} \meta{number A} |!| \meta{number B} |!|
% \end{declcs}
% Subtraction, $A\geq B$.
%
% \begin{declcs}{BigIntCalcShl} \meta{number} |!|
% \end{declcs}
% Left shift (multiplication with two).
%
% \begin{declcs}{BigIntCalcShr} \meta{number} |!|
% \end{declcs}
% Right shift (integer division by two).
%
% \begin{declcs}{BigIntCalcMul} \meta{number A} |!| \meta{number B} |!|
% \end{declcs}
% Multiplication, $A\geq B$.
%
% \begin{declcs}{BigIntCalcDiv} \meta{number A} |!| \meta{number B} |!|
% \end{declcs}
% Division operation.
%
% \begin{declcs}{BigIntCalcMod} \meta{number A} |!| \meta{number B} |!|
% \end{declcs}
% Modulo operation.
%
%
% \StopEventually{
% }
%
% \section{Implementation}
%
%    \begin{macrocode}
%<*package>
%    \end{macrocode}
%
% \subsection{Reload check and package identification}
%    Reload check, especially if the package is not used with \LaTeX.
%    \begin{macrocode}
\begingroup\catcode61\catcode48\catcode32=10\relax%
  \catcode13=5 % ^^M
  \endlinechar=13 %
  \catcode35=6 % #
  \catcode39=12 % '
  \catcode44=12 % ,
  \catcode45=12 % -
  \catcode46=12 % .
  \catcode58=12 % :
  \catcode64=11 % @
  \catcode123=1 % {
  \catcode125=2 % }
  \expandafter\let\expandafter\x\csname ver@bigintcalc.sty\endcsname
  \ifx\x\relax % plain-TeX, first loading
  \else
    \def\empty{}%
    \ifx\x\empty % LaTeX, first loading,
      % variable is initialized, but \ProvidesPackage not yet seen
    \else
      \expandafter\ifx\csname PackageInfo\endcsname\relax
        \def\x#1#2{%
          \immediate\write-1{Package #1 Info: #2.}%
        }%
      \else
        \def\x#1#2{\PackageInfo{#1}{#2, stopped}}%
      \fi
      \x{bigintcalc}{The package is already loaded}%
      \aftergroup\endinput
    \fi
  \fi
\endgroup%
%    \end{macrocode}
%    Package identification:
%    \begin{macrocode}
\begingroup\catcode61\catcode48\catcode32=10\relax%
  \catcode13=5 % ^^M
  \endlinechar=13 %
  \catcode35=6 % #
  \catcode39=12 % '
  \catcode40=12 % (
  \catcode41=12 % )
  \catcode44=12 % ,
  \catcode45=12 % -
  \catcode46=12 % .
  \catcode47=12 % /
  \catcode58=12 % :
  \catcode64=11 % @
  \catcode91=12 % [
  \catcode93=12 % ]
  \catcode123=1 % {
  \catcode125=2 % }
  \expandafter\ifx\csname ProvidesPackage\endcsname\relax
    \def\x#1#2#3[#4]{\endgroup
      \immediate\write-1{Package: #3 #4}%
      \xdef#1{#4}%
    }%
  \else
    \def\x#1#2[#3]{\endgroup
      #2[{#3}]%
      \ifx#1\@undefined
        \xdef#1{#3}%
      \fi
      \ifx#1\relax
        \xdef#1{#3}%
      \fi
    }%
  \fi
\expandafter\x\csname ver@bigintcalc.sty\endcsname
\ProvidesPackage{bigintcalc}%
  [2012/04/08 v1.3 Expandable calculations on big integers (HO)]%
%    \end{macrocode}
%
% \subsection{Catcodes}
%
%    \begin{macrocode}
\begingroup\catcode61\catcode48\catcode32=10\relax%
  \catcode13=5 % ^^M
  \endlinechar=13 %
  \catcode123=1 % {
  \catcode125=2 % }
  \catcode64=11 % @
  \def\x{\endgroup
    \expandafter\edef\csname BIC@AtEnd\endcsname{%
      \endlinechar=\the\endlinechar\relax
      \catcode13=\the\catcode13\relax
      \catcode32=\the\catcode32\relax
      \catcode35=\the\catcode35\relax
      \catcode61=\the\catcode61\relax
      \catcode64=\the\catcode64\relax
      \catcode123=\the\catcode123\relax
      \catcode125=\the\catcode125\relax
    }%
  }%
\x\catcode61\catcode48\catcode32=10\relax%
\catcode13=5 % ^^M
\endlinechar=13 %
\catcode35=6 % #
\catcode64=11 % @
\catcode123=1 % {
\catcode125=2 % }
\def\TMP@EnsureCode#1#2{%
  \edef\BIC@AtEnd{%
    \BIC@AtEnd
    \catcode#1=\the\catcode#1\relax
  }%
  \catcode#1=#2\relax
}
\TMP@EnsureCode{33}{12}% !
\TMP@EnsureCode{36}{14}% $ (comment!)
\TMP@EnsureCode{38}{14}% & (comment!)
\TMP@EnsureCode{40}{12}% (
\TMP@EnsureCode{41}{12}% )
\TMP@EnsureCode{42}{12}% *
\TMP@EnsureCode{43}{12}% +
\TMP@EnsureCode{45}{12}% -
\TMP@EnsureCode{46}{12}% .
\TMP@EnsureCode{47}{12}% /
\TMP@EnsureCode{58}{11}% : (letter!)
\TMP@EnsureCode{60}{12}% <
\TMP@EnsureCode{62}{12}% >
\TMP@EnsureCode{63}{14}% ? (comment!)
\TMP@EnsureCode{91}{12}% [
\TMP@EnsureCode{93}{12}% ]
\edef\BIC@AtEnd{\BIC@AtEnd\noexpand\endinput}
\begingroup\expandafter\expandafter\expandafter\endgroup
\expandafter\ifx\csname BIC@TestMode\endcsname\relax
\else
  \catcode63=9 % ? (ignore)
\fi
? \let\BIC@@TestMode\BIC@TestMode
%    \end{macrocode}
%
% \subsection{\eTeX\ detection}
%
%    \begin{macrocode}
\begingroup\expandafter\expandafter\expandafter\endgroup
\expandafter\ifx\csname numexpr\endcsname\relax
  \catcode36=9 % $ (ignore)
\else
  \catcode38=9 % & (ignore)
\fi
%    \end{macrocode}
%
% \subsection{Help macros}
%
%    \begin{macro}{\BIC@Fi}
%    \begin{macrocode}
\let\BIC@Fi\fi
%    \end{macrocode}
%    \end{macro}
%    \begin{macro}{\BIC@AfterFi}
%    \begin{macrocode}
\def\BIC@AfterFi#1#2\BIC@Fi{\fi#1}%
%    \end{macrocode}
%    \end{macro}
%    \begin{macro}{\BIC@AfterFiFi}
%    \begin{macrocode}
\def\BIC@AfterFiFi#1#2\BIC@Fi{\fi\fi#1}%
%    \end{macrocode}
%    \end{macro}
%    \begin{macro}{\BIC@AfterFiFiFi}
%    \begin{macrocode}
\def\BIC@AfterFiFiFi#1#2\BIC@Fi{\fi\fi\fi#1}%
%    \end{macrocode}
%    \end{macro}
%
%    \begin{macro}{\BIC@Space}
%    \begin{macrocode}
\begingroup
  \def\x#1{\endgroup
    \let\BIC@Space= #1%
  }%
\x{ }
%    \end{macrocode}
%    \end{macro}
%
% \subsection{Expand number}
%
%    \begin{macrocode}
\begingroup\expandafter\expandafter\expandafter\endgroup
\expandafter\ifx\csname RequirePackage\endcsname\relax
  \def\TMP@RequirePackage#1[#2]{%
    \begingroup\expandafter\expandafter\expandafter\endgroup
    \expandafter\ifx\csname ver@#1.sty\endcsname\relax
      \input #1.sty\relax
    \fi
  }%
  \TMP@RequirePackage{pdftexcmds}[2007/11/11]%
\else
  \RequirePackage{pdftexcmds}[2007/11/11]%
\fi
%    \end{macrocode}
%
%    \begin{macrocode}
\begingroup\expandafter\expandafter\expandafter\endgroup
\expandafter\ifx\csname pdf@escapehex\endcsname\relax
%    \end{macrocode}
%
%    \begin{macro}{\BIC@Expand}
%    \begin{macrocode}
  \def\BIC@Expand#1{%
    \romannumeral0%
    \BIC@@Expand#1!\@nil{}%
  }%
%    \end{macrocode}
%    \end{macro}
%    \begin{macro}{\BIC@@Expand}
%    \begin{macrocode}
  \def\BIC@@Expand#1#2\@nil#3{%
    \expandafter\ifcat\noexpand#1\relax
      \expandafter\@firstoftwo
    \else
      \expandafter\@secondoftwo
    \fi
    {%
      \expandafter\BIC@@Expand#1#2\@nil{#3}%
    }{%
      \ifx#1!%
        \expandafter\@firstoftwo
      \else
        \expandafter\@secondoftwo
      \fi
      { #3}{%
        \BIC@@Expand#2\@nil{#3#1}%
      }%
    }%
  }%
%    \end{macrocode}
%    \end{macro}
%    \begin{macro}{\@firstoftwo}
%    \begin{macrocode}
  \expandafter\ifx\csname @firstoftwo\endcsname\relax
    \long\def\@firstoftwo#1#2{#1}%
  \fi
%    \end{macrocode}
%    \end{macro}
%    \begin{macro}{\@secondoftwo}
%    \begin{macrocode}
  \expandafter\ifx\csname @secondoftwo\endcsname\relax
    \long\def\@secondoftwo#1#2{#2}%
  \fi
%    \end{macrocode}
%    \end{macro}
%
%    \begin{macrocode}
\else
%    \end{macrocode}
%    \begin{macro}{\BIC@Expand}
%    \begin{macrocode}
  \def\BIC@Expand#1{%
    \romannumeral0\expandafter\expandafter\expandafter\BIC@Space
    \pdf@unescapehex{%
      \expandafter\expandafter\expandafter
      \BIC@StripHexSpace\pdf@escapehex{#1}20\@nil
    }%
  }%
%    \end{macrocode}
%    \end{macro}
%    \begin{macro}{\BIC@StripHexSpace}
%    \begin{macrocode}
  \def\BIC@StripHexSpace#120#2\@nil{%
    #1%
    \ifx\\#2\\%
    \else
      \BIC@AfterFi{%
        \BIC@StripHexSpace#2\@nil
      }%
    \BIC@Fi
  }%
%    \end{macrocode}
%    \end{macro}
%    \begin{macrocode}
\fi
%    \end{macrocode}
%
% \subsection{Normalize expanded number}
%
%    \begin{macro}{\BIC@Normalize}
%    |#1|: result sign\\
%    |#2|: first token of number
%    \begin{macrocode}
\def\BIC@Normalize#1#2{%
  \ifx#2-%
    \ifx\\#1\\%
      \BIC@AfterFiFi{%
        \BIC@Normalize-%
      }%
    \else
      \BIC@AfterFiFi{%
        \BIC@Normalize{}%
      }%
    \fi
  \else
    \ifx#2+%
      \BIC@AfterFiFi{%
        \BIC@Normalize{#1}%
      }%
    \else
      \ifx#20%
        \BIC@AfterFiFiFi{%
          \BIC@NormalizeZero{#1}%
        }%
      \else
        \BIC@AfterFiFiFi{%
          \BIC@NormalizeDigits#1#2%
        }%
      \fi
    \fi
  \BIC@Fi
}
%    \end{macrocode}
%    \end{macro}
%    \begin{macro}{\BIC@NormalizeZero}
%    \begin{macrocode}
\def\BIC@NormalizeZero#1#2{%
  \ifx#2!%
    \BIC@AfterFi{ 0}%
  \else
    \ifx#20%
      \BIC@AfterFiFi{%
        \BIC@NormalizeZero{#1}%
      }%
    \else
      \BIC@AfterFiFi{%
        \BIC@NormalizeDigits#1#2%
      }%
    \fi
  \BIC@Fi
}
%    \end{macrocode}
%    \end{macro}
%    \begin{macro}{\BIC@NormalizeDigits}
%    \begin{macrocode}
\def\BIC@NormalizeDigits#1!{ #1}
%    \end{macrocode}
%    \end{macro}
%
% \subsection{\op{Num}}
%
%    \begin{macro}{\bigintcalcNum}
%    \begin{macrocode}
\def\bigintcalcNum#1{%
  \romannumeral0%
  \expandafter\expandafter\expandafter\BIC@Normalize
  \expandafter\expandafter\expandafter{%
  \expandafter\expandafter\expandafter}%
  \BIC@Expand{#1}!%
}
%    \end{macrocode}
%    \end{macro}
%
% \subsection{\op{Inv}, \op{Abs}, \op{Sgn}}
%
%
%    \begin{macro}{\bigintcalcInv}
%    \begin{macrocode}
\def\bigintcalcInv#1{%
  \romannumeral0\expandafter\expandafter\expandafter\BIC@Space
  \bigintcalcNum{-#1}%
}
%    \end{macrocode}
%    \end{macro}
%
%    \begin{macro}{\bigintcalcAbs}
%    \begin{macrocode}
\def\bigintcalcAbs#1{%
  \romannumeral0%
  \expandafter\expandafter\expandafter\BIC@Abs
  \bigintcalcNum{#1}%
}
%    \end{macrocode}
%    \end{macro}
%    \begin{macro}{\BIC@Abs}
%    \begin{macrocode}
\def\BIC@Abs#1{%
  \ifx#1-%
    \expandafter\BIC@Space
  \else
    \expandafter\BIC@Space
    \expandafter#1%
  \fi
}
%    \end{macrocode}
%    \end{macro}
%
%    \begin{macro}{\bigintcalcSgn}
%    \begin{macrocode}
\def\bigintcalcSgn#1{%
  \number
  \expandafter\expandafter\expandafter\BIC@Sgn
  \bigintcalcNum{#1}! %
}
%    \end{macrocode}
%    \end{macro}
%    \begin{macro}{\BIC@Sgn}
%    \begin{macrocode}
\def\BIC@Sgn#1#2!{%
  \ifx#1-%
    -1%
  \else
    \ifx#10%
      0%
    \else
      1%
    \fi
  \fi
}
%    \end{macrocode}
%    \end{macro}
%
% \subsection{\op{Cmp}, \op{Min}, \op{Max}}
%
%    \begin{macro}{\bigintcalcCmp}
%    \begin{macrocode}
\def\bigintcalcCmp#1#2{%
  \number
  \expandafter\expandafter\expandafter\BIC@Cmp
  \bigintcalcNum{#2}!{#1}%
}
%    \end{macrocode}
%    \end{macro}
%    \begin{macro}{\BIC@Cmp}
%    \begin{macrocode}
\def\BIC@Cmp#1!#2{%
  \expandafter\expandafter\expandafter\BIC@@Cmp
  \bigintcalcNum{#2}!#1!%
}
%    \end{macrocode}
%    \end{macro}
%    \begin{macro}{\BIC@@Cmp}
%    \begin{macrocode}
\def\BIC@@Cmp#1#2!#3#4!{%
  \ifx#1-%
    \ifx#3-%
      \BIC@AfterFiFi{%
        \BIC@@Cmp#4!#2!%
      }%
    \else
      \BIC@AfterFiFi{%
        -1 %
      }%
    \fi
  \else
    \ifx#3-%
      \BIC@AfterFiFi{%
        1 %
      }%
    \else
      \BIC@AfterFiFi{%
        \BIC@CmpLength#1#2!#3#4!#1#2!#3#4!%
      }%
    \fi
  \BIC@Fi
}
%    \end{macrocode}
%    \end{macro}
%    \begin{macro}{\BIC@PosCmp}
%    \begin{macrocode}
\def\BIC@PosCmp#1!#2!{%
  \BIC@CmpLength#1!#2!#1!#2!%
}
%    \end{macrocode}
%    \end{macro}
%    \begin{macro}{\BIC@CmpLength}
%    \begin{macrocode}
\def\BIC@CmpLength#1#2!#3#4!{%
  \ifx\\#2\\%
    \ifx\\#4\\%
      \BIC@AfterFiFi\BIC@CmpDiff
    \else
      \BIC@AfterFiFi{%
        \BIC@CmpResult{-1}%
      }%
    \fi
  \else
    \ifx\\#4\\%
      \BIC@AfterFiFi{%
        \BIC@CmpResult1%
      }%
    \else
      \BIC@AfterFiFi{%
        \BIC@CmpLength#2!#4!%
      }%
    \fi
  \BIC@Fi
}
%    \end{macrocode}
%    \end{macro}
%    \begin{macro}{\BIC@CmpResult}
%    \begin{macrocode}
\def\BIC@CmpResult#1#2!#3!{#1 }
%    \end{macrocode}
%    \end{macro}
%    \begin{macro}{\BIC@CmpDiff}
%    \begin{macrocode}
\def\BIC@CmpDiff#1#2!#3#4!{%
  \ifnum#1<#3 %
    \BIC@AfterFi{%
      -1 %
    }%
  \else
    \ifnum#1>#3 %
      \BIC@AfterFiFi{%
        1 %
      }%
    \else
      \ifx\\#2\\%
        \BIC@AfterFiFiFi{%
          0 %
        }%
      \else
        \BIC@AfterFiFiFi{%
          \BIC@CmpDiff#2!#4!%
        }%
      \fi
    \fi
  \BIC@Fi
}
%    \end{macrocode}
%    \end{macro}
%
%    \begin{macro}{\bigintcalcMin}
%    \begin{macrocode}
\def\bigintcalcMin#1{%
  \romannumeral0%
  \expandafter\expandafter\expandafter\BIC@MinMax
  \bigintcalcNum{#1}!-!%
}
%    \end{macrocode}
%    \end{macro}
%    \begin{macro}{\bigintcalcMax}
%    \begin{macrocode}
\def\bigintcalcMax#1{%
  \romannumeral0%
  \expandafter\expandafter\expandafter\BIC@MinMax
  \bigintcalcNum{#1}!!%
}
%    \end{macrocode}
%    \end{macro}
%    \begin{macro}{\BIC@MinMax}
%    |#1|: $x$\\
%    |#2|: sign for comparison\\
%    |#3|: $y$
%    \begin{macrocode}
\def\BIC@MinMax#1!#2!#3{%
  \expandafter\expandafter\expandafter\BIC@@MinMax
  \bigintcalcNum{#3}!#1!#2!%
}
%    \end{macrocode}
%    \end{macro}
%    \begin{macro}{\BIC@@MinMax}
%    |#1|: $y$\\
%    |#2|: $x$\\
%    |#3|: sign for comparison
%    \begin{macrocode}
\def\BIC@@MinMax#1!#2!#3!{%
  \ifnum\BIC@@Cmp#1!#2!=#31 %
    \BIC@AfterFi{ #1}%
  \else
    \BIC@AfterFi{ #2}%
  \BIC@Fi
}
%    \end{macrocode}
%    \end{macro}
%
% \subsection{\op{Odd}}
%
%    \begin{macro}{\bigintcalcOdd}
%    \begin{macrocode}
\def\bigintcalcOdd#1{%
  \romannumeral0%
  \expandafter\expandafter\expandafter\BIC@Odd
  \bigintcalcAbs{#1}!%
}
%    \end{macrocode}
%    \end{macro}
%    \begin{macro}{\BigIntCalcOdd}
%    \begin{macrocode}
\def\BigIntCalcOdd#1!{%
  \romannumeral0%
  \BIC@Odd#1!%
}
%    \end{macrocode}
%    \end{macro}
%    \begin{macro}{\BIC@Odd}
%    |#1|: $x$
%    \begin{macrocode}
\def\BIC@Odd#1#2{%
  \ifx#2!%
    \ifodd#1 %
      \BIC@AfterFiFi{ 1}%
    \else
      \BIC@AfterFiFi{ 0}%
    \fi
  \else
    \expandafter\BIC@Odd\expandafter#2%
  \BIC@Fi
}
%    \end{macrocode}
%    \end{macro}
%
% \subsection{\op{Inc}, \op{Dec}}
%
%    \begin{macro}{\bigintcalcInc}
%    \begin{macrocode}
\def\bigintcalcInc#1{%
  \romannumeral0%
  \expandafter\expandafter\expandafter\BIC@IncSwitch
  \bigintcalcNum{#1}!%
}
%    \end{macrocode}
%    \end{macro}
%    \begin{macro}{\BIC@IncSwitch}
%    \begin{macrocode}
\def\BIC@IncSwitch#1#2!{%
  \ifcase\BIC@@Cmp#1#2!-1!%
    \BIC@AfterFi{ 0}%
  \or
    \BIC@AfterFi{%
      \BIC@Inc#1#2!{}%
    }%
  \else
    \BIC@AfterFi{%
      \expandafter-\romannumeral0%
      \BIC@Dec#2!{}%
    }%
  \BIC@Fi
}
%    \end{macrocode}
%    \end{macro}
%
%    \begin{macro}{\bigintcalcDec}
%    \begin{macrocode}
\def\bigintcalcDec#1{%
  \romannumeral0%
  \expandafter\expandafter\expandafter\BIC@DecSwitch
  \bigintcalcNum{#1}!%
}
%    \end{macrocode}
%    \end{macro}
%    \begin{macro}{\BIC@DecSwitch}
%    \begin{macrocode}
\def\BIC@DecSwitch#1#2!{%
  \ifcase\BIC@Sgn#1#2! %
    \BIC@AfterFi{ -1}%
  \or
    \BIC@AfterFi{%
      \BIC@Dec#1#2!{}%
    }%
  \else
    \BIC@AfterFi{%
      \expandafter-\romannumeral0%
      \BIC@Inc#2!{}%
    }%
  \BIC@Fi
}
%    \end{macrocode}
%    \end{macro}
%
%    \begin{macro}{\BigIntCalcInc}
%    \begin{macrocode}
\def\BigIntCalcInc#1!{%
  \romannumeral0\BIC@Inc#1!{}%
}
%    \end{macrocode}
%    \end{macro}
%    \begin{macro}{\BigIntCalcDec}
%    \begin{macrocode}
\def\BigIntCalcDec#1!{%
  \romannumeral0\BIC@Dec#1!{}%
}
%    \end{macrocode}
%    \end{macro}
%
%    \begin{macro}{\BIC@Inc}
%    \begin{macrocode}
\def\BIC@Inc#1#2!#3{%
  \ifx\\#2\\%
    \BIC@AfterFi{%
      \BIC@@Inc1#1#3!{}%
    }%
  \else
    \BIC@AfterFi{%
      \BIC@Inc#2!{#1#3}%
    }%
  \BIC@Fi
}
%    \end{macrocode}
%    \end{macro}
%    \begin{macro}{\BIC@@Inc}
%    \begin{macrocode}
\def\BIC@@Inc#1#2#3!#4{%
  \ifcase#1 %
    \ifx\\#3\\%
      \BIC@AfterFiFi{ #2#4}%
    \else
      \BIC@AfterFiFi{%
        \BIC@@Inc0#3!{#2#4}%
      }%
    \fi
  \else
    \ifnum#2<9 %
      \BIC@AfterFiFi{%
&       \expandafter\BIC@@@Inc\the\numexpr#2+1\relax
$       \expandafter\expandafter\expandafter\BIC@@@Inc
$       \ifcase#2 \expandafter1%
$       \or\expandafter2%
$       \or\expandafter3%
$       \or\expandafter4%
$       \or\expandafter5%
$       \or\expandafter6%
$       \or\expandafter7%
$       \or\expandafter8%
$       \or\expandafter9%
$?      \else\BigIntCalcError:ThisCannotHappen%
$       \fi
        0#3!{#4}%
      }%
    \else
      \BIC@AfterFiFi{%
        \BIC@@@Inc01#3!{#4}%
      }%
    \fi
  \BIC@Fi
}
%    \end{macrocode}
%    \end{macro}
%    \begin{macro}{\BIC@@@Inc}
%    \begin{macrocode}
\def\BIC@@@Inc#1#2#3!#4{%
  \ifx\\#3\\%
    \ifnum#2=1 %
      \BIC@AfterFiFi{ 1#1#4}%
    \else
      \BIC@AfterFiFi{ #1#4}%
    \fi
  \else
    \BIC@AfterFi{%
      \BIC@@Inc#2#3!{#1#4}%
    }%
  \BIC@Fi
}
%    \end{macrocode}
%    \end{macro}
%
%    \begin{macro}{\BIC@Dec}
%    \begin{macrocode}
\def\BIC@Dec#1#2!#3{%
  \ifx\\#2\\%
    \BIC@AfterFi{%
      \BIC@@Dec1#1#3!{}%
    }%
  \else
    \BIC@AfterFi{%
      \BIC@Dec#2!{#1#3}%
    }%
  \BIC@Fi
}
%    \end{macrocode}
%    \end{macro}
%    \begin{macro}{\BIC@@Dec}
%    \begin{macrocode}
\def\BIC@@Dec#1#2#3!#4{%
  \ifcase#1 %
    \ifx\\#3\\%
      \BIC@AfterFiFi{ #2#4}%
    \else
      \BIC@AfterFiFi{%
        \BIC@@Dec0#3!{#2#4}%
      }%
    \fi
  \else
    \ifnum#2>0 %
      \BIC@AfterFiFi{%
&       \expandafter\BIC@@@Dec\the\numexpr#2-1\relax
$       \expandafter\expandafter\expandafter\BIC@@@Dec
$       \ifcase#2
$?        \BigIntCalcError:ThisCannotHappen%
$       \or\expandafter0%
$       \or\expandafter1%
$       \or\expandafter2%
$       \or\expandafter3%
$       \or\expandafter4%
$       \or\expandafter5%
$       \or\expandafter6%
$       \or\expandafter7%
$       \or\expandafter8%
$?      \else\BigIntCalcError:ThisCannotHappen%
$       \fi
        0#3!{#4}%
      }%
    \else
      \BIC@AfterFiFi{%
        \BIC@@@Dec91#3!{#4}%
      }%
    \fi
  \BIC@Fi
}
%    \end{macrocode}
%    \end{macro}
%    \begin{macro}{\BIC@@@Dec}
%    \begin{macrocode}
\def\BIC@@@Dec#1#2#3!#4{%
  \ifx\\#3\\%
    \ifcase#1 %
      \ifx\\#4\\%
        \BIC@AfterFiFiFi{ 0}%
      \else
        \BIC@AfterFiFiFi{ #4}%
      \fi
    \else
      \BIC@AfterFiFi{ #1#4}%
    \fi
  \else
    \BIC@AfterFi{%
      \BIC@@Dec#2#3!{#1#4}%
    }%
  \BIC@Fi
}
%    \end{macrocode}
%    \end{macro}
%
% \subsection{\op{Add}, \op{Sub}}
%
%    \begin{macro}{\bigintcalcAdd}
%    \begin{macrocode}
\def\bigintcalcAdd#1{%
  \romannumeral0%
  \expandafter\expandafter\expandafter\BIC@Add
  \bigintcalcNum{#1}!%
}
%    \end{macrocode}
%    \end{macro}
%    \begin{macro}{\BIC@Add}
%    \begin{macrocode}
\def\BIC@Add#1!#2{%
  \expandafter\expandafter\expandafter
  \BIC@AddSwitch\bigintcalcNum{#2}!#1!%
}
%    \end{macrocode}
%    \end{macro}
%    \begin{macro}{\bigintcalcSub}
%    \begin{macrocode}
\def\bigintcalcSub#1#2{%
  \romannumeral0%
  \expandafter\expandafter\expandafter\BIC@Add
  \bigintcalcNum{-#2}!{#1}%
}
%    \end{macrocode}
%    \end{macro}
%
%    \begin{macro}{\BIC@AddSwitch}
%    Decision table for \cs{BIC@AddSwitch}.
%    \begin{quote}
%    \begin{tabular}[t]{@{}|l|l|l|l|l|@{}}
%      \hline
%      $x<0$ & $y<0$ & $-x>-y$ & $-$ & $\opAdd(-x,-y)$\\
%      \cline{3-3}\cline{5-5}
%            &       & else    &     & $\opAdd(-y,-x)$\\
%      \cline{2-5}
%            & else  & $-x> y$ & $-$ & $\opSub(-x, y)$\\
%      \cline{3-5}
%            &       & $-x= y$ &     & $0$\\
%      \cline{3-5}
%            &       & else    & $+$ & $\opSub( y,-x)$\\
%      \hline
%      else  & $y<0$ & $ x>-y$ & $+$ & $\opSub( x,-y)$\\
%      \cline{3-5}
%            &       & $ x=-y$ &     & $0$\\
%      \cline{3-5}
%            &       & else    & $-$ & $\opSub(-y, x)$\\
%      \cline{2-5}
%            & else  & $ x> y$ & $+$ & $\opAdd( x, y)$\\
%      \cline{3-3}\cline{5-5}
%            &       & else    &     & $\opAdd( y, x)$\\
%      \hline
%    \end{tabular}
%    \end{quote}
%    \begin{macrocode}
\def\BIC@AddSwitch#1#2!#3#4!{%
  \ifx#1-% x < 0
    \ifx#3-% y < 0
      \expandafter-\romannumeral0%
      \ifnum\BIC@PosCmp#2!#4!=1 % -x > -y
        \BIC@AfterFiFiFi{%
          \BIC@AddXY#2!#4!!!%
        }%
      \else % -x <= -y
        \BIC@AfterFiFiFi{%
          \BIC@AddXY#4!#2!!!%
        }%
      \fi
    \else % y >= 0
      \ifcase\BIC@PosCmp#2!#3#4!% -x = y
        \BIC@AfterFiFiFi{ 0}%
      \or % -x > y
        \expandafter-\romannumeral0%
        \BIC@AfterFiFiFi{%
          \BIC@SubXY#2!#3#4!!!%
        }%
      \else % -x <= y
        \BIC@AfterFiFiFi{%
          \BIC@SubXY#3#4!#2!!!%
        }%
      \fi
    \fi
  \else % x >= 0
    \ifx#3-% y < 0
      \ifcase\BIC@PosCmp#1#2!#4!% x = -y
        \BIC@AfterFiFiFi{ 0}%
      \or % x > -y
        \BIC@AfterFiFiFi{%
          \BIC@SubXY#1#2!#4!!!%
        }%
      \else % x <= -y
        \expandafter-\romannumeral0%
        \BIC@AfterFiFiFi{%
          \BIC@SubXY#4!#1#2!!!%
        }%
      \fi
    \else % y >= 0
      \ifnum\BIC@PosCmp#1#2!#3#4!=1 % x > y
        \BIC@AfterFiFiFi{%
          \BIC@AddXY#1#2!#3#4!!!%
        }%
      \else % x <= y
        \BIC@AfterFiFiFi{%
          \BIC@AddXY#3#4!#1#2!!!%
        }%
      \fi
    \fi
  \BIC@Fi
}
%    \end{macrocode}
%    \end{macro}
%
%    \begin{macro}{\BigIntCalcAdd}
%    \begin{macrocode}
\def\BigIntCalcAdd#1!#2!{%
  \romannumeral0\BIC@AddXY#1!#2!!!%
}
%    \end{macrocode}
%    \end{macro}
%    \begin{macro}{\BigIntCalcSub}
%    \begin{macrocode}
\def\BigIntCalcSub#1!#2!{%
  \romannumeral0\BIC@SubXY#1!#2!!!%
}
%    \end{macrocode}
%    \end{macro}
%
%    \begin{macro}{\BIC@AddXY}
%    \begin{macrocode}
\def\BIC@AddXY#1#2!#3#4!#5!#6!{%
  \ifx\\#2\\%
    \ifx\\#3\\%
      \BIC@AfterFiFi{%
        \BIC@DoAdd0!#1#5!#60!%
      }%
    \else
      \BIC@AfterFiFi{%
        \BIC@DoAdd0!#1#5!#3#6!%
      }%
    \fi
  \else
    \ifx\\#4\\%
      \ifx\\#3\\%
        \BIC@AfterFiFiFi{%
          \BIC@AddXY#2!{}!#1#5!#60!%
        }%
      \else
        \BIC@AfterFiFiFi{%
          \BIC@AddXY#2!{}!#1#5!#3#6!%
        }%
      \fi
    \else
      \BIC@AfterFiFi{%
        \BIC@AddXY#2!#4!#1#5!#3#6!%
      }%
    \fi
  \BIC@Fi
}
%    \end{macrocode}
%    \end{macro}
%    \begin{macro}{\BIC@DoAdd}
%    |#1|: carry\\
%    |#2|: reverted result\\
%    |#3#4|: reverted $x$\\
%    |#5#6|: reverted $y$
%    \begin{macrocode}
\def\BIC@DoAdd#1#2!#3#4!#5#6!{%
  \ifx\\#4\\%
    \BIC@AfterFi{%
&     \expandafter\BIC@Space
&     \the\numexpr#1+#3+#5\relax#2%
$     \expandafter\expandafter\expandafter\BIC@AddResult
$     \BIC@AddDigit#1#3#5#2%
    }%
  \else
    \BIC@AfterFi{%
      \expandafter\expandafter\expandafter\BIC@DoAdd
      \BIC@AddDigit#1#3#5#2!#4!#6!%
    }%
  \BIC@Fi
}
%    \end{macrocode}
%    \end{macro}
%    \begin{macro}{\BIC@AddResult}
%    \begin{macrocode}
$ \def\BIC@AddResult#1{%
$   \ifx#10%
$     \expandafter\BIC@Space
$   \else
$     \expandafter\BIC@Space\expandafter#1%
$   \fi
$ }%
%    \end{macrocode}
%    \end{macro}
%    \begin{macro}{\BIC@AddDigit}
%    |#1|: carry\\
%    |#2|: digit of $x$\\
%    |#3|: digit of $y$
%    \begin{macrocode}
\def\BIC@AddDigit#1#2#3{%
  \romannumeral0%
& \expandafter\BIC@@AddDigit\the\numexpr#1+#2+#3!%
$ \expandafter\BIC@@AddDigit\number%
$ \csname
$   BIC@AddCarry%
$   \ifcase#1 %
$     #2%
$   \else
$     \ifcase#2 1\or2\or3\or4\or5\or6\or7\or8\or9\or10\fi
$   \fi
$ \endcsname#3!%
}
%    \end{macrocode}
%    \end{macro}
%    \begin{macro}{\BIC@@AddDigit}
%    \begin{macrocode}
\def\BIC@@AddDigit#1!{%
  \ifnum#1<10 %
    \BIC@AfterFi{ 0#1}%
  \else
    \BIC@AfterFi{ #1}%
  \BIC@Fi
}
%    \end{macrocode}
%    \end{macro}
%    \begin{macro}{\BIC@AddCarry0}
%    \begin{macrocode}
$ \expandafter\def\csname BIC@AddCarry0\endcsname#1{#1}%
%    \end{macrocode}
%    \end{macro}
%    \begin{macro}{\BIC@AddCarry10}
%    \begin{macrocode}
$ \expandafter\def\csname BIC@AddCarry10\endcsname#1{1#1}%
%    \end{macrocode}
%    \end{macro}
%    \begin{macro}{\BIC@AddCarry[1-9]}
%    \begin{macrocode}
$ \def\BIC@Temp#1#2{%
$   \expandafter\def\csname BIC@AddCarry#1\endcsname##1{%
$     \ifcase##1 #1\or
$     #2%
$?    \else\BigIntCalcError:ThisCannotHappen%
$     \fi
$   }%
$ }%
$ \BIC@Temp 0{1\or2\or3\or4\or5\or6\or7\or8\or9}%
$ \BIC@Temp 1{2\or3\or4\or5\or6\or7\or8\or9\or10}%
$ \BIC@Temp 2{3\or4\or5\or6\or7\or8\or9\or10\or11}%
$ \BIC@Temp 3{4\or5\or6\or7\or8\or9\or10\or11\or12}%
$ \BIC@Temp 4{5\or6\or7\or8\or9\or10\or11\or12\or13}%
$ \BIC@Temp 5{6\or7\or8\or9\or10\or11\or12\or13\or14}%
$ \BIC@Temp 6{7\or8\or9\or10\or11\or12\or13\or14\or15}%
$ \BIC@Temp 7{8\or9\or10\or11\or12\or13\or14\or15\or16}%
$ \BIC@Temp 8{9\or10\or11\or12\or13\or14\or15\or16\or17}%
$ \BIC@Temp 9{10\or11\or12\or13\or14\or15\or16\or17\or18}%
%    \end{macrocode}
%    \end{macro}
%
%    \begin{macro}{\BIC@SubXY}
%    Preconditions:
%    \begin{itemize}
%    \item $x > y$, $x \geq 0$, and $y >= 0$
%    \item digits($x$) = digits($y$)
%    \end{itemize}
%    \begin{macrocode}
\def\BIC@SubXY#1#2!#3#4!#5!#6!{%
  \ifx\\#2\\%
    \ifx\\#3\\%
      \BIC@AfterFiFi{%
        \BIC@DoSub0!#1#5!#60!%
      }%
    \else
      \BIC@AfterFiFi{%
        \BIC@DoSub0!#1#5!#3#6!%
      }%
    \fi
  \else
    \ifx\\#4\\%
      \ifx\\#3\\%
        \BIC@AfterFiFiFi{%
          \BIC@SubXY#2!{}!#1#5!#60!%
        }%
      \else
        \BIC@AfterFiFiFi{%
          \BIC@SubXY#2!{}!#1#5!#3#6!%
        }%
      \fi
    \else
      \BIC@AfterFiFi{%
        \BIC@SubXY#2!#4!#1#5!#3#6!%
      }%
    \fi
  \BIC@Fi
}
%    \end{macrocode}
%    \end{macro}
%    \begin{macro}{\BIC@DoSub}
%    |#1|: carry\\
%    |#2|: reverted result\\
%    |#3#4|: reverted x\\
%    |#5#6|: reverted y
%    \begin{macrocode}
\def\BIC@DoSub#1#2!#3#4!#5#6!{%
  \ifx\\#4\\%
    \BIC@AfterFi{%
      \expandafter\expandafter\expandafter\BIC@SubResult
      \BIC@SubDigit#1#3#5#2%
    }%
  \else
    \BIC@AfterFi{%
      \expandafter\expandafter\expandafter\BIC@DoSub
      \BIC@SubDigit#1#3#5#2!#4!#6!%
    }%
  \BIC@Fi
}
%    \end{macrocode}
%    \end{macro}
%    \begin{macro}{\BIC@SubResult}
%    \begin{macrocode}
\def\BIC@SubResult#1{%
  \ifx#10%
    \expandafter\BIC@SubResult
  \else
    \expandafter\BIC@Space\expandafter#1%
  \fi
}
%    \end{macrocode}
%    \end{macro}
%    \begin{macro}{\BIC@SubDigit}
%    |#1|: carry\\
%    |#2|: digit of $x$\\
%    |#3|: digit of $y$
%    \begin{macrocode}
\def\BIC@SubDigit#1#2#3{%
  \romannumeral0%
& \expandafter\BIC@@SubDigit\the\numexpr#2-#3-#1!%
$ \expandafter\BIC@@AddDigit\number
$   \csname
$     BIC@SubCarry%
$     \ifcase#1 %
$       #3%
$     \else
$       \ifcase#3 1\or2\or3\or4\or5\or6\or7\or8\or9\or10\fi
$     \fi
$   \endcsname#2!%
}
%    \end{macrocode}
%    \end{macro}
%    \begin{macro}{\BIC@@SubDigit}
%    \begin{macrocode}
& \def\BIC@@SubDigit#1!{%
&   \ifnum#1<0 %
&     \BIC@AfterFi{%
&       \expandafter\BIC@Space
&       \expandafter1\the\numexpr#1+10\relax
&     }%
&   \else
&     \BIC@AfterFi{ 0#1}%
&   \BIC@Fi
& }%
%    \end{macrocode}
%    \end{macro}
%    \begin{macro}{\BIC@SubCarry0}
%    \begin{macrocode}
$ \expandafter\def\csname BIC@SubCarry0\endcsname#1{#1}%
%    \end{macrocode}
%    \end{macro}
%    \begin{macro}{\BIC@SubCarry10}%
%    \begin{macrocode}
$ \expandafter\def\csname BIC@SubCarry10\endcsname#1{1#1}%
%    \end{macrocode}
%    \end{macro}
%    \begin{macro}{\BIC@SubCarry[1-9]}
%    \begin{macrocode}
$ \def\BIC@Temp#1#2{%
$   \expandafter\def\csname BIC@SubCarry#1\endcsname##1{%
$     \ifcase##1 #2%
$?    \else\BigIntCalcError:ThisCannotHappen%
$     \fi
$   }%
$ }%
$ \BIC@Temp 1{19\or0\or1\or2\or3\or4\or5\or6\or7\or8}%
$ \BIC@Temp 2{18\or19\or0\or1\or2\or3\or4\or5\or6\or7}%
$ \BIC@Temp 3{17\or18\or19\or0\or1\or2\or3\or4\or5\or6}%
$ \BIC@Temp 4{16\or17\or18\or19\or0\or1\or2\or3\or4\or5}%
$ \BIC@Temp 5{15\or16\or17\or18\or19\or0\or1\or2\or3\or4}%
$ \BIC@Temp 6{14\or15\or16\or17\or18\or19\or0\or1\or2\or3}%
$ \BIC@Temp 7{13\or14\or15\or16\or17\or18\or19\or0\or1\or2}%
$ \BIC@Temp 8{12\or13\or14\or15\or16\or17\or18\or19\or0\or1}%
$ \BIC@Temp 9{11\or12\or13\or14\or15\or16\or17\or18\or19\or0}%
%    \end{macrocode}
%    \end{macro}
%
% \subsection{\op{Shl}, \op{Shr}}
%
%    \begin{macro}{\bigintcalcShl}
%    \begin{macrocode}
\def\bigintcalcShl#1{%
  \romannumeral0%
  \expandafter\expandafter\expandafter\BIC@Shl
  \bigintcalcNum{#1}!%
}
%    \end{macrocode}
%    \end{macro}
%    \begin{macro}{\BIC@Shl}
%    \begin{macrocode}
\def\BIC@Shl#1#2!{%
  \ifx#1-%
    \BIC@AfterFi{%
      \expandafter-\romannumeral0%
&     \BIC@@Shl#2!!%
$     \BIC@AddXY#2!#2!!!%
    }%
  \else
    \BIC@AfterFi{%
&     \BIC@@Shl#1#2!!%
$     \BIC@AddXY#1#2!#1#2!!!%
    }%
  \BIC@Fi
}
%    \end{macrocode}
%    \end{macro}
%    \begin{macro}{\BigIntCalcShl}
%    \begin{macrocode}
\def\BigIntCalcShl#1!{%
  \romannumeral0%
& \BIC@@Shl#1!!%
$ \BIC@AddXY#1!#1!!!%
}
%    \end{macrocode}
%    \end{macro}
%    \begin{macro}{\BIC@@Shl}
%    \begin{macrocode}
& \def\BIC@@Shl#1#2!{%
&   \ifx\\#2\\%
&     \BIC@AfterFi{%
&       \BIC@@@Shl0!#1%
&     }%
&   \else
&     \BIC@AfterFi{%
&       \BIC@@Shl#2!#1%
&     }%
&   \BIC@Fi
& }%
%    \end{macrocode}
%    \end{macro}
%    \begin{macro}{\BIC@@@Shl}
%    |#1|: carry\\
%    |#2|: result\\
%    |#3#4|: reverted number
%    \begin{macrocode}
& \def\BIC@@@Shl#1#2!#3#4!{%
&   \ifx\\#4\\%
&     \BIC@AfterFi{%
&       \expandafter\BIC@Space
&       \the\numexpr#3*2+#1\relax#2%
&     }%
&   \else
&     \BIC@AfterFi{%
&       \expandafter\BIC@@@@Shl\the\numexpr#3*2+#1!#2!#4!%
&     }%
&   \BIC@Fi
& }%
%    \end{macrocode}
%    \end{macro}
%    \begin{macro}{\BIC@@@@Shl}
%    \begin{macrocode}
& \def\BIC@@@@Shl#1!{%
&   \ifnum#1<10 %
&     \BIC@AfterFi{%
&       \BIC@@@Shl0#1%
&     }%
&   \else
&     \BIC@AfterFi{%
&       \BIC@@@Shl#1%
&     }%
&   \BIC@Fi
& }%
%    \end{macrocode}
%    \end{macro}
%
%    \begin{macro}{\bigintcalcShr}
%    \begin{macrocode}
\def\bigintcalcShr#1{%
  \romannumeral0%
  \expandafter\expandafter\expandafter\BIC@Shr
  \bigintcalcNum{#1}!%
}
%    \end{macrocode}
%    \end{macro}
%    \begin{macro}{\BIC@Shr}
%    \begin{macrocode}
\def\BIC@Shr#1#2!{%
  \ifx#1-%
    \expandafter-\romannumeral0%
    \BIC@AfterFi{%
      \BIC@@Shr#2!%
    }%
  \else
    \BIC@AfterFi{%
      \BIC@@Shr#1#2!%
    }%
  \BIC@Fi
}
%    \end{macrocode}
%    \end{macro}
%    \begin{macro}{\BigIntCalcShr}
%    \begin{macrocode}
\def\BigIntCalcShr#1!{%
  \romannumeral0%
  \BIC@@Shr#1!%
}
%    \end{macrocode}
%    \end{macro}
%    \begin{macro}{\BIC@@Shr}
%    \begin{macrocode}
\def\BIC@@Shr#1#2!{%
  \ifcase#1 %
    \BIC@AfterFi{ 0}%
  \or
    \ifx\\#2\\%
      \BIC@AfterFiFi{ 0}%
    \else
      \BIC@AfterFiFi{%
        \BIC@@@Shr#1#2!!%
      }%
    \fi
  \else
    \BIC@AfterFi{%
      \BIC@@@Shr0#1#2!!%
    }%
  \BIC@Fi
}
%    \end{macrocode}
%    \end{macro}
%    \begin{macro}{\BIC@@@Shr}
%    |#1|: carry\\
%    |#2#3|: number\\
%    |#4|: result
%    \begin{macrocode}
\def\BIC@@@Shr#1#2#3!#4!{%
  \ifx\\#3\\%
    \ifodd#1#2 %
      \BIC@AfterFiFi{%
&       \expandafter\BIC@ShrResult\the\numexpr(#1#2-1)/2\relax
$       \expandafter\expandafter\expandafter\BIC@ShrResult
$       \csname BIC@ShrDigit#1#2\endcsname
        #4!%
      }%
    \else
      \BIC@AfterFiFi{%
&       \expandafter\BIC@ShrResult\the\numexpr#1#2/2\relax
$       \expandafter\expandafter\expandafter\BIC@ShrResult
$       \csname BIC@ShrDigit#1#2\endcsname
        #4!%
      }%
    \fi
  \else
    \ifodd#1#2 %
      \BIC@AfterFiFi{%
&       \expandafter\BIC@@@@Shr\the\numexpr(#1#2-1)/2\relax1%
$       \expandafter\expandafter\expandafter\BIC@@@@Shr
$       \csname BIC@ShrDigit#1#2\endcsname
        #3!#4!%
      }%
    \else
      \BIC@AfterFiFi{%
&       \expandafter\BIC@@@@Shr\the\numexpr#1#2/2\relax0%
$       \expandafter\expandafter\expandafter\BIC@@@@Shr
$       \csname BIC@ShrDigit#1#2\endcsname
        #3!#4!%
      }%
    \fi
  \BIC@Fi
}
%    \end{macrocode}
%    \end{macro}
%    \begin{macro}{\BIC@ShrResult}
%    \begin{macrocode}
& \def\BIC@ShrResult#1#2!{ #2#1}%
$ \def\BIC@ShrResult#1#2#3!{ #3#1}%
%    \end{macrocode}
%    \end{macro}
%    \begin{macro}{\BIC@@@@Shr}
%    |#1|: new digit\\
%    |#2|: carry\\
%    |#3|: remaining number\\
%    |#4|: result
%    \begin{macrocode}
\def\BIC@@@@Shr#1#2#3!#4!{%
  \BIC@@@Shr#2#3!#4#1!%
}
%    \end{macrocode}
%    \end{macro}
%    \begin{macro}{\BIC@ShrDigit[00-19]}
%    \begin{macrocode}
$ \def\BIC@Temp#1#2#3#4{%
$   \expandafter\def\csname BIC@ShrDigit#1#2\endcsname{#3#4}%
$ }%
$ \BIC@Temp 0000%
$ \BIC@Temp 0101%
$ \BIC@Temp 0210%
$ \BIC@Temp 0311%
$ \BIC@Temp 0420%
$ \BIC@Temp 0521%
$ \BIC@Temp 0630%
$ \BIC@Temp 0731%
$ \BIC@Temp 0840%
$ \BIC@Temp 0941%
$ \BIC@Temp 1050%
$ \BIC@Temp 1151%
$ \BIC@Temp 1260%
$ \BIC@Temp 1361%
$ \BIC@Temp 1470%
$ \BIC@Temp 1571%
$ \BIC@Temp 1680%
$ \BIC@Temp 1781%
$ \BIC@Temp 1890%
$ \BIC@Temp 1991%
%    \end{macrocode}
%    \end{macro}
%
% \subsection{\cs{BIC@Tim}}
%
%    \begin{macro}{\BIC@Tim}
%    Macro \cs{BIC@Tim} implements ``Number \emph{tim}es digit''.\\
%    |#1|: plain number without sign\\
%    |#2|: digit
\def\BIC@Tim#1!#2{%
  \romannumeral0%
  \ifcase#2 % 0
    \BIC@AfterFi{ 0}%
  \or % 1
    \BIC@AfterFi{ #1}%
  \or % 2
    \BIC@AfterFi{%
      \BIC@Shl#1!%
    }%
  \else % 3-9
    \BIC@AfterFi{%
      \BIC@@Tim#1!!#2%
    }%
  \BIC@Fi
}
%    \begin{macrocode}
%    \end{macrocode}
%    \end{macro}
%    \begin{macro}{\BIC@@Tim}
%    |#1#2|: number\\
%    |#3|: reverted number
%    \begin{macrocode}
\def\BIC@@Tim#1#2!{%
  \ifx\\#2\\%
    \BIC@AfterFi{%
      \BIC@ProcessTim0!#1%
    }%
  \else
    \BIC@AfterFi{%
      \BIC@@Tim#2!#1%
    }%
  \BIC@Fi
}
%    \end{macrocode}
%    \end{macro}
%    \begin{macro}{\BIC@ProcessTim}
%    |#1|: carry\\
%    |#2|: result\\
%    |#3#4|: reverted number\\
%    |#5|: digit
%    \begin{macrocode}
\def\BIC@ProcessTim#1#2!#3#4!#5{%
  \ifx\\#4\\%
    \BIC@AfterFi{%
      \expandafter\BIC@Space
&     \the\numexpr#3*#5+#1\relax
$     \romannumeral0\BIC@TimDigit#3#5#1%
      #2%
    }%
  \else
    \BIC@AfterFi{%
      \expandafter\BIC@@ProcessTim
&     \the\numexpr#3*#5+#1%
$     \romannumeral0\BIC@TimDigit#3#5#1%
      !#2!#4!#5%
    }%
  \BIC@Fi
}
%    \end{macrocode}
%    \end{macro}
%    \begin{macro}{\BIC@@ProcessTim}
%    |#1#2|: carry?, new digit\\
%    |#3|: new number\\
%    |#4|: old number\\
%    |#5|: digit
%    \begin{macrocode}
\def\BIC@@ProcessTim#1#2!{%
  \ifx\\#2\\%
    \BIC@AfterFi{%
      \BIC@ProcessTim0#1%
    }%
  \else
    \BIC@AfterFi{%
      \BIC@ProcessTim#1#2%
    }%
  \BIC@Fi
}
%    \end{macrocode}
%    \end{macro}
%    \begin{macro}{\BIC@TimDigit}
%    |#1|: digit 0--9\\
%    |#2|: digit 3--9\\
%    |#3|: carry 0--9
%    \begin{macrocode}
$ \def\BIC@TimDigit#1#2#3{%
$   \ifcase#1 % 0
$     \BIC@AfterFi{ #3}%
$   \or % 1
$     \BIC@AfterFi{%
$       \expandafter\BIC@Space
$       \number\csname BIC@AddCarry#2\endcsname#3 %
$     }%
$   \else
$     \ifcase#3 %
$       \BIC@AfterFiFi{%
$         \expandafter\BIC@Space
$         \number\csname BIC@MulDigit#2\endcsname#1 %
$       }%
$     \else
$       \BIC@AfterFiFi{%
$         \expandafter\BIC@Space
$         \romannumeral0%
$         \expandafter\BIC@AddXY
$         \number\csname BIC@MulDigit#2\endcsname#1!%
$         #3!!!%
$       }%
$     \fi
$   \BIC@Fi
$ }%
%    \end{macrocode}
%    \end{macro}
%    \begin{macro}{\BIC@MulDigit[3-9]}
%    \begin{macrocode}
$ \def\BIC@Temp#1#2{%
$   \expandafter\def\csname BIC@MulDigit#1\endcsname##1{%
$     \ifcase##1 0%
$     \or ##1%
$     \or #2%
$?    \else\BigIntCalcError:ThisCannotHappen%
$     \fi
$   }%
$ }%
$ \BIC@Temp 3{6\or9\or12\or15\or18\or21\or24\or27}%
$ \BIC@Temp 4{8\or12\or16\or20\or24\or28\or32\or36}%
$ \BIC@Temp 5{10\or15\or20\or25\or30\or35\or40\or45}%
$ \BIC@Temp 6{12\or18\or24\or30\or36\or42\or48\or54}%
$ \BIC@Temp 7{14\or21\or28\or35\or42\or49\or56\or63}%
$ \BIC@Temp 8{16\or24\or32\or40\or48\or56\or64\or72}%
$ \BIC@Temp 9{18\or27\or36\or45\or54\or63\or72\or81}%
%    \end{macrocode}
%    \end{macro}
%
% \subsection{\op{Mul}}
%
%    \begin{macro}{\bigintcalcMul}
%    \begin{macrocode}
\def\bigintcalcMul#1#2{%
  \romannumeral0%
  \expandafter\expandafter\expandafter\BIC@Mul
  \bigintcalcNum{#1}!{#2}%
}
%    \end{macrocode}
%    \end{macro}
%    \begin{macro}{\BIC@Mul}
%    \begin{macrocode}
\def\BIC@Mul#1!#2{%
  \expandafter\expandafter\expandafter\BIC@MulSwitch
  \bigintcalcNum{#2}!#1!%
}
%    \end{macrocode}
%    \end{macro}
%    \begin{macro}{\BIC@MulSwitch}
%    Decision table for \cs{BIC@MulSwitch}.
%    \begin{quote}
%    \begin{tabular}[t]{@{}|l|l|l|l|l|@{}}
%      \hline
%      $x=0$ &\multicolumn{3}{l}{}    &$0$\\
%      \hline
%      $x>0$ & $y=0$ &\multicolumn2l{}&$0$\\
%      \cline{2-5}
%            & $y>0$ & $ x> y$ & $+$ & $\opMul( x, y)$\\
%      \cline{3-3}\cline{5-5}
%            &       & else    &     & $\opMul( y, x)$\\
%      \cline{2-5}
%            & $y<0$ & $ x>-y$ & $-$ & $\opMul( x,-y)$\\
%      \cline{3-3}\cline{5-5}
%            &       & else    &     & $\opMul(-y, x)$\\
%      \hline
%      $x<0$ & $y=0$ &\multicolumn2l{}&$0$\\
%      \cline{2-5}
%            & $y>0$ & $-x> y$ & $-$ & $\opMul(-x, y)$\\
%      \cline{3-3}\cline{5-5}
%            &       & else    &     & $\opMul( y,-x)$\\
%      \cline{2-5}
%            & $y<0$ & $-x>-y$ & $+$ & $\opMul(-x,-y)$\\
%      \cline{3-3}\cline{5-5}
%            &       & else    &     & $\opMul(-y,-x)$\\
%      \hline
%    \end{tabular}
%    \end{quote}
%    \begin{macrocode}
\def\BIC@MulSwitch#1#2!#3#4!{%
  \ifcase\BIC@Sgn#1#2! % x = 0
    \BIC@AfterFi{ 0}%
  \or % x > 0
    \ifcase\BIC@Sgn#3#4! % y = 0
      \BIC@AfterFiFi{ 0}%
    \or % y > 0
      \ifnum\BIC@PosCmp#1#2!#3#4!=1 % x > y
        \BIC@AfterFiFiFi{%
          \BIC@ProcessMul0!#1#2!#3#4!%
        }%
      \else % x <= y
        \BIC@AfterFiFiFi{%
          \BIC@ProcessMul0!#3#4!#1#2!%
        }%
      \fi
    \else % y < 0
      \expandafter-\romannumeral0%
      \ifnum\BIC@PosCmp#1#2!#4!=1 % x > -y
        \BIC@AfterFiFiFi{%
          \BIC@ProcessMul0!#1#2!#4!%
        }%
      \else % x <= -y
        \BIC@AfterFiFiFi{%
          \BIC@ProcessMul0!#4!#1#2!%
        }%
      \fi
    \fi
  \else % x < 0
    \ifcase\BIC@Sgn#3#4! % y = 0
      \BIC@AfterFiFi{ 0}%
    \or % y > 0
      \expandafter-\romannumeral0%
      \ifnum\BIC@PosCmp#2!#3#4!=1 % -x > y
        \BIC@AfterFiFiFi{%
          \BIC@ProcessMul0!#2!#3#4!%
        }%
      \else % -x <= y
        \BIC@AfterFiFiFi{%
          \BIC@ProcessMul0!#3#4!#2!%
        }%
      \fi
    \else % y < 0
      \ifnum\BIC@PosCmp#2!#4!=1 % -x > -y
        \BIC@AfterFiFiFi{%
          \BIC@ProcessMul0!#2!#4!%
        }%
      \else % -x <= -y
        \BIC@AfterFiFiFi{%
          \BIC@ProcessMul0!#4!#2!%
        }%
      \fi
    \fi
  \BIC@Fi
}
%    \end{macrocode}
%    \end{macro}
%    \begin{macro}{\BigIntCalcMul}
%    \begin{macrocode}
\def\BigIntCalcMul#1!#2!{%
  \romannumeral0%
  \BIC@ProcessMul0!#1!#2!%
}
%    \end{macrocode}
%    \end{macro}
%    \begin{macro}{\BIC@ProcessMul}
%    |#1|: result\\
%    |#2|: number $x$\\
%    |#3#4|: number $y$
%    \begin{macrocode}
\def\BIC@ProcessMul#1!#2!#3#4!{%
  \ifx\\#4\\%
    \BIC@AfterFi{%
      \expandafter\expandafter\expandafter\BIC@Space
      \bigintcalcAdd{\BIC@Tim#2!#3}{#10}%
    }%
  \else
    \BIC@AfterFi{%
      \expandafter\expandafter\expandafter\BIC@ProcessMul
      \bigintcalcAdd{\BIC@Tim#2!#3}{#10}!#2!#4!%
    }%
  \BIC@Fi
}
%    \end{macrocode}
%    \end{macro}
%
% \subsection{\op{Sqr}}
%
%    \begin{macro}{\bigintcalcSqr}
%    \begin{macrocode}
\def\bigintcalcSqr#1{%
  \romannumeral0%
  \expandafter\expandafter\expandafter\BIC@Sqr
  \bigintcalcNum{#1}!%
}
%    \end{macrocode}
%    \end{macro}
%    \begin{macro}{\BIC@Sqr}
%    \begin{macrocode}
\def\BIC@Sqr#1{%
   \ifx#1-%
     \expandafter\BIC@@Sqr
   \else
     \expandafter\BIC@@Sqr\expandafter#1%
   \fi
}
%    \end{macrocode}
%    \end{macro}
%    \begin{macro}{\BIC@@Sqr}
%    \begin{macrocode}
\def\BIC@@Sqr#1!{%
  \BIC@ProcessMul0!#1!#1!%
}
%    \end{macrocode}
%    \end{macro}
%
% \subsection{\op{Fac}}
%
%    \begin{macro}{\bigintcalcFac}
%    \begin{macrocode}
\def\bigintcalcFac#1{%
  \romannumeral0%
  \expandafter\expandafter\expandafter\BIC@Fac
  \bigintcalcNum{#1}!%
}
%    \end{macrocode}
%    \end{macro}
%    \begin{macro}{\BIC@Fac}
%    \begin{macrocode}
\def\BIC@Fac#1#2!{%
  \ifx#1-%
    \BIC@AfterFi{ 0\BigIntCalcError:FacNegative}%
  \else
    \ifnum\BIC@PosCmp#1#2!13!<0 %
      \ifcase#1#2 %
         \BIC@AfterFiFiFi{ 1}% 0!
      \or\BIC@AfterFiFiFi{ 1}% 1!
      \or\BIC@AfterFiFiFi{ 2}% 2!
      \or\BIC@AfterFiFiFi{ 6}% 3!
      \or\BIC@AfterFiFiFi{ 24}% 4!
      \or\BIC@AfterFiFiFi{ 120}% 5!
      \or\BIC@AfterFiFiFi{ 720}% 6!
      \or\BIC@AfterFiFiFi{ 5040}% 7!
      \or\BIC@AfterFiFiFi{ 40320}% 8!
      \or\BIC@AfterFiFiFi{ 362880}% 9!
      \or\BIC@AfterFiFiFi{ 3628800}% 10!
      \or\BIC@AfterFiFiFi{ 39916800}% 11!
      \or\BIC@AfterFiFiFi{ 479001600}% 12!
?     \else\BigIntCalcError:ThisCannotHappen%
      \fi
    \else
      \BIC@AfterFiFi{%
        \BIC@ProcessFac#1#2!479001600!%
      }%
    \fi
  \BIC@Fi
}
%    \end{macrocode}
%    \end{macro}
%    \begin{macro}{\BIC@ProcessFac}
%    |#1|: $n$\\
%    |#2|: result
%    \begin{macrocode}
\def\BIC@ProcessFac#1!#2!{%
  \ifnum\BIC@PosCmp#1!12!=0 %
    \BIC@AfterFi{ #2}%
  \else
    \BIC@AfterFi{%
      \expandafter\BIC@@ProcessFac
      \romannumeral0\BIC@ProcessMul0!#2!#1!%
      !#1!%
    }%
  \BIC@Fi
}
%    \end{macrocode}
%    \end{macro}
%    \begin{macro}{\BIC@@ProcessFac}
%    |#1|: result\\
%    |#2|: $n$
%    \begin{macrocode}
\def\BIC@@ProcessFac#1!#2!{%
  \expandafter\BIC@ProcessFac
  \romannumeral0\BIC@Dec#2!{}%
  !#1!%
}
%    \end{macrocode}
%    \end{macro}
%
% \subsection{\op{Pow}}
%
%    \begin{macro}{\bigintcalcPow}
%    |#1|: basis\\
%    |#2|: power
%    \begin{macrocode}
\def\bigintcalcPow#1{%
  \romannumeral0%
  \expandafter\expandafter\expandafter\BIC@Pow
  \bigintcalcNum{#1}!%
}
%    \end{macrocode}
%    \end{macro}
%    \begin{macro}{\BIC@Pow}
%    |#1|: basis\\
%    |#2|: power
%    \begin{macrocode}
\def\BIC@Pow#1!#2{%
  \expandafter\expandafter\expandafter\BIC@PowSwitch
  \bigintcalcNum{#2}!#1!%
}
%    \end{macrocode}
%    \end{macro}
%    \begin{macro}{\BIC@PowSwitch}
%    |#1#2|: power $y$\\
%    |#3#4|: basis $x$\\
%    Decision table for \cs{BIC@PowSwitch}.
%    \begin{quote}
%    \def\M#1{\multicolumn{#1}{l}{}}%
%    \begin{tabular}[t]{@{}|l|l|l|l|@{}}
%    \hline
%    $y=0$ & \M{2}                        & $1$\\
%    \hline
%    $y=1$ & \M{2}                        & $x$\\
%    \hline
%    $y=2$ & $x<0$          & \M{1}       & $\opMul(-x,-x)$\\
%    \cline{2-4}
%          & else           & \M{1}       & $\opMul(x,x)$\\
%    \hline
%    $y<0$ & $x=0$          & \M{1}       & DivisionByZero\\
%    \cline{2-4}
%          & $x=1$          & \M{1}       & $1$\\
%    \cline{2-4}
%          & $x=-1$         & $\opODD(y)$ & $-1$\\
%    \cline{3-4}
%          &                & else        & $1$\\
%    \cline{2-4}
%          & else ($\string|x\string|>1$) & \M{1}       & $0$\\
%    \hline
%    $y>2$ & $x=0$          & \M{1}       & $0$\\
%    \cline{2-4}
%          & $x=1$          & \M{1}       & $1$\\
%    \cline{2-4}
%          & $x=-1$         & $\opODD(y)$ & $-1$\\
%    \cline{3-4}
%          &                & else        & $1$\\
%    \cline{2-4}
%          & $x<-1$ ($x<0$) & $\opODD(y)$ & $-\opPow(-x,y)$\\
%    \cline{3-4}
%          &                & else        & $\opPow(-x,y)$\\
%    \cline{2-4}
%          & else ($x>1$)   & \M{1}       & $\opPow(x,y)$\\
%    \hline
%    \end{tabular}
%    \end{quote}
%    \begin{macrocode}
\def\BIC@PowSwitch#1#2!#3#4!{%
  \ifcase\ifx\\#2\\%
           \ifx#100 % y = 0
           \else\ifx#111 % y = 1
           \else\ifx#122 % y = 2
           \else4 % y > 2
           \fi\fi\fi
         \else
           \ifx#1-3 % y < 0
           \else4 % y > 2
           \fi
         \fi
    \BIC@AfterFi{ 1}% y = 0
  \or % y = 1
    \BIC@AfterFi{ #3#4}%
  \or % y = 2
    \ifx#3-% x < 0
      \BIC@AfterFiFi{%
        \BIC@ProcessMul0!#4!#4!%
      }%
    \else % x >= 0
      \BIC@AfterFiFi{%
        \BIC@ProcessMul0!#3#4!#3#4!%
      }%
    \fi
  \or % y < 0
    \ifcase\ifx\\#4\\%
             \ifx#300 % x = 0
             \else\ifx#311 % x = 1
             \else3 % x > 1
             \fi\fi
           \else
             \ifcase\BIC@MinusOne#3#4! %
               3 % |x| > 1
             \or
               2 % x = -1
?            \else\BigIntCalcError:ThisCannotHappen%
             \fi
           \fi
      \BIC@AfterFiFi{ 0\BigIntCalcError:DivisionByZero}% x = 0
    \or % x = 1
      \BIC@AfterFiFi{ 1}% x = 1
    \or % x = -1
      \ifcase\BIC@ModTwo#2! % even(y)
        \BIC@AfterFiFiFi{ 1}%
      \or % odd(y)
        \BIC@AfterFiFiFi{ -1}%
?     \else\BigIntCalcError:ThisCannotHappen%
      \fi
    \or % |x| > 1
      \BIC@AfterFiFi{ 0}%
?   \else\BigIntCalcError:ThisCannotHappen%
    \fi
  \or % y > 2
    \ifcase\ifx\\#4\\%
             \ifx#300 % x = 0
             \else\ifx#311 % x = 1
             \else4 % x > 1
             \fi\fi
           \else
             \ifx#3-%
               \ifcase\BIC@MinusOne#3#4! %
                 3 % x < -1
               \else
                 2 % x = -1
               \fi
             \else
               4 % x > 1
             \fi
           \fi
      \BIC@AfterFiFi{ 0}% x = 0
    \or % x = 1
      \BIC@AfterFiFi{ 1}% x = 1
    \or % x = -1
      \ifcase\BIC@ModTwo#1#2! % even(y)
        \BIC@AfterFiFiFi{ 1}%
      \or % odd(y)
        \BIC@AfterFiFiFi{ -1}%
?     \else\BigIntCalcError:ThisCannotHappen%
      \fi
    \or % x < -1
      \ifcase\BIC@ModTwo#1#2! % even(y)
        \BIC@AfterFiFiFi{%
          \BIC@PowRec#4!#1#2!1!%
        }%
      \or % odd(y)
        \expandafter-\romannumeral0%
        \BIC@AfterFiFiFi{%
          \BIC@PowRec#4!#1#2!1!%
        }%
?     \else\BigIntCalcError:ThisCannotHappen%
      \fi
    \or % x > 1
      \BIC@AfterFiFi{%
        \BIC@PowRec#3#4!#1#2!1!%
      }%
?   \else\BigIntCalcError:ThisCannotHappen%
    \fi
? \else\BigIntCalcError:ThisCannotHappen%
  \BIC@Fi
}
%    \end{macrocode}
%    \end{macro}
%
% \subsubsection{Help macros}
%
%    \begin{macro}{\BIC@ModTwo}
%    Macro \cs{BIC@ModTwo} expects a number without sign
%    and returns digit |1| or |0| if the number is odd or even.
%    \begin{macrocode}
\def\BIC@ModTwo#1#2!{%
  \ifx\\#2\\%
    \ifodd#1 %
      \BIC@AfterFiFi1%
    \else
      \BIC@AfterFiFi0%
    \fi
  \else
    \BIC@AfterFi{%
      \BIC@ModTwo#2!%
    }%
  \BIC@Fi
}
%    \end{macrocode}
%    \end{macro}
%
%    \begin{macro}{\BIC@MinusOne}
%    Macro \cs{BIC@MinusOne} expects a number and returns
%    digit |1| if the number equals minus one and returns |0| otherwise.
%    \begin{macrocode}
\def\BIC@MinusOne#1#2!{%
  \ifx#1-%
    \BIC@@MinusOne#2!%
  \else
    0%
  \fi
}
%    \end{macrocode}
%    \end{macro}
%    \begin{macro}{\BIC@@MinusOne}
%    \begin{macrocode}
\def\BIC@@MinusOne#1#2!{%
  \ifx#11%
    \ifx\\#2\\%
      1%
    \else
      0%
    \fi
  \else
    0%
  \fi
}
%    \end{macrocode}
%    \end{macro}
%
% \subsubsection{Recursive calculation}
%
%    \begin{macro}{\BIC@PowRec}
%\begin{quote}
%\begin{verbatim}
%Pow(x, y) {
%  PowRec(x, y, 1)
%}
%PowRec(x, y, r) {
%  if y == 1 then
%    return r
%  else
%    ifodd y then
%      return PowRec(x*x, y div 2, r*x) % y div 2 = (y-1)/2
%    else
%      return PowRec(x*x, y div 2, r)
%    fi
%  fi
%}
%\end{verbatim}
%\end{quote}
%    |#1|: $x$ (basis)\\
%    |#2#3|: $y$ (power)\\
%    |#4|: $r$ (result)
%    \begin{macrocode}
\def\BIC@PowRec#1!#2#3!#4!{%
  \ifcase\ifx#21\ifx\\#3\\0 \else1 \fi\else1 \fi % y = 1
    \ifnum\BIC@PosCmp#1!#4!=1 % x > r
      \BIC@AfterFiFi{%
        \BIC@ProcessMul0!#1!#4!%
      }%
    \else
      \BIC@AfterFiFi{%
        \BIC@ProcessMul0!#4!#1!%
      }%
    \fi
  \or
    \ifcase\BIC@ModTwo#2#3! % even(y)
      \BIC@AfterFiFi{%
        \expandafter\BIC@@PowRec\romannumeral0%
        \BIC@@Shr#2#3!%
        !#1!#4!%
      }%
    \or % odd(y)
      \ifnum\BIC@PosCmp#1!#4!=1 % x > r
        \BIC@AfterFiFiFi{%
          \expandafter\BIC@@@PowRec\romannumeral0%
          \BIC@ProcessMul0!#1!#4!%
          !#1!#2#3!%
        }%
      \else
        \BIC@AfterFiFiFi{%
          \expandafter\BIC@@@PowRec\romannumeral0%
          \BIC@ProcessMul0!#1!#4!%
          !#1!#2#3!%
        }%
      \fi
?   \else\BigIntCalcError:ThisCannotHappen%
    \fi
? \else\BigIntCalcError:ThisCannotHappen%
  \BIC@Fi
}
%    \end{macrocode}
%    \end{macro}
%    \begin{macro}{\BIC@@PowRec}
%    |#1|: $y/2$\\
%    |#2|: $x$\\
%    |#3|: new $r$ ($r$ or $r*x$)
%    \begin{macrocode}
\def\BIC@@PowRec#1!#2!#3!{%
  \expandafter\BIC@PowRec\romannumeral0%
  \BIC@ProcessMul0!#2!#2!%
  !#1!#3!%
}
%    \end{macrocode}
%    \end{macro}
%    \begin{macro}{\BIC@@@PowRec}
%    |#1|: $r*x$
%    |#2|: $x$
%    |#3|: $y$
%    \begin{macrocode}
\def\BIC@@@PowRec#1!#2!#3!{%
  \expandafter\BIC@@PowRec\romannumeral0%
  \BIC@@Shr#3!%
  !#2!#1!%
}
%    \end{macrocode}
%    \end{macro}
%
% \subsection{\op{Div}}
%
%    \begin{macro}{\bigintcalcDiv}
%    |#1|: $x$\\
%    |#2|: $y$ (divisor)
%    \begin{macrocode}
\def\bigintcalcDiv#1{%
  \romannumeral0%
  \expandafter\expandafter\expandafter\BIC@Div
  \bigintcalcNum{#1}!%
}
%    \end{macrocode}
%    \end{macro}
%    \begin{macro}{\BIC@Div}
%    |#1|: $x$\\
%    |#2|: $y$
%    \begin{macrocode}
\def\BIC@Div#1!#2{%
  \expandafter\expandafter\expandafter\BIC@DivSwitchSign
  \bigintcalcNum{#2}!#1!%
}
%    \end{macrocode}
%    \end{macro}
%    \begin{macro}{\BigIntCalcDiv}
%    \begin{macrocode}
\def\BigIntCalcDiv#1!#2!{%
  \romannumeral0%
  \BIC@DivSwitchSign#2!#1!%
}
%    \end{macrocode}
%    \end{macro}
%    \begin{macro}{\BIC@DivSwitchSign}
%    Decision table for \cs{BIC@DivSwitchSign}.
%    \begin{quote}
%    \begin{tabular}{@{}|l|l|l|@{}}
%    \hline
%    $y=0$ & \multicolumn{1}{l}{} & DivisionByZero\\
%    \hline
%    $y>0$ & $x=0$ & $0$\\
%    \cline{2-3}
%          & $x>0$ & DivSwitch$(+,x,y)$\\
%    \cline{2-3}
%          & $x<0$ & DivSwitch$(-,-x,y)$\\
%    \hline
%    $y<0$ & $x=0$ & $0$\\
%    \cline{2-3}
%          & $x>0$ & DivSwitch$(-,x,-y)$\\
%    \cline{2-3}
%          & $x<0$ & DivSwitch$(+,-x,-y)$\\
%    \hline
%    \end{tabular}
%    \end{quote}
%    |#1|: $y$ (divisor)\\
%    |#2|: $x$
%    \begin{macrocode}
\def\BIC@DivSwitchSign#1#2!#3#4!{%
  \ifcase\BIC@Sgn#1#2! % y = 0
    \BIC@AfterFi{ 0\BigIntCalcError:DivisionByZero}%
  \or % y > 0
    \ifcase\BIC@Sgn#3#4! % x = 0
      \BIC@AfterFiFi{ 0}%
    \or % x > 0
      \BIC@AfterFiFi{%
        \BIC@DivSwitch{}#3#4!#1#2!%
      }%
    \else % x < 0
      \BIC@AfterFiFi{%
        \BIC@DivSwitch-#4!#1#2!%
      }%
    \fi
  \else % y < 0
    \ifcase\BIC@Sgn#3#4! % x = 0
      \BIC@AfterFiFi{ 0}%
    \or % x > 0
      \BIC@AfterFiFi{%
        \BIC@DivSwitch-#3#4!#2!%
      }%
    \else % x < 0
      \BIC@AfterFiFi{%
        \BIC@DivSwitch{}#4!#2!%
      }%
    \fi
  \BIC@Fi
}
%    \end{macrocode}
%    \end{macro}
%    \begin{macro}{\BIC@DivSwitch}
%    Decision table for \cs{BIC@DivSwitch}.
%    \begin{quote}
%    \begin{tabular}{@{}|l|l|l|@{}}
%    \hline
%    $y=x$ & \multicolumn{1}{l}{} & sign $1$\\
%    \hline
%    $y>x$ & \multicolumn{1}{l}{} & $0$\\
%    \hline
%    $y<x$ & $y=1$                & sign $x$\\
%    \cline{2-3}
%          & $y=2$                & sign Shr$(x)$\\
%    \cline{2-3}
%          & $y=4$                & sign Shr(Shr$(x)$)\\
%    \cline{2-3}
%          & else                 & sign ProcessDiv$(x,y)$\\
%    \hline
%    \end{tabular}
%    \end{quote}
%    |#1|: sign\\
%    |#2|: $x$\\
%    |#3#4|: $y$ ($y\ne 0$)
%    \begin{macrocode}
\def\BIC@DivSwitch#1#2!#3#4!{%
  \ifcase\BIC@PosCmp#3#4!#2!% y = x
    \BIC@AfterFi{ #11}%
  \or % y > x
    \BIC@AfterFi{ 0}%
  \else % y < x
    \ifx\\#1\\%
    \else
      \expandafter-\romannumeral0%
    \fi
    \ifcase\ifx\\#4\\%
             \ifx#310 % y = 1
             \else\ifx#321 % y = 2
             \else\ifx#342 % y = 4
             \else3 % y > 2
             \fi\fi\fi
           \else
             3 % y > 2
           \fi
      \BIC@AfterFiFi{ #2}% y = 1
    \or % y = 2
      \BIC@AfterFiFi{%
        \BIC@@Shr#2!%
      }%
    \or % y = 4
      \BIC@AfterFiFi{%
        \expandafter\BIC@@Shr\romannumeral0%
          \BIC@@Shr#2!!%
      }%
    \or % y > 2
      \BIC@AfterFiFi{%
        \BIC@DivStartX#2!#3#4!!!%
      }%
?   \else\BigIntCalcError:ThisCannotHappen%
    \fi
  \BIC@Fi
}
%    \end{macrocode}
%    \end{macro}
%    \begin{macro}{\BIC@ProcessDiv}
%    |#1#2|: $x$\\
%    |#3#4|: $y$\\
%    |#5|: collect first digits of $x$\\
%    |#6|: corresponding digits of $y$
%    \begin{macrocode}
\def\BIC@DivStartX#1#2!#3#4!#5!#6!{%
  \ifx\\#4\\%
    \BIC@AfterFi{%
      \BIC@DivStartYii#6#3#4!{#5#1}#2=!%
    }%
  \else
    \BIC@AfterFi{%
      \BIC@DivStartX#2!#4!#5#1!#6#3!%
    }%
  \BIC@Fi
}
%    \end{macrocode}
%    \end{macro}
%    \begin{macro}{\BIC@DivStartYii}
%    |#1|: $y$\\
%    |#2|: $x$, |=|
%    \begin{macrocode}
\def\BIC@DivStartYii#1!{%
  \expandafter\BIC@DivStartYiv\romannumeral0%
  \BIC@Shl#1!%
  !#1!%
}
%    \end{macrocode}
%    \end{macro}
%    \begin{macro}{\BIC@DivStartYiv}
%    |#1|: $2y$\\
%    |#2|: $y$\\
%    |#3|: $x$, |=|
%    \begin{macrocode}
\def\BIC@DivStartYiv#1!{%
  \expandafter\BIC@DivStartYvi\romannumeral0%
  \BIC@Shl#1!%
  !#1!%
}
%    \end{macrocode}
%    \end{macro}
%    \begin{macro}{\BIC@DivStartYvi}
%    |#1|: $4y$\\
%    |#2|: $2y$\\
%    |#3|: $y$\\
%    |#4|: $x$, |=|
%    \begin{macrocode}
\def\BIC@DivStartYvi#1!#2!{%
  \expandafter\BIC@DivStartYviii\romannumeral0%
  \BIC@AddXY#1!#2!!!%
  !#1!#2!%
}
%    \end{macrocode}
%    \end{macro}
%    \begin{macro}{\BIC@DivStartYviii}
%    |#1|: $6y$\\
%    |#2|: $4y$\\
%    |#3|: $2y$\\
%    |#4|: $y$\\
%    |#5|: $x$, |=|
%    \begin{macrocode}
\def\BIC@DivStartYviii#1!#2!{%
  \expandafter\BIC@DivStart\romannumeral0%
  \BIC@Shl#2!%
  !#1!#2!%
}
%    \end{macrocode}
%    \end{macro}
%    \begin{macro}{\BIC@DivStart}
%    |#1|: $8y$\\
%    |#2|: $6y$\\
%    |#3|: $4y$\\
%    |#4|: $2y$\\
%    |#5|: $y$\\
%    |#6|: $x$, |=|
%    \begin{macrocode}
\def\BIC@DivStart#1!#2!#3!#4!#5!#6!{%
  \BIC@ProcessDiv#6!!#5!#4!#3!#2!#1!=%
}
%    \end{macrocode}
%    \end{macro}
%    \begin{macro}{\BIC@ProcessDiv}
%    |#1#2#3|: $x$, |=|\\
%    |#4|: result\\
%    |#5|: $y$\\
%    |#6|: $2y$\\
%    |#7|: $4y$\\
%    |#8|: $6y$\\
%    |#9|: $8y$
%    \begin{macrocode}
\def\BIC@ProcessDiv#1#2#3!#4!#5!{%
  \ifcase\BIC@PosCmp#5!#1!% y = #1
    \ifx#2=%
      \BIC@AfterFiFi{\BIC@DivCleanup{#41}}%
    \else
      \BIC@AfterFiFi{%
        \BIC@ProcessDiv#2#3!#41!#5!%
      }%
    \fi
  \or % y > #1
    \ifx#2=%
      \BIC@AfterFiFi{\BIC@DivCleanup{#40}}%
    \else
      \ifx\\#4\\%
        \BIC@AfterFiFiFi{%
          \BIC@ProcessDiv{#1#2}#3!!#5!%
        }%
      \else
        \BIC@AfterFiFiFi{%
          \BIC@ProcessDiv{#1#2}#3!#40!#5!%
        }%
      \fi
    \fi
  \else % y < #1
    \BIC@AfterFi{%
      \BIC@@ProcessDiv{#1}#2#3!#4!#5!%
    }%
  \BIC@Fi
}
%    \end{macrocode}
%    \end{macro}
%    \begin{macro}{\BIC@DivCleanup}
%    |#1|: result\\
%    |#2|: garbage
%    \begin{macrocode}
\def\BIC@DivCleanup#1#2={ #1}%
%    \end{macrocode}
%    \end{macro}
%    \begin{macro}{\BIC@@ProcessDiv}
%    \begin{macrocode}
\def\BIC@@ProcessDiv#1#2#3!#4!#5!#6!#7!{%
  \ifcase\BIC@PosCmp#7!#1!% 4y = #1
    \ifx#2=%
      \BIC@AfterFiFi{\BIC@DivCleanup{#44}}%
    \else
     \BIC@AfterFiFi{%
       \BIC@ProcessDiv#2#3!#44!#5!#6!#7!%
     }%
    \fi
  \or % 4y > #1
    \ifcase\BIC@PosCmp#6!#1!% 2y = #1
      \ifx#2=%
        \BIC@AfterFiFiFi{\BIC@DivCleanup{#42}}%
      \else
        \BIC@AfterFiFiFi{%
          \BIC@ProcessDiv#2#3!#42!#5!#6!#7!%
        }%
      \fi
    \or % 2y > #1
      \ifx#2=%
        \BIC@AfterFiFiFi{\BIC@DivCleanup{#41}}%
      \else
        \BIC@AfterFiFiFi{%
          \BIC@DivSub#1!#5!#2#3!#41!#5!#6!#7!%
        }%
      \fi
    \else % 2y < #1
      \BIC@AfterFiFi{%
        \expandafter\BIC@ProcessDivII\romannumeral0%
        \BIC@SubXY#1!#6!!!%
        !#2#3!#4!#5!23%
        #6!#7!%
      }%
    \fi
  \else % 4y < #1
    \BIC@AfterFi{%
      \BIC@@@ProcessDiv{#1}#2#3!#4!#5!#6!#7!%
    }%
  \BIC@Fi
}
%    \end{macrocode}
%    \end{macro}
%    \begin{macro}{\BIC@DivSub}
%    Next token group: |#1|-|#2| and next digit |#3|.
%    \begin{macrocode}
\def\BIC@DivSub#1!#2!#3{%
  \expandafter\BIC@ProcessDiv\expandafter{%
    \romannumeral0%
    \BIC@SubXY#1!#2!!!%
    #3%
  }%
}
%    \end{macrocode}
%    \end{macro}
%    \begin{macro}{\BIC@ProcessDivII}
%    |#1|: $x'-2y$\\
%    |#2#3|: remaining $x$, |=|\\
%    |#4|: result\\
%    |#5|: $y$\\
%    |#6|: first possible result digit\\
%    |#7|: second possible result digit
%    \begin{macrocode}
\def\BIC@ProcessDivII#1!#2#3!#4!#5!#6#7{%
  \ifcase\BIC@PosCmp#5!#1!% y = #1
    \ifx#2=%
      \BIC@AfterFiFi{\BIC@DivCleanup{#4#7}}%
    \else
      \BIC@AfterFiFi{%
        \BIC@ProcessDiv#2#3!#4#7!#5!%
      }%
    \fi
  \or % y > #1
    \ifx#2=%
      \BIC@AfterFiFi{\BIC@DivCleanup{#4#6}}%
    \else
      \BIC@AfterFiFi{%
        \BIC@ProcessDiv{#1#2}#3!#4#6!#5!%
      }%
    \fi
  \else % y < #1
    \ifx#2=%
      \BIC@AfterFiFi{\BIC@DivCleanup{#4#7}}%
    \else
      \BIC@AfterFiFi{%
        \BIC@DivSub#1!#5!#2#3!#4#7!#5!%
      }%
    \fi
  \BIC@Fi
}
%    \end{macrocode}
%    \end{macro}
%    \begin{macro}{\BIC@ProcessDivIV}
%    |#1#2#3|: $x$, |=|, $x>4y$\\
%    |#4|: result\\
%    |#5|: $y$\\
%    |#6|: $2y$\\
%    |#7|: $4y$\\
%    |#8|: $6y$\\
%    |#9|: $8y$
%    \begin{macrocode}
\def\BIC@@@ProcessDiv#1#2#3!#4!#5!#6!#7!#8!#9!{%
  \ifcase\BIC@PosCmp#8!#1!% 6y = #1
    \ifx#2=%
      \BIC@AfterFiFi{\BIC@DivCleanup{#46}}%
    \else
      \BIC@AfterFiFi{%
        \BIC@ProcessDiv#2#3!#46!#5!#6!#7!#8!#9!%
      }%
    \fi
  \or % 6y > #1
    \BIC@AfterFi{%
      \expandafter\BIC@ProcessDivII\romannumeral0%
      \BIC@SubXY#1!#7!!!%
      !#2#3!#4!#5!45%
      #6!#7!#8!#9!%
    }%
  \else % 6y < #1
    \ifcase\BIC@PosCmp#9!#1!% 8y = #1
      \ifx#2=%
        \BIC@AfterFiFiFi{\BIC@DivCleanup{#48}}%
      \else
        \BIC@AfterFiFiFi{%
          \BIC@ProcessDiv#2#3!#48!#5!#6!#7!#8!#9!%
        }%
      \fi
    \or % 8y > #1
      \BIC@AfterFiFi{%
        \expandafter\BIC@ProcessDivII\romannumeral0%
        \BIC@SubXY#1!#8!!!%
        !#2#3!#4!#5!67%
        #6!#7!#8!#9!%
      }%
    \else % 8y < #1
      \BIC@AfterFiFi{%
        \expandafter\BIC@ProcessDivII\romannumeral0%
        \BIC@SubXY#1!#9!!!%
        !#2#3!#4!#5!89%
        #6!#7!#8!#9!%
      }%
    \fi
  \BIC@Fi
}
%    \end{macrocode}
%    \end{macro}
%
% \subsection{\op{Mod}}
%
%    \begin{macro}{\bigintcalcMod}
%    |#1|: $x$\\
%    |#2|: $y$
%    \begin{macrocode}
\def\bigintcalcMod#1{%
  \romannumeral0%
  \expandafter\expandafter\expandafter\BIC@Mod
  \bigintcalcNum{#1}!%
}
%    \end{macrocode}
%    \end{macro}
%    \begin{macro}{\BIC@Mod}
%    |#1|: $x$\\
%    |#2|: $y$
%    \begin{macrocode}
\def\BIC@Mod#1!#2{%
  \expandafter\expandafter\expandafter\BIC@ModSwitchSign
  \bigintcalcNum{#2}!#1!%
}
%    \end{macrocode}
%    \end{macro}
%    \begin{macro}{\BigIntCalcMod}
%    \begin{macrocode}
\def\BigIntCalcMod#1!#2!{%
  \romannumeral0%
  \BIC@ModSwitchSign#2!#1!%
}
%    \end{macrocode}
%    \end{macro}
%    \begin{macro}{\BIC@ModSwitchSign}
%    Decision table for \cs{BIC@ModSwitchSign}.
%    \begin{quote}
%    \begin{tabular}{@{}|l|l|l|@{}}
%    \hline
%    $y=0$ & \multicolumn{1}{l}{} & DivisionByZero\\
%    \hline
%    $y>0$ & $x=0$ & $0$\\
%    \cline{2-3}
%          & else  & ModSwitch$(+,x,y)$\\
%    \hline
%    $y<0$ & \multicolumn{1}{l}{} & ModSwitch$(-,-x,-y)$\\
%    \hline
%    \end{tabular}
%    \end{quote}
%    |#1#2|: $y$\\
%    |#3#4|: $x$
%    \begin{macrocode}
\def\BIC@ModSwitchSign#1#2!#3#4!{%
  \ifcase\ifx\\#2\\%
           \ifx#100 % y = 0
           \else1 % y > 0
           \fi
          \else
            \ifx#1-2 % y < 0
            \else1 % y > 0
            \fi
          \fi
    \BIC@AfterFi{ 0\BigIntCalcError:DivisionByZero}%
  \or % y > 0
    \ifcase\ifx\\#4\\\ifx#300 \else1 \fi\else1 \fi % x = 0
      \BIC@AfterFiFi{ 0}%
    \else
      \BIC@AfterFiFi{%
        \BIC@ModSwitch{}#3#4!#1#2!%
      }%
    \fi
  \else % y < 0
    \ifcase\ifx\\#4\\%
             \ifx#300 % x = 0
             \else1 % x > 0
             \fi
           \else
             \ifx#3-2 % x < 0
             \else1 % x > 0
             \fi
           \fi
      \BIC@AfterFiFi{ 0}%
    \or % x > 0
      \BIC@AfterFiFi{%
        \BIC@ModSwitch--#3#4!#2!%
      }%
    \else % x < 0
      \BIC@AfterFiFi{%
        \BIC@ModSwitch-#4!#2!%
      }%
    \fi
  \BIC@Fi
}
%    \end{macrocode}
%    \end{macro}
%    \begin{macro}{\BIC@ModSwitch}
%    Decision table for \cs{BIC@ModSwitch}.
%    \begin{quote}
%    \begin{tabular}{@{}|l|l|l|@{}}
%    \hline
%    $y=1$ & \multicolumn{1}{l}{} & $0$\\
%    \hline
%    $y=2$ & ifodd$(x)$ & sign $1$\\
%    \cline{2-3}
%          & else       & $0$\\
%    \hline
%    $y>2$ & $x<0$      &
%        $z\leftarrow x-(x/y)*y;\quad (z<0)\mathbin{?}z+y\mathbin{:}z$\\
%    \cline{2-3}
%          & $x>0$      & $x - (x/y) * y$\\
%    \hline
%    \end{tabular}
%    \end{quote}
%    |#1|: sign\\
%    |#2#3|: $x$\\
%    |#4#5|: $y$
%    \begin{macrocode}
\def\BIC@ModSwitch#1#2#3!#4#5!{%
  \ifcase\ifx\\#5\\%
           \ifx#410 % y = 1
           \else\ifx#421 % y = 2
           \else2 % y > 2
           \fi\fi
         \else2 % y > 2
         \fi
    \BIC@AfterFi{ 0}% y = 1
  \or % y = 2
    \ifcase\BIC@ModTwo#2#3! % even(x)
      \BIC@AfterFiFi{ 0}%
    \or % odd(x)
      \BIC@AfterFiFi{ #11}%
?   \else\BigIntCalcError:ThisCannotHappen%
    \fi
  \or % y > 2
    \ifx\\#1\\%
    \else
      \expandafter\BIC@Space\romannumeral0%
      \expandafter\BIC@ModMinus\romannumeral0%
    \fi
    \ifx#2-% x < 0
      \BIC@AfterFiFi{%
        \expandafter\expandafter\expandafter\BIC@ModX
        \bigintcalcSub{#2#3}{%
          \bigintcalcMul{#4#5}{\bigintcalcDiv{#2#3}{#4#5}}%
        }!#4#5!%
      }%
    \else % x > 0
      \BIC@AfterFiFi{%
        \expandafter\expandafter\expandafter\BIC@Space
        \bigintcalcSub{#2#3}{%
          \bigintcalcMul{#4#5}{\bigintcalcDiv{#2#3}{#4#5}}%
        }%
      }%
    \fi
? \else\BigIntCalcError:ThisCannotHappen%
  \BIC@Fi
}
%    \end{macrocode}
%    \end{macro}
%    \begin{macro}{\BIC@ModMinus}
%    \begin{macrocode}
\def\BIC@ModMinus#1{%
  \ifx#10%
    \BIC@AfterFi{ 0}%
  \else
    \BIC@AfterFi{ -#1}%
  \BIC@Fi
}
%    \end{macrocode}
%    \end{macro}
%    \begin{macro}{\BIC@ModX}
%    |#1#2|: $z$\\
%    |#3|: $x$
%    \begin{macrocode}
\def\BIC@ModX#1#2!#3!{%
  \ifx#1-% z < 0
    \BIC@AfterFi{%
      \expandafter\BIC@Space\romannumeral0%
      \BIC@SubXY#3!#2!!!%
    }%
  \else % z >= 0
    \BIC@AfterFi{ #1#2}%
  \BIC@Fi
}
%    \end{macrocode}
%    \end{macro}
%
%    \begin{macrocode}
\BIC@AtEnd%
%    \end{macrocode}
%
%    \begin{macrocode}
%</package>
%    \end{macrocode}
%
% \section{Test}
%
% \subsection{Catcode checks for loading}
%
%    \begin{macrocode}
%<*test1>
%    \end{macrocode}
%    \begin{macrocode}
\catcode`\{=1 %
\catcode`\}=2 %
\catcode`\#=6 %
\catcode`\@=11 %
\expandafter\ifx\csname count@\endcsname\relax
  \countdef\count@=255 %
\fi
\expandafter\ifx\csname @gobble\endcsname\relax
  \long\def\@gobble#1{}%
\fi
\expandafter\ifx\csname @firstofone\endcsname\relax
  \long\def\@firstofone#1{#1}%
\fi
\expandafter\ifx\csname loop\endcsname\relax
  \expandafter\@firstofone
\else
  \expandafter\@gobble
\fi
{%
  \def\loop#1\repeat{%
    \def\body{#1}%
    \iterate
  }%
  \def\iterate{%
    \body
      \let\next\iterate
    \else
      \let\next\relax
    \fi
    \next
  }%
  \let\repeat=\fi
}%
\def\RestoreCatcodes{}
\count@=0 %
\loop
  \edef\RestoreCatcodes{%
    \RestoreCatcodes
    \catcode\the\count@=\the\catcode\count@\relax
  }%
\ifnum\count@<255 %
  \advance\count@ 1 %
\repeat

\def\RangeCatcodeInvalid#1#2{%
  \count@=#1\relax
  \loop
    \catcode\count@=15 %
  \ifnum\count@<#2\relax
    \advance\count@ 1 %
  \repeat
}
\def\RangeCatcodeCheck#1#2#3{%
  \count@=#1\relax
  \loop
    \ifnum#3=\catcode\count@
    \else
      \errmessage{%
        Character \the\count@\space
        with wrong catcode \the\catcode\count@\space
        instead of \number#3%
      }%
    \fi
  \ifnum\count@<#2\relax
    \advance\count@ 1 %
  \repeat
}
\def\space{ }
\expandafter\ifx\csname LoadCommand\endcsname\relax
  \def\LoadCommand{\input bigintcalc.sty\relax}%
\fi
\def\Test{%
  \RangeCatcodeInvalid{0}{47}%
  \RangeCatcodeInvalid{58}{64}%
  \RangeCatcodeInvalid{91}{96}%
  \RangeCatcodeInvalid{123}{255}%
  \catcode`\@=12 %
  \catcode`\\=0 %
  \catcode`\%=14 %
  \LoadCommand
  \RangeCatcodeCheck{0}{36}{15}%
  \RangeCatcodeCheck{37}{37}{14}%
  \RangeCatcodeCheck{38}{47}{15}%
  \RangeCatcodeCheck{48}{57}{12}%
  \RangeCatcodeCheck{58}{63}{15}%
  \RangeCatcodeCheck{64}{64}{12}%
  \RangeCatcodeCheck{65}{90}{11}%
  \RangeCatcodeCheck{91}{91}{15}%
  \RangeCatcodeCheck{92}{92}{0}%
  \RangeCatcodeCheck{93}{96}{15}%
  \RangeCatcodeCheck{97}{122}{11}%
  \RangeCatcodeCheck{123}{255}{15}%
  \RestoreCatcodes
}
\Test
\csname @@end\endcsname
\end
%    \end{macrocode}
%    \begin{macrocode}
%</test1>
%    \end{macrocode}
%
% \subsection{Macro tests}
%
% \subsubsection{Preamble with test macro definitions}
%
%    \begin{macrocode}
%<*test2>
\NeedsTeXFormat{LaTeX2e}
\nofiles
\documentclass{article}
%<noetex>\let\SavedNumexpr\numexpr
%<noetex>\let\numexpr\UNDEFINED
\makeatletter
\chardef\BIC@TestMode=1 %
\makeatother
\usepackage{bigintcalc}[2012/04/08]
%<noetex>\let\numexpr\SavedNumexpr
\usepackage{qstest}
\IncludeTests{*}
\LogTests{log}{*}{*}
\newcommand*{\TestSpaceAtEnd}[1]{%
%<noetex>  \let\SavedNumexpr\numexpr
%<noetex>  \let\numexpr\UNDEFINED
  \edef\resultA{#1}%
  \edef\resultB{#1 }%
%<noetex>  \let\numexpr\SavedNumexpr
  \Expect*{\resultA\space}*{\resultB}%
}
\newcommand*{\TestResult}[2]{%
%<noetex>  \let\SavedNumexpr\numexpr
%<noetex>  \let\numexpr\UNDEFINED
  \edef\result{#1}%
%<noetex>  \let\numexpr\SavedNumexpr
  \Expect*{\result}{#2}%
}
\newcommand*{\TestResultTwoExpansions}[2]{%
%<*noetex>
  \begingroup
    \let\numexpr\UNDEFINED
    \expandafter\expandafter\expandafter
  \endgroup
%</noetex>
  \expandafter\expandafter\expandafter\Expect
  \expandafter\expandafter\expandafter{#1}{#2}%
}
\newcount\TestCount
%<etex>\newcommand*{\TestArg}[1]{\numexpr#1\relax}
%<noetex>\newcommand*{\TestArg}[1]{#1}
\newcommand*{\TestTeXDivide}[2]{%
  \TestCount=\TestArg{#1}\relax
  \divide\TestCount by \TestArg{#2}\relax
  \Expect*{\bigintcalcDiv{#1}{#2}}*{\the\TestCount}%
}
\newcommand*{\Test}[2]{%
  \TestResult{#1}{#2}%
  \TestResultTwoExpansions{#1}{#2}%
  \TestSpaceAtEnd{#1}%
}
\newcommand*{\TestExch}[2]{\Test{#2}{#1}}
\newcommand*{\TestInv}[2]{%
  \Test{\bigintcalcInv{#1}}{#2}%
}
\newcommand*{\TestAbs}[2]{%
  \Test{\bigintcalcAbs{#1}}{#2}%
}
\newcommand*{\TestSgn}[2]{%
  \Test{\bigintcalcSgn{#1}}{#2}%
}
\newcommand*{\TestMin}[3]{%
  \Test{\bigintcalcMin{#1}{#2}}{#3}%
}
\newcommand*{\TestMax}[3]{%
  \Test{\bigintcalcMax{#1}{#2}}{#3}%
}
\newcommand*{\TestCmp}[3]{%
  \Test{\bigintcalcCmp{#1}{#2}}{#3}%
}
\newcommand*{\TestOdd}[2]{%
  \Test{\bigintcalcOdd{#1}}{#2}%
  \edef\x{%
    \noexpand\Test{%
      \noexpand\BigIntCalcOdd
      \bigintcalcAbs{#1}!%
    }{#2}%
  }%
  \x
}
\newcommand*{\TestInc}[2]{%
  \Test{\bigintcalcInc{#1}}{#2}%
  \ifnum\bigintcalcSgn{#1}>-1 %
    \edef\x{%
      \noexpand\Test{%
        \noexpand\BigIntCalcInc\bigintcalcNum{#1}!%
      }{#2}%
    }%
    \x
  \fi
}
\newcommand*{\TestDec}[2]{%
  \Test{\bigintcalcDec{#1}}{#2}%
  \ifnum\bigintcalcSgn{#1}>0 %
    \edef\x{%
      \noexpand\Test{%
        \noexpand\BigIntCalcDec\bigintcalcNum{#1}!%
      }{#2}%
    }%
    \x
  \fi
}
\newcommand*{\TestAdd}[3]{%
  \Test{\bigintcalcAdd{#1}{#2}}{#3}%
  \ifnum\bigintcalcSgn{#1}>0 %
    \ifnum\bigintcalcSgn{#2}> 0 %
      \ifnum\bigintcalcCmp{#1}{#2}>0 %
        \edef\x{%
          \noexpand\Test{%
            \noexpand\BigIntCalcAdd
            \bigintcalcNum{#1}!\bigintcalcNum{#2}!%
          }{#3}%
        }%
        \x
      \else
        \edef\x{%
          \noexpand\Test{%
            \noexpand\BigIntCalcAdd
            \bigintcalcNum{#2}!\bigintcalcNum{#1}!%
          }{#3}%
        }%
        \x
      \fi
    \fi
  \fi
}
\newcommand*{\TestSub}[3]{%
  \Test{\bigintcalcSub{#1}{#2}}{#3}%
  \ifnum\bigintcalcSgn{#1}>0 %
    \ifnum\bigintcalcSgn{#2}> 0 %
      \ifnum\bigintcalcCmp{#1}{#2}>0 %
        \edef\x{%
          \noexpand\Test{%
            \noexpand\BigIntCalcSub
            \bigintcalcNum{#1}!\bigintcalcNum{#2}!%
          }{#3}%
        }%
        \x
      \fi
    \fi
  \fi
}
\newcommand*{\TestShl}[2]{%
  \Test{\bigintcalcShl{#1}}{#2}%
  \edef\x{%
    \noexpand\Test{%
      \noexpand\BigIntCalcShl\bigintcalcAbs{#1}!%
    }{\bigintcalcAbs{#2}}%
  }%
  \x
}
\newcommand*{\TestShr}[2]{%
  \Test{\bigintcalcShr{#1}}{#2}%
  \edef\x{%
    \noexpand\Test{%
      \noexpand\BigIntCalcShr\bigintcalcAbs{#1}!%
    }{\bigintcalcAbs{#2}}%
  }%
  \x
}
\newcommand*{\TestMul}[3]{%
  \Test{\bigintcalcMul{#1}{#2}}{#3}%
  \edef\x{%
    \noexpand\Test{%
      \noexpand\BigIntCalcMul
      \bigintcalcAbs{#1}!\bigintcalcAbs{#2}!%
    }{\bigintcalcAbs{#3}}%
  }%
  \x
}
\newcommand*{\TestSqr}[2]{%
  \Test{\bigintcalcSqr{#1}}{#2}%
}
\newcommand*{\TestFac}[2]{%
  \expandafter\TestExch\expandafter{%
    \the\numexpr#2%
  }{\bigintcalcFac{#1}}%
}
\newcommand*{\TestFacBig}[2]{%
  \Test{\bigintcalcFac{#1}}{#2}%
}
\newcommand*{\TestPow}[3]{%
  \Test{\bigintcalcPow{#1}{#2}}{#3}%
}
\newcommand*{\TestDiv}[3]{%
  \Test{\bigintcalcDiv{#1}{#2}}{#3}%
  \TestTeXDivide{#1}{#2}%
}
\newcommand*{\TestDivBig}[3]{%
  \Test{\bigintcalcDiv{#1}{#2}}{#3}%
  \edef\x{%
    \noexpand\Test{%
      \noexpand\BigIntCalcDiv\bigintcalcAbs{#1}!\bigintcalcAbs{#2}!%
    }{\bigintcalcAbs{#3}}%
  }%
}
\newcommand*{\TestMod}[3]{%
  \Test{\bigintcalcMod{#1}{#2}}{#3}%
  \ifcase\ifcase\bigintcalcSgn{#1} 0%
         \or
           \ifcase\bigintcalcSgn{#2} 1%
           \or 0%
           \else 1%
           \fi
         \else
           \ifcase\bigintcalcSgn{#2} 1%
           \or 1%
           \else 0%
           \fi
         \fi\relax
    \edef\x{%
      \noexpand\Test{%
        \noexpand\BigIntCalcMod
        \bigintcalcAbs{#1}!\bigintcalcAbs{#2}!%
      }{\bigintcalcAbs{#3}}%
    }%
    \x
  \fi
}
%    \end{macrocode}
%
% \subsubsection{Time}
%
%    \begin{macrocode}
\begingroup\expandafter\expandafter\expandafter\endgroup
\expandafter\ifx\csname pdfresettimer\endcsname\relax
\else
  \makeatletter
  \newcount\SummaryTime
  \newcount\TestTime
  \SummaryTime=\z@
  \newcommand*{\PrintTime}[2]{%
    \typeout{%
      [Time #1: \strip@pt\dimexpr\number#2sp\relax\space s]%
    }%
  }%
  \newcommand*{\StartTime}[1]{%
    \renewcommand*{\TimeDescription}{#1}%
    \pdfresettimer
  }%
  \newcommand*{\TimeDescription}{}%
  \newcommand*{\StopTime}{%
    \TestTime=\pdfelapsedtime
    \global\advance\SummaryTime\TestTime
    \PrintTime\TimeDescription\TestTime
  }%
  \let\saved@qstest\qstest
  \let\saved@endqstest\endqstest
  \def\qstest#1#2{%
    \saved@qstest{#1}{#2}%
    \StartTime{#1}%
  }%
  \def\endqstest{%
    \StopTime
    \saved@endqstest
  }%
  \AtEndDocument{%
    \PrintTime{summary}\SummaryTime
  }%
  \makeatother
\fi
%    \end{macrocode}
%
% \subsubsection{Test sets}
%
%    \begin{macrocode}
\makeatletter

\begin{qstest}{inv}{inv}%
  \TestInv{0}{0}%
  \TestInv{1}{-1}%
  \TestInv{-1}{1}%
  \TestInv{10}{-10}%
  \TestInv{-10}{10}%
  \TestInv{2147483647}{-2147483647}%
  \TestInv{-2147483647}{2147483647}%
  \TestInv{12345678901234567890}{-12345678901234567890}%
  \TestInv{-12345678901234567890}{12345678901234567890}%
  \TestInv{ 0 }{0}%
  \TestInv{ 1 }{-1}%
  \TestInv{--1}{-1}%
  \TestInv{\number\z@}{0}%
  \TestInv{\ifx\relax\relax1\fi}{-1}%
  \TestInv{\ifx\relax\relax-\fi\ifx234\else1\fi}{1}%
\end{qstest}

\begin{qstest}{abs}{abs}%
  \TestAbs{0}{0}%
  \TestAbs{1}{1}%
  \TestAbs{-1}{1}%
  \TestAbs{10}{10}%
  \TestAbs{-10}{10}%
  \TestAbs{2147483647}{2147483647}%
  \TestAbs{-2147483647}{2147483647}%
  \TestAbs{12345678901234567890}{12345678901234567890}%
  \TestAbs{-12345678901234567890}{12345678901234567890}%
  \TestAbs{ 0 }{0}%
  \TestAbs{ 1 }{1}%
  \TestAbs{--1}{1}%
  \TestAbs{-+-+1}{1}%
  \TestAbs{00000000000}{0}%
  \TestAbs{00000001000}{1000}%
  \TestAbs{\ifx\relax\relax 0\else 1\fi}{0}%
\end{qstest}

\begin{qstest}{sign}{sign}%
  \TestSgn{0}{0}%
  \TestSgn{1}{1}%
  \TestSgn{-1}{-1}%
  \TestSgn{10}{1}%
  \TestSgn{-10}{-1}%
  \TestSgn{2147483647}{1}%
  \TestSgn{-2147483647}{-1}%
  \TestSgn{12345678901234567890}{1}%
  \TestSgn{-12345678901234567890}{-1}%
  \TestSgn{ 0 }{0}%
  \TestSgn{ 2 }{1}%
  \TestSgn{ -2 }{-1}%
  \TestSgn{--2}{1}%
  \TestSgn{\number\z@}{0}%
  \TestSgn{\number\@ne}{1}%
  \TestSgn{\number\m@ne}{-1}%
  \TestSgn{%
    -+-+\number\z@\number\z@
    \iftrue1\fi\iftrue2\fi\iftrue3\fi
  }{1}%
\end{qstest}

\begin{qstest}{min}{min}%
  \TestMin{0}{1}{0}%
  \TestMin{1}{0}{0}%
  \TestMin{-10}{-20}{-20}%
  \TestMin{ 1 }{ 2 }{1}%
  \TestMin{ 2 }{ 1 }{1}%
  \TestMin{1}{1}{1}%
  \TestMin{\number\z@}{\number\@ne}{0}%
  \TestMin{\number\@ne}{\number\m@ne}{-1}%
\end{qstest}

\begin{qstest}{max}{max}%
  \TestMax{0}{1}{1}%
  \TestMax{1}{0}{1}%
  \TestMax{-10}{-20}{-10}%
  \TestMax{ 1 }{ 2 }{2}%
  \TestMax{ 2 }{ 1 }{2}%
  \TestMax{1}{1}{1}%
  \TestMax{\number\z@}{\number\@ne}{1}%
  \TestMax{\number\@ne}{\number\m@ne}{1}%
\end{qstest}

\begin{qstest}{cmp}{cmp}%
  \TestCmp{0}{0}{0}%
  \TestCmp{-21}{17}{-1}%
  \TestCmp{3}{4}{-1}%
  \TestCmp{-10}{-10}{0}%
  \TestCmp{-10}{-11}{1}%
  \TestCmp{100}{5}{1}%
  \TestCmp{9}{10}{-1}%
  \TestCmp{10}{9}{1}%
  \TestCmp{ 3 }{ 3 }{0}%
  \TestCmp{-9}{-10}{1}%
  \TestCmp{-10}{-9}{-1}%
  \TestCmp{-3}{-3}{0}%
  \TestCmp{0}{-2}{1}%
  \TestCmp{0}{2}{-1}%
  \TestCmp{2}{0}{1}%
  \TestCmp{-2}{0}{-1}%
  \TestCmp{12}{11}{1}%
  \TestCmp{11}{12}{-1}%
  \TestCmp{2147483647}{-2147483647}{1}%
  \TestCmp{-2147483647}{2147483647}{-1}%
  \TestCmp{2147483647}{2147483647}{0}%
  \TestCmp{\number\z@}{\number\@ne}{-1}%
  \TestCmp{\number\@ne}{\number\m@ne}{1}%
  \TestCmp{ 4 }{ 5 }{-1}%
  \TestCmp{ -3 }{ -7 }{1}%
\end{qstest}

\begin{qstest}{odd}{odd}
\tracingmacros=1
  \TestOdd{0}{0}%
  \TestOdd{1}{1}%
  \TestOdd{2}{0}%
  \TestOdd{3}{1}%
  \TestOdd{14}{0}%
  \TestOdd{15}{1}%
  \TestOdd{12345678901234567896}{0}%
  \TestOdd{12345678901234567897}{1}%
\end{qstest}

\begin{qstest}{inc}{inc}%
  \TestInc{0}{1}%
  \TestInc{1}{2}%
  \TestInc{-1}{0}%
  \TestInc{10}{11}%
  \TestInc{-10}{-9}%
  \TestInc{ 3 }{4}%
  \TestInc{999}{1000}%
  \TestInc{-1000}{-999}%
  \TestInc{129}{130}%
  \TestInc{2147483646}{2147483647}%
  \TestInc{-2147483647}{-2147483646}%
  \TestInc{12345678901234567890}{12345678901234567891}%
  \TestInc{99999999999999999999}{100000000000000000000}%
  \TestInc{-12345678901234567891}{-12345678901234567890}%
  \TestInc{-100000000000000000000}{-99999999999999999999}%
\end{qstest}

\begin{qstest}{dec}{dec}%
  \TestDec{0}{-1}%
  \TestDec{1}{0}%
  \TestDec{-1}{-2}%
  \TestDec{10}{9}%
  \TestDec{-10}{-11}%
  \TestDec{1000}{999}%
  \TestDec{-999}{-1000}%
  \TestDec{130}{129}%
  \TestDec{2147483647}{2147483646}%
  \TestDec{-2147483646}{-2147483647}%
  \TestDec{12345678901234567891}{12345678901234567890}%
  \TestDec{100000000000000000000}{99999999999999999999}%
  \TestDec{-12345678901234567890}{-12345678901234567891}%
  \TestDec{-99999999999999999999}{-100000000000000000000}%
\end{qstest}

\begin{qstest}{add}{add}%
  \TestAdd{0}{0}{0}%
  \TestAdd{1}{0}{1}%
  \TestAdd{0}{1}{1}%
  \TestAdd{1}{2}{3}%
  \TestAdd{-1}{-1}{-2}%
  \TestAdd{2147483646}{1}{2147483647}%
  \TestAdd{-2147483647}{2147483647}{0}%
  \TestAdd{20}{-5}{15}%
  \TestAdd{-4}{-1}{-5}%
  \TestAdd{-1}{-4}{-5}%
  \TestAdd{-4}{1}{-3}%
  \TestAdd{-1}{4}{3}%
  \TestAdd{4}{-1}{3}%
  \TestAdd{1}{-4}{-3}%
  \TestAdd{-4}{-1}{-5}%
  \TestAdd{-1}{-4}{-5}%
  \TestAdd{ -4 }{ -1 }{-5}%
  \TestAdd{ -1 }{ -4 }{-5}%
  \TestAdd{ -4 }{ 1 }{-3}%
  \TestAdd{ -1 }{ 4 }{3}%
  \TestAdd{ 4 }{ -1 }{3}%
  \TestAdd{ 1 }{ -4 }{-3}%
  \TestAdd{ -4 }{ -1 }{-5}%
  \TestAdd{ -1 }{ -4 }{-5}%
  \TestAdd{876543210}{111111111}{987654321}%
  \TestAdd{999999999}{2}{1000000001}%
\end{qstest}

\begin{qstest}{sub}{sub}
  \TestSub{0}{0}{0}%
  \TestSub{1}{0}{1}%
  \TestSub{1}{2}{-1}%
  \TestSub{-1}{-1}{0}%
  \TestSub{2147483646}{-1}{2147483647}%
  \TestSub{-2147483647}{-2147483647}{0}%
  \TestSub{-4}{-1}{-3}%
  \TestSub{-1}{-4}{3}%
  \TestSub{-4}{1}{-5}%
  \TestSub{-1}{4}{-5}%
  \TestSub{4}{-1}{5}%
  \TestSub{1}{-4}{5}%
  \TestSub{-4}{-1}{-3}%
  \TestSub{-1}{-4}{3}%
  \TestSub{ -4 }{ -1 }{-3}%
  \TestSub{ -1 }{ -4 }{3}%
  \TestSub{ -4 }{ 1 }{-5}%
  \TestSub{ -1 }{ 4 }{-5}%
  \TestSub{ 4 }{ -1 }{5}%
  \TestSub{ 1 }{ -4 }{5}%
  \TestSub{ -4 }{ -1 }{-3}%
  \TestSub{ -1 }{ -4 }{3}%
  \TestSub{1000000000}{2}{999999998}%
  \TestSub{987654321}{111111111}{876543210}%
\end{qstest}

\begin{qstest}{shl}{shl}
  \TestShl{0}{0}%
  \TestShl{1}{2}%
  \TestShl{2}{4}%
  \TestShl{5621}{11242}%
  \TestShl{1073741823}{2147483646}%
\end{qstest}

\begin{qstest}{shr}{shr}
  \TestShr{0}{0}%
  \TestShr{1}{0}%
  \TestShr{2}{1}%
  \TestShr{3}{1}%
  \TestShr{4}{2}%
  \TestShr{5}{2}%
  \TestShr{6}{3}%
  \TestShr{7}{3}%
  \TestShr{8}{4}%
  \TestShr{9}{4}%
  \TestShr{10}{5}%
  \TestShr{11}{5}%
  \TestShr{12}{6}%
  \TestShr{13}{6}%
  \TestShr{14}{7}%
  \TestShr{15}{7}%
  \TestShr{16}{8}%
  \TestShr{17}{8}%
  \TestShr{18}{9}%
  \TestShr{19}{9}%
  \TestShr{20}{10}%
  \TestShr{21}{10}%
  \TestShr{22}{11}%
  \TestShr{11241}{5620}%
  \TestShr{73054202}{36527101}%
  \TestShr{2147483646}{1073741823}%
\end{qstest}

\begin{qstest}{mul}{mul}
  \TestMul{0}{0}{0}%
  \TestMul{1}{0}{0}%
  \TestMul{0}{1}{0}%
  \TestMul{1}{1}{1}%
  \TestMul{3}{1}{3}%
  \TestMul{1}{-3}{-3}%
  \TestMul{-4}{-5}{20}%
  \TestMul{3}{7}{21}%
  \TestMul{7}{3}{21}%
  \TestMul{3}{-7}{-21}%
  \TestMul{7}{-3}{-21}%
  \TestMul{-3}{7}{-21}%
  \TestMul{-7}{3}{-21}%
  \TestMul{-3}{-7}{21}%
  \TestMul{-7}{-3}{21}%
  \TestMul{12}{11}{132}%
  \TestMul{999}{333}{332667}%
  \TestMul{1000}{4321}{4321000}%
  \TestMul{12345}{173955}{2147474475}%
  \TestMul{1073741823}{2}{2147483646}%
  \TestMul{2}{1073741823}{2147483646}%
  \TestMul{-1073741823}{2}{-2147483646}%
  \TestMul{2}{-1073741823}{-2147483646}%
  \TestMul{6706022400}{13}{87178291200}%
\end{qstest}

\begin{qstest}{sqr}{sqr}
  \TestSqr{0}{0}%
  \TestSqr{1}{1}%
  \TestSqr{2}{4}%
  \TestSqr{3}{9}%
  \TestSqr{4}{16}%
  \TestSqr{9}{81}%
  \TestSqr{10}{100}%
  \TestSqr{46340}{2147395600}%
  \TestSqr{-1}{1}%
  \TestSqr{-2}{4}%
  \TestSqr{-46340}{2147395600}%
\end{qstest}

\begin{qstest}{fac}{fac}
  \TestFac{0}{1}%
  \TestFac{1}{1}%
  \TestFac{2}{2}%
  \TestFac{3}{2*3}%
  \TestFac{4}{2*3*4}%
  \TestFac{5}{2*3*4*5}%
  \TestFac{6}{2*3*4*5*6}%
  \TestFac{7}{2*3*4*5*6*7}%
  \TestFac{8}{2*3*4*5*6*7*8}%
  \TestFac{9}{2*3*4*5*6*7*8*9}%
  \TestFac{10}{2*3*4*5*6*7*8*9*10}%
  \TestFac{11}{2*3*4*5*6*7*8*9*10*11}%
  \TestFac{12}{2*3*4*5*6*7*8*9*10*11*12}%
  \TestFacBig{13}{6227020800}%
  \TestFacBig{14}{87178291200}%
  \TestFacBig{15}{1307674368000}%
  \TestFacBig{16}{20922789888000}%
  \TestFacBig{17}{355687428096000}%
  \TestFacBig{18}{6402373705728000}%
  \TestFacBig{19}{121645100408832000}%
  \TestFacBig{20}{2432902008176640000}%
  \TestFacBig{21}{51090942171709440000}%
  \TestFacBig{22}{1124000727777607680000}%
\end{qstest}

\begin{qstest}{pow}{pow}
  \TestPow{-2}{0}{1}%
  \TestPow{-1}{0}{1}%
  \TestPow{0}{0}{1}%
  \TestPow{1}{0}{1}%
  \TestPow{2}{0}{1}%
  \TestPow{3}{0}{1}%
  \TestPow{-2}{1}{-2}%
  \TestPow{-1}{1}{-1}%
  \TestPow{1}{1}{1}%
  \TestPow{2}{1}{2}%
  \TestPow{3}{1}{3}%
  \TestPow{-2}{2}{4}%
  \TestPow{-1}{2}{1}%
  \TestPow{0}{2}{0}%
  \TestPow{1}{2}{1}%
  \TestPow{2}{2}{4}%
  \TestPow{3}{2}{9}%
  \TestPow{0}{1}{0}%
  \TestPow{1}{-2}{1}%
  \TestPow{1}{-1}{1}%
  \TestPow{-1}{-2}{1}%
  \TestPow{-1}{-1}{-1}%
  \TestPow{-1}{3}{-1}%
  \TestPow{-1}{4}{1}%
  \TestPow{-2}{-1}{0}%
  \TestPow{-2}{-2}{0}%
  \TestPow{2}{3}{8}%
  \TestPow{2}{4}{16}%
  \TestPow{2}{5}{32}%
  \TestPow{2}{6}{64}%
  \TestPow{2}{7}{128}%
  \TestPow{2}{8}{256}%
  \TestPow{2}{9}{512}%
  \TestPow{2}{10}{1024}%
  \TestPow{-2}{3}{-8}%
  \TestPow{-2}{4}{16}%
  \TestPow{-2}{5}{-32}%
  \TestPow{-2}{6}{64}%
  \TestPow{-2}{7}{-128}%
  \TestPow{-2}{8}{256}%
  \TestPow{-2}{9}{-512}%
  \TestPow{-2}{10}{1024}%
  \TestPow{3}{3}{27}%
  \TestPow{3}{4}{81}%
  \TestPow{3}{5}{243}%
  \TestPow{-3}{3}{-27}%
  \TestPow{-3}{4}{81}%
  \TestPow{-3}{5}{-243}%
  \TestPow{2}{30}{1073741824}%
  \TestPow{-3}{19}{-1162261467}%
  \TestPow{5}{13}{1220703125}%
  \TestPow{-7}{11}{-1977326743}%
\end{qstest}

\begin{qstest}{div}{div}
  \TestDiv{1}{1}{1}%
  \TestDiv{2}{1}{2}%
  \TestDiv{-2}{1}{-2}%
  \TestDiv{2}{-1}{-2}%
  \TestDiv{-2}{-1}{2}%
  \TestDiv{15}{2}{7}%
  \TestDiv{-16}{2}{-8}%
  \TestDiv{1}{2}{0}%
  \TestDiv{1}{3}{0}%
  \TestDiv{2}{3}{0}%
  \TestDiv{-2}{3}{0}%
  \TestDiv{2}{-3}{0}%
  \TestDiv{-2}{-3}{0}%
  \TestDiv{13}{3}{4}%
  \TestDiv{-13}{-3}{4}%
  \TestDiv{-13}{3}{-4}%
  \TestDiv{-6}{5}{-1}%
  \TestDiv{-5}{5}{-1}%
  \TestDiv{-4}{5}{0}%
  \TestDiv{-3}{5}{0}%
  \TestDiv{-2}{5}{0}%
  \TestDiv{-1}{5}{0}%
  \TestDiv{0}{5}{0}%
  \TestDiv{1}{5}{0}%
  \TestDiv{2}{5}{0}%
  \TestDiv{3}{5}{0}%
  \TestDiv{4}{5}{0}%
  \TestDiv{5}{5}{1}%
  \TestDiv{6}{5}{1}%
  \TestDiv{-5}{4}{-1}%
  \TestDiv{-4}{4}{-1}%
  \TestDiv{-3}{4}{0}%
  \TestDiv{-2}{4}{0}%
  \TestDiv{-1}{4}{0}%
  \TestDiv{0}{4}{0}%
  \TestDiv{1}{4}{0}%
  \TestDiv{2}{4}{0}%
  \TestDiv{3}{4}{0}%
  \TestDiv{4}{4}{1}%
  \TestDiv{5}{4}{1}%
  \TestDiv{12345}{678}{18}%
  \TestDiv{32372}{5952}{5}%
  \TestDiv{284271294}{18162}{15651}%
  \TestDiv{217652429}{12561}{17327}%
  \TestDiv{462028434}{5439}{84947}%
  \TestDiv{2147483647}{1000}{2147483}%
  \TestDiv{2147483647}{-1000}{-2147483}%
  \TestDiv{-2147483647}{1000}{-2147483}%
  \TestDiv{-2147483647}{-1000}{2147483}%
  \TestDiv{0}{3}{0}%
  \TestDiv{1}{3}{0}%
  \TestDiv{2}{3}{0}%
  \TestDiv{3}{3}{1}%
  \TestDiv{4}{3}{1}%
  \TestDiv{5}{3}{1}%
  \TestDiv{6}{3}{2}%
  \TestDiv{7}{3}{2}%
  \TestDiv{8}{3}{2}%
  \TestDiv{9}{3}{3}%
  \TestDiv{10}{3}{3}%
  \TestDiv{11}{3}{3}%
  \TestDiv{12}{3}{4}%
  \TestDiv{13}{3}{4}%
  \TestDiv{14}{3}{4}%
  \TestDiv{15}{3}{5}%
  \TestDiv{16}{3}{5}%
  \TestDiv{17}{3}{5}%
  \TestDiv{18}{3}{6}%
  \TestDiv{19}{3}{6}%
  \TestDiv{20}{3}{6}%
  \TestDiv{21}{3}{7}%
  \TestDiv{22}{3}{7}%
  \TestDiv{23}{3}{7}%
  \TestDiv{24}{3}{8}%
  \TestDiv{25}{3}{8}%
  \TestDiv{26}{3}{8}%
  \TestDiv{27}{3}{9}%
  \TestDiv{28}{3}{9}%
  \TestDiv{29}{3}{9}%
  \TestDiv{30}{3}{10}%
  \TestDiv{31}{3}{10}%
  \TestDivBig{17363436332507}{24702}{702916214}%
\end{qstest}

\begin{qstest}{mod}{mod}
  \TestMod{-6}{5}{4}%
  \TestMod{-5}{5}{0}%
  \TestMod{-4}{5}{1}%
  \TestMod{-3}{5}{2}%
  \TestMod{-2}{5}{3}%
  \TestMod{-1}{5}{4}%
  \TestMod{0}{5}{0}%
  \TestMod{1}{5}{1}%
  \TestMod{2}{5}{2}%
  \TestMod{3}{5}{3}%
  \TestMod{4}{5}{4}%
  \TestMod{5}{5}{0}%
  \TestMod{6}{5}{1}%
  \TestMod{-5}{4}{3}%
  \TestMod{-4}{4}{0}%
  \TestMod{-3}{4}{1}%
  \TestMod{-2}{4}{2}%
  \TestMod{-1}{4}{3}%
  \TestMod{0}{4}{0}%
  \TestMod{1}{4}{1}%
  \TestMod{2}{4}{2}%
  \TestMod{3}{4}{3}%
  \TestMod{4}{4}{0}%
  \TestMod{5}{4}{1}%
  \TestMod{-6}{-5}{-1}%
  \TestMod{-5}{-5}{0}%
  \TestMod{-4}{-5}{-4}%
  \TestMod{-3}{-5}{-3}%
  \TestMod{-2}{-5}{-2}%
  \TestMod{-1}{-5}{-1}%
  \TestMod{0}{-5}{0}%
  \TestMod{1}{-5}{-4}%
  \TestMod{2}{-5}{-3}%
  \TestMod{3}{-5}{-2}%
  \TestMod{4}{-5}{-1}%
  \TestMod{5}{-5}{0}%
  \TestMod{6}{-5}{-4}%
  \TestMod{-5}{-4}{-1}%
  \TestMod{-4}{-4}{0}%
  \TestMod{-3}{-4}{-3}%
  \TestMod{-2}{-4}{-2}%
  \TestMod{-1}{-4}{-1}%
  \TestMod{0}{-4}{0}%
  \TestMod{1}{-4}{-3}%
  \TestMod{2}{-4}{-2}%
  \TestMod{3}{-4}{-1}%
  \TestMod{4}{-4}{0}%
  \TestMod{5}{-4}{-3}%
  \TestMod{2147483647}{1000}{647}%
  \TestMod{2147483647}{-1000}{-353}%
  \TestMod{-2147483647}{1000}{353}%
  \TestMod{-2147483647}{-1000}{-647}%
  \TestMod{ 0 }{ 4 }{0}%
  \TestMod{ 1 }{ 4 }{1}%
  \TestMod{ -1 }{ 4 }{3}%
  \TestMod{ 0 }{ -4 }{0}%
  \TestMod{ 1 }{ -4 }{-3}%
  \TestMod{ -1 }{ -4 }{-1}%
  \TestMod{18362}{25}{12}%
\end{qstest}

\newcommand*{\TestError}[2]{%
  \begingroup
    \expandafter\def\csname BigIntCalcError:#1\endcsname{}%
    \Expect*{#2}{0}%
    \expandafter\def\csname BigIntCalcError:#1\endcsname{ERROR}%
    \Expect*{#2}{0ERROR}%
  \endgroup
}
\begin{qstest}{error}{error}
  \TestError{FacNegative}{\bigintcalcFac{-1}}%
  \TestError{FacNegative}{\bigintcalcFac{-2147483647}}%
  \TestError{DivisionByZero}{\bigintcalcPow{0}{-1}}%
  \TestError{DivisionByZero}{\bigintcalcDiv{1}{0}}%
  \TestError{DivisionByZero}{\bigintcalcMod{1}{0}}%
\end{qstest}

\begin{document}
\end{document}
%</test2>
%    \end{macrocode}
%
% \section{Installation}
%
% \subsection{Download}
%
% \paragraph{Package.} This package is available on
% CTAN\footnote{\url{ftp://ftp.ctan.org/tex-archive/}}:
% \begin{description}
% \item[\CTAN{macros/latex/contrib/oberdiek/bigintcalc.dtx}] The source file.
% \item[\CTAN{macros/latex/contrib/oberdiek/bigintcalc.pdf}] Documentation.
% \end{description}
%
%
% \paragraph{Bundle.} All the packages of the bundle `oberdiek'
% are also available in a TDS compliant ZIP archive. There
% the packages are already unpacked and the documentation files
% are generated. The files and directories obey the TDS standard.
% \begin{description}
% \item[\CTAN{install/macros/latex/contrib/oberdiek.tds.zip}]
% \end{description}
% \emph{TDS} refers to the standard ``A Directory Structure
% for \TeX\ Files'' (\CTAN{tds/tds.pdf}). Directories
% with \xfile{texmf} in their name are usually organized this way.
%
% \subsection{Bundle installation}
%
% \paragraph{Unpacking.} Unpack the \xfile{oberdiek.tds.zip} in the
% TDS tree (also known as \xfile{texmf} tree) of your choice.
% Example (linux):
% \begin{quote}
%   |unzip oberdiek.tds.zip -d ~/texmf|
% \end{quote}
%
% \paragraph{Script installation.}
% Check the directory \xfile{TDS:scripts/oberdiek/} for
% scripts that need further installation steps.
% Package \xpackage{attachfile2} comes with the Perl script
% \xfile{pdfatfi.pl} that should be installed in such a way
% that it can be called as \texttt{pdfatfi}.
% Example (linux):
% \begin{quote}
%   |chmod +x scripts/oberdiek/pdfatfi.pl|\\
%   |cp scripts/oberdiek/pdfatfi.pl /usr/local/bin/|
% \end{quote}
%
% \subsection{Package installation}
%
% \paragraph{Unpacking.} The \xfile{.dtx} file is a self-extracting
% \docstrip\ archive. The files are extracted by running the
% \xfile{.dtx} through \plainTeX:
% \begin{quote}
%   \verb|tex bigintcalc.dtx|
% \end{quote}
%
% \paragraph{TDS.} Now the different files must be moved into
% the different directories in your installation TDS tree
% (also known as \xfile{texmf} tree):
% \begin{quote}
% \def\t{^^A
% \begin{tabular}{@{}>{\ttfamily}l@{ $\rightarrow$ }>{\ttfamily}l@{}}
%   bigintcalc.sty & tex/generic/oberdiek/bigintcalc.sty\\
%   bigintcalc.pdf & doc/latex/oberdiek/bigintcalc.pdf\\
%   test/bigintcalc-test1.tex & doc/latex/oberdiek/test/bigintcalc-test1.tex\\
%   test/bigintcalc-test2.tex & doc/latex/oberdiek/test/bigintcalc-test2.tex\\
%   test/bigintcalc-test3.tex & doc/latex/oberdiek/test/bigintcalc-test3.tex\\
%   bigintcalc.dtx & source/latex/oberdiek/bigintcalc.dtx\\
% \end{tabular}^^A
% }^^A
% \sbox0{\t}^^A
% \ifdim\wd0>\linewidth
%   \begingroup
%     \advance\linewidth by\leftmargin
%     \advance\linewidth by\rightmargin
%   \edef\x{\endgroup
%     \def\noexpand\lw{\the\linewidth}^^A
%   }\x
%   \def\lwbox{^^A
%     \leavevmode
%     \hbox to \linewidth{^^A
%       \kern-\leftmargin\relax
%       \hss
%       \usebox0
%       \hss
%       \kern-\rightmargin\relax
%     }^^A
%   }^^A
%   \ifdim\wd0>\lw
%     \sbox0{\small\t}^^A
%     \ifdim\wd0>\linewidth
%       \ifdim\wd0>\lw
%         \sbox0{\footnotesize\t}^^A
%         \ifdim\wd0>\linewidth
%           \ifdim\wd0>\lw
%             \sbox0{\scriptsize\t}^^A
%             \ifdim\wd0>\linewidth
%               \ifdim\wd0>\lw
%                 \sbox0{\tiny\t}^^A
%                 \ifdim\wd0>\linewidth
%                   \lwbox
%                 \else
%                   \usebox0
%                 \fi
%               \else
%                 \lwbox
%               \fi
%             \else
%               \usebox0
%             \fi
%           \else
%             \lwbox
%           \fi
%         \else
%           \usebox0
%         \fi
%       \else
%         \lwbox
%       \fi
%     \else
%       \usebox0
%     \fi
%   \else
%     \lwbox
%   \fi
% \else
%   \usebox0
% \fi
% \end{quote}
% If you have a \xfile{docstrip.cfg} that configures and enables \docstrip's
% TDS installing feature, then some files can already be in the right
% place, see the documentation of \docstrip.
%
% \subsection{Refresh file name databases}
%
% If your \TeX~distribution
% (\teTeX, \mikTeX, \dots) relies on file name databases, you must refresh
% these. For example, \teTeX\ users run \verb|texhash| or
% \verb|mktexlsr|.
%
% \subsection{Some details for the interested}
%
% \paragraph{Attached source.}
%
% The PDF documentation on CTAN also includes the
% \xfile{.dtx} source file. It can be extracted by
% AcrobatReader 6 or higher. Another option is \textsf{pdftk},
% e.g. unpack the file into the current directory:
% \begin{quote}
%   \verb|pdftk bigintcalc.pdf unpack_files output .|
% \end{quote}
%
% \paragraph{Unpacking with \LaTeX.}
% The \xfile{.dtx} chooses its action depending on the format:
% \begin{description}
% \item[\plainTeX:] Run \docstrip\ and extract the files.
% \item[\LaTeX:] Generate the documentation.
% \end{description}
% If you insist on using \LaTeX\ for \docstrip\ (really,
% \docstrip\ does not need \LaTeX), then inform the autodetect routine
% about your intention:
% \begin{quote}
%   \verb|latex \let\install=y% \iffalse meta-comment
%
% File: bigintcalc.dtx
% Version: 2016/05/16 v1.4
% Info: Expandable calculations on big integers
%
% Copyright (C) 2007, 2011, 2012 by
%    Heiko Oberdiek <heiko.oberdiek at googlemail.com>
%    2016
%    https://github.com/ho-tex/oberdiek/issues
%
% This work may be distributed and/or modified under the
% conditions of the LaTeX Project Public License, either
% version 1.3c of this license or (at your option) any later
% version. This version of this license is in
%    http://www.latex-project.org/lppl/lppl-1-3c.txt
% and the latest version of this license is in
%    http://www.latex-project.org/lppl.txt
% and version 1.3 or later is part of all distributions of
% LaTeX version 2005/12/01 or later.
%
% This work has the LPPL maintenance status "maintained".
%
% This Current Maintainer of this work is Heiko Oberdiek.
%
% The Base Interpreter refers to any `TeX-Format',
% because some files are installed in TDS:tex/generic//.
%
% This work consists of the main source file bigintcalc.dtx
% and the derived files
%    bigintcalc.sty, bigintcalc.pdf, bigintcalc.ins, bigintcalc.drv,
%    bigintcalc-test1.tex, bigintcalc-test2.tex,
%    bigintcalc-test3.tex.
%
% Distribution:
%    CTAN:macros/latex/contrib/oberdiek/bigintcalc.dtx
%    CTAN:macros/latex/contrib/oberdiek/bigintcalc.pdf
%
% Unpacking:
%    (a) If bigintcalc.ins is present:
%           tex bigintcalc.ins
%    (b) Without bigintcalc.ins:
%           tex bigintcalc.dtx
%    (c) If you insist on using LaTeX
%           latex \let\install=y% \iffalse meta-comment
%
% File: bigintcalc.dtx
% Version: 2016/05/16 v1.4
% Info: Expandable calculations on big integers
%
% Copyright (C) 2007, 2011, 2012 by
%    Heiko Oberdiek <heiko.oberdiek at googlemail.com>
%    2016
%    https://github.com/ho-tex/oberdiek/issues
%
% This work may be distributed and/or modified under the
% conditions of the LaTeX Project Public License, either
% version 1.3c of this license or (at your option) any later
% version. This version of this license is in
%    http://www.latex-project.org/lppl/lppl-1-3c.txt
% and the latest version of this license is in
%    http://www.latex-project.org/lppl.txt
% and version 1.3 or later is part of all distributions of
% LaTeX version 2005/12/01 or later.
%
% This work has the LPPL maintenance status "maintained".
%
% This Current Maintainer of this work is Heiko Oberdiek.
%
% The Base Interpreter refers to any `TeX-Format',
% because some files are installed in TDS:tex/generic//.
%
% This work consists of the main source file bigintcalc.dtx
% and the derived files
%    bigintcalc.sty, bigintcalc.pdf, bigintcalc.ins, bigintcalc.drv,
%    bigintcalc-test1.tex, bigintcalc-test2.tex,
%    bigintcalc-test3.tex.
%
% Distribution:
%    CTAN:macros/latex/contrib/oberdiek/bigintcalc.dtx
%    CTAN:macros/latex/contrib/oberdiek/bigintcalc.pdf
%
% Unpacking:
%    (a) If bigintcalc.ins is present:
%           tex bigintcalc.ins
%    (b) Without bigintcalc.ins:
%           tex bigintcalc.dtx
%    (c) If you insist on using LaTeX
%           latex \let\install=y% \iffalse meta-comment
%
% File: bigintcalc.dtx
% Version: 2016/05/16 v1.4
% Info: Expandable calculations on big integers
%
% Copyright (C) 2007, 2011, 2012 by
%    Heiko Oberdiek <heiko.oberdiek at googlemail.com>
%    2016
%    https://github.com/ho-tex/oberdiek/issues
%
% This work may be distributed and/or modified under the
% conditions of the LaTeX Project Public License, either
% version 1.3c of this license or (at your option) any later
% version. This version of this license is in
%    http://www.latex-project.org/lppl/lppl-1-3c.txt
% and the latest version of this license is in
%    http://www.latex-project.org/lppl.txt
% and version 1.3 or later is part of all distributions of
% LaTeX version 2005/12/01 or later.
%
% This work has the LPPL maintenance status "maintained".
%
% This Current Maintainer of this work is Heiko Oberdiek.
%
% The Base Interpreter refers to any `TeX-Format',
% because some files are installed in TDS:tex/generic//.
%
% This work consists of the main source file bigintcalc.dtx
% and the derived files
%    bigintcalc.sty, bigintcalc.pdf, bigintcalc.ins, bigintcalc.drv,
%    bigintcalc-test1.tex, bigintcalc-test2.tex,
%    bigintcalc-test3.tex.
%
% Distribution:
%    CTAN:macros/latex/contrib/oberdiek/bigintcalc.dtx
%    CTAN:macros/latex/contrib/oberdiek/bigintcalc.pdf
%
% Unpacking:
%    (a) If bigintcalc.ins is present:
%           tex bigintcalc.ins
%    (b) Without bigintcalc.ins:
%           tex bigintcalc.dtx
%    (c) If you insist on using LaTeX
%           latex \let\install=y\input{bigintcalc.dtx}
%        (quote the arguments according to the demands of your shell)
%
% Documentation:
%    (a) If bigintcalc.drv is present:
%           latex bigintcalc.drv
%    (b) Without bigintcalc.drv:
%           latex bigintcalc.dtx; ...
%    The class ltxdoc loads the configuration file ltxdoc.cfg
%    if available. Here you can specify further options, e.g.
%    use A4 as paper format:
%       \PassOptionsToClass{a4paper}{article}
%
%    Programm calls to get the documentation (example):
%       pdflatex bigintcalc.dtx
%       makeindex -s gind.ist bigintcalc.idx
%       pdflatex bigintcalc.dtx
%       makeindex -s gind.ist bigintcalc.idx
%       pdflatex bigintcalc.dtx
%
% Installation:
%    TDS:tex/generic/oberdiek/bigintcalc.sty
%    TDS:doc/latex/oberdiek/bigintcalc.pdf
%    TDS:doc/latex/oberdiek/test/bigintcalc-test1.tex
%    TDS:doc/latex/oberdiek/test/bigintcalc-test2.tex
%    TDS:doc/latex/oberdiek/test/bigintcalc-test3.tex
%    TDS:source/latex/oberdiek/bigintcalc.dtx
%
%<*ignore>
\begingroup
  \catcode123=1 %
  \catcode125=2 %
  \def\x{LaTeX2e}%
\expandafter\endgroup
\ifcase 0\ifx\install y1\fi\expandafter
         \ifx\csname processbatchFile\endcsname\relax\else1\fi
         \ifx\fmtname\x\else 1\fi\relax
\else\csname fi\endcsname
%</ignore>
%<*install>
\input docstrip.tex
\Msg{************************************************************************}
\Msg{* Installation}
\Msg{* Package: bigintcalc 2016/05/16 v1.4 Expandable calculations on big integers (HO)}
\Msg{************************************************************************}

\keepsilent
\askforoverwritefalse

\let\MetaPrefix\relax
\preamble

This is a generated file.

Project: bigintcalc
Version: 2016/05/16 v1.4

Copyright (C) 2007, 2011, 2012 by
   Heiko Oberdiek <heiko.oberdiek at googlemail.com>

This work may be distributed and/or modified under the
conditions of the LaTeX Project Public License, either
version 1.3c of this license or (at your option) any later
version. This version of this license is in
   http://www.latex-project.org/lppl/lppl-1-3c.txt
and the latest version of this license is in
   http://www.latex-project.org/lppl.txt
and version 1.3 or later is part of all distributions of
LaTeX version 2005/12/01 or later.

This work has the LPPL maintenance status "maintained".

This Current Maintainer of this work is Heiko Oberdiek.

The Base Interpreter refers to any `TeX-Format',
because some files are installed in TDS:tex/generic//.

This work consists of the main source file bigintcalc.dtx
and the derived files
   bigintcalc.sty, bigintcalc.pdf, bigintcalc.ins, bigintcalc.drv,
   bigintcalc-test1.tex, bigintcalc-test2.tex,
   bigintcalc-test3.tex.

\endpreamble
\let\MetaPrefix\DoubleperCent

\generate{%
  \file{bigintcalc.ins}{\from{bigintcalc.dtx}{install}}%
  \file{bigintcalc.drv}{\from{bigintcalc.dtx}{driver}}%
  \usedir{tex/generic/oberdiek}%
  \file{bigintcalc.sty}{\from{bigintcalc.dtx}{package}}%
%  \usedir{doc/latex/oberdiek/test}%
%  \file{bigintcalc-test1.tex}{\from{bigintcalc.dtx}{test1}}%
%  \file{bigintcalc-test2.tex}{\from{bigintcalc.dtx}{test2,etex}}%
%  \file{bigintcalc-test3.tex}{\from{bigintcalc.dtx}{test2,noetex}}%
  \nopreamble
  \nopostamble
  \usedir{source/latex/oberdiek/catalogue}%
  \file{bigintcalc.xml}{\from{bigintcalc.dtx}{catalogue}}%
}

\catcode32=13\relax% active space
\let =\space%
\Msg{************************************************************************}
\Msg{*}
\Msg{* To finish the installation you have to move the following}
\Msg{* file into a directory searched by TeX:}
\Msg{*}
\Msg{*     bigintcalc.sty}
\Msg{*}
\Msg{* To produce the documentation run the file `bigintcalc.drv'}
\Msg{* through LaTeX.}
\Msg{*}
\Msg{* Happy TeXing!}
\Msg{*}
\Msg{************************************************************************}

\endbatchfile
%</install>
%<*ignore>
\fi
%</ignore>
%<*driver>
\NeedsTeXFormat{LaTeX2e}
\ProvidesFile{bigintcalc.drv}%
  [2016/05/16 v1.4 Expandable calculations on big integers (HO)]%
\documentclass{ltxdoc}
\usepackage{wasysym}
\let\iint\relax
\let\iiint\relax
\usepackage[fleqn]{amsmath}

\DeclareMathOperator{\opInv}{Inv}
\DeclareMathOperator{\opAbs}{Abs}
\DeclareMathOperator{\opSgn}{Sgn}
\DeclareMathOperator{\opMin}{Min}
\DeclareMathOperator{\opMax}{Max}
\DeclareMathOperator{\opCmp}{Cmp}
\DeclareMathOperator{\opOdd}{Odd}
\DeclareMathOperator{\opInc}{Inc}
\DeclareMathOperator{\opDec}{Dec}
\DeclareMathOperator{\opAdd}{Add}
\DeclareMathOperator{\opSub}{Sub}
\DeclareMathOperator{\opShl}{Shl}
\DeclareMathOperator{\opShr}{Shr}
\DeclareMathOperator{\opMul}{Mul}
\DeclareMathOperator{\opSqr}{Sqr}
\DeclareMathOperator{\opFac}{Fac}
\DeclareMathOperator{\opPow}{Pow}
\DeclareMathOperator{\opDiv}{Div}
\DeclareMathOperator{\opMod}{Mod}
\DeclareMathOperator{\opInt}{Int}
\DeclareMathOperator{\opODD}{ifodd}

\newcommand*{\Def}{%
  \ensuremath{%
    \mathrel{\mathop{:}}=%
  }%
}
\newcommand*{\op}[1]{%
  \textsf{#1}%
}
\usepackage{holtxdoc}[2011/11/22]
\begin{document}
  \DocInput{bigintcalc.dtx}%
\end{document}
%</driver>
% \fi
%
%
% \CharacterTable
%  {Upper-case    \A\B\C\D\E\F\G\H\I\J\K\L\M\N\O\P\Q\R\S\T\U\V\W\X\Y\Z
%   Lower-case    \a\b\c\d\e\f\g\h\i\j\k\l\m\n\o\p\q\r\s\t\u\v\w\x\y\z
%   Digits        \0\1\2\3\4\5\6\7\8\9
%   Exclamation   \!     Double quote  \"     Hash (number) \#
%   Dollar        \$     Percent       \%     Ampersand     \&
%   Acute accent  \'     Left paren    \(     Right paren   \)
%   Asterisk      \*     Plus          \+     Comma         \,
%   Minus         \-     Point         \.     Solidus       \/
%   Colon         \:     Semicolon     \;     Less than     \<
%   Equals        \=     Greater than  \>     Question mark \?
%   Commercial at \@     Left bracket  \[     Backslash     \\
%   Right bracket \]     Circumflex    \^     Underscore    \_
%   Grave accent  \`     Left brace    \{     Vertical bar  \|
%   Right brace   \}     Tilde         \~}
%
% \GetFileInfo{bigintcalc.drv}
%
% \title{The \xpackage{bigintcalc} package}
% \date{2016/05/16 v1.4}
% \author{Heiko Oberdiek\thanks
% {Please report any issues at https://github.com/ho-tex/oberdiek/issues}\\
% \xemail{heiko.oberdiek at googlemail.com}}
%
% \maketitle
%
% \begin{abstract}
% This package provides expandable arithmetic operations
% with big integers that can exceed \TeX's number limits.
% \end{abstract}
%
% \tableofcontents
%
% \section{Documentation}
%
% \subsection{Introduction}
%
% Package \xpackage{bigintcalc} defines arithmetic operations
% that deal with big integers. Big integers can be given
% either as explicit integer number or as macro code
% that expands to an explicit number. \emph{Big} means that
% there is no limit on the size of the number. Big integers
% may exceed \TeX's range limitation of -2147483647 and 2147483647.
% Only memory issues will limit the usable range.
%
% In opposite to package \xpackage{intcalc}
% unexpandable command tokens are not supported, even if
% they are valid \TeX\ numbers like count registers or
% commands created by \cs{chardef}. Nevertheless they may
% be used, if they are prefixed by \cs{number}.
%
% Also \eTeX's \cs{numexpr} expressions are not supported
% directly in the manner of package \xpackage{intcalc}.
% However they can be given if \cs{the}\cs{numexpr}
% or \cs{number}\cs{numexpr} are used.
%
% The operations have the form of macros that take one or
% two integers as parameter and return the integer result.
% The macro name is a three letter operation name prefixed
% by the package name, e.g. \cs{bigintcalcAdd}|{10}{43}| returns
% |53|.
%
% The macros are fully expandable, exactly two expansion
% steps generate the result. Therefore the operations
% may be used nearly everywhere in \TeX, even inside
% \cs{csname}, file names,  or other
% expandable contexts.
%
% \subsection{Conditions}
%
% \subsubsection{Preconditions}
%
% \begin{itemize}
% \item
%   Arguments can be anything that expands to a number that consists
%   of optional signs and digits.
% \item
%   The arguments and return values must be sound.
%   Zero as divisor or factorials of negative numbers will cause errors.
% \end{itemize}
%
% \subsubsection{Postconditions}
%
% Additional properties of the macros apart from calculating
% a correct result (of course \smiley):
% \begin{itemize}
% \item
%   The macros are fully expandable. Thus they can
%   be used inside \cs{edef}, \cs{csname},
%   for example.
% \item
%   Furthermore exactly two expansion steps calculate the result.
% \item
%   The number consists of one optional minus sign and one or more
%   digits. The first digit is larger than zero for
%   numbers that consists of more than one digit.
%
%   In short, the number format is exactly the same as
%   \cs{number} generates, but without its range limitation.
%   And the tokens (minus sign, digits)
%   have catcode 12 (other).
% \item
%   Call by value is simulated. First the arguments are
%   converted to numbers. Then these numbers are used
%   in the calculations.
%
%   Remember that arguments
%   may contain expensive macros or \eTeX\ expressions.
%   This strategy avoids multiple evaluations of such
%   arguments.
% \end{itemize}
%
% \subsection{Error handling}
%
% Some errors are detected by the macros, example: division by zero.
% In this cases an undefined control sequence is called and causes
% a TeX error message, example: \cs{BigIntCalcError:DivisionByZero}.
% The name of the control sequence contains
% the reason for the error. The \TeX\ error may be ignored.
% Then the operation returns zero as result.
% Because the macros are supposed to work in expandible contexts.
% An traditional error message, however, is not expandable and
% would break these contexts.
%
% \subsection{Operations}
%
% Some definition equations below use the function $\opInt$
% that converts a real number to an integer. The number
% is truncated that means rounding to zero:
% \begin{gather*}
%   \opInt(x) \Def
%   \begin{cases}
%     \lfloor x\rfloor & \text{if $x\geq0$}\\
%     \lceil x\rceil & \text{otherwise}
%   \end{cases}
% \end{gather*}
%
% \subsubsection{\op{Num}}
%
% \begin{declcs}{bigintcalcNum} \M{x}
% \end{declcs}
% Macro \cs{bigintcalcNum} converts its argument to a normalized integer
% number without unnecessary leading zeros or signs.
% The result matches the regular expression:
%\begin{quote}
%\begin{verbatim}
%0|-?[1-9][0-9]*
%\end{verbatim}
%\end{quote}
%
% \subsubsection{\op{Inv}, \op{Abs}, \op{Sgn}}
%
% \begin{declcs}{bigintcalcInv} \M{x}
% \end{declcs}
% Macro \cs{bigintcalcInv} switches the sign.
% \begin{gather*}
%   \opInv(x) \Def -x
% \end{gather*}
%
% \begin{declcs}{bigintcalcAbs} \M{x}
% \end{declcs}
% Macro \cs{bigintcalcAbs} returns the absolute value
% of integer \meta{x}.
% \begin{gather*}
%   \opAbs(x) \Def \vert x\vert
% \end{gather*}
%
% \begin{declcs}{bigintcalcSgn} \M{x}
% \end{declcs}
% Macro \cs{bigintcalcSgn} encodes the sign of \meta{x} as number.
% \begin{gather*}
%   \opSgn(x) \Def
%   \begin{cases}
%     -1& \text{if $x<0$}\\
%      0& \text{if $x=0$}\\
%      1& \text{if $x>0$}
%   \end{cases}
% \end{gather*}
% These return values can easily be distinguished by \cs{ifcase}:
%\begin{quote}
%\begin{verbatim}
%\ifcase\bigintcalcSgn{<x>}
%  $x=0$
%\or
%  $x>0$
%\else
%  $x<0$
%\fi
%\end{verbatim}
%\end{quote}
%
% \subsubsection{\op{Min}, \op{Max}, \op{Cmp}}
%
% \begin{declcs}{bigintcalcMin} \M{x} \M{y}
% \end{declcs}
% Macro \cs{bigintcalcMin} returns the smaller of the two integers.
% \begin{gather*}
%   \opMin(x,y) \Def
%   \begin{cases}
%     x & \text{if $x<y$}\\
%     y & \text{otherwise}
%   \end{cases}
% \end{gather*}
%
% \begin{declcs}{bigintcalcMax} \M{x} \M{y}
% \end{declcs}
% Macro \cs{bigintcalcMax} returns the larger of the two integers.
% \begin{gather*}
%   \opMax(x,y) \Def
%   \begin{cases}
%     x & \text{if $x>y$}\\
%     y & \text{otherwise}
%   \end{cases}
% \end{gather*}
%
% \begin{declcs}{bigintcalcCmp} \M{x} \M{y}
% \end{declcs}
% Macro \cs{bigintcalcCmp} encodes the comparison result as number:
% \begin{gather*}
%   \opCmp(x,y) \Def
%   \begin{cases}
%     -1 & \text{if $x<y$}\\
%      0 & \text{if $x=y$}\\
%      1 & \text{if $x>y$}
%   \end{cases}
% \end{gather*}
% These values can be distinguished by \cs{ifcase}:
%\begin{quote}
%\begin{verbatim}
%\ifcase\bigintcalcCmp{<x>}{<y>}
%  $x=y$
%\or
%  $x>y$
%\else
%  $x<y$
%\fi
%\end{verbatim}
%\end{quote}
%
% \subsubsection{\op{Odd}}
%
% \begin{declcs}{bigintcalcOdd} \M{x}
% \end{declcs}
% \begin{gather*}
%   \opOdd(x) \Def
%   \begin{cases}
%     1 & \text{if $x$ is odd}\\
%     0 & \text{if $x$ is even}
%   \end{cases}
% \end{gather*}
%
% \subsubsection{\op{Inc}, \op{Dec}, \op{Add}, \op{Sub}}
%
% \begin{declcs}{bigintcalcInc} \M{x}
% \end{declcs}
% Macro \cs{bigintcalcInc} increments \meta{x} by one.
% \begin{gather*}
%   \opInc(x) \Def x + 1
% \end{gather*}
%
% \begin{declcs}{bigintcalcDec} \M{x}
% \end{declcs}
% Macro \cs{bigintcalcDec} decrements \meta{x} by one.
% \begin{gather*}
%   \opDec(x) \Def x - 1
% \end{gather*}
%
% \begin{declcs}{bigintcalcAdd} \M{x} \M{y}
% \end{declcs}
% Macro \cs{bigintcalcAdd} adds the two numbers.
% \begin{gather*}
%   \opAdd(x, y) \Def x + y
% \end{gather*}
%
% \begin{declcs}{bigintcalcSub} \M{x} \M{y}
% \end{declcs}
% Macro \cs{bigintcalcSub} calculates the difference.
% \begin{gather*}
%   \opSub(x, y) \Def x - y
% \end{gather*}
%
% \subsubsection{\op{Shl}, \op{Shr}}
%
% \begin{declcs}{bigintcalcShl} \M{x}
% \end{declcs}
% Macro \cs{bigintcalcShl} implements shifting to the left that
% means the number is multiplied by two.
% The sign is preserved.
% \begin{gather*}
%   \opShl(x) \Def x*2
% \end{gather*}
%
% \begin{declcs}{bigintcalcShr} \M{x}
% \end{declcs}
% Macro \cs{bigintcalcShr} implements shifting to the right.
% That is equivalent to an integer division by two.
% The sign is preserved.
% \begin{gather*}
%   \opShr(x) \Def \opInt(x/2)
% \end{gather*}
%
% \subsubsection{\op{Mul}, \op{Sqr}, \op{Fac}, \op{Pow}}
%
% \begin{declcs}{bigintcalcMul} \M{x} \M{y}
% \end{declcs}
% Macro \cs{bigintcalcMul} calculates the product of
% \meta{x} and \meta{y}.
% \begin{gather*}
%   \opMul(x,y) \Def x*y
% \end{gather*}
%
% \begin{declcs}{bigintcalcSqr} \M{x}
% \end{declcs}
% Macro \cs{bigintcalcSqr} returns the square product.
% \begin{gather*}
%   \opSqr(x) \Def x^2
% \end{gather*}
%
% \begin{declcs}{bigintcalcFac} \M{x}
% \end{declcs}
% Macro \cs{bigintcalcFac} returns the factorial of \meta{x}.
% Negative numbers are not permitted.
% \begin{gather*}
%   \opFac(x) \Def x!\qquad\text{for $x\geq0$}
% \end{gather*}
% ($0! = 1$)
%
% \begin{declcs}{bigintcalcPow} M{x} M{y}
% \end{declcs}
% Macro \cs{bigintcalcPow} calculates the value of \meta{x} to the
% power of \meta{y}. The error ``division by zero'' is thrown
% if \meta{x} is zero and \meta{y} is negative.
% permitted:
% \begin{gather*}
%   \opPow(x,y) \Def
%   \opInt(x^y)\qquad\text{for $x\neq0$ or $y\geq0$}
% \end{gather*}
% ($0^0 = 1$)
%
% \subsubsection{\op{Div}, \op{Mul}}
%
% \begin{declcs}{bigintcalcDiv} \M{x} \M{y}
% \end{declcs}
% Macro \cs{bigintcalcDiv} performs an integer division.
% Argument \meta{y} must not be zero.
% \begin{gather*}
%   \opDiv(x,y) \Def \opInt(x/y)\qquad\text{for $y\neq0$}
% \end{gather*}
%
% \begin{declcs}{bigintcalcMod} \M{x} \M{y}
% \end{declcs}
% Macro \cs{bigintcalcMod} gets the remainder of the integer
% division. The sign follows the divisor \meta{y}.
% Argument \meta{y} must not be zero.
% \begin{gather*}
%   \opMod(x,y) \Def x\mathrel{\%}y\qquad\text{for $y\neq0$}
% \end{gather*}
% The result ranges:
% \begin{gather*}
%   -\vert y\vert < \opMod(x,y) \leq 0\qquad\text{for $y<0$}\\
%   0 \leq \opMod(x,y) < y\qquad\text{for $y\geq0$}
% \end{gather*}
%
% \subsection{Interface for programmers}
%
% If the programmer can ensure some more properties about
% the arguments of the operations, then the following
% macros are a little more efficient.
%
% In general numbers must obey the following constraints:
% \begin{itemize}
% \item Plain number: digit tokens only, no command tokens.
% \item Non-negative. Signs are forbidden.
% \item Delimited by exclamation mark. Curly braces
%       around the number are not allowed and will
%       break the code.
% \end{itemize}
%
% \begin{declcs}{BigIntCalcOdd} \meta{number} |!|
% \end{declcs}
% |1|/|0| is returned if \meta{number} is odd/even.
%
% \begin{declcs}{BigIntCalcInc} \meta{number} |!|
% \end{declcs}
% Incrementation.
%
% \begin{declcs}{BigIntCalcDec} \meta{number} |!|
% \end{declcs}
% Decrementation, positive number without zero.
%
% \begin{declcs}{BigIntCalcAdd} \meta{number A} |!| \meta{number B} |!|
% \end{declcs}
% Addition, $A\geq B$.
%
% \begin{declcs}{BigIntCalcSub} \meta{number A} |!| \meta{number B} |!|
% \end{declcs}
% Subtraction, $A\geq B$.
%
% \begin{declcs}{BigIntCalcShl} \meta{number} |!|
% \end{declcs}
% Left shift (multiplication with two).
%
% \begin{declcs}{BigIntCalcShr} \meta{number} |!|
% \end{declcs}
% Right shift (integer division by two).
%
% \begin{declcs}{BigIntCalcMul} \meta{number A} |!| \meta{number B} |!|
% \end{declcs}
% Multiplication, $A\geq B$.
%
% \begin{declcs}{BigIntCalcDiv} \meta{number A} |!| \meta{number B} |!|
% \end{declcs}
% Division operation.
%
% \begin{declcs}{BigIntCalcMod} \meta{number A} |!| \meta{number B} |!|
% \end{declcs}
% Modulo operation.
%
%
% \StopEventually{
% }
%
% \section{Implementation}
%
%    \begin{macrocode}
%<*package>
%    \end{macrocode}
%
% \subsection{Reload check and package identification}
%    Reload check, especially if the package is not used with \LaTeX.
%    \begin{macrocode}
\begingroup\catcode61\catcode48\catcode32=10\relax%
  \catcode13=5 % ^^M
  \endlinechar=13 %
  \catcode35=6 % #
  \catcode39=12 % '
  \catcode44=12 % ,
  \catcode45=12 % -
  \catcode46=12 % .
  \catcode58=12 % :
  \catcode64=11 % @
  \catcode123=1 % {
  \catcode125=2 % }
  \expandafter\let\expandafter\x\csname ver@bigintcalc.sty\endcsname
  \ifx\x\relax % plain-TeX, first loading
  \else
    \def\empty{}%
    \ifx\x\empty % LaTeX, first loading,
      % variable is initialized, but \ProvidesPackage not yet seen
    \else
      \expandafter\ifx\csname PackageInfo\endcsname\relax
        \def\x#1#2{%
          \immediate\write-1{Package #1 Info: #2.}%
        }%
      \else
        \def\x#1#2{\PackageInfo{#1}{#2, stopped}}%
      \fi
      \x{bigintcalc}{The package is already loaded}%
      \aftergroup\endinput
    \fi
  \fi
\endgroup%
%    \end{macrocode}
%    Package identification:
%    \begin{macrocode}
\begingroup\catcode61\catcode48\catcode32=10\relax%
  \catcode13=5 % ^^M
  \endlinechar=13 %
  \catcode35=6 % #
  \catcode39=12 % '
  \catcode40=12 % (
  \catcode41=12 % )
  \catcode44=12 % ,
  \catcode45=12 % -
  \catcode46=12 % .
  \catcode47=12 % /
  \catcode58=12 % :
  \catcode64=11 % @
  \catcode91=12 % [
  \catcode93=12 % ]
  \catcode123=1 % {
  \catcode125=2 % }
  \expandafter\ifx\csname ProvidesPackage\endcsname\relax
    \def\x#1#2#3[#4]{\endgroup
      \immediate\write-1{Package: #3 #4}%
      \xdef#1{#4}%
    }%
  \else
    \def\x#1#2[#3]{\endgroup
      #2[{#3}]%
      \ifx#1\@undefined
        \xdef#1{#3}%
      \fi
      \ifx#1\relax
        \xdef#1{#3}%
      \fi
    }%
  \fi
\expandafter\x\csname ver@bigintcalc.sty\endcsname
\ProvidesPackage{bigintcalc}%
  [2016/05/16 v1.4 Expandable calculations on big integers (HO)]%
%    \end{macrocode}
%
% \subsection{Catcodes}
%
%    \begin{macrocode}
\begingroup\catcode61\catcode48\catcode32=10\relax%
  \catcode13=5 % ^^M
  \endlinechar=13 %
  \catcode123=1 % {
  \catcode125=2 % }
  \catcode64=11 % @
  \def\x{\endgroup
    \expandafter\edef\csname BIC@AtEnd\endcsname{%
      \endlinechar=\the\endlinechar\relax
      \catcode13=\the\catcode13\relax
      \catcode32=\the\catcode32\relax
      \catcode35=\the\catcode35\relax
      \catcode61=\the\catcode61\relax
      \catcode64=\the\catcode64\relax
      \catcode123=\the\catcode123\relax
      \catcode125=\the\catcode125\relax
    }%
  }%
\x\catcode61\catcode48\catcode32=10\relax%
\catcode13=5 % ^^M
\endlinechar=13 %
\catcode35=6 % #
\catcode64=11 % @
\catcode123=1 % {
\catcode125=2 % }
\def\TMP@EnsureCode#1#2{%
  \edef\BIC@AtEnd{%
    \BIC@AtEnd
    \catcode#1=\the\catcode#1\relax
  }%
  \catcode#1=#2\relax
}
\TMP@EnsureCode{33}{12}% !
\TMP@EnsureCode{36}{14}% $ (comment!)
\TMP@EnsureCode{38}{14}% & (comment!)
\TMP@EnsureCode{40}{12}% (
\TMP@EnsureCode{41}{12}% )
\TMP@EnsureCode{42}{12}% *
\TMP@EnsureCode{43}{12}% +
\TMP@EnsureCode{45}{12}% -
\TMP@EnsureCode{46}{12}% .
\TMP@EnsureCode{47}{12}% /
\TMP@EnsureCode{58}{11}% : (letter!)
\TMP@EnsureCode{60}{12}% <
\TMP@EnsureCode{62}{12}% >
\TMP@EnsureCode{63}{14}% ? (comment!)
\TMP@EnsureCode{91}{12}% [
\TMP@EnsureCode{93}{12}% ]
\edef\BIC@AtEnd{\BIC@AtEnd\noexpand\endinput}
\begingroup\expandafter\expandafter\expandafter\endgroup
\expandafter\ifx\csname BIC@TestMode\endcsname\relax
\else
  \catcode63=9 % ? (ignore)
\fi
? \let\BIC@@TestMode\BIC@TestMode
%    \end{macrocode}
%
% \subsection{\eTeX\ detection}
%
%    \begin{macrocode}
\begingroup\expandafter\expandafter\expandafter\endgroup
\expandafter\ifx\csname numexpr\endcsname\relax
  \catcode36=9 % $ (ignore)
\else
  \catcode38=9 % & (ignore)
\fi
%    \end{macrocode}
%
% \subsection{Help macros}
%
%    \begin{macro}{\BIC@Fi}
%    \begin{macrocode}
\let\BIC@Fi\fi
%    \end{macrocode}
%    \end{macro}
%    \begin{macro}{\BIC@AfterFi}
%    \begin{macrocode}
\def\BIC@AfterFi#1#2\BIC@Fi{\fi#1}%
%    \end{macrocode}
%    \end{macro}
%    \begin{macro}{\BIC@AfterFiFi}
%    \begin{macrocode}
\def\BIC@AfterFiFi#1#2\BIC@Fi{\fi\fi#1}%
%    \end{macrocode}
%    \end{macro}
%    \begin{macro}{\BIC@AfterFiFiFi}
%    \begin{macrocode}
\def\BIC@AfterFiFiFi#1#2\BIC@Fi{\fi\fi\fi#1}%
%    \end{macrocode}
%    \end{macro}
%
%    \begin{macro}{\BIC@Space}
%    \begin{macrocode}
\begingroup
  \def\x#1{\endgroup
    \let\BIC@Space= #1%
  }%
\x{ }
%    \end{macrocode}
%    \end{macro}
%
% \subsection{Expand number}
%
%    \begin{macrocode}
\begingroup\expandafter\expandafter\expandafter\endgroup
\expandafter\ifx\csname RequirePackage\endcsname\relax
  \def\TMP@RequirePackage#1[#2]{%
    \begingroup\expandafter\expandafter\expandafter\endgroup
    \expandafter\ifx\csname ver@#1.sty\endcsname\relax
      \input #1.sty\relax
    \fi
  }%
  \TMP@RequirePackage{pdftexcmds}[2007/11/11]%
\else
  \RequirePackage{pdftexcmds}[2007/11/11]%
\fi
%    \end{macrocode}
%
%    \begin{macrocode}
\begingroup\expandafter\expandafter\expandafter\endgroup
\expandafter\ifx\csname pdf@escapehex\endcsname\relax
%    \end{macrocode}
%
%    \begin{macro}{\BIC@Expand}
%    \begin{macrocode}
  \def\BIC@Expand#1{%
    \romannumeral0%
    \BIC@@Expand#1!\@nil{}%
  }%
%    \end{macrocode}
%    \end{macro}
%    \begin{macro}{\BIC@@Expand}
%    \begin{macrocode}
  \def\BIC@@Expand#1#2\@nil#3{%
    \expandafter\ifcat\noexpand#1\relax
      \expandafter\@firstoftwo
    \else
      \expandafter\@secondoftwo
    \fi
    {%
      \expandafter\BIC@@Expand#1#2\@nil{#3}%
    }{%
      \ifx#1!%
        \expandafter\@firstoftwo
      \else
        \expandafter\@secondoftwo
      \fi
      { #3}{%
        \BIC@@Expand#2\@nil{#3#1}%
      }%
    }%
  }%
%    \end{macrocode}
%    \end{macro}
%    \begin{macro}{\@firstoftwo}
%    \begin{macrocode}
  \expandafter\ifx\csname @firstoftwo\endcsname\relax
    \long\def\@firstoftwo#1#2{#1}%
  \fi
%    \end{macrocode}
%    \end{macro}
%    \begin{macro}{\@secondoftwo}
%    \begin{macrocode}
  \expandafter\ifx\csname @secondoftwo\endcsname\relax
    \long\def\@secondoftwo#1#2{#2}%
  \fi
%    \end{macrocode}
%    \end{macro}
%
%    \begin{macrocode}
\else
%    \end{macrocode}
%    \begin{macro}{\BIC@Expand}
%    \begin{macrocode}
  \def\BIC@Expand#1{%
    \romannumeral0\expandafter\expandafter\expandafter\BIC@Space
    \pdf@unescapehex{%
      \expandafter\expandafter\expandafter
      \BIC@StripHexSpace\pdf@escapehex{#1}20\@nil
    }%
  }%
%    \end{macrocode}
%    \end{macro}
%    \begin{macro}{\BIC@StripHexSpace}
%    \begin{macrocode}
  \def\BIC@StripHexSpace#120#2\@nil{%
    #1%
    \ifx\\#2\\%
    \else
      \BIC@AfterFi{%
        \BIC@StripHexSpace#2\@nil
      }%
    \BIC@Fi
  }%
%    \end{macrocode}
%    \end{macro}
%    \begin{macrocode}
\fi
%    \end{macrocode}
%
% \subsection{Normalize expanded number}
%
%    \begin{macro}{\BIC@Normalize}
%    |#1|: result sign\\
%    |#2|: first token of number
%    \begin{macrocode}
\def\BIC@Normalize#1#2{%
  \ifx#2-%
    \ifx\\#1\\%
      \BIC@AfterFiFi{%
        \BIC@Normalize-%
      }%
    \else
      \BIC@AfterFiFi{%
        \BIC@Normalize{}%
      }%
    \fi
  \else
    \ifx#2+%
      \BIC@AfterFiFi{%
        \BIC@Normalize{#1}%
      }%
    \else
      \ifx#20%
        \BIC@AfterFiFiFi{%
          \BIC@NormalizeZero{#1}%
        }%
      \else
        \BIC@AfterFiFiFi{%
          \BIC@NormalizeDigits#1#2%
        }%
      \fi
    \fi
  \BIC@Fi
}
%    \end{macrocode}
%    \end{macro}
%    \begin{macro}{\BIC@NormalizeZero}
%    \begin{macrocode}
\def\BIC@NormalizeZero#1#2{%
  \ifx#2!%
    \BIC@AfterFi{ 0}%
  \else
    \ifx#20%
      \BIC@AfterFiFi{%
        \BIC@NormalizeZero{#1}%
      }%
    \else
      \BIC@AfterFiFi{%
        \BIC@NormalizeDigits#1#2%
      }%
    \fi
  \BIC@Fi
}
%    \end{macrocode}
%    \end{macro}
%    \begin{macro}{\BIC@NormalizeDigits}
%    \begin{macrocode}
\def\BIC@NormalizeDigits#1!{ #1}
%    \end{macrocode}
%    \end{macro}
%
% \subsection{\op{Num}}
%
%    \begin{macro}{\bigintcalcNum}
%    \begin{macrocode}
\def\bigintcalcNum#1{%
  \romannumeral0%
  \expandafter\expandafter\expandafter\BIC@Normalize
  \expandafter\expandafter\expandafter{%
  \expandafter\expandafter\expandafter}%
  \BIC@Expand{#1}!%
}
%    \end{macrocode}
%    \end{macro}
%
% \subsection{\op{Inv}, \op{Abs}, \op{Sgn}}
%
%
%    \begin{macro}{\bigintcalcInv}
%    \begin{macrocode}
\def\bigintcalcInv#1{%
  \romannumeral0\expandafter\expandafter\expandafter\BIC@Space
  \bigintcalcNum{-#1}%
}
%    \end{macrocode}
%    \end{macro}
%
%    \begin{macro}{\bigintcalcAbs}
%    \begin{macrocode}
\def\bigintcalcAbs#1{%
  \romannumeral0%
  \expandafter\expandafter\expandafter\BIC@Abs
  \bigintcalcNum{#1}%
}
%    \end{macrocode}
%    \end{macro}
%    \begin{macro}{\BIC@Abs}
%    \begin{macrocode}
\def\BIC@Abs#1{%
  \ifx#1-%
    \expandafter\BIC@Space
  \else
    \expandafter\BIC@Space
    \expandafter#1%
  \fi
}
%    \end{macrocode}
%    \end{macro}
%
%    \begin{macro}{\bigintcalcSgn}
%    \begin{macrocode}
\def\bigintcalcSgn#1{%
  \number
  \expandafter\expandafter\expandafter\BIC@Sgn
  \bigintcalcNum{#1}! %
}
%    \end{macrocode}
%    \end{macro}
%    \begin{macro}{\BIC@Sgn}
%    \begin{macrocode}
\def\BIC@Sgn#1#2!{%
  \ifx#1-%
    -1%
  \else
    \ifx#10%
      0%
    \else
      1%
    \fi
  \fi
}
%    \end{macrocode}
%    \end{macro}
%
% \subsection{\op{Cmp}, \op{Min}, \op{Max}}
%
%    \begin{macro}{\bigintcalcCmp}
%    \begin{macrocode}
\def\bigintcalcCmp#1#2{%
  \number
  \expandafter\expandafter\expandafter\BIC@Cmp
  \bigintcalcNum{#2}!{#1}%
}
%    \end{macrocode}
%    \end{macro}
%    \begin{macro}{\BIC@Cmp}
%    \begin{macrocode}
\def\BIC@Cmp#1!#2{%
  \expandafter\expandafter\expandafter\BIC@@Cmp
  \bigintcalcNum{#2}!#1!%
}
%    \end{macrocode}
%    \end{macro}
%    \begin{macro}{\BIC@@Cmp}
%    \begin{macrocode}
\def\BIC@@Cmp#1#2!#3#4!{%
  \ifx#1-%
    \ifx#3-%
      \BIC@AfterFiFi{%
        \BIC@@Cmp#4!#2!%
      }%
    \else
      \BIC@AfterFiFi{%
        -1 %
      }%
    \fi
  \else
    \ifx#3-%
      \BIC@AfterFiFi{%
        1 %
      }%
    \else
      \BIC@AfterFiFi{%
        \BIC@CmpLength#1#2!#3#4!#1#2!#3#4!%
      }%
    \fi
  \BIC@Fi
}
%    \end{macrocode}
%    \end{macro}
%    \begin{macro}{\BIC@PosCmp}
%    \begin{macrocode}
\def\BIC@PosCmp#1!#2!{%
  \BIC@CmpLength#1!#2!#1!#2!%
}
%    \end{macrocode}
%    \end{macro}
%    \begin{macro}{\BIC@CmpLength}
%    \begin{macrocode}
\def\BIC@CmpLength#1#2!#3#4!{%
  \ifx\\#2\\%
    \ifx\\#4\\%
      \BIC@AfterFiFi\BIC@CmpDiff
    \else
      \BIC@AfterFiFi{%
        \BIC@CmpResult{-1}%
      }%
    \fi
  \else
    \ifx\\#4\\%
      \BIC@AfterFiFi{%
        \BIC@CmpResult1%
      }%
    \else
      \BIC@AfterFiFi{%
        \BIC@CmpLength#2!#4!%
      }%
    \fi
  \BIC@Fi
}
%    \end{macrocode}
%    \end{macro}
%    \begin{macro}{\BIC@CmpResult}
%    \begin{macrocode}
\def\BIC@CmpResult#1#2!#3!{#1 }
%    \end{macrocode}
%    \end{macro}
%    \begin{macro}{\BIC@CmpDiff}
%    \begin{macrocode}
\def\BIC@CmpDiff#1#2!#3#4!{%
  \ifnum#1<#3 %
    \BIC@AfterFi{%
      -1 %
    }%
  \else
    \ifnum#1>#3 %
      \BIC@AfterFiFi{%
        1 %
      }%
    \else
      \ifx\\#2\\%
        \BIC@AfterFiFiFi{%
          0 %
        }%
      \else
        \BIC@AfterFiFiFi{%
          \BIC@CmpDiff#2!#4!%
        }%
      \fi
    \fi
  \BIC@Fi
}
%    \end{macrocode}
%    \end{macro}
%
%    \begin{macro}{\bigintcalcMin}
%    \begin{macrocode}
\def\bigintcalcMin#1{%
  \romannumeral0%
  \expandafter\expandafter\expandafter\BIC@MinMax
  \bigintcalcNum{#1}!-!%
}
%    \end{macrocode}
%    \end{macro}
%    \begin{macro}{\bigintcalcMax}
%    \begin{macrocode}
\def\bigintcalcMax#1{%
  \romannumeral0%
  \expandafter\expandafter\expandafter\BIC@MinMax
  \bigintcalcNum{#1}!!%
}
%    \end{macrocode}
%    \end{macro}
%    \begin{macro}{\BIC@MinMax}
%    |#1|: $x$\\
%    |#2|: sign for comparison\\
%    |#3|: $y$
%    \begin{macrocode}
\def\BIC@MinMax#1!#2!#3{%
  \expandafter\expandafter\expandafter\BIC@@MinMax
  \bigintcalcNum{#3}!#1!#2!%
}
%    \end{macrocode}
%    \end{macro}
%    \begin{macro}{\BIC@@MinMax}
%    |#1|: $y$\\
%    |#2|: $x$\\
%    |#3|: sign for comparison
%    \begin{macrocode}
\def\BIC@@MinMax#1!#2!#3!{%
  \ifnum\BIC@@Cmp#1!#2!=#31 %
    \BIC@AfterFi{ #1}%
  \else
    \BIC@AfterFi{ #2}%
  \BIC@Fi
}
%    \end{macrocode}
%    \end{macro}
%
% \subsection{\op{Odd}}
%
%    \begin{macro}{\bigintcalcOdd}
%    \begin{macrocode}
\def\bigintcalcOdd#1{%
  \romannumeral0%
  \expandafter\expandafter\expandafter\BIC@Odd
  \bigintcalcAbs{#1}!%
}
%    \end{macrocode}
%    \end{macro}
%    \begin{macro}{\BigIntCalcOdd}
%    \begin{macrocode}
\def\BigIntCalcOdd#1!{%
  \romannumeral0%
  \BIC@Odd#1!%
}
%    \end{macrocode}
%    \end{macro}
%    \begin{macro}{\BIC@Odd}
%    |#1|: $x$
%    \begin{macrocode}
\def\BIC@Odd#1#2{%
  \ifx#2!%
    \ifodd#1 %
      \BIC@AfterFiFi{ 1}%
    \else
      \BIC@AfterFiFi{ 0}%
    \fi
  \else
    \expandafter\BIC@Odd\expandafter#2%
  \BIC@Fi
}
%    \end{macrocode}
%    \end{macro}
%
% \subsection{\op{Inc}, \op{Dec}}
%
%    \begin{macro}{\bigintcalcInc}
%    \begin{macrocode}
\def\bigintcalcInc#1{%
  \romannumeral0%
  \expandafter\expandafter\expandafter\BIC@IncSwitch
  \bigintcalcNum{#1}!%
}
%    \end{macrocode}
%    \end{macro}
%    \begin{macro}{\BIC@IncSwitch}
%    \begin{macrocode}
\def\BIC@IncSwitch#1#2!{%
  \ifcase\BIC@@Cmp#1#2!-1!%
    \BIC@AfterFi{ 0}%
  \or
    \BIC@AfterFi{%
      \BIC@Inc#1#2!{}%
    }%
  \else
    \BIC@AfterFi{%
      \expandafter-\romannumeral0%
      \BIC@Dec#2!{}%
    }%
  \BIC@Fi
}
%    \end{macrocode}
%    \end{macro}
%
%    \begin{macro}{\bigintcalcDec}
%    \begin{macrocode}
\def\bigintcalcDec#1{%
  \romannumeral0%
  \expandafter\expandafter\expandafter\BIC@DecSwitch
  \bigintcalcNum{#1}!%
}
%    \end{macrocode}
%    \end{macro}
%    \begin{macro}{\BIC@DecSwitch}
%    \begin{macrocode}
\def\BIC@DecSwitch#1#2!{%
  \ifcase\BIC@Sgn#1#2! %
    \BIC@AfterFi{ -1}%
  \or
    \BIC@AfterFi{%
      \BIC@Dec#1#2!{}%
    }%
  \else
    \BIC@AfterFi{%
      \expandafter-\romannumeral0%
      \BIC@Inc#2!{}%
    }%
  \BIC@Fi
}
%    \end{macrocode}
%    \end{macro}
%
%    \begin{macro}{\BigIntCalcInc}
%    \begin{macrocode}
\def\BigIntCalcInc#1!{%
  \romannumeral0\BIC@Inc#1!{}%
}
%    \end{macrocode}
%    \end{macro}
%    \begin{macro}{\BigIntCalcDec}
%    \begin{macrocode}
\def\BigIntCalcDec#1!{%
  \romannumeral0\BIC@Dec#1!{}%
}
%    \end{macrocode}
%    \end{macro}
%
%    \begin{macro}{\BIC@Inc}
%    \begin{macrocode}
\def\BIC@Inc#1#2!#3{%
  \ifx\\#2\\%
    \BIC@AfterFi{%
      \BIC@@Inc1#1#3!{}%
    }%
  \else
    \BIC@AfterFi{%
      \BIC@Inc#2!{#1#3}%
    }%
  \BIC@Fi
}
%    \end{macrocode}
%    \end{macro}
%    \begin{macro}{\BIC@@Inc}
%    \begin{macrocode}
\def\BIC@@Inc#1#2#3!#4{%
  \ifcase#1 %
    \ifx\\#3\\%
      \BIC@AfterFiFi{ #2#4}%
    \else
      \BIC@AfterFiFi{%
        \BIC@@Inc0#3!{#2#4}%
      }%
    \fi
  \else
    \ifnum#2<9 %
      \BIC@AfterFiFi{%
&       \expandafter\BIC@@@Inc\the\numexpr#2+1\relax
$       \expandafter\expandafter\expandafter\BIC@@@Inc
$       \ifcase#2 \expandafter1%
$       \or\expandafter2%
$       \or\expandafter3%
$       \or\expandafter4%
$       \or\expandafter5%
$       \or\expandafter6%
$       \or\expandafter7%
$       \or\expandafter8%
$       \or\expandafter9%
$?      \else\BigIntCalcError:ThisCannotHappen%
$       \fi
        0#3!{#4}%
      }%
    \else
      \BIC@AfterFiFi{%
        \BIC@@@Inc01#3!{#4}%
      }%
    \fi
  \BIC@Fi
}
%    \end{macrocode}
%    \end{macro}
%    \begin{macro}{\BIC@@@Inc}
%    \begin{macrocode}
\def\BIC@@@Inc#1#2#3!#4{%
  \ifx\\#3\\%
    \ifnum#2=1 %
      \BIC@AfterFiFi{ 1#1#4}%
    \else
      \BIC@AfterFiFi{ #1#4}%
    \fi
  \else
    \BIC@AfterFi{%
      \BIC@@Inc#2#3!{#1#4}%
    }%
  \BIC@Fi
}
%    \end{macrocode}
%    \end{macro}
%
%    \begin{macro}{\BIC@Dec}
%    \begin{macrocode}
\def\BIC@Dec#1#2!#3{%
  \ifx\\#2\\%
    \BIC@AfterFi{%
      \BIC@@Dec1#1#3!{}%
    }%
  \else
    \BIC@AfterFi{%
      \BIC@Dec#2!{#1#3}%
    }%
  \BIC@Fi
}
%    \end{macrocode}
%    \end{macro}
%    \begin{macro}{\BIC@@Dec}
%    \begin{macrocode}
\def\BIC@@Dec#1#2#3!#4{%
  \ifcase#1 %
    \ifx\\#3\\%
      \BIC@AfterFiFi{ #2#4}%
    \else
      \BIC@AfterFiFi{%
        \BIC@@Dec0#3!{#2#4}%
      }%
    \fi
  \else
    \ifnum#2>0 %
      \BIC@AfterFiFi{%
&       \expandafter\BIC@@@Dec\the\numexpr#2-1\relax
$       \expandafter\expandafter\expandafter\BIC@@@Dec
$       \ifcase#2
$?        \BigIntCalcError:ThisCannotHappen%
$       \or\expandafter0%
$       \or\expandafter1%
$       \or\expandafter2%
$       \or\expandafter3%
$       \or\expandafter4%
$       \or\expandafter5%
$       \or\expandafter6%
$       \or\expandafter7%
$       \or\expandafter8%
$?      \else\BigIntCalcError:ThisCannotHappen%
$       \fi
        0#3!{#4}%
      }%
    \else
      \BIC@AfterFiFi{%
        \BIC@@@Dec91#3!{#4}%
      }%
    \fi
  \BIC@Fi
}
%    \end{macrocode}
%    \end{macro}
%    \begin{macro}{\BIC@@@Dec}
%    \begin{macrocode}
\def\BIC@@@Dec#1#2#3!#4{%
  \ifx\\#3\\%
    \ifcase#1 %
      \ifx\\#4\\%
        \BIC@AfterFiFiFi{ 0}%
      \else
        \BIC@AfterFiFiFi{ #4}%
      \fi
    \else
      \BIC@AfterFiFi{ #1#4}%
    \fi
  \else
    \BIC@AfterFi{%
      \BIC@@Dec#2#3!{#1#4}%
    }%
  \BIC@Fi
}
%    \end{macrocode}
%    \end{macro}
%
% \subsection{\op{Add}, \op{Sub}}
%
%    \begin{macro}{\bigintcalcAdd}
%    \begin{macrocode}
\def\bigintcalcAdd#1{%
  \romannumeral0%
  \expandafter\expandafter\expandafter\BIC@Add
  \bigintcalcNum{#1}!%
}
%    \end{macrocode}
%    \end{macro}
%    \begin{macro}{\BIC@Add}
%    \begin{macrocode}
\def\BIC@Add#1!#2{%
  \expandafter\expandafter\expandafter
  \BIC@AddSwitch\bigintcalcNum{#2}!#1!%
}
%    \end{macrocode}
%    \end{macro}
%    \begin{macro}{\bigintcalcSub}
%    \begin{macrocode}
\def\bigintcalcSub#1#2{%
  \romannumeral0%
  \expandafter\expandafter\expandafter\BIC@Add
  \bigintcalcNum{-#2}!{#1}%
}
%    \end{macrocode}
%    \end{macro}
%
%    \begin{macro}{\BIC@AddSwitch}
%    Decision table for \cs{BIC@AddSwitch}.
%    \begin{quote}
%    \begin{tabular}[t]{@{}|l|l|l|l|l|@{}}
%      \hline
%      $x<0$ & $y<0$ & $-x>-y$ & $-$ & $\opAdd(-x,-y)$\\
%      \cline{3-3}\cline{5-5}
%            &       & else    &     & $\opAdd(-y,-x)$\\
%      \cline{2-5}
%            & else  & $-x> y$ & $-$ & $\opSub(-x, y)$\\
%      \cline{3-5}
%            &       & $-x= y$ &     & $0$\\
%      \cline{3-5}
%            &       & else    & $+$ & $\opSub( y,-x)$\\
%      \hline
%      else  & $y<0$ & $ x>-y$ & $+$ & $\opSub( x,-y)$\\
%      \cline{3-5}
%            &       & $ x=-y$ &     & $0$\\
%      \cline{3-5}
%            &       & else    & $-$ & $\opSub(-y, x)$\\
%      \cline{2-5}
%            & else  & $ x> y$ & $+$ & $\opAdd( x, y)$\\
%      \cline{3-3}\cline{5-5}
%            &       & else    &     & $\opAdd( y, x)$\\
%      \hline
%    \end{tabular}
%    \end{quote}
%    \begin{macrocode}
\def\BIC@AddSwitch#1#2!#3#4!{%
  \ifx#1-% x < 0
    \ifx#3-% y < 0
      \expandafter-\romannumeral0%
      \ifnum\BIC@PosCmp#2!#4!=1 % -x > -y
        \BIC@AfterFiFiFi{%
          \BIC@AddXY#2!#4!!!%
        }%
      \else % -x <= -y
        \BIC@AfterFiFiFi{%
          \BIC@AddXY#4!#2!!!%
        }%
      \fi
    \else % y >= 0
      \ifcase\BIC@PosCmp#2!#3#4!% -x = y
        \BIC@AfterFiFiFi{ 0}%
      \or % -x > y
        \expandafter-\romannumeral0%
        \BIC@AfterFiFiFi{%
          \BIC@SubXY#2!#3#4!!!%
        }%
      \else % -x <= y
        \BIC@AfterFiFiFi{%
          \BIC@SubXY#3#4!#2!!!%
        }%
      \fi
    \fi
  \else % x >= 0
    \ifx#3-% y < 0
      \ifcase\BIC@PosCmp#1#2!#4!% x = -y
        \BIC@AfterFiFiFi{ 0}%
      \or % x > -y
        \BIC@AfterFiFiFi{%
          \BIC@SubXY#1#2!#4!!!%
        }%
      \else % x <= -y
        \expandafter-\romannumeral0%
        \BIC@AfterFiFiFi{%
          \BIC@SubXY#4!#1#2!!!%
        }%
      \fi
    \else % y >= 0
      \ifnum\BIC@PosCmp#1#2!#3#4!=1 % x > y
        \BIC@AfterFiFiFi{%
          \BIC@AddXY#1#2!#3#4!!!%
        }%
      \else % x <= y
        \BIC@AfterFiFiFi{%
          \BIC@AddXY#3#4!#1#2!!!%
        }%
      \fi
    \fi
  \BIC@Fi
}
%    \end{macrocode}
%    \end{macro}
%
%    \begin{macro}{\BigIntCalcAdd}
%    \begin{macrocode}
\def\BigIntCalcAdd#1!#2!{%
  \romannumeral0\BIC@AddXY#1!#2!!!%
}
%    \end{macrocode}
%    \end{macro}
%    \begin{macro}{\BigIntCalcSub}
%    \begin{macrocode}
\def\BigIntCalcSub#1!#2!{%
  \romannumeral0\BIC@SubXY#1!#2!!!%
}
%    \end{macrocode}
%    \end{macro}
%
%    \begin{macro}{\BIC@AddXY}
%    \begin{macrocode}
\def\BIC@AddXY#1#2!#3#4!#5!#6!{%
  \ifx\\#2\\%
    \ifx\\#3\\%
      \BIC@AfterFiFi{%
        \BIC@DoAdd0!#1#5!#60!%
      }%
    \else
      \BIC@AfterFiFi{%
        \BIC@DoAdd0!#1#5!#3#6!%
      }%
    \fi
  \else
    \ifx\\#4\\%
      \ifx\\#3\\%
        \BIC@AfterFiFiFi{%
          \BIC@AddXY#2!{}!#1#5!#60!%
        }%
      \else
        \BIC@AfterFiFiFi{%
          \BIC@AddXY#2!{}!#1#5!#3#6!%
        }%
      \fi
    \else
      \BIC@AfterFiFi{%
        \BIC@AddXY#2!#4!#1#5!#3#6!%
      }%
    \fi
  \BIC@Fi
}
%    \end{macrocode}
%    \end{macro}
%    \begin{macro}{\BIC@DoAdd}
%    |#1|: carry\\
%    |#2|: reverted result\\
%    |#3#4|: reverted $x$\\
%    |#5#6|: reverted $y$
%    \begin{macrocode}
\def\BIC@DoAdd#1#2!#3#4!#5#6!{%
  \ifx\\#4\\%
    \BIC@AfterFi{%
&     \expandafter\BIC@Space
&     \the\numexpr#1+#3+#5\relax#2%
$     \expandafter\expandafter\expandafter\BIC@AddResult
$     \BIC@AddDigit#1#3#5#2%
    }%
  \else
    \BIC@AfterFi{%
      \expandafter\expandafter\expandafter\BIC@DoAdd
      \BIC@AddDigit#1#3#5#2!#4!#6!%
    }%
  \BIC@Fi
}
%    \end{macrocode}
%    \end{macro}
%    \begin{macro}{\BIC@AddResult}
%    \begin{macrocode}
$ \def\BIC@AddResult#1{%
$   \ifx#10%
$     \expandafter\BIC@Space
$   \else
$     \expandafter\BIC@Space\expandafter#1%
$   \fi
$ }%
%    \end{macrocode}
%    \end{macro}
%    \begin{macro}{\BIC@AddDigit}
%    |#1|: carry\\
%    |#2|: digit of $x$\\
%    |#3|: digit of $y$
%    \begin{macrocode}
\def\BIC@AddDigit#1#2#3{%
  \romannumeral0%
& \expandafter\BIC@@AddDigit\the\numexpr#1+#2+#3!%
$ \expandafter\BIC@@AddDigit\number%
$ \csname
$   BIC@AddCarry%
$   \ifcase#1 %
$     #2%
$   \else
$     \ifcase#2 1\or2\or3\or4\or5\or6\or7\or8\or9\or10\fi
$   \fi
$ \endcsname#3!%
}
%    \end{macrocode}
%    \end{macro}
%    \begin{macro}{\BIC@@AddDigit}
%    \begin{macrocode}
\def\BIC@@AddDigit#1!{%
  \ifnum#1<10 %
    \BIC@AfterFi{ 0#1}%
  \else
    \BIC@AfterFi{ #1}%
  \BIC@Fi
}
%    \end{macrocode}
%    \end{macro}
%    \begin{macro}{\BIC@AddCarry0}
%    \begin{macrocode}
$ \expandafter\def\csname BIC@AddCarry0\endcsname#1{#1}%
%    \end{macrocode}
%    \end{macro}
%    \begin{macro}{\BIC@AddCarry10}
%    \begin{macrocode}
$ \expandafter\def\csname BIC@AddCarry10\endcsname#1{1#1}%
%    \end{macrocode}
%    \end{macro}
%    \begin{macro}{\BIC@AddCarry[1-9]}
%    \begin{macrocode}
$ \def\BIC@Temp#1#2{%
$   \expandafter\def\csname BIC@AddCarry#1\endcsname##1{%
$     \ifcase##1 #1\or
$     #2%
$?    \else\BigIntCalcError:ThisCannotHappen%
$     \fi
$   }%
$ }%
$ \BIC@Temp 0{1\or2\or3\or4\or5\or6\or7\or8\or9}%
$ \BIC@Temp 1{2\or3\or4\or5\or6\or7\or8\or9\or10}%
$ \BIC@Temp 2{3\or4\or5\or6\or7\or8\or9\or10\or11}%
$ \BIC@Temp 3{4\or5\or6\or7\or8\or9\or10\or11\or12}%
$ \BIC@Temp 4{5\or6\or7\or8\or9\or10\or11\or12\or13}%
$ \BIC@Temp 5{6\or7\or8\or9\or10\or11\or12\or13\or14}%
$ \BIC@Temp 6{7\or8\or9\or10\or11\or12\or13\or14\or15}%
$ \BIC@Temp 7{8\or9\or10\or11\or12\or13\or14\or15\or16}%
$ \BIC@Temp 8{9\or10\or11\or12\or13\or14\or15\or16\or17}%
$ \BIC@Temp 9{10\or11\or12\or13\or14\or15\or16\or17\or18}%
%    \end{macrocode}
%    \end{macro}
%
%    \begin{macro}{\BIC@SubXY}
%    Preconditions:
%    \begin{itemize}
%    \item $x > y$, $x \geq 0$, and $y >= 0$
%    \item digits($x$) = digits($y$)
%    \end{itemize}
%    \begin{macrocode}
\def\BIC@SubXY#1#2!#3#4!#5!#6!{%
  \ifx\\#2\\%
    \ifx\\#3\\%
      \BIC@AfterFiFi{%
        \BIC@DoSub0!#1#5!#60!%
      }%
    \else
      \BIC@AfterFiFi{%
        \BIC@DoSub0!#1#5!#3#6!%
      }%
    \fi
  \else
    \ifx\\#4\\%
      \ifx\\#3\\%
        \BIC@AfterFiFiFi{%
          \BIC@SubXY#2!{}!#1#5!#60!%
        }%
      \else
        \BIC@AfterFiFiFi{%
          \BIC@SubXY#2!{}!#1#5!#3#6!%
        }%
      \fi
    \else
      \BIC@AfterFiFi{%
        \BIC@SubXY#2!#4!#1#5!#3#6!%
      }%
    \fi
  \BIC@Fi
}
%    \end{macrocode}
%    \end{macro}
%    \begin{macro}{\BIC@DoSub}
%    |#1|: carry\\
%    |#2|: reverted result\\
%    |#3#4|: reverted x\\
%    |#5#6|: reverted y
%    \begin{macrocode}
\def\BIC@DoSub#1#2!#3#4!#5#6!{%
  \ifx\\#4\\%
    \BIC@AfterFi{%
      \expandafter\expandafter\expandafter\BIC@SubResult
      \BIC@SubDigit#1#3#5#2%
    }%
  \else
    \BIC@AfterFi{%
      \expandafter\expandafter\expandafter\BIC@DoSub
      \BIC@SubDigit#1#3#5#2!#4!#6!%
    }%
  \BIC@Fi
}
%    \end{macrocode}
%    \end{macro}
%    \begin{macro}{\BIC@SubResult}
%    \begin{macrocode}
\def\BIC@SubResult#1{%
  \ifx#10%
    \expandafter\BIC@SubResult
  \else
    \expandafter\BIC@Space\expandafter#1%
  \fi
}
%    \end{macrocode}
%    \end{macro}
%    \begin{macro}{\BIC@SubDigit}
%    |#1|: carry\\
%    |#2|: digit of $x$\\
%    |#3|: digit of $y$
%    \begin{macrocode}
\def\BIC@SubDigit#1#2#3{%
  \romannumeral0%
& \expandafter\BIC@@SubDigit\the\numexpr#2-#3-#1!%
$ \expandafter\BIC@@AddDigit\number
$   \csname
$     BIC@SubCarry%
$     \ifcase#1 %
$       #3%
$     \else
$       \ifcase#3 1\or2\or3\or4\or5\or6\or7\or8\or9\or10\fi
$     \fi
$   \endcsname#2!%
}
%    \end{macrocode}
%    \end{macro}
%    \begin{macro}{\BIC@@SubDigit}
%    \begin{macrocode}
& \def\BIC@@SubDigit#1!{%
&   \ifnum#1<0 %
&     \BIC@AfterFi{%
&       \expandafter\BIC@Space
&       \expandafter1\the\numexpr#1+10\relax
&     }%
&   \else
&     \BIC@AfterFi{ 0#1}%
&   \BIC@Fi
& }%
%    \end{macrocode}
%    \end{macro}
%    \begin{macro}{\BIC@SubCarry0}
%    \begin{macrocode}
$ \expandafter\def\csname BIC@SubCarry0\endcsname#1{#1}%
%    \end{macrocode}
%    \end{macro}
%    \begin{macro}{\BIC@SubCarry10}%
%    \begin{macrocode}
$ \expandafter\def\csname BIC@SubCarry10\endcsname#1{1#1}%
%    \end{macrocode}
%    \end{macro}
%    \begin{macro}{\BIC@SubCarry[1-9]}
%    \begin{macrocode}
$ \def\BIC@Temp#1#2{%
$   \expandafter\def\csname BIC@SubCarry#1\endcsname##1{%
$     \ifcase##1 #2%
$?    \else\BigIntCalcError:ThisCannotHappen%
$     \fi
$   }%
$ }%
$ \BIC@Temp 1{19\or0\or1\or2\or3\or4\or5\or6\or7\or8}%
$ \BIC@Temp 2{18\or19\or0\or1\or2\or3\or4\or5\or6\or7}%
$ \BIC@Temp 3{17\or18\or19\or0\or1\or2\or3\or4\or5\or6}%
$ \BIC@Temp 4{16\or17\or18\or19\or0\or1\or2\or3\or4\or5}%
$ \BIC@Temp 5{15\or16\or17\or18\or19\or0\or1\or2\or3\or4}%
$ \BIC@Temp 6{14\or15\or16\or17\or18\or19\or0\or1\or2\or3}%
$ \BIC@Temp 7{13\or14\or15\or16\or17\or18\or19\or0\or1\or2}%
$ \BIC@Temp 8{12\or13\or14\or15\or16\or17\or18\or19\or0\or1}%
$ \BIC@Temp 9{11\or12\or13\or14\or15\or16\or17\or18\or19\or0}%
%    \end{macrocode}
%    \end{macro}
%
% \subsection{\op{Shl}, \op{Shr}}
%
%    \begin{macro}{\bigintcalcShl}
%    \begin{macrocode}
\def\bigintcalcShl#1{%
  \romannumeral0%
  \expandafter\expandafter\expandafter\BIC@Shl
  \bigintcalcNum{#1}!%
}
%    \end{macrocode}
%    \end{macro}
%    \begin{macro}{\BIC@Shl}
%    \begin{macrocode}
\def\BIC@Shl#1#2!{%
  \ifx#1-%
    \BIC@AfterFi{%
      \expandafter-\romannumeral0%
&     \BIC@@Shl#2!!%
$     \BIC@AddXY#2!#2!!!%
    }%
  \else
    \BIC@AfterFi{%
&     \BIC@@Shl#1#2!!%
$     \BIC@AddXY#1#2!#1#2!!!%
    }%
  \BIC@Fi
}
%    \end{macrocode}
%    \end{macro}
%    \begin{macro}{\BigIntCalcShl}
%    \begin{macrocode}
\def\BigIntCalcShl#1!{%
  \romannumeral0%
& \BIC@@Shl#1!!%
$ \BIC@AddXY#1!#1!!!%
}
%    \end{macrocode}
%    \end{macro}
%    \begin{macro}{\BIC@@Shl}
%    \begin{macrocode}
& \def\BIC@@Shl#1#2!{%
&   \ifx\\#2\\%
&     \BIC@AfterFi{%
&       \BIC@@@Shl0!#1%
&     }%
&   \else
&     \BIC@AfterFi{%
&       \BIC@@Shl#2!#1%
&     }%
&   \BIC@Fi
& }%
%    \end{macrocode}
%    \end{macro}
%    \begin{macro}{\BIC@@@Shl}
%    |#1|: carry\\
%    |#2|: result\\
%    |#3#4|: reverted number
%    \begin{macrocode}
& \def\BIC@@@Shl#1#2!#3#4!{%
&   \ifx\\#4\\%
&     \BIC@AfterFi{%
&       \expandafter\BIC@Space
&       \the\numexpr#3*2+#1\relax#2%
&     }%
&   \else
&     \BIC@AfterFi{%
&       \expandafter\BIC@@@@Shl\the\numexpr#3*2+#1!#2!#4!%
&     }%
&   \BIC@Fi
& }%
%    \end{macrocode}
%    \end{macro}
%    \begin{macro}{\BIC@@@@Shl}
%    \begin{macrocode}
& \def\BIC@@@@Shl#1!{%
&   \ifnum#1<10 %
&     \BIC@AfterFi{%
&       \BIC@@@Shl0#1%
&     }%
&   \else
&     \BIC@AfterFi{%
&       \BIC@@@Shl#1%
&     }%
&   \BIC@Fi
& }%
%    \end{macrocode}
%    \end{macro}
%
%    \begin{macro}{\bigintcalcShr}
%    \begin{macrocode}
\def\bigintcalcShr#1{%
  \romannumeral0%
  \expandafter\expandafter\expandafter\BIC@Shr
  \bigintcalcNum{#1}!%
}
%    \end{macrocode}
%    \end{macro}
%    \begin{macro}{\BIC@Shr}
%    \begin{macrocode}
\def\BIC@Shr#1#2!{%
  \ifx#1-%
    \expandafter-\romannumeral0%
    \BIC@AfterFi{%
      \BIC@@Shr#2!%
    }%
  \else
    \BIC@AfterFi{%
      \BIC@@Shr#1#2!%
    }%
  \BIC@Fi
}
%    \end{macrocode}
%    \end{macro}
%    \begin{macro}{\BigIntCalcShr}
%    \begin{macrocode}
\def\BigIntCalcShr#1!{%
  \romannumeral0%
  \BIC@@Shr#1!%
}
%    \end{macrocode}
%    \end{macro}
%    \begin{macro}{\BIC@@Shr}
%    \begin{macrocode}
\def\BIC@@Shr#1#2!{%
  \ifcase#1 %
    \BIC@AfterFi{ 0}%
  \or
    \ifx\\#2\\%
      \BIC@AfterFiFi{ 0}%
    \else
      \BIC@AfterFiFi{%
        \BIC@@@Shr#1#2!!%
      }%
    \fi
  \else
    \BIC@AfterFi{%
      \BIC@@@Shr0#1#2!!%
    }%
  \BIC@Fi
}
%    \end{macrocode}
%    \end{macro}
%    \begin{macro}{\BIC@@@Shr}
%    |#1|: carry\\
%    |#2#3|: number\\
%    |#4|: result
%    \begin{macrocode}
\def\BIC@@@Shr#1#2#3!#4!{%
  \ifx\\#3\\%
    \ifodd#1#2 %
      \BIC@AfterFiFi{%
&       \expandafter\BIC@ShrResult\the\numexpr(#1#2-1)/2\relax
$       \expandafter\expandafter\expandafter\BIC@ShrResult
$       \csname BIC@ShrDigit#1#2\endcsname
        #4!%
      }%
    \else
      \BIC@AfterFiFi{%
&       \expandafter\BIC@ShrResult\the\numexpr#1#2/2\relax
$       \expandafter\expandafter\expandafter\BIC@ShrResult
$       \csname BIC@ShrDigit#1#2\endcsname
        #4!%
      }%
    \fi
  \else
    \ifodd#1#2 %
      \BIC@AfterFiFi{%
&       \expandafter\BIC@@@@Shr\the\numexpr(#1#2-1)/2\relax1%
$       \expandafter\expandafter\expandafter\BIC@@@@Shr
$       \csname BIC@ShrDigit#1#2\endcsname
        #3!#4!%
      }%
    \else
      \BIC@AfterFiFi{%
&       \expandafter\BIC@@@@Shr\the\numexpr#1#2/2\relax0%
$       \expandafter\expandafter\expandafter\BIC@@@@Shr
$       \csname BIC@ShrDigit#1#2\endcsname
        #3!#4!%
      }%
    \fi
  \BIC@Fi
}
%    \end{macrocode}
%    \end{macro}
%    \begin{macro}{\BIC@ShrResult}
%    \begin{macrocode}
& \def\BIC@ShrResult#1#2!{ #2#1}%
$ \def\BIC@ShrResult#1#2#3!{ #3#1}%
%    \end{macrocode}
%    \end{macro}
%    \begin{macro}{\BIC@@@@Shr}
%    |#1|: new digit\\
%    |#2|: carry\\
%    |#3|: remaining number\\
%    |#4|: result
%    \begin{macrocode}
\def\BIC@@@@Shr#1#2#3!#4!{%
  \BIC@@@Shr#2#3!#4#1!%
}
%    \end{macrocode}
%    \end{macro}
%    \begin{macro}{\BIC@ShrDigit[00-19]}
%    \begin{macrocode}
$ \def\BIC@Temp#1#2#3#4{%
$   \expandafter\def\csname BIC@ShrDigit#1#2\endcsname{#3#4}%
$ }%
$ \BIC@Temp 0000%
$ \BIC@Temp 0101%
$ \BIC@Temp 0210%
$ \BIC@Temp 0311%
$ \BIC@Temp 0420%
$ \BIC@Temp 0521%
$ \BIC@Temp 0630%
$ \BIC@Temp 0731%
$ \BIC@Temp 0840%
$ \BIC@Temp 0941%
$ \BIC@Temp 1050%
$ \BIC@Temp 1151%
$ \BIC@Temp 1260%
$ \BIC@Temp 1361%
$ \BIC@Temp 1470%
$ \BIC@Temp 1571%
$ \BIC@Temp 1680%
$ \BIC@Temp 1781%
$ \BIC@Temp 1890%
$ \BIC@Temp 1991%
%    \end{macrocode}
%    \end{macro}
%
% \subsection{\cs{BIC@Tim}}
%
%    \begin{macro}{\BIC@Tim}
%    Macro \cs{BIC@Tim} implements ``Number \emph{tim}es digit''.\\
%    |#1|: plain number without sign\\
%    |#2|: digit
\def\BIC@Tim#1!#2{%
  \romannumeral0%
  \ifcase#2 % 0
    \BIC@AfterFi{ 0}%
  \or % 1
    \BIC@AfterFi{ #1}%
  \or % 2
    \BIC@AfterFi{%
      \BIC@Shl#1!%
    }%
  \else % 3-9
    \BIC@AfterFi{%
      \BIC@@Tim#1!!#2%
    }%
  \BIC@Fi
}
%    \begin{macrocode}
%    \end{macrocode}
%    \end{macro}
%    \begin{macro}{\BIC@@Tim}
%    |#1#2|: number\\
%    |#3|: reverted number
%    \begin{macrocode}
\def\BIC@@Tim#1#2!{%
  \ifx\\#2\\%
    \BIC@AfterFi{%
      \BIC@ProcessTim0!#1%
    }%
  \else
    \BIC@AfterFi{%
      \BIC@@Tim#2!#1%
    }%
  \BIC@Fi
}
%    \end{macrocode}
%    \end{macro}
%    \begin{macro}{\BIC@ProcessTim}
%    |#1|: carry\\
%    |#2|: result\\
%    |#3#4|: reverted number\\
%    |#5|: digit
%    \begin{macrocode}
\def\BIC@ProcessTim#1#2!#3#4!#5{%
  \ifx\\#4\\%
    \BIC@AfterFi{%
      \expandafter\BIC@Space
&     \the\numexpr#3*#5+#1\relax
$     \romannumeral0\BIC@TimDigit#3#5#1%
      #2%
    }%
  \else
    \BIC@AfterFi{%
      \expandafter\BIC@@ProcessTim
&     \the\numexpr#3*#5+#1%
$     \romannumeral0\BIC@TimDigit#3#5#1%
      !#2!#4!#5%
    }%
  \BIC@Fi
}
%    \end{macrocode}
%    \end{macro}
%    \begin{macro}{\BIC@@ProcessTim}
%    |#1#2|: carry?, new digit\\
%    |#3|: new number\\
%    |#4|: old number\\
%    |#5|: digit
%    \begin{macrocode}
\def\BIC@@ProcessTim#1#2!{%
  \ifx\\#2\\%
    \BIC@AfterFi{%
      \BIC@ProcessTim0#1%
    }%
  \else
    \BIC@AfterFi{%
      \BIC@ProcessTim#1#2%
    }%
  \BIC@Fi
}
%    \end{macrocode}
%    \end{macro}
%    \begin{macro}{\BIC@TimDigit}
%    |#1|: digit 0--9\\
%    |#2|: digit 3--9\\
%    |#3|: carry 0--9
%    \begin{macrocode}
$ \def\BIC@TimDigit#1#2#3{%
$   \ifcase#1 % 0
$     \BIC@AfterFi{ #3}%
$   \or % 1
$     \BIC@AfterFi{%
$       \expandafter\BIC@Space
$       \number\csname BIC@AddCarry#2\endcsname#3 %
$     }%
$   \else
$     \ifcase#3 %
$       \BIC@AfterFiFi{%
$         \expandafter\BIC@Space
$         \number\csname BIC@MulDigit#2\endcsname#1 %
$       }%
$     \else
$       \BIC@AfterFiFi{%
$         \expandafter\BIC@Space
$         \romannumeral0%
$         \expandafter\BIC@AddXY
$         \number\csname BIC@MulDigit#2\endcsname#1!%
$         #3!!!%
$       }%
$     \fi
$   \BIC@Fi
$ }%
%    \end{macrocode}
%    \end{macro}
%    \begin{macro}{\BIC@MulDigit[3-9]}
%    \begin{macrocode}
$ \def\BIC@Temp#1#2{%
$   \expandafter\def\csname BIC@MulDigit#1\endcsname##1{%
$     \ifcase##1 0%
$     \or ##1%
$     \or #2%
$?    \else\BigIntCalcError:ThisCannotHappen%
$     \fi
$   }%
$ }%
$ \BIC@Temp 3{6\or9\or12\or15\or18\or21\or24\or27}%
$ \BIC@Temp 4{8\or12\or16\or20\or24\or28\or32\or36}%
$ \BIC@Temp 5{10\or15\or20\or25\or30\or35\or40\or45}%
$ \BIC@Temp 6{12\or18\or24\or30\or36\or42\or48\or54}%
$ \BIC@Temp 7{14\or21\or28\or35\or42\or49\or56\or63}%
$ \BIC@Temp 8{16\or24\or32\or40\or48\or56\or64\or72}%
$ \BIC@Temp 9{18\or27\or36\or45\or54\or63\or72\or81}%
%    \end{macrocode}
%    \end{macro}
%
% \subsection{\op{Mul}}
%
%    \begin{macro}{\bigintcalcMul}
%    \begin{macrocode}
\def\bigintcalcMul#1#2{%
  \romannumeral0%
  \expandafter\expandafter\expandafter\BIC@Mul
  \bigintcalcNum{#1}!{#2}%
}
%    \end{macrocode}
%    \end{macro}
%    \begin{macro}{\BIC@Mul}
%    \begin{macrocode}
\def\BIC@Mul#1!#2{%
  \expandafter\expandafter\expandafter\BIC@MulSwitch
  \bigintcalcNum{#2}!#1!%
}
%    \end{macrocode}
%    \end{macro}
%    \begin{macro}{\BIC@MulSwitch}
%    Decision table for \cs{BIC@MulSwitch}.
%    \begin{quote}
%    \begin{tabular}[t]{@{}|l|l|l|l|l|@{}}
%      \hline
%      $x=0$ &\multicolumn{3}{l}{}    &$0$\\
%      \hline
%      $x>0$ & $y=0$ &\multicolumn2l{}&$0$\\
%      \cline{2-5}
%            & $y>0$ & $ x> y$ & $+$ & $\opMul( x, y)$\\
%      \cline{3-3}\cline{5-5}
%            &       & else    &     & $\opMul( y, x)$\\
%      \cline{2-5}
%            & $y<0$ & $ x>-y$ & $-$ & $\opMul( x,-y)$\\
%      \cline{3-3}\cline{5-5}
%            &       & else    &     & $\opMul(-y, x)$\\
%      \hline
%      $x<0$ & $y=0$ &\multicolumn2l{}&$0$\\
%      \cline{2-5}
%            & $y>0$ & $-x> y$ & $-$ & $\opMul(-x, y)$\\
%      \cline{3-3}\cline{5-5}
%            &       & else    &     & $\opMul( y,-x)$\\
%      \cline{2-5}
%            & $y<0$ & $-x>-y$ & $+$ & $\opMul(-x,-y)$\\
%      \cline{3-3}\cline{5-5}
%            &       & else    &     & $\opMul(-y,-x)$\\
%      \hline
%    \end{tabular}
%    \end{quote}
%    \begin{macrocode}
\def\BIC@MulSwitch#1#2!#3#4!{%
  \ifcase\BIC@Sgn#1#2! % x = 0
    \BIC@AfterFi{ 0}%
  \or % x > 0
    \ifcase\BIC@Sgn#3#4! % y = 0
      \BIC@AfterFiFi{ 0}%
    \or % y > 0
      \ifnum\BIC@PosCmp#1#2!#3#4!=1 % x > y
        \BIC@AfterFiFiFi{%
          \BIC@ProcessMul0!#1#2!#3#4!%
        }%
      \else % x <= y
        \BIC@AfterFiFiFi{%
          \BIC@ProcessMul0!#3#4!#1#2!%
        }%
      \fi
    \else % y < 0
      \expandafter-\romannumeral0%
      \ifnum\BIC@PosCmp#1#2!#4!=1 % x > -y
        \BIC@AfterFiFiFi{%
          \BIC@ProcessMul0!#1#2!#4!%
        }%
      \else % x <= -y
        \BIC@AfterFiFiFi{%
          \BIC@ProcessMul0!#4!#1#2!%
        }%
      \fi
    \fi
  \else % x < 0
    \ifcase\BIC@Sgn#3#4! % y = 0
      \BIC@AfterFiFi{ 0}%
    \or % y > 0
      \expandafter-\romannumeral0%
      \ifnum\BIC@PosCmp#2!#3#4!=1 % -x > y
        \BIC@AfterFiFiFi{%
          \BIC@ProcessMul0!#2!#3#4!%
        }%
      \else % -x <= y
        \BIC@AfterFiFiFi{%
          \BIC@ProcessMul0!#3#4!#2!%
        }%
      \fi
    \else % y < 0
      \ifnum\BIC@PosCmp#2!#4!=1 % -x > -y
        \BIC@AfterFiFiFi{%
          \BIC@ProcessMul0!#2!#4!%
        }%
      \else % -x <= -y
        \BIC@AfterFiFiFi{%
          \BIC@ProcessMul0!#4!#2!%
        }%
      \fi
    \fi
  \BIC@Fi
}
%    \end{macrocode}
%    \end{macro}
%    \begin{macro}{\BigIntCalcMul}
%    \begin{macrocode}
\def\BigIntCalcMul#1!#2!{%
  \romannumeral0%
  \BIC@ProcessMul0!#1!#2!%
}
%    \end{macrocode}
%    \end{macro}
%    \begin{macro}{\BIC@ProcessMul}
%    |#1|: result\\
%    |#2|: number $x$\\
%    |#3#4|: number $y$
%    \begin{macrocode}
\def\BIC@ProcessMul#1!#2!#3#4!{%
  \ifx\\#4\\%
    \BIC@AfterFi{%
      \expandafter\expandafter\expandafter\BIC@Space
      \bigintcalcAdd{\BIC@Tim#2!#3}{#10}%
    }%
  \else
    \BIC@AfterFi{%
      \expandafter\expandafter\expandafter\BIC@ProcessMul
      \bigintcalcAdd{\BIC@Tim#2!#3}{#10}!#2!#4!%
    }%
  \BIC@Fi
}
%    \end{macrocode}
%    \end{macro}
%
% \subsection{\op{Sqr}}
%
%    \begin{macro}{\bigintcalcSqr}
%    \begin{macrocode}
\def\bigintcalcSqr#1{%
  \romannumeral0%
  \expandafter\expandafter\expandafter\BIC@Sqr
  \bigintcalcNum{#1}!%
}
%    \end{macrocode}
%    \end{macro}
%    \begin{macro}{\BIC@Sqr}
%    \begin{macrocode}
\def\BIC@Sqr#1{%
   \ifx#1-%
     \expandafter\BIC@@Sqr
   \else
     \expandafter\BIC@@Sqr\expandafter#1%
   \fi
}
%    \end{macrocode}
%    \end{macro}
%    \begin{macro}{\BIC@@Sqr}
%    \begin{macrocode}
\def\BIC@@Sqr#1!{%
  \BIC@ProcessMul0!#1!#1!%
}
%    \end{macrocode}
%    \end{macro}
%
% \subsection{\op{Fac}}
%
%    \begin{macro}{\bigintcalcFac}
%    \begin{macrocode}
\def\bigintcalcFac#1{%
  \romannumeral0%
  \expandafter\expandafter\expandafter\BIC@Fac
  \bigintcalcNum{#1}!%
}
%    \end{macrocode}
%    \end{macro}
%    \begin{macro}{\BIC@Fac}
%    \begin{macrocode}
\def\BIC@Fac#1#2!{%
  \ifx#1-%
    \BIC@AfterFi{ 0\BigIntCalcError:FacNegative}%
  \else
    \ifnum\BIC@PosCmp#1#2!13!<0 %
      \ifcase#1#2 %
         \BIC@AfterFiFiFi{ 1}% 0!
      \or\BIC@AfterFiFiFi{ 1}% 1!
      \or\BIC@AfterFiFiFi{ 2}% 2!
      \or\BIC@AfterFiFiFi{ 6}% 3!
      \or\BIC@AfterFiFiFi{ 24}% 4!
      \or\BIC@AfterFiFiFi{ 120}% 5!
      \or\BIC@AfterFiFiFi{ 720}% 6!
      \or\BIC@AfterFiFiFi{ 5040}% 7!
      \or\BIC@AfterFiFiFi{ 40320}% 8!
      \or\BIC@AfterFiFiFi{ 362880}% 9!
      \or\BIC@AfterFiFiFi{ 3628800}% 10!
      \or\BIC@AfterFiFiFi{ 39916800}% 11!
      \or\BIC@AfterFiFiFi{ 479001600}% 12!
?     \else\BigIntCalcError:ThisCannotHappen%
      \fi
    \else
      \BIC@AfterFiFi{%
        \BIC@ProcessFac#1#2!479001600!%
      }%
    \fi
  \BIC@Fi
}
%    \end{macrocode}
%    \end{macro}
%    \begin{macro}{\BIC@ProcessFac}
%    |#1|: $n$\\
%    |#2|: result
%    \begin{macrocode}
\def\BIC@ProcessFac#1!#2!{%
  \ifnum\BIC@PosCmp#1!12!=0 %
    \BIC@AfterFi{ #2}%
  \else
    \BIC@AfterFi{%
      \expandafter\BIC@@ProcessFac
      \romannumeral0\BIC@ProcessMul0!#2!#1!%
      !#1!%
    }%
  \BIC@Fi
}
%    \end{macrocode}
%    \end{macro}
%    \begin{macro}{\BIC@@ProcessFac}
%    |#1|: result\\
%    |#2|: $n$
%    \begin{macrocode}
\def\BIC@@ProcessFac#1!#2!{%
  \expandafter\BIC@ProcessFac
  \romannumeral0\BIC@Dec#2!{}%
  !#1!%
}
%    \end{macrocode}
%    \end{macro}
%
% \subsection{\op{Pow}}
%
%    \begin{macro}{\bigintcalcPow}
%    |#1|: basis\\
%    |#2|: power
%    \begin{macrocode}
\def\bigintcalcPow#1{%
  \romannumeral0%
  \expandafter\expandafter\expandafter\BIC@Pow
  \bigintcalcNum{#1}!%
}
%    \end{macrocode}
%    \end{macro}
%    \begin{macro}{\BIC@Pow}
%    |#1|: basis\\
%    |#2|: power
%    \begin{macrocode}
\def\BIC@Pow#1!#2{%
  \expandafter\expandafter\expandafter\BIC@PowSwitch
  \bigintcalcNum{#2}!#1!%
}
%    \end{macrocode}
%    \end{macro}
%    \begin{macro}{\BIC@PowSwitch}
%    |#1#2|: power $y$\\
%    |#3#4|: basis $x$\\
%    Decision table for \cs{BIC@PowSwitch}.
%    \begin{quote}
%    \def\M#1{\multicolumn{#1}{l}{}}%
%    \begin{tabular}[t]{@{}|l|l|l|l|@{}}
%    \hline
%    $y=0$ & \M{2}                        & $1$\\
%    \hline
%    $y=1$ & \M{2}                        & $x$\\
%    \hline
%    $y=2$ & $x<0$          & \M{1}       & $\opMul(-x,-x)$\\
%    \cline{2-4}
%          & else           & \M{1}       & $\opMul(x,x)$\\
%    \hline
%    $y<0$ & $x=0$          & \M{1}       & DivisionByZero\\
%    \cline{2-4}
%          & $x=1$          & \M{1}       & $1$\\
%    \cline{2-4}
%          & $x=-1$         & $\opODD(y)$ & $-1$\\
%    \cline{3-4}
%          &                & else        & $1$\\
%    \cline{2-4}
%          & else ($\string|x\string|>1$) & \M{1}       & $0$\\
%    \hline
%    $y>2$ & $x=0$          & \M{1}       & $0$\\
%    \cline{2-4}
%          & $x=1$          & \M{1}       & $1$\\
%    \cline{2-4}
%          & $x=-1$         & $\opODD(y)$ & $-1$\\
%    \cline{3-4}
%          &                & else        & $1$\\
%    \cline{2-4}
%          & $x<-1$ ($x<0$) & $\opODD(y)$ & $-\opPow(-x,y)$\\
%    \cline{3-4}
%          &                & else        & $\opPow(-x,y)$\\
%    \cline{2-4}
%          & else ($x>1$)   & \M{1}       & $\opPow(x,y)$\\
%    \hline
%    \end{tabular}
%    \end{quote}
%    \begin{macrocode}
\def\BIC@PowSwitch#1#2!#3#4!{%
  \ifcase\ifx\\#2\\%
           \ifx#100 % y = 0
           \else\ifx#111 % y = 1
           \else\ifx#122 % y = 2
           \else4 % y > 2
           \fi\fi\fi
         \else
           \ifx#1-3 % y < 0
           \else4 % y > 2
           \fi
         \fi
    \BIC@AfterFi{ 1}% y = 0
  \or % y = 1
    \BIC@AfterFi{ #3#4}%
  \or % y = 2
    \ifx#3-% x < 0
      \BIC@AfterFiFi{%
        \BIC@ProcessMul0!#4!#4!%
      }%
    \else % x >= 0
      \BIC@AfterFiFi{%
        \BIC@ProcessMul0!#3#4!#3#4!%
      }%
    \fi
  \or % y < 0
    \ifcase\ifx\\#4\\%
             \ifx#300 % x = 0
             \else\ifx#311 % x = 1
             \else3 % x > 1
             \fi\fi
           \else
             \ifcase\BIC@MinusOne#3#4! %
               3 % |x| > 1
             \or
               2 % x = -1
?            \else\BigIntCalcError:ThisCannotHappen%
             \fi
           \fi
      \BIC@AfterFiFi{ 0\BigIntCalcError:DivisionByZero}% x = 0
    \or % x = 1
      \BIC@AfterFiFi{ 1}% x = 1
    \or % x = -1
      \ifcase\BIC@ModTwo#2! % even(y)
        \BIC@AfterFiFiFi{ 1}%
      \or % odd(y)
        \BIC@AfterFiFiFi{ -1}%
?     \else\BigIntCalcError:ThisCannotHappen%
      \fi
    \or % |x| > 1
      \BIC@AfterFiFi{ 0}%
?   \else\BigIntCalcError:ThisCannotHappen%
    \fi
  \or % y > 2
    \ifcase\ifx\\#4\\%
             \ifx#300 % x = 0
             \else\ifx#311 % x = 1
             \else4 % x > 1
             \fi\fi
           \else
             \ifx#3-%
               \ifcase\BIC@MinusOne#3#4! %
                 3 % x < -1
               \else
                 2 % x = -1
               \fi
             \else
               4 % x > 1
             \fi
           \fi
      \BIC@AfterFiFi{ 0}% x = 0
    \or % x = 1
      \BIC@AfterFiFi{ 1}% x = 1
    \or % x = -1
      \ifcase\BIC@ModTwo#1#2! % even(y)
        \BIC@AfterFiFiFi{ 1}%
      \or % odd(y)
        \BIC@AfterFiFiFi{ -1}%
?     \else\BigIntCalcError:ThisCannotHappen%
      \fi
    \or % x < -1
      \ifcase\BIC@ModTwo#1#2! % even(y)
        \BIC@AfterFiFiFi{%
          \BIC@PowRec#4!#1#2!1!%
        }%
      \or % odd(y)
        \expandafter-\romannumeral0%
        \BIC@AfterFiFiFi{%
          \BIC@PowRec#4!#1#2!1!%
        }%
?     \else\BigIntCalcError:ThisCannotHappen%
      \fi
    \or % x > 1
      \BIC@AfterFiFi{%
        \BIC@PowRec#3#4!#1#2!1!%
      }%
?   \else\BigIntCalcError:ThisCannotHappen%
    \fi
? \else\BigIntCalcError:ThisCannotHappen%
  \BIC@Fi
}
%    \end{macrocode}
%    \end{macro}
%
% \subsubsection{Help macros}
%
%    \begin{macro}{\BIC@ModTwo}
%    Macro \cs{BIC@ModTwo} expects a number without sign
%    and returns digit |1| or |0| if the number is odd or even.
%    \begin{macrocode}
\def\BIC@ModTwo#1#2!{%
  \ifx\\#2\\%
    \ifodd#1 %
      \BIC@AfterFiFi1%
    \else
      \BIC@AfterFiFi0%
    \fi
  \else
    \BIC@AfterFi{%
      \BIC@ModTwo#2!%
    }%
  \BIC@Fi
}
%    \end{macrocode}
%    \end{macro}
%
%    \begin{macro}{\BIC@MinusOne}
%    Macro \cs{BIC@MinusOne} expects a number and returns
%    digit |1| if the number equals minus one and returns |0| otherwise.
%    \begin{macrocode}
\def\BIC@MinusOne#1#2!{%
  \ifx#1-%
    \BIC@@MinusOne#2!%
  \else
    0%
  \fi
}
%    \end{macrocode}
%    \end{macro}
%    \begin{macro}{\BIC@@MinusOne}
%    \begin{macrocode}
\def\BIC@@MinusOne#1#2!{%
  \ifx#11%
    \ifx\\#2\\%
      1%
    \else
      0%
    \fi
  \else
    0%
  \fi
}
%    \end{macrocode}
%    \end{macro}
%
% \subsubsection{Recursive calculation}
%
%    \begin{macro}{\BIC@PowRec}
%\begin{quote}
%\begin{verbatim}
%Pow(x, y) {
%  PowRec(x, y, 1)
%}
%PowRec(x, y, r) {
%  if y == 1 then
%    return r
%  else
%    ifodd y then
%      return PowRec(x*x, y div 2, r*x) % y div 2 = (y-1)/2
%    else
%      return PowRec(x*x, y div 2, r)
%    fi
%  fi
%}
%\end{verbatim}
%\end{quote}
%    |#1|: $x$ (basis)\\
%    |#2#3|: $y$ (power)\\
%    |#4|: $r$ (result)
%    \begin{macrocode}
\def\BIC@PowRec#1!#2#3!#4!{%
  \ifcase\ifx#21\ifx\\#3\\0 \else1 \fi\else1 \fi % y = 1
    \ifnum\BIC@PosCmp#1!#4!=1 % x > r
      \BIC@AfterFiFi{%
        \BIC@ProcessMul0!#1!#4!%
      }%
    \else
      \BIC@AfterFiFi{%
        \BIC@ProcessMul0!#4!#1!%
      }%
    \fi
  \or
    \ifcase\BIC@ModTwo#2#3! % even(y)
      \BIC@AfterFiFi{%
        \expandafter\BIC@@PowRec\romannumeral0%
        \BIC@@Shr#2#3!%
        !#1!#4!%
      }%
    \or % odd(y)
      \ifnum\BIC@PosCmp#1!#4!=1 % x > r
        \BIC@AfterFiFiFi{%
          \expandafter\BIC@@@PowRec\romannumeral0%
          \BIC@ProcessMul0!#1!#4!%
          !#1!#2#3!%
        }%
      \else
        \BIC@AfterFiFiFi{%
          \expandafter\BIC@@@PowRec\romannumeral0%
          \BIC@ProcessMul0!#1!#4!%
          !#1!#2#3!%
        }%
      \fi
?   \else\BigIntCalcError:ThisCannotHappen%
    \fi
? \else\BigIntCalcError:ThisCannotHappen%
  \BIC@Fi
}
%    \end{macrocode}
%    \end{macro}
%    \begin{macro}{\BIC@@PowRec}
%    |#1|: $y/2$\\
%    |#2|: $x$\\
%    |#3|: new $r$ ($r$ or $r*x$)
%    \begin{macrocode}
\def\BIC@@PowRec#1!#2!#3!{%
  \expandafter\BIC@PowRec\romannumeral0%
  \BIC@ProcessMul0!#2!#2!%
  !#1!#3!%
}
%    \end{macrocode}
%    \end{macro}
%    \begin{macro}{\BIC@@@PowRec}
%    |#1|: $r*x$
%    |#2|: $x$
%    |#3|: $y$
%    \begin{macrocode}
\def\BIC@@@PowRec#1!#2!#3!{%
  \expandafter\BIC@@PowRec\romannumeral0%
  \BIC@@Shr#3!%
  !#2!#1!%
}
%    \end{macrocode}
%    \end{macro}
%
% \subsection{\op{Div}}
%
%    \begin{macro}{\bigintcalcDiv}
%    |#1|: $x$\\
%    |#2|: $y$ (divisor)
%    \begin{macrocode}
\def\bigintcalcDiv#1{%
  \romannumeral0%
  \expandafter\expandafter\expandafter\BIC@Div
  \bigintcalcNum{#1}!%
}
%    \end{macrocode}
%    \end{macro}
%    \begin{macro}{\BIC@Div}
%    |#1|: $x$\\
%    |#2|: $y$
%    \begin{macrocode}
\def\BIC@Div#1!#2{%
  \expandafter\expandafter\expandafter\BIC@DivSwitchSign
  \bigintcalcNum{#2}!#1!%
}
%    \end{macrocode}
%    \end{macro}
%    \begin{macro}{\BigIntCalcDiv}
%    \begin{macrocode}
\def\BigIntCalcDiv#1!#2!{%
  \romannumeral0%
  \BIC@DivSwitchSign#2!#1!%
}
%    \end{macrocode}
%    \end{macro}
%    \begin{macro}{\BIC@DivSwitchSign}
%    Decision table for \cs{BIC@DivSwitchSign}.
%    \begin{quote}
%    \begin{tabular}{@{}|l|l|l|@{}}
%    \hline
%    $y=0$ & \multicolumn{1}{l}{} & DivisionByZero\\
%    \hline
%    $y>0$ & $x=0$ & $0$\\
%    \cline{2-3}
%          & $x>0$ & DivSwitch$(+,x,y)$\\
%    \cline{2-3}
%          & $x<0$ & DivSwitch$(-,-x,y)$\\
%    \hline
%    $y<0$ & $x=0$ & $0$\\
%    \cline{2-3}
%          & $x>0$ & DivSwitch$(-,x,-y)$\\
%    \cline{2-3}
%          & $x<0$ & DivSwitch$(+,-x,-y)$\\
%    \hline
%    \end{tabular}
%    \end{quote}
%    |#1|: $y$ (divisor)\\
%    |#2|: $x$
%    \begin{macrocode}
\def\BIC@DivSwitchSign#1#2!#3#4!{%
  \ifcase\BIC@Sgn#1#2! % y = 0
    \BIC@AfterFi{ 0\BigIntCalcError:DivisionByZero}%
  \or % y > 0
    \ifcase\BIC@Sgn#3#4! % x = 0
      \BIC@AfterFiFi{ 0}%
    \or % x > 0
      \BIC@AfterFiFi{%
        \BIC@DivSwitch{}#3#4!#1#2!%
      }%
    \else % x < 0
      \BIC@AfterFiFi{%
        \BIC@DivSwitch-#4!#1#2!%
      }%
    \fi
  \else % y < 0
    \ifcase\BIC@Sgn#3#4! % x = 0
      \BIC@AfterFiFi{ 0}%
    \or % x > 0
      \BIC@AfterFiFi{%
        \BIC@DivSwitch-#3#4!#2!%
      }%
    \else % x < 0
      \BIC@AfterFiFi{%
        \BIC@DivSwitch{}#4!#2!%
      }%
    \fi
  \BIC@Fi
}
%    \end{macrocode}
%    \end{macro}
%    \begin{macro}{\BIC@DivSwitch}
%    Decision table for \cs{BIC@DivSwitch}.
%    \begin{quote}
%    \begin{tabular}{@{}|l|l|l|@{}}
%    \hline
%    $y=x$ & \multicolumn{1}{l}{} & sign $1$\\
%    \hline
%    $y>x$ & \multicolumn{1}{l}{} & $0$\\
%    \hline
%    $y<x$ & $y=1$                & sign $x$\\
%    \cline{2-3}
%          & $y=2$                & sign Shr$(x)$\\
%    \cline{2-3}
%          & $y=4$                & sign Shr(Shr$(x)$)\\
%    \cline{2-3}
%          & else                 & sign ProcessDiv$(x,y)$\\
%    \hline
%    \end{tabular}
%    \end{quote}
%    |#1|: sign\\
%    |#2|: $x$\\
%    |#3#4|: $y$ ($y\ne 0$)
%    \begin{macrocode}
\def\BIC@DivSwitch#1#2!#3#4!{%
  \ifcase\BIC@PosCmp#3#4!#2!% y = x
    \BIC@AfterFi{ #11}%
  \or % y > x
    \BIC@AfterFi{ 0}%
  \else % y < x
    \ifx\\#1\\%
    \else
      \expandafter-\romannumeral0%
    \fi
    \ifcase\ifx\\#4\\%
             \ifx#310 % y = 1
             \else\ifx#321 % y = 2
             \else\ifx#342 % y = 4
             \else3 % y > 2
             \fi\fi\fi
           \else
             3 % y > 2
           \fi
      \BIC@AfterFiFi{ #2}% y = 1
    \or % y = 2
      \BIC@AfterFiFi{%
        \BIC@@Shr#2!%
      }%
    \or % y = 4
      \BIC@AfterFiFi{%
        \expandafter\BIC@@Shr\romannumeral0%
          \BIC@@Shr#2!!%
      }%
    \or % y > 2
      \BIC@AfterFiFi{%
        \BIC@DivStartX#2!#3#4!!!%
      }%
?   \else\BigIntCalcError:ThisCannotHappen%
    \fi
  \BIC@Fi
}
%    \end{macrocode}
%    \end{macro}
%    \begin{macro}{\BIC@ProcessDiv}
%    |#1#2|: $x$\\
%    |#3#4|: $y$\\
%    |#5|: collect first digits of $x$\\
%    |#6|: corresponding digits of $y$
%    \begin{macrocode}
\def\BIC@DivStartX#1#2!#3#4!#5!#6!{%
  \ifx\\#4\\%
    \BIC@AfterFi{%
      \BIC@DivStartYii#6#3#4!{#5#1}#2=!%
    }%
  \else
    \BIC@AfterFi{%
      \BIC@DivStartX#2!#4!#5#1!#6#3!%
    }%
  \BIC@Fi
}
%    \end{macrocode}
%    \end{macro}
%    \begin{macro}{\BIC@DivStartYii}
%    |#1|: $y$\\
%    |#2|: $x$, |=|
%    \begin{macrocode}
\def\BIC@DivStartYii#1!{%
  \expandafter\BIC@DivStartYiv\romannumeral0%
  \BIC@Shl#1!%
  !#1!%
}
%    \end{macrocode}
%    \end{macro}
%    \begin{macro}{\BIC@DivStartYiv}
%    |#1|: $2y$\\
%    |#2|: $y$\\
%    |#3|: $x$, |=|
%    \begin{macrocode}
\def\BIC@DivStartYiv#1!{%
  \expandafter\BIC@DivStartYvi\romannumeral0%
  \BIC@Shl#1!%
  !#1!%
}
%    \end{macrocode}
%    \end{macro}
%    \begin{macro}{\BIC@DivStartYvi}
%    |#1|: $4y$\\
%    |#2|: $2y$\\
%    |#3|: $y$\\
%    |#4|: $x$, |=|
%    \begin{macrocode}
\def\BIC@DivStartYvi#1!#2!{%
  \expandafter\BIC@DivStartYviii\romannumeral0%
  \BIC@AddXY#1!#2!!!%
  !#1!#2!%
}
%    \end{macrocode}
%    \end{macro}
%    \begin{macro}{\BIC@DivStartYviii}
%    |#1|: $6y$\\
%    |#2|: $4y$\\
%    |#3|: $2y$\\
%    |#4|: $y$\\
%    |#5|: $x$, |=|
%    \begin{macrocode}
\def\BIC@DivStartYviii#1!#2!{%
  \expandafter\BIC@DivStart\romannumeral0%
  \BIC@Shl#2!%
  !#1!#2!%
}
%    \end{macrocode}
%    \end{macro}
%    \begin{macro}{\BIC@DivStart}
%    |#1|: $8y$\\
%    |#2|: $6y$\\
%    |#3|: $4y$\\
%    |#4|: $2y$\\
%    |#5|: $y$\\
%    |#6|: $x$, |=|
%    \begin{macrocode}
\def\BIC@DivStart#1!#2!#3!#4!#5!#6!{%
  \BIC@ProcessDiv#6!!#5!#4!#3!#2!#1!=%
}
%    \end{macrocode}
%    \end{macro}
%    \begin{macro}{\BIC@ProcessDiv}
%    |#1#2#3|: $x$, |=|\\
%    |#4|: result\\
%    |#5|: $y$\\
%    |#6|: $2y$\\
%    |#7|: $4y$\\
%    |#8|: $6y$\\
%    |#9|: $8y$
%    \begin{macrocode}
\def\BIC@ProcessDiv#1#2#3!#4!#5!{%
  \ifcase\BIC@PosCmp#5!#1!% y = #1
    \ifx#2=%
      \BIC@AfterFiFi{\BIC@DivCleanup{#41}}%
    \else
      \BIC@AfterFiFi{%
        \BIC@ProcessDiv#2#3!#41!#5!%
      }%
    \fi
  \or % y > #1
    \ifx#2=%
      \BIC@AfterFiFi{\BIC@DivCleanup{#40}}%
    \else
      \ifx\\#4\\%
        \BIC@AfterFiFiFi{%
          \BIC@ProcessDiv{#1#2}#3!!#5!%
        }%
      \else
        \BIC@AfterFiFiFi{%
          \BIC@ProcessDiv{#1#2}#3!#40!#5!%
        }%
      \fi
    \fi
  \else % y < #1
    \BIC@AfterFi{%
      \BIC@@ProcessDiv{#1}#2#3!#4!#5!%
    }%
  \BIC@Fi
}
%    \end{macrocode}
%    \end{macro}
%    \begin{macro}{\BIC@DivCleanup}
%    |#1|: result\\
%    |#2|: garbage
%    \begin{macrocode}
\def\BIC@DivCleanup#1#2={ #1}%
%    \end{macrocode}
%    \end{macro}
%    \begin{macro}{\BIC@@ProcessDiv}
%    \begin{macrocode}
\def\BIC@@ProcessDiv#1#2#3!#4!#5!#6!#7!{%
  \ifcase\BIC@PosCmp#7!#1!% 4y = #1
    \ifx#2=%
      \BIC@AfterFiFi{\BIC@DivCleanup{#44}}%
    \else
     \BIC@AfterFiFi{%
       \BIC@ProcessDiv#2#3!#44!#5!#6!#7!%
     }%
    \fi
  \or % 4y > #1
    \ifcase\BIC@PosCmp#6!#1!% 2y = #1
      \ifx#2=%
        \BIC@AfterFiFiFi{\BIC@DivCleanup{#42}}%
      \else
        \BIC@AfterFiFiFi{%
          \BIC@ProcessDiv#2#3!#42!#5!#6!#7!%
        }%
      \fi
    \or % 2y > #1
      \ifx#2=%
        \BIC@AfterFiFiFi{\BIC@DivCleanup{#41}}%
      \else
        \BIC@AfterFiFiFi{%
          \BIC@DivSub#1!#5!#2#3!#41!#5!#6!#7!%
        }%
      \fi
    \else % 2y < #1
      \BIC@AfterFiFi{%
        \expandafter\BIC@ProcessDivII\romannumeral0%
        \BIC@SubXY#1!#6!!!%
        !#2#3!#4!#5!23%
        #6!#7!%
      }%
    \fi
  \else % 4y < #1
    \BIC@AfterFi{%
      \BIC@@@ProcessDiv{#1}#2#3!#4!#5!#6!#7!%
    }%
  \BIC@Fi
}
%    \end{macrocode}
%    \end{macro}
%    \begin{macro}{\BIC@DivSub}
%    Next token group: |#1|-|#2| and next digit |#3|.
%    \begin{macrocode}
\def\BIC@DivSub#1!#2!#3{%
  \expandafter\BIC@ProcessDiv\expandafter{%
    \romannumeral0%
    \BIC@SubXY#1!#2!!!%
    #3%
  }%
}
%    \end{macrocode}
%    \end{macro}
%    \begin{macro}{\BIC@ProcessDivII}
%    |#1|: $x'-2y$\\
%    |#2#3|: remaining $x$, |=|\\
%    |#4|: result\\
%    |#5|: $y$\\
%    |#6|: first possible result digit\\
%    |#7|: second possible result digit
%    \begin{macrocode}
\def\BIC@ProcessDivII#1!#2#3!#4!#5!#6#7{%
  \ifcase\BIC@PosCmp#5!#1!% y = #1
    \ifx#2=%
      \BIC@AfterFiFi{\BIC@DivCleanup{#4#7}}%
    \else
      \BIC@AfterFiFi{%
        \BIC@ProcessDiv#2#3!#4#7!#5!%
      }%
    \fi
  \or % y > #1
    \ifx#2=%
      \BIC@AfterFiFi{\BIC@DivCleanup{#4#6}}%
    \else
      \BIC@AfterFiFi{%
        \BIC@ProcessDiv{#1#2}#3!#4#6!#5!%
      }%
    \fi
  \else % y < #1
    \ifx#2=%
      \BIC@AfterFiFi{\BIC@DivCleanup{#4#7}}%
    \else
      \BIC@AfterFiFi{%
        \BIC@DivSub#1!#5!#2#3!#4#7!#5!%
      }%
    \fi
  \BIC@Fi
}
%    \end{macrocode}
%    \end{macro}
%    \begin{macro}{\BIC@ProcessDivIV}
%    |#1#2#3|: $x$, |=|, $x>4y$\\
%    |#4|: result\\
%    |#5|: $y$\\
%    |#6|: $2y$\\
%    |#7|: $4y$\\
%    |#8|: $6y$\\
%    |#9|: $8y$
%    \begin{macrocode}
\def\BIC@@@ProcessDiv#1#2#3!#4!#5!#6!#7!#8!#9!{%
  \ifcase\BIC@PosCmp#8!#1!% 6y = #1
    \ifx#2=%
      \BIC@AfterFiFi{\BIC@DivCleanup{#46}}%
    \else
      \BIC@AfterFiFi{%
        \BIC@ProcessDiv#2#3!#46!#5!#6!#7!#8!#9!%
      }%
    \fi
  \or % 6y > #1
    \BIC@AfterFi{%
      \expandafter\BIC@ProcessDivII\romannumeral0%
      \BIC@SubXY#1!#7!!!%
      !#2#3!#4!#5!45%
      #6!#7!#8!#9!%
    }%
  \else % 6y < #1
    \ifcase\BIC@PosCmp#9!#1!% 8y = #1
      \ifx#2=%
        \BIC@AfterFiFiFi{\BIC@DivCleanup{#48}}%
      \else
        \BIC@AfterFiFiFi{%
          \BIC@ProcessDiv#2#3!#48!#5!#6!#7!#8!#9!%
        }%
      \fi
    \or % 8y > #1
      \BIC@AfterFiFi{%
        \expandafter\BIC@ProcessDivII\romannumeral0%
        \BIC@SubXY#1!#8!!!%
        !#2#3!#4!#5!67%
        #6!#7!#8!#9!%
      }%
    \else % 8y < #1
      \BIC@AfterFiFi{%
        \expandafter\BIC@ProcessDivII\romannumeral0%
        \BIC@SubXY#1!#9!!!%
        !#2#3!#4!#5!89%
        #6!#7!#8!#9!%
      }%
    \fi
  \BIC@Fi
}
%    \end{macrocode}
%    \end{macro}
%
% \subsection{\op{Mod}}
%
%    \begin{macro}{\bigintcalcMod}
%    |#1|: $x$\\
%    |#2|: $y$
%    \begin{macrocode}
\def\bigintcalcMod#1{%
  \romannumeral0%
  \expandafter\expandafter\expandafter\BIC@Mod
  \bigintcalcNum{#1}!%
}
%    \end{macrocode}
%    \end{macro}
%    \begin{macro}{\BIC@Mod}
%    |#1|: $x$\\
%    |#2|: $y$
%    \begin{macrocode}
\def\BIC@Mod#1!#2{%
  \expandafter\expandafter\expandafter\BIC@ModSwitchSign
  \bigintcalcNum{#2}!#1!%
}
%    \end{macrocode}
%    \end{macro}
%    \begin{macro}{\BigIntCalcMod}
%    \begin{macrocode}
\def\BigIntCalcMod#1!#2!{%
  \romannumeral0%
  \BIC@ModSwitchSign#2!#1!%
}
%    \end{macrocode}
%    \end{macro}
%    \begin{macro}{\BIC@ModSwitchSign}
%    Decision table for \cs{BIC@ModSwitchSign}.
%    \begin{quote}
%    \begin{tabular}{@{}|l|l|l|@{}}
%    \hline
%    $y=0$ & \multicolumn{1}{l}{} & DivisionByZero\\
%    \hline
%    $y>0$ & $x=0$ & $0$\\
%    \cline{2-3}
%          & else  & ModSwitch$(+,x,y)$\\
%    \hline
%    $y<0$ & \multicolumn{1}{l}{} & ModSwitch$(-,-x,-y)$\\
%    \hline
%    \end{tabular}
%    \end{quote}
%    |#1#2|: $y$\\
%    |#3#4|: $x$
%    \begin{macrocode}
\def\BIC@ModSwitchSign#1#2!#3#4!{%
  \ifcase\ifx\\#2\\%
           \ifx#100 % y = 0
           \else1 % y > 0
           \fi
          \else
            \ifx#1-2 % y < 0
            \else1 % y > 0
            \fi
          \fi
    \BIC@AfterFi{ 0\BigIntCalcError:DivisionByZero}%
  \or % y > 0
    \ifcase\ifx\\#4\\\ifx#300 \else1 \fi\else1 \fi % x = 0
      \BIC@AfterFiFi{ 0}%
    \else
      \BIC@AfterFiFi{%
        \BIC@ModSwitch{}#3#4!#1#2!%
      }%
    \fi
  \else % y < 0
    \ifcase\ifx\\#4\\%
             \ifx#300 % x = 0
             \else1 % x > 0
             \fi
           \else
             \ifx#3-2 % x < 0
             \else1 % x > 0
             \fi
           \fi
      \BIC@AfterFiFi{ 0}%
    \or % x > 0
      \BIC@AfterFiFi{%
        \BIC@ModSwitch--#3#4!#2!%
      }%
    \else % x < 0
      \BIC@AfterFiFi{%
        \BIC@ModSwitch-#4!#2!%
      }%
    \fi
  \BIC@Fi
}
%    \end{macrocode}
%    \end{macro}
%    \begin{macro}{\BIC@ModSwitch}
%    Decision table for \cs{BIC@ModSwitch}.
%    \begin{quote}
%    \begin{tabular}{@{}|l|l|l|@{}}
%    \hline
%    $y=1$ & \multicolumn{1}{l}{} & $0$\\
%    \hline
%    $y=2$ & ifodd$(x)$ & sign $1$\\
%    \cline{2-3}
%          & else       & $0$\\
%    \hline
%    $y>2$ & $x<0$      &
%        $z\leftarrow x-(x/y)*y;\quad (z<0)\mathbin{?}z+y\mathbin{:}z$\\
%    \cline{2-3}
%          & $x>0$      & $x - (x/y) * y$\\
%    \hline
%    \end{tabular}
%    \end{quote}
%    |#1|: sign\\
%    |#2#3|: $x$\\
%    |#4#5|: $y$
%    \begin{macrocode}
\def\BIC@ModSwitch#1#2#3!#4#5!{%
  \ifcase\ifx\\#5\\%
           \ifx#410 % y = 1
           \else\ifx#421 % y = 2
           \else2 % y > 2
           \fi\fi
         \else2 % y > 2
         \fi
    \BIC@AfterFi{ 0}% y = 1
  \or % y = 2
    \ifcase\BIC@ModTwo#2#3! % even(x)
      \BIC@AfterFiFi{ 0}%
    \or % odd(x)
      \BIC@AfterFiFi{ #11}%
?   \else\BigIntCalcError:ThisCannotHappen%
    \fi
  \or % y > 2
    \ifx\\#1\\%
    \else
      \expandafter\BIC@Space\romannumeral0%
      \expandafter\BIC@ModMinus\romannumeral0%
    \fi
    \ifx#2-% x < 0
      \BIC@AfterFiFi{%
        \expandafter\expandafter\expandafter\BIC@ModX
        \bigintcalcSub{#2#3}{%
          \bigintcalcMul{#4#5}{\bigintcalcDiv{#2#3}{#4#5}}%
        }!#4#5!%
      }%
    \else % x > 0
      \BIC@AfterFiFi{%
        \expandafter\expandafter\expandafter\BIC@Space
        \bigintcalcSub{#2#3}{%
          \bigintcalcMul{#4#5}{\bigintcalcDiv{#2#3}{#4#5}}%
        }%
      }%
    \fi
? \else\BigIntCalcError:ThisCannotHappen%
  \BIC@Fi
}
%    \end{macrocode}
%    \end{macro}
%    \begin{macro}{\BIC@ModMinus}
%    \begin{macrocode}
\def\BIC@ModMinus#1{%
  \ifx#10%
    \BIC@AfterFi{ 0}%
  \else
    \BIC@AfterFi{ -#1}%
  \BIC@Fi
}
%    \end{macrocode}
%    \end{macro}
%    \begin{macro}{\BIC@ModX}
%    |#1#2|: $z$\\
%    |#3|: $x$
%    \begin{macrocode}
\def\BIC@ModX#1#2!#3!{%
  \ifx#1-% z < 0
    \BIC@AfterFi{%
      \expandafter\BIC@Space\romannumeral0%
      \BIC@SubXY#3!#2!!!%
    }%
  \else % z >= 0
    \BIC@AfterFi{ #1#2}%
  \BIC@Fi
}
%    \end{macrocode}
%    \end{macro}
%
%    \begin{macrocode}
\BIC@AtEnd%
%    \end{macrocode}
%
%    \begin{macrocode}
%</package>
%    \end{macrocode}
%
% \section{Test}
%
% \subsection{Catcode checks for loading}
%
%    \begin{macrocode}
%<*test1>
%    \end{macrocode}
%    \begin{macrocode}
\catcode`\{=1 %
\catcode`\}=2 %
\catcode`\#=6 %
\catcode`\@=11 %
\expandafter\ifx\csname count@\endcsname\relax
  \countdef\count@=255 %
\fi
\expandafter\ifx\csname @gobble\endcsname\relax
  \long\def\@gobble#1{}%
\fi
\expandafter\ifx\csname @firstofone\endcsname\relax
  \long\def\@firstofone#1{#1}%
\fi
\expandafter\ifx\csname loop\endcsname\relax
  \expandafter\@firstofone
\else
  \expandafter\@gobble
\fi
{%
  \def\loop#1\repeat{%
    \def\body{#1}%
    \iterate
  }%
  \def\iterate{%
    \body
      \let\next\iterate
    \else
      \let\next\relax
    \fi
    \next
  }%
  \let\repeat=\fi
}%
\def\RestoreCatcodes{}
\count@=0 %
\loop
  \edef\RestoreCatcodes{%
    \RestoreCatcodes
    \catcode\the\count@=\the\catcode\count@\relax
  }%
\ifnum\count@<255 %
  \advance\count@ 1 %
\repeat

\def\RangeCatcodeInvalid#1#2{%
  \count@=#1\relax
  \loop
    \catcode\count@=15 %
  \ifnum\count@<#2\relax
    \advance\count@ 1 %
  \repeat
}
\def\RangeCatcodeCheck#1#2#3{%
  \count@=#1\relax
  \loop
    \ifnum#3=\catcode\count@
    \else
      \errmessage{%
        Character \the\count@\space
        with wrong catcode \the\catcode\count@\space
        instead of \number#3%
      }%
    \fi
  \ifnum\count@<#2\relax
    \advance\count@ 1 %
  \repeat
}
\def\space{ }
\expandafter\ifx\csname LoadCommand\endcsname\relax
  \def\LoadCommand{\input bigintcalc.sty\relax}%
\fi
\def\Test{%
  \RangeCatcodeInvalid{0}{47}%
  \RangeCatcodeInvalid{58}{64}%
  \RangeCatcodeInvalid{91}{96}%
  \RangeCatcodeInvalid{123}{255}%
  \catcode`\@=12 %
  \catcode`\\=0 %
  \catcode`\%=14 %
  \LoadCommand
  \RangeCatcodeCheck{0}{36}{15}%
  \RangeCatcodeCheck{37}{37}{14}%
  \RangeCatcodeCheck{38}{47}{15}%
  \RangeCatcodeCheck{48}{57}{12}%
  \RangeCatcodeCheck{58}{63}{15}%
  \RangeCatcodeCheck{64}{64}{12}%
  \RangeCatcodeCheck{65}{90}{11}%
  \RangeCatcodeCheck{91}{91}{15}%
  \RangeCatcodeCheck{92}{92}{0}%
  \RangeCatcodeCheck{93}{96}{15}%
  \RangeCatcodeCheck{97}{122}{11}%
  \RangeCatcodeCheck{123}{255}{15}%
  \RestoreCatcodes
}
\Test
\csname @@end\endcsname
\end
%    \end{macrocode}
%    \begin{macrocode}
%</test1>
%    \end{macrocode}
%
% \subsection{Macro tests}
%
% \subsubsection{Preamble with test macro definitions}
%
%    \begin{macrocode}
%<*test2>
\NeedsTeXFormat{LaTeX2e}
\nofiles
\documentclass{article}
%<noetex>\let\SavedNumexpr\numexpr
%<noetex>\let\numexpr\UNDEFINED
\makeatletter
\chardef\BIC@TestMode=1 %
\makeatother
\usepackage{bigintcalc}[2016/05/16]
%<noetex>\let\numexpr\SavedNumexpr
\usepackage{qstest}
\IncludeTests{*}
\LogTests{log}{*}{*}
\newcommand*{\TestSpaceAtEnd}[1]{%
%<noetex>  \let\SavedNumexpr\numexpr
%<noetex>  \let\numexpr\UNDEFINED
  \edef\resultA{#1}%
  \edef\resultB{#1 }%
%<noetex>  \let\numexpr\SavedNumexpr
  \Expect*{\resultA\space}*{\resultB}%
}
\newcommand*{\TestResult}[2]{%
%<noetex>  \let\SavedNumexpr\numexpr
%<noetex>  \let\numexpr\UNDEFINED
  \edef\result{#1}%
%<noetex>  \let\numexpr\SavedNumexpr
  \Expect*{\result}{#2}%
}
\newcommand*{\TestResultTwoExpansions}[2]{%
%<*noetex>
  \begingroup
    \let\numexpr\UNDEFINED
    \expandafter\expandafter\expandafter
  \endgroup
%</noetex>
  \expandafter\expandafter\expandafter\Expect
  \expandafter\expandafter\expandafter{#1}{#2}%
}
\newcount\TestCount
%<etex>\newcommand*{\TestArg}[1]{\numexpr#1\relax}
%<noetex>\newcommand*{\TestArg}[1]{#1}
\newcommand*{\TestTeXDivide}[2]{%
  \TestCount=\TestArg{#1}\relax
  \divide\TestCount by \TestArg{#2}\relax
  \Expect*{\bigintcalcDiv{#1}{#2}}*{\the\TestCount}%
}
\newcommand*{\Test}[2]{%
  \TestResult{#1}{#2}%
  \TestResultTwoExpansions{#1}{#2}%
  \TestSpaceAtEnd{#1}%
}
\newcommand*{\TestExch}[2]{\Test{#2}{#1}}
\newcommand*{\TestInv}[2]{%
  \Test{\bigintcalcInv{#1}}{#2}%
}
\newcommand*{\TestAbs}[2]{%
  \Test{\bigintcalcAbs{#1}}{#2}%
}
\newcommand*{\TestSgn}[2]{%
  \Test{\bigintcalcSgn{#1}}{#2}%
}
\newcommand*{\TestMin}[3]{%
  \Test{\bigintcalcMin{#1}{#2}}{#3}%
}
\newcommand*{\TestMax}[3]{%
  \Test{\bigintcalcMax{#1}{#2}}{#3}%
}
\newcommand*{\TestCmp}[3]{%
  \Test{\bigintcalcCmp{#1}{#2}}{#3}%
}
\newcommand*{\TestOdd}[2]{%
  \Test{\bigintcalcOdd{#1}}{#2}%
  \edef\x{%
    \noexpand\Test{%
      \noexpand\BigIntCalcOdd
      \bigintcalcAbs{#1}!%
    }{#2}%
  }%
  \x
}
\newcommand*{\TestInc}[2]{%
  \Test{\bigintcalcInc{#1}}{#2}%
  \ifnum\bigintcalcSgn{#1}>-1 %
    \edef\x{%
      \noexpand\Test{%
        \noexpand\BigIntCalcInc\bigintcalcNum{#1}!%
      }{#2}%
    }%
    \x
  \fi
}
\newcommand*{\TestDec}[2]{%
  \Test{\bigintcalcDec{#1}}{#2}%
  \ifnum\bigintcalcSgn{#1}>0 %
    \edef\x{%
      \noexpand\Test{%
        \noexpand\BigIntCalcDec\bigintcalcNum{#1}!%
      }{#2}%
    }%
    \x
  \fi
}
\newcommand*{\TestAdd}[3]{%
  \Test{\bigintcalcAdd{#1}{#2}}{#3}%
  \ifnum\bigintcalcSgn{#1}>0 %
    \ifnum\bigintcalcSgn{#2}> 0 %
      \ifnum\bigintcalcCmp{#1}{#2}>0 %
        \edef\x{%
          \noexpand\Test{%
            \noexpand\BigIntCalcAdd
            \bigintcalcNum{#1}!\bigintcalcNum{#2}!%
          }{#3}%
        }%
        \x
      \else
        \edef\x{%
          \noexpand\Test{%
            \noexpand\BigIntCalcAdd
            \bigintcalcNum{#2}!\bigintcalcNum{#1}!%
          }{#3}%
        }%
        \x
      \fi
    \fi
  \fi
}
\newcommand*{\TestSub}[3]{%
  \Test{\bigintcalcSub{#1}{#2}}{#3}%
  \ifnum\bigintcalcSgn{#1}>0 %
    \ifnum\bigintcalcSgn{#2}> 0 %
      \ifnum\bigintcalcCmp{#1}{#2}>0 %
        \edef\x{%
          \noexpand\Test{%
            \noexpand\BigIntCalcSub
            \bigintcalcNum{#1}!\bigintcalcNum{#2}!%
          }{#3}%
        }%
        \x
      \fi
    \fi
  \fi
}
\newcommand*{\TestShl}[2]{%
  \Test{\bigintcalcShl{#1}}{#2}%
  \edef\x{%
    \noexpand\Test{%
      \noexpand\BigIntCalcShl\bigintcalcAbs{#1}!%
    }{\bigintcalcAbs{#2}}%
  }%
  \x
}
\newcommand*{\TestShr}[2]{%
  \Test{\bigintcalcShr{#1}}{#2}%
  \edef\x{%
    \noexpand\Test{%
      \noexpand\BigIntCalcShr\bigintcalcAbs{#1}!%
    }{\bigintcalcAbs{#2}}%
  }%
  \x
}
\newcommand*{\TestMul}[3]{%
  \Test{\bigintcalcMul{#1}{#2}}{#3}%
  \edef\x{%
    \noexpand\Test{%
      \noexpand\BigIntCalcMul
      \bigintcalcAbs{#1}!\bigintcalcAbs{#2}!%
    }{\bigintcalcAbs{#3}}%
  }%
  \x
}
\newcommand*{\TestSqr}[2]{%
  \Test{\bigintcalcSqr{#1}}{#2}%
}
\newcommand*{\TestFac}[2]{%
  \expandafter\TestExch\expandafter{%
    \the\numexpr#2%
  }{\bigintcalcFac{#1}}%
}
\newcommand*{\TestFacBig}[2]{%
  \Test{\bigintcalcFac{#1}}{#2}%
}
\newcommand*{\TestPow}[3]{%
  \Test{\bigintcalcPow{#1}{#2}}{#3}%
}
\newcommand*{\TestDiv}[3]{%
  \Test{\bigintcalcDiv{#1}{#2}}{#3}%
  \TestTeXDivide{#1}{#2}%
}
\newcommand*{\TestDivBig}[3]{%
  \Test{\bigintcalcDiv{#1}{#2}}{#3}%
  \edef\x{%
    \noexpand\Test{%
      \noexpand\BigIntCalcDiv\bigintcalcAbs{#1}!\bigintcalcAbs{#2}!%
    }{\bigintcalcAbs{#3}}%
  }%
}
\newcommand*{\TestMod}[3]{%
  \Test{\bigintcalcMod{#1}{#2}}{#3}%
  \ifcase\ifcase\bigintcalcSgn{#1} 0%
         \or
           \ifcase\bigintcalcSgn{#2} 1%
           \or 0%
           \else 1%
           \fi
         \else
           \ifcase\bigintcalcSgn{#2} 1%
           \or 1%
           \else 0%
           \fi
         \fi\relax
    \edef\x{%
      \noexpand\Test{%
        \noexpand\BigIntCalcMod
        \bigintcalcAbs{#1}!\bigintcalcAbs{#2}!%
      }{\bigintcalcAbs{#3}}%
    }%
    \x
  \fi
}
%    \end{macrocode}
%
% \subsubsection{Time}
%
%    \begin{macrocode}
\begingroup\expandafter\expandafter\expandafter\endgroup
\expandafter\ifx\csname pdfresettimer\endcsname\relax
\else
  \makeatletter
  \newcount\SummaryTime
  \newcount\TestTime
  \SummaryTime=\z@
  \newcommand*{\PrintTime}[2]{%
    \typeout{%
      [Time #1: \strip@pt\dimexpr\number#2sp\relax\space s]%
    }%
  }%
  \newcommand*{\StartTime}[1]{%
    \renewcommand*{\TimeDescription}{#1}%
    \pdfresettimer
  }%
  \newcommand*{\TimeDescription}{}%
  \newcommand*{\StopTime}{%
    \TestTime=\pdfelapsedtime
    \global\advance\SummaryTime\TestTime
    \PrintTime\TimeDescription\TestTime
  }%
  \let\saved@qstest\qstest
  \let\saved@endqstest\endqstest
  \def\qstest#1#2{%
    \saved@qstest{#1}{#2}%
    \StartTime{#1}%
  }%
  \def\endqstest{%
    \StopTime
    \saved@endqstest
  }%
  \AtEndDocument{%
    \PrintTime{summary}\SummaryTime
  }%
  \makeatother
\fi
%    \end{macrocode}
%
% \subsubsection{Test sets}
%
%    \begin{macrocode}
\makeatletter

\begin{qstest}{inv}{inv}%
  \TestInv{0}{0}%
  \TestInv{1}{-1}%
  \TestInv{-1}{1}%
  \TestInv{10}{-10}%
  \TestInv{-10}{10}%
  \TestInv{2147483647}{-2147483647}%
  \TestInv{-2147483647}{2147483647}%
  \TestInv{12345678901234567890}{-12345678901234567890}%
  \TestInv{-12345678901234567890}{12345678901234567890}%
  \TestInv{ 0 }{0}%
  \TestInv{ 1 }{-1}%
  \TestInv{--1}{-1}%
  \TestInv{\number\z@}{0}%
  \TestInv{\ifx\relax\relax1\fi}{-1}%
  \TestInv{\ifx\relax\relax-\fi\ifx234\else1\fi}{1}%
\end{qstest}

\begin{qstest}{abs}{abs}%
  \TestAbs{0}{0}%
  \TestAbs{1}{1}%
  \TestAbs{-1}{1}%
  \TestAbs{10}{10}%
  \TestAbs{-10}{10}%
  \TestAbs{2147483647}{2147483647}%
  \TestAbs{-2147483647}{2147483647}%
  \TestAbs{12345678901234567890}{12345678901234567890}%
  \TestAbs{-12345678901234567890}{12345678901234567890}%
  \TestAbs{ 0 }{0}%
  \TestAbs{ 1 }{1}%
  \TestAbs{--1}{1}%
  \TestAbs{-+-+1}{1}%
  \TestAbs{00000000000}{0}%
  \TestAbs{00000001000}{1000}%
  \TestAbs{\ifx\relax\relax 0\else 1\fi}{0}%
\end{qstest}

\begin{qstest}{sign}{sign}%
  \TestSgn{0}{0}%
  \TestSgn{1}{1}%
  \TestSgn{-1}{-1}%
  \TestSgn{10}{1}%
  \TestSgn{-10}{-1}%
  \TestSgn{2147483647}{1}%
  \TestSgn{-2147483647}{-1}%
  \TestSgn{12345678901234567890}{1}%
  \TestSgn{-12345678901234567890}{-1}%
  \TestSgn{ 0 }{0}%
  \TestSgn{ 2 }{1}%
  \TestSgn{ -2 }{-1}%
  \TestSgn{--2}{1}%
  \TestSgn{\number\z@}{0}%
  \TestSgn{\number\@ne}{1}%
  \TestSgn{\number\m@ne}{-1}%
  \TestSgn{%
    -+-+\number\z@\number\z@
    \iftrue1\fi\iftrue2\fi\iftrue3\fi
  }{1}%
\end{qstest}

\begin{qstest}{min}{min}%
  \TestMin{0}{1}{0}%
  \TestMin{1}{0}{0}%
  \TestMin{-10}{-20}{-20}%
  \TestMin{ 1 }{ 2 }{1}%
  \TestMin{ 2 }{ 1 }{1}%
  \TestMin{1}{1}{1}%
  \TestMin{\number\z@}{\number\@ne}{0}%
  \TestMin{\number\@ne}{\number\m@ne}{-1}%
\end{qstest}

\begin{qstest}{max}{max}%
  \TestMax{0}{1}{1}%
  \TestMax{1}{0}{1}%
  \TestMax{-10}{-20}{-10}%
  \TestMax{ 1 }{ 2 }{2}%
  \TestMax{ 2 }{ 1 }{2}%
  \TestMax{1}{1}{1}%
  \TestMax{\number\z@}{\number\@ne}{1}%
  \TestMax{\number\@ne}{\number\m@ne}{1}%
\end{qstest}

\begin{qstest}{cmp}{cmp}%
  \TestCmp{0}{0}{0}%
  \TestCmp{-21}{17}{-1}%
  \TestCmp{3}{4}{-1}%
  \TestCmp{-10}{-10}{0}%
  \TestCmp{-10}{-11}{1}%
  \TestCmp{100}{5}{1}%
  \TestCmp{9}{10}{-1}%
  \TestCmp{10}{9}{1}%
  \TestCmp{ 3 }{ 3 }{0}%
  \TestCmp{-9}{-10}{1}%
  \TestCmp{-10}{-9}{-1}%
  \TestCmp{-3}{-3}{0}%
  \TestCmp{0}{-2}{1}%
  \TestCmp{0}{2}{-1}%
  \TestCmp{2}{0}{1}%
  \TestCmp{-2}{0}{-1}%
  \TestCmp{12}{11}{1}%
  \TestCmp{11}{12}{-1}%
  \TestCmp{2147483647}{-2147483647}{1}%
  \TestCmp{-2147483647}{2147483647}{-1}%
  \TestCmp{2147483647}{2147483647}{0}%
  \TestCmp{\number\z@}{\number\@ne}{-1}%
  \TestCmp{\number\@ne}{\number\m@ne}{1}%
  \TestCmp{ 4 }{ 5 }{-1}%
  \TestCmp{ -3 }{ -7 }{1}%
\end{qstest}

\begin{qstest}{odd}{odd}
\tracingmacros=1
  \TestOdd{0}{0}%
  \TestOdd{1}{1}%
  \TestOdd{2}{0}%
  \TestOdd{3}{1}%
  \TestOdd{14}{0}%
  \TestOdd{15}{1}%
  \TestOdd{12345678901234567896}{0}%
  \TestOdd{12345678901234567897}{1}%
\end{qstest}

\begin{qstest}{inc}{inc}%
  \TestInc{0}{1}%
  \TestInc{1}{2}%
  \TestInc{-1}{0}%
  \TestInc{10}{11}%
  \TestInc{-10}{-9}%
  \TestInc{ 3 }{4}%
  \TestInc{999}{1000}%
  \TestInc{-1000}{-999}%
  \TestInc{129}{130}%
  \TestInc{2147483646}{2147483647}%
  \TestInc{-2147483647}{-2147483646}%
  \TestInc{12345678901234567890}{12345678901234567891}%
  \TestInc{99999999999999999999}{100000000000000000000}%
  \TestInc{-12345678901234567891}{-12345678901234567890}%
  \TestInc{-100000000000000000000}{-99999999999999999999}%
\end{qstest}

\begin{qstest}{dec}{dec}%
  \TestDec{0}{-1}%
  \TestDec{1}{0}%
  \TestDec{-1}{-2}%
  \TestDec{10}{9}%
  \TestDec{-10}{-11}%
  \TestDec{1000}{999}%
  \TestDec{-999}{-1000}%
  \TestDec{130}{129}%
  \TestDec{2147483647}{2147483646}%
  \TestDec{-2147483646}{-2147483647}%
  \TestDec{12345678901234567891}{12345678901234567890}%
  \TestDec{100000000000000000000}{99999999999999999999}%
  \TestDec{-12345678901234567890}{-12345678901234567891}%
  \TestDec{-99999999999999999999}{-100000000000000000000}%
\end{qstest}

\begin{qstest}{add}{add}%
  \TestAdd{0}{0}{0}%
  \TestAdd{1}{0}{1}%
  \TestAdd{0}{1}{1}%
  \TestAdd{1}{2}{3}%
  \TestAdd{-1}{-1}{-2}%
  \TestAdd{2147483646}{1}{2147483647}%
  \TestAdd{-2147483647}{2147483647}{0}%
  \TestAdd{20}{-5}{15}%
  \TestAdd{-4}{-1}{-5}%
  \TestAdd{-1}{-4}{-5}%
  \TestAdd{-4}{1}{-3}%
  \TestAdd{-1}{4}{3}%
  \TestAdd{4}{-1}{3}%
  \TestAdd{1}{-4}{-3}%
  \TestAdd{-4}{-1}{-5}%
  \TestAdd{-1}{-4}{-5}%
  \TestAdd{ -4 }{ -1 }{-5}%
  \TestAdd{ -1 }{ -4 }{-5}%
  \TestAdd{ -4 }{ 1 }{-3}%
  \TestAdd{ -1 }{ 4 }{3}%
  \TestAdd{ 4 }{ -1 }{3}%
  \TestAdd{ 1 }{ -4 }{-3}%
  \TestAdd{ -4 }{ -1 }{-5}%
  \TestAdd{ -1 }{ -4 }{-5}%
  \TestAdd{876543210}{111111111}{987654321}%
  \TestAdd{999999999}{2}{1000000001}%
\end{qstest}

\begin{qstest}{sub}{sub}
  \TestSub{0}{0}{0}%
  \TestSub{1}{0}{1}%
  \TestSub{1}{2}{-1}%
  \TestSub{-1}{-1}{0}%
  \TestSub{2147483646}{-1}{2147483647}%
  \TestSub{-2147483647}{-2147483647}{0}%
  \TestSub{-4}{-1}{-3}%
  \TestSub{-1}{-4}{3}%
  \TestSub{-4}{1}{-5}%
  \TestSub{-1}{4}{-5}%
  \TestSub{4}{-1}{5}%
  \TestSub{1}{-4}{5}%
  \TestSub{-4}{-1}{-3}%
  \TestSub{-1}{-4}{3}%
  \TestSub{ -4 }{ -1 }{-3}%
  \TestSub{ -1 }{ -4 }{3}%
  \TestSub{ -4 }{ 1 }{-5}%
  \TestSub{ -1 }{ 4 }{-5}%
  \TestSub{ 4 }{ -1 }{5}%
  \TestSub{ 1 }{ -4 }{5}%
  \TestSub{ -4 }{ -1 }{-3}%
  \TestSub{ -1 }{ -4 }{3}%
  \TestSub{1000000000}{2}{999999998}%
  \TestSub{987654321}{111111111}{876543210}%
\end{qstest}

\begin{qstest}{shl}{shl}
  \TestShl{0}{0}%
  \TestShl{1}{2}%
  \TestShl{2}{4}%
  \TestShl{5621}{11242}%
  \TestShl{1073741823}{2147483646}%
\end{qstest}

\begin{qstest}{shr}{shr}
  \TestShr{0}{0}%
  \TestShr{1}{0}%
  \TestShr{2}{1}%
  \TestShr{3}{1}%
  \TestShr{4}{2}%
  \TestShr{5}{2}%
  \TestShr{6}{3}%
  \TestShr{7}{3}%
  \TestShr{8}{4}%
  \TestShr{9}{4}%
  \TestShr{10}{5}%
  \TestShr{11}{5}%
  \TestShr{12}{6}%
  \TestShr{13}{6}%
  \TestShr{14}{7}%
  \TestShr{15}{7}%
  \TestShr{16}{8}%
  \TestShr{17}{8}%
  \TestShr{18}{9}%
  \TestShr{19}{9}%
  \TestShr{20}{10}%
  \TestShr{21}{10}%
  \TestShr{22}{11}%
  \TestShr{11241}{5620}%
  \TestShr{73054202}{36527101}%
  \TestShr{2147483646}{1073741823}%
\end{qstest}

\begin{qstest}{mul}{mul}
  \TestMul{0}{0}{0}%
  \TestMul{1}{0}{0}%
  \TestMul{0}{1}{0}%
  \TestMul{1}{1}{1}%
  \TestMul{3}{1}{3}%
  \TestMul{1}{-3}{-3}%
  \TestMul{-4}{-5}{20}%
  \TestMul{3}{7}{21}%
  \TestMul{7}{3}{21}%
  \TestMul{3}{-7}{-21}%
  \TestMul{7}{-3}{-21}%
  \TestMul{-3}{7}{-21}%
  \TestMul{-7}{3}{-21}%
  \TestMul{-3}{-7}{21}%
  \TestMul{-7}{-3}{21}%
  \TestMul{12}{11}{132}%
  \TestMul{999}{333}{332667}%
  \TestMul{1000}{4321}{4321000}%
  \TestMul{12345}{173955}{2147474475}%
  \TestMul{1073741823}{2}{2147483646}%
  \TestMul{2}{1073741823}{2147483646}%
  \TestMul{-1073741823}{2}{-2147483646}%
  \TestMul{2}{-1073741823}{-2147483646}%
  \TestMul{6706022400}{13}{87178291200}%
\end{qstest}

\begin{qstest}{sqr}{sqr}
  \TestSqr{0}{0}%
  \TestSqr{1}{1}%
  \TestSqr{2}{4}%
  \TestSqr{3}{9}%
  \TestSqr{4}{16}%
  \TestSqr{9}{81}%
  \TestSqr{10}{100}%
  \TestSqr{46340}{2147395600}%
  \TestSqr{-1}{1}%
  \TestSqr{-2}{4}%
  \TestSqr{-46340}{2147395600}%
\end{qstest}

\begin{qstest}{fac}{fac}
  \TestFac{0}{1}%
  \TestFac{1}{1}%
  \TestFac{2}{2}%
  \TestFac{3}{2*3}%
  \TestFac{4}{2*3*4}%
  \TestFac{5}{2*3*4*5}%
  \TestFac{6}{2*3*4*5*6}%
  \TestFac{7}{2*3*4*5*6*7}%
  \TestFac{8}{2*3*4*5*6*7*8}%
  \TestFac{9}{2*3*4*5*6*7*8*9}%
  \TestFac{10}{2*3*4*5*6*7*8*9*10}%
  \TestFac{11}{2*3*4*5*6*7*8*9*10*11}%
  \TestFac{12}{2*3*4*5*6*7*8*9*10*11*12}%
  \TestFacBig{13}{6227020800}%
  \TestFacBig{14}{87178291200}%
  \TestFacBig{15}{1307674368000}%
  \TestFacBig{16}{20922789888000}%
  \TestFacBig{17}{355687428096000}%
  \TestFacBig{18}{6402373705728000}%
  \TestFacBig{19}{121645100408832000}%
  \TestFacBig{20}{2432902008176640000}%
  \TestFacBig{21}{51090942171709440000}%
  \TestFacBig{22}{1124000727777607680000}%
\end{qstest}

\begin{qstest}{pow}{pow}
  \TestPow{-2}{0}{1}%
  \TestPow{-1}{0}{1}%
  \TestPow{0}{0}{1}%
  \TestPow{1}{0}{1}%
  \TestPow{2}{0}{1}%
  \TestPow{3}{0}{1}%
  \TestPow{-2}{1}{-2}%
  \TestPow{-1}{1}{-1}%
  \TestPow{1}{1}{1}%
  \TestPow{2}{1}{2}%
  \TestPow{3}{1}{3}%
  \TestPow{-2}{2}{4}%
  \TestPow{-1}{2}{1}%
  \TestPow{0}{2}{0}%
  \TestPow{1}{2}{1}%
  \TestPow{2}{2}{4}%
  \TestPow{3}{2}{9}%
  \TestPow{0}{1}{0}%
  \TestPow{1}{-2}{1}%
  \TestPow{1}{-1}{1}%
  \TestPow{-1}{-2}{1}%
  \TestPow{-1}{-1}{-1}%
  \TestPow{-1}{3}{-1}%
  \TestPow{-1}{4}{1}%
  \TestPow{-2}{-1}{0}%
  \TestPow{-2}{-2}{0}%
  \TestPow{2}{3}{8}%
  \TestPow{2}{4}{16}%
  \TestPow{2}{5}{32}%
  \TestPow{2}{6}{64}%
  \TestPow{2}{7}{128}%
  \TestPow{2}{8}{256}%
  \TestPow{2}{9}{512}%
  \TestPow{2}{10}{1024}%
  \TestPow{-2}{3}{-8}%
  \TestPow{-2}{4}{16}%
  \TestPow{-2}{5}{-32}%
  \TestPow{-2}{6}{64}%
  \TestPow{-2}{7}{-128}%
  \TestPow{-2}{8}{256}%
  \TestPow{-2}{9}{-512}%
  \TestPow{-2}{10}{1024}%
  \TestPow{3}{3}{27}%
  \TestPow{3}{4}{81}%
  \TestPow{3}{5}{243}%
  \TestPow{-3}{3}{-27}%
  \TestPow{-3}{4}{81}%
  \TestPow{-3}{5}{-243}%
  \TestPow{2}{30}{1073741824}%
  \TestPow{-3}{19}{-1162261467}%
  \TestPow{5}{13}{1220703125}%
  \TestPow{-7}{11}{-1977326743}%
\end{qstest}

\begin{qstest}{div}{div}
  \TestDiv{1}{1}{1}%
  \TestDiv{2}{1}{2}%
  \TestDiv{-2}{1}{-2}%
  \TestDiv{2}{-1}{-2}%
  \TestDiv{-2}{-1}{2}%
  \TestDiv{15}{2}{7}%
  \TestDiv{-16}{2}{-8}%
  \TestDiv{1}{2}{0}%
  \TestDiv{1}{3}{0}%
  \TestDiv{2}{3}{0}%
  \TestDiv{-2}{3}{0}%
  \TestDiv{2}{-3}{0}%
  \TestDiv{-2}{-3}{0}%
  \TestDiv{13}{3}{4}%
  \TestDiv{-13}{-3}{4}%
  \TestDiv{-13}{3}{-4}%
  \TestDiv{-6}{5}{-1}%
  \TestDiv{-5}{5}{-1}%
  \TestDiv{-4}{5}{0}%
  \TestDiv{-3}{5}{0}%
  \TestDiv{-2}{5}{0}%
  \TestDiv{-1}{5}{0}%
  \TestDiv{0}{5}{0}%
  \TestDiv{1}{5}{0}%
  \TestDiv{2}{5}{0}%
  \TestDiv{3}{5}{0}%
  \TestDiv{4}{5}{0}%
  \TestDiv{5}{5}{1}%
  \TestDiv{6}{5}{1}%
  \TestDiv{-5}{4}{-1}%
  \TestDiv{-4}{4}{-1}%
  \TestDiv{-3}{4}{0}%
  \TestDiv{-2}{4}{0}%
  \TestDiv{-1}{4}{0}%
  \TestDiv{0}{4}{0}%
  \TestDiv{1}{4}{0}%
  \TestDiv{2}{4}{0}%
  \TestDiv{3}{4}{0}%
  \TestDiv{4}{4}{1}%
  \TestDiv{5}{4}{1}%
  \TestDiv{12345}{678}{18}%
  \TestDiv{32372}{5952}{5}%
  \TestDiv{284271294}{18162}{15651}%
  \TestDiv{217652429}{12561}{17327}%
  \TestDiv{462028434}{5439}{84947}%
  \TestDiv{2147483647}{1000}{2147483}%
  \TestDiv{2147483647}{-1000}{-2147483}%
  \TestDiv{-2147483647}{1000}{-2147483}%
  \TestDiv{-2147483647}{-1000}{2147483}%
  \TestDiv{0}{3}{0}%
  \TestDiv{1}{3}{0}%
  \TestDiv{2}{3}{0}%
  \TestDiv{3}{3}{1}%
  \TestDiv{4}{3}{1}%
  \TestDiv{5}{3}{1}%
  \TestDiv{6}{3}{2}%
  \TestDiv{7}{3}{2}%
  \TestDiv{8}{3}{2}%
  \TestDiv{9}{3}{3}%
  \TestDiv{10}{3}{3}%
  \TestDiv{11}{3}{3}%
  \TestDiv{12}{3}{4}%
  \TestDiv{13}{3}{4}%
  \TestDiv{14}{3}{4}%
  \TestDiv{15}{3}{5}%
  \TestDiv{16}{3}{5}%
  \TestDiv{17}{3}{5}%
  \TestDiv{18}{3}{6}%
  \TestDiv{19}{3}{6}%
  \TestDiv{20}{3}{6}%
  \TestDiv{21}{3}{7}%
  \TestDiv{22}{3}{7}%
  \TestDiv{23}{3}{7}%
  \TestDiv{24}{3}{8}%
  \TestDiv{25}{3}{8}%
  \TestDiv{26}{3}{8}%
  \TestDiv{27}{3}{9}%
  \TestDiv{28}{3}{9}%
  \TestDiv{29}{3}{9}%
  \TestDiv{30}{3}{10}%
  \TestDiv{31}{3}{10}%
  \TestDivBig{17363436332507}{24702}{702916214}%
\end{qstest}

\begin{qstest}{mod}{mod}
  \TestMod{-6}{5}{4}%
  \TestMod{-5}{5}{0}%
  \TestMod{-4}{5}{1}%
  \TestMod{-3}{5}{2}%
  \TestMod{-2}{5}{3}%
  \TestMod{-1}{5}{4}%
  \TestMod{0}{5}{0}%
  \TestMod{1}{5}{1}%
  \TestMod{2}{5}{2}%
  \TestMod{3}{5}{3}%
  \TestMod{4}{5}{4}%
  \TestMod{5}{5}{0}%
  \TestMod{6}{5}{1}%
  \TestMod{-5}{4}{3}%
  \TestMod{-4}{4}{0}%
  \TestMod{-3}{4}{1}%
  \TestMod{-2}{4}{2}%
  \TestMod{-1}{4}{3}%
  \TestMod{0}{4}{0}%
  \TestMod{1}{4}{1}%
  \TestMod{2}{4}{2}%
  \TestMod{3}{4}{3}%
  \TestMod{4}{4}{0}%
  \TestMod{5}{4}{1}%
  \TestMod{-6}{-5}{-1}%
  \TestMod{-5}{-5}{0}%
  \TestMod{-4}{-5}{-4}%
  \TestMod{-3}{-5}{-3}%
  \TestMod{-2}{-5}{-2}%
  \TestMod{-1}{-5}{-1}%
  \TestMod{0}{-5}{0}%
  \TestMod{1}{-5}{-4}%
  \TestMod{2}{-5}{-3}%
  \TestMod{3}{-5}{-2}%
  \TestMod{4}{-5}{-1}%
  \TestMod{5}{-5}{0}%
  \TestMod{6}{-5}{-4}%
  \TestMod{-5}{-4}{-1}%
  \TestMod{-4}{-4}{0}%
  \TestMod{-3}{-4}{-3}%
  \TestMod{-2}{-4}{-2}%
  \TestMod{-1}{-4}{-1}%
  \TestMod{0}{-4}{0}%
  \TestMod{1}{-4}{-3}%
  \TestMod{2}{-4}{-2}%
  \TestMod{3}{-4}{-1}%
  \TestMod{4}{-4}{0}%
  \TestMod{5}{-4}{-3}%
  \TestMod{2147483647}{1000}{647}%
  \TestMod{2147483647}{-1000}{-353}%
  \TestMod{-2147483647}{1000}{353}%
  \TestMod{-2147483647}{-1000}{-647}%
  \TestMod{ 0 }{ 4 }{0}%
  \TestMod{ 1 }{ 4 }{1}%
  \TestMod{ -1 }{ 4 }{3}%
  \TestMod{ 0 }{ -4 }{0}%
  \TestMod{ 1 }{ -4 }{-3}%
  \TestMod{ -1 }{ -4 }{-1}%
  \TestMod{18362}{25}{12}%
\end{qstest}

\newcommand*{\TestError}[2]{%
  \begingroup
    \expandafter\def\csname BigIntCalcError:#1\endcsname{}%
    \Expect*{#2}{0}%
    \expandafter\def\csname BigIntCalcError:#1\endcsname{ERROR}%
    \Expect*{#2}{0ERROR}%
  \endgroup
}
\begin{qstest}{error}{error}
  \TestError{FacNegative}{\bigintcalcFac{-1}}%
  \TestError{FacNegative}{\bigintcalcFac{-2147483647}}%
  \TestError{DivisionByZero}{\bigintcalcPow{0}{-1}}%
  \TestError{DivisionByZero}{\bigintcalcDiv{1}{0}}%
  \TestError{DivisionByZero}{\bigintcalcMod{1}{0}}%
\end{qstest}

\begin{document}
\end{document}
%</test2>
%    \end{macrocode}
%
% \section{Installation}
%
% \subsection{Download}
%
% \paragraph{Package.} This package is available on
% CTAN\footnote{\url{http://ctan.org/pkg/bigintcalc}}:
% \begin{description}
% \item[\CTAN{macros/latex/contrib/oberdiek/bigintcalc.dtx}] The source file.
% \item[\CTAN{macros/latex/contrib/oberdiek/bigintcalc.pdf}] Documentation.
% \end{description}
%
%
% \paragraph{Bundle.} All the packages of the bundle `oberdiek'
% are also available in a TDS compliant ZIP archive. There
% the packages are already unpacked and the documentation files
% are generated. The files and directories obey the TDS standard.
% \begin{description}
% \item[\CTAN{install/macros/latex/contrib/oberdiek.tds.zip}]
% \end{description}
% \emph{TDS} refers to the standard ``A Directory Structure
% for \TeX\ Files'' (\CTAN{tds/tds.pdf}). Directories
% with \xfile{texmf} in their name are usually organized this way.
%
% \subsection{Bundle installation}
%
% \paragraph{Unpacking.} Unpack the \xfile{oberdiek.tds.zip} in the
% TDS tree (also known as \xfile{texmf} tree) of your choice.
% Example (linux):
% \begin{quote}
%   |unzip oberdiek.tds.zip -d ~/texmf|
% \end{quote}
%
% \paragraph{Script installation.}
% Check the directory \xfile{TDS:scripts/oberdiek/} for
% scripts that need further installation steps.
% Package \xpackage{attachfile2} comes with the Perl script
% \xfile{pdfatfi.pl} that should be installed in such a way
% that it can be called as \texttt{pdfatfi}.
% Example (linux):
% \begin{quote}
%   |chmod +x scripts/oberdiek/pdfatfi.pl|\\
%   |cp scripts/oberdiek/pdfatfi.pl /usr/local/bin/|
% \end{quote}
%
% \subsection{Package installation}
%
% \paragraph{Unpacking.} The \xfile{.dtx} file is a self-extracting
% \docstrip\ archive. The files are extracted by running the
% \xfile{.dtx} through \plainTeX:
% \begin{quote}
%   \verb|tex bigintcalc.dtx|
% \end{quote}
%
% \paragraph{TDS.} Now the different files must be moved into
% the different directories in your installation TDS tree
% (also known as \xfile{texmf} tree):
% \begin{quote}
% \def\t{^^A
% \begin{tabular}{@{}>{\ttfamily}l@{ $\rightarrow$ }>{\ttfamily}l@{}}
%   bigintcalc.sty & tex/generic/oberdiek/bigintcalc.sty\\
%   bigintcalc.pdf & doc/latex/oberdiek/bigintcalc.pdf\\
%   test/bigintcalc-test1.tex & doc/latex/oberdiek/test/bigintcalc-test1.tex\\
%   test/bigintcalc-test2.tex & doc/latex/oberdiek/test/bigintcalc-test2.tex\\
%   test/bigintcalc-test3.tex & doc/latex/oberdiek/test/bigintcalc-test3.tex\\
%   bigintcalc.dtx & source/latex/oberdiek/bigintcalc.dtx\\
% \end{tabular}^^A
% }^^A
% \sbox0{\t}^^A
% \ifdim\wd0>\linewidth
%   \begingroup
%     \advance\linewidth by\leftmargin
%     \advance\linewidth by\rightmargin
%   \edef\x{\endgroup
%     \def\noexpand\lw{\the\linewidth}^^A
%   }\x
%   \def\lwbox{^^A
%     \leavevmode
%     \hbox to \linewidth{^^A
%       \kern-\leftmargin\relax
%       \hss
%       \usebox0
%       \hss
%       \kern-\rightmargin\relax
%     }^^A
%   }^^A
%   \ifdim\wd0>\lw
%     \sbox0{\small\t}^^A
%     \ifdim\wd0>\linewidth
%       \ifdim\wd0>\lw
%         \sbox0{\footnotesize\t}^^A
%         \ifdim\wd0>\linewidth
%           \ifdim\wd0>\lw
%             \sbox0{\scriptsize\t}^^A
%             \ifdim\wd0>\linewidth
%               \ifdim\wd0>\lw
%                 \sbox0{\tiny\t}^^A
%                 \ifdim\wd0>\linewidth
%                   \lwbox
%                 \else
%                   \usebox0
%                 \fi
%               \else
%                 \lwbox
%               \fi
%             \else
%               \usebox0
%             \fi
%           \else
%             \lwbox
%           \fi
%         \else
%           \usebox0
%         \fi
%       \else
%         \lwbox
%       \fi
%     \else
%       \usebox0
%     \fi
%   \else
%     \lwbox
%   \fi
% \else
%   \usebox0
% \fi
% \end{quote}
% If you have a \xfile{docstrip.cfg} that configures and enables \docstrip's
% TDS installing feature, then some files can already be in the right
% place, see the documentation of \docstrip.
%
% \subsection{Refresh file name databases}
%
% If your \TeX~distribution
% (\teTeX, \mikTeX, \dots) relies on file name databases, you must refresh
% these. For example, \teTeX\ users run \verb|texhash| or
% \verb|mktexlsr|.
%
% \subsection{Some details for the interested}
%
% \paragraph{Attached source.}
%
% The PDF documentation on CTAN also includes the
% \xfile{.dtx} source file. It can be extracted by
% AcrobatReader 6 or higher. Another option is \textsf{pdftk},
% e.g. unpack the file into the current directory:
% \begin{quote}
%   \verb|pdftk bigintcalc.pdf unpack_files output .|
% \end{quote}
%
% \paragraph{Unpacking with \LaTeX.}
% The \xfile{.dtx} chooses its action depending on the format:
% \begin{description}
% \item[\plainTeX:] Run \docstrip\ and extract the files.
% \item[\LaTeX:] Generate the documentation.
% \end{description}
% If you insist on using \LaTeX\ for \docstrip\ (really,
% \docstrip\ does not need \LaTeX), then inform the autodetect routine
% about your intention:
% \begin{quote}
%   \verb|latex \let\install=y\input{bigintcalc.dtx}|
% \end{quote}
% Do not forget to quote the argument according to the demands
% of your shell.
%
% \paragraph{Generating the documentation.}
% You can use both the \xfile{.dtx} or the \xfile{.drv} to generate
% the documentation. The process can be configured by the
% configuration file \xfile{ltxdoc.cfg}. For instance, put this
% line into this file, if you want to have A4 as paper format:
% \begin{quote}
%   \verb|\PassOptionsToClass{a4paper}{article}|
% \end{quote}
% An example follows how to generate the
% documentation with pdf\LaTeX:
% \begin{quote}
%\begin{verbatim}
%pdflatex bigintcalc.dtx
%makeindex -s gind.ist bigintcalc.idx
%pdflatex bigintcalc.dtx
%makeindex -s gind.ist bigintcalc.idx
%pdflatex bigintcalc.dtx
%\end{verbatim}
% \end{quote}
%
% \section{Catalogue}
%
% The following XML file can be used as source for the
% \href{http://mirror.ctan.org/help/Catalogue/catalogue.html}{\TeX\ Catalogue}.
% The elements \texttt{caption} and \texttt{description} are imported
% from the original XML file from the Catalogue.
% The name of the XML file in the Catalogue is \xfile{bigintcalc.xml}.
%    \begin{macrocode}
%<*catalogue>
<?xml version='1.0' encoding='us-ascii'?>
<!DOCTYPE entry SYSTEM 'catalogue.dtd'>
<entry datestamp='$Date$' modifier='$Author$' id='bigintcalc'>
  <name>bigintcalc</name>
  <caption>Integer calculations on very large numbers.</caption>
  <authorref id='auth:oberdiek'/>
  <copyright owner='Heiko Oberdiek' year='2007,2011,2012'/>
  <license type='lppl1.3'/>
  <version number='1.4'/>
  <description>
    This package provides expandable arithmetic operations
    with big integers that can exceed TeX's number limits.
    <p/>
    The package is part of the <xref refid='oberdiek'>oberdiek</xref> bundle.
  </description>
  <documentation details='Package documentation'
      href='ctan:/macros/latex/contrib/oberdiek/bigintcalc.pdf'/>
  <ctan file='true' path='/macros/latex/contrib/oberdiek/bigintcalc.dtx'/>
  <miktex location='oberdiek'/>
  <texlive location='oberdiek'/>
  <install path='/macros/latex/contrib/oberdiek/oberdiek.tds.zip'/>
</entry>
%</catalogue>
%    \end{macrocode}
%
% \begin{History}
%   \begin{Version}{2007/09/27 v1.0}
%   \item
%     First version.
%   \end{Version}
%   \begin{Version}{2007/11/11 v1.1}
%   \item
%     Use of package \xpackage{pdftexcmds} for \LuaTeX\ support.
%   \end{Version}
%   \begin{Version}{2011/01/30 v1.2}
%   \item
%     Already loaded package files are not input in \hologo{plainTeX}.
%   \end{Version}
%   \begin{Version}{2012/04/08 v1.3}
%   \item
%     Fix: \xpackage{pdftexcmds} wasn't loaded in case of \hologo{LaTeX}.
%   \end{Version}
%   \begin{Version}{2016/05/16 v1.4}
%   \item
%     Documentation updates.
%   \end{Version}
% \end{History}
%
% \PrintIndex
%
% \Finale
\endinput

%        (quote the arguments according to the demands of your shell)
%
% Documentation:
%    (a) If bigintcalc.drv is present:
%           latex bigintcalc.drv
%    (b) Without bigintcalc.drv:
%           latex bigintcalc.dtx; ...
%    The class ltxdoc loads the configuration file ltxdoc.cfg
%    if available. Here you can specify further options, e.g.
%    use A4 as paper format:
%       \PassOptionsToClass{a4paper}{article}
%
%    Programm calls to get the documentation (example):
%       pdflatex bigintcalc.dtx
%       makeindex -s gind.ist bigintcalc.idx
%       pdflatex bigintcalc.dtx
%       makeindex -s gind.ist bigintcalc.idx
%       pdflatex bigintcalc.dtx
%
% Installation:
%    TDS:tex/generic/oberdiek/bigintcalc.sty
%    TDS:doc/latex/oberdiek/bigintcalc.pdf
%    TDS:doc/latex/oberdiek/test/bigintcalc-test1.tex
%    TDS:doc/latex/oberdiek/test/bigintcalc-test2.tex
%    TDS:doc/latex/oberdiek/test/bigintcalc-test3.tex
%    TDS:source/latex/oberdiek/bigintcalc.dtx
%
%<*ignore>
\begingroup
  \catcode123=1 %
  \catcode125=2 %
  \def\x{LaTeX2e}%
\expandafter\endgroup
\ifcase 0\ifx\install y1\fi\expandafter
         \ifx\csname processbatchFile\endcsname\relax\else1\fi
         \ifx\fmtname\x\else 1\fi\relax
\else\csname fi\endcsname
%</ignore>
%<*install>
\input docstrip.tex
\Msg{************************************************************************}
\Msg{* Installation}
\Msg{* Package: bigintcalc 2016/05/16 v1.4 Expandable calculations on big integers (HO)}
\Msg{************************************************************************}

\keepsilent
\askforoverwritefalse

\let\MetaPrefix\relax
\preamble

This is a generated file.

Project: bigintcalc
Version: 2016/05/16 v1.4

Copyright (C) 2007, 2011, 2012 by
   Heiko Oberdiek <heiko.oberdiek at googlemail.com>

This work may be distributed and/or modified under the
conditions of the LaTeX Project Public License, either
version 1.3c of this license or (at your option) any later
version. This version of this license is in
   http://www.latex-project.org/lppl/lppl-1-3c.txt
and the latest version of this license is in
   http://www.latex-project.org/lppl.txt
and version 1.3 or later is part of all distributions of
LaTeX version 2005/12/01 or later.

This work has the LPPL maintenance status "maintained".

This Current Maintainer of this work is Heiko Oberdiek.

The Base Interpreter refers to any `TeX-Format',
because some files are installed in TDS:tex/generic//.

This work consists of the main source file bigintcalc.dtx
and the derived files
   bigintcalc.sty, bigintcalc.pdf, bigintcalc.ins, bigintcalc.drv,
   bigintcalc-test1.tex, bigintcalc-test2.tex,
   bigintcalc-test3.tex.

\endpreamble
\let\MetaPrefix\DoubleperCent

\generate{%
  \file{bigintcalc.ins}{\from{bigintcalc.dtx}{install}}%
  \file{bigintcalc.drv}{\from{bigintcalc.dtx}{driver}}%
  \usedir{tex/generic/oberdiek}%
  \file{bigintcalc.sty}{\from{bigintcalc.dtx}{package}}%
%  \usedir{doc/latex/oberdiek/test}%
%  \file{bigintcalc-test1.tex}{\from{bigintcalc.dtx}{test1}}%
%  \file{bigintcalc-test2.tex}{\from{bigintcalc.dtx}{test2,etex}}%
%  \file{bigintcalc-test3.tex}{\from{bigintcalc.dtx}{test2,noetex}}%
  \nopreamble
  \nopostamble
  \usedir{source/latex/oberdiek/catalogue}%
  \file{bigintcalc.xml}{\from{bigintcalc.dtx}{catalogue}}%
}

\catcode32=13\relax% active space
\let =\space%
\Msg{************************************************************************}
\Msg{*}
\Msg{* To finish the installation you have to move the following}
\Msg{* file into a directory searched by TeX:}
\Msg{*}
\Msg{*     bigintcalc.sty}
\Msg{*}
\Msg{* To produce the documentation run the file `bigintcalc.drv'}
\Msg{* through LaTeX.}
\Msg{*}
\Msg{* Happy TeXing!}
\Msg{*}
\Msg{************************************************************************}

\endbatchfile
%</install>
%<*ignore>
\fi
%</ignore>
%<*driver>
\NeedsTeXFormat{LaTeX2e}
\ProvidesFile{bigintcalc.drv}%
  [2016/05/16 v1.4 Expandable calculations on big integers (HO)]%
\documentclass{ltxdoc}
\usepackage{wasysym}
\let\iint\relax
\let\iiint\relax
\usepackage[fleqn]{amsmath}

\DeclareMathOperator{\opInv}{Inv}
\DeclareMathOperator{\opAbs}{Abs}
\DeclareMathOperator{\opSgn}{Sgn}
\DeclareMathOperator{\opMin}{Min}
\DeclareMathOperator{\opMax}{Max}
\DeclareMathOperator{\opCmp}{Cmp}
\DeclareMathOperator{\opOdd}{Odd}
\DeclareMathOperator{\opInc}{Inc}
\DeclareMathOperator{\opDec}{Dec}
\DeclareMathOperator{\opAdd}{Add}
\DeclareMathOperator{\opSub}{Sub}
\DeclareMathOperator{\opShl}{Shl}
\DeclareMathOperator{\opShr}{Shr}
\DeclareMathOperator{\opMul}{Mul}
\DeclareMathOperator{\opSqr}{Sqr}
\DeclareMathOperator{\opFac}{Fac}
\DeclareMathOperator{\opPow}{Pow}
\DeclareMathOperator{\opDiv}{Div}
\DeclareMathOperator{\opMod}{Mod}
\DeclareMathOperator{\opInt}{Int}
\DeclareMathOperator{\opODD}{ifodd}

\newcommand*{\Def}{%
  \ensuremath{%
    \mathrel{\mathop{:}}=%
  }%
}
\newcommand*{\op}[1]{%
  \textsf{#1}%
}
\usepackage{holtxdoc}[2011/11/22]
\begin{document}
  \DocInput{bigintcalc.dtx}%
\end{document}
%</driver>
% \fi
%
%
% \CharacterTable
%  {Upper-case    \A\B\C\D\E\F\G\H\I\J\K\L\M\N\O\P\Q\R\S\T\U\V\W\X\Y\Z
%   Lower-case    \a\b\c\d\e\f\g\h\i\j\k\l\m\n\o\p\q\r\s\t\u\v\w\x\y\z
%   Digits        \0\1\2\3\4\5\6\7\8\9
%   Exclamation   \!     Double quote  \"     Hash (number) \#
%   Dollar        \$     Percent       \%     Ampersand     \&
%   Acute accent  \'     Left paren    \(     Right paren   \)
%   Asterisk      \*     Plus          \+     Comma         \,
%   Minus         \-     Point         \.     Solidus       \/
%   Colon         \:     Semicolon     \;     Less than     \<
%   Equals        \=     Greater than  \>     Question mark \?
%   Commercial at \@     Left bracket  \[     Backslash     \\
%   Right bracket \]     Circumflex    \^     Underscore    \_
%   Grave accent  \`     Left brace    \{     Vertical bar  \|
%   Right brace   \}     Tilde         \~}
%
% \GetFileInfo{bigintcalc.drv}
%
% \title{The \xpackage{bigintcalc} package}
% \date{2016/05/16 v1.4}
% \author{Heiko Oberdiek\thanks
% {Please report any issues at https://github.com/ho-tex/oberdiek/issues}\\
% \xemail{heiko.oberdiek at googlemail.com}}
%
% \maketitle
%
% \begin{abstract}
% This package provides expandable arithmetic operations
% with big integers that can exceed \TeX's number limits.
% \end{abstract}
%
% \tableofcontents
%
% \section{Documentation}
%
% \subsection{Introduction}
%
% Package \xpackage{bigintcalc} defines arithmetic operations
% that deal with big integers. Big integers can be given
% either as explicit integer number or as macro code
% that expands to an explicit number. \emph{Big} means that
% there is no limit on the size of the number. Big integers
% may exceed \TeX's range limitation of -2147483647 and 2147483647.
% Only memory issues will limit the usable range.
%
% In opposite to package \xpackage{intcalc}
% unexpandable command tokens are not supported, even if
% they are valid \TeX\ numbers like count registers or
% commands created by \cs{chardef}. Nevertheless they may
% be used, if they are prefixed by \cs{number}.
%
% Also \eTeX's \cs{numexpr} expressions are not supported
% directly in the manner of package \xpackage{intcalc}.
% However they can be given if \cs{the}\cs{numexpr}
% or \cs{number}\cs{numexpr} are used.
%
% The operations have the form of macros that take one or
% two integers as parameter and return the integer result.
% The macro name is a three letter operation name prefixed
% by the package name, e.g. \cs{bigintcalcAdd}|{10}{43}| returns
% |53|.
%
% The macros are fully expandable, exactly two expansion
% steps generate the result. Therefore the operations
% may be used nearly everywhere in \TeX, even inside
% \cs{csname}, file names,  or other
% expandable contexts.
%
% \subsection{Conditions}
%
% \subsubsection{Preconditions}
%
% \begin{itemize}
% \item
%   Arguments can be anything that expands to a number that consists
%   of optional signs and digits.
% \item
%   The arguments and return values must be sound.
%   Zero as divisor or factorials of negative numbers will cause errors.
% \end{itemize}
%
% \subsubsection{Postconditions}
%
% Additional properties of the macros apart from calculating
% a correct result (of course \smiley):
% \begin{itemize}
% \item
%   The macros are fully expandable. Thus they can
%   be used inside \cs{edef}, \cs{csname},
%   for example.
% \item
%   Furthermore exactly two expansion steps calculate the result.
% \item
%   The number consists of one optional minus sign and one or more
%   digits. The first digit is larger than zero for
%   numbers that consists of more than one digit.
%
%   In short, the number format is exactly the same as
%   \cs{number} generates, but without its range limitation.
%   And the tokens (minus sign, digits)
%   have catcode 12 (other).
% \item
%   Call by value is simulated. First the arguments are
%   converted to numbers. Then these numbers are used
%   in the calculations.
%
%   Remember that arguments
%   may contain expensive macros or \eTeX\ expressions.
%   This strategy avoids multiple evaluations of such
%   arguments.
% \end{itemize}
%
% \subsection{Error handling}
%
% Some errors are detected by the macros, example: division by zero.
% In this cases an undefined control sequence is called and causes
% a TeX error message, example: \cs{BigIntCalcError:DivisionByZero}.
% The name of the control sequence contains
% the reason for the error. The \TeX\ error may be ignored.
% Then the operation returns zero as result.
% Because the macros are supposed to work in expandible contexts.
% An traditional error message, however, is not expandable and
% would break these contexts.
%
% \subsection{Operations}
%
% Some definition equations below use the function $\opInt$
% that converts a real number to an integer. The number
% is truncated that means rounding to zero:
% \begin{gather*}
%   \opInt(x) \Def
%   \begin{cases}
%     \lfloor x\rfloor & \text{if $x\geq0$}\\
%     \lceil x\rceil & \text{otherwise}
%   \end{cases}
% \end{gather*}
%
% \subsubsection{\op{Num}}
%
% \begin{declcs}{bigintcalcNum} \M{x}
% \end{declcs}
% Macro \cs{bigintcalcNum} converts its argument to a normalized integer
% number without unnecessary leading zeros or signs.
% The result matches the regular expression:
%\begin{quote}
%\begin{verbatim}
%0|-?[1-9][0-9]*
%\end{verbatim}
%\end{quote}
%
% \subsubsection{\op{Inv}, \op{Abs}, \op{Sgn}}
%
% \begin{declcs}{bigintcalcInv} \M{x}
% \end{declcs}
% Macro \cs{bigintcalcInv} switches the sign.
% \begin{gather*}
%   \opInv(x) \Def -x
% \end{gather*}
%
% \begin{declcs}{bigintcalcAbs} \M{x}
% \end{declcs}
% Macro \cs{bigintcalcAbs} returns the absolute value
% of integer \meta{x}.
% \begin{gather*}
%   \opAbs(x) \Def \vert x\vert
% \end{gather*}
%
% \begin{declcs}{bigintcalcSgn} \M{x}
% \end{declcs}
% Macro \cs{bigintcalcSgn} encodes the sign of \meta{x} as number.
% \begin{gather*}
%   \opSgn(x) \Def
%   \begin{cases}
%     -1& \text{if $x<0$}\\
%      0& \text{if $x=0$}\\
%      1& \text{if $x>0$}
%   \end{cases}
% \end{gather*}
% These return values can easily be distinguished by \cs{ifcase}:
%\begin{quote}
%\begin{verbatim}
%\ifcase\bigintcalcSgn{<x>}
%  $x=0$
%\or
%  $x>0$
%\else
%  $x<0$
%\fi
%\end{verbatim}
%\end{quote}
%
% \subsubsection{\op{Min}, \op{Max}, \op{Cmp}}
%
% \begin{declcs}{bigintcalcMin} \M{x} \M{y}
% \end{declcs}
% Macro \cs{bigintcalcMin} returns the smaller of the two integers.
% \begin{gather*}
%   \opMin(x,y) \Def
%   \begin{cases}
%     x & \text{if $x<y$}\\
%     y & \text{otherwise}
%   \end{cases}
% \end{gather*}
%
% \begin{declcs}{bigintcalcMax} \M{x} \M{y}
% \end{declcs}
% Macro \cs{bigintcalcMax} returns the larger of the two integers.
% \begin{gather*}
%   \opMax(x,y) \Def
%   \begin{cases}
%     x & \text{if $x>y$}\\
%     y & \text{otherwise}
%   \end{cases}
% \end{gather*}
%
% \begin{declcs}{bigintcalcCmp} \M{x} \M{y}
% \end{declcs}
% Macro \cs{bigintcalcCmp} encodes the comparison result as number:
% \begin{gather*}
%   \opCmp(x,y) \Def
%   \begin{cases}
%     -1 & \text{if $x<y$}\\
%      0 & \text{if $x=y$}\\
%      1 & \text{if $x>y$}
%   \end{cases}
% \end{gather*}
% These values can be distinguished by \cs{ifcase}:
%\begin{quote}
%\begin{verbatim}
%\ifcase\bigintcalcCmp{<x>}{<y>}
%  $x=y$
%\or
%  $x>y$
%\else
%  $x<y$
%\fi
%\end{verbatim}
%\end{quote}
%
% \subsubsection{\op{Odd}}
%
% \begin{declcs}{bigintcalcOdd} \M{x}
% \end{declcs}
% \begin{gather*}
%   \opOdd(x) \Def
%   \begin{cases}
%     1 & \text{if $x$ is odd}\\
%     0 & \text{if $x$ is even}
%   \end{cases}
% \end{gather*}
%
% \subsubsection{\op{Inc}, \op{Dec}, \op{Add}, \op{Sub}}
%
% \begin{declcs}{bigintcalcInc} \M{x}
% \end{declcs}
% Macro \cs{bigintcalcInc} increments \meta{x} by one.
% \begin{gather*}
%   \opInc(x) \Def x + 1
% \end{gather*}
%
% \begin{declcs}{bigintcalcDec} \M{x}
% \end{declcs}
% Macro \cs{bigintcalcDec} decrements \meta{x} by one.
% \begin{gather*}
%   \opDec(x) \Def x - 1
% \end{gather*}
%
% \begin{declcs}{bigintcalcAdd} \M{x} \M{y}
% \end{declcs}
% Macro \cs{bigintcalcAdd} adds the two numbers.
% \begin{gather*}
%   \opAdd(x, y) \Def x + y
% \end{gather*}
%
% \begin{declcs}{bigintcalcSub} \M{x} \M{y}
% \end{declcs}
% Macro \cs{bigintcalcSub} calculates the difference.
% \begin{gather*}
%   \opSub(x, y) \Def x - y
% \end{gather*}
%
% \subsubsection{\op{Shl}, \op{Shr}}
%
% \begin{declcs}{bigintcalcShl} \M{x}
% \end{declcs}
% Macro \cs{bigintcalcShl} implements shifting to the left that
% means the number is multiplied by two.
% The sign is preserved.
% \begin{gather*}
%   \opShl(x) \Def x*2
% \end{gather*}
%
% \begin{declcs}{bigintcalcShr} \M{x}
% \end{declcs}
% Macro \cs{bigintcalcShr} implements shifting to the right.
% That is equivalent to an integer division by two.
% The sign is preserved.
% \begin{gather*}
%   \opShr(x) \Def \opInt(x/2)
% \end{gather*}
%
% \subsubsection{\op{Mul}, \op{Sqr}, \op{Fac}, \op{Pow}}
%
% \begin{declcs}{bigintcalcMul} \M{x} \M{y}
% \end{declcs}
% Macro \cs{bigintcalcMul} calculates the product of
% \meta{x} and \meta{y}.
% \begin{gather*}
%   \opMul(x,y) \Def x*y
% \end{gather*}
%
% \begin{declcs}{bigintcalcSqr} \M{x}
% \end{declcs}
% Macro \cs{bigintcalcSqr} returns the square product.
% \begin{gather*}
%   \opSqr(x) \Def x^2
% \end{gather*}
%
% \begin{declcs}{bigintcalcFac} \M{x}
% \end{declcs}
% Macro \cs{bigintcalcFac} returns the factorial of \meta{x}.
% Negative numbers are not permitted.
% \begin{gather*}
%   \opFac(x) \Def x!\qquad\text{for $x\geq0$}
% \end{gather*}
% ($0! = 1$)
%
% \begin{declcs}{bigintcalcPow} M{x} M{y}
% \end{declcs}
% Macro \cs{bigintcalcPow} calculates the value of \meta{x} to the
% power of \meta{y}. The error ``division by zero'' is thrown
% if \meta{x} is zero and \meta{y} is negative.
% permitted:
% \begin{gather*}
%   \opPow(x,y) \Def
%   \opInt(x^y)\qquad\text{for $x\neq0$ or $y\geq0$}
% \end{gather*}
% ($0^0 = 1$)
%
% \subsubsection{\op{Div}, \op{Mul}}
%
% \begin{declcs}{bigintcalcDiv} \M{x} \M{y}
% \end{declcs}
% Macro \cs{bigintcalcDiv} performs an integer division.
% Argument \meta{y} must not be zero.
% \begin{gather*}
%   \opDiv(x,y) \Def \opInt(x/y)\qquad\text{for $y\neq0$}
% \end{gather*}
%
% \begin{declcs}{bigintcalcMod} \M{x} \M{y}
% \end{declcs}
% Macro \cs{bigintcalcMod} gets the remainder of the integer
% division. The sign follows the divisor \meta{y}.
% Argument \meta{y} must not be zero.
% \begin{gather*}
%   \opMod(x,y) \Def x\mathrel{\%}y\qquad\text{for $y\neq0$}
% \end{gather*}
% The result ranges:
% \begin{gather*}
%   -\vert y\vert < \opMod(x,y) \leq 0\qquad\text{for $y<0$}\\
%   0 \leq \opMod(x,y) < y\qquad\text{for $y\geq0$}
% \end{gather*}
%
% \subsection{Interface for programmers}
%
% If the programmer can ensure some more properties about
% the arguments of the operations, then the following
% macros are a little more efficient.
%
% In general numbers must obey the following constraints:
% \begin{itemize}
% \item Plain number: digit tokens only, no command tokens.
% \item Non-negative. Signs are forbidden.
% \item Delimited by exclamation mark. Curly braces
%       around the number are not allowed and will
%       break the code.
% \end{itemize}
%
% \begin{declcs}{BigIntCalcOdd} \meta{number} |!|
% \end{declcs}
% |1|/|0| is returned if \meta{number} is odd/even.
%
% \begin{declcs}{BigIntCalcInc} \meta{number} |!|
% \end{declcs}
% Incrementation.
%
% \begin{declcs}{BigIntCalcDec} \meta{number} |!|
% \end{declcs}
% Decrementation, positive number without zero.
%
% \begin{declcs}{BigIntCalcAdd} \meta{number A} |!| \meta{number B} |!|
% \end{declcs}
% Addition, $A\geq B$.
%
% \begin{declcs}{BigIntCalcSub} \meta{number A} |!| \meta{number B} |!|
% \end{declcs}
% Subtraction, $A\geq B$.
%
% \begin{declcs}{BigIntCalcShl} \meta{number} |!|
% \end{declcs}
% Left shift (multiplication with two).
%
% \begin{declcs}{BigIntCalcShr} \meta{number} |!|
% \end{declcs}
% Right shift (integer division by two).
%
% \begin{declcs}{BigIntCalcMul} \meta{number A} |!| \meta{number B} |!|
% \end{declcs}
% Multiplication, $A\geq B$.
%
% \begin{declcs}{BigIntCalcDiv} \meta{number A} |!| \meta{number B} |!|
% \end{declcs}
% Division operation.
%
% \begin{declcs}{BigIntCalcMod} \meta{number A} |!| \meta{number B} |!|
% \end{declcs}
% Modulo operation.
%
%
% \StopEventually{
% }
%
% \section{Implementation}
%
%    \begin{macrocode}
%<*package>
%    \end{macrocode}
%
% \subsection{Reload check and package identification}
%    Reload check, especially if the package is not used with \LaTeX.
%    \begin{macrocode}
\begingroup\catcode61\catcode48\catcode32=10\relax%
  \catcode13=5 % ^^M
  \endlinechar=13 %
  \catcode35=6 % #
  \catcode39=12 % '
  \catcode44=12 % ,
  \catcode45=12 % -
  \catcode46=12 % .
  \catcode58=12 % :
  \catcode64=11 % @
  \catcode123=1 % {
  \catcode125=2 % }
  \expandafter\let\expandafter\x\csname ver@bigintcalc.sty\endcsname
  \ifx\x\relax % plain-TeX, first loading
  \else
    \def\empty{}%
    \ifx\x\empty % LaTeX, first loading,
      % variable is initialized, but \ProvidesPackage not yet seen
    \else
      \expandafter\ifx\csname PackageInfo\endcsname\relax
        \def\x#1#2{%
          \immediate\write-1{Package #1 Info: #2.}%
        }%
      \else
        \def\x#1#2{\PackageInfo{#1}{#2, stopped}}%
      \fi
      \x{bigintcalc}{The package is already loaded}%
      \aftergroup\endinput
    \fi
  \fi
\endgroup%
%    \end{macrocode}
%    Package identification:
%    \begin{macrocode}
\begingroup\catcode61\catcode48\catcode32=10\relax%
  \catcode13=5 % ^^M
  \endlinechar=13 %
  \catcode35=6 % #
  \catcode39=12 % '
  \catcode40=12 % (
  \catcode41=12 % )
  \catcode44=12 % ,
  \catcode45=12 % -
  \catcode46=12 % .
  \catcode47=12 % /
  \catcode58=12 % :
  \catcode64=11 % @
  \catcode91=12 % [
  \catcode93=12 % ]
  \catcode123=1 % {
  \catcode125=2 % }
  \expandafter\ifx\csname ProvidesPackage\endcsname\relax
    \def\x#1#2#3[#4]{\endgroup
      \immediate\write-1{Package: #3 #4}%
      \xdef#1{#4}%
    }%
  \else
    \def\x#1#2[#3]{\endgroup
      #2[{#3}]%
      \ifx#1\@undefined
        \xdef#1{#3}%
      \fi
      \ifx#1\relax
        \xdef#1{#3}%
      \fi
    }%
  \fi
\expandafter\x\csname ver@bigintcalc.sty\endcsname
\ProvidesPackage{bigintcalc}%
  [2016/05/16 v1.4 Expandable calculations on big integers (HO)]%
%    \end{macrocode}
%
% \subsection{Catcodes}
%
%    \begin{macrocode}
\begingroup\catcode61\catcode48\catcode32=10\relax%
  \catcode13=5 % ^^M
  \endlinechar=13 %
  \catcode123=1 % {
  \catcode125=2 % }
  \catcode64=11 % @
  \def\x{\endgroup
    \expandafter\edef\csname BIC@AtEnd\endcsname{%
      \endlinechar=\the\endlinechar\relax
      \catcode13=\the\catcode13\relax
      \catcode32=\the\catcode32\relax
      \catcode35=\the\catcode35\relax
      \catcode61=\the\catcode61\relax
      \catcode64=\the\catcode64\relax
      \catcode123=\the\catcode123\relax
      \catcode125=\the\catcode125\relax
    }%
  }%
\x\catcode61\catcode48\catcode32=10\relax%
\catcode13=5 % ^^M
\endlinechar=13 %
\catcode35=6 % #
\catcode64=11 % @
\catcode123=1 % {
\catcode125=2 % }
\def\TMP@EnsureCode#1#2{%
  \edef\BIC@AtEnd{%
    \BIC@AtEnd
    \catcode#1=\the\catcode#1\relax
  }%
  \catcode#1=#2\relax
}
\TMP@EnsureCode{33}{12}% !
\TMP@EnsureCode{36}{14}% $ (comment!)
\TMP@EnsureCode{38}{14}% & (comment!)
\TMP@EnsureCode{40}{12}% (
\TMP@EnsureCode{41}{12}% )
\TMP@EnsureCode{42}{12}% *
\TMP@EnsureCode{43}{12}% +
\TMP@EnsureCode{45}{12}% -
\TMP@EnsureCode{46}{12}% .
\TMP@EnsureCode{47}{12}% /
\TMP@EnsureCode{58}{11}% : (letter!)
\TMP@EnsureCode{60}{12}% <
\TMP@EnsureCode{62}{12}% >
\TMP@EnsureCode{63}{14}% ? (comment!)
\TMP@EnsureCode{91}{12}% [
\TMP@EnsureCode{93}{12}% ]
\edef\BIC@AtEnd{\BIC@AtEnd\noexpand\endinput}
\begingroup\expandafter\expandafter\expandafter\endgroup
\expandafter\ifx\csname BIC@TestMode\endcsname\relax
\else
  \catcode63=9 % ? (ignore)
\fi
? \let\BIC@@TestMode\BIC@TestMode
%    \end{macrocode}
%
% \subsection{\eTeX\ detection}
%
%    \begin{macrocode}
\begingroup\expandafter\expandafter\expandafter\endgroup
\expandafter\ifx\csname numexpr\endcsname\relax
  \catcode36=9 % $ (ignore)
\else
  \catcode38=9 % & (ignore)
\fi
%    \end{macrocode}
%
% \subsection{Help macros}
%
%    \begin{macro}{\BIC@Fi}
%    \begin{macrocode}
\let\BIC@Fi\fi
%    \end{macrocode}
%    \end{macro}
%    \begin{macro}{\BIC@AfterFi}
%    \begin{macrocode}
\def\BIC@AfterFi#1#2\BIC@Fi{\fi#1}%
%    \end{macrocode}
%    \end{macro}
%    \begin{macro}{\BIC@AfterFiFi}
%    \begin{macrocode}
\def\BIC@AfterFiFi#1#2\BIC@Fi{\fi\fi#1}%
%    \end{macrocode}
%    \end{macro}
%    \begin{macro}{\BIC@AfterFiFiFi}
%    \begin{macrocode}
\def\BIC@AfterFiFiFi#1#2\BIC@Fi{\fi\fi\fi#1}%
%    \end{macrocode}
%    \end{macro}
%
%    \begin{macro}{\BIC@Space}
%    \begin{macrocode}
\begingroup
  \def\x#1{\endgroup
    \let\BIC@Space= #1%
  }%
\x{ }
%    \end{macrocode}
%    \end{macro}
%
% \subsection{Expand number}
%
%    \begin{macrocode}
\begingroup\expandafter\expandafter\expandafter\endgroup
\expandafter\ifx\csname RequirePackage\endcsname\relax
  \def\TMP@RequirePackage#1[#2]{%
    \begingroup\expandafter\expandafter\expandafter\endgroup
    \expandafter\ifx\csname ver@#1.sty\endcsname\relax
      \input #1.sty\relax
    \fi
  }%
  \TMP@RequirePackage{pdftexcmds}[2007/11/11]%
\else
  \RequirePackage{pdftexcmds}[2007/11/11]%
\fi
%    \end{macrocode}
%
%    \begin{macrocode}
\begingroup\expandafter\expandafter\expandafter\endgroup
\expandafter\ifx\csname pdf@escapehex\endcsname\relax
%    \end{macrocode}
%
%    \begin{macro}{\BIC@Expand}
%    \begin{macrocode}
  \def\BIC@Expand#1{%
    \romannumeral0%
    \BIC@@Expand#1!\@nil{}%
  }%
%    \end{macrocode}
%    \end{macro}
%    \begin{macro}{\BIC@@Expand}
%    \begin{macrocode}
  \def\BIC@@Expand#1#2\@nil#3{%
    \expandafter\ifcat\noexpand#1\relax
      \expandafter\@firstoftwo
    \else
      \expandafter\@secondoftwo
    \fi
    {%
      \expandafter\BIC@@Expand#1#2\@nil{#3}%
    }{%
      \ifx#1!%
        \expandafter\@firstoftwo
      \else
        \expandafter\@secondoftwo
      \fi
      { #3}{%
        \BIC@@Expand#2\@nil{#3#1}%
      }%
    }%
  }%
%    \end{macrocode}
%    \end{macro}
%    \begin{macro}{\@firstoftwo}
%    \begin{macrocode}
  \expandafter\ifx\csname @firstoftwo\endcsname\relax
    \long\def\@firstoftwo#1#2{#1}%
  \fi
%    \end{macrocode}
%    \end{macro}
%    \begin{macro}{\@secondoftwo}
%    \begin{macrocode}
  \expandafter\ifx\csname @secondoftwo\endcsname\relax
    \long\def\@secondoftwo#1#2{#2}%
  \fi
%    \end{macrocode}
%    \end{macro}
%
%    \begin{macrocode}
\else
%    \end{macrocode}
%    \begin{macro}{\BIC@Expand}
%    \begin{macrocode}
  \def\BIC@Expand#1{%
    \romannumeral0\expandafter\expandafter\expandafter\BIC@Space
    \pdf@unescapehex{%
      \expandafter\expandafter\expandafter
      \BIC@StripHexSpace\pdf@escapehex{#1}20\@nil
    }%
  }%
%    \end{macrocode}
%    \end{macro}
%    \begin{macro}{\BIC@StripHexSpace}
%    \begin{macrocode}
  \def\BIC@StripHexSpace#120#2\@nil{%
    #1%
    \ifx\\#2\\%
    \else
      \BIC@AfterFi{%
        \BIC@StripHexSpace#2\@nil
      }%
    \BIC@Fi
  }%
%    \end{macrocode}
%    \end{macro}
%    \begin{macrocode}
\fi
%    \end{macrocode}
%
% \subsection{Normalize expanded number}
%
%    \begin{macro}{\BIC@Normalize}
%    |#1|: result sign\\
%    |#2|: first token of number
%    \begin{macrocode}
\def\BIC@Normalize#1#2{%
  \ifx#2-%
    \ifx\\#1\\%
      \BIC@AfterFiFi{%
        \BIC@Normalize-%
      }%
    \else
      \BIC@AfterFiFi{%
        \BIC@Normalize{}%
      }%
    \fi
  \else
    \ifx#2+%
      \BIC@AfterFiFi{%
        \BIC@Normalize{#1}%
      }%
    \else
      \ifx#20%
        \BIC@AfterFiFiFi{%
          \BIC@NormalizeZero{#1}%
        }%
      \else
        \BIC@AfterFiFiFi{%
          \BIC@NormalizeDigits#1#2%
        }%
      \fi
    \fi
  \BIC@Fi
}
%    \end{macrocode}
%    \end{macro}
%    \begin{macro}{\BIC@NormalizeZero}
%    \begin{macrocode}
\def\BIC@NormalizeZero#1#2{%
  \ifx#2!%
    \BIC@AfterFi{ 0}%
  \else
    \ifx#20%
      \BIC@AfterFiFi{%
        \BIC@NormalizeZero{#1}%
      }%
    \else
      \BIC@AfterFiFi{%
        \BIC@NormalizeDigits#1#2%
      }%
    \fi
  \BIC@Fi
}
%    \end{macrocode}
%    \end{macro}
%    \begin{macro}{\BIC@NormalizeDigits}
%    \begin{macrocode}
\def\BIC@NormalizeDigits#1!{ #1}
%    \end{macrocode}
%    \end{macro}
%
% \subsection{\op{Num}}
%
%    \begin{macro}{\bigintcalcNum}
%    \begin{macrocode}
\def\bigintcalcNum#1{%
  \romannumeral0%
  \expandafter\expandafter\expandafter\BIC@Normalize
  \expandafter\expandafter\expandafter{%
  \expandafter\expandafter\expandafter}%
  \BIC@Expand{#1}!%
}
%    \end{macrocode}
%    \end{macro}
%
% \subsection{\op{Inv}, \op{Abs}, \op{Sgn}}
%
%
%    \begin{macro}{\bigintcalcInv}
%    \begin{macrocode}
\def\bigintcalcInv#1{%
  \romannumeral0\expandafter\expandafter\expandafter\BIC@Space
  \bigintcalcNum{-#1}%
}
%    \end{macrocode}
%    \end{macro}
%
%    \begin{macro}{\bigintcalcAbs}
%    \begin{macrocode}
\def\bigintcalcAbs#1{%
  \romannumeral0%
  \expandafter\expandafter\expandafter\BIC@Abs
  \bigintcalcNum{#1}%
}
%    \end{macrocode}
%    \end{macro}
%    \begin{macro}{\BIC@Abs}
%    \begin{macrocode}
\def\BIC@Abs#1{%
  \ifx#1-%
    \expandafter\BIC@Space
  \else
    \expandafter\BIC@Space
    \expandafter#1%
  \fi
}
%    \end{macrocode}
%    \end{macro}
%
%    \begin{macro}{\bigintcalcSgn}
%    \begin{macrocode}
\def\bigintcalcSgn#1{%
  \number
  \expandafter\expandafter\expandafter\BIC@Sgn
  \bigintcalcNum{#1}! %
}
%    \end{macrocode}
%    \end{macro}
%    \begin{macro}{\BIC@Sgn}
%    \begin{macrocode}
\def\BIC@Sgn#1#2!{%
  \ifx#1-%
    -1%
  \else
    \ifx#10%
      0%
    \else
      1%
    \fi
  \fi
}
%    \end{macrocode}
%    \end{macro}
%
% \subsection{\op{Cmp}, \op{Min}, \op{Max}}
%
%    \begin{macro}{\bigintcalcCmp}
%    \begin{macrocode}
\def\bigintcalcCmp#1#2{%
  \number
  \expandafter\expandafter\expandafter\BIC@Cmp
  \bigintcalcNum{#2}!{#1}%
}
%    \end{macrocode}
%    \end{macro}
%    \begin{macro}{\BIC@Cmp}
%    \begin{macrocode}
\def\BIC@Cmp#1!#2{%
  \expandafter\expandafter\expandafter\BIC@@Cmp
  \bigintcalcNum{#2}!#1!%
}
%    \end{macrocode}
%    \end{macro}
%    \begin{macro}{\BIC@@Cmp}
%    \begin{macrocode}
\def\BIC@@Cmp#1#2!#3#4!{%
  \ifx#1-%
    \ifx#3-%
      \BIC@AfterFiFi{%
        \BIC@@Cmp#4!#2!%
      }%
    \else
      \BIC@AfterFiFi{%
        -1 %
      }%
    \fi
  \else
    \ifx#3-%
      \BIC@AfterFiFi{%
        1 %
      }%
    \else
      \BIC@AfterFiFi{%
        \BIC@CmpLength#1#2!#3#4!#1#2!#3#4!%
      }%
    \fi
  \BIC@Fi
}
%    \end{macrocode}
%    \end{macro}
%    \begin{macro}{\BIC@PosCmp}
%    \begin{macrocode}
\def\BIC@PosCmp#1!#2!{%
  \BIC@CmpLength#1!#2!#1!#2!%
}
%    \end{macrocode}
%    \end{macro}
%    \begin{macro}{\BIC@CmpLength}
%    \begin{macrocode}
\def\BIC@CmpLength#1#2!#3#4!{%
  \ifx\\#2\\%
    \ifx\\#4\\%
      \BIC@AfterFiFi\BIC@CmpDiff
    \else
      \BIC@AfterFiFi{%
        \BIC@CmpResult{-1}%
      }%
    \fi
  \else
    \ifx\\#4\\%
      \BIC@AfterFiFi{%
        \BIC@CmpResult1%
      }%
    \else
      \BIC@AfterFiFi{%
        \BIC@CmpLength#2!#4!%
      }%
    \fi
  \BIC@Fi
}
%    \end{macrocode}
%    \end{macro}
%    \begin{macro}{\BIC@CmpResult}
%    \begin{macrocode}
\def\BIC@CmpResult#1#2!#3!{#1 }
%    \end{macrocode}
%    \end{macro}
%    \begin{macro}{\BIC@CmpDiff}
%    \begin{macrocode}
\def\BIC@CmpDiff#1#2!#3#4!{%
  \ifnum#1<#3 %
    \BIC@AfterFi{%
      -1 %
    }%
  \else
    \ifnum#1>#3 %
      \BIC@AfterFiFi{%
        1 %
      }%
    \else
      \ifx\\#2\\%
        \BIC@AfterFiFiFi{%
          0 %
        }%
      \else
        \BIC@AfterFiFiFi{%
          \BIC@CmpDiff#2!#4!%
        }%
      \fi
    \fi
  \BIC@Fi
}
%    \end{macrocode}
%    \end{macro}
%
%    \begin{macro}{\bigintcalcMin}
%    \begin{macrocode}
\def\bigintcalcMin#1{%
  \romannumeral0%
  \expandafter\expandafter\expandafter\BIC@MinMax
  \bigintcalcNum{#1}!-!%
}
%    \end{macrocode}
%    \end{macro}
%    \begin{macro}{\bigintcalcMax}
%    \begin{macrocode}
\def\bigintcalcMax#1{%
  \romannumeral0%
  \expandafter\expandafter\expandafter\BIC@MinMax
  \bigintcalcNum{#1}!!%
}
%    \end{macrocode}
%    \end{macro}
%    \begin{macro}{\BIC@MinMax}
%    |#1|: $x$\\
%    |#2|: sign for comparison\\
%    |#3|: $y$
%    \begin{macrocode}
\def\BIC@MinMax#1!#2!#3{%
  \expandafter\expandafter\expandafter\BIC@@MinMax
  \bigintcalcNum{#3}!#1!#2!%
}
%    \end{macrocode}
%    \end{macro}
%    \begin{macro}{\BIC@@MinMax}
%    |#1|: $y$\\
%    |#2|: $x$\\
%    |#3|: sign for comparison
%    \begin{macrocode}
\def\BIC@@MinMax#1!#2!#3!{%
  \ifnum\BIC@@Cmp#1!#2!=#31 %
    \BIC@AfterFi{ #1}%
  \else
    \BIC@AfterFi{ #2}%
  \BIC@Fi
}
%    \end{macrocode}
%    \end{macro}
%
% \subsection{\op{Odd}}
%
%    \begin{macro}{\bigintcalcOdd}
%    \begin{macrocode}
\def\bigintcalcOdd#1{%
  \romannumeral0%
  \expandafter\expandafter\expandafter\BIC@Odd
  \bigintcalcAbs{#1}!%
}
%    \end{macrocode}
%    \end{macro}
%    \begin{macro}{\BigIntCalcOdd}
%    \begin{macrocode}
\def\BigIntCalcOdd#1!{%
  \romannumeral0%
  \BIC@Odd#1!%
}
%    \end{macrocode}
%    \end{macro}
%    \begin{macro}{\BIC@Odd}
%    |#1|: $x$
%    \begin{macrocode}
\def\BIC@Odd#1#2{%
  \ifx#2!%
    \ifodd#1 %
      \BIC@AfterFiFi{ 1}%
    \else
      \BIC@AfterFiFi{ 0}%
    \fi
  \else
    \expandafter\BIC@Odd\expandafter#2%
  \BIC@Fi
}
%    \end{macrocode}
%    \end{macro}
%
% \subsection{\op{Inc}, \op{Dec}}
%
%    \begin{macro}{\bigintcalcInc}
%    \begin{macrocode}
\def\bigintcalcInc#1{%
  \romannumeral0%
  \expandafter\expandafter\expandafter\BIC@IncSwitch
  \bigintcalcNum{#1}!%
}
%    \end{macrocode}
%    \end{macro}
%    \begin{macro}{\BIC@IncSwitch}
%    \begin{macrocode}
\def\BIC@IncSwitch#1#2!{%
  \ifcase\BIC@@Cmp#1#2!-1!%
    \BIC@AfterFi{ 0}%
  \or
    \BIC@AfterFi{%
      \BIC@Inc#1#2!{}%
    }%
  \else
    \BIC@AfterFi{%
      \expandafter-\romannumeral0%
      \BIC@Dec#2!{}%
    }%
  \BIC@Fi
}
%    \end{macrocode}
%    \end{macro}
%
%    \begin{macro}{\bigintcalcDec}
%    \begin{macrocode}
\def\bigintcalcDec#1{%
  \romannumeral0%
  \expandafter\expandafter\expandafter\BIC@DecSwitch
  \bigintcalcNum{#1}!%
}
%    \end{macrocode}
%    \end{macro}
%    \begin{macro}{\BIC@DecSwitch}
%    \begin{macrocode}
\def\BIC@DecSwitch#1#2!{%
  \ifcase\BIC@Sgn#1#2! %
    \BIC@AfterFi{ -1}%
  \or
    \BIC@AfterFi{%
      \BIC@Dec#1#2!{}%
    }%
  \else
    \BIC@AfterFi{%
      \expandafter-\romannumeral0%
      \BIC@Inc#2!{}%
    }%
  \BIC@Fi
}
%    \end{macrocode}
%    \end{macro}
%
%    \begin{macro}{\BigIntCalcInc}
%    \begin{macrocode}
\def\BigIntCalcInc#1!{%
  \romannumeral0\BIC@Inc#1!{}%
}
%    \end{macrocode}
%    \end{macro}
%    \begin{macro}{\BigIntCalcDec}
%    \begin{macrocode}
\def\BigIntCalcDec#1!{%
  \romannumeral0\BIC@Dec#1!{}%
}
%    \end{macrocode}
%    \end{macro}
%
%    \begin{macro}{\BIC@Inc}
%    \begin{macrocode}
\def\BIC@Inc#1#2!#3{%
  \ifx\\#2\\%
    \BIC@AfterFi{%
      \BIC@@Inc1#1#3!{}%
    }%
  \else
    \BIC@AfterFi{%
      \BIC@Inc#2!{#1#3}%
    }%
  \BIC@Fi
}
%    \end{macrocode}
%    \end{macro}
%    \begin{macro}{\BIC@@Inc}
%    \begin{macrocode}
\def\BIC@@Inc#1#2#3!#4{%
  \ifcase#1 %
    \ifx\\#3\\%
      \BIC@AfterFiFi{ #2#4}%
    \else
      \BIC@AfterFiFi{%
        \BIC@@Inc0#3!{#2#4}%
      }%
    \fi
  \else
    \ifnum#2<9 %
      \BIC@AfterFiFi{%
&       \expandafter\BIC@@@Inc\the\numexpr#2+1\relax
$       \expandafter\expandafter\expandafter\BIC@@@Inc
$       \ifcase#2 \expandafter1%
$       \or\expandafter2%
$       \or\expandafter3%
$       \or\expandafter4%
$       \or\expandafter5%
$       \or\expandafter6%
$       \or\expandafter7%
$       \or\expandafter8%
$       \or\expandafter9%
$?      \else\BigIntCalcError:ThisCannotHappen%
$       \fi
        0#3!{#4}%
      }%
    \else
      \BIC@AfterFiFi{%
        \BIC@@@Inc01#3!{#4}%
      }%
    \fi
  \BIC@Fi
}
%    \end{macrocode}
%    \end{macro}
%    \begin{macro}{\BIC@@@Inc}
%    \begin{macrocode}
\def\BIC@@@Inc#1#2#3!#4{%
  \ifx\\#3\\%
    \ifnum#2=1 %
      \BIC@AfterFiFi{ 1#1#4}%
    \else
      \BIC@AfterFiFi{ #1#4}%
    \fi
  \else
    \BIC@AfterFi{%
      \BIC@@Inc#2#3!{#1#4}%
    }%
  \BIC@Fi
}
%    \end{macrocode}
%    \end{macro}
%
%    \begin{macro}{\BIC@Dec}
%    \begin{macrocode}
\def\BIC@Dec#1#2!#3{%
  \ifx\\#2\\%
    \BIC@AfterFi{%
      \BIC@@Dec1#1#3!{}%
    }%
  \else
    \BIC@AfterFi{%
      \BIC@Dec#2!{#1#3}%
    }%
  \BIC@Fi
}
%    \end{macrocode}
%    \end{macro}
%    \begin{macro}{\BIC@@Dec}
%    \begin{macrocode}
\def\BIC@@Dec#1#2#3!#4{%
  \ifcase#1 %
    \ifx\\#3\\%
      \BIC@AfterFiFi{ #2#4}%
    \else
      \BIC@AfterFiFi{%
        \BIC@@Dec0#3!{#2#4}%
      }%
    \fi
  \else
    \ifnum#2>0 %
      \BIC@AfterFiFi{%
&       \expandafter\BIC@@@Dec\the\numexpr#2-1\relax
$       \expandafter\expandafter\expandafter\BIC@@@Dec
$       \ifcase#2
$?        \BigIntCalcError:ThisCannotHappen%
$       \or\expandafter0%
$       \or\expandafter1%
$       \or\expandafter2%
$       \or\expandafter3%
$       \or\expandafter4%
$       \or\expandafter5%
$       \or\expandafter6%
$       \or\expandafter7%
$       \or\expandafter8%
$?      \else\BigIntCalcError:ThisCannotHappen%
$       \fi
        0#3!{#4}%
      }%
    \else
      \BIC@AfterFiFi{%
        \BIC@@@Dec91#3!{#4}%
      }%
    \fi
  \BIC@Fi
}
%    \end{macrocode}
%    \end{macro}
%    \begin{macro}{\BIC@@@Dec}
%    \begin{macrocode}
\def\BIC@@@Dec#1#2#3!#4{%
  \ifx\\#3\\%
    \ifcase#1 %
      \ifx\\#4\\%
        \BIC@AfterFiFiFi{ 0}%
      \else
        \BIC@AfterFiFiFi{ #4}%
      \fi
    \else
      \BIC@AfterFiFi{ #1#4}%
    \fi
  \else
    \BIC@AfterFi{%
      \BIC@@Dec#2#3!{#1#4}%
    }%
  \BIC@Fi
}
%    \end{macrocode}
%    \end{macro}
%
% \subsection{\op{Add}, \op{Sub}}
%
%    \begin{macro}{\bigintcalcAdd}
%    \begin{macrocode}
\def\bigintcalcAdd#1{%
  \romannumeral0%
  \expandafter\expandafter\expandafter\BIC@Add
  \bigintcalcNum{#1}!%
}
%    \end{macrocode}
%    \end{macro}
%    \begin{macro}{\BIC@Add}
%    \begin{macrocode}
\def\BIC@Add#1!#2{%
  \expandafter\expandafter\expandafter
  \BIC@AddSwitch\bigintcalcNum{#2}!#1!%
}
%    \end{macrocode}
%    \end{macro}
%    \begin{macro}{\bigintcalcSub}
%    \begin{macrocode}
\def\bigintcalcSub#1#2{%
  \romannumeral0%
  \expandafter\expandafter\expandafter\BIC@Add
  \bigintcalcNum{-#2}!{#1}%
}
%    \end{macrocode}
%    \end{macro}
%
%    \begin{macro}{\BIC@AddSwitch}
%    Decision table for \cs{BIC@AddSwitch}.
%    \begin{quote}
%    \begin{tabular}[t]{@{}|l|l|l|l|l|@{}}
%      \hline
%      $x<0$ & $y<0$ & $-x>-y$ & $-$ & $\opAdd(-x,-y)$\\
%      \cline{3-3}\cline{5-5}
%            &       & else    &     & $\opAdd(-y,-x)$\\
%      \cline{2-5}
%            & else  & $-x> y$ & $-$ & $\opSub(-x, y)$\\
%      \cline{3-5}
%            &       & $-x= y$ &     & $0$\\
%      \cline{3-5}
%            &       & else    & $+$ & $\opSub( y,-x)$\\
%      \hline
%      else  & $y<0$ & $ x>-y$ & $+$ & $\opSub( x,-y)$\\
%      \cline{3-5}
%            &       & $ x=-y$ &     & $0$\\
%      \cline{3-5}
%            &       & else    & $-$ & $\opSub(-y, x)$\\
%      \cline{2-5}
%            & else  & $ x> y$ & $+$ & $\opAdd( x, y)$\\
%      \cline{3-3}\cline{5-5}
%            &       & else    &     & $\opAdd( y, x)$\\
%      \hline
%    \end{tabular}
%    \end{quote}
%    \begin{macrocode}
\def\BIC@AddSwitch#1#2!#3#4!{%
  \ifx#1-% x < 0
    \ifx#3-% y < 0
      \expandafter-\romannumeral0%
      \ifnum\BIC@PosCmp#2!#4!=1 % -x > -y
        \BIC@AfterFiFiFi{%
          \BIC@AddXY#2!#4!!!%
        }%
      \else % -x <= -y
        \BIC@AfterFiFiFi{%
          \BIC@AddXY#4!#2!!!%
        }%
      \fi
    \else % y >= 0
      \ifcase\BIC@PosCmp#2!#3#4!% -x = y
        \BIC@AfterFiFiFi{ 0}%
      \or % -x > y
        \expandafter-\romannumeral0%
        \BIC@AfterFiFiFi{%
          \BIC@SubXY#2!#3#4!!!%
        }%
      \else % -x <= y
        \BIC@AfterFiFiFi{%
          \BIC@SubXY#3#4!#2!!!%
        }%
      \fi
    \fi
  \else % x >= 0
    \ifx#3-% y < 0
      \ifcase\BIC@PosCmp#1#2!#4!% x = -y
        \BIC@AfterFiFiFi{ 0}%
      \or % x > -y
        \BIC@AfterFiFiFi{%
          \BIC@SubXY#1#2!#4!!!%
        }%
      \else % x <= -y
        \expandafter-\romannumeral0%
        \BIC@AfterFiFiFi{%
          \BIC@SubXY#4!#1#2!!!%
        }%
      \fi
    \else % y >= 0
      \ifnum\BIC@PosCmp#1#2!#3#4!=1 % x > y
        \BIC@AfterFiFiFi{%
          \BIC@AddXY#1#2!#3#4!!!%
        }%
      \else % x <= y
        \BIC@AfterFiFiFi{%
          \BIC@AddXY#3#4!#1#2!!!%
        }%
      \fi
    \fi
  \BIC@Fi
}
%    \end{macrocode}
%    \end{macro}
%
%    \begin{macro}{\BigIntCalcAdd}
%    \begin{macrocode}
\def\BigIntCalcAdd#1!#2!{%
  \romannumeral0\BIC@AddXY#1!#2!!!%
}
%    \end{macrocode}
%    \end{macro}
%    \begin{macro}{\BigIntCalcSub}
%    \begin{macrocode}
\def\BigIntCalcSub#1!#2!{%
  \romannumeral0\BIC@SubXY#1!#2!!!%
}
%    \end{macrocode}
%    \end{macro}
%
%    \begin{macro}{\BIC@AddXY}
%    \begin{macrocode}
\def\BIC@AddXY#1#2!#3#4!#5!#6!{%
  \ifx\\#2\\%
    \ifx\\#3\\%
      \BIC@AfterFiFi{%
        \BIC@DoAdd0!#1#5!#60!%
      }%
    \else
      \BIC@AfterFiFi{%
        \BIC@DoAdd0!#1#5!#3#6!%
      }%
    \fi
  \else
    \ifx\\#4\\%
      \ifx\\#3\\%
        \BIC@AfterFiFiFi{%
          \BIC@AddXY#2!{}!#1#5!#60!%
        }%
      \else
        \BIC@AfterFiFiFi{%
          \BIC@AddXY#2!{}!#1#5!#3#6!%
        }%
      \fi
    \else
      \BIC@AfterFiFi{%
        \BIC@AddXY#2!#4!#1#5!#3#6!%
      }%
    \fi
  \BIC@Fi
}
%    \end{macrocode}
%    \end{macro}
%    \begin{macro}{\BIC@DoAdd}
%    |#1|: carry\\
%    |#2|: reverted result\\
%    |#3#4|: reverted $x$\\
%    |#5#6|: reverted $y$
%    \begin{macrocode}
\def\BIC@DoAdd#1#2!#3#4!#5#6!{%
  \ifx\\#4\\%
    \BIC@AfterFi{%
&     \expandafter\BIC@Space
&     \the\numexpr#1+#3+#5\relax#2%
$     \expandafter\expandafter\expandafter\BIC@AddResult
$     \BIC@AddDigit#1#3#5#2%
    }%
  \else
    \BIC@AfterFi{%
      \expandafter\expandafter\expandafter\BIC@DoAdd
      \BIC@AddDigit#1#3#5#2!#4!#6!%
    }%
  \BIC@Fi
}
%    \end{macrocode}
%    \end{macro}
%    \begin{macro}{\BIC@AddResult}
%    \begin{macrocode}
$ \def\BIC@AddResult#1{%
$   \ifx#10%
$     \expandafter\BIC@Space
$   \else
$     \expandafter\BIC@Space\expandafter#1%
$   \fi
$ }%
%    \end{macrocode}
%    \end{macro}
%    \begin{macro}{\BIC@AddDigit}
%    |#1|: carry\\
%    |#2|: digit of $x$\\
%    |#3|: digit of $y$
%    \begin{macrocode}
\def\BIC@AddDigit#1#2#3{%
  \romannumeral0%
& \expandafter\BIC@@AddDigit\the\numexpr#1+#2+#3!%
$ \expandafter\BIC@@AddDigit\number%
$ \csname
$   BIC@AddCarry%
$   \ifcase#1 %
$     #2%
$   \else
$     \ifcase#2 1\or2\or3\or4\or5\or6\or7\or8\or9\or10\fi
$   \fi
$ \endcsname#3!%
}
%    \end{macrocode}
%    \end{macro}
%    \begin{macro}{\BIC@@AddDigit}
%    \begin{macrocode}
\def\BIC@@AddDigit#1!{%
  \ifnum#1<10 %
    \BIC@AfterFi{ 0#1}%
  \else
    \BIC@AfterFi{ #1}%
  \BIC@Fi
}
%    \end{macrocode}
%    \end{macro}
%    \begin{macro}{\BIC@AddCarry0}
%    \begin{macrocode}
$ \expandafter\def\csname BIC@AddCarry0\endcsname#1{#1}%
%    \end{macrocode}
%    \end{macro}
%    \begin{macro}{\BIC@AddCarry10}
%    \begin{macrocode}
$ \expandafter\def\csname BIC@AddCarry10\endcsname#1{1#1}%
%    \end{macrocode}
%    \end{macro}
%    \begin{macro}{\BIC@AddCarry[1-9]}
%    \begin{macrocode}
$ \def\BIC@Temp#1#2{%
$   \expandafter\def\csname BIC@AddCarry#1\endcsname##1{%
$     \ifcase##1 #1\or
$     #2%
$?    \else\BigIntCalcError:ThisCannotHappen%
$     \fi
$   }%
$ }%
$ \BIC@Temp 0{1\or2\or3\or4\or5\or6\or7\or8\or9}%
$ \BIC@Temp 1{2\or3\or4\or5\or6\or7\or8\or9\or10}%
$ \BIC@Temp 2{3\or4\or5\or6\or7\or8\or9\or10\or11}%
$ \BIC@Temp 3{4\or5\or6\or7\or8\or9\or10\or11\or12}%
$ \BIC@Temp 4{5\or6\or7\or8\or9\or10\or11\or12\or13}%
$ \BIC@Temp 5{6\or7\or8\or9\or10\or11\or12\or13\or14}%
$ \BIC@Temp 6{7\or8\or9\or10\or11\or12\or13\or14\or15}%
$ \BIC@Temp 7{8\or9\or10\or11\or12\or13\or14\or15\or16}%
$ \BIC@Temp 8{9\or10\or11\or12\or13\or14\or15\or16\or17}%
$ \BIC@Temp 9{10\or11\or12\or13\or14\or15\or16\or17\or18}%
%    \end{macrocode}
%    \end{macro}
%
%    \begin{macro}{\BIC@SubXY}
%    Preconditions:
%    \begin{itemize}
%    \item $x > y$, $x \geq 0$, and $y >= 0$
%    \item digits($x$) = digits($y$)
%    \end{itemize}
%    \begin{macrocode}
\def\BIC@SubXY#1#2!#3#4!#5!#6!{%
  \ifx\\#2\\%
    \ifx\\#3\\%
      \BIC@AfterFiFi{%
        \BIC@DoSub0!#1#5!#60!%
      }%
    \else
      \BIC@AfterFiFi{%
        \BIC@DoSub0!#1#5!#3#6!%
      }%
    \fi
  \else
    \ifx\\#4\\%
      \ifx\\#3\\%
        \BIC@AfterFiFiFi{%
          \BIC@SubXY#2!{}!#1#5!#60!%
        }%
      \else
        \BIC@AfterFiFiFi{%
          \BIC@SubXY#2!{}!#1#5!#3#6!%
        }%
      \fi
    \else
      \BIC@AfterFiFi{%
        \BIC@SubXY#2!#4!#1#5!#3#6!%
      }%
    \fi
  \BIC@Fi
}
%    \end{macrocode}
%    \end{macro}
%    \begin{macro}{\BIC@DoSub}
%    |#1|: carry\\
%    |#2|: reverted result\\
%    |#3#4|: reverted x\\
%    |#5#6|: reverted y
%    \begin{macrocode}
\def\BIC@DoSub#1#2!#3#4!#5#6!{%
  \ifx\\#4\\%
    \BIC@AfterFi{%
      \expandafter\expandafter\expandafter\BIC@SubResult
      \BIC@SubDigit#1#3#5#2%
    }%
  \else
    \BIC@AfterFi{%
      \expandafter\expandafter\expandafter\BIC@DoSub
      \BIC@SubDigit#1#3#5#2!#4!#6!%
    }%
  \BIC@Fi
}
%    \end{macrocode}
%    \end{macro}
%    \begin{macro}{\BIC@SubResult}
%    \begin{macrocode}
\def\BIC@SubResult#1{%
  \ifx#10%
    \expandafter\BIC@SubResult
  \else
    \expandafter\BIC@Space\expandafter#1%
  \fi
}
%    \end{macrocode}
%    \end{macro}
%    \begin{macro}{\BIC@SubDigit}
%    |#1|: carry\\
%    |#2|: digit of $x$\\
%    |#3|: digit of $y$
%    \begin{macrocode}
\def\BIC@SubDigit#1#2#3{%
  \romannumeral0%
& \expandafter\BIC@@SubDigit\the\numexpr#2-#3-#1!%
$ \expandafter\BIC@@AddDigit\number
$   \csname
$     BIC@SubCarry%
$     \ifcase#1 %
$       #3%
$     \else
$       \ifcase#3 1\or2\or3\or4\or5\or6\or7\or8\or9\or10\fi
$     \fi
$   \endcsname#2!%
}
%    \end{macrocode}
%    \end{macro}
%    \begin{macro}{\BIC@@SubDigit}
%    \begin{macrocode}
& \def\BIC@@SubDigit#1!{%
&   \ifnum#1<0 %
&     \BIC@AfterFi{%
&       \expandafter\BIC@Space
&       \expandafter1\the\numexpr#1+10\relax
&     }%
&   \else
&     \BIC@AfterFi{ 0#1}%
&   \BIC@Fi
& }%
%    \end{macrocode}
%    \end{macro}
%    \begin{macro}{\BIC@SubCarry0}
%    \begin{macrocode}
$ \expandafter\def\csname BIC@SubCarry0\endcsname#1{#1}%
%    \end{macrocode}
%    \end{macro}
%    \begin{macro}{\BIC@SubCarry10}%
%    \begin{macrocode}
$ \expandafter\def\csname BIC@SubCarry10\endcsname#1{1#1}%
%    \end{macrocode}
%    \end{macro}
%    \begin{macro}{\BIC@SubCarry[1-9]}
%    \begin{macrocode}
$ \def\BIC@Temp#1#2{%
$   \expandafter\def\csname BIC@SubCarry#1\endcsname##1{%
$     \ifcase##1 #2%
$?    \else\BigIntCalcError:ThisCannotHappen%
$     \fi
$   }%
$ }%
$ \BIC@Temp 1{19\or0\or1\or2\or3\or4\or5\or6\or7\or8}%
$ \BIC@Temp 2{18\or19\or0\or1\or2\or3\or4\or5\or6\or7}%
$ \BIC@Temp 3{17\or18\or19\or0\or1\or2\or3\or4\or5\or6}%
$ \BIC@Temp 4{16\or17\or18\or19\or0\or1\or2\or3\or4\or5}%
$ \BIC@Temp 5{15\or16\or17\or18\or19\or0\or1\or2\or3\or4}%
$ \BIC@Temp 6{14\or15\or16\or17\or18\or19\or0\or1\or2\or3}%
$ \BIC@Temp 7{13\or14\or15\or16\or17\or18\or19\or0\or1\or2}%
$ \BIC@Temp 8{12\or13\or14\or15\or16\or17\or18\or19\or0\or1}%
$ \BIC@Temp 9{11\or12\or13\or14\or15\or16\or17\or18\or19\or0}%
%    \end{macrocode}
%    \end{macro}
%
% \subsection{\op{Shl}, \op{Shr}}
%
%    \begin{macro}{\bigintcalcShl}
%    \begin{macrocode}
\def\bigintcalcShl#1{%
  \romannumeral0%
  \expandafter\expandafter\expandafter\BIC@Shl
  \bigintcalcNum{#1}!%
}
%    \end{macrocode}
%    \end{macro}
%    \begin{macro}{\BIC@Shl}
%    \begin{macrocode}
\def\BIC@Shl#1#2!{%
  \ifx#1-%
    \BIC@AfterFi{%
      \expandafter-\romannumeral0%
&     \BIC@@Shl#2!!%
$     \BIC@AddXY#2!#2!!!%
    }%
  \else
    \BIC@AfterFi{%
&     \BIC@@Shl#1#2!!%
$     \BIC@AddXY#1#2!#1#2!!!%
    }%
  \BIC@Fi
}
%    \end{macrocode}
%    \end{macro}
%    \begin{macro}{\BigIntCalcShl}
%    \begin{macrocode}
\def\BigIntCalcShl#1!{%
  \romannumeral0%
& \BIC@@Shl#1!!%
$ \BIC@AddXY#1!#1!!!%
}
%    \end{macrocode}
%    \end{macro}
%    \begin{macro}{\BIC@@Shl}
%    \begin{macrocode}
& \def\BIC@@Shl#1#2!{%
&   \ifx\\#2\\%
&     \BIC@AfterFi{%
&       \BIC@@@Shl0!#1%
&     }%
&   \else
&     \BIC@AfterFi{%
&       \BIC@@Shl#2!#1%
&     }%
&   \BIC@Fi
& }%
%    \end{macrocode}
%    \end{macro}
%    \begin{macro}{\BIC@@@Shl}
%    |#1|: carry\\
%    |#2|: result\\
%    |#3#4|: reverted number
%    \begin{macrocode}
& \def\BIC@@@Shl#1#2!#3#4!{%
&   \ifx\\#4\\%
&     \BIC@AfterFi{%
&       \expandafter\BIC@Space
&       \the\numexpr#3*2+#1\relax#2%
&     }%
&   \else
&     \BIC@AfterFi{%
&       \expandafter\BIC@@@@Shl\the\numexpr#3*2+#1!#2!#4!%
&     }%
&   \BIC@Fi
& }%
%    \end{macrocode}
%    \end{macro}
%    \begin{macro}{\BIC@@@@Shl}
%    \begin{macrocode}
& \def\BIC@@@@Shl#1!{%
&   \ifnum#1<10 %
&     \BIC@AfterFi{%
&       \BIC@@@Shl0#1%
&     }%
&   \else
&     \BIC@AfterFi{%
&       \BIC@@@Shl#1%
&     }%
&   \BIC@Fi
& }%
%    \end{macrocode}
%    \end{macro}
%
%    \begin{macro}{\bigintcalcShr}
%    \begin{macrocode}
\def\bigintcalcShr#1{%
  \romannumeral0%
  \expandafter\expandafter\expandafter\BIC@Shr
  \bigintcalcNum{#1}!%
}
%    \end{macrocode}
%    \end{macro}
%    \begin{macro}{\BIC@Shr}
%    \begin{macrocode}
\def\BIC@Shr#1#2!{%
  \ifx#1-%
    \expandafter-\romannumeral0%
    \BIC@AfterFi{%
      \BIC@@Shr#2!%
    }%
  \else
    \BIC@AfterFi{%
      \BIC@@Shr#1#2!%
    }%
  \BIC@Fi
}
%    \end{macrocode}
%    \end{macro}
%    \begin{macro}{\BigIntCalcShr}
%    \begin{macrocode}
\def\BigIntCalcShr#1!{%
  \romannumeral0%
  \BIC@@Shr#1!%
}
%    \end{macrocode}
%    \end{macro}
%    \begin{macro}{\BIC@@Shr}
%    \begin{macrocode}
\def\BIC@@Shr#1#2!{%
  \ifcase#1 %
    \BIC@AfterFi{ 0}%
  \or
    \ifx\\#2\\%
      \BIC@AfterFiFi{ 0}%
    \else
      \BIC@AfterFiFi{%
        \BIC@@@Shr#1#2!!%
      }%
    \fi
  \else
    \BIC@AfterFi{%
      \BIC@@@Shr0#1#2!!%
    }%
  \BIC@Fi
}
%    \end{macrocode}
%    \end{macro}
%    \begin{macro}{\BIC@@@Shr}
%    |#1|: carry\\
%    |#2#3|: number\\
%    |#4|: result
%    \begin{macrocode}
\def\BIC@@@Shr#1#2#3!#4!{%
  \ifx\\#3\\%
    \ifodd#1#2 %
      \BIC@AfterFiFi{%
&       \expandafter\BIC@ShrResult\the\numexpr(#1#2-1)/2\relax
$       \expandafter\expandafter\expandafter\BIC@ShrResult
$       \csname BIC@ShrDigit#1#2\endcsname
        #4!%
      }%
    \else
      \BIC@AfterFiFi{%
&       \expandafter\BIC@ShrResult\the\numexpr#1#2/2\relax
$       \expandafter\expandafter\expandafter\BIC@ShrResult
$       \csname BIC@ShrDigit#1#2\endcsname
        #4!%
      }%
    \fi
  \else
    \ifodd#1#2 %
      \BIC@AfterFiFi{%
&       \expandafter\BIC@@@@Shr\the\numexpr(#1#2-1)/2\relax1%
$       \expandafter\expandafter\expandafter\BIC@@@@Shr
$       \csname BIC@ShrDigit#1#2\endcsname
        #3!#4!%
      }%
    \else
      \BIC@AfterFiFi{%
&       \expandafter\BIC@@@@Shr\the\numexpr#1#2/2\relax0%
$       \expandafter\expandafter\expandafter\BIC@@@@Shr
$       \csname BIC@ShrDigit#1#2\endcsname
        #3!#4!%
      }%
    \fi
  \BIC@Fi
}
%    \end{macrocode}
%    \end{macro}
%    \begin{macro}{\BIC@ShrResult}
%    \begin{macrocode}
& \def\BIC@ShrResult#1#2!{ #2#1}%
$ \def\BIC@ShrResult#1#2#3!{ #3#1}%
%    \end{macrocode}
%    \end{macro}
%    \begin{macro}{\BIC@@@@Shr}
%    |#1|: new digit\\
%    |#2|: carry\\
%    |#3|: remaining number\\
%    |#4|: result
%    \begin{macrocode}
\def\BIC@@@@Shr#1#2#3!#4!{%
  \BIC@@@Shr#2#3!#4#1!%
}
%    \end{macrocode}
%    \end{macro}
%    \begin{macro}{\BIC@ShrDigit[00-19]}
%    \begin{macrocode}
$ \def\BIC@Temp#1#2#3#4{%
$   \expandafter\def\csname BIC@ShrDigit#1#2\endcsname{#3#4}%
$ }%
$ \BIC@Temp 0000%
$ \BIC@Temp 0101%
$ \BIC@Temp 0210%
$ \BIC@Temp 0311%
$ \BIC@Temp 0420%
$ \BIC@Temp 0521%
$ \BIC@Temp 0630%
$ \BIC@Temp 0731%
$ \BIC@Temp 0840%
$ \BIC@Temp 0941%
$ \BIC@Temp 1050%
$ \BIC@Temp 1151%
$ \BIC@Temp 1260%
$ \BIC@Temp 1361%
$ \BIC@Temp 1470%
$ \BIC@Temp 1571%
$ \BIC@Temp 1680%
$ \BIC@Temp 1781%
$ \BIC@Temp 1890%
$ \BIC@Temp 1991%
%    \end{macrocode}
%    \end{macro}
%
% \subsection{\cs{BIC@Tim}}
%
%    \begin{macro}{\BIC@Tim}
%    Macro \cs{BIC@Tim} implements ``Number \emph{tim}es digit''.\\
%    |#1|: plain number without sign\\
%    |#2|: digit
\def\BIC@Tim#1!#2{%
  \romannumeral0%
  \ifcase#2 % 0
    \BIC@AfterFi{ 0}%
  \or % 1
    \BIC@AfterFi{ #1}%
  \or % 2
    \BIC@AfterFi{%
      \BIC@Shl#1!%
    }%
  \else % 3-9
    \BIC@AfterFi{%
      \BIC@@Tim#1!!#2%
    }%
  \BIC@Fi
}
%    \begin{macrocode}
%    \end{macrocode}
%    \end{macro}
%    \begin{macro}{\BIC@@Tim}
%    |#1#2|: number\\
%    |#3|: reverted number
%    \begin{macrocode}
\def\BIC@@Tim#1#2!{%
  \ifx\\#2\\%
    \BIC@AfterFi{%
      \BIC@ProcessTim0!#1%
    }%
  \else
    \BIC@AfterFi{%
      \BIC@@Tim#2!#1%
    }%
  \BIC@Fi
}
%    \end{macrocode}
%    \end{macro}
%    \begin{macro}{\BIC@ProcessTim}
%    |#1|: carry\\
%    |#2|: result\\
%    |#3#4|: reverted number\\
%    |#5|: digit
%    \begin{macrocode}
\def\BIC@ProcessTim#1#2!#3#4!#5{%
  \ifx\\#4\\%
    \BIC@AfterFi{%
      \expandafter\BIC@Space
&     \the\numexpr#3*#5+#1\relax
$     \romannumeral0\BIC@TimDigit#3#5#1%
      #2%
    }%
  \else
    \BIC@AfterFi{%
      \expandafter\BIC@@ProcessTim
&     \the\numexpr#3*#5+#1%
$     \romannumeral0\BIC@TimDigit#3#5#1%
      !#2!#4!#5%
    }%
  \BIC@Fi
}
%    \end{macrocode}
%    \end{macro}
%    \begin{macro}{\BIC@@ProcessTim}
%    |#1#2|: carry?, new digit\\
%    |#3|: new number\\
%    |#4|: old number\\
%    |#5|: digit
%    \begin{macrocode}
\def\BIC@@ProcessTim#1#2!{%
  \ifx\\#2\\%
    \BIC@AfterFi{%
      \BIC@ProcessTim0#1%
    }%
  \else
    \BIC@AfterFi{%
      \BIC@ProcessTim#1#2%
    }%
  \BIC@Fi
}
%    \end{macrocode}
%    \end{macro}
%    \begin{macro}{\BIC@TimDigit}
%    |#1|: digit 0--9\\
%    |#2|: digit 3--9\\
%    |#3|: carry 0--9
%    \begin{macrocode}
$ \def\BIC@TimDigit#1#2#3{%
$   \ifcase#1 % 0
$     \BIC@AfterFi{ #3}%
$   \or % 1
$     \BIC@AfterFi{%
$       \expandafter\BIC@Space
$       \number\csname BIC@AddCarry#2\endcsname#3 %
$     }%
$   \else
$     \ifcase#3 %
$       \BIC@AfterFiFi{%
$         \expandafter\BIC@Space
$         \number\csname BIC@MulDigit#2\endcsname#1 %
$       }%
$     \else
$       \BIC@AfterFiFi{%
$         \expandafter\BIC@Space
$         \romannumeral0%
$         \expandafter\BIC@AddXY
$         \number\csname BIC@MulDigit#2\endcsname#1!%
$         #3!!!%
$       }%
$     \fi
$   \BIC@Fi
$ }%
%    \end{macrocode}
%    \end{macro}
%    \begin{macro}{\BIC@MulDigit[3-9]}
%    \begin{macrocode}
$ \def\BIC@Temp#1#2{%
$   \expandafter\def\csname BIC@MulDigit#1\endcsname##1{%
$     \ifcase##1 0%
$     \or ##1%
$     \or #2%
$?    \else\BigIntCalcError:ThisCannotHappen%
$     \fi
$   }%
$ }%
$ \BIC@Temp 3{6\or9\or12\or15\or18\or21\or24\or27}%
$ \BIC@Temp 4{8\or12\or16\or20\or24\or28\or32\or36}%
$ \BIC@Temp 5{10\or15\or20\or25\or30\or35\or40\or45}%
$ \BIC@Temp 6{12\or18\or24\or30\or36\or42\or48\or54}%
$ \BIC@Temp 7{14\or21\or28\or35\or42\or49\or56\or63}%
$ \BIC@Temp 8{16\or24\or32\or40\or48\or56\or64\or72}%
$ \BIC@Temp 9{18\or27\or36\or45\or54\or63\or72\or81}%
%    \end{macrocode}
%    \end{macro}
%
% \subsection{\op{Mul}}
%
%    \begin{macro}{\bigintcalcMul}
%    \begin{macrocode}
\def\bigintcalcMul#1#2{%
  \romannumeral0%
  \expandafter\expandafter\expandafter\BIC@Mul
  \bigintcalcNum{#1}!{#2}%
}
%    \end{macrocode}
%    \end{macro}
%    \begin{macro}{\BIC@Mul}
%    \begin{macrocode}
\def\BIC@Mul#1!#2{%
  \expandafter\expandafter\expandafter\BIC@MulSwitch
  \bigintcalcNum{#2}!#1!%
}
%    \end{macrocode}
%    \end{macro}
%    \begin{macro}{\BIC@MulSwitch}
%    Decision table for \cs{BIC@MulSwitch}.
%    \begin{quote}
%    \begin{tabular}[t]{@{}|l|l|l|l|l|@{}}
%      \hline
%      $x=0$ &\multicolumn{3}{l}{}    &$0$\\
%      \hline
%      $x>0$ & $y=0$ &\multicolumn2l{}&$0$\\
%      \cline{2-5}
%            & $y>0$ & $ x> y$ & $+$ & $\opMul( x, y)$\\
%      \cline{3-3}\cline{5-5}
%            &       & else    &     & $\opMul( y, x)$\\
%      \cline{2-5}
%            & $y<0$ & $ x>-y$ & $-$ & $\opMul( x,-y)$\\
%      \cline{3-3}\cline{5-5}
%            &       & else    &     & $\opMul(-y, x)$\\
%      \hline
%      $x<0$ & $y=0$ &\multicolumn2l{}&$0$\\
%      \cline{2-5}
%            & $y>0$ & $-x> y$ & $-$ & $\opMul(-x, y)$\\
%      \cline{3-3}\cline{5-5}
%            &       & else    &     & $\opMul( y,-x)$\\
%      \cline{2-5}
%            & $y<0$ & $-x>-y$ & $+$ & $\opMul(-x,-y)$\\
%      \cline{3-3}\cline{5-5}
%            &       & else    &     & $\opMul(-y,-x)$\\
%      \hline
%    \end{tabular}
%    \end{quote}
%    \begin{macrocode}
\def\BIC@MulSwitch#1#2!#3#4!{%
  \ifcase\BIC@Sgn#1#2! % x = 0
    \BIC@AfterFi{ 0}%
  \or % x > 0
    \ifcase\BIC@Sgn#3#4! % y = 0
      \BIC@AfterFiFi{ 0}%
    \or % y > 0
      \ifnum\BIC@PosCmp#1#2!#3#4!=1 % x > y
        \BIC@AfterFiFiFi{%
          \BIC@ProcessMul0!#1#2!#3#4!%
        }%
      \else % x <= y
        \BIC@AfterFiFiFi{%
          \BIC@ProcessMul0!#3#4!#1#2!%
        }%
      \fi
    \else % y < 0
      \expandafter-\romannumeral0%
      \ifnum\BIC@PosCmp#1#2!#4!=1 % x > -y
        \BIC@AfterFiFiFi{%
          \BIC@ProcessMul0!#1#2!#4!%
        }%
      \else % x <= -y
        \BIC@AfterFiFiFi{%
          \BIC@ProcessMul0!#4!#1#2!%
        }%
      \fi
    \fi
  \else % x < 0
    \ifcase\BIC@Sgn#3#4! % y = 0
      \BIC@AfterFiFi{ 0}%
    \or % y > 0
      \expandafter-\romannumeral0%
      \ifnum\BIC@PosCmp#2!#3#4!=1 % -x > y
        \BIC@AfterFiFiFi{%
          \BIC@ProcessMul0!#2!#3#4!%
        }%
      \else % -x <= y
        \BIC@AfterFiFiFi{%
          \BIC@ProcessMul0!#3#4!#2!%
        }%
      \fi
    \else % y < 0
      \ifnum\BIC@PosCmp#2!#4!=1 % -x > -y
        \BIC@AfterFiFiFi{%
          \BIC@ProcessMul0!#2!#4!%
        }%
      \else % -x <= -y
        \BIC@AfterFiFiFi{%
          \BIC@ProcessMul0!#4!#2!%
        }%
      \fi
    \fi
  \BIC@Fi
}
%    \end{macrocode}
%    \end{macro}
%    \begin{macro}{\BigIntCalcMul}
%    \begin{macrocode}
\def\BigIntCalcMul#1!#2!{%
  \romannumeral0%
  \BIC@ProcessMul0!#1!#2!%
}
%    \end{macrocode}
%    \end{macro}
%    \begin{macro}{\BIC@ProcessMul}
%    |#1|: result\\
%    |#2|: number $x$\\
%    |#3#4|: number $y$
%    \begin{macrocode}
\def\BIC@ProcessMul#1!#2!#3#4!{%
  \ifx\\#4\\%
    \BIC@AfterFi{%
      \expandafter\expandafter\expandafter\BIC@Space
      \bigintcalcAdd{\BIC@Tim#2!#3}{#10}%
    }%
  \else
    \BIC@AfterFi{%
      \expandafter\expandafter\expandafter\BIC@ProcessMul
      \bigintcalcAdd{\BIC@Tim#2!#3}{#10}!#2!#4!%
    }%
  \BIC@Fi
}
%    \end{macrocode}
%    \end{macro}
%
% \subsection{\op{Sqr}}
%
%    \begin{macro}{\bigintcalcSqr}
%    \begin{macrocode}
\def\bigintcalcSqr#1{%
  \romannumeral0%
  \expandafter\expandafter\expandafter\BIC@Sqr
  \bigintcalcNum{#1}!%
}
%    \end{macrocode}
%    \end{macro}
%    \begin{macro}{\BIC@Sqr}
%    \begin{macrocode}
\def\BIC@Sqr#1{%
   \ifx#1-%
     \expandafter\BIC@@Sqr
   \else
     \expandafter\BIC@@Sqr\expandafter#1%
   \fi
}
%    \end{macrocode}
%    \end{macro}
%    \begin{macro}{\BIC@@Sqr}
%    \begin{macrocode}
\def\BIC@@Sqr#1!{%
  \BIC@ProcessMul0!#1!#1!%
}
%    \end{macrocode}
%    \end{macro}
%
% \subsection{\op{Fac}}
%
%    \begin{macro}{\bigintcalcFac}
%    \begin{macrocode}
\def\bigintcalcFac#1{%
  \romannumeral0%
  \expandafter\expandafter\expandafter\BIC@Fac
  \bigintcalcNum{#1}!%
}
%    \end{macrocode}
%    \end{macro}
%    \begin{macro}{\BIC@Fac}
%    \begin{macrocode}
\def\BIC@Fac#1#2!{%
  \ifx#1-%
    \BIC@AfterFi{ 0\BigIntCalcError:FacNegative}%
  \else
    \ifnum\BIC@PosCmp#1#2!13!<0 %
      \ifcase#1#2 %
         \BIC@AfterFiFiFi{ 1}% 0!
      \or\BIC@AfterFiFiFi{ 1}% 1!
      \or\BIC@AfterFiFiFi{ 2}% 2!
      \or\BIC@AfterFiFiFi{ 6}% 3!
      \or\BIC@AfterFiFiFi{ 24}% 4!
      \or\BIC@AfterFiFiFi{ 120}% 5!
      \or\BIC@AfterFiFiFi{ 720}% 6!
      \or\BIC@AfterFiFiFi{ 5040}% 7!
      \or\BIC@AfterFiFiFi{ 40320}% 8!
      \or\BIC@AfterFiFiFi{ 362880}% 9!
      \or\BIC@AfterFiFiFi{ 3628800}% 10!
      \or\BIC@AfterFiFiFi{ 39916800}% 11!
      \or\BIC@AfterFiFiFi{ 479001600}% 12!
?     \else\BigIntCalcError:ThisCannotHappen%
      \fi
    \else
      \BIC@AfterFiFi{%
        \BIC@ProcessFac#1#2!479001600!%
      }%
    \fi
  \BIC@Fi
}
%    \end{macrocode}
%    \end{macro}
%    \begin{macro}{\BIC@ProcessFac}
%    |#1|: $n$\\
%    |#2|: result
%    \begin{macrocode}
\def\BIC@ProcessFac#1!#2!{%
  \ifnum\BIC@PosCmp#1!12!=0 %
    \BIC@AfterFi{ #2}%
  \else
    \BIC@AfterFi{%
      \expandafter\BIC@@ProcessFac
      \romannumeral0\BIC@ProcessMul0!#2!#1!%
      !#1!%
    }%
  \BIC@Fi
}
%    \end{macrocode}
%    \end{macro}
%    \begin{macro}{\BIC@@ProcessFac}
%    |#1|: result\\
%    |#2|: $n$
%    \begin{macrocode}
\def\BIC@@ProcessFac#1!#2!{%
  \expandafter\BIC@ProcessFac
  \romannumeral0\BIC@Dec#2!{}%
  !#1!%
}
%    \end{macrocode}
%    \end{macro}
%
% \subsection{\op{Pow}}
%
%    \begin{macro}{\bigintcalcPow}
%    |#1|: basis\\
%    |#2|: power
%    \begin{macrocode}
\def\bigintcalcPow#1{%
  \romannumeral0%
  \expandafter\expandafter\expandafter\BIC@Pow
  \bigintcalcNum{#1}!%
}
%    \end{macrocode}
%    \end{macro}
%    \begin{macro}{\BIC@Pow}
%    |#1|: basis\\
%    |#2|: power
%    \begin{macrocode}
\def\BIC@Pow#1!#2{%
  \expandafter\expandafter\expandafter\BIC@PowSwitch
  \bigintcalcNum{#2}!#1!%
}
%    \end{macrocode}
%    \end{macro}
%    \begin{macro}{\BIC@PowSwitch}
%    |#1#2|: power $y$\\
%    |#3#4|: basis $x$\\
%    Decision table for \cs{BIC@PowSwitch}.
%    \begin{quote}
%    \def\M#1{\multicolumn{#1}{l}{}}%
%    \begin{tabular}[t]{@{}|l|l|l|l|@{}}
%    \hline
%    $y=0$ & \M{2}                        & $1$\\
%    \hline
%    $y=1$ & \M{2}                        & $x$\\
%    \hline
%    $y=2$ & $x<0$          & \M{1}       & $\opMul(-x,-x)$\\
%    \cline{2-4}
%          & else           & \M{1}       & $\opMul(x,x)$\\
%    \hline
%    $y<0$ & $x=0$          & \M{1}       & DivisionByZero\\
%    \cline{2-4}
%          & $x=1$          & \M{1}       & $1$\\
%    \cline{2-4}
%          & $x=-1$         & $\opODD(y)$ & $-1$\\
%    \cline{3-4}
%          &                & else        & $1$\\
%    \cline{2-4}
%          & else ($\string|x\string|>1$) & \M{1}       & $0$\\
%    \hline
%    $y>2$ & $x=0$          & \M{1}       & $0$\\
%    \cline{2-4}
%          & $x=1$          & \M{1}       & $1$\\
%    \cline{2-4}
%          & $x=-1$         & $\opODD(y)$ & $-1$\\
%    \cline{3-4}
%          &                & else        & $1$\\
%    \cline{2-4}
%          & $x<-1$ ($x<0$) & $\opODD(y)$ & $-\opPow(-x,y)$\\
%    \cline{3-4}
%          &                & else        & $\opPow(-x,y)$\\
%    \cline{2-4}
%          & else ($x>1$)   & \M{1}       & $\opPow(x,y)$\\
%    \hline
%    \end{tabular}
%    \end{quote}
%    \begin{macrocode}
\def\BIC@PowSwitch#1#2!#3#4!{%
  \ifcase\ifx\\#2\\%
           \ifx#100 % y = 0
           \else\ifx#111 % y = 1
           \else\ifx#122 % y = 2
           \else4 % y > 2
           \fi\fi\fi
         \else
           \ifx#1-3 % y < 0
           \else4 % y > 2
           \fi
         \fi
    \BIC@AfterFi{ 1}% y = 0
  \or % y = 1
    \BIC@AfterFi{ #3#4}%
  \or % y = 2
    \ifx#3-% x < 0
      \BIC@AfterFiFi{%
        \BIC@ProcessMul0!#4!#4!%
      }%
    \else % x >= 0
      \BIC@AfterFiFi{%
        \BIC@ProcessMul0!#3#4!#3#4!%
      }%
    \fi
  \or % y < 0
    \ifcase\ifx\\#4\\%
             \ifx#300 % x = 0
             \else\ifx#311 % x = 1
             \else3 % x > 1
             \fi\fi
           \else
             \ifcase\BIC@MinusOne#3#4! %
               3 % |x| > 1
             \or
               2 % x = -1
?            \else\BigIntCalcError:ThisCannotHappen%
             \fi
           \fi
      \BIC@AfterFiFi{ 0\BigIntCalcError:DivisionByZero}% x = 0
    \or % x = 1
      \BIC@AfterFiFi{ 1}% x = 1
    \or % x = -1
      \ifcase\BIC@ModTwo#2! % even(y)
        \BIC@AfterFiFiFi{ 1}%
      \or % odd(y)
        \BIC@AfterFiFiFi{ -1}%
?     \else\BigIntCalcError:ThisCannotHappen%
      \fi
    \or % |x| > 1
      \BIC@AfterFiFi{ 0}%
?   \else\BigIntCalcError:ThisCannotHappen%
    \fi
  \or % y > 2
    \ifcase\ifx\\#4\\%
             \ifx#300 % x = 0
             \else\ifx#311 % x = 1
             \else4 % x > 1
             \fi\fi
           \else
             \ifx#3-%
               \ifcase\BIC@MinusOne#3#4! %
                 3 % x < -1
               \else
                 2 % x = -1
               \fi
             \else
               4 % x > 1
             \fi
           \fi
      \BIC@AfterFiFi{ 0}% x = 0
    \or % x = 1
      \BIC@AfterFiFi{ 1}% x = 1
    \or % x = -1
      \ifcase\BIC@ModTwo#1#2! % even(y)
        \BIC@AfterFiFiFi{ 1}%
      \or % odd(y)
        \BIC@AfterFiFiFi{ -1}%
?     \else\BigIntCalcError:ThisCannotHappen%
      \fi
    \or % x < -1
      \ifcase\BIC@ModTwo#1#2! % even(y)
        \BIC@AfterFiFiFi{%
          \BIC@PowRec#4!#1#2!1!%
        }%
      \or % odd(y)
        \expandafter-\romannumeral0%
        \BIC@AfterFiFiFi{%
          \BIC@PowRec#4!#1#2!1!%
        }%
?     \else\BigIntCalcError:ThisCannotHappen%
      \fi
    \or % x > 1
      \BIC@AfterFiFi{%
        \BIC@PowRec#3#4!#1#2!1!%
      }%
?   \else\BigIntCalcError:ThisCannotHappen%
    \fi
? \else\BigIntCalcError:ThisCannotHappen%
  \BIC@Fi
}
%    \end{macrocode}
%    \end{macro}
%
% \subsubsection{Help macros}
%
%    \begin{macro}{\BIC@ModTwo}
%    Macro \cs{BIC@ModTwo} expects a number without sign
%    and returns digit |1| or |0| if the number is odd or even.
%    \begin{macrocode}
\def\BIC@ModTwo#1#2!{%
  \ifx\\#2\\%
    \ifodd#1 %
      \BIC@AfterFiFi1%
    \else
      \BIC@AfterFiFi0%
    \fi
  \else
    \BIC@AfterFi{%
      \BIC@ModTwo#2!%
    }%
  \BIC@Fi
}
%    \end{macrocode}
%    \end{macro}
%
%    \begin{macro}{\BIC@MinusOne}
%    Macro \cs{BIC@MinusOne} expects a number and returns
%    digit |1| if the number equals minus one and returns |0| otherwise.
%    \begin{macrocode}
\def\BIC@MinusOne#1#2!{%
  \ifx#1-%
    \BIC@@MinusOne#2!%
  \else
    0%
  \fi
}
%    \end{macrocode}
%    \end{macro}
%    \begin{macro}{\BIC@@MinusOne}
%    \begin{macrocode}
\def\BIC@@MinusOne#1#2!{%
  \ifx#11%
    \ifx\\#2\\%
      1%
    \else
      0%
    \fi
  \else
    0%
  \fi
}
%    \end{macrocode}
%    \end{macro}
%
% \subsubsection{Recursive calculation}
%
%    \begin{macro}{\BIC@PowRec}
%\begin{quote}
%\begin{verbatim}
%Pow(x, y) {
%  PowRec(x, y, 1)
%}
%PowRec(x, y, r) {
%  if y == 1 then
%    return r
%  else
%    ifodd y then
%      return PowRec(x*x, y div 2, r*x) % y div 2 = (y-1)/2
%    else
%      return PowRec(x*x, y div 2, r)
%    fi
%  fi
%}
%\end{verbatim}
%\end{quote}
%    |#1|: $x$ (basis)\\
%    |#2#3|: $y$ (power)\\
%    |#4|: $r$ (result)
%    \begin{macrocode}
\def\BIC@PowRec#1!#2#3!#4!{%
  \ifcase\ifx#21\ifx\\#3\\0 \else1 \fi\else1 \fi % y = 1
    \ifnum\BIC@PosCmp#1!#4!=1 % x > r
      \BIC@AfterFiFi{%
        \BIC@ProcessMul0!#1!#4!%
      }%
    \else
      \BIC@AfterFiFi{%
        \BIC@ProcessMul0!#4!#1!%
      }%
    \fi
  \or
    \ifcase\BIC@ModTwo#2#3! % even(y)
      \BIC@AfterFiFi{%
        \expandafter\BIC@@PowRec\romannumeral0%
        \BIC@@Shr#2#3!%
        !#1!#4!%
      }%
    \or % odd(y)
      \ifnum\BIC@PosCmp#1!#4!=1 % x > r
        \BIC@AfterFiFiFi{%
          \expandafter\BIC@@@PowRec\romannumeral0%
          \BIC@ProcessMul0!#1!#4!%
          !#1!#2#3!%
        }%
      \else
        \BIC@AfterFiFiFi{%
          \expandafter\BIC@@@PowRec\romannumeral0%
          \BIC@ProcessMul0!#1!#4!%
          !#1!#2#3!%
        }%
      \fi
?   \else\BigIntCalcError:ThisCannotHappen%
    \fi
? \else\BigIntCalcError:ThisCannotHappen%
  \BIC@Fi
}
%    \end{macrocode}
%    \end{macro}
%    \begin{macro}{\BIC@@PowRec}
%    |#1|: $y/2$\\
%    |#2|: $x$\\
%    |#3|: new $r$ ($r$ or $r*x$)
%    \begin{macrocode}
\def\BIC@@PowRec#1!#2!#3!{%
  \expandafter\BIC@PowRec\romannumeral0%
  \BIC@ProcessMul0!#2!#2!%
  !#1!#3!%
}
%    \end{macrocode}
%    \end{macro}
%    \begin{macro}{\BIC@@@PowRec}
%    |#1|: $r*x$
%    |#2|: $x$
%    |#3|: $y$
%    \begin{macrocode}
\def\BIC@@@PowRec#1!#2!#3!{%
  \expandafter\BIC@@PowRec\romannumeral0%
  \BIC@@Shr#3!%
  !#2!#1!%
}
%    \end{macrocode}
%    \end{macro}
%
% \subsection{\op{Div}}
%
%    \begin{macro}{\bigintcalcDiv}
%    |#1|: $x$\\
%    |#2|: $y$ (divisor)
%    \begin{macrocode}
\def\bigintcalcDiv#1{%
  \romannumeral0%
  \expandafter\expandafter\expandafter\BIC@Div
  \bigintcalcNum{#1}!%
}
%    \end{macrocode}
%    \end{macro}
%    \begin{macro}{\BIC@Div}
%    |#1|: $x$\\
%    |#2|: $y$
%    \begin{macrocode}
\def\BIC@Div#1!#2{%
  \expandafter\expandafter\expandafter\BIC@DivSwitchSign
  \bigintcalcNum{#2}!#1!%
}
%    \end{macrocode}
%    \end{macro}
%    \begin{macro}{\BigIntCalcDiv}
%    \begin{macrocode}
\def\BigIntCalcDiv#1!#2!{%
  \romannumeral0%
  \BIC@DivSwitchSign#2!#1!%
}
%    \end{macrocode}
%    \end{macro}
%    \begin{macro}{\BIC@DivSwitchSign}
%    Decision table for \cs{BIC@DivSwitchSign}.
%    \begin{quote}
%    \begin{tabular}{@{}|l|l|l|@{}}
%    \hline
%    $y=0$ & \multicolumn{1}{l}{} & DivisionByZero\\
%    \hline
%    $y>0$ & $x=0$ & $0$\\
%    \cline{2-3}
%          & $x>0$ & DivSwitch$(+,x,y)$\\
%    \cline{2-3}
%          & $x<0$ & DivSwitch$(-,-x,y)$\\
%    \hline
%    $y<0$ & $x=0$ & $0$\\
%    \cline{2-3}
%          & $x>0$ & DivSwitch$(-,x,-y)$\\
%    \cline{2-3}
%          & $x<0$ & DivSwitch$(+,-x,-y)$\\
%    \hline
%    \end{tabular}
%    \end{quote}
%    |#1|: $y$ (divisor)\\
%    |#2|: $x$
%    \begin{macrocode}
\def\BIC@DivSwitchSign#1#2!#3#4!{%
  \ifcase\BIC@Sgn#1#2! % y = 0
    \BIC@AfterFi{ 0\BigIntCalcError:DivisionByZero}%
  \or % y > 0
    \ifcase\BIC@Sgn#3#4! % x = 0
      \BIC@AfterFiFi{ 0}%
    \or % x > 0
      \BIC@AfterFiFi{%
        \BIC@DivSwitch{}#3#4!#1#2!%
      }%
    \else % x < 0
      \BIC@AfterFiFi{%
        \BIC@DivSwitch-#4!#1#2!%
      }%
    \fi
  \else % y < 0
    \ifcase\BIC@Sgn#3#4! % x = 0
      \BIC@AfterFiFi{ 0}%
    \or % x > 0
      \BIC@AfterFiFi{%
        \BIC@DivSwitch-#3#4!#2!%
      }%
    \else % x < 0
      \BIC@AfterFiFi{%
        \BIC@DivSwitch{}#4!#2!%
      }%
    \fi
  \BIC@Fi
}
%    \end{macrocode}
%    \end{macro}
%    \begin{macro}{\BIC@DivSwitch}
%    Decision table for \cs{BIC@DivSwitch}.
%    \begin{quote}
%    \begin{tabular}{@{}|l|l|l|@{}}
%    \hline
%    $y=x$ & \multicolumn{1}{l}{} & sign $1$\\
%    \hline
%    $y>x$ & \multicolumn{1}{l}{} & $0$\\
%    \hline
%    $y<x$ & $y=1$                & sign $x$\\
%    \cline{2-3}
%          & $y=2$                & sign Shr$(x)$\\
%    \cline{2-3}
%          & $y=4$                & sign Shr(Shr$(x)$)\\
%    \cline{2-3}
%          & else                 & sign ProcessDiv$(x,y)$\\
%    \hline
%    \end{tabular}
%    \end{quote}
%    |#1|: sign\\
%    |#2|: $x$\\
%    |#3#4|: $y$ ($y\ne 0$)
%    \begin{macrocode}
\def\BIC@DivSwitch#1#2!#3#4!{%
  \ifcase\BIC@PosCmp#3#4!#2!% y = x
    \BIC@AfterFi{ #11}%
  \or % y > x
    \BIC@AfterFi{ 0}%
  \else % y < x
    \ifx\\#1\\%
    \else
      \expandafter-\romannumeral0%
    \fi
    \ifcase\ifx\\#4\\%
             \ifx#310 % y = 1
             \else\ifx#321 % y = 2
             \else\ifx#342 % y = 4
             \else3 % y > 2
             \fi\fi\fi
           \else
             3 % y > 2
           \fi
      \BIC@AfterFiFi{ #2}% y = 1
    \or % y = 2
      \BIC@AfterFiFi{%
        \BIC@@Shr#2!%
      }%
    \or % y = 4
      \BIC@AfterFiFi{%
        \expandafter\BIC@@Shr\romannumeral0%
          \BIC@@Shr#2!!%
      }%
    \or % y > 2
      \BIC@AfterFiFi{%
        \BIC@DivStartX#2!#3#4!!!%
      }%
?   \else\BigIntCalcError:ThisCannotHappen%
    \fi
  \BIC@Fi
}
%    \end{macrocode}
%    \end{macro}
%    \begin{macro}{\BIC@ProcessDiv}
%    |#1#2|: $x$\\
%    |#3#4|: $y$\\
%    |#5|: collect first digits of $x$\\
%    |#6|: corresponding digits of $y$
%    \begin{macrocode}
\def\BIC@DivStartX#1#2!#3#4!#5!#6!{%
  \ifx\\#4\\%
    \BIC@AfterFi{%
      \BIC@DivStartYii#6#3#4!{#5#1}#2=!%
    }%
  \else
    \BIC@AfterFi{%
      \BIC@DivStartX#2!#4!#5#1!#6#3!%
    }%
  \BIC@Fi
}
%    \end{macrocode}
%    \end{macro}
%    \begin{macro}{\BIC@DivStartYii}
%    |#1|: $y$\\
%    |#2|: $x$, |=|
%    \begin{macrocode}
\def\BIC@DivStartYii#1!{%
  \expandafter\BIC@DivStartYiv\romannumeral0%
  \BIC@Shl#1!%
  !#1!%
}
%    \end{macrocode}
%    \end{macro}
%    \begin{macro}{\BIC@DivStartYiv}
%    |#1|: $2y$\\
%    |#2|: $y$\\
%    |#3|: $x$, |=|
%    \begin{macrocode}
\def\BIC@DivStartYiv#1!{%
  \expandafter\BIC@DivStartYvi\romannumeral0%
  \BIC@Shl#1!%
  !#1!%
}
%    \end{macrocode}
%    \end{macro}
%    \begin{macro}{\BIC@DivStartYvi}
%    |#1|: $4y$\\
%    |#2|: $2y$\\
%    |#3|: $y$\\
%    |#4|: $x$, |=|
%    \begin{macrocode}
\def\BIC@DivStartYvi#1!#2!{%
  \expandafter\BIC@DivStartYviii\romannumeral0%
  \BIC@AddXY#1!#2!!!%
  !#1!#2!%
}
%    \end{macrocode}
%    \end{macro}
%    \begin{macro}{\BIC@DivStartYviii}
%    |#1|: $6y$\\
%    |#2|: $4y$\\
%    |#3|: $2y$\\
%    |#4|: $y$\\
%    |#5|: $x$, |=|
%    \begin{macrocode}
\def\BIC@DivStartYviii#1!#2!{%
  \expandafter\BIC@DivStart\romannumeral0%
  \BIC@Shl#2!%
  !#1!#2!%
}
%    \end{macrocode}
%    \end{macro}
%    \begin{macro}{\BIC@DivStart}
%    |#1|: $8y$\\
%    |#2|: $6y$\\
%    |#3|: $4y$\\
%    |#4|: $2y$\\
%    |#5|: $y$\\
%    |#6|: $x$, |=|
%    \begin{macrocode}
\def\BIC@DivStart#1!#2!#3!#4!#5!#6!{%
  \BIC@ProcessDiv#6!!#5!#4!#3!#2!#1!=%
}
%    \end{macrocode}
%    \end{macro}
%    \begin{macro}{\BIC@ProcessDiv}
%    |#1#2#3|: $x$, |=|\\
%    |#4|: result\\
%    |#5|: $y$\\
%    |#6|: $2y$\\
%    |#7|: $4y$\\
%    |#8|: $6y$\\
%    |#9|: $8y$
%    \begin{macrocode}
\def\BIC@ProcessDiv#1#2#3!#4!#5!{%
  \ifcase\BIC@PosCmp#5!#1!% y = #1
    \ifx#2=%
      \BIC@AfterFiFi{\BIC@DivCleanup{#41}}%
    \else
      \BIC@AfterFiFi{%
        \BIC@ProcessDiv#2#3!#41!#5!%
      }%
    \fi
  \or % y > #1
    \ifx#2=%
      \BIC@AfterFiFi{\BIC@DivCleanup{#40}}%
    \else
      \ifx\\#4\\%
        \BIC@AfterFiFiFi{%
          \BIC@ProcessDiv{#1#2}#3!!#5!%
        }%
      \else
        \BIC@AfterFiFiFi{%
          \BIC@ProcessDiv{#1#2}#3!#40!#5!%
        }%
      \fi
    \fi
  \else % y < #1
    \BIC@AfterFi{%
      \BIC@@ProcessDiv{#1}#2#3!#4!#5!%
    }%
  \BIC@Fi
}
%    \end{macrocode}
%    \end{macro}
%    \begin{macro}{\BIC@DivCleanup}
%    |#1|: result\\
%    |#2|: garbage
%    \begin{macrocode}
\def\BIC@DivCleanup#1#2={ #1}%
%    \end{macrocode}
%    \end{macro}
%    \begin{macro}{\BIC@@ProcessDiv}
%    \begin{macrocode}
\def\BIC@@ProcessDiv#1#2#3!#4!#5!#6!#7!{%
  \ifcase\BIC@PosCmp#7!#1!% 4y = #1
    \ifx#2=%
      \BIC@AfterFiFi{\BIC@DivCleanup{#44}}%
    \else
     \BIC@AfterFiFi{%
       \BIC@ProcessDiv#2#3!#44!#5!#6!#7!%
     }%
    \fi
  \or % 4y > #1
    \ifcase\BIC@PosCmp#6!#1!% 2y = #1
      \ifx#2=%
        \BIC@AfterFiFiFi{\BIC@DivCleanup{#42}}%
      \else
        \BIC@AfterFiFiFi{%
          \BIC@ProcessDiv#2#3!#42!#5!#6!#7!%
        }%
      \fi
    \or % 2y > #1
      \ifx#2=%
        \BIC@AfterFiFiFi{\BIC@DivCleanup{#41}}%
      \else
        \BIC@AfterFiFiFi{%
          \BIC@DivSub#1!#5!#2#3!#41!#5!#6!#7!%
        }%
      \fi
    \else % 2y < #1
      \BIC@AfterFiFi{%
        \expandafter\BIC@ProcessDivII\romannumeral0%
        \BIC@SubXY#1!#6!!!%
        !#2#3!#4!#5!23%
        #6!#7!%
      }%
    \fi
  \else % 4y < #1
    \BIC@AfterFi{%
      \BIC@@@ProcessDiv{#1}#2#3!#4!#5!#6!#7!%
    }%
  \BIC@Fi
}
%    \end{macrocode}
%    \end{macro}
%    \begin{macro}{\BIC@DivSub}
%    Next token group: |#1|-|#2| and next digit |#3|.
%    \begin{macrocode}
\def\BIC@DivSub#1!#2!#3{%
  \expandafter\BIC@ProcessDiv\expandafter{%
    \romannumeral0%
    \BIC@SubXY#1!#2!!!%
    #3%
  }%
}
%    \end{macrocode}
%    \end{macro}
%    \begin{macro}{\BIC@ProcessDivII}
%    |#1|: $x'-2y$\\
%    |#2#3|: remaining $x$, |=|\\
%    |#4|: result\\
%    |#5|: $y$\\
%    |#6|: first possible result digit\\
%    |#7|: second possible result digit
%    \begin{macrocode}
\def\BIC@ProcessDivII#1!#2#3!#4!#5!#6#7{%
  \ifcase\BIC@PosCmp#5!#1!% y = #1
    \ifx#2=%
      \BIC@AfterFiFi{\BIC@DivCleanup{#4#7}}%
    \else
      \BIC@AfterFiFi{%
        \BIC@ProcessDiv#2#3!#4#7!#5!%
      }%
    \fi
  \or % y > #1
    \ifx#2=%
      \BIC@AfterFiFi{\BIC@DivCleanup{#4#6}}%
    \else
      \BIC@AfterFiFi{%
        \BIC@ProcessDiv{#1#2}#3!#4#6!#5!%
      }%
    \fi
  \else % y < #1
    \ifx#2=%
      \BIC@AfterFiFi{\BIC@DivCleanup{#4#7}}%
    \else
      \BIC@AfterFiFi{%
        \BIC@DivSub#1!#5!#2#3!#4#7!#5!%
      }%
    \fi
  \BIC@Fi
}
%    \end{macrocode}
%    \end{macro}
%    \begin{macro}{\BIC@ProcessDivIV}
%    |#1#2#3|: $x$, |=|, $x>4y$\\
%    |#4|: result\\
%    |#5|: $y$\\
%    |#6|: $2y$\\
%    |#7|: $4y$\\
%    |#8|: $6y$\\
%    |#9|: $8y$
%    \begin{macrocode}
\def\BIC@@@ProcessDiv#1#2#3!#4!#5!#6!#7!#8!#9!{%
  \ifcase\BIC@PosCmp#8!#1!% 6y = #1
    \ifx#2=%
      \BIC@AfterFiFi{\BIC@DivCleanup{#46}}%
    \else
      \BIC@AfterFiFi{%
        \BIC@ProcessDiv#2#3!#46!#5!#6!#7!#8!#9!%
      }%
    \fi
  \or % 6y > #1
    \BIC@AfterFi{%
      \expandafter\BIC@ProcessDivII\romannumeral0%
      \BIC@SubXY#1!#7!!!%
      !#2#3!#4!#5!45%
      #6!#7!#8!#9!%
    }%
  \else % 6y < #1
    \ifcase\BIC@PosCmp#9!#1!% 8y = #1
      \ifx#2=%
        \BIC@AfterFiFiFi{\BIC@DivCleanup{#48}}%
      \else
        \BIC@AfterFiFiFi{%
          \BIC@ProcessDiv#2#3!#48!#5!#6!#7!#8!#9!%
        }%
      \fi
    \or % 8y > #1
      \BIC@AfterFiFi{%
        \expandafter\BIC@ProcessDivII\romannumeral0%
        \BIC@SubXY#1!#8!!!%
        !#2#3!#4!#5!67%
        #6!#7!#8!#9!%
      }%
    \else % 8y < #1
      \BIC@AfterFiFi{%
        \expandafter\BIC@ProcessDivII\romannumeral0%
        \BIC@SubXY#1!#9!!!%
        !#2#3!#4!#5!89%
        #6!#7!#8!#9!%
      }%
    \fi
  \BIC@Fi
}
%    \end{macrocode}
%    \end{macro}
%
% \subsection{\op{Mod}}
%
%    \begin{macro}{\bigintcalcMod}
%    |#1|: $x$\\
%    |#2|: $y$
%    \begin{macrocode}
\def\bigintcalcMod#1{%
  \romannumeral0%
  \expandafter\expandafter\expandafter\BIC@Mod
  \bigintcalcNum{#1}!%
}
%    \end{macrocode}
%    \end{macro}
%    \begin{macro}{\BIC@Mod}
%    |#1|: $x$\\
%    |#2|: $y$
%    \begin{macrocode}
\def\BIC@Mod#1!#2{%
  \expandafter\expandafter\expandafter\BIC@ModSwitchSign
  \bigintcalcNum{#2}!#1!%
}
%    \end{macrocode}
%    \end{macro}
%    \begin{macro}{\BigIntCalcMod}
%    \begin{macrocode}
\def\BigIntCalcMod#1!#2!{%
  \romannumeral0%
  \BIC@ModSwitchSign#2!#1!%
}
%    \end{macrocode}
%    \end{macro}
%    \begin{macro}{\BIC@ModSwitchSign}
%    Decision table for \cs{BIC@ModSwitchSign}.
%    \begin{quote}
%    \begin{tabular}{@{}|l|l|l|@{}}
%    \hline
%    $y=0$ & \multicolumn{1}{l}{} & DivisionByZero\\
%    \hline
%    $y>0$ & $x=0$ & $0$\\
%    \cline{2-3}
%          & else  & ModSwitch$(+,x,y)$\\
%    \hline
%    $y<0$ & \multicolumn{1}{l}{} & ModSwitch$(-,-x,-y)$\\
%    \hline
%    \end{tabular}
%    \end{quote}
%    |#1#2|: $y$\\
%    |#3#4|: $x$
%    \begin{macrocode}
\def\BIC@ModSwitchSign#1#2!#3#4!{%
  \ifcase\ifx\\#2\\%
           \ifx#100 % y = 0
           \else1 % y > 0
           \fi
          \else
            \ifx#1-2 % y < 0
            \else1 % y > 0
            \fi
          \fi
    \BIC@AfterFi{ 0\BigIntCalcError:DivisionByZero}%
  \or % y > 0
    \ifcase\ifx\\#4\\\ifx#300 \else1 \fi\else1 \fi % x = 0
      \BIC@AfterFiFi{ 0}%
    \else
      \BIC@AfterFiFi{%
        \BIC@ModSwitch{}#3#4!#1#2!%
      }%
    \fi
  \else % y < 0
    \ifcase\ifx\\#4\\%
             \ifx#300 % x = 0
             \else1 % x > 0
             \fi
           \else
             \ifx#3-2 % x < 0
             \else1 % x > 0
             \fi
           \fi
      \BIC@AfterFiFi{ 0}%
    \or % x > 0
      \BIC@AfterFiFi{%
        \BIC@ModSwitch--#3#4!#2!%
      }%
    \else % x < 0
      \BIC@AfterFiFi{%
        \BIC@ModSwitch-#4!#2!%
      }%
    \fi
  \BIC@Fi
}
%    \end{macrocode}
%    \end{macro}
%    \begin{macro}{\BIC@ModSwitch}
%    Decision table for \cs{BIC@ModSwitch}.
%    \begin{quote}
%    \begin{tabular}{@{}|l|l|l|@{}}
%    \hline
%    $y=1$ & \multicolumn{1}{l}{} & $0$\\
%    \hline
%    $y=2$ & ifodd$(x)$ & sign $1$\\
%    \cline{2-3}
%          & else       & $0$\\
%    \hline
%    $y>2$ & $x<0$      &
%        $z\leftarrow x-(x/y)*y;\quad (z<0)\mathbin{?}z+y\mathbin{:}z$\\
%    \cline{2-3}
%          & $x>0$      & $x - (x/y) * y$\\
%    \hline
%    \end{tabular}
%    \end{quote}
%    |#1|: sign\\
%    |#2#3|: $x$\\
%    |#4#5|: $y$
%    \begin{macrocode}
\def\BIC@ModSwitch#1#2#3!#4#5!{%
  \ifcase\ifx\\#5\\%
           \ifx#410 % y = 1
           \else\ifx#421 % y = 2
           \else2 % y > 2
           \fi\fi
         \else2 % y > 2
         \fi
    \BIC@AfterFi{ 0}% y = 1
  \or % y = 2
    \ifcase\BIC@ModTwo#2#3! % even(x)
      \BIC@AfterFiFi{ 0}%
    \or % odd(x)
      \BIC@AfterFiFi{ #11}%
?   \else\BigIntCalcError:ThisCannotHappen%
    \fi
  \or % y > 2
    \ifx\\#1\\%
    \else
      \expandafter\BIC@Space\romannumeral0%
      \expandafter\BIC@ModMinus\romannumeral0%
    \fi
    \ifx#2-% x < 0
      \BIC@AfterFiFi{%
        \expandafter\expandafter\expandafter\BIC@ModX
        \bigintcalcSub{#2#3}{%
          \bigintcalcMul{#4#5}{\bigintcalcDiv{#2#3}{#4#5}}%
        }!#4#5!%
      }%
    \else % x > 0
      \BIC@AfterFiFi{%
        \expandafter\expandafter\expandafter\BIC@Space
        \bigintcalcSub{#2#3}{%
          \bigintcalcMul{#4#5}{\bigintcalcDiv{#2#3}{#4#5}}%
        }%
      }%
    \fi
? \else\BigIntCalcError:ThisCannotHappen%
  \BIC@Fi
}
%    \end{macrocode}
%    \end{macro}
%    \begin{macro}{\BIC@ModMinus}
%    \begin{macrocode}
\def\BIC@ModMinus#1{%
  \ifx#10%
    \BIC@AfterFi{ 0}%
  \else
    \BIC@AfterFi{ -#1}%
  \BIC@Fi
}
%    \end{macrocode}
%    \end{macro}
%    \begin{macro}{\BIC@ModX}
%    |#1#2|: $z$\\
%    |#3|: $x$
%    \begin{macrocode}
\def\BIC@ModX#1#2!#3!{%
  \ifx#1-% z < 0
    \BIC@AfterFi{%
      \expandafter\BIC@Space\romannumeral0%
      \BIC@SubXY#3!#2!!!%
    }%
  \else % z >= 0
    \BIC@AfterFi{ #1#2}%
  \BIC@Fi
}
%    \end{macrocode}
%    \end{macro}
%
%    \begin{macrocode}
\BIC@AtEnd%
%    \end{macrocode}
%
%    \begin{macrocode}
%</package>
%    \end{macrocode}
%
% \section{Test}
%
% \subsection{Catcode checks for loading}
%
%    \begin{macrocode}
%<*test1>
%    \end{macrocode}
%    \begin{macrocode}
\catcode`\{=1 %
\catcode`\}=2 %
\catcode`\#=6 %
\catcode`\@=11 %
\expandafter\ifx\csname count@\endcsname\relax
  \countdef\count@=255 %
\fi
\expandafter\ifx\csname @gobble\endcsname\relax
  \long\def\@gobble#1{}%
\fi
\expandafter\ifx\csname @firstofone\endcsname\relax
  \long\def\@firstofone#1{#1}%
\fi
\expandafter\ifx\csname loop\endcsname\relax
  \expandafter\@firstofone
\else
  \expandafter\@gobble
\fi
{%
  \def\loop#1\repeat{%
    \def\body{#1}%
    \iterate
  }%
  \def\iterate{%
    \body
      \let\next\iterate
    \else
      \let\next\relax
    \fi
    \next
  }%
  \let\repeat=\fi
}%
\def\RestoreCatcodes{}
\count@=0 %
\loop
  \edef\RestoreCatcodes{%
    \RestoreCatcodes
    \catcode\the\count@=\the\catcode\count@\relax
  }%
\ifnum\count@<255 %
  \advance\count@ 1 %
\repeat

\def\RangeCatcodeInvalid#1#2{%
  \count@=#1\relax
  \loop
    \catcode\count@=15 %
  \ifnum\count@<#2\relax
    \advance\count@ 1 %
  \repeat
}
\def\RangeCatcodeCheck#1#2#3{%
  \count@=#1\relax
  \loop
    \ifnum#3=\catcode\count@
    \else
      \errmessage{%
        Character \the\count@\space
        with wrong catcode \the\catcode\count@\space
        instead of \number#3%
      }%
    \fi
  \ifnum\count@<#2\relax
    \advance\count@ 1 %
  \repeat
}
\def\space{ }
\expandafter\ifx\csname LoadCommand\endcsname\relax
  \def\LoadCommand{\input bigintcalc.sty\relax}%
\fi
\def\Test{%
  \RangeCatcodeInvalid{0}{47}%
  \RangeCatcodeInvalid{58}{64}%
  \RangeCatcodeInvalid{91}{96}%
  \RangeCatcodeInvalid{123}{255}%
  \catcode`\@=12 %
  \catcode`\\=0 %
  \catcode`\%=14 %
  \LoadCommand
  \RangeCatcodeCheck{0}{36}{15}%
  \RangeCatcodeCheck{37}{37}{14}%
  \RangeCatcodeCheck{38}{47}{15}%
  \RangeCatcodeCheck{48}{57}{12}%
  \RangeCatcodeCheck{58}{63}{15}%
  \RangeCatcodeCheck{64}{64}{12}%
  \RangeCatcodeCheck{65}{90}{11}%
  \RangeCatcodeCheck{91}{91}{15}%
  \RangeCatcodeCheck{92}{92}{0}%
  \RangeCatcodeCheck{93}{96}{15}%
  \RangeCatcodeCheck{97}{122}{11}%
  \RangeCatcodeCheck{123}{255}{15}%
  \RestoreCatcodes
}
\Test
\csname @@end\endcsname
\end
%    \end{macrocode}
%    \begin{macrocode}
%</test1>
%    \end{macrocode}
%
% \subsection{Macro tests}
%
% \subsubsection{Preamble with test macro definitions}
%
%    \begin{macrocode}
%<*test2>
\NeedsTeXFormat{LaTeX2e}
\nofiles
\documentclass{article}
%<noetex>\let\SavedNumexpr\numexpr
%<noetex>\let\numexpr\UNDEFINED
\makeatletter
\chardef\BIC@TestMode=1 %
\makeatother
\usepackage{bigintcalc}[2016/05/16]
%<noetex>\let\numexpr\SavedNumexpr
\usepackage{qstest}
\IncludeTests{*}
\LogTests{log}{*}{*}
\newcommand*{\TestSpaceAtEnd}[1]{%
%<noetex>  \let\SavedNumexpr\numexpr
%<noetex>  \let\numexpr\UNDEFINED
  \edef\resultA{#1}%
  \edef\resultB{#1 }%
%<noetex>  \let\numexpr\SavedNumexpr
  \Expect*{\resultA\space}*{\resultB}%
}
\newcommand*{\TestResult}[2]{%
%<noetex>  \let\SavedNumexpr\numexpr
%<noetex>  \let\numexpr\UNDEFINED
  \edef\result{#1}%
%<noetex>  \let\numexpr\SavedNumexpr
  \Expect*{\result}{#2}%
}
\newcommand*{\TestResultTwoExpansions}[2]{%
%<*noetex>
  \begingroup
    \let\numexpr\UNDEFINED
    \expandafter\expandafter\expandafter
  \endgroup
%</noetex>
  \expandafter\expandafter\expandafter\Expect
  \expandafter\expandafter\expandafter{#1}{#2}%
}
\newcount\TestCount
%<etex>\newcommand*{\TestArg}[1]{\numexpr#1\relax}
%<noetex>\newcommand*{\TestArg}[1]{#1}
\newcommand*{\TestTeXDivide}[2]{%
  \TestCount=\TestArg{#1}\relax
  \divide\TestCount by \TestArg{#2}\relax
  \Expect*{\bigintcalcDiv{#1}{#2}}*{\the\TestCount}%
}
\newcommand*{\Test}[2]{%
  \TestResult{#1}{#2}%
  \TestResultTwoExpansions{#1}{#2}%
  \TestSpaceAtEnd{#1}%
}
\newcommand*{\TestExch}[2]{\Test{#2}{#1}}
\newcommand*{\TestInv}[2]{%
  \Test{\bigintcalcInv{#1}}{#2}%
}
\newcommand*{\TestAbs}[2]{%
  \Test{\bigintcalcAbs{#1}}{#2}%
}
\newcommand*{\TestSgn}[2]{%
  \Test{\bigintcalcSgn{#1}}{#2}%
}
\newcommand*{\TestMin}[3]{%
  \Test{\bigintcalcMin{#1}{#2}}{#3}%
}
\newcommand*{\TestMax}[3]{%
  \Test{\bigintcalcMax{#1}{#2}}{#3}%
}
\newcommand*{\TestCmp}[3]{%
  \Test{\bigintcalcCmp{#1}{#2}}{#3}%
}
\newcommand*{\TestOdd}[2]{%
  \Test{\bigintcalcOdd{#1}}{#2}%
  \edef\x{%
    \noexpand\Test{%
      \noexpand\BigIntCalcOdd
      \bigintcalcAbs{#1}!%
    }{#2}%
  }%
  \x
}
\newcommand*{\TestInc}[2]{%
  \Test{\bigintcalcInc{#1}}{#2}%
  \ifnum\bigintcalcSgn{#1}>-1 %
    \edef\x{%
      \noexpand\Test{%
        \noexpand\BigIntCalcInc\bigintcalcNum{#1}!%
      }{#2}%
    }%
    \x
  \fi
}
\newcommand*{\TestDec}[2]{%
  \Test{\bigintcalcDec{#1}}{#2}%
  \ifnum\bigintcalcSgn{#1}>0 %
    \edef\x{%
      \noexpand\Test{%
        \noexpand\BigIntCalcDec\bigintcalcNum{#1}!%
      }{#2}%
    }%
    \x
  \fi
}
\newcommand*{\TestAdd}[3]{%
  \Test{\bigintcalcAdd{#1}{#2}}{#3}%
  \ifnum\bigintcalcSgn{#1}>0 %
    \ifnum\bigintcalcSgn{#2}> 0 %
      \ifnum\bigintcalcCmp{#1}{#2}>0 %
        \edef\x{%
          \noexpand\Test{%
            \noexpand\BigIntCalcAdd
            \bigintcalcNum{#1}!\bigintcalcNum{#2}!%
          }{#3}%
        }%
        \x
      \else
        \edef\x{%
          \noexpand\Test{%
            \noexpand\BigIntCalcAdd
            \bigintcalcNum{#2}!\bigintcalcNum{#1}!%
          }{#3}%
        }%
        \x
      \fi
    \fi
  \fi
}
\newcommand*{\TestSub}[3]{%
  \Test{\bigintcalcSub{#1}{#2}}{#3}%
  \ifnum\bigintcalcSgn{#1}>0 %
    \ifnum\bigintcalcSgn{#2}> 0 %
      \ifnum\bigintcalcCmp{#1}{#2}>0 %
        \edef\x{%
          \noexpand\Test{%
            \noexpand\BigIntCalcSub
            \bigintcalcNum{#1}!\bigintcalcNum{#2}!%
          }{#3}%
        }%
        \x
      \fi
    \fi
  \fi
}
\newcommand*{\TestShl}[2]{%
  \Test{\bigintcalcShl{#1}}{#2}%
  \edef\x{%
    \noexpand\Test{%
      \noexpand\BigIntCalcShl\bigintcalcAbs{#1}!%
    }{\bigintcalcAbs{#2}}%
  }%
  \x
}
\newcommand*{\TestShr}[2]{%
  \Test{\bigintcalcShr{#1}}{#2}%
  \edef\x{%
    \noexpand\Test{%
      \noexpand\BigIntCalcShr\bigintcalcAbs{#1}!%
    }{\bigintcalcAbs{#2}}%
  }%
  \x
}
\newcommand*{\TestMul}[3]{%
  \Test{\bigintcalcMul{#1}{#2}}{#3}%
  \edef\x{%
    \noexpand\Test{%
      \noexpand\BigIntCalcMul
      \bigintcalcAbs{#1}!\bigintcalcAbs{#2}!%
    }{\bigintcalcAbs{#3}}%
  }%
  \x
}
\newcommand*{\TestSqr}[2]{%
  \Test{\bigintcalcSqr{#1}}{#2}%
}
\newcommand*{\TestFac}[2]{%
  \expandafter\TestExch\expandafter{%
    \the\numexpr#2%
  }{\bigintcalcFac{#1}}%
}
\newcommand*{\TestFacBig}[2]{%
  \Test{\bigintcalcFac{#1}}{#2}%
}
\newcommand*{\TestPow}[3]{%
  \Test{\bigintcalcPow{#1}{#2}}{#3}%
}
\newcommand*{\TestDiv}[3]{%
  \Test{\bigintcalcDiv{#1}{#2}}{#3}%
  \TestTeXDivide{#1}{#2}%
}
\newcommand*{\TestDivBig}[3]{%
  \Test{\bigintcalcDiv{#1}{#2}}{#3}%
  \edef\x{%
    \noexpand\Test{%
      \noexpand\BigIntCalcDiv\bigintcalcAbs{#1}!\bigintcalcAbs{#2}!%
    }{\bigintcalcAbs{#3}}%
  }%
}
\newcommand*{\TestMod}[3]{%
  \Test{\bigintcalcMod{#1}{#2}}{#3}%
  \ifcase\ifcase\bigintcalcSgn{#1} 0%
         \or
           \ifcase\bigintcalcSgn{#2} 1%
           \or 0%
           \else 1%
           \fi
         \else
           \ifcase\bigintcalcSgn{#2} 1%
           \or 1%
           \else 0%
           \fi
         \fi\relax
    \edef\x{%
      \noexpand\Test{%
        \noexpand\BigIntCalcMod
        \bigintcalcAbs{#1}!\bigintcalcAbs{#2}!%
      }{\bigintcalcAbs{#3}}%
    }%
    \x
  \fi
}
%    \end{macrocode}
%
% \subsubsection{Time}
%
%    \begin{macrocode}
\begingroup\expandafter\expandafter\expandafter\endgroup
\expandafter\ifx\csname pdfresettimer\endcsname\relax
\else
  \makeatletter
  \newcount\SummaryTime
  \newcount\TestTime
  \SummaryTime=\z@
  \newcommand*{\PrintTime}[2]{%
    \typeout{%
      [Time #1: \strip@pt\dimexpr\number#2sp\relax\space s]%
    }%
  }%
  \newcommand*{\StartTime}[1]{%
    \renewcommand*{\TimeDescription}{#1}%
    \pdfresettimer
  }%
  \newcommand*{\TimeDescription}{}%
  \newcommand*{\StopTime}{%
    \TestTime=\pdfelapsedtime
    \global\advance\SummaryTime\TestTime
    \PrintTime\TimeDescription\TestTime
  }%
  \let\saved@qstest\qstest
  \let\saved@endqstest\endqstest
  \def\qstest#1#2{%
    \saved@qstest{#1}{#2}%
    \StartTime{#1}%
  }%
  \def\endqstest{%
    \StopTime
    \saved@endqstest
  }%
  \AtEndDocument{%
    \PrintTime{summary}\SummaryTime
  }%
  \makeatother
\fi
%    \end{macrocode}
%
% \subsubsection{Test sets}
%
%    \begin{macrocode}
\makeatletter

\begin{qstest}{inv}{inv}%
  \TestInv{0}{0}%
  \TestInv{1}{-1}%
  \TestInv{-1}{1}%
  \TestInv{10}{-10}%
  \TestInv{-10}{10}%
  \TestInv{2147483647}{-2147483647}%
  \TestInv{-2147483647}{2147483647}%
  \TestInv{12345678901234567890}{-12345678901234567890}%
  \TestInv{-12345678901234567890}{12345678901234567890}%
  \TestInv{ 0 }{0}%
  \TestInv{ 1 }{-1}%
  \TestInv{--1}{-1}%
  \TestInv{\number\z@}{0}%
  \TestInv{\ifx\relax\relax1\fi}{-1}%
  \TestInv{\ifx\relax\relax-\fi\ifx234\else1\fi}{1}%
\end{qstest}

\begin{qstest}{abs}{abs}%
  \TestAbs{0}{0}%
  \TestAbs{1}{1}%
  \TestAbs{-1}{1}%
  \TestAbs{10}{10}%
  \TestAbs{-10}{10}%
  \TestAbs{2147483647}{2147483647}%
  \TestAbs{-2147483647}{2147483647}%
  \TestAbs{12345678901234567890}{12345678901234567890}%
  \TestAbs{-12345678901234567890}{12345678901234567890}%
  \TestAbs{ 0 }{0}%
  \TestAbs{ 1 }{1}%
  \TestAbs{--1}{1}%
  \TestAbs{-+-+1}{1}%
  \TestAbs{00000000000}{0}%
  \TestAbs{00000001000}{1000}%
  \TestAbs{\ifx\relax\relax 0\else 1\fi}{0}%
\end{qstest}

\begin{qstest}{sign}{sign}%
  \TestSgn{0}{0}%
  \TestSgn{1}{1}%
  \TestSgn{-1}{-1}%
  \TestSgn{10}{1}%
  \TestSgn{-10}{-1}%
  \TestSgn{2147483647}{1}%
  \TestSgn{-2147483647}{-1}%
  \TestSgn{12345678901234567890}{1}%
  \TestSgn{-12345678901234567890}{-1}%
  \TestSgn{ 0 }{0}%
  \TestSgn{ 2 }{1}%
  \TestSgn{ -2 }{-1}%
  \TestSgn{--2}{1}%
  \TestSgn{\number\z@}{0}%
  \TestSgn{\number\@ne}{1}%
  \TestSgn{\number\m@ne}{-1}%
  \TestSgn{%
    -+-+\number\z@\number\z@
    \iftrue1\fi\iftrue2\fi\iftrue3\fi
  }{1}%
\end{qstest}

\begin{qstest}{min}{min}%
  \TestMin{0}{1}{0}%
  \TestMin{1}{0}{0}%
  \TestMin{-10}{-20}{-20}%
  \TestMin{ 1 }{ 2 }{1}%
  \TestMin{ 2 }{ 1 }{1}%
  \TestMin{1}{1}{1}%
  \TestMin{\number\z@}{\number\@ne}{0}%
  \TestMin{\number\@ne}{\number\m@ne}{-1}%
\end{qstest}

\begin{qstest}{max}{max}%
  \TestMax{0}{1}{1}%
  \TestMax{1}{0}{1}%
  \TestMax{-10}{-20}{-10}%
  \TestMax{ 1 }{ 2 }{2}%
  \TestMax{ 2 }{ 1 }{2}%
  \TestMax{1}{1}{1}%
  \TestMax{\number\z@}{\number\@ne}{1}%
  \TestMax{\number\@ne}{\number\m@ne}{1}%
\end{qstest}

\begin{qstest}{cmp}{cmp}%
  \TestCmp{0}{0}{0}%
  \TestCmp{-21}{17}{-1}%
  \TestCmp{3}{4}{-1}%
  \TestCmp{-10}{-10}{0}%
  \TestCmp{-10}{-11}{1}%
  \TestCmp{100}{5}{1}%
  \TestCmp{9}{10}{-1}%
  \TestCmp{10}{9}{1}%
  \TestCmp{ 3 }{ 3 }{0}%
  \TestCmp{-9}{-10}{1}%
  \TestCmp{-10}{-9}{-1}%
  \TestCmp{-3}{-3}{0}%
  \TestCmp{0}{-2}{1}%
  \TestCmp{0}{2}{-1}%
  \TestCmp{2}{0}{1}%
  \TestCmp{-2}{0}{-1}%
  \TestCmp{12}{11}{1}%
  \TestCmp{11}{12}{-1}%
  \TestCmp{2147483647}{-2147483647}{1}%
  \TestCmp{-2147483647}{2147483647}{-1}%
  \TestCmp{2147483647}{2147483647}{0}%
  \TestCmp{\number\z@}{\number\@ne}{-1}%
  \TestCmp{\number\@ne}{\number\m@ne}{1}%
  \TestCmp{ 4 }{ 5 }{-1}%
  \TestCmp{ -3 }{ -7 }{1}%
\end{qstest}

\begin{qstest}{odd}{odd}
\tracingmacros=1
  \TestOdd{0}{0}%
  \TestOdd{1}{1}%
  \TestOdd{2}{0}%
  \TestOdd{3}{1}%
  \TestOdd{14}{0}%
  \TestOdd{15}{1}%
  \TestOdd{12345678901234567896}{0}%
  \TestOdd{12345678901234567897}{1}%
\end{qstest}

\begin{qstest}{inc}{inc}%
  \TestInc{0}{1}%
  \TestInc{1}{2}%
  \TestInc{-1}{0}%
  \TestInc{10}{11}%
  \TestInc{-10}{-9}%
  \TestInc{ 3 }{4}%
  \TestInc{999}{1000}%
  \TestInc{-1000}{-999}%
  \TestInc{129}{130}%
  \TestInc{2147483646}{2147483647}%
  \TestInc{-2147483647}{-2147483646}%
  \TestInc{12345678901234567890}{12345678901234567891}%
  \TestInc{99999999999999999999}{100000000000000000000}%
  \TestInc{-12345678901234567891}{-12345678901234567890}%
  \TestInc{-100000000000000000000}{-99999999999999999999}%
\end{qstest}

\begin{qstest}{dec}{dec}%
  \TestDec{0}{-1}%
  \TestDec{1}{0}%
  \TestDec{-1}{-2}%
  \TestDec{10}{9}%
  \TestDec{-10}{-11}%
  \TestDec{1000}{999}%
  \TestDec{-999}{-1000}%
  \TestDec{130}{129}%
  \TestDec{2147483647}{2147483646}%
  \TestDec{-2147483646}{-2147483647}%
  \TestDec{12345678901234567891}{12345678901234567890}%
  \TestDec{100000000000000000000}{99999999999999999999}%
  \TestDec{-12345678901234567890}{-12345678901234567891}%
  \TestDec{-99999999999999999999}{-100000000000000000000}%
\end{qstest}

\begin{qstest}{add}{add}%
  \TestAdd{0}{0}{0}%
  \TestAdd{1}{0}{1}%
  \TestAdd{0}{1}{1}%
  \TestAdd{1}{2}{3}%
  \TestAdd{-1}{-1}{-2}%
  \TestAdd{2147483646}{1}{2147483647}%
  \TestAdd{-2147483647}{2147483647}{0}%
  \TestAdd{20}{-5}{15}%
  \TestAdd{-4}{-1}{-5}%
  \TestAdd{-1}{-4}{-5}%
  \TestAdd{-4}{1}{-3}%
  \TestAdd{-1}{4}{3}%
  \TestAdd{4}{-1}{3}%
  \TestAdd{1}{-4}{-3}%
  \TestAdd{-4}{-1}{-5}%
  \TestAdd{-1}{-4}{-5}%
  \TestAdd{ -4 }{ -1 }{-5}%
  \TestAdd{ -1 }{ -4 }{-5}%
  \TestAdd{ -4 }{ 1 }{-3}%
  \TestAdd{ -1 }{ 4 }{3}%
  \TestAdd{ 4 }{ -1 }{3}%
  \TestAdd{ 1 }{ -4 }{-3}%
  \TestAdd{ -4 }{ -1 }{-5}%
  \TestAdd{ -1 }{ -4 }{-5}%
  \TestAdd{876543210}{111111111}{987654321}%
  \TestAdd{999999999}{2}{1000000001}%
\end{qstest}

\begin{qstest}{sub}{sub}
  \TestSub{0}{0}{0}%
  \TestSub{1}{0}{1}%
  \TestSub{1}{2}{-1}%
  \TestSub{-1}{-1}{0}%
  \TestSub{2147483646}{-1}{2147483647}%
  \TestSub{-2147483647}{-2147483647}{0}%
  \TestSub{-4}{-1}{-3}%
  \TestSub{-1}{-4}{3}%
  \TestSub{-4}{1}{-5}%
  \TestSub{-1}{4}{-5}%
  \TestSub{4}{-1}{5}%
  \TestSub{1}{-4}{5}%
  \TestSub{-4}{-1}{-3}%
  \TestSub{-1}{-4}{3}%
  \TestSub{ -4 }{ -1 }{-3}%
  \TestSub{ -1 }{ -4 }{3}%
  \TestSub{ -4 }{ 1 }{-5}%
  \TestSub{ -1 }{ 4 }{-5}%
  \TestSub{ 4 }{ -1 }{5}%
  \TestSub{ 1 }{ -4 }{5}%
  \TestSub{ -4 }{ -1 }{-3}%
  \TestSub{ -1 }{ -4 }{3}%
  \TestSub{1000000000}{2}{999999998}%
  \TestSub{987654321}{111111111}{876543210}%
\end{qstest}

\begin{qstest}{shl}{shl}
  \TestShl{0}{0}%
  \TestShl{1}{2}%
  \TestShl{2}{4}%
  \TestShl{5621}{11242}%
  \TestShl{1073741823}{2147483646}%
\end{qstest}

\begin{qstest}{shr}{shr}
  \TestShr{0}{0}%
  \TestShr{1}{0}%
  \TestShr{2}{1}%
  \TestShr{3}{1}%
  \TestShr{4}{2}%
  \TestShr{5}{2}%
  \TestShr{6}{3}%
  \TestShr{7}{3}%
  \TestShr{8}{4}%
  \TestShr{9}{4}%
  \TestShr{10}{5}%
  \TestShr{11}{5}%
  \TestShr{12}{6}%
  \TestShr{13}{6}%
  \TestShr{14}{7}%
  \TestShr{15}{7}%
  \TestShr{16}{8}%
  \TestShr{17}{8}%
  \TestShr{18}{9}%
  \TestShr{19}{9}%
  \TestShr{20}{10}%
  \TestShr{21}{10}%
  \TestShr{22}{11}%
  \TestShr{11241}{5620}%
  \TestShr{73054202}{36527101}%
  \TestShr{2147483646}{1073741823}%
\end{qstest}

\begin{qstest}{mul}{mul}
  \TestMul{0}{0}{0}%
  \TestMul{1}{0}{0}%
  \TestMul{0}{1}{0}%
  \TestMul{1}{1}{1}%
  \TestMul{3}{1}{3}%
  \TestMul{1}{-3}{-3}%
  \TestMul{-4}{-5}{20}%
  \TestMul{3}{7}{21}%
  \TestMul{7}{3}{21}%
  \TestMul{3}{-7}{-21}%
  \TestMul{7}{-3}{-21}%
  \TestMul{-3}{7}{-21}%
  \TestMul{-7}{3}{-21}%
  \TestMul{-3}{-7}{21}%
  \TestMul{-7}{-3}{21}%
  \TestMul{12}{11}{132}%
  \TestMul{999}{333}{332667}%
  \TestMul{1000}{4321}{4321000}%
  \TestMul{12345}{173955}{2147474475}%
  \TestMul{1073741823}{2}{2147483646}%
  \TestMul{2}{1073741823}{2147483646}%
  \TestMul{-1073741823}{2}{-2147483646}%
  \TestMul{2}{-1073741823}{-2147483646}%
  \TestMul{6706022400}{13}{87178291200}%
\end{qstest}

\begin{qstest}{sqr}{sqr}
  \TestSqr{0}{0}%
  \TestSqr{1}{1}%
  \TestSqr{2}{4}%
  \TestSqr{3}{9}%
  \TestSqr{4}{16}%
  \TestSqr{9}{81}%
  \TestSqr{10}{100}%
  \TestSqr{46340}{2147395600}%
  \TestSqr{-1}{1}%
  \TestSqr{-2}{4}%
  \TestSqr{-46340}{2147395600}%
\end{qstest}

\begin{qstest}{fac}{fac}
  \TestFac{0}{1}%
  \TestFac{1}{1}%
  \TestFac{2}{2}%
  \TestFac{3}{2*3}%
  \TestFac{4}{2*3*4}%
  \TestFac{5}{2*3*4*5}%
  \TestFac{6}{2*3*4*5*6}%
  \TestFac{7}{2*3*4*5*6*7}%
  \TestFac{8}{2*3*4*5*6*7*8}%
  \TestFac{9}{2*3*4*5*6*7*8*9}%
  \TestFac{10}{2*3*4*5*6*7*8*9*10}%
  \TestFac{11}{2*3*4*5*6*7*8*9*10*11}%
  \TestFac{12}{2*3*4*5*6*7*8*9*10*11*12}%
  \TestFacBig{13}{6227020800}%
  \TestFacBig{14}{87178291200}%
  \TestFacBig{15}{1307674368000}%
  \TestFacBig{16}{20922789888000}%
  \TestFacBig{17}{355687428096000}%
  \TestFacBig{18}{6402373705728000}%
  \TestFacBig{19}{121645100408832000}%
  \TestFacBig{20}{2432902008176640000}%
  \TestFacBig{21}{51090942171709440000}%
  \TestFacBig{22}{1124000727777607680000}%
\end{qstest}

\begin{qstest}{pow}{pow}
  \TestPow{-2}{0}{1}%
  \TestPow{-1}{0}{1}%
  \TestPow{0}{0}{1}%
  \TestPow{1}{0}{1}%
  \TestPow{2}{0}{1}%
  \TestPow{3}{0}{1}%
  \TestPow{-2}{1}{-2}%
  \TestPow{-1}{1}{-1}%
  \TestPow{1}{1}{1}%
  \TestPow{2}{1}{2}%
  \TestPow{3}{1}{3}%
  \TestPow{-2}{2}{4}%
  \TestPow{-1}{2}{1}%
  \TestPow{0}{2}{0}%
  \TestPow{1}{2}{1}%
  \TestPow{2}{2}{4}%
  \TestPow{3}{2}{9}%
  \TestPow{0}{1}{0}%
  \TestPow{1}{-2}{1}%
  \TestPow{1}{-1}{1}%
  \TestPow{-1}{-2}{1}%
  \TestPow{-1}{-1}{-1}%
  \TestPow{-1}{3}{-1}%
  \TestPow{-1}{4}{1}%
  \TestPow{-2}{-1}{0}%
  \TestPow{-2}{-2}{0}%
  \TestPow{2}{3}{8}%
  \TestPow{2}{4}{16}%
  \TestPow{2}{5}{32}%
  \TestPow{2}{6}{64}%
  \TestPow{2}{7}{128}%
  \TestPow{2}{8}{256}%
  \TestPow{2}{9}{512}%
  \TestPow{2}{10}{1024}%
  \TestPow{-2}{3}{-8}%
  \TestPow{-2}{4}{16}%
  \TestPow{-2}{5}{-32}%
  \TestPow{-2}{6}{64}%
  \TestPow{-2}{7}{-128}%
  \TestPow{-2}{8}{256}%
  \TestPow{-2}{9}{-512}%
  \TestPow{-2}{10}{1024}%
  \TestPow{3}{3}{27}%
  \TestPow{3}{4}{81}%
  \TestPow{3}{5}{243}%
  \TestPow{-3}{3}{-27}%
  \TestPow{-3}{4}{81}%
  \TestPow{-3}{5}{-243}%
  \TestPow{2}{30}{1073741824}%
  \TestPow{-3}{19}{-1162261467}%
  \TestPow{5}{13}{1220703125}%
  \TestPow{-7}{11}{-1977326743}%
\end{qstest}

\begin{qstest}{div}{div}
  \TestDiv{1}{1}{1}%
  \TestDiv{2}{1}{2}%
  \TestDiv{-2}{1}{-2}%
  \TestDiv{2}{-1}{-2}%
  \TestDiv{-2}{-1}{2}%
  \TestDiv{15}{2}{7}%
  \TestDiv{-16}{2}{-8}%
  \TestDiv{1}{2}{0}%
  \TestDiv{1}{3}{0}%
  \TestDiv{2}{3}{0}%
  \TestDiv{-2}{3}{0}%
  \TestDiv{2}{-3}{0}%
  \TestDiv{-2}{-3}{0}%
  \TestDiv{13}{3}{4}%
  \TestDiv{-13}{-3}{4}%
  \TestDiv{-13}{3}{-4}%
  \TestDiv{-6}{5}{-1}%
  \TestDiv{-5}{5}{-1}%
  \TestDiv{-4}{5}{0}%
  \TestDiv{-3}{5}{0}%
  \TestDiv{-2}{5}{0}%
  \TestDiv{-1}{5}{0}%
  \TestDiv{0}{5}{0}%
  \TestDiv{1}{5}{0}%
  \TestDiv{2}{5}{0}%
  \TestDiv{3}{5}{0}%
  \TestDiv{4}{5}{0}%
  \TestDiv{5}{5}{1}%
  \TestDiv{6}{5}{1}%
  \TestDiv{-5}{4}{-1}%
  \TestDiv{-4}{4}{-1}%
  \TestDiv{-3}{4}{0}%
  \TestDiv{-2}{4}{0}%
  \TestDiv{-1}{4}{0}%
  \TestDiv{0}{4}{0}%
  \TestDiv{1}{4}{0}%
  \TestDiv{2}{4}{0}%
  \TestDiv{3}{4}{0}%
  \TestDiv{4}{4}{1}%
  \TestDiv{5}{4}{1}%
  \TestDiv{12345}{678}{18}%
  \TestDiv{32372}{5952}{5}%
  \TestDiv{284271294}{18162}{15651}%
  \TestDiv{217652429}{12561}{17327}%
  \TestDiv{462028434}{5439}{84947}%
  \TestDiv{2147483647}{1000}{2147483}%
  \TestDiv{2147483647}{-1000}{-2147483}%
  \TestDiv{-2147483647}{1000}{-2147483}%
  \TestDiv{-2147483647}{-1000}{2147483}%
  \TestDiv{0}{3}{0}%
  \TestDiv{1}{3}{0}%
  \TestDiv{2}{3}{0}%
  \TestDiv{3}{3}{1}%
  \TestDiv{4}{3}{1}%
  \TestDiv{5}{3}{1}%
  \TestDiv{6}{3}{2}%
  \TestDiv{7}{3}{2}%
  \TestDiv{8}{3}{2}%
  \TestDiv{9}{3}{3}%
  \TestDiv{10}{3}{3}%
  \TestDiv{11}{3}{3}%
  \TestDiv{12}{3}{4}%
  \TestDiv{13}{3}{4}%
  \TestDiv{14}{3}{4}%
  \TestDiv{15}{3}{5}%
  \TestDiv{16}{3}{5}%
  \TestDiv{17}{3}{5}%
  \TestDiv{18}{3}{6}%
  \TestDiv{19}{3}{6}%
  \TestDiv{20}{3}{6}%
  \TestDiv{21}{3}{7}%
  \TestDiv{22}{3}{7}%
  \TestDiv{23}{3}{7}%
  \TestDiv{24}{3}{8}%
  \TestDiv{25}{3}{8}%
  \TestDiv{26}{3}{8}%
  \TestDiv{27}{3}{9}%
  \TestDiv{28}{3}{9}%
  \TestDiv{29}{3}{9}%
  \TestDiv{30}{3}{10}%
  \TestDiv{31}{3}{10}%
  \TestDivBig{17363436332507}{24702}{702916214}%
\end{qstest}

\begin{qstest}{mod}{mod}
  \TestMod{-6}{5}{4}%
  \TestMod{-5}{5}{0}%
  \TestMod{-4}{5}{1}%
  \TestMod{-3}{5}{2}%
  \TestMod{-2}{5}{3}%
  \TestMod{-1}{5}{4}%
  \TestMod{0}{5}{0}%
  \TestMod{1}{5}{1}%
  \TestMod{2}{5}{2}%
  \TestMod{3}{5}{3}%
  \TestMod{4}{5}{4}%
  \TestMod{5}{5}{0}%
  \TestMod{6}{5}{1}%
  \TestMod{-5}{4}{3}%
  \TestMod{-4}{4}{0}%
  \TestMod{-3}{4}{1}%
  \TestMod{-2}{4}{2}%
  \TestMod{-1}{4}{3}%
  \TestMod{0}{4}{0}%
  \TestMod{1}{4}{1}%
  \TestMod{2}{4}{2}%
  \TestMod{3}{4}{3}%
  \TestMod{4}{4}{0}%
  \TestMod{5}{4}{1}%
  \TestMod{-6}{-5}{-1}%
  \TestMod{-5}{-5}{0}%
  \TestMod{-4}{-5}{-4}%
  \TestMod{-3}{-5}{-3}%
  \TestMod{-2}{-5}{-2}%
  \TestMod{-1}{-5}{-1}%
  \TestMod{0}{-5}{0}%
  \TestMod{1}{-5}{-4}%
  \TestMod{2}{-5}{-3}%
  \TestMod{3}{-5}{-2}%
  \TestMod{4}{-5}{-1}%
  \TestMod{5}{-5}{0}%
  \TestMod{6}{-5}{-4}%
  \TestMod{-5}{-4}{-1}%
  \TestMod{-4}{-4}{0}%
  \TestMod{-3}{-4}{-3}%
  \TestMod{-2}{-4}{-2}%
  \TestMod{-1}{-4}{-1}%
  \TestMod{0}{-4}{0}%
  \TestMod{1}{-4}{-3}%
  \TestMod{2}{-4}{-2}%
  \TestMod{3}{-4}{-1}%
  \TestMod{4}{-4}{0}%
  \TestMod{5}{-4}{-3}%
  \TestMod{2147483647}{1000}{647}%
  \TestMod{2147483647}{-1000}{-353}%
  \TestMod{-2147483647}{1000}{353}%
  \TestMod{-2147483647}{-1000}{-647}%
  \TestMod{ 0 }{ 4 }{0}%
  \TestMod{ 1 }{ 4 }{1}%
  \TestMod{ -1 }{ 4 }{3}%
  \TestMod{ 0 }{ -4 }{0}%
  \TestMod{ 1 }{ -4 }{-3}%
  \TestMod{ -1 }{ -4 }{-1}%
  \TestMod{18362}{25}{12}%
\end{qstest}

\newcommand*{\TestError}[2]{%
  \begingroup
    \expandafter\def\csname BigIntCalcError:#1\endcsname{}%
    \Expect*{#2}{0}%
    \expandafter\def\csname BigIntCalcError:#1\endcsname{ERROR}%
    \Expect*{#2}{0ERROR}%
  \endgroup
}
\begin{qstest}{error}{error}
  \TestError{FacNegative}{\bigintcalcFac{-1}}%
  \TestError{FacNegative}{\bigintcalcFac{-2147483647}}%
  \TestError{DivisionByZero}{\bigintcalcPow{0}{-1}}%
  \TestError{DivisionByZero}{\bigintcalcDiv{1}{0}}%
  \TestError{DivisionByZero}{\bigintcalcMod{1}{0}}%
\end{qstest}

\begin{document}
\end{document}
%</test2>
%    \end{macrocode}
%
% \section{Installation}
%
% \subsection{Download}
%
% \paragraph{Package.} This package is available on
% CTAN\footnote{\url{http://ctan.org/pkg/bigintcalc}}:
% \begin{description}
% \item[\CTAN{macros/latex/contrib/oberdiek/bigintcalc.dtx}] The source file.
% \item[\CTAN{macros/latex/contrib/oberdiek/bigintcalc.pdf}] Documentation.
% \end{description}
%
%
% \paragraph{Bundle.} All the packages of the bundle `oberdiek'
% are also available in a TDS compliant ZIP archive. There
% the packages are already unpacked and the documentation files
% are generated. The files and directories obey the TDS standard.
% \begin{description}
% \item[\CTAN{install/macros/latex/contrib/oberdiek.tds.zip}]
% \end{description}
% \emph{TDS} refers to the standard ``A Directory Structure
% for \TeX\ Files'' (\CTAN{tds/tds.pdf}). Directories
% with \xfile{texmf} in their name are usually organized this way.
%
% \subsection{Bundle installation}
%
% \paragraph{Unpacking.} Unpack the \xfile{oberdiek.tds.zip} in the
% TDS tree (also known as \xfile{texmf} tree) of your choice.
% Example (linux):
% \begin{quote}
%   |unzip oberdiek.tds.zip -d ~/texmf|
% \end{quote}
%
% \paragraph{Script installation.}
% Check the directory \xfile{TDS:scripts/oberdiek/} for
% scripts that need further installation steps.
% Package \xpackage{attachfile2} comes with the Perl script
% \xfile{pdfatfi.pl} that should be installed in such a way
% that it can be called as \texttt{pdfatfi}.
% Example (linux):
% \begin{quote}
%   |chmod +x scripts/oberdiek/pdfatfi.pl|\\
%   |cp scripts/oberdiek/pdfatfi.pl /usr/local/bin/|
% \end{quote}
%
% \subsection{Package installation}
%
% \paragraph{Unpacking.} The \xfile{.dtx} file is a self-extracting
% \docstrip\ archive. The files are extracted by running the
% \xfile{.dtx} through \plainTeX:
% \begin{quote}
%   \verb|tex bigintcalc.dtx|
% \end{quote}
%
% \paragraph{TDS.} Now the different files must be moved into
% the different directories in your installation TDS tree
% (also known as \xfile{texmf} tree):
% \begin{quote}
% \def\t{^^A
% \begin{tabular}{@{}>{\ttfamily}l@{ $\rightarrow$ }>{\ttfamily}l@{}}
%   bigintcalc.sty & tex/generic/oberdiek/bigintcalc.sty\\
%   bigintcalc.pdf & doc/latex/oberdiek/bigintcalc.pdf\\
%   test/bigintcalc-test1.tex & doc/latex/oberdiek/test/bigintcalc-test1.tex\\
%   test/bigintcalc-test2.tex & doc/latex/oberdiek/test/bigintcalc-test2.tex\\
%   test/bigintcalc-test3.tex & doc/latex/oberdiek/test/bigintcalc-test3.tex\\
%   bigintcalc.dtx & source/latex/oberdiek/bigintcalc.dtx\\
% \end{tabular}^^A
% }^^A
% \sbox0{\t}^^A
% \ifdim\wd0>\linewidth
%   \begingroup
%     \advance\linewidth by\leftmargin
%     \advance\linewidth by\rightmargin
%   \edef\x{\endgroup
%     \def\noexpand\lw{\the\linewidth}^^A
%   }\x
%   \def\lwbox{^^A
%     \leavevmode
%     \hbox to \linewidth{^^A
%       \kern-\leftmargin\relax
%       \hss
%       \usebox0
%       \hss
%       \kern-\rightmargin\relax
%     }^^A
%   }^^A
%   \ifdim\wd0>\lw
%     \sbox0{\small\t}^^A
%     \ifdim\wd0>\linewidth
%       \ifdim\wd0>\lw
%         \sbox0{\footnotesize\t}^^A
%         \ifdim\wd0>\linewidth
%           \ifdim\wd0>\lw
%             \sbox0{\scriptsize\t}^^A
%             \ifdim\wd0>\linewidth
%               \ifdim\wd0>\lw
%                 \sbox0{\tiny\t}^^A
%                 \ifdim\wd0>\linewidth
%                   \lwbox
%                 \else
%                   \usebox0
%                 \fi
%               \else
%                 \lwbox
%               \fi
%             \else
%               \usebox0
%             \fi
%           \else
%             \lwbox
%           \fi
%         \else
%           \usebox0
%         \fi
%       \else
%         \lwbox
%       \fi
%     \else
%       \usebox0
%     \fi
%   \else
%     \lwbox
%   \fi
% \else
%   \usebox0
% \fi
% \end{quote}
% If you have a \xfile{docstrip.cfg} that configures and enables \docstrip's
% TDS installing feature, then some files can already be in the right
% place, see the documentation of \docstrip.
%
% \subsection{Refresh file name databases}
%
% If your \TeX~distribution
% (\teTeX, \mikTeX, \dots) relies on file name databases, you must refresh
% these. For example, \teTeX\ users run \verb|texhash| or
% \verb|mktexlsr|.
%
% \subsection{Some details for the interested}
%
% \paragraph{Attached source.}
%
% The PDF documentation on CTAN also includes the
% \xfile{.dtx} source file. It can be extracted by
% AcrobatReader 6 or higher. Another option is \textsf{pdftk},
% e.g. unpack the file into the current directory:
% \begin{quote}
%   \verb|pdftk bigintcalc.pdf unpack_files output .|
% \end{quote}
%
% \paragraph{Unpacking with \LaTeX.}
% The \xfile{.dtx} chooses its action depending on the format:
% \begin{description}
% \item[\plainTeX:] Run \docstrip\ and extract the files.
% \item[\LaTeX:] Generate the documentation.
% \end{description}
% If you insist on using \LaTeX\ for \docstrip\ (really,
% \docstrip\ does not need \LaTeX), then inform the autodetect routine
% about your intention:
% \begin{quote}
%   \verb|latex \let\install=y% \iffalse meta-comment
%
% File: bigintcalc.dtx
% Version: 2016/05/16 v1.4
% Info: Expandable calculations on big integers
%
% Copyright (C) 2007, 2011, 2012 by
%    Heiko Oberdiek <heiko.oberdiek at googlemail.com>
%    2016
%    https://github.com/ho-tex/oberdiek/issues
%
% This work may be distributed and/or modified under the
% conditions of the LaTeX Project Public License, either
% version 1.3c of this license or (at your option) any later
% version. This version of this license is in
%    http://www.latex-project.org/lppl/lppl-1-3c.txt
% and the latest version of this license is in
%    http://www.latex-project.org/lppl.txt
% and version 1.3 or later is part of all distributions of
% LaTeX version 2005/12/01 or later.
%
% This work has the LPPL maintenance status "maintained".
%
% This Current Maintainer of this work is Heiko Oberdiek.
%
% The Base Interpreter refers to any `TeX-Format',
% because some files are installed in TDS:tex/generic//.
%
% This work consists of the main source file bigintcalc.dtx
% and the derived files
%    bigintcalc.sty, bigintcalc.pdf, bigintcalc.ins, bigintcalc.drv,
%    bigintcalc-test1.tex, bigintcalc-test2.tex,
%    bigintcalc-test3.tex.
%
% Distribution:
%    CTAN:macros/latex/contrib/oberdiek/bigintcalc.dtx
%    CTAN:macros/latex/contrib/oberdiek/bigintcalc.pdf
%
% Unpacking:
%    (a) If bigintcalc.ins is present:
%           tex bigintcalc.ins
%    (b) Without bigintcalc.ins:
%           tex bigintcalc.dtx
%    (c) If you insist on using LaTeX
%           latex \let\install=y\input{bigintcalc.dtx}
%        (quote the arguments according to the demands of your shell)
%
% Documentation:
%    (a) If bigintcalc.drv is present:
%           latex bigintcalc.drv
%    (b) Without bigintcalc.drv:
%           latex bigintcalc.dtx; ...
%    The class ltxdoc loads the configuration file ltxdoc.cfg
%    if available. Here you can specify further options, e.g.
%    use A4 as paper format:
%       \PassOptionsToClass{a4paper}{article}
%
%    Programm calls to get the documentation (example):
%       pdflatex bigintcalc.dtx
%       makeindex -s gind.ist bigintcalc.idx
%       pdflatex bigintcalc.dtx
%       makeindex -s gind.ist bigintcalc.idx
%       pdflatex bigintcalc.dtx
%
% Installation:
%    TDS:tex/generic/oberdiek/bigintcalc.sty
%    TDS:doc/latex/oberdiek/bigintcalc.pdf
%    TDS:doc/latex/oberdiek/test/bigintcalc-test1.tex
%    TDS:doc/latex/oberdiek/test/bigintcalc-test2.tex
%    TDS:doc/latex/oberdiek/test/bigintcalc-test3.tex
%    TDS:source/latex/oberdiek/bigintcalc.dtx
%
%<*ignore>
\begingroup
  \catcode123=1 %
  \catcode125=2 %
  \def\x{LaTeX2e}%
\expandafter\endgroup
\ifcase 0\ifx\install y1\fi\expandafter
         \ifx\csname processbatchFile\endcsname\relax\else1\fi
         \ifx\fmtname\x\else 1\fi\relax
\else\csname fi\endcsname
%</ignore>
%<*install>
\input docstrip.tex
\Msg{************************************************************************}
\Msg{* Installation}
\Msg{* Package: bigintcalc 2016/05/16 v1.4 Expandable calculations on big integers (HO)}
\Msg{************************************************************************}

\keepsilent
\askforoverwritefalse

\let\MetaPrefix\relax
\preamble

This is a generated file.

Project: bigintcalc
Version: 2016/05/16 v1.4

Copyright (C) 2007, 2011, 2012 by
   Heiko Oberdiek <heiko.oberdiek at googlemail.com>

This work may be distributed and/or modified under the
conditions of the LaTeX Project Public License, either
version 1.3c of this license or (at your option) any later
version. This version of this license is in
   http://www.latex-project.org/lppl/lppl-1-3c.txt
and the latest version of this license is in
   http://www.latex-project.org/lppl.txt
and version 1.3 or later is part of all distributions of
LaTeX version 2005/12/01 or later.

This work has the LPPL maintenance status "maintained".

This Current Maintainer of this work is Heiko Oberdiek.

The Base Interpreter refers to any `TeX-Format',
because some files are installed in TDS:tex/generic//.

This work consists of the main source file bigintcalc.dtx
and the derived files
   bigintcalc.sty, bigintcalc.pdf, bigintcalc.ins, bigintcalc.drv,
   bigintcalc-test1.tex, bigintcalc-test2.tex,
   bigintcalc-test3.tex.

\endpreamble
\let\MetaPrefix\DoubleperCent

\generate{%
  \file{bigintcalc.ins}{\from{bigintcalc.dtx}{install}}%
  \file{bigintcalc.drv}{\from{bigintcalc.dtx}{driver}}%
  \usedir{tex/generic/oberdiek}%
  \file{bigintcalc.sty}{\from{bigintcalc.dtx}{package}}%
%  \usedir{doc/latex/oberdiek/test}%
%  \file{bigintcalc-test1.tex}{\from{bigintcalc.dtx}{test1}}%
%  \file{bigintcalc-test2.tex}{\from{bigintcalc.dtx}{test2,etex}}%
%  \file{bigintcalc-test3.tex}{\from{bigintcalc.dtx}{test2,noetex}}%
  \nopreamble
  \nopostamble
  \usedir{source/latex/oberdiek/catalogue}%
  \file{bigintcalc.xml}{\from{bigintcalc.dtx}{catalogue}}%
}

\catcode32=13\relax% active space
\let =\space%
\Msg{************************************************************************}
\Msg{*}
\Msg{* To finish the installation you have to move the following}
\Msg{* file into a directory searched by TeX:}
\Msg{*}
\Msg{*     bigintcalc.sty}
\Msg{*}
\Msg{* To produce the documentation run the file `bigintcalc.drv'}
\Msg{* through LaTeX.}
\Msg{*}
\Msg{* Happy TeXing!}
\Msg{*}
\Msg{************************************************************************}

\endbatchfile
%</install>
%<*ignore>
\fi
%</ignore>
%<*driver>
\NeedsTeXFormat{LaTeX2e}
\ProvidesFile{bigintcalc.drv}%
  [2016/05/16 v1.4 Expandable calculations on big integers (HO)]%
\documentclass{ltxdoc}
\usepackage{wasysym}
\let\iint\relax
\let\iiint\relax
\usepackage[fleqn]{amsmath}

\DeclareMathOperator{\opInv}{Inv}
\DeclareMathOperator{\opAbs}{Abs}
\DeclareMathOperator{\opSgn}{Sgn}
\DeclareMathOperator{\opMin}{Min}
\DeclareMathOperator{\opMax}{Max}
\DeclareMathOperator{\opCmp}{Cmp}
\DeclareMathOperator{\opOdd}{Odd}
\DeclareMathOperator{\opInc}{Inc}
\DeclareMathOperator{\opDec}{Dec}
\DeclareMathOperator{\opAdd}{Add}
\DeclareMathOperator{\opSub}{Sub}
\DeclareMathOperator{\opShl}{Shl}
\DeclareMathOperator{\opShr}{Shr}
\DeclareMathOperator{\opMul}{Mul}
\DeclareMathOperator{\opSqr}{Sqr}
\DeclareMathOperator{\opFac}{Fac}
\DeclareMathOperator{\opPow}{Pow}
\DeclareMathOperator{\opDiv}{Div}
\DeclareMathOperator{\opMod}{Mod}
\DeclareMathOperator{\opInt}{Int}
\DeclareMathOperator{\opODD}{ifodd}

\newcommand*{\Def}{%
  \ensuremath{%
    \mathrel{\mathop{:}}=%
  }%
}
\newcommand*{\op}[1]{%
  \textsf{#1}%
}
\usepackage{holtxdoc}[2011/11/22]
\begin{document}
  \DocInput{bigintcalc.dtx}%
\end{document}
%</driver>
% \fi
%
%
% \CharacterTable
%  {Upper-case    \A\B\C\D\E\F\G\H\I\J\K\L\M\N\O\P\Q\R\S\T\U\V\W\X\Y\Z
%   Lower-case    \a\b\c\d\e\f\g\h\i\j\k\l\m\n\o\p\q\r\s\t\u\v\w\x\y\z
%   Digits        \0\1\2\3\4\5\6\7\8\9
%   Exclamation   \!     Double quote  \"     Hash (number) \#
%   Dollar        \$     Percent       \%     Ampersand     \&
%   Acute accent  \'     Left paren    \(     Right paren   \)
%   Asterisk      \*     Plus          \+     Comma         \,
%   Minus         \-     Point         \.     Solidus       \/
%   Colon         \:     Semicolon     \;     Less than     \<
%   Equals        \=     Greater than  \>     Question mark \?
%   Commercial at \@     Left bracket  \[     Backslash     \\
%   Right bracket \]     Circumflex    \^     Underscore    \_
%   Grave accent  \`     Left brace    \{     Vertical bar  \|
%   Right brace   \}     Tilde         \~}
%
% \GetFileInfo{bigintcalc.drv}
%
% \title{The \xpackage{bigintcalc} package}
% \date{2016/05/16 v1.4}
% \author{Heiko Oberdiek\thanks
% {Please report any issues at https://github.com/ho-tex/oberdiek/issues}\\
% \xemail{heiko.oberdiek at googlemail.com}}
%
% \maketitle
%
% \begin{abstract}
% This package provides expandable arithmetic operations
% with big integers that can exceed \TeX's number limits.
% \end{abstract}
%
% \tableofcontents
%
% \section{Documentation}
%
% \subsection{Introduction}
%
% Package \xpackage{bigintcalc} defines arithmetic operations
% that deal with big integers. Big integers can be given
% either as explicit integer number or as macro code
% that expands to an explicit number. \emph{Big} means that
% there is no limit on the size of the number. Big integers
% may exceed \TeX's range limitation of -2147483647 and 2147483647.
% Only memory issues will limit the usable range.
%
% In opposite to package \xpackage{intcalc}
% unexpandable command tokens are not supported, even if
% they are valid \TeX\ numbers like count registers or
% commands created by \cs{chardef}. Nevertheless they may
% be used, if they are prefixed by \cs{number}.
%
% Also \eTeX's \cs{numexpr} expressions are not supported
% directly in the manner of package \xpackage{intcalc}.
% However they can be given if \cs{the}\cs{numexpr}
% or \cs{number}\cs{numexpr} are used.
%
% The operations have the form of macros that take one or
% two integers as parameter and return the integer result.
% The macro name is a three letter operation name prefixed
% by the package name, e.g. \cs{bigintcalcAdd}|{10}{43}| returns
% |53|.
%
% The macros are fully expandable, exactly two expansion
% steps generate the result. Therefore the operations
% may be used nearly everywhere in \TeX, even inside
% \cs{csname}, file names,  or other
% expandable contexts.
%
% \subsection{Conditions}
%
% \subsubsection{Preconditions}
%
% \begin{itemize}
% \item
%   Arguments can be anything that expands to a number that consists
%   of optional signs and digits.
% \item
%   The arguments and return values must be sound.
%   Zero as divisor or factorials of negative numbers will cause errors.
% \end{itemize}
%
% \subsubsection{Postconditions}
%
% Additional properties of the macros apart from calculating
% a correct result (of course \smiley):
% \begin{itemize}
% \item
%   The macros are fully expandable. Thus they can
%   be used inside \cs{edef}, \cs{csname},
%   for example.
% \item
%   Furthermore exactly two expansion steps calculate the result.
% \item
%   The number consists of one optional minus sign and one or more
%   digits. The first digit is larger than zero for
%   numbers that consists of more than one digit.
%
%   In short, the number format is exactly the same as
%   \cs{number} generates, but without its range limitation.
%   And the tokens (minus sign, digits)
%   have catcode 12 (other).
% \item
%   Call by value is simulated. First the arguments are
%   converted to numbers. Then these numbers are used
%   in the calculations.
%
%   Remember that arguments
%   may contain expensive macros or \eTeX\ expressions.
%   This strategy avoids multiple evaluations of such
%   arguments.
% \end{itemize}
%
% \subsection{Error handling}
%
% Some errors are detected by the macros, example: division by zero.
% In this cases an undefined control sequence is called and causes
% a TeX error message, example: \cs{BigIntCalcError:DivisionByZero}.
% The name of the control sequence contains
% the reason for the error. The \TeX\ error may be ignored.
% Then the operation returns zero as result.
% Because the macros are supposed to work in expandible contexts.
% An traditional error message, however, is not expandable and
% would break these contexts.
%
% \subsection{Operations}
%
% Some definition equations below use the function $\opInt$
% that converts a real number to an integer. The number
% is truncated that means rounding to zero:
% \begin{gather*}
%   \opInt(x) \Def
%   \begin{cases}
%     \lfloor x\rfloor & \text{if $x\geq0$}\\
%     \lceil x\rceil & \text{otherwise}
%   \end{cases}
% \end{gather*}
%
% \subsubsection{\op{Num}}
%
% \begin{declcs}{bigintcalcNum} \M{x}
% \end{declcs}
% Macro \cs{bigintcalcNum} converts its argument to a normalized integer
% number without unnecessary leading zeros or signs.
% The result matches the regular expression:
%\begin{quote}
%\begin{verbatim}
%0|-?[1-9][0-9]*
%\end{verbatim}
%\end{quote}
%
% \subsubsection{\op{Inv}, \op{Abs}, \op{Sgn}}
%
% \begin{declcs}{bigintcalcInv} \M{x}
% \end{declcs}
% Macro \cs{bigintcalcInv} switches the sign.
% \begin{gather*}
%   \opInv(x) \Def -x
% \end{gather*}
%
% \begin{declcs}{bigintcalcAbs} \M{x}
% \end{declcs}
% Macro \cs{bigintcalcAbs} returns the absolute value
% of integer \meta{x}.
% \begin{gather*}
%   \opAbs(x) \Def \vert x\vert
% \end{gather*}
%
% \begin{declcs}{bigintcalcSgn} \M{x}
% \end{declcs}
% Macro \cs{bigintcalcSgn} encodes the sign of \meta{x} as number.
% \begin{gather*}
%   \opSgn(x) \Def
%   \begin{cases}
%     -1& \text{if $x<0$}\\
%      0& \text{if $x=0$}\\
%      1& \text{if $x>0$}
%   \end{cases}
% \end{gather*}
% These return values can easily be distinguished by \cs{ifcase}:
%\begin{quote}
%\begin{verbatim}
%\ifcase\bigintcalcSgn{<x>}
%  $x=0$
%\or
%  $x>0$
%\else
%  $x<0$
%\fi
%\end{verbatim}
%\end{quote}
%
% \subsubsection{\op{Min}, \op{Max}, \op{Cmp}}
%
% \begin{declcs}{bigintcalcMin} \M{x} \M{y}
% \end{declcs}
% Macro \cs{bigintcalcMin} returns the smaller of the two integers.
% \begin{gather*}
%   \opMin(x,y) \Def
%   \begin{cases}
%     x & \text{if $x<y$}\\
%     y & \text{otherwise}
%   \end{cases}
% \end{gather*}
%
% \begin{declcs}{bigintcalcMax} \M{x} \M{y}
% \end{declcs}
% Macro \cs{bigintcalcMax} returns the larger of the two integers.
% \begin{gather*}
%   \opMax(x,y) \Def
%   \begin{cases}
%     x & \text{if $x>y$}\\
%     y & \text{otherwise}
%   \end{cases}
% \end{gather*}
%
% \begin{declcs}{bigintcalcCmp} \M{x} \M{y}
% \end{declcs}
% Macro \cs{bigintcalcCmp} encodes the comparison result as number:
% \begin{gather*}
%   \opCmp(x,y) \Def
%   \begin{cases}
%     -1 & \text{if $x<y$}\\
%      0 & \text{if $x=y$}\\
%      1 & \text{if $x>y$}
%   \end{cases}
% \end{gather*}
% These values can be distinguished by \cs{ifcase}:
%\begin{quote}
%\begin{verbatim}
%\ifcase\bigintcalcCmp{<x>}{<y>}
%  $x=y$
%\or
%  $x>y$
%\else
%  $x<y$
%\fi
%\end{verbatim}
%\end{quote}
%
% \subsubsection{\op{Odd}}
%
% \begin{declcs}{bigintcalcOdd} \M{x}
% \end{declcs}
% \begin{gather*}
%   \opOdd(x) \Def
%   \begin{cases}
%     1 & \text{if $x$ is odd}\\
%     0 & \text{if $x$ is even}
%   \end{cases}
% \end{gather*}
%
% \subsubsection{\op{Inc}, \op{Dec}, \op{Add}, \op{Sub}}
%
% \begin{declcs}{bigintcalcInc} \M{x}
% \end{declcs}
% Macro \cs{bigintcalcInc} increments \meta{x} by one.
% \begin{gather*}
%   \opInc(x) \Def x + 1
% \end{gather*}
%
% \begin{declcs}{bigintcalcDec} \M{x}
% \end{declcs}
% Macro \cs{bigintcalcDec} decrements \meta{x} by one.
% \begin{gather*}
%   \opDec(x) \Def x - 1
% \end{gather*}
%
% \begin{declcs}{bigintcalcAdd} \M{x} \M{y}
% \end{declcs}
% Macro \cs{bigintcalcAdd} adds the two numbers.
% \begin{gather*}
%   \opAdd(x, y) \Def x + y
% \end{gather*}
%
% \begin{declcs}{bigintcalcSub} \M{x} \M{y}
% \end{declcs}
% Macro \cs{bigintcalcSub} calculates the difference.
% \begin{gather*}
%   \opSub(x, y) \Def x - y
% \end{gather*}
%
% \subsubsection{\op{Shl}, \op{Shr}}
%
% \begin{declcs}{bigintcalcShl} \M{x}
% \end{declcs}
% Macro \cs{bigintcalcShl} implements shifting to the left that
% means the number is multiplied by two.
% The sign is preserved.
% \begin{gather*}
%   \opShl(x) \Def x*2
% \end{gather*}
%
% \begin{declcs}{bigintcalcShr} \M{x}
% \end{declcs}
% Macro \cs{bigintcalcShr} implements shifting to the right.
% That is equivalent to an integer division by two.
% The sign is preserved.
% \begin{gather*}
%   \opShr(x) \Def \opInt(x/2)
% \end{gather*}
%
% \subsubsection{\op{Mul}, \op{Sqr}, \op{Fac}, \op{Pow}}
%
% \begin{declcs}{bigintcalcMul} \M{x} \M{y}
% \end{declcs}
% Macro \cs{bigintcalcMul} calculates the product of
% \meta{x} and \meta{y}.
% \begin{gather*}
%   \opMul(x,y) \Def x*y
% \end{gather*}
%
% \begin{declcs}{bigintcalcSqr} \M{x}
% \end{declcs}
% Macro \cs{bigintcalcSqr} returns the square product.
% \begin{gather*}
%   \opSqr(x) \Def x^2
% \end{gather*}
%
% \begin{declcs}{bigintcalcFac} \M{x}
% \end{declcs}
% Macro \cs{bigintcalcFac} returns the factorial of \meta{x}.
% Negative numbers are not permitted.
% \begin{gather*}
%   \opFac(x) \Def x!\qquad\text{for $x\geq0$}
% \end{gather*}
% ($0! = 1$)
%
% \begin{declcs}{bigintcalcPow} M{x} M{y}
% \end{declcs}
% Macro \cs{bigintcalcPow} calculates the value of \meta{x} to the
% power of \meta{y}. The error ``division by zero'' is thrown
% if \meta{x} is zero and \meta{y} is negative.
% permitted:
% \begin{gather*}
%   \opPow(x,y) \Def
%   \opInt(x^y)\qquad\text{for $x\neq0$ or $y\geq0$}
% \end{gather*}
% ($0^0 = 1$)
%
% \subsubsection{\op{Div}, \op{Mul}}
%
% \begin{declcs}{bigintcalcDiv} \M{x} \M{y}
% \end{declcs}
% Macro \cs{bigintcalcDiv} performs an integer division.
% Argument \meta{y} must not be zero.
% \begin{gather*}
%   \opDiv(x,y) \Def \opInt(x/y)\qquad\text{for $y\neq0$}
% \end{gather*}
%
% \begin{declcs}{bigintcalcMod} \M{x} \M{y}
% \end{declcs}
% Macro \cs{bigintcalcMod} gets the remainder of the integer
% division. The sign follows the divisor \meta{y}.
% Argument \meta{y} must not be zero.
% \begin{gather*}
%   \opMod(x,y) \Def x\mathrel{\%}y\qquad\text{for $y\neq0$}
% \end{gather*}
% The result ranges:
% \begin{gather*}
%   -\vert y\vert < \opMod(x,y) \leq 0\qquad\text{for $y<0$}\\
%   0 \leq \opMod(x,y) < y\qquad\text{for $y\geq0$}
% \end{gather*}
%
% \subsection{Interface for programmers}
%
% If the programmer can ensure some more properties about
% the arguments of the operations, then the following
% macros are a little more efficient.
%
% In general numbers must obey the following constraints:
% \begin{itemize}
% \item Plain number: digit tokens only, no command tokens.
% \item Non-negative. Signs are forbidden.
% \item Delimited by exclamation mark. Curly braces
%       around the number are not allowed and will
%       break the code.
% \end{itemize}
%
% \begin{declcs}{BigIntCalcOdd} \meta{number} |!|
% \end{declcs}
% |1|/|0| is returned if \meta{number} is odd/even.
%
% \begin{declcs}{BigIntCalcInc} \meta{number} |!|
% \end{declcs}
% Incrementation.
%
% \begin{declcs}{BigIntCalcDec} \meta{number} |!|
% \end{declcs}
% Decrementation, positive number without zero.
%
% \begin{declcs}{BigIntCalcAdd} \meta{number A} |!| \meta{number B} |!|
% \end{declcs}
% Addition, $A\geq B$.
%
% \begin{declcs}{BigIntCalcSub} \meta{number A} |!| \meta{number B} |!|
% \end{declcs}
% Subtraction, $A\geq B$.
%
% \begin{declcs}{BigIntCalcShl} \meta{number} |!|
% \end{declcs}
% Left shift (multiplication with two).
%
% \begin{declcs}{BigIntCalcShr} \meta{number} |!|
% \end{declcs}
% Right shift (integer division by two).
%
% \begin{declcs}{BigIntCalcMul} \meta{number A} |!| \meta{number B} |!|
% \end{declcs}
% Multiplication, $A\geq B$.
%
% \begin{declcs}{BigIntCalcDiv} \meta{number A} |!| \meta{number B} |!|
% \end{declcs}
% Division operation.
%
% \begin{declcs}{BigIntCalcMod} \meta{number A} |!| \meta{number B} |!|
% \end{declcs}
% Modulo operation.
%
%
% \StopEventually{
% }
%
% \section{Implementation}
%
%    \begin{macrocode}
%<*package>
%    \end{macrocode}
%
% \subsection{Reload check and package identification}
%    Reload check, especially if the package is not used with \LaTeX.
%    \begin{macrocode}
\begingroup\catcode61\catcode48\catcode32=10\relax%
  \catcode13=5 % ^^M
  \endlinechar=13 %
  \catcode35=6 % #
  \catcode39=12 % '
  \catcode44=12 % ,
  \catcode45=12 % -
  \catcode46=12 % .
  \catcode58=12 % :
  \catcode64=11 % @
  \catcode123=1 % {
  \catcode125=2 % }
  \expandafter\let\expandafter\x\csname ver@bigintcalc.sty\endcsname
  \ifx\x\relax % plain-TeX, first loading
  \else
    \def\empty{}%
    \ifx\x\empty % LaTeX, first loading,
      % variable is initialized, but \ProvidesPackage not yet seen
    \else
      \expandafter\ifx\csname PackageInfo\endcsname\relax
        \def\x#1#2{%
          \immediate\write-1{Package #1 Info: #2.}%
        }%
      \else
        \def\x#1#2{\PackageInfo{#1}{#2, stopped}}%
      \fi
      \x{bigintcalc}{The package is already loaded}%
      \aftergroup\endinput
    \fi
  \fi
\endgroup%
%    \end{macrocode}
%    Package identification:
%    \begin{macrocode}
\begingroup\catcode61\catcode48\catcode32=10\relax%
  \catcode13=5 % ^^M
  \endlinechar=13 %
  \catcode35=6 % #
  \catcode39=12 % '
  \catcode40=12 % (
  \catcode41=12 % )
  \catcode44=12 % ,
  \catcode45=12 % -
  \catcode46=12 % .
  \catcode47=12 % /
  \catcode58=12 % :
  \catcode64=11 % @
  \catcode91=12 % [
  \catcode93=12 % ]
  \catcode123=1 % {
  \catcode125=2 % }
  \expandafter\ifx\csname ProvidesPackage\endcsname\relax
    \def\x#1#2#3[#4]{\endgroup
      \immediate\write-1{Package: #3 #4}%
      \xdef#1{#4}%
    }%
  \else
    \def\x#1#2[#3]{\endgroup
      #2[{#3}]%
      \ifx#1\@undefined
        \xdef#1{#3}%
      \fi
      \ifx#1\relax
        \xdef#1{#3}%
      \fi
    }%
  \fi
\expandafter\x\csname ver@bigintcalc.sty\endcsname
\ProvidesPackage{bigintcalc}%
  [2016/05/16 v1.4 Expandable calculations on big integers (HO)]%
%    \end{macrocode}
%
% \subsection{Catcodes}
%
%    \begin{macrocode}
\begingroup\catcode61\catcode48\catcode32=10\relax%
  \catcode13=5 % ^^M
  \endlinechar=13 %
  \catcode123=1 % {
  \catcode125=2 % }
  \catcode64=11 % @
  \def\x{\endgroup
    \expandafter\edef\csname BIC@AtEnd\endcsname{%
      \endlinechar=\the\endlinechar\relax
      \catcode13=\the\catcode13\relax
      \catcode32=\the\catcode32\relax
      \catcode35=\the\catcode35\relax
      \catcode61=\the\catcode61\relax
      \catcode64=\the\catcode64\relax
      \catcode123=\the\catcode123\relax
      \catcode125=\the\catcode125\relax
    }%
  }%
\x\catcode61\catcode48\catcode32=10\relax%
\catcode13=5 % ^^M
\endlinechar=13 %
\catcode35=6 % #
\catcode64=11 % @
\catcode123=1 % {
\catcode125=2 % }
\def\TMP@EnsureCode#1#2{%
  \edef\BIC@AtEnd{%
    \BIC@AtEnd
    \catcode#1=\the\catcode#1\relax
  }%
  \catcode#1=#2\relax
}
\TMP@EnsureCode{33}{12}% !
\TMP@EnsureCode{36}{14}% $ (comment!)
\TMP@EnsureCode{38}{14}% & (comment!)
\TMP@EnsureCode{40}{12}% (
\TMP@EnsureCode{41}{12}% )
\TMP@EnsureCode{42}{12}% *
\TMP@EnsureCode{43}{12}% +
\TMP@EnsureCode{45}{12}% -
\TMP@EnsureCode{46}{12}% .
\TMP@EnsureCode{47}{12}% /
\TMP@EnsureCode{58}{11}% : (letter!)
\TMP@EnsureCode{60}{12}% <
\TMP@EnsureCode{62}{12}% >
\TMP@EnsureCode{63}{14}% ? (comment!)
\TMP@EnsureCode{91}{12}% [
\TMP@EnsureCode{93}{12}% ]
\edef\BIC@AtEnd{\BIC@AtEnd\noexpand\endinput}
\begingroup\expandafter\expandafter\expandafter\endgroup
\expandafter\ifx\csname BIC@TestMode\endcsname\relax
\else
  \catcode63=9 % ? (ignore)
\fi
? \let\BIC@@TestMode\BIC@TestMode
%    \end{macrocode}
%
% \subsection{\eTeX\ detection}
%
%    \begin{macrocode}
\begingroup\expandafter\expandafter\expandafter\endgroup
\expandafter\ifx\csname numexpr\endcsname\relax
  \catcode36=9 % $ (ignore)
\else
  \catcode38=9 % & (ignore)
\fi
%    \end{macrocode}
%
% \subsection{Help macros}
%
%    \begin{macro}{\BIC@Fi}
%    \begin{macrocode}
\let\BIC@Fi\fi
%    \end{macrocode}
%    \end{macro}
%    \begin{macro}{\BIC@AfterFi}
%    \begin{macrocode}
\def\BIC@AfterFi#1#2\BIC@Fi{\fi#1}%
%    \end{macrocode}
%    \end{macro}
%    \begin{macro}{\BIC@AfterFiFi}
%    \begin{macrocode}
\def\BIC@AfterFiFi#1#2\BIC@Fi{\fi\fi#1}%
%    \end{macrocode}
%    \end{macro}
%    \begin{macro}{\BIC@AfterFiFiFi}
%    \begin{macrocode}
\def\BIC@AfterFiFiFi#1#2\BIC@Fi{\fi\fi\fi#1}%
%    \end{macrocode}
%    \end{macro}
%
%    \begin{macro}{\BIC@Space}
%    \begin{macrocode}
\begingroup
  \def\x#1{\endgroup
    \let\BIC@Space= #1%
  }%
\x{ }
%    \end{macrocode}
%    \end{macro}
%
% \subsection{Expand number}
%
%    \begin{macrocode}
\begingroup\expandafter\expandafter\expandafter\endgroup
\expandafter\ifx\csname RequirePackage\endcsname\relax
  \def\TMP@RequirePackage#1[#2]{%
    \begingroup\expandafter\expandafter\expandafter\endgroup
    \expandafter\ifx\csname ver@#1.sty\endcsname\relax
      \input #1.sty\relax
    \fi
  }%
  \TMP@RequirePackage{pdftexcmds}[2007/11/11]%
\else
  \RequirePackage{pdftexcmds}[2007/11/11]%
\fi
%    \end{macrocode}
%
%    \begin{macrocode}
\begingroup\expandafter\expandafter\expandafter\endgroup
\expandafter\ifx\csname pdf@escapehex\endcsname\relax
%    \end{macrocode}
%
%    \begin{macro}{\BIC@Expand}
%    \begin{macrocode}
  \def\BIC@Expand#1{%
    \romannumeral0%
    \BIC@@Expand#1!\@nil{}%
  }%
%    \end{macrocode}
%    \end{macro}
%    \begin{macro}{\BIC@@Expand}
%    \begin{macrocode}
  \def\BIC@@Expand#1#2\@nil#3{%
    \expandafter\ifcat\noexpand#1\relax
      \expandafter\@firstoftwo
    \else
      \expandafter\@secondoftwo
    \fi
    {%
      \expandafter\BIC@@Expand#1#2\@nil{#3}%
    }{%
      \ifx#1!%
        \expandafter\@firstoftwo
      \else
        \expandafter\@secondoftwo
      \fi
      { #3}{%
        \BIC@@Expand#2\@nil{#3#1}%
      }%
    }%
  }%
%    \end{macrocode}
%    \end{macro}
%    \begin{macro}{\@firstoftwo}
%    \begin{macrocode}
  \expandafter\ifx\csname @firstoftwo\endcsname\relax
    \long\def\@firstoftwo#1#2{#1}%
  \fi
%    \end{macrocode}
%    \end{macro}
%    \begin{macro}{\@secondoftwo}
%    \begin{macrocode}
  \expandafter\ifx\csname @secondoftwo\endcsname\relax
    \long\def\@secondoftwo#1#2{#2}%
  \fi
%    \end{macrocode}
%    \end{macro}
%
%    \begin{macrocode}
\else
%    \end{macrocode}
%    \begin{macro}{\BIC@Expand}
%    \begin{macrocode}
  \def\BIC@Expand#1{%
    \romannumeral0\expandafter\expandafter\expandafter\BIC@Space
    \pdf@unescapehex{%
      \expandafter\expandafter\expandafter
      \BIC@StripHexSpace\pdf@escapehex{#1}20\@nil
    }%
  }%
%    \end{macrocode}
%    \end{macro}
%    \begin{macro}{\BIC@StripHexSpace}
%    \begin{macrocode}
  \def\BIC@StripHexSpace#120#2\@nil{%
    #1%
    \ifx\\#2\\%
    \else
      \BIC@AfterFi{%
        \BIC@StripHexSpace#2\@nil
      }%
    \BIC@Fi
  }%
%    \end{macrocode}
%    \end{macro}
%    \begin{macrocode}
\fi
%    \end{macrocode}
%
% \subsection{Normalize expanded number}
%
%    \begin{macro}{\BIC@Normalize}
%    |#1|: result sign\\
%    |#2|: first token of number
%    \begin{macrocode}
\def\BIC@Normalize#1#2{%
  \ifx#2-%
    \ifx\\#1\\%
      \BIC@AfterFiFi{%
        \BIC@Normalize-%
      }%
    \else
      \BIC@AfterFiFi{%
        \BIC@Normalize{}%
      }%
    \fi
  \else
    \ifx#2+%
      \BIC@AfterFiFi{%
        \BIC@Normalize{#1}%
      }%
    \else
      \ifx#20%
        \BIC@AfterFiFiFi{%
          \BIC@NormalizeZero{#1}%
        }%
      \else
        \BIC@AfterFiFiFi{%
          \BIC@NormalizeDigits#1#2%
        }%
      \fi
    \fi
  \BIC@Fi
}
%    \end{macrocode}
%    \end{macro}
%    \begin{macro}{\BIC@NormalizeZero}
%    \begin{macrocode}
\def\BIC@NormalizeZero#1#2{%
  \ifx#2!%
    \BIC@AfterFi{ 0}%
  \else
    \ifx#20%
      \BIC@AfterFiFi{%
        \BIC@NormalizeZero{#1}%
      }%
    \else
      \BIC@AfterFiFi{%
        \BIC@NormalizeDigits#1#2%
      }%
    \fi
  \BIC@Fi
}
%    \end{macrocode}
%    \end{macro}
%    \begin{macro}{\BIC@NormalizeDigits}
%    \begin{macrocode}
\def\BIC@NormalizeDigits#1!{ #1}
%    \end{macrocode}
%    \end{macro}
%
% \subsection{\op{Num}}
%
%    \begin{macro}{\bigintcalcNum}
%    \begin{macrocode}
\def\bigintcalcNum#1{%
  \romannumeral0%
  \expandafter\expandafter\expandafter\BIC@Normalize
  \expandafter\expandafter\expandafter{%
  \expandafter\expandafter\expandafter}%
  \BIC@Expand{#1}!%
}
%    \end{macrocode}
%    \end{macro}
%
% \subsection{\op{Inv}, \op{Abs}, \op{Sgn}}
%
%
%    \begin{macro}{\bigintcalcInv}
%    \begin{macrocode}
\def\bigintcalcInv#1{%
  \romannumeral0\expandafter\expandafter\expandafter\BIC@Space
  \bigintcalcNum{-#1}%
}
%    \end{macrocode}
%    \end{macro}
%
%    \begin{macro}{\bigintcalcAbs}
%    \begin{macrocode}
\def\bigintcalcAbs#1{%
  \romannumeral0%
  \expandafter\expandafter\expandafter\BIC@Abs
  \bigintcalcNum{#1}%
}
%    \end{macrocode}
%    \end{macro}
%    \begin{macro}{\BIC@Abs}
%    \begin{macrocode}
\def\BIC@Abs#1{%
  \ifx#1-%
    \expandafter\BIC@Space
  \else
    \expandafter\BIC@Space
    \expandafter#1%
  \fi
}
%    \end{macrocode}
%    \end{macro}
%
%    \begin{macro}{\bigintcalcSgn}
%    \begin{macrocode}
\def\bigintcalcSgn#1{%
  \number
  \expandafter\expandafter\expandafter\BIC@Sgn
  \bigintcalcNum{#1}! %
}
%    \end{macrocode}
%    \end{macro}
%    \begin{macro}{\BIC@Sgn}
%    \begin{macrocode}
\def\BIC@Sgn#1#2!{%
  \ifx#1-%
    -1%
  \else
    \ifx#10%
      0%
    \else
      1%
    \fi
  \fi
}
%    \end{macrocode}
%    \end{macro}
%
% \subsection{\op{Cmp}, \op{Min}, \op{Max}}
%
%    \begin{macro}{\bigintcalcCmp}
%    \begin{macrocode}
\def\bigintcalcCmp#1#2{%
  \number
  \expandafter\expandafter\expandafter\BIC@Cmp
  \bigintcalcNum{#2}!{#1}%
}
%    \end{macrocode}
%    \end{macro}
%    \begin{macro}{\BIC@Cmp}
%    \begin{macrocode}
\def\BIC@Cmp#1!#2{%
  \expandafter\expandafter\expandafter\BIC@@Cmp
  \bigintcalcNum{#2}!#1!%
}
%    \end{macrocode}
%    \end{macro}
%    \begin{macro}{\BIC@@Cmp}
%    \begin{macrocode}
\def\BIC@@Cmp#1#2!#3#4!{%
  \ifx#1-%
    \ifx#3-%
      \BIC@AfterFiFi{%
        \BIC@@Cmp#4!#2!%
      }%
    \else
      \BIC@AfterFiFi{%
        -1 %
      }%
    \fi
  \else
    \ifx#3-%
      \BIC@AfterFiFi{%
        1 %
      }%
    \else
      \BIC@AfterFiFi{%
        \BIC@CmpLength#1#2!#3#4!#1#2!#3#4!%
      }%
    \fi
  \BIC@Fi
}
%    \end{macrocode}
%    \end{macro}
%    \begin{macro}{\BIC@PosCmp}
%    \begin{macrocode}
\def\BIC@PosCmp#1!#2!{%
  \BIC@CmpLength#1!#2!#1!#2!%
}
%    \end{macrocode}
%    \end{macro}
%    \begin{macro}{\BIC@CmpLength}
%    \begin{macrocode}
\def\BIC@CmpLength#1#2!#3#4!{%
  \ifx\\#2\\%
    \ifx\\#4\\%
      \BIC@AfterFiFi\BIC@CmpDiff
    \else
      \BIC@AfterFiFi{%
        \BIC@CmpResult{-1}%
      }%
    \fi
  \else
    \ifx\\#4\\%
      \BIC@AfterFiFi{%
        \BIC@CmpResult1%
      }%
    \else
      \BIC@AfterFiFi{%
        \BIC@CmpLength#2!#4!%
      }%
    \fi
  \BIC@Fi
}
%    \end{macrocode}
%    \end{macro}
%    \begin{macro}{\BIC@CmpResult}
%    \begin{macrocode}
\def\BIC@CmpResult#1#2!#3!{#1 }
%    \end{macrocode}
%    \end{macro}
%    \begin{macro}{\BIC@CmpDiff}
%    \begin{macrocode}
\def\BIC@CmpDiff#1#2!#3#4!{%
  \ifnum#1<#3 %
    \BIC@AfterFi{%
      -1 %
    }%
  \else
    \ifnum#1>#3 %
      \BIC@AfterFiFi{%
        1 %
      }%
    \else
      \ifx\\#2\\%
        \BIC@AfterFiFiFi{%
          0 %
        }%
      \else
        \BIC@AfterFiFiFi{%
          \BIC@CmpDiff#2!#4!%
        }%
      \fi
    \fi
  \BIC@Fi
}
%    \end{macrocode}
%    \end{macro}
%
%    \begin{macro}{\bigintcalcMin}
%    \begin{macrocode}
\def\bigintcalcMin#1{%
  \romannumeral0%
  \expandafter\expandafter\expandafter\BIC@MinMax
  \bigintcalcNum{#1}!-!%
}
%    \end{macrocode}
%    \end{macro}
%    \begin{macro}{\bigintcalcMax}
%    \begin{macrocode}
\def\bigintcalcMax#1{%
  \romannumeral0%
  \expandafter\expandafter\expandafter\BIC@MinMax
  \bigintcalcNum{#1}!!%
}
%    \end{macrocode}
%    \end{macro}
%    \begin{macro}{\BIC@MinMax}
%    |#1|: $x$\\
%    |#2|: sign for comparison\\
%    |#3|: $y$
%    \begin{macrocode}
\def\BIC@MinMax#1!#2!#3{%
  \expandafter\expandafter\expandafter\BIC@@MinMax
  \bigintcalcNum{#3}!#1!#2!%
}
%    \end{macrocode}
%    \end{macro}
%    \begin{macro}{\BIC@@MinMax}
%    |#1|: $y$\\
%    |#2|: $x$\\
%    |#3|: sign for comparison
%    \begin{macrocode}
\def\BIC@@MinMax#1!#2!#3!{%
  \ifnum\BIC@@Cmp#1!#2!=#31 %
    \BIC@AfterFi{ #1}%
  \else
    \BIC@AfterFi{ #2}%
  \BIC@Fi
}
%    \end{macrocode}
%    \end{macro}
%
% \subsection{\op{Odd}}
%
%    \begin{macro}{\bigintcalcOdd}
%    \begin{macrocode}
\def\bigintcalcOdd#1{%
  \romannumeral0%
  \expandafter\expandafter\expandafter\BIC@Odd
  \bigintcalcAbs{#1}!%
}
%    \end{macrocode}
%    \end{macro}
%    \begin{macro}{\BigIntCalcOdd}
%    \begin{macrocode}
\def\BigIntCalcOdd#1!{%
  \romannumeral0%
  \BIC@Odd#1!%
}
%    \end{macrocode}
%    \end{macro}
%    \begin{macro}{\BIC@Odd}
%    |#1|: $x$
%    \begin{macrocode}
\def\BIC@Odd#1#2{%
  \ifx#2!%
    \ifodd#1 %
      \BIC@AfterFiFi{ 1}%
    \else
      \BIC@AfterFiFi{ 0}%
    \fi
  \else
    \expandafter\BIC@Odd\expandafter#2%
  \BIC@Fi
}
%    \end{macrocode}
%    \end{macro}
%
% \subsection{\op{Inc}, \op{Dec}}
%
%    \begin{macro}{\bigintcalcInc}
%    \begin{macrocode}
\def\bigintcalcInc#1{%
  \romannumeral0%
  \expandafter\expandafter\expandafter\BIC@IncSwitch
  \bigintcalcNum{#1}!%
}
%    \end{macrocode}
%    \end{macro}
%    \begin{macro}{\BIC@IncSwitch}
%    \begin{macrocode}
\def\BIC@IncSwitch#1#2!{%
  \ifcase\BIC@@Cmp#1#2!-1!%
    \BIC@AfterFi{ 0}%
  \or
    \BIC@AfterFi{%
      \BIC@Inc#1#2!{}%
    }%
  \else
    \BIC@AfterFi{%
      \expandafter-\romannumeral0%
      \BIC@Dec#2!{}%
    }%
  \BIC@Fi
}
%    \end{macrocode}
%    \end{macro}
%
%    \begin{macro}{\bigintcalcDec}
%    \begin{macrocode}
\def\bigintcalcDec#1{%
  \romannumeral0%
  \expandafter\expandafter\expandafter\BIC@DecSwitch
  \bigintcalcNum{#1}!%
}
%    \end{macrocode}
%    \end{macro}
%    \begin{macro}{\BIC@DecSwitch}
%    \begin{macrocode}
\def\BIC@DecSwitch#1#2!{%
  \ifcase\BIC@Sgn#1#2! %
    \BIC@AfterFi{ -1}%
  \or
    \BIC@AfterFi{%
      \BIC@Dec#1#2!{}%
    }%
  \else
    \BIC@AfterFi{%
      \expandafter-\romannumeral0%
      \BIC@Inc#2!{}%
    }%
  \BIC@Fi
}
%    \end{macrocode}
%    \end{macro}
%
%    \begin{macro}{\BigIntCalcInc}
%    \begin{macrocode}
\def\BigIntCalcInc#1!{%
  \romannumeral0\BIC@Inc#1!{}%
}
%    \end{macrocode}
%    \end{macro}
%    \begin{macro}{\BigIntCalcDec}
%    \begin{macrocode}
\def\BigIntCalcDec#1!{%
  \romannumeral0\BIC@Dec#1!{}%
}
%    \end{macrocode}
%    \end{macro}
%
%    \begin{macro}{\BIC@Inc}
%    \begin{macrocode}
\def\BIC@Inc#1#2!#3{%
  \ifx\\#2\\%
    \BIC@AfterFi{%
      \BIC@@Inc1#1#3!{}%
    }%
  \else
    \BIC@AfterFi{%
      \BIC@Inc#2!{#1#3}%
    }%
  \BIC@Fi
}
%    \end{macrocode}
%    \end{macro}
%    \begin{macro}{\BIC@@Inc}
%    \begin{macrocode}
\def\BIC@@Inc#1#2#3!#4{%
  \ifcase#1 %
    \ifx\\#3\\%
      \BIC@AfterFiFi{ #2#4}%
    \else
      \BIC@AfterFiFi{%
        \BIC@@Inc0#3!{#2#4}%
      }%
    \fi
  \else
    \ifnum#2<9 %
      \BIC@AfterFiFi{%
&       \expandafter\BIC@@@Inc\the\numexpr#2+1\relax
$       \expandafter\expandafter\expandafter\BIC@@@Inc
$       \ifcase#2 \expandafter1%
$       \or\expandafter2%
$       \or\expandafter3%
$       \or\expandafter4%
$       \or\expandafter5%
$       \or\expandafter6%
$       \or\expandafter7%
$       \or\expandafter8%
$       \or\expandafter9%
$?      \else\BigIntCalcError:ThisCannotHappen%
$       \fi
        0#3!{#4}%
      }%
    \else
      \BIC@AfterFiFi{%
        \BIC@@@Inc01#3!{#4}%
      }%
    \fi
  \BIC@Fi
}
%    \end{macrocode}
%    \end{macro}
%    \begin{macro}{\BIC@@@Inc}
%    \begin{macrocode}
\def\BIC@@@Inc#1#2#3!#4{%
  \ifx\\#3\\%
    \ifnum#2=1 %
      \BIC@AfterFiFi{ 1#1#4}%
    \else
      \BIC@AfterFiFi{ #1#4}%
    \fi
  \else
    \BIC@AfterFi{%
      \BIC@@Inc#2#3!{#1#4}%
    }%
  \BIC@Fi
}
%    \end{macrocode}
%    \end{macro}
%
%    \begin{macro}{\BIC@Dec}
%    \begin{macrocode}
\def\BIC@Dec#1#2!#3{%
  \ifx\\#2\\%
    \BIC@AfterFi{%
      \BIC@@Dec1#1#3!{}%
    }%
  \else
    \BIC@AfterFi{%
      \BIC@Dec#2!{#1#3}%
    }%
  \BIC@Fi
}
%    \end{macrocode}
%    \end{macro}
%    \begin{macro}{\BIC@@Dec}
%    \begin{macrocode}
\def\BIC@@Dec#1#2#3!#4{%
  \ifcase#1 %
    \ifx\\#3\\%
      \BIC@AfterFiFi{ #2#4}%
    \else
      \BIC@AfterFiFi{%
        \BIC@@Dec0#3!{#2#4}%
      }%
    \fi
  \else
    \ifnum#2>0 %
      \BIC@AfterFiFi{%
&       \expandafter\BIC@@@Dec\the\numexpr#2-1\relax
$       \expandafter\expandafter\expandafter\BIC@@@Dec
$       \ifcase#2
$?        \BigIntCalcError:ThisCannotHappen%
$       \or\expandafter0%
$       \or\expandafter1%
$       \or\expandafter2%
$       \or\expandafter3%
$       \or\expandafter4%
$       \or\expandafter5%
$       \or\expandafter6%
$       \or\expandafter7%
$       \or\expandafter8%
$?      \else\BigIntCalcError:ThisCannotHappen%
$       \fi
        0#3!{#4}%
      }%
    \else
      \BIC@AfterFiFi{%
        \BIC@@@Dec91#3!{#4}%
      }%
    \fi
  \BIC@Fi
}
%    \end{macrocode}
%    \end{macro}
%    \begin{macro}{\BIC@@@Dec}
%    \begin{macrocode}
\def\BIC@@@Dec#1#2#3!#4{%
  \ifx\\#3\\%
    \ifcase#1 %
      \ifx\\#4\\%
        \BIC@AfterFiFiFi{ 0}%
      \else
        \BIC@AfterFiFiFi{ #4}%
      \fi
    \else
      \BIC@AfterFiFi{ #1#4}%
    \fi
  \else
    \BIC@AfterFi{%
      \BIC@@Dec#2#3!{#1#4}%
    }%
  \BIC@Fi
}
%    \end{macrocode}
%    \end{macro}
%
% \subsection{\op{Add}, \op{Sub}}
%
%    \begin{macro}{\bigintcalcAdd}
%    \begin{macrocode}
\def\bigintcalcAdd#1{%
  \romannumeral0%
  \expandafter\expandafter\expandafter\BIC@Add
  \bigintcalcNum{#1}!%
}
%    \end{macrocode}
%    \end{macro}
%    \begin{macro}{\BIC@Add}
%    \begin{macrocode}
\def\BIC@Add#1!#2{%
  \expandafter\expandafter\expandafter
  \BIC@AddSwitch\bigintcalcNum{#2}!#1!%
}
%    \end{macrocode}
%    \end{macro}
%    \begin{macro}{\bigintcalcSub}
%    \begin{macrocode}
\def\bigintcalcSub#1#2{%
  \romannumeral0%
  \expandafter\expandafter\expandafter\BIC@Add
  \bigintcalcNum{-#2}!{#1}%
}
%    \end{macrocode}
%    \end{macro}
%
%    \begin{macro}{\BIC@AddSwitch}
%    Decision table for \cs{BIC@AddSwitch}.
%    \begin{quote}
%    \begin{tabular}[t]{@{}|l|l|l|l|l|@{}}
%      \hline
%      $x<0$ & $y<0$ & $-x>-y$ & $-$ & $\opAdd(-x,-y)$\\
%      \cline{3-3}\cline{5-5}
%            &       & else    &     & $\opAdd(-y,-x)$\\
%      \cline{2-5}
%            & else  & $-x> y$ & $-$ & $\opSub(-x, y)$\\
%      \cline{3-5}
%            &       & $-x= y$ &     & $0$\\
%      \cline{3-5}
%            &       & else    & $+$ & $\opSub( y,-x)$\\
%      \hline
%      else  & $y<0$ & $ x>-y$ & $+$ & $\opSub( x,-y)$\\
%      \cline{3-5}
%            &       & $ x=-y$ &     & $0$\\
%      \cline{3-5}
%            &       & else    & $-$ & $\opSub(-y, x)$\\
%      \cline{2-5}
%            & else  & $ x> y$ & $+$ & $\opAdd( x, y)$\\
%      \cline{3-3}\cline{5-5}
%            &       & else    &     & $\opAdd( y, x)$\\
%      \hline
%    \end{tabular}
%    \end{quote}
%    \begin{macrocode}
\def\BIC@AddSwitch#1#2!#3#4!{%
  \ifx#1-% x < 0
    \ifx#3-% y < 0
      \expandafter-\romannumeral0%
      \ifnum\BIC@PosCmp#2!#4!=1 % -x > -y
        \BIC@AfterFiFiFi{%
          \BIC@AddXY#2!#4!!!%
        }%
      \else % -x <= -y
        \BIC@AfterFiFiFi{%
          \BIC@AddXY#4!#2!!!%
        }%
      \fi
    \else % y >= 0
      \ifcase\BIC@PosCmp#2!#3#4!% -x = y
        \BIC@AfterFiFiFi{ 0}%
      \or % -x > y
        \expandafter-\romannumeral0%
        \BIC@AfterFiFiFi{%
          \BIC@SubXY#2!#3#4!!!%
        }%
      \else % -x <= y
        \BIC@AfterFiFiFi{%
          \BIC@SubXY#3#4!#2!!!%
        }%
      \fi
    \fi
  \else % x >= 0
    \ifx#3-% y < 0
      \ifcase\BIC@PosCmp#1#2!#4!% x = -y
        \BIC@AfterFiFiFi{ 0}%
      \or % x > -y
        \BIC@AfterFiFiFi{%
          \BIC@SubXY#1#2!#4!!!%
        }%
      \else % x <= -y
        \expandafter-\romannumeral0%
        \BIC@AfterFiFiFi{%
          \BIC@SubXY#4!#1#2!!!%
        }%
      \fi
    \else % y >= 0
      \ifnum\BIC@PosCmp#1#2!#3#4!=1 % x > y
        \BIC@AfterFiFiFi{%
          \BIC@AddXY#1#2!#3#4!!!%
        }%
      \else % x <= y
        \BIC@AfterFiFiFi{%
          \BIC@AddXY#3#4!#1#2!!!%
        }%
      \fi
    \fi
  \BIC@Fi
}
%    \end{macrocode}
%    \end{macro}
%
%    \begin{macro}{\BigIntCalcAdd}
%    \begin{macrocode}
\def\BigIntCalcAdd#1!#2!{%
  \romannumeral0\BIC@AddXY#1!#2!!!%
}
%    \end{macrocode}
%    \end{macro}
%    \begin{macro}{\BigIntCalcSub}
%    \begin{macrocode}
\def\BigIntCalcSub#1!#2!{%
  \romannumeral0\BIC@SubXY#1!#2!!!%
}
%    \end{macrocode}
%    \end{macro}
%
%    \begin{macro}{\BIC@AddXY}
%    \begin{macrocode}
\def\BIC@AddXY#1#2!#3#4!#5!#6!{%
  \ifx\\#2\\%
    \ifx\\#3\\%
      \BIC@AfterFiFi{%
        \BIC@DoAdd0!#1#5!#60!%
      }%
    \else
      \BIC@AfterFiFi{%
        \BIC@DoAdd0!#1#5!#3#6!%
      }%
    \fi
  \else
    \ifx\\#4\\%
      \ifx\\#3\\%
        \BIC@AfterFiFiFi{%
          \BIC@AddXY#2!{}!#1#5!#60!%
        }%
      \else
        \BIC@AfterFiFiFi{%
          \BIC@AddXY#2!{}!#1#5!#3#6!%
        }%
      \fi
    \else
      \BIC@AfterFiFi{%
        \BIC@AddXY#2!#4!#1#5!#3#6!%
      }%
    \fi
  \BIC@Fi
}
%    \end{macrocode}
%    \end{macro}
%    \begin{macro}{\BIC@DoAdd}
%    |#1|: carry\\
%    |#2|: reverted result\\
%    |#3#4|: reverted $x$\\
%    |#5#6|: reverted $y$
%    \begin{macrocode}
\def\BIC@DoAdd#1#2!#3#4!#5#6!{%
  \ifx\\#4\\%
    \BIC@AfterFi{%
&     \expandafter\BIC@Space
&     \the\numexpr#1+#3+#5\relax#2%
$     \expandafter\expandafter\expandafter\BIC@AddResult
$     \BIC@AddDigit#1#3#5#2%
    }%
  \else
    \BIC@AfterFi{%
      \expandafter\expandafter\expandafter\BIC@DoAdd
      \BIC@AddDigit#1#3#5#2!#4!#6!%
    }%
  \BIC@Fi
}
%    \end{macrocode}
%    \end{macro}
%    \begin{macro}{\BIC@AddResult}
%    \begin{macrocode}
$ \def\BIC@AddResult#1{%
$   \ifx#10%
$     \expandafter\BIC@Space
$   \else
$     \expandafter\BIC@Space\expandafter#1%
$   \fi
$ }%
%    \end{macrocode}
%    \end{macro}
%    \begin{macro}{\BIC@AddDigit}
%    |#1|: carry\\
%    |#2|: digit of $x$\\
%    |#3|: digit of $y$
%    \begin{macrocode}
\def\BIC@AddDigit#1#2#3{%
  \romannumeral0%
& \expandafter\BIC@@AddDigit\the\numexpr#1+#2+#3!%
$ \expandafter\BIC@@AddDigit\number%
$ \csname
$   BIC@AddCarry%
$   \ifcase#1 %
$     #2%
$   \else
$     \ifcase#2 1\or2\or3\or4\or5\or6\or7\or8\or9\or10\fi
$   \fi
$ \endcsname#3!%
}
%    \end{macrocode}
%    \end{macro}
%    \begin{macro}{\BIC@@AddDigit}
%    \begin{macrocode}
\def\BIC@@AddDigit#1!{%
  \ifnum#1<10 %
    \BIC@AfterFi{ 0#1}%
  \else
    \BIC@AfterFi{ #1}%
  \BIC@Fi
}
%    \end{macrocode}
%    \end{macro}
%    \begin{macro}{\BIC@AddCarry0}
%    \begin{macrocode}
$ \expandafter\def\csname BIC@AddCarry0\endcsname#1{#1}%
%    \end{macrocode}
%    \end{macro}
%    \begin{macro}{\BIC@AddCarry10}
%    \begin{macrocode}
$ \expandafter\def\csname BIC@AddCarry10\endcsname#1{1#1}%
%    \end{macrocode}
%    \end{macro}
%    \begin{macro}{\BIC@AddCarry[1-9]}
%    \begin{macrocode}
$ \def\BIC@Temp#1#2{%
$   \expandafter\def\csname BIC@AddCarry#1\endcsname##1{%
$     \ifcase##1 #1\or
$     #2%
$?    \else\BigIntCalcError:ThisCannotHappen%
$     \fi
$   }%
$ }%
$ \BIC@Temp 0{1\or2\or3\or4\or5\or6\or7\or8\or9}%
$ \BIC@Temp 1{2\or3\or4\or5\or6\or7\or8\or9\or10}%
$ \BIC@Temp 2{3\or4\or5\or6\or7\or8\or9\or10\or11}%
$ \BIC@Temp 3{4\or5\or6\or7\or8\or9\or10\or11\or12}%
$ \BIC@Temp 4{5\or6\or7\or8\or9\or10\or11\or12\or13}%
$ \BIC@Temp 5{6\or7\or8\or9\or10\or11\or12\or13\or14}%
$ \BIC@Temp 6{7\or8\or9\or10\or11\or12\or13\or14\or15}%
$ \BIC@Temp 7{8\or9\or10\or11\or12\or13\or14\or15\or16}%
$ \BIC@Temp 8{9\or10\or11\or12\or13\or14\or15\or16\or17}%
$ \BIC@Temp 9{10\or11\or12\or13\or14\or15\or16\or17\or18}%
%    \end{macrocode}
%    \end{macro}
%
%    \begin{macro}{\BIC@SubXY}
%    Preconditions:
%    \begin{itemize}
%    \item $x > y$, $x \geq 0$, and $y >= 0$
%    \item digits($x$) = digits($y$)
%    \end{itemize}
%    \begin{macrocode}
\def\BIC@SubXY#1#2!#3#4!#5!#6!{%
  \ifx\\#2\\%
    \ifx\\#3\\%
      \BIC@AfterFiFi{%
        \BIC@DoSub0!#1#5!#60!%
      }%
    \else
      \BIC@AfterFiFi{%
        \BIC@DoSub0!#1#5!#3#6!%
      }%
    \fi
  \else
    \ifx\\#4\\%
      \ifx\\#3\\%
        \BIC@AfterFiFiFi{%
          \BIC@SubXY#2!{}!#1#5!#60!%
        }%
      \else
        \BIC@AfterFiFiFi{%
          \BIC@SubXY#2!{}!#1#5!#3#6!%
        }%
      \fi
    \else
      \BIC@AfterFiFi{%
        \BIC@SubXY#2!#4!#1#5!#3#6!%
      }%
    \fi
  \BIC@Fi
}
%    \end{macrocode}
%    \end{macro}
%    \begin{macro}{\BIC@DoSub}
%    |#1|: carry\\
%    |#2|: reverted result\\
%    |#3#4|: reverted x\\
%    |#5#6|: reverted y
%    \begin{macrocode}
\def\BIC@DoSub#1#2!#3#4!#5#6!{%
  \ifx\\#4\\%
    \BIC@AfterFi{%
      \expandafter\expandafter\expandafter\BIC@SubResult
      \BIC@SubDigit#1#3#5#2%
    }%
  \else
    \BIC@AfterFi{%
      \expandafter\expandafter\expandafter\BIC@DoSub
      \BIC@SubDigit#1#3#5#2!#4!#6!%
    }%
  \BIC@Fi
}
%    \end{macrocode}
%    \end{macro}
%    \begin{macro}{\BIC@SubResult}
%    \begin{macrocode}
\def\BIC@SubResult#1{%
  \ifx#10%
    \expandafter\BIC@SubResult
  \else
    \expandafter\BIC@Space\expandafter#1%
  \fi
}
%    \end{macrocode}
%    \end{macro}
%    \begin{macro}{\BIC@SubDigit}
%    |#1|: carry\\
%    |#2|: digit of $x$\\
%    |#3|: digit of $y$
%    \begin{macrocode}
\def\BIC@SubDigit#1#2#3{%
  \romannumeral0%
& \expandafter\BIC@@SubDigit\the\numexpr#2-#3-#1!%
$ \expandafter\BIC@@AddDigit\number
$   \csname
$     BIC@SubCarry%
$     \ifcase#1 %
$       #3%
$     \else
$       \ifcase#3 1\or2\or3\or4\or5\or6\or7\or8\or9\or10\fi
$     \fi
$   \endcsname#2!%
}
%    \end{macrocode}
%    \end{macro}
%    \begin{macro}{\BIC@@SubDigit}
%    \begin{macrocode}
& \def\BIC@@SubDigit#1!{%
&   \ifnum#1<0 %
&     \BIC@AfterFi{%
&       \expandafter\BIC@Space
&       \expandafter1\the\numexpr#1+10\relax
&     }%
&   \else
&     \BIC@AfterFi{ 0#1}%
&   \BIC@Fi
& }%
%    \end{macrocode}
%    \end{macro}
%    \begin{macro}{\BIC@SubCarry0}
%    \begin{macrocode}
$ \expandafter\def\csname BIC@SubCarry0\endcsname#1{#1}%
%    \end{macrocode}
%    \end{macro}
%    \begin{macro}{\BIC@SubCarry10}%
%    \begin{macrocode}
$ \expandafter\def\csname BIC@SubCarry10\endcsname#1{1#1}%
%    \end{macrocode}
%    \end{macro}
%    \begin{macro}{\BIC@SubCarry[1-9]}
%    \begin{macrocode}
$ \def\BIC@Temp#1#2{%
$   \expandafter\def\csname BIC@SubCarry#1\endcsname##1{%
$     \ifcase##1 #2%
$?    \else\BigIntCalcError:ThisCannotHappen%
$     \fi
$   }%
$ }%
$ \BIC@Temp 1{19\or0\or1\or2\or3\or4\or5\or6\or7\or8}%
$ \BIC@Temp 2{18\or19\or0\or1\or2\or3\or4\or5\or6\or7}%
$ \BIC@Temp 3{17\or18\or19\or0\or1\or2\or3\or4\or5\or6}%
$ \BIC@Temp 4{16\or17\or18\or19\or0\or1\or2\or3\or4\or5}%
$ \BIC@Temp 5{15\or16\or17\or18\or19\or0\or1\or2\or3\or4}%
$ \BIC@Temp 6{14\or15\or16\or17\or18\or19\or0\or1\or2\or3}%
$ \BIC@Temp 7{13\or14\or15\or16\or17\or18\or19\or0\or1\or2}%
$ \BIC@Temp 8{12\or13\or14\or15\or16\or17\or18\or19\or0\or1}%
$ \BIC@Temp 9{11\or12\or13\or14\or15\or16\or17\or18\or19\or0}%
%    \end{macrocode}
%    \end{macro}
%
% \subsection{\op{Shl}, \op{Shr}}
%
%    \begin{macro}{\bigintcalcShl}
%    \begin{macrocode}
\def\bigintcalcShl#1{%
  \romannumeral0%
  \expandafter\expandafter\expandafter\BIC@Shl
  \bigintcalcNum{#1}!%
}
%    \end{macrocode}
%    \end{macro}
%    \begin{macro}{\BIC@Shl}
%    \begin{macrocode}
\def\BIC@Shl#1#2!{%
  \ifx#1-%
    \BIC@AfterFi{%
      \expandafter-\romannumeral0%
&     \BIC@@Shl#2!!%
$     \BIC@AddXY#2!#2!!!%
    }%
  \else
    \BIC@AfterFi{%
&     \BIC@@Shl#1#2!!%
$     \BIC@AddXY#1#2!#1#2!!!%
    }%
  \BIC@Fi
}
%    \end{macrocode}
%    \end{macro}
%    \begin{macro}{\BigIntCalcShl}
%    \begin{macrocode}
\def\BigIntCalcShl#1!{%
  \romannumeral0%
& \BIC@@Shl#1!!%
$ \BIC@AddXY#1!#1!!!%
}
%    \end{macrocode}
%    \end{macro}
%    \begin{macro}{\BIC@@Shl}
%    \begin{macrocode}
& \def\BIC@@Shl#1#2!{%
&   \ifx\\#2\\%
&     \BIC@AfterFi{%
&       \BIC@@@Shl0!#1%
&     }%
&   \else
&     \BIC@AfterFi{%
&       \BIC@@Shl#2!#1%
&     }%
&   \BIC@Fi
& }%
%    \end{macrocode}
%    \end{macro}
%    \begin{macro}{\BIC@@@Shl}
%    |#1|: carry\\
%    |#2|: result\\
%    |#3#4|: reverted number
%    \begin{macrocode}
& \def\BIC@@@Shl#1#2!#3#4!{%
&   \ifx\\#4\\%
&     \BIC@AfterFi{%
&       \expandafter\BIC@Space
&       \the\numexpr#3*2+#1\relax#2%
&     }%
&   \else
&     \BIC@AfterFi{%
&       \expandafter\BIC@@@@Shl\the\numexpr#3*2+#1!#2!#4!%
&     }%
&   \BIC@Fi
& }%
%    \end{macrocode}
%    \end{macro}
%    \begin{macro}{\BIC@@@@Shl}
%    \begin{macrocode}
& \def\BIC@@@@Shl#1!{%
&   \ifnum#1<10 %
&     \BIC@AfterFi{%
&       \BIC@@@Shl0#1%
&     }%
&   \else
&     \BIC@AfterFi{%
&       \BIC@@@Shl#1%
&     }%
&   \BIC@Fi
& }%
%    \end{macrocode}
%    \end{macro}
%
%    \begin{macro}{\bigintcalcShr}
%    \begin{macrocode}
\def\bigintcalcShr#1{%
  \romannumeral0%
  \expandafter\expandafter\expandafter\BIC@Shr
  \bigintcalcNum{#1}!%
}
%    \end{macrocode}
%    \end{macro}
%    \begin{macro}{\BIC@Shr}
%    \begin{macrocode}
\def\BIC@Shr#1#2!{%
  \ifx#1-%
    \expandafter-\romannumeral0%
    \BIC@AfterFi{%
      \BIC@@Shr#2!%
    }%
  \else
    \BIC@AfterFi{%
      \BIC@@Shr#1#2!%
    }%
  \BIC@Fi
}
%    \end{macrocode}
%    \end{macro}
%    \begin{macro}{\BigIntCalcShr}
%    \begin{macrocode}
\def\BigIntCalcShr#1!{%
  \romannumeral0%
  \BIC@@Shr#1!%
}
%    \end{macrocode}
%    \end{macro}
%    \begin{macro}{\BIC@@Shr}
%    \begin{macrocode}
\def\BIC@@Shr#1#2!{%
  \ifcase#1 %
    \BIC@AfterFi{ 0}%
  \or
    \ifx\\#2\\%
      \BIC@AfterFiFi{ 0}%
    \else
      \BIC@AfterFiFi{%
        \BIC@@@Shr#1#2!!%
      }%
    \fi
  \else
    \BIC@AfterFi{%
      \BIC@@@Shr0#1#2!!%
    }%
  \BIC@Fi
}
%    \end{macrocode}
%    \end{macro}
%    \begin{macro}{\BIC@@@Shr}
%    |#1|: carry\\
%    |#2#3|: number\\
%    |#4|: result
%    \begin{macrocode}
\def\BIC@@@Shr#1#2#3!#4!{%
  \ifx\\#3\\%
    \ifodd#1#2 %
      \BIC@AfterFiFi{%
&       \expandafter\BIC@ShrResult\the\numexpr(#1#2-1)/2\relax
$       \expandafter\expandafter\expandafter\BIC@ShrResult
$       \csname BIC@ShrDigit#1#2\endcsname
        #4!%
      }%
    \else
      \BIC@AfterFiFi{%
&       \expandafter\BIC@ShrResult\the\numexpr#1#2/2\relax
$       \expandafter\expandafter\expandafter\BIC@ShrResult
$       \csname BIC@ShrDigit#1#2\endcsname
        #4!%
      }%
    \fi
  \else
    \ifodd#1#2 %
      \BIC@AfterFiFi{%
&       \expandafter\BIC@@@@Shr\the\numexpr(#1#2-1)/2\relax1%
$       \expandafter\expandafter\expandafter\BIC@@@@Shr
$       \csname BIC@ShrDigit#1#2\endcsname
        #3!#4!%
      }%
    \else
      \BIC@AfterFiFi{%
&       \expandafter\BIC@@@@Shr\the\numexpr#1#2/2\relax0%
$       \expandafter\expandafter\expandafter\BIC@@@@Shr
$       \csname BIC@ShrDigit#1#2\endcsname
        #3!#4!%
      }%
    \fi
  \BIC@Fi
}
%    \end{macrocode}
%    \end{macro}
%    \begin{macro}{\BIC@ShrResult}
%    \begin{macrocode}
& \def\BIC@ShrResult#1#2!{ #2#1}%
$ \def\BIC@ShrResult#1#2#3!{ #3#1}%
%    \end{macrocode}
%    \end{macro}
%    \begin{macro}{\BIC@@@@Shr}
%    |#1|: new digit\\
%    |#2|: carry\\
%    |#3|: remaining number\\
%    |#4|: result
%    \begin{macrocode}
\def\BIC@@@@Shr#1#2#3!#4!{%
  \BIC@@@Shr#2#3!#4#1!%
}
%    \end{macrocode}
%    \end{macro}
%    \begin{macro}{\BIC@ShrDigit[00-19]}
%    \begin{macrocode}
$ \def\BIC@Temp#1#2#3#4{%
$   \expandafter\def\csname BIC@ShrDigit#1#2\endcsname{#3#4}%
$ }%
$ \BIC@Temp 0000%
$ \BIC@Temp 0101%
$ \BIC@Temp 0210%
$ \BIC@Temp 0311%
$ \BIC@Temp 0420%
$ \BIC@Temp 0521%
$ \BIC@Temp 0630%
$ \BIC@Temp 0731%
$ \BIC@Temp 0840%
$ \BIC@Temp 0941%
$ \BIC@Temp 1050%
$ \BIC@Temp 1151%
$ \BIC@Temp 1260%
$ \BIC@Temp 1361%
$ \BIC@Temp 1470%
$ \BIC@Temp 1571%
$ \BIC@Temp 1680%
$ \BIC@Temp 1781%
$ \BIC@Temp 1890%
$ \BIC@Temp 1991%
%    \end{macrocode}
%    \end{macro}
%
% \subsection{\cs{BIC@Tim}}
%
%    \begin{macro}{\BIC@Tim}
%    Macro \cs{BIC@Tim} implements ``Number \emph{tim}es digit''.\\
%    |#1|: plain number without sign\\
%    |#2|: digit
\def\BIC@Tim#1!#2{%
  \romannumeral0%
  \ifcase#2 % 0
    \BIC@AfterFi{ 0}%
  \or % 1
    \BIC@AfterFi{ #1}%
  \or % 2
    \BIC@AfterFi{%
      \BIC@Shl#1!%
    }%
  \else % 3-9
    \BIC@AfterFi{%
      \BIC@@Tim#1!!#2%
    }%
  \BIC@Fi
}
%    \begin{macrocode}
%    \end{macrocode}
%    \end{macro}
%    \begin{macro}{\BIC@@Tim}
%    |#1#2|: number\\
%    |#3|: reverted number
%    \begin{macrocode}
\def\BIC@@Tim#1#2!{%
  \ifx\\#2\\%
    \BIC@AfterFi{%
      \BIC@ProcessTim0!#1%
    }%
  \else
    \BIC@AfterFi{%
      \BIC@@Tim#2!#1%
    }%
  \BIC@Fi
}
%    \end{macrocode}
%    \end{macro}
%    \begin{macro}{\BIC@ProcessTim}
%    |#1|: carry\\
%    |#2|: result\\
%    |#3#4|: reverted number\\
%    |#5|: digit
%    \begin{macrocode}
\def\BIC@ProcessTim#1#2!#3#4!#5{%
  \ifx\\#4\\%
    \BIC@AfterFi{%
      \expandafter\BIC@Space
&     \the\numexpr#3*#5+#1\relax
$     \romannumeral0\BIC@TimDigit#3#5#1%
      #2%
    }%
  \else
    \BIC@AfterFi{%
      \expandafter\BIC@@ProcessTim
&     \the\numexpr#3*#5+#1%
$     \romannumeral0\BIC@TimDigit#3#5#1%
      !#2!#4!#5%
    }%
  \BIC@Fi
}
%    \end{macrocode}
%    \end{macro}
%    \begin{macro}{\BIC@@ProcessTim}
%    |#1#2|: carry?, new digit\\
%    |#3|: new number\\
%    |#4|: old number\\
%    |#5|: digit
%    \begin{macrocode}
\def\BIC@@ProcessTim#1#2!{%
  \ifx\\#2\\%
    \BIC@AfterFi{%
      \BIC@ProcessTim0#1%
    }%
  \else
    \BIC@AfterFi{%
      \BIC@ProcessTim#1#2%
    }%
  \BIC@Fi
}
%    \end{macrocode}
%    \end{macro}
%    \begin{macro}{\BIC@TimDigit}
%    |#1|: digit 0--9\\
%    |#2|: digit 3--9\\
%    |#3|: carry 0--9
%    \begin{macrocode}
$ \def\BIC@TimDigit#1#2#3{%
$   \ifcase#1 % 0
$     \BIC@AfterFi{ #3}%
$   \or % 1
$     \BIC@AfterFi{%
$       \expandafter\BIC@Space
$       \number\csname BIC@AddCarry#2\endcsname#3 %
$     }%
$   \else
$     \ifcase#3 %
$       \BIC@AfterFiFi{%
$         \expandafter\BIC@Space
$         \number\csname BIC@MulDigit#2\endcsname#1 %
$       }%
$     \else
$       \BIC@AfterFiFi{%
$         \expandafter\BIC@Space
$         \romannumeral0%
$         \expandafter\BIC@AddXY
$         \number\csname BIC@MulDigit#2\endcsname#1!%
$         #3!!!%
$       }%
$     \fi
$   \BIC@Fi
$ }%
%    \end{macrocode}
%    \end{macro}
%    \begin{macro}{\BIC@MulDigit[3-9]}
%    \begin{macrocode}
$ \def\BIC@Temp#1#2{%
$   \expandafter\def\csname BIC@MulDigit#1\endcsname##1{%
$     \ifcase##1 0%
$     \or ##1%
$     \or #2%
$?    \else\BigIntCalcError:ThisCannotHappen%
$     \fi
$   }%
$ }%
$ \BIC@Temp 3{6\or9\or12\or15\or18\or21\or24\or27}%
$ \BIC@Temp 4{8\or12\or16\or20\or24\or28\or32\or36}%
$ \BIC@Temp 5{10\or15\or20\or25\or30\or35\or40\or45}%
$ \BIC@Temp 6{12\or18\or24\or30\or36\or42\or48\or54}%
$ \BIC@Temp 7{14\or21\or28\or35\or42\or49\or56\or63}%
$ \BIC@Temp 8{16\or24\or32\or40\or48\or56\or64\or72}%
$ \BIC@Temp 9{18\or27\or36\or45\or54\or63\or72\or81}%
%    \end{macrocode}
%    \end{macro}
%
% \subsection{\op{Mul}}
%
%    \begin{macro}{\bigintcalcMul}
%    \begin{macrocode}
\def\bigintcalcMul#1#2{%
  \romannumeral0%
  \expandafter\expandafter\expandafter\BIC@Mul
  \bigintcalcNum{#1}!{#2}%
}
%    \end{macrocode}
%    \end{macro}
%    \begin{macro}{\BIC@Mul}
%    \begin{macrocode}
\def\BIC@Mul#1!#2{%
  \expandafter\expandafter\expandafter\BIC@MulSwitch
  \bigintcalcNum{#2}!#1!%
}
%    \end{macrocode}
%    \end{macro}
%    \begin{macro}{\BIC@MulSwitch}
%    Decision table for \cs{BIC@MulSwitch}.
%    \begin{quote}
%    \begin{tabular}[t]{@{}|l|l|l|l|l|@{}}
%      \hline
%      $x=0$ &\multicolumn{3}{l}{}    &$0$\\
%      \hline
%      $x>0$ & $y=0$ &\multicolumn2l{}&$0$\\
%      \cline{2-5}
%            & $y>0$ & $ x> y$ & $+$ & $\opMul( x, y)$\\
%      \cline{3-3}\cline{5-5}
%            &       & else    &     & $\opMul( y, x)$\\
%      \cline{2-5}
%            & $y<0$ & $ x>-y$ & $-$ & $\opMul( x,-y)$\\
%      \cline{3-3}\cline{5-5}
%            &       & else    &     & $\opMul(-y, x)$\\
%      \hline
%      $x<0$ & $y=0$ &\multicolumn2l{}&$0$\\
%      \cline{2-5}
%            & $y>0$ & $-x> y$ & $-$ & $\opMul(-x, y)$\\
%      \cline{3-3}\cline{5-5}
%            &       & else    &     & $\opMul( y,-x)$\\
%      \cline{2-5}
%            & $y<0$ & $-x>-y$ & $+$ & $\opMul(-x,-y)$\\
%      \cline{3-3}\cline{5-5}
%            &       & else    &     & $\opMul(-y,-x)$\\
%      \hline
%    \end{tabular}
%    \end{quote}
%    \begin{macrocode}
\def\BIC@MulSwitch#1#2!#3#4!{%
  \ifcase\BIC@Sgn#1#2! % x = 0
    \BIC@AfterFi{ 0}%
  \or % x > 0
    \ifcase\BIC@Sgn#3#4! % y = 0
      \BIC@AfterFiFi{ 0}%
    \or % y > 0
      \ifnum\BIC@PosCmp#1#2!#3#4!=1 % x > y
        \BIC@AfterFiFiFi{%
          \BIC@ProcessMul0!#1#2!#3#4!%
        }%
      \else % x <= y
        \BIC@AfterFiFiFi{%
          \BIC@ProcessMul0!#3#4!#1#2!%
        }%
      \fi
    \else % y < 0
      \expandafter-\romannumeral0%
      \ifnum\BIC@PosCmp#1#2!#4!=1 % x > -y
        \BIC@AfterFiFiFi{%
          \BIC@ProcessMul0!#1#2!#4!%
        }%
      \else % x <= -y
        \BIC@AfterFiFiFi{%
          \BIC@ProcessMul0!#4!#1#2!%
        }%
      \fi
    \fi
  \else % x < 0
    \ifcase\BIC@Sgn#3#4! % y = 0
      \BIC@AfterFiFi{ 0}%
    \or % y > 0
      \expandafter-\romannumeral0%
      \ifnum\BIC@PosCmp#2!#3#4!=1 % -x > y
        \BIC@AfterFiFiFi{%
          \BIC@ProcessMul0!#2!#3#4!%
        }%
      \else % -x <= y
        \BIC@AfterFiFiFi{%
          \BIC@ProcessMul0!#3#4!#2!%
        }%
      \fi
    \else % y < 0
      \ifnum\BIC@PosCmp#2!#4!=1 % -x > -y
        \BIC@AfterFiFiFi{%
          \BIC@ProcessMul0!#2!#4!%
        }%
      \else % -x <= -y
        \BIC@AfterFiFiFi{%
          \BIC@ProcessMul0!#4!#2!%
        }%
      \fi
    \fi
  \BIC@Fi
}
%    \end{macrocode}
%    \end{macro}
%    \begin{macro}{\BigIntCalcMul}
%    \begin{macrocode}
\def\BigIntCalcMul#1!#2!{%
  \romannumeral0%
  \BIC@ProcessMul0!#1!#2!%
}
%    \end{macrocode}
%    \end{macro}
%    \begin{macro}{\BIC@ProcessMul}
%    |#1|: result\\
%    |#2|: number $x$\\
%    |#3#4|: number $y$
%    \begin{macrocode}
\def\BIC@ProcessMul#1!#2!#3#4!{%
  \ifx\\#4\\%
    \BIC@AfterFi{%
      \expandafter\expandafter\expandafter\BIC@Space
      \bigintcalcAdd{\BIC@Tim#2!#3}{#10}%
    }%
  \else
    \BIC@AfterFi{%
      \expandafter\expandafter\expandafter\BIC@ProcessMul
      \bigintcalcAdd{\BIC@Tim#2!#3}{#10}!#2!#4!%
    }%
  \BIC@Fi
}
%    \end{macrocode}
%    \end{macro}
%
% \subsection{\op{Sqr}}
%
%    \begin{macro}{\bigintcalcSqr}
%    \begin{macrocode}
\def\bigintcalcSqr#1{%
  \romannumeral0%
  \expandafter\expandafter\expandafter\BIC@Sqr
  \bigintcalcNum{#1}!%
}
%    \end{macrocode}
%    \end{macro}
%    \begin{macro}{\BIC@Sqr}
%    \begin{macrocode}
\def\BIC@Sqr#1{%
   \ifx#1-%
     \expandafter\BIC@@Sqr
   \else
     \expandafter\BIC@@Sqr\expandafter#1%
   \fi
}
%    \end{macrocode}
%    \end{macro}
%    \begin{macro}{\BIC@@Sqr}
%    \begin{macrocode}
\def\BIC@@Sqr#1!{%
  \BIC@ProcessMul0!#1!#1!%
}
%    \end{macrocode}
%    \end{macro}
%
% \subsection{\op{Fac}}
%
%    \begin{macro}{\bigintcalcFac}
%    \begin{macrocode}
\def\bigintcalcFac#1{%
  \romannumeral0%
  \expandafter\expandafter\expandafter\BIC@Fac
  \bigintcalcNum{#1}!%
}
%    \end{macrocode}
%    \end{macro}
%    \begin{macro}{\BIC@Fac}
%    \begin{macrocode}
\def\BIC@Fac#1#2!{%
  \ifx#1-%
    \BIC@AfterFi{ 0\BigIntCalcError:FacNegative}%
  \else
    \ifnum\BIC@PosCmp#1#2!13!<0 %
      \ifcase#1#2 %
         \BIC@AfterFiFiFi{ 1}% 0!
      \or\BIC@AfterFiFiFi{ 1}% 1!
      \or\BIC@AfterFiFiFi{ 2}% 2!
      \or\BIC@AfterFiFiFi{ 6}% 3!
      \or\BIC@AfterFiFiFi{ 24}% 4!
      \or\BIC@AfterFiFiFi{ 120}% 5!
      \or\BIC@AfterFiFiFi{ 720}% 6!
      \or\BIC@AfterFiFiFi{ 5040}% 7!
      \or\BIC@AfterFiFiFi{ 40320}% 8!
      \or\BIC@AfterFiFiFi{ 362880}% 9!
      \or\BIC@AfterFiFiFi{ 3628800}% 10!
      \or\BIC@AfterFiFiFi{ 39916800}% 11!
      \or\BIC@AfterFiFiFi{ 479001600}% 12!
?     \else\BigIntCalcError:ThisCannotHappen%
      \fi
    \else
      \BIC@AfterFiFi{%
        \BIC@ProcessFac#1#2!479001600!%
      }%
    \fi
  \BIC@Fi
}
%    \end{macrocode}
%    \end{macro}
%    \begin{macro}{\BIC@ProcessFac}
%    |#1|: $n$\\
%    |#2|: result
%    \begin{macrocode}
\def\BIC@ProcessFac#1!#2!{%
  \ifnum\BIC@PosCmp#1!12!=0 %
    \BIC@AfterFi{ #2}%
  \else
    \BIC@AfterFi{%
      \expandafter\BIC@@ProcessFac
      \romannumeral0\BIC@ProcessMul0!#2!#1!%
      !#1!%
    }%
  \BIC@Fi
}
%    \end{macrocode}
%    \end{macro}
%    \begin{macro}{\BIC@@ProcessFac}
%    |#1|: result\\
%    |#2|: $n$
%    \begin{macrocode}
\def\BIC@@ProcessFac#1!#2!{%
  \expandafter\BIC@ProcessFac
  \romannumeral0\BIC@Dec#2!{}%
  !#1!%
}
%    \end{macrocode}
%    \end{macro}
%
% \subsection{\op{Pow}}
%
%    \begin{macro}{\bigintcalcPow}
%    |#1|: basis\\
%    |#2|: power
%    \begin{macrocode}
\def\bigintcalcPow#1{%
  \romannumeral0%
  \expandafter\expandafter\expandafter\BIC@Pow
  \bigintcalcNum{#1}!%
}
%    \end{macrocode}
%    \end{macro}
%    \begin{macro}{\BIC@Pow}
%    |#1|: basis\\
%    |#2|: power
%    \begin{macrocode}
\def\BIC@Pow#1!#2{%
  \expandafter\expandafter\expandafter\BIC@PowSwitch
  \bigintcalcNum{#2}!#1!%
}
%    \end{macrocode}
%    \end{macro}
%    \begin{macro}{\BIC@PowSwitch}
%    |#1#2|: power $y$\\
%    |#3#4|: basis $x$\\
%    Decision table for \cs{BIC@PowSwitch}.
%    \begin{quote}
%    \def\M#1{\multicolumn{#1}{l}{}}%
%    \begin{tabular}[t]{@{}|l|l|l|l|@{}}
%    \hline
%    $y=0$ & \M{2}                        & $1$\\
%    \hline
%    $y=1$ & \M{2}                        & $x$\\
%    \hline
%    $y=2$ & $x<0$          & \M{1}       & $\opMul(-x,-x)$\\
%    \cline{2-4}
%          & else           & \M{1}       & $\opMul(x,x)$\\
%    \hline
%    $y<0$ & $x=0$          & \M{1}       & DivisionByZero\\
%    \cline{2-4}
%          & $x=1$          & \M{1}       & $1$\\
%    \cline{2-4}
%          & $x=-1$         & $\opODD(y)$ & $-1$\\
%    \cline{3-4}
%          &                & else        & $1$\\
%    \cline{2-4}
%          & else ($\string|x\string|>1$) & \M{1}       & $0$\\
%    \hline
%    $y>2$ & $x=0$          & \M{1}       & $0$\\
%    \cline{2-4}
%          & $x=1$          & \M{1}       & $1$\\
%    \cline{2-4}
%          & $x=-1$         & $\opODD(y)$ & $-1$\\
%    \cline{3-4}
%          &                & else        & $1$\\
%    \cline{2-4}
%          & $x<-1$ ($x<0$) & $\opODD(y)$ & $-\opPow(-x,y)$\\
%    \cline{3-4}
%          &                & else        & $\opPow(-x,y)$\\
%    \cline{2-4}
%          & else ($x>1$)   & \M{1}       & $\opPow(x,y)$\\
%    \hline
%    \end{tabular}
%    \end{quote}
%    \begin{macrocode}
\def\BIC@PowSwitch#1#2!#3#4!{%
  \ifcase\ifx\\#2\\%
           \ifx#100 % y = 0
           \else\ifx#111 % y = 1
           \else\ifx#122 % y = 2
           \else4 % y > 2
           \fi\fi\fi
         \else
           \ifx#1-3 % y < 0
           \else4 % y > 2
           \fi
         \fi
    \BIC@AfterFi{ 1}% y = 0
  \or % y = 1
    \BIC@AfterFi{ #3#4}%
  \or % y = 2
    \ifx#3-% x < 0
      \BIC@AfterFiFi{%
        \BIC@ProcessMul0!#4!#4!%
      }%
    \else % x >= 0
      \BIC@AfterFiFi{%
        \BIC@ProcessMul0!#3#4!#3#4!%
      }%
    \fi
  \or % y < 0
    \ifcase\ifx\\#4\\%
             \ifx#300 % x = 0
             \else\ifx#311 % x = 1
             \else3 % x > 1
             \fi\fi
           \else
             \ifcase\BIC@MinusOne#3#4! %
               3 % |x| > 1
             \or
               2 % x = -1
?            \else\BigIntCalcError:ThisCannotHappen%
             \fi
           \fi
      \BIC@AfterFiFi{ 0\BigIntCalcError:DivisionByZero}% x = 0
    \or % x = 1
      \BIC@AfterFiFi{ 1}% x = 1
    \or % x = -1
      \ifcase\BIC@ModTwo#2! % even(y)
        \BIC@AfterFiFiFi{ 1}%
      \or % odd(y)
        \BIC@AfterFiFiFi{ -1}%
?     \else\BigIntCalcError:ThisCannotHappen%
      \fi
    \or % |x| > 1
      \BIC@AfterFiFi{ 0}%
?   \else\BigIntCalcError:ThisCannotHappen%
    \fi
  \or % y > 2
    \ifcase\ifx\\#4\\%
             \ifx#300 % x = 0
             \else\ifx#311 % x = 1
             \else4 % x > 1
             \fi\fi
           \else
             \ifx#3-%
               \ifcase\BIC@MinusOne#3#4! %
                 3 % x < -1
               \else
                 2 % x = -1
               \fi
             \else
               4 % x > 1
             \fi
           \fi
      \BIC@AfterFiFi{ 0}% x = 0
    \or % x = 1
      \BIC@AfterFiFi{ 1}% x = 1
    \or % x = -1
      \ifcase\BIC@ModTwo#1#2! % even(y)
        \BIC@AfterFiFiFi{ 1}%
      \or % odd(y)
        \BIC@AfterFiFiFi{ -1}%
?     \else\BigIntCalcError:ThisCannotHappen%
      \fi
    \or % x < -1
      \ifcase\BIC@ModTwo#1#2! % even(y)
        \BIC@AfterFiFiFi{%
          \BIC@PowRec#4!#1#2!1!%
        }%
      \or % odd(y)
        \expandafter-\romannumeral0%
        \BIC@AfterFiFiFi{%
          \BIC@PowRec#4!#1#2!1!%
        }%
?     \else\BigIntCalcError:ThisCannotHappen%
      \fi
    \or % x > 1
      \BIC@AfterFiFi{%
        \BIC@PowRec#3#4!#1#2!1!%
      }%
?   \else\BigIntCalcError:ThisCannotHappen%
    \fi
? \else\BigIntCalcError:ThisCannotHappen%
  \BIC@Fi
}
%    \end{macrocode}
%    \end{macro}
%
% \subsubsection{Help macros}
%
%    \begin{macro}{\BIC@ModTwo}
%    Macro \cs{BIC@ModTwo} expects a number without sign
%    and returns digit |1| or |0| if the number is odd or even.
%    \begin{macrocode}
\def\BIC@ModTwo#1#2!{%
  \ifx\\#2\\%
    \ifodd#1 %
      \BIC@AfterFiFi1%
    \else
      \BIC@AfterFiFi0%
    \fi
  \else
    \BIC@AfterFi{%
      \BIC@ModTwo#2!%
    }%
  \BIC@Fi
}
%    \end{macrocode}
%    \end{macro}
%
%    \begin{macro}{\BIC@MinusOne}
%    Macro \cs{BIC@MinusOne} expects a number and returns
%    digit |1| if the number equals minus one and returns |0| otherwise.
%    \begin{macrocode}
\def\BIC@MinusOne#1#2!{%
  \ifx#1-%
    \BIC@@MinusOne#2!%
  \else
    0%
  \fi
}
%    \end{macrocode}
%    \end{macro}
%    \begin{macro}{\BIC@@MinusOne}
%    \begin{macrocode}
\def\BIC@@MinusOne#1#2!{%
  \ifx#11%
    \ifx\\#2\\%
      1%
    \else
      0%
    \fi
  \else
    0%
  \fi
}
%    \end{macrocode}
%    \end{macro}
%
% \subsubsection{Recursive calculation}
%
%    \begin{macro}{\BIC@PowRec}
%\begin{quote}
%\begin{verbatim}
%Pow(x, y) {
%  PowRec(x, y, 1)
%}
%PowRec(x, y, r) {
%  if y == 1 then
%    return r
%  else
%    ifodd y then
%      return PowRec(x*x, y div 2, r*x) % y div 2 = (y-1)/2
%    else
%      return PowRec(x*x, y div 2, r)
%    fi
%  fi
%}
%\end{verbatim}
%\end{quote}
%    |#1|: $x$ (basis)\\
%    |#2#3|: $y$ (power)\\
%    |#4|: $r$ (result)
%    \begin{macrocode}
\def\BIC@PowRec#1!#2#3!#4!{%
  \ifcase\ifx#21\ifx\\#3\\0 \else1 \fi\else1 \fi % y = 1
    \ifnum\BIC@PosCmp#1!#4!=1 % x > r
      \BIC@AfterFiFi{%
        \BIC@ProcessMul0!#1!#4!%
      }%
    \else
      \BIC@AfterFiFi{%
        \BIC@ProcessMul0!#4!#1!%
      }%
    \fi
  \or
    \ifcase\BIC@ModTwo#2#3! % even(y)
      \BIC@AfterFiFi{%
        \expandafter\BIC@@PowRec\romannumeral0%
        \BIC@@Shr#2#3!%
        !#1!#4!%
      }%
    \or % odd(y)
      \ifnum\BIC@PosCmp#1!#4!=1 % x > r
        \BIC@AfterFiFiFi{%
          \expandafter\BIC@@@PowRec\romannumeral0%
          \BIC@ProcessMul0!#1!#4!%
          !#1!#2#3!%
        }%
      \else
        \BIC@AfterFiFiFi{%
          \expandafter\BIC@@@PowRec\romannumeral0%
          \BIC@ProcessMul0!#1!#4!%
          !#1!#2#3!%
        }%
      \fi
?   \else\BigIntCalcError:ThisCannotHappen%
    \fi
? \else\BigIntCalcError:ThisCannotHappen%
  \BIC@Fi
}
%    \end{macrocode}
%    \end{macro}
%    \begin{macro}{\BIC@@PowRec}
%    |#1|: $y/2$\\
%    |#2|: $x$\\
%    |#3|: new $r$ ($r$ or $r*x$)
%    \begin{macrocode}
\def\BIC@@PowRec#1!#2!#3!{%
  \expandafter\BIC@PowRec\romannumeral0%
  \BIC@ProcessMul0!#2!#2!%
  !#1!#3!%
}
%    \end{macrocode}
%    \end{macro}
%    \begin{macro}{\BIC@@@PowRec}
%    |#1|: $r*x$
%    |#2|: $x$
%    |#3|: $y$
%    \begin{macrocode}
\def\BIC@@@PowRec#1!#2!#3!{%
  \expandafter\BIC@@PowRec\romannumeral0%
  \BIC@@Shr#3!%
  !#2!#1!%
}
%    \end{macrocode}
%    \end{macro}
%
% \subsection{\op{Div}}
%
%    \begin{macro}{\bigintcalcDiv}
%    |#1|: $x$\\
%    |#2|: $y$ (divisor)
%    \begin{macrocode}
\def\bigintcalcDiv#1{%
  \romannumeral0%
  \expandafter\expandafter\expandafter\BIC@Div
  \bigintcalcNum{#1}!%
}
%    \end{macrocode}
%    \end{macro}
%    \begin{macro}{\BIC@Div}
%    |#1|: $x$\\
%    |#2|: $y$
%    \begin{macrocode}
\def\BIC@Div#1!#2{%
  \expandafter\expandafter\expandafter\BIC@DivSwitchSign
  \bigintcalcNum{#2}!#1!%
}
%    \end{macrocode}
%    \end{macro}
%    \begin{macro}{\BigIntCalcDiv}
%    \begin{macrocode}
\def\BigIntCalcDiv#1!#2!{%
  \romannumeral0%
  \BIC@DivSwitchSign#2!#1!%
}
%    \end{macrocode}
%    \end{macro}
%    \begin{macro}{\BIC@DivSwitchSign}
%    Decision table for \cs{BIC@DivSwitchSign}.
%    \begin{quote}
%    \begin{tabular}{@{}|l|l|l|@{}}
%    \hline
%    $y=0$ & \multicolumn{1}{l}{} & DivisionByZero\\
%    \hline
%    $y>0$ & $x=0$ & $0$\\
%    \cline{2-3}
%          & $x>0$ & DivSwitch$(+,x,y)$\\
%    \cline{2-3}
%          & $x<0$ & DivSwitch$(-,-x,y)$\\
%    \hline
%    $y<0$ & $x=0$ & $0$\\
%    \cline{2-3}
%          & $x>0$ & DivSwitch$(-,x,-y)$\\
%    \cline{2-3}
%          & $x<0$ & DivSwitch$(+,-x,-y)$\\
%    \hline
%    \end{tabular}
%    \end{quote}
%    |#1|: $y$ (divisor)\\
%    |#2|: $x$
%    \begin{macrocode}
\def\BIC@DivSwitchSign#1#2!#3#4!{%
  \ifcase\BIC@Sgn#1#2! % y = 0
    \BIC@AfterFi{ 0\BigIntCalcError:DivisionByZero}%
  \or % y > 0
    \ifcase\BIC@Sgn#3#4! % x = 0
      \BIC@AfterFiFi{ 0}%
    \or % x > 0
      \BIC@AfterFiFi{%
        \BIC@DivSwitch{}#3#4!#1#2!%
      }%
    \else % x < 0
      \BIC@AfterFiFi{%
        \BIC@DivSwitch-#4!#1#2!%
      }%
    \fi
  \else % y < 0
    \ifcase\BIC@Sgn#3#4! % x = 0
      \BIC@AfterFiFi{ 0}%
    \or % x > 0
      \BIC@AfterFiFi{%
        \BIC@DivSwitch-#3#4!#2!%
      }%
    \else % x < 0
      \BIC@AfterFiFi{%
        \BIC@DivSwitch{}#4!#2!%
      }%
    \fi
  \BIC@Fi
}
%    \end{macrocode}
%    \end{macro}
%    \begin{macro}{\BIC@DivSwitch}
%    Decision table for \cs{BIC@DivSwitch}.
%    \begin{quote}
%    \begin{tabular}{@{}|l|l|l|@{}}
%    \hline
%    $y=x$ & \multicolumn{1}{l}{} & sign $1$\\
%    \hline
%    $y>x$ & \multicolumn{1}{l}{} & $0$\\
%    \hline
%    $y<x$ & $y=1$                & sign $x$\\
%    \cline{2-3}
%          & $y=2$                & sign Shr$(x)$\\
%    \cline{2-3}
%          & $y=4$                & sign Shr(Shr$(x)$)\\
%    \cline{2-3}
%          & else                 & sign ProcessDiv$(x,y)$\\
%    \hline
%    \end{tabular}
%    \end{quote}
%    |#1|: sign\\
%    |#2|: $x$\\
%    |#3#4|: $y$ ($y\ne 0$)
%    \begin{macrocode}
\def\BIC@DivSwitch#1#2!#3#4!{%
  \ifcase\BIC@PosCmp#3#4!#2!% y = x
    \BIC@AfterFi{ #11}%
  \or % y > x
    \BIC@AfterFi{ 0}%
  \else % y < x
    \ifx\\#1\\%
    \else
      \expandafter-\romannumeral0%
    \fi
    \ifcase\ifx\\#4\\%
             \ifx#310 % y = 1
             \else\ifx#321 % y = 2
             \else\ifx#342 % y = 4
             \else3 % y > 2
             \fi\fi\fi
           \else
             3 % y > 2
           \fi
      \BIC@AfterFiFi{ #2}% y = 1
    \or % y = 2
      \BIC@AfterFiFi{%
        \BIC@@Shr#2!%
      }%
    \or % y = 4
      \BIC@AfterFiFi{%
        \expandafter\BIC@@Shr\romannumeral0%
          \BIC@@Shr#2!!%
      }%
    \or % y > 2
      \BIC@AfterFiFi{%
        \BIC@DivStartX#2!#3#4!!!%
      }%
?   \else\BigIntCalcError:ThisCannotHappen%
    \fi
  \BIC@Fi
}
%    \end{macrocode}
%    \end{macro}
%    \begin{macro}{\BIC@ProcessDiv}
%    |#1#2|: $x$\\
%    |#3#4|: $y$\\
%    |#5|: collect first digits of $x$\\
%    |#6|: corresponding digits of $y$
%    \begin{macrocode}
\def\BIC@DivStartX#1#2!#3#4!#5!#6!{%
  \ifx\\#4\\%
    \BIC@AfterFi{%
      \BIC@DivStartYii#6#3#4!{#5#1}#2=!%
    }%
  \else
    \BIC@AfterFi{%
      \BIC@DivStartX#2!#4!#5#1!#6#3!%
    }%
  \BIC@Fi
}
%    \end{macrocode}
%    \end{macro}
%    \begin{macro}{\BIC@DivStartYii}
%    |#1|: $y$\\
%    |#2|: $x$, |=|
%    \begin{macrocode}
\def\BIC@DivStartYii#1!{%
  \expandafter\BIC@DivStartYiv\romannumeral0%
  \BIC@Shl#1!%
  !#1!%
}
%    \end{macrocode}
%    \end{macro}
%    \begin{macro}{\BIC@DivStartYiv}
%    |#1|: $2y$\\
%    |#2|: $y$\\
%    |#3|: $x$, |=|
%    \begin{macrocode}
\def\BIC@DivStartYiv#1!{%
  \expandafter\BIC@DivStartYvi\romannumeral0%
  \BIC@Shl#1!%
  !#1!%
}
%    \end{macrocode}
%    \end{macro}
%    \begin{macro}{\BIC@DivStartYvi}
%    |#1|: $4y$\\
%    |#2|: $2y$\\
%    |#3|: $y$\\
%    |#4|: $x$, |=|
%    \begin{macrocode}
\def\BIC@DivStartYvi#1!#2!{%
  \expandafter\BIC@DivStartYviii\romannumeral0%
  \BIC@AddXY#1!#2!!!%
  !#1!#2!%
}
%    \end{macrocode}
%    \end{macro}
%    \begin{macro}{\BIC@DivStartYviii}
%    |#1|: $6y$\\
%    |#2|: $4y$\\
%    |#3|: $2y$\\
%    |#4|: $y$\\
%    |#5|: $x$, |=|
%    \begin{macrocode}
\def\BIC@DivStartYviii#1!#2!{%
  \expandafter\BIC@DivStart\romannumeral0%
  \BIC@Shl#2!%
  !#1!#2!%
}
%    \end{macrocode}
%    \end{macro}
%    \begin{macro}{\BIC@DivStart}
%    |#1|: $8y$\\
%    |#2|: $6y$\\
%    |#3|: $4y$\\
%    |#4|: $2y$\\
%    |#5|: $y$\\
%    |#6|: $x$, |=|
%    \begin{macrocode}
\def\BIC@DivStart#1!#2!#3!#4!#5!#6!{%
  \BIC@ProcessDiv#6!!#5!#4!#3!#2!#1!=%
}
%    \end{macrocode}
%    \end{macro}
%    \begin{macro}{\BIC@ProcessDiv}
%    |#1#2#3|: $x$, |=|\\
%    |#4|: result\\
%    |#5|: $y$\\
%    |#6|: $2y$\\
%    |#7|: $4y$\\
%    |#8|: $6y$\\
%    |#9|: $8y$
%    \begin{macrocode}
\def\BIC@ProcessDiv#1#2#3!#4!#5!{%
  \ifcase\BIC@PosCmp#5!#1!% y = #1
    \ifx#2=%
      \BIC@AfterFiFi{\BIC@DivCleanup{#41}}%
    \else
      \BIC@AfterFiFi{%
        \BIC@ProcessDiv#2#3!#41!#5!%
      }%
    \fi
  \or % y > #1
    \ifx#2=%
      \BIC@AfterFiFi{\BIC@DivCleanup{#40}}%
    \else
      \ifx\\#4\\%
        \BIC@AfterFiFiFi{%
          \BIC@ProcessDiv{#1#2}#3!!#5!%
        }%
      \else
        \BIC@AfterFiFiFi{%
          \BIC@ProcessDiv{#1#2}#3!#40!#5!%
        }%
      \fi
    \fi
  \else % y < #1
    \BIC@AfterFi{%
      \BIC@@ProcessDiv{#1}#2#3!#4!#5!%
    }%
  \BIC@Fi
}
%    \end{macrocode}
%    \end{macro}
%    \begin{macro}{\BIC@DivCleanup}
%    |#1|: result\\
%    |#2|: garbage
%    \begin{macrocode}
\def\BIC@DivCleanup#1#2={ #1}%
%    \end{macrocode}
%    \end{macro}
%    \begin{macro}{\BIC@@ProcessDiv}
%    \begin{macrocode}
\def\BIC@@ProcessDiv#1#2#3!#4!#5!#6!#7!{%
  \ifcase\BIC@PosCmp#7!#1!% 4y = #1
    \ifx#2=%
      \BIC@AfterFiFi{\BIC@DivCleanup{#44}}%
    \else
     \BIC@AfterFiFi{%
       \BIC@ProcessDiv#2#3!#44!#5!#6!#7!%
     }%
    \fi
  \or % 4y > #1
    \ifcase\BIC@PosCmp#6!#1!% 2y = #1
      \ifx#2=%
        \BIC@AfterFiFiFi{\BIC@DivCleanup{#42}}%
      \else
        \BIC@AfterFiFiFi{%
          \BIC@ProcessDiv#2#3!#42!#5!#6!#7!%
        }%
      \fi
    \or % 2y > #1
      \ifx#2=%
        \BIC@AfterFiFiFi{\BIC@DivCleanup{#41}}%
      \else
        \BIC@AfterFiFiFi{%
          \BIC@DivSub#1!#5!#2#3!#41!#5!#6!#7!%
        }%
      \fi
    \else % 2y < #1
      \BIC@AfterFiFi{%
        \expandafter\BIC@ProcessDivII\romannumeral0%
        \BIC@SubXY#1!#6!!!%
        !#2#3!#4!#5!23%
        #6!#7!%
      }%
    \fi
  \else % 4y < #1
    \BIC@AfterFi{%
      \BIC@@@ProcessDiv{#1}#2#3!#4!#5!#6!#7!%
    }%
  \BIC@Fi
}
%    \end{macrocode}
%    \end{macro}
%    \begin{macro}{\BIC@DivSub}
%    Next token group: |#1|-|#2| and next digit |#3|.
%    \begin{macrocode}
\def\BIC@DivSub#1!#2!#3{%
  \expandafter\BIC@ProcessDiv\expandafter{%
    \romannumeral0%
    \BIC@SubXY#1!#2!!!%
    #3%
  }%
}
%    \end{macrocode}
%    \end{macro}
%    \begin{macro}{\BIC@ProcessDivII}
%    |#1|: $x'-2y$\\
%    |#2#3|: remaining $x$, |=|\\
%    |#4|: result\\
%    |#5|: $y$\\
%    |#6|: first possible result digit\\
%    |#7|: second possible result digit
%    \begin{macrocode}
\def\BIC@ProcessDivII#1!#2#3!#4!#5!#6#7{%
  \ifcase\BIC@PosCmp#5!#1!% y = #1
    \ifx#2=%
      \BIC@AfterFiFi{\BIC@DivCleanup{#4#7}}%
    \else
      \BIC@AfterFiFi{%
        \BIC@ProcessDiv#2#3!#4#7!#5!%
      }%
    \fi
  \or % y > #1
    \ifx#2=%
      \BIC@AfterFiFi{\BIC@DivCleanup{#4#6}}%
    \else
      \BIC@AfterFiFi{%
        \BIC@ProcessDiv{#1#2}#3!#4#6!#5!%
      }%
    \fi
  \else % y < #1
    \ifx#2=%
      \BIC@AfterFiFi{\BIC@DivCleanup{#4#7}}%
    \else
      \BIC@AfterFiFi{%
        \BIC@DivSub#1!#5!#2#3!#4#7!#5!%
      }%
    \fi
  \BIC@Fi
}
%    \end{macrocode}
%    \end{macro}
%    \begin{macro}{\BIC@ProcessDivIV}
%    |#1#2#3|: $x$, |=|, $x>4y$\\
%    |#4|: result\\
%    |#5|: $y$\\
%    |#6|: $2y$\\
%    |#7|: $4y$\\
%    |#8|: $6y$\\
%    |#9|: $8y$
%    \begin{macrocode}
\def\BIC@@@ProcessDiv#1#2#3!#4!#5!#6!#7!#8!#9!{%
  \ifcase\BIC@PosCmp#8!#1!% 6y = #1
    \ifx#2=%
      \BIC@AfterFiFi{\BIC@DivCleanup{#46}}%
    \else
      \BIC@AfterFiFi{%
        \BIC@ProcessDiv#2#3!#46!#5!#6!#7!#8!#9!%
      }%
    \fi
  \or % 6y > #1
    \BIC@AfterFi{%
      \expandafter\BIC@ProcessDivII\romannumeral0%
      \BIC@SubXY#1!#7!!!%
      !#2#3!#4!#5!45%
      #6!#7!#8!#9!%
    }%
  \else % 6y < #1
    \ifcase\BIC@PosCmp#9!#1!% 8y = #1
      \ifx#2=%
        \BIC@AfterFiFiFi{\BIC@DivCleanup{#48}}%
      \else
        \BIC@AfterFiFiFi{%
          \BIC@ProcessDiv#2#3!#48!#5!#6!#7!#8!#9!%
        }%
      \fi
    \or % 8y > #1
      \BIC@AfterFiFi{%
        \expandafter\BIC@ProcessDivII\romannumeral0%
        \BIC@SubXY#1!#8!!!%
        !#2#3!#4!#5!67%
        #6!#7!#8!#9!%
      }%
    \else % 8y < #1
      \BIC@AfterFiFi{%
        \expandafter\BIC@ProcessDivII\romannumeral0%
        \BIC@SubXY#1!#9!!!%
        !#2#3!#4!#5!89%
        #6!#7!#8!#9!%
      }%
    \fi
  \BIC@Fi
}
%    \end{macrocode}
%    \end{macro}
%
% \subsection{\op{Mod}}
%
%    \begin{macro}{\bigintcalcMod}
%    |#1|: $x$\\
%    |#2|: $y$
%    \begin{macrocode}
\def\bigintcalcMod#1{%
  \romannumeral0%
  \expandafter\expandafter\expandafter\BIC@Mod
  \bigintcalcNum{#1}!%
}
%    \end{macrocode}
%    \end{macro}
%    \begin{macro}{\BIC@Mod}
%    |#1|: $x$\\
%    |#2|: $y$
%    \begin{macrocode}
\def\BIC@Mod#1!#2{%
  \expandafter\expandafter\expandafter\BIC@ModSwitchSign
  \bigintcalcNum{#2}!#1!%
}
%    \end{macrocode}
%    \end{macro}
%    \begin{macro}{\BigIntCalcMod}
%    \begin{macrocode}
\def\BigIntCalcMod#1!#2!{%
  \romannumeral0%
  \BIC@ModSwitchSign#2!#1!%
}
%    \end{macrocode}
%    \end{macro}
%    \begin{macro}{\BIC@ModSwitchSign}
%    Decision table for \cs{BIC@ModSwitchSign}.
%    \begin{quote}
%    \begin{tabular}{@{}|l|l|l|@{}}
%    \hline
%    $y=0$ & \multicolumn{1}{l}{} & DivisionByZero\\
%    \hline
%    $y>0$ & $x=0$ & $0$\\
%    \cline{2-3}
%          & else  & ModSwitch$(+,x,y)$\\
%    \hline
%    $y<0$ & \multicolumn{1}{l}{} & ModSwitch$(-,-x,-y)$\\
%    \hline
%    \end{tabular}
%    \end{quote}
%    |#1#2|: $y$\\
%    |#3#4|: $x$
%    \begin{macrocode}
\def\BIC@ModSwitchSign#1#2!#3#4!{%
  \ifcase\ifx\\#2\\%
           \ifx#100 % y = 0
           \else1 % y > 0
           \fi
          \else
            \ifx#1-2 % y < 0
            \else1 % y > 0
            \fi
          \fi
    \BIC@AfterFi{ 0\BigIntCalcError:DivisionByZero}%
  \or % y > 0
    \ifcase\ifx\\#4\\\ifx#300 \else1 \fi\else1 \fi % x = 0
      \BIC@AfterFiFi{ 0}%
    \else
      \BIC@AfterFiFi{%
        \BIC@ModSwitch{}#3#4!#1#2!%
      }%
    \fi
  \else % y < 0
    \ifcase\ifx\\#4\\%
             \ifx#300 % x = 0
             \else1 % x > 0
             \fi
           \else
             \ifx#3-2 % x < 0
             \else1 % x > 0
             \fi
           \fi
      \BIC@AfterFiFi{ 0}%
    \or % x > 0
      \BIC@AfterFiFi{%
        \BIC@ModSwitch--#3#4!#2!%
      }%
    \else % x < 0
      \BIC@AfterFiFi{%
        \BIC@ModSwitch-#4!#2!%
      }%
    \fi
  \BIC@Fi
}
%    \end{macrocode}
%    \end{macro}
%    \begin{macro}{\BIC@ModSwitch}
%    Decision table for \cs{BIC@ModSwitch}.
%    \begin{quote}
%    \begin{tabular}{@{}|l|l|l|@{}}
%    \hline
%    $y=1$ & \multicolumn{1}{l}{} & $0$\\
%    \hline
%    $y=2$ & ifodd$(x)$ & sign $1$\\
%    \cline{2-3}
%          & else       & $0$\\
%    \hline
%    $y>2$ & $x<0$      &
%        $z\leftarrow x-(x/y)*y;\quad (z<0)\mathbin{?}z+y\mathbin{:}z$\\
%    \cline{2-3}
%          & $x>0$      & $x - (x/y) * y$\\
%    \hline
%    \end{tabular}
%    \end{quote}
%    |#1|: sign\\
%    |#2#3|: $x$\\
%    |#4#5|: $y$
%    \begin{macrocode}
\def\BIC@ModSwitch#1#2#3!#4#5!{%
  \ifcase\ifx\\#5\\%
           \ifx#410 % y = 1
           \else\ifx#421 % y = 2
           \else2 % y > 2
           \fi\fi
         \else2 % y > 2
         \fi
    \BIC@AfterFi{ 0}% y = 1
  \or % y = 2
    \ifcase\BIC@ModTwo#2#3! % even(x)
      \BIC@AfterFiFi{ 0}%
    \or % odd(x)
      \BIC@AfterFiFi{ #11}%
?   \else\BigIntCalcError:ThisCannotHappen%
    \fi
  \or % y > 2
    \ifx\\#1\\%
    \else
      \expandafter\BIC@Space\romannumeral0%
      \expandafter\BIC@ModMinus\romannumeral0%
    \fi
    \ifx#2-% x < 0
      \BIC@AfterFiFi{%
        \expandafter\expandafter\expandafter\BIC@ModX
        \bigintcalcSub{#2#3}{%
          \bigintcalcMul{#4#5}{\bigintcalcDiv{#2#3}{#4#5}}%
        }!#4#5!%
      }%
    \else % x > 0
      \BIC@AfterFiFi{%
        \expandafter\expandafter\expandafter\BIC@Space
        \bigintcalcSub{#2#3}{%
          \bigintcalcMul{#4#5}{\bigintcalcDiv{#2#3}{#4#5}}%
        }%
      }%
    \fi
? \else\BigIntCalcError:ThisCannotHappen%
  \BIC@Fi
}
%    \end{macrocode}
%    \end{macro}
%    \begin{macro}{\BIC@ModMinus}
%    \begin{macrocode}
\def\BIC@ModMinus#1{%
  \ifx#10%
    \BIC@AfterFi{ 0}%
  \else
    \BIC@AfterFi{ -#1}%
  \BIC@Fi
}
%    \end{macrocode}
%    \end{macro}
%    \begin{macro}{\BIC@ModX}
%    |#1#2|: $z$\\
%    |#3|: $x$
%    \begin{macrocode}
\def\BIC@ModX#1#2!#3!{%
  \ifx#1-% z < 0
    \BIC@AfterFi{%
      \expandafter\BIC@Space\romannumeral0%
      \BIC@SubXY#3!#2!!!%
    }%
  \else % z >= 0
    \BIC@AfterFi{ #1#2}%
  \BIC@Fi
}
%    \end{macrocode}
%    \end{macro}
%
%    \begin{macrocode}
\BIC@AtEnd%
%    \end{macrocode}
%
%    \begin{macrocode}
%</package>
%    \end{macrocode}
%
% \section{Test}
%
% \subsection{Catcode checks for loading}
%
%    \begin{macrocode}
%<*test1>
%    \end{macrocode}
%    \begin{macrocode}
\catcode`\{=1 %
\catcode`\}=2 %
\catcode`\#=6 %
\catcode`\@=11 %
\expandafter\ifx\csname count@\endcsname\relax
  \countdef\count@=255 %
\fi
\expandafter\ifx\csname @gobble\endcsname\relax
  \long\def\@gobble#1{}%
\fi
\expandafter\ifx\csname @firstofone\endcsname\relax
  \long\def\@firstofone#1{#1}%
\fi
\expandafter\ifx\csname loop\endcsname\relax
  \expandafter\@firstofone
\else
  \expandafter\@gobble
\fi
{%
  \def\loop#1\repeat{%
    \def\body{#1}%
    \iterate
  }%
  \def\iterate{%
    \body
      \let\next\iterate
    \else
      \let\next\relax
    \fi
    \next
  }%
  \let\repeat=\fi
}%
\def\RestoreCatcodes{}
\count@=0 %
\loop
  \edef\RestoreCatcodes{%
    \RestoreCatcodes
    \catcode\the\count@=\the\catcode\count@\relax
  }%
\ifnum\count@<255 %
  \advance\count@ 1 %
\repeat

\def\RangeCatcodeInvalid#1#2{%
  \count@=#1\relax
  \loop
    \catcode\count@=15 %
  \ifnum\count@<#2\relax
    \advance\count@ 1 %
  \repeat
}
\def\RangeCatcodeCheck#1#2#3{%
  \count@=#1\relax
  \loop
    \ifnum#3=\catcode\count@
    \else
      \errmessage{%
        Character \the\count@\space
        with wrong catcode \the\catcode\count@\space
        instead of \number#3%
      }%
    \fi
  \ifnum\count@<#2\relax
    \advance\count@ 1 %
  \repeat
}
\def\space{ }
\expandafter\ifx\csname LoadCommand\endcsname\relax
  \def\LoadCommand{\input bigintcalc.sty\relax}%
\fi
\def\Test{%
  \RangeCatcodeInvalid{0}{47}%
  \RangeCatcodeInvalid{58}{64}%
  \RangeCatcodeInvalid{91}{96}%
  \RangeCatcodeInvalid{123}{255}%
  \catcode`\@=12 %
  \catcode`\\=0 %
  \catcode`\%=14 %
  \LoadCommand
  \RangeCatcodeCheck{0}{36}{15}%
  \RangeCatcodeCheck{37}{37}{14}%
  \RangeCatcodeCheck{38}{47}{15}%
  \RangeCatcodeCheck{48}{57}{12}%
  \RangeCatcodeCheck{58}{63}{15}%
  \RangeCatcodeCheck{64}{64}{12}%
  \RangeCatcodeCheck{65}{90}{11}%
  \RangeCatcodeCheck{91}{91}{15}%
  \RangeCatcodeCheck{92}{92}{0}%
  \RangeCatcodeCheck{93}{96}{15}%
  \RangeCatcodeCheck{97}{122}{11}%
  \RangeCatcodeCheck{123}{255}{15}%
  \RestoreCatcodes
}
\Test
\csname @@end\endcsname
\end
%    \end{macrocode}
%    \begin{macrocode}
%</test1>
%    \end{macrocode}
%
% \subsection{Macro tests}
%
% \subsubsection{Preamble with test macro definitions}
%
%    \begin{macrocode}
%<*test2>
\NeedsTeXFormat{LaTeX2e}
\nofiles
\documentclass{article}
%<noetex>\let\SavedNumexpr\numexpr
%<noetex>\let\numexpr\UNDEFINED
\makeatletter
\chardef\BIC@TestMode=1 %
\makeatother
\usepackage{bigintcalc}[2016/05/16]
%<noetex>\let\numexpr\SavedNumexpr
\usepackage{qstest}
\IncludeTests{*}
\LogTests{log}{*}{*}
\newcommand*{\TestSpaceAtEnd}[1]{%
%<noetex>  \let\SavedNumexpr\numexpr
%<noetex>  \let\numexpr\UNDEFINED
  \edef\resultA{#1}%
  \edef\resultB{#1 }%
%<noetex>  \let\numexpr\SavedNumexpr
  \Expect*{\resultA\space}*{\resultB}%
}
\newcommand*{\TestResult}[2]{%
%<noetex>  \let\SavedNumexpr\numexpr
%<noetex>  \let\numexpr\UNDEFINED
  \edef\result{#1}%
%<noetex>  \let\numexpr\SavedNumexpr
  \Expect*{\result}{#2}%
}
\newcommand*{\TestResultTwoExpansions}[2]{%
%<*noetex>
  \begingroup
    \let\numexpr\UNDEFINED
    \expandafter\expandafter\expandafter
  \endgroup
%</noetex>
  \expandafter\expandafter\expandafter\Expect
  \expandafter\expandafter\expandafter{#1}{#2}%
}
\newcount\TestCount
%<etex>\newcommand*{\TestArg}[1]{\numexpr#1\relax}
%<noetex>\newcommand*{\TestArg}[1]{#1}
\newcommand*{\TestTeXDivide}[2]{%
  \TestCount=\TestArg{#1}\relax
  \divide\TestCount by \TestArg{#2}\relax
  \Expect*{\bigintcalcDiv{#1}{#2}}*{\the\TestCount}%
}
\newcommand*{\Test}[2]{%
  \TestResult{#1}{#2}%
  \TestResultTwoExpansions{#1}{#2}%
  \TestSpaceAtEnd{#1}%
}
\newcommand*{\TestExch}[2]{\Test{#2}{#1}}
\newcommand*{\TestInv}[2]{%
  \Test{\bigintcalcInv{#1}}{#2}%
}
\newcommand*{\TestAbs}[2]{%
  \Test{\bigintcalcAbs{#1}}{#2}%
}
\newcommand*{\TestSgn}[2]{%
  \Test{\bigintcalcSgn{#1}}{#2}%
}
\newcommand*{\TestMin}[3]{%
  \Test{\bigintcalcMin{#1}{#2}}{#3}%
}
\newcommand*{\TestMax}[3]{%
  \Test{\bigintcalcMax{#1}{#2}}{#3}%
}
\newcommand*{\TestCmp}[3]{%
  \Test{\bigintcalcCmp{#1}{#2}}{#3}%
}
\newcommand*{\TestOdd}[2]{%
  \Test{\bigintcalcOdd{#1}}{#2}%
  \edef\x{%
    \noexpand\Test{%
      \noexpand\BigIntCalcOdd
      \bigintcalcAbs{#1}!%
    }{#2}%
  }%
  \x
}
\newcommand*{\TestInc}[2]{%
  \Test{\bigintcalcInc{#1}}{#2}%
  \ifnum\bigintcalcSgn{#1}>-1 %
    \edef\x{%
      \noexpand\Test{%
        \noexpand\BigIntCalcInc\bigintcalcNum{#1}!%
      }{#2}%
    }%
    \x
  \fi
}
\newcommand*{\TestDec}[2]{%
  \Test{\bigintcalcDec{#1}}{#2}%
  \ifnum\bigintcalcSgn{#1}>0 %
    \edef\x{%
      \noexpand\Test{%
        \noexpand\BigIntCalcDec\bigintcalcNum{#1}!%
      }{#2}%
    }%
    \x
  \fi
}
\newcommand*{\TestAdd}[3]{%
  \Test{\bigintcalcAdd{#1}{#2}}{#3}%
  \ifnum\bigintcalcSgn{#1}>0 %
    \ifnum\bigintcalcSgn{#2}> 0 %
      \ifnum\bigintcalcCmp{#1}{#2}>0 %
        \edef\x{%
          \noexpand\Test{%
            \noexpand\BigIntCalcAdd
            \bigintcalcNum{#1}!\bigintcalcNum{#2}!%
          }{#3}%
        }%
        \x
      \else
        \edef\x{%
          \noexpand\Test{%
            \noexpand\BigIntCalcAdd
            \bigintcalcNum{#2}!\bigintcalcNum{#1}!%
          }{#3}%
        }%
        \x
      \fi
    \fi
  \fi
}
\newcommand*{\TestSub}[3]{%
  \Test{\bigintcalcSub{#1}{#2}}{#3}%
  \ifnum\bigintcalcSgn{#1}>0 %
    \ifnum\bigintcalcSgn{#2}> 0 %
      \ifnum\bigintcalcCmp{#1}{#2}>0 %
        \edef\x{%
          \noexpand\Test{%
            \noexpand\BigIntCalcSub
            \bigintcalcNum{#1}!\bigintcalcNum{#2}!%
          }{#3}%
        }%
        \x
      \fi
    \fi
  \fi
}
\newcommand*{\TestShl}[2]{%
  \Test{\bigintcalcShl{#1}}{#2}%
  \edef\x{%
    \noexpand\Test{%
      \noexpand\BigIntCalcShl\bigintcalcAbs{#1}!%
    }{\bigintcalcAbs{#2}}%
  }%
  \x
}
\newcommand*{\TestShr}[2]{%
  \Test{\bigintcalcShr{#1}}{#2}%
  \edef\x{%
    \noexpand\Test{%
      \noexpand\BigIntCalcShr\bigintcalcAbs{#1}!%
    }{\bigintcalcAbs{#2}}%
  }%
  \x
}
\newcommand*{\TestMul}[3]{%
  \Test{\bigintcalcMul{#1}{#2}}{#3}%
  \edef\x{%
    \noexpand\Test{%
      \noexpand\BigIntCalcMul
      \bigintcalcAbs{#1}!\bigintcalcAbs{#2}!%
    }{\bigintcalcAbs{#3}}%
  }%
  \x
}
\newcommand*{\TestSqr}[2]{%
  \Test{\bigintcalcSqr{#1}}{#2}%
}
\newcommand*{\TestFac}[2]{%
  \expandafter\TestExch\expandafter{%
    \the\numexpr#2%
  }{\bigintcalcFac{#1}}%
}
\newcommand*{\TestFacBig}[2]{%
  \Test{\bigintcalcFac{#1}}{#2}%
}
\newcommand*{\TestPow}[3]{%
  \Test{\bigintcalcPow{#1}{#2}}{#3}%
}
\newcommand*{\TestDiv}[3]{%
  \Test{\bigintcalcDiv{#1}{#2}}{#3}%
  \TestTeXDivide{#1}{#2}%
}
\newcommand*{\TestDivBig}[3]{%
  \Test{\bigintcalcDiv{#1}{#2}}{#3}%
  \edef\x{%
    \noexpand\Test{%
      \noexpand\BigIntCalcDiv\bigintcalcAbs{#1}!\bigintcalcAbs{#2}!%
    }{\bigintcalcAbs{#3}}%
  }%
}
\newcommand*{\TestMod}[3]{%
  \Test{\bigintcalcMod{#1}{#2}}{#3}%
  \ifcase\ifcase\bigintcalcSgn{#1} 0%
         \or
           \ifcase\bigintcalcSgn{#2} 1%
           \or 0%
           \else 1%
           \fi
         \else
           \ifcase\bigintcalcSgn{#2} 1%
           \or 1%
           \else 0%
           \fi
         \fi\relax
    \edef\x{%
      \noexpand\Test{%
        \noexpand\BigIntCalcMod
        \bigintcalcAbs{#1}!\bigintcalcAbs{#2}!%
      }{\bigintcalcAbs{#3}}%
    }%
    \x
  \fi
}
%    \end{macrocode}
%
% \subsubsection{Time}
%
%    \begin{macrocode}
\begingroup\expandafter\expandafter\expandafter\endgroup
\expandafter\ifx\csname pdfresettimer\endcsname\relax
\else
  \makeatletter
  \newcount\SummaryTime
  \newcount\TestTime
  \SummaryTime=\z@
  \newcommand*{\PrintTime}[2]{%
    \typeout{%
      [Time #1: \strip@pt\dimexpr\number#2sp\relax\space s]%
    }%
  }%
  \newcommand*{\StartTime}[1]{%
    \renewcommand*{\TimeDescription}{#1}%
    \pdfresettimer
  }%
  \newcommand*{\TimeDescription}{}%
  \newcommand*{\StopTime}{%
    \TestTime=\pdfelapsedtime
    \global\advance\SummaryTime\TestTime
    \PrintTime\TimeDescription\TestTime
  }%
  \let\saved@qstest\qstest
  \let\saved@endqstest\endqstest
  \def\qstest#1#2{%
    \saved@qstest{#1}{#2}%
    \StartTime{#1}%
  }%
  \def\endqstest{%
    \StopTime
    \saved@endqstest
  }%
  \AtEndDocument{%
    \PrintTime{summary}\SummaryTime
  }%
  \makeatother
\fi
%    \end{macrocode}
%
% \subsubsection{Test sets}
%
%    \begin{macrocode}
\makeatletter

\begin{qstest}{inv}{inv}%
  \TestInv{0}{0}%
  \TestInv{1}{-1}%
  \TestInv{-1}{1}%
  \TestInv{10}{-10}%
  \TestInv{-10}{10}%
  \TestInv{2147483647}{-2147483647}%
  \TestInv{-2147483647}{2147483647}%
  \TestInv{12345678901234567890}{-12345678901234567890}%
  \TestInv{-12345678901234567890}{12345678901234567890}%
  \TestInv{ 0 }{0}%
  \TestInv{ 1 }{-1}%
  \TestInv{--1}{-1}%
  \TestInv{\number\z@}{0}%
  \TestInv{\ifx\relax\relax1\fi}{-1}%
  \TestInv{\ifx\relax\relax-\fi\ifx234\else1\fi}{1}%
\end{qstest}

\begin{qstest}{abs}{abs}%
  \TestAbs{0}{0}%
  \TestAbs{1}{1}%
  \TestAbs{-1}{1}%
  \TestAbs{10}{10}%
  \TestAbs{-10}{10}%
  \TestAbs{2147483647}{2147483647}%
  \TestAbs{-2147483647}{2147483647}%
  \TestAbs{12345678901234567890}{12345678901234567890}%
  \TestAbs{-12345678901234567890}{12345678901234567890}%
  \TestAbs{ 0 }{0}%
  \TestAbs{ 1 }{1}%
  \TestAbs{--1}{1}%
  \TestAbs{-+-+1}{1}%
  \TestAbs{00000000000}{0}%
  \TestAbs{00000001000}{1000}%
  \TestAbs{\ifx\relax\relax 0\else 1\fi}{0}%
\end{qstest}

\begin{qstest}{sign}{sign}%
  \TestSgn{0}{0}%
  \TestSgn{1}{1}%
  \TestSgn{-1}{-1}%
  \TestSgn{10}{1}%
  \TestSgn{-10}{-1}%
  \TestSgn{2147483647}{1}%
  \TestSgn{-2147483647}{-1}%
  \TestSgn{12345678901234567890}{1}%
  \TestSgn{-12345678901234567890}{-1}%
  \TestSgn{ 0 }{0}%
  \TestSgn{ 2 }{1}%
  \TestSgn{ -2 }{-1}%
  \TestSgn{--2}{1}%
  \TestSgn{\number\z@}{0}%
  \TestSgn{\number\@ne}{1}%
  \TestSgn{\number\m@ne}{-1}%
  \TestSgn{%
    -+-+\number\z@\number\z@
    \iftrue1\fi\iftrue2\fi\iftrue3\fi
  }{1}%
\end{qstest}

\begin{qstest}{min}{min}%
  \TestMin{0}{1}{0}%
  \TestMin{1}{0}{0}%
  \TestMin{-10}{-20}{-20}%
  \TestMin{ 1 }{ 2 }{1}%
  \TestMin{ 2 }{ 1 }{1}%
  \TestMin{1}{1}{1}%
  \TestMin{\number\z@}{\number\@ne}{0}%
  \TestMin{\number\@ne}{\number\m@ne}{-1}%
\end{qstest}

\begin{qstest}{max}{max}%
  \TestMax{0}{1}{1}%
  \TestMax{1}{0}{1}%
  \TestMax{-10}{-20}{-10}%
  \TestMax{ 1 }{ 2 }{2}%
  \TestMax{ 2 }{ 1 }{2}%
  \TestMax{1}{1}{1}%
  \TestMax{\number\z@}{\number\@ne}{1}%
  \TestMax{\number\@ne}{\number\m@ne}{1}%
\end{qstest}

\begin{qstest}{cmp}{cmp}%
  \TestCmp{0}{0}{0}%
  \TestCmp{-21}{17}{-1}%
  \TestCmp{3}{4}{-1}%
  \TestCmp{-10}{-10}{0}%
  \TestCmp{-10}{-11}{1}%
  \TestCmp{100}{5}{1}%
  \TestCmp{9}{10}{-1}%
  \TestCmp{10}{9}{1}%
  \TestCmp{ 3 }{ 3 }{0}%
  \TestCmp{-9}{-10}{1}%
  \TestCmp{-10}{-9}{-1}%
  \TestCmp{-3}{-3}{0}%
  \TestCmp{0}{-2}{1}%
  \TestCmp{0}{2}{-1}%
  \TestCmp{2}{0}{1}%
  \TestCmp{-2}{0}{-1}%
  \TestCmp{12}{11}{1}%
  \TestCmp{11}{12}{-1}%
  \TestCmp{2147483647}{-2147483647}{1}%
  \TestCmp{-2147483647}{2147483647}{-1}%
  \TestCmp{2147483647}{2147483647}{0}%
  \TestCmp{\number\z@}{\number\@ne}{-1}%
  \TestCmp{\number\@ne}{\number\m@ne}{1}%
  \TestCmp{ 4 }{ 5 }{-1}%
  \TestCmp{ -3 }{ -7 }{1}%
\end{qstest}

\begin{qstest}{odd}{odd}
\tracingmacros=1
  \TestOdd{0}{0}%
  \TestOdd{1}{1}%
  \TestOdd{2}{0}%
  \TestOdd{3}{1}%
  \TestOdd{14}{0}%
  \TestOdd{15}{1}%
  \TestOdd{12345678901234567896}{0}%
  \TestOdd{12345678901234567897}{1}%
\end{qstest}

\begin{qstest}{inc}{inc}%
  \TestInc{0}{1}%
  \TestInc{1}{2}%
  \TestInc{-1}{0}%
  \TestInc{10}{11}%
  \TestInc{-10}{-9}%
  \TestInc{ 3 }{4}%
  \TestInc{999}{1000}%
  \TestInc{-1000}{-999}%
  \TestInc{129}{130}%
  \TestInc{2147483646}{2147483647}%
  \TestInc{-2147483647}{-2147483646}%
  \TestInc{12345678901234567890}{12345678901234567891}%
  \TestInc{99999999999999999999}{100000000000000000000}%
  \TestInc{-12345678901234567891}{-12345678901234567890}%
  \TestInc{-100000000000000000000}{-99999999999999999999}%
\end{qstest}

\begin{qstest}{dec}{dec}%
  \TestDec{0}{-1}%
  \TestDec{1}{0}%
  \TestDec{-1}{-2}%
  \TestDec{10}{9}%
  \TestDec{-10}{-11}%
  \TestDec{1000}{999}%
  \TestDec{-999}{-1000}%
  \TestDec{130}{129}%
  \TestDec{2147483647}{2147483646}%
  \TestDec{-2147483646}{-2147483647}%
  \TestDec{12345678901234567891}{12345678901234567890}%
  \TestDec{100000000000000000000}{99999999999999999999}%
  \TestDec{-12345678901234567890}{-12345678901234567891}%
  \TestDec{-99999999999999999999}{-100000000000000000000}%
\end{qstest}

\begin{qstest}{add}{add}%
  \TestAdd{0}{0}{0}%
  \TestAdd{1}{0}{1}%
  \TestAdd{0}{1}{1}%
  \TestAdd{1}{2}{3}%
  \TestAdd{-1}{-1}{-2}%
  \TestAdd{2147483646}{1}{2147483647}%
  \TestAdd{-2147483647}{2147483647}{0}%
  \TestAdd{20}{-5}{15}%
  \TestAdd{-4}{-1}{-5}%
  \TestAdd{-1}{-4}{-5}%
  \TestAdd{-4}{1}{-3}%
  \TestAdd{-1}{4}{3}%
  \TestAdd{4}{-1}{3}%
  \TestAdd{1}{-4}{-3}%
  \TestAdd{-4}{-1}{-5}%
  \TestAdd{-1}{-4}{-5}%
  \TestAdd{ -4 }{ -1 }{-5}%
  \TestAdd{ -1 }{ -4 }{-5}%
  \TestAdd{ -4 }{ 1 }{-3}%
  \TestAdd{ -1 }{ 4 }{3}%
  \TestAdd{ 4 }{ -1 }{3}%
  \TestAdd{ 1 }{ -4 }{-3}%
  \TestAdd{ -4 }{ -1 }{-5}%
  \TestAdd{ -1 }{ -4 }{-5}%
  \TestAdd{876543210}{111111111}{987654321}%
  \TestAdd{999999999}{2}{1000000001}%
\end{qstest}

\begin{qstest}{sub}{sub}
  \TestSub{0}{0}{0}%
  \TestSub{1}{0}{1}%
  \TestSub{1}{2}{-1}%
  \TestSub{-1}{-1}{0}%
  \TestSub{2147483646}{-1}{2147483647}%
  \TestSub{-2147483647}{-2147483647}{0}%
  \TestSub{-4}{-1}{-3}%
  \TestSub{-1}{-4}{3}%
  \TestSub{-4}{1}{-5}%
  \TestSub{-1}{4}{-5}%
  \TestSub{4}{-1}{5}%
  \TestSub{1}{-4}{5}%
  \TestSub{-4}{-1}{-3}%
  \TestSub{-1}{-4}{3}%
  \TestSub{ -4 }{ -1 }{-3}%
  \TestSub{ -1 }{ -4 }{3}%
  \TestSub{ -4 }{ 1 }{-5}%
  \TestSub{ -1 }{ 4 }{-5}%
  \TestSub{ 4 }{ -1 }{5}%
  \TestSub{ 1 }{ -4 }{5}%
  \TestSub{ -4 }{ -1 }{-3}%
  \TestSub{ -1 }{ -4 }{3}%
  \TestSub{1000000000}{2}{999999998}%
  \TestSub{987654321}{111111111}{876543210}%
\end{qstest}

\begin{qstest}{shl}{shl}
  \TestShl{0}{0}%
  \TestShl{1}{2}%
  \TestShl{2}{4}%
  \TestShl{5621}{11242}%
  \TestShl{1073741823}{2147483646}%
\end{qstest}

\begin{qstest}{shr}{shr}
  \TestShr{0}{0}%
  \TestShr{1}{0}%
  \TestShr{2}{1}%
  \TestShr{3}{1}%
  \TestShr{4}{2}%
  \TestShr{5}{2}%
  \TestShr{6}{3}%
  \TestShr{7}{3}%
  \TestShr{8}{4}%
  \TestShr{9}{4}%
  \TestShr{10}{5}%
  \TestShr{11}{5}%
  \TestShr{12}{6}%
  \TestShr{13}{6}%
  \TestShr{14}{7}%
  \TestShr{15}{7}%
  \TestShr{16}{8}%
  \TestShr{17}{8}%
  \TestShr{18}{9}%
  \TestShr{19}{9}%
  \TestShr{20}{10}%
  \TestShr{21}{10}%
  \TestShr{22}{11}%
  \TestShr{11241}{5620}%
  \TestShr{73054202}{36527101}%
  \TestShr{2147483646}{1073741823}%
\end{qstest}

\begin{qstest}{mul}{mul}
  \TestMul{0}{0}{0}%
  \TestMul{1}{0}{0}%
  \TestMul{0}{1}{0}%
  \TestMul{1}{1}{1}%
  \TestMul{3}{1}{3}%
  \TestMul{1}{-3}{-3}%
  \TestMul{-4}{-5}{20}%
  \TestMul{3}{7}{21}%
  \TestMul{7}{3}{21}%
  \TestMul{3}{-7}{-21}%
  \TestMul{7}{-3}{-21}%
  \TestMul{-3}{7}{-21}%
  \TestMul{-7}{3}{-21}%
  \TestMul{-3}{-7}{21}%
  \TestMul{-7}{-3}{21}%
  \TestMul{12}{11}{132}%
  \TestMul{999}{333}{332667}%
  \TestMul{1000}{4321}{4321000}%
  \TestMul{12345}{173955}{2147474475}%
  \TestMul{1073741823}{2}{2147483646}%
  \TestMul{2}{1073741823}{2147483646}%
  \TestMul{-1073741823}{2}{-2147483646}%
  \TestMul{2}{-1073741823}{-2147483646}%
  \TestMul{6706022400}{13}{87178291200}%
\end{qstest}

\begin{qstest}{sqr}{sqr}
  \TestSqr{0}{0}%
  \TestSqr{1}{1}%
  \TestSqr{2}{4}%
  \TestSqr{3}{9}%
  \TestSqr{4}{16}%
  \TestSqr{9}{81}%
  \TestSqr{10}{100}%
  \TestSqr{46340}{2147395600}%
  \TestSqr{-1}{1}%
  \TestSqr{-2}{4}%
  \TestSqr{-46340}{2147395600}%
\end{qstest}

\begin{qstest}{fac}{fac}
  \TestFac{0}{1}%
  \TestFac{1}{1}%
  \TestFac{2}{2}%
  \TestFac{3}{2*3}%
  \TestFac{4}{2*3*4}%
  \TestFac{5}{2*3*4*5}%
  \TestFac{6}{2*3*4*5*6}%
  \TestFac{7}{2*3*4*5*6*7}%
  \TestFac{8}{2*3*4*5*6*7*8}%
  \TestFac{9}{2*3*4*5*6*7*8*9}%
  \TestFac{10}{2*3*4*5*6*7*8*9*10}%
  \TestFac{11}{2*3*4*5*6*7*8*9*10*11}%
  \TestFac{12}{2*3*4*5*6*7*8*9*10*11*12}%
  \TestFacBig{13}{6227020800}%
  \TestFacBig{14}{87178291200}%
  \TestFacBig{15}{1307674368000}%
  \TestFacBig{16}{20922789888000}%
  \TestFacBig{17}{355687428096000}%
  \TestFacBig{18}{6402373705728000}%
  \TestFacBig{19}{121645100408832000}%
  \TestFacBig{20}{2432902008176640000}%
  \TestFacBig{21}{51090942171709440000}%
  \TestFacBig{22}{1124000727777607680000}%
\end{qstest}

\begin{qstest}{pow}{pow}
  \TestPow{-2}{0}{1}%
  \TestPow{-1}{0}{1}%
  \TestPow{0}{0}{1}%
  \TestPow{1}{0}{1}%
  \TestPow{2}{0}{1}%
  \TestPow{3}{0}{1}%
  \TestPow{-2}{1}{-2}%
  \TestPow{-1}{1}{-1}%
  \TestPow{1}{1}{1}%
  \TestPow{2}{1}{2}%
  \TestPow{3}{1}{3}%
  \TestPow{-2}{2}{4}%
  \TestPow{-1}{2}{1}%
  \TestPow{0}{2}{0}%
  \TestPow{1}{2}{1}%
  \TestPow{2}{2}{4}%
  \TestPow{3}{2}{9}%
  \TestPow{0}{1}{0}%
  \TestPow{1}{-2}{1}%
  \TestPow{1}{-1}{1}%
  \TestPow{-1}{-2}{1}%
  \TestPow{-1}{-1}{-1}%
  \TestPow{-1}{3}{-1}%
  \TestPow{-1}{4}{1}%
  \TestPow{-2}{-1}{0}%
  \TestPow{-2}{-2}{0}%
  \TestPow{2}{3}{8}%
  \TestPow{2}{4}{16}%
  \TestPow{2}{5}{32}%
  \TestPow{2}{6}{64}%
  \TestPow{2}{7}{128}%
  \TestPow{2}{8}{256}%
  \TestPow{2}{9}{512}%
  \TestPow{2}{10}{1024}%
  \TestPow{-2}{3}{-8}%
  \TestPow{-2}{4}{16}%
  \TestPow{-2}{5}{-32}%
  \TestPow{-2}{6}{64}%
  \TestPow{-2}{7}{-128}%
  \TestPow{-2}{8}{256}%
  \TestPow{-2}{9}{-512}%
  \TestPow{-2}{10}{1024}%
  \TestPow{3}{3}{27}%
  \TestPow{3}{4}{81}%
  \TestPow{3}{5}{243}%
  \TestPow{-3}{3}{-27}%
  \TestPow{-3}{4}{81}%
  \TestPow{-3}{5}{-243}%
  \TestPow{2}{30}{1073741824}%
  \TestPow{-3}{19}{-1162261467}%
  \TestPow{5}{13}{1220703125}%
  \TestPow{-7}{11}{-1977326743}%
\end{qstest}

\begin{qstest}{div}{div}
  \TestDiv{1}{1}{1}%
  \TestDiv{2}{1}{2}%
  \TestDiv{-2}{1}{-2}%
  \TestDiv{2}{-1}{-2}%
  \TestDiv{-2}{-1}{2}%
  \TestDiv{15}{2}{7}%
  \TestDiv{-16}{2}{-8}%
  \TestDiv{1}{2}{0}%
  \TestDiv{1}{3}{0}%
  \TestDiv{2}{3}{0}%
  \TestDiv{-2}{3}{0}%
  \TestDiv{2}{-3}{0}%
  \TestDiv{-2}{-3}{0}%
  \TestDiv{13}{3}{4}%
  \TestDiv{-13}{-3}{4}%
  \TestDiv{-13}{3}{-4}%
  \TestDiv{-6}{5}{-1}%
  \TestDiv{-5}{5}{-1}%
  \TestDiv{-4}{5}{0}%
  \TestDiv{-3}{5}{0}%
  \TestDiv{-2}{5}{0}%
  \TestDiv{-1}{5}{0}%
  \TestDiv{0}{5}{0}%
  \TestDiv{1}{5}{0}%
  \TestDiv{2}{5}{0}%
  \TestDiv{3}{5}{0}%
  \TestDiv{4}{5}{0}%
  \TestDiv{5}{5}{1}%
  \TestDiv{6}{5}{1}%
  \TestDiv{-5}{4}{-1}%
  \TestDiv{-4}{4}{-1}%
  \TestDiv{-3}{4}{0}%
  \TestDiv{-2}{4}{0}%
  \TestDiv{-1}{4}{0}%
  \TestDiv{0}{4}{0}%
  \TestDiv{1}{4}{0}%
  \TestDiv{2}{4}{0}%
  \TestDiv{3}{4}{0}%
  \TestDiv{4}{4}{1}%
  \TestDiv{5}{4}{1}%
  \TestDiv{12345}{678}{18}%
  \TestDiv{32372}{5952}{5}%
  \TestDiv{284271294}{18162}{15651}%
  \TestDiv{217652429}{12561}{17327}%
  \TestDiv{462028434}{5439}{84947}%
  \TestDiv{2147483647}{1000}{2147483}%
  \TestDiv{2147483647}{-1000}{-2147483}%
  \TestDiv{-2147483647}{1000}{-2147483}%
  \TestDiv{-2147483647}{-1000}{2147483}%
  \TestDiv{0}{3}{0}%
  \TestDiv{1}{3}{0}%
  \TestDiv{2}{3}{0}%
  \TestDiv{3}{3}{1}%
  \TestDiv{4}{3}{1}%
  \TestDiv{5}{3}{1}%
  \TestDiv{6}{3}{2}%
  \TestDiv{7}{3}{2}%
  \TestDiv{8}{3}{2}%
  \TestDiv{9}{3}{3}%
  \TestDiv{10}{3}{3}%
  \TestDiv{11}{3}{3}%
  \TestDiv{12}{3}{4}%
  \TestDiv{13}{3}{4}%
  \TestDiv{14}{3}{4}%
  \TestDiv{15}{3}{5}%
  \TestDiv{16}{3}{5}%
  \TestDiv{17}{3}{5}%
  \TestDiv{18}{3}{6}%
  \TestDiv{19}{3}{6}%
  \TestDiv{20}{3}{6}%
  \TestDiv{21}{3}{7}%
  \TestDiv{22}{3}{7}%
  \TestDiv{23}{3}{7}%
  \TestDiv{24}{3}{8}%
  \TestDiv{25}{3}{8}%
  \TestDiv{26}{3}{8}%
  \TestDiv{27}{3}{9}%
  \TestDiv{28}{3}{9}%
  \TestDiv{29}{3}{9}%
  \TestDiv{30}{3}{10}%
  \TestDiv{31}{3}{10}%
  \TestDivBig{17363436332507}{24702}{702916214}%
\end{qstest}

\begin{qstest}{mod}{mod}
  \TestMod{-6}{5}{4}%
  \TestMod{-5}{5}{0}%
  \TestMod{-4}{5}{1}%
  \TestMod{-3}{5}{2}%
  \TestMod{-2}{5}{3}%
  \TestMod{-1}{5}{4}%
  \TestMod{0}{5}{0}%
  \TestMod{1}{5}{1}%
  \TestMod{2}{5}{2}%
  \TestMod{3}{5}{3}%
  \TestMod{4}{5}{4}%
  \TestMod{5}{5}{0}%
  \TestMod{6}{5}{1}%
  \TestMod{-5}{4}{3}%
  \TestMod{-4}{4}{0}%
  \TestMod{-3}{4}{1}%
  \TestMod{-2}{4}{2}%
  \TestMod{-1}{4}{3}%
  \TestMod{0}{4}{0}%
  \TestMod{1}{4}{1}%
  \TestMod{2}{4}{2}%
  \TestMod{3}{4}{3}%
  \TestMod{4}{4}{0}%
  \TestMod{5}{4}{1}%
  \TestMod{-6}{-5}{-1}%
  \TestMod{-5}{-5}{0}%
  \TestMod{-4}{-5}{-4}%
  \TestMod{-3}{-5}{-3}%
  \TestMod{-2}{-5}{-2}%
  \TestMod{-1}{-5}{-1}%
  \TestMod{0}{-5}{0}%
  \TestMod{1}{-5}{-4}%
  \TestMod{2}{-5}{-3}%
  \TestMod{3}{-5}{-2}%
  \TestMod{4}{-5}{-1}%
  \TestMod{5}{-5}{0}%
  \TestMod{6}{-5}{-4}%
  \TestMod{-5}{-4}{-1}%
  \TestMod{-4}{-4}{0}%
  \TestMod{-3}{-4}{-3}%
  \TestMod{-2}{-4}{-2}%
  \TestMod{-1}{-4}{-1}%
  \TestMod{0}{-4}{0}%
  \TestMod{1}{-4}{-3}%
  \TestMod{2}{-4}{-2}%
  \TestMod{3}{-4}{-1}%
  \TestMod{4}{-4}{0}%
  \TestMod{5}{-4}{-3}%
  \TestMod{2147483647}{1000}{647}%
  \TestMod{2147483647}{-1000}{-353}%
  \TestMod{-2147483647}{1000}{353}%
  \TestMod{-2147483647}{-1000}{-647}%
  \TestMod{ 0 }{ 4 }{0}%
  \TestMod{ 1 }{ 4 }{1}%
  \TestMod{ -1 }{ 4 }{3}%
  \TestMod{ 0 }{ -4 }{0}%
  \TestMod{ 1 }{ -4 }{-3}%
  \TestMod{ -1 }{ -4 }{-1}%
  \TestMod{18362}{25}{12}%
\end{qstest}

\newcommand*{\TestError}[2]{%
  \begingroup
    \expandafter\def\csname BigIntCalcError:#1\endcsname{}%
    \Expect*{#2}{0}%
    \expandafter\def\csname BigIntCalcError:#1\endcsname{ERROR}%
    \Expect*{#2}{0ERROR}%
  \endgroup
}
\begin{qstest}{error}{error}
  \TestError{FacNegative}{\bigintcalcFac{-1}}%
  \TestError{FacNegative}{\bigintcalcFac{-2147483647}}%
  \TestError{DivisionByZero}{\bigintcalcPow{0}{-1}}%
  \TestError{DivisionByZero}{\bigintcalcDiv{1}{0}}%
  \TestError{DivisionByZero}{\bigintcalcMod{1}{0}}%
\end{qstest}

\begin{document}
\end{document}
%</test2>
%    \end{macrocode}
%
% \section{Installation}
%
% \subsection{Download}
%
% \paragraph{Package.} This package is available on
% CTAN\footnote{\url{http://ctan.org/pkg/bigintcalc}}:
% \begin{description}
% \item[\CTAN{macros/latex/contrib/oberdiek/bigintcalc.dtx}] The source file.
% \item[\CTAN{macros/latex/contrib/oberdiek/bigintcalc.pdf}] Documentation.
% \end{description}
%
%
% \paragraph{Bundle.} All the packages of the bundle `oberdiek'
% are also available in a TDS compliant ZIP archive. There
% the packages are already unpacked and the documentation files
% are generated. The files and directories obey the TDS standard.
% \begin{description}
% \item[\CTAN{install/macros/latex/contrib/oberdiek.tds.zip}]
% \end{description}
% \emph{TDS} refers to the standard ``A Directory Structure
% for \TeX\ Files'' (\CTAN{tds/tds.pdf}). Directories
% with \xfile{texmf} in their name are usually organized this way.
%
% \subsection{Bundle installation}
%
% \paragraph{Unpacking.} Unpack the \xfile{oberdiek.tds.zip} in the
% TDS tree (also known as \xfile{texmf} tree) of your choice.
% Example (linux):
% \begin{quote}
%   |unzip oberdiek.tds.zip -d ~/texmf|
% \end{quote}
%
% \paragraph{Script installation.}
% Check the directory \xfile{TDS:scripts/oberdiek/} for
% scripts that need further installation steps.
% Package \xpackage{attachfile2} comes with the Perl script
% \xfile{pdfatfi.pl} that should be installed in such a way
% that it can be called as \texttt{pdfatfi}.
% Example (linux):
% \begin{quote}
%   |chmod +x scripts/oberdiek/pdfatfi.pl|\\
%   |cp scripts/oberdiek/pdfatfi.pl /usr/local/bin/|
% \end{quote}
%
% \subsection{Package installation}
%
% \paragraph{Unpacking.} The \xfile{.dtx} file is a self-extracting
% \docstrip\ archive. The files are extracted by running the
% \xfile{.dtx} through \plainTeX:
% \begin{quote}
%   \verb|tex bigintcalc.dtx|
% \end{quote}
%
% \paragraph{TDS.} Now the different files must be moved into
% the different directories in your installation TDS tree
% (also known as \xfile{texmf} tree):
% \begin{quote}
% \def\t{^^A
% \begin{tabular}{@{}>{\ttfamily}l@{ $\rightarrow$ }>{\ttfamily}l@{}}
%   bigintcalc.sty & tex/generic/oberdiek/bigintcalc.sty\\
%   bigintcalc.pdf & doc/latex/oberdiek/bigintcalc.pdf\\
%   test/bigintcalc-test1.tex & doc/latex/oberdiek/test/bigintcalc-test1.tex\\
%   test/bigintcalc-test2.tex & doc/latex/oberdiek/test/bigintcalc-test2.tex\\
%   test/bigintcalc-test3.tex & doc/latex/oberdiek/test/bigintcalc-test3.tex\\
%   bigintcalc.dtx & source/latex/oberdiek/bigintcalc.dtx\\
% \end{tabular}^^A
% }^^A
% \sbox0{\t}^^A
% \ifdim\wd0>\linewidth
%   \begingroup
%     \advance\linewidth by\leftmargin
%     \advance\linewidth by\rightmargin
%   \edef\x{\endgroup
%     \def\noexpand\lw{\the\linewidth}^^A
%   }\x
%   \def\lwbox{^^A
%     \leavevmode
%     \hbox to \linewidth{^^A
%       \kern-\leftmargin\relax
%       \hss
%       \usebox0
%       \hss
%       \kern-\rightmargin\relax
%     }^^A
%   }^^A
%   \ifdim\wd0>\lw
%     \sbox0{\small\t}^^A
%     \ifdim\wd0>\linewidth
%       \ifdim\wd0>\lw
%         \sbox0{\footnotesize\t}^^A
%         \ifdim\wd0>\linewidth
%           \ifdim\wd0>\lw
%             \sbox0{\scriptsize\t}^^A
%             \ifdim\wd0>\linewidth
%               \ifdim\wd0>\lw
%                 \sbox0{\tiny\t}^^A
%                 \ifdim\wd0>\linewidth
%                   \lwbox
%                 \else
%                   \usebox0
%                 \fi
%               \else
%                 \lwbox
%               \fi
%             \else
%               \usebox0
%             \fi
%           \else
%             \lwbox
%           \fi
%         \else
%           \usebox0
%         \fi
%       \else
%         \lwbox
%       \fi
%     \else
%       \usebox0
%     \fi
%   \else
%     \lwbox
%   \fi
% \else
%   \usebox0
% \fi
% \end{quote}
% If you have a \xfile{docstrip.cfg} that configures and enables \docstrip's
% TDS installing feature, then some files can already be in the right
% place, see the documentation of \docstrip.
%
% \subsection{Refresh file name databases}
%
% If your \TeX~distribution
% (\teTeX, \mikTeX, \dots) relies on file name databases, you must refresh
% these. For example, \teTeX\ users run \verb|texhash| or
% \verb|mktexlsr|.
%
% \subsection{Some details for the interested}
%
% \paragraph{Attached source.}
%
% The PDF documentation on CTAN also includes the
% \xfile{.dtx} source file. It can be extracted by
% AcrobatReader 6 or higher. Another option is \textsf{pdftk},
% e.g. unpack the file into the current directory:
% \begin{quote}
%   \verb|pdftk bigintcalc.pdf unpack_files output .|
% \end{quote}
%
% \paragraph{Unpacking with \LaTeX.}
% The \xfile{.dtx} chooses its action depending on the format:
% \begin{description}
% \item[\plainTeX:] Run \docstrip\ and extract the files.
% \item[\LaTeX:] Generate the documentation.
% \end{description}
% If you insist on using \LaTeX\ for \docstrip\ (really,
% \docstrip\ does not need \LaTeX), then inform the autodetect routine
% about your intention:
% \begin{quote}
%   \verb|latex \let\install=y\input{bigintcalc.dtx}|
% \end{quote}
% Do not forget to quote the argument according to the demands
% of your shell.
%
% \paragraph{Generating the documentation.}
% You can use both the \xfile{.dtx} or the \xfile{.drv} to generate
% the documentation. The process can be configured by the
% configuration file \xfile{ltxdoc.cfg}. For instance, put this
% line into this file, if you want to have A4 as paper format:
% \begin{quote}
%   \verb|\PassOptionsToClass{a4paper}{article}|
% \end{quote}
% An example follows how to generate the
% documentation with pdf\LaTeX:
% \begin{quote}
%\begin{verbatim}
%pdflatex bigintcalc.dtx
%makeindex -s gind.ist bigintcalc.idx
%pdflatex bigintcalc.dtx
%makeindex -s gind.ist bigintcalc.idx
%pdflatex bigintcalc.dtx
%\end{verbatim}
% \end{quote}
%
% \section{Catalogue}
%
% The following XML file can be used as source for the
% \href{http://mirror.ctan.org/help/Catalogue/catalogue.html}{\TeX\ Catalogue}.
% The elements \texttt{caption} and \texttt{description} are imported
% from the original XML file from the Catalogue.
% The name of the XML file in the Catalogue is \xfile{bigintcalc.xml}.
%    \begin{macrocode}
%<*catalogue>
<?xml version='1.0' encoding='us-ascii'?>
<!DOCTYPE entry SYSTEM 'catalogue.dtd'>
<entry datestamp='$Date$' modifier='$Author$' id='bigintcalc'>
  <name>bigintcalc</name>
  <caption>Integer calculations on very large numbers.</caption>
  <authorref id='auth:oberdiek'/>
  <copyright owner='Heiko Oberdiek' year='2007,2011,2012'/>
  <license type='lppl1.3'/>
  <version number='1.4'/>
  <description>
    This package provides expandable arithmetic operations
    with big integers that can exceed TeX's number limits.
    <p/>
    The package is part of the <xref refid='oberdiek'>oberdiek</xref> bundle.
  </description>
  <documentation details='Package documentation'
      href='ctan:/macros/latex/contrib/oberdiek/bigintcalc.pdf'/>
  <ctan file='true' path='/macros/latex/contrib/oberdiek/bigintcalc.dtx'/>
  <miktex location='oberdiek'/>
  <texlive location='oberdiek'/>
  <install path='/macros/latex/contrib/oberdiek/oberdiek.tds.zip'/>
</entry>
%</catalogue>
%    \end{macrocode}
%
% \begin{History}
%   \begin{Version}{2007/09/27 v1.0}
%   \item
%     First version.
%   \end{Version}
%   \begin{Version}{2007/11/11 v1.1}
%   \item
%     Use of package \xpackage{pdftexcmds} for \LuaTeX\ support.
%   \end{Version}
%   \begin{Version}{2011/01/30 v1.2}
%   \item
%     Already loaded package files are not input in \hologo{plainTeX}.
%   \end{Version}
%   \begin{Version}{2012/04/08 v1.3}
%   \item
%     Fix: \xpackage{pdftexcmds} wasn't loaded in case of \hologo{LaTeX}.
%   \end{Version}
%   \begin{Version}{2016/05/16 v1.4}
%   \item
%     Documentation updates.
%   \end{Version}
% \end{History}
%
% \PrintIndex
%
% \Finale
\endinput
|
% \end{quote}
% Do not forget to quote the argument according to the demands
% of your shell.
%
% \paragraph{Generating the documentation.}
% You can use both the \xfile{.dtx} or the \xfile{.drv} to generate
% the documentation. The process can be configured by the
% configuration file \xfile{ltxdoc.cfg}. For instance, put this
% line into this file, if you want to have A4 as paper format:
% \begin{quote}
%   \verb|\PassOptionsToClass{a4paper}{article}|
% \end{quote}
% An example follows how to generate the
% documentation with pdf\LaTeX:
% \begin{quote}
%\begin{verbatim}
%pdflatex bigintcalc.dtx
%makeindex -s gind.ist bigintcalc.idx
%pdflatex bigintcalc.dtx
%makeindex -s gind.ist bigintcalc.idx
%pdflatex bigintcalc.dtx
%\end{verbatim}
% \end{quote}
%
% \section{Catalogue}
%
% The following XML file can be used as source for the
% \href{http://mirror.ctan.org/help/Catalogue/catalogue.html}{\TeX\ Catalogue}.
% The elements \texttt{caption} and \texttt{description} are imported
% from the original XML file from the Catalogue.
% The name of the XML file in the Catalogue is \xfile{bigintcalc.xml}.
%    \begin{macrocode}
%<*catalogue>
<?xml version='1.0' encoding='us-ascii'?>
<!DOCTYPE entry SYSTEM 'catalogue.dtd'>
<entry datestamp='$Date$' modifier='$Author$' id='bigintcalc'>
  <name>bigintcalc</name>
  <caption>Integer calculations on very large numbers.</caption>
  <authorref id='auth:oberdiek'/>
  <copyright owner='Heiko Oberdiek' year='2007,2011,2012'/>
  <license type='lppl1.3'/>
  <version number='1.4'/>
  <description>
    This package provides expandable arithmetic operations
    with big integers that can exceed TeX's number limits.
    <p/>
    The package is part of the <xref refid='oberdiek'>oberdiek</xref> bundle.
  </description>
  <documentation details='Package documentation'
      href='ctan:/macros/latex/contrib/oberdiek/bigintcalc.pdf'/>
  <ctan file='true' path='/macros/latex/contrib/oberdiek/bigintcalc.dtx'/>
  <miktex location='oberdiek'/>
  <texlive location='oberdiek'/>
  <install path='/macros/latex/contrib/oberdiek/oberdiek.tds.zip'/>
</entry>
%</catalogue>
%    \end{macrocode}
%
% \begin{History}
%   \begin{Version}{2007/09/27 v1.0}
%   \item
%     First version.
%   \end{Version}
%   \begin{Version}{2007/11/11 v1.1}
%   \item
%     Use of package \xpackage{pdftexcmds} for \LuaTeX\ support.
%   \end{Version}
%   \begin{Version}{2011/01/30 v1.2}
%   \item
%     Already loaded package files are not input in \hologo{plainTeX}.
%   \end{Version}
%   \begin{Version}{2012/04/08 v1.3}
%   \item
%     Fix: \xpackage{pdftexcmds} wasn't loaded in case of \hologo{LaTeX}.
%   \end{Version}
%   \begin{Version}{2016/05/16 v1.4}
%   \item
%     Documentation updates.
%   \end{Version}
% \end{History}
%
% \PrintIndex
%
% \Finale
\endinput

%        (quote the arguments according to the demands of your shell)
%
% Documentation:
%    (a) If bigintcalc.drv is present:
%           latex bigintcalc.drv
%    (b) Without bigintcalc.drv:
%           latex bigintcalc.dtx; ...
%    The class ltxdoc loads the configuration file ltxdoc.cfg
%    if available. Here you can specify further options, e.g.
%    use A4 as paper format:
%       \PassOptionsToClass{a4paper}{article}
%
%    Programm calls to get the documentation (example):
%       pdflatex bigintcalc.dtx
%       makeindex -s gind.ist bigintcalc.idx
%       pdflatex bigintcalc.dtx
%       makeindex -s gind.ist bigintcalc.idx
%       pdflatex bigintcalc.dtx
%
% Installation:
%    TDS:tex/generic/oberdiek/bigintcalc.sty
%    TDS:doc/latex/oberdiek/bigintcalc.pdf
%    TDS:doc/latex/oberdiek/test/bigintcalc-test1.tex
%    TDS:doc/latex/oberdiek/test/bigintcalc-test2.tex
%    TDS:doc/latex/oberdiek/test/bigintcalc-test3.tex
%    TDS:source/latex/oberdiek/bigintcalc.dtx
%
%<*ignore>
\begingroup
  \catcode123=1 %
  \catcode125=2 %
  \def\x{LaTeX2e}%
\expandafter\endgroup
\ifcase 0\ifx\install y1\fi\expandafter
         \ifx\csname processbatchFile\endcsname\relax\else1\fi
         \ifx\fmtname\x\else 1\fi\relax
\else\csname fi\endcsname
%</ignore>
%<*install>
\input docstrip.tex
\Msg{************************************************************************}
\Msg{* Installation}
\Msg{* Package: bigintcalc 2016/05/16 v1.4 Expandable calculations on big integers (HO)}
\Msg{************************************************************************}

\keepsilent
\askforoverwritefalse

\let\MetaPrefix\relax
\preamble

This is a generated file.

Project: bigintcalc
Version: 2016/05/16 v1.4

Copyright (C) 2007, 2011, 2012 by
   Heiko Oberdiek <heiko.oberdiek at googlemail.com>

This work may be distributed and/or modified under the
conditions of the LaTeX Project Public License, either
version 1.3c of this license or (at your option) any later
version. This version of this license is in
   http://www.latex-project.org/lppl/lppl-1-3c.txt
and the latest version of this license is in
   http://www.latex-project.org/lppl.txt
and version 1.3 or later is part of all distributions of
LaTeX version 2005/12/01 or later.

This work has the LPPL maintenance status "maintained".

This Current Maintainer of this work is Heiko Oberdiek.

The Base Interpreter refers to any `TeX-Format',
because some files are installed in TDS:tex/generic//.

This work consists of the main source file bigintcalc.dtx
and the derived files
   bigintcalc.sty, bigintcalc.pdf, bigintcalc.ins, bigintcalc.drv,
   bigintcalc-test1.tex, bigintcalc-test2.tex,
   bigintcalc-test3.tex.

\endpreamble
\let\MetaPrefix\DoubleperCent

\generate{%
  \file{bigintcalc.ins}{\from{bigintcalc.dtx}{install}}%
  \file{bigintcalc.drv}{\from{bigintcalc.dtx}{driver}}%
  \usedir{tex/generic/oberdiek}%
  \file{bigintcalc.sty}{\from{bigintcalc.dtx}{package}}%
%  \usedir{doc/latex/oberdiek/test}%
%  \file{bigintcalc-test1.tex}{\from{bigintcalc.dtx}{test1}}%
%  \file{bigintcalc-test2.tex}{\from{bigintcalc.dtx}{test2,etex}}%
%  \file{bigintcalc-test3.tex}{\from{bigintcalc.dtx}{test2,noetex}}%
  \nopreamble
  \nopostamble
  \usedir{source/latex/oberdiek/catalogue}%
  \file{bigintcalc.xml}{\from{bigintcalc.dtx}{catalogue}}%
}

\catcode32=13\relax% active space
\let =\space%
\Msg{************************************************************************}
\Msg{*}
\Msg{* To finish the installation you have to move the following}
\Msg{* file into a directory searched by TeX:}
\Msg{*}
\Msg{*     bigintcalc.sty}
\Msg{*}
\Msg{* To produce the documentation run the file `bigintcalc.drv'}
\Msg{* through LaTeX.}
\Msg{*}
\Msg{* Happy TeXing!}
\Msg{*}
\Msg{************************************************************************}

\endbatchfile
%</install>
%<*ignore>
\fi
%</ignore>
%<*driver>
\NeedsTeXFormat{LaTeX2e}
\ProvidesFile{bigintcalc.drv}%
  [2016/05/16 v1.4 Expandable calculations on big integers (HO)]%
\documentclass{ltxdoc}
\usepackage{wasysym}
\let\iint\relax
\let\iiint\relax
\usepackage[fleqn]{amsmath}

\DeclareMathOperator{\opInv}{Inv}
\DeclareMathOperator{\opAbs}{Abs}
\DeclareMathOperator{\opSgn}{Sgn}
\DeclareMathOperator{\opMin}{Min}
\DeclareMathOperator{\opMax}{Max}
\DeclareMathOperator{\opCmp}{Cmp}
\DeclareMathOperator{\opOdd}{Odd}
\DeclareMathOperator{\opInc}{Inc}
\DeclareMathOperator{\opDec}{Dec}
\DeclareMathOperator{\opAdd}{Add}
\DeclareMathOperator{\opSub}{Sub}
\DeclareMathOperator{\opShl}{Shl}
\DeclareMathOperator{\opShr}{Shr}
\DeclareMathOperator{\opMul}{Mul}
\DeclareMathOperator{\opSqr}{Sqr}
\DeclareMathOperator{\opFac}{Fac}
\DeclareMathOperator{\opPow}{Pow}
\DeclareMathOperator{\opDiv}{Div}
\DeclareMathOperator{\opMod}{Mod}
\DeclareMathOperator{\opInt}{Int}
\DeclareMathOperator{\opODD}{ifodd}

\newcommand*{\Def}{%
  \ensuremath{%
    \mathrel{\mathop{:}}=%
  }%
}
\newcommand*{\op}[1]{%
  \textsf{#1}%
}
\usepackage{holtxdoc}[2011/11/22]
\begin{document}
  \DocInput{bigintcalc.dtx}%
\end{document}
%</driver>
% \fi
%
%
% \CharacterTable
%  {Upper-case    \A\B\C\D\E\F\G\H\I\J\K\L\M\N\O\P\Q\R\S\T\U\V\W\X\Y\Z
%   Lower-case    \a\b\c\d\e\f\g\h\i\j\k\l\m\n\o\p\q\r\s\t\u\v\w\x\y\z
%   Digits        \0\1\2\3\4\5\6\7\8\9
%   Exclamation   \!     Double quote  \"     Hash (number) \#
%   Dollar        \$     Percent       \%     Ampersand     \&
%   Acute accent  \'     Left paren    \(     Right paren   \)
%   Asterisk      \*     Plus          \+     Comma         \,
%   Minus         \-     Point         \.     Solidus       \/
%   Colon         \:     Semicolon     \;     Less than     \<
%   Equals        \=     Greater than  \>     Question mark \?
%   Commercial at \@     Left bracket  \[     Backslash     \\
%   Right bracket \]     Circumflex    \^     Underscore    \_
%   Grave accent  \`     Left brace    \{     Vertical bar  \|
%   Right brace   \}     Tilde         \~}
%
% \GetFileInfo{bigintcalc.drv}
%
% \title{The \xpackage{bigintcalc} package}
% \date{2016/05/16 v1.4}
% \author{Heiko Oberdiek\thanks
% {Please report any issues at https://github.com/ho-tex/oberdiek/issues}\\
% \xemail{heiko.oberdiek at googlemail.com}}
%
% \maketitle
%
% \begin{abstract}
% This package provides expandable arithmetic operations
% with big integers that can exceed \TeX's number limits.
% \end{abstract}
%
% \tableofcontents
%
% \section{Documentation}
%
% \subsection{Introduction}
%
% Package \xpackage{bigintcalc} defines arithmetic operations
% that deal with big integers. Big integers can be given
% either as explicit integer number or as macro code
% that expands to an explicit number. \emph{Big} means that
% there is no limit on the size of the number. Big integers
% may exceed \TeX's range limitation of -2147483647 and 2147483647.
% Only memory issues will limit the usable range.
%
% In opposite to package \xpackage{intcalc}
% unexpandable command tokens are not supported, even if
% they are valid \TeX\ numbers like count registers or
% commands created by \cs{chardef}. Nevertheless they may
% be used, if they are prefixed by \cs{number}.
%
% Also \eTeX's \cs{numexpr} expressions are not supported
% directly in the manner of package \xpackage{intcalc}.
% However they can be given if \cs{the}\cs{numexpr}
% or \cs{number}\cs{numexpr} are used.
%
% The operations have the form of macros that take one or
% two integers as parameter and return the integer result.
% The macro name is a three letter operation name prefixed
% by the package name, e.g. \cs{bigintcalcAdd}|{10}{43}| returns
% |53|.
%
% The macros are fully expandable, exactly two expansion
% steps generate the result. Therefore the operations
% may be used nearly everywhere in \TeX, even inside
% \cs{csname}, file names,  or other
% expandable contexts.
%
% \subsection{Conditions}
%
% \subsubsection{Preconditions}
%
% \begin{itemize}
% \item
%   Arguments can be anything that expands to a number that consists
%   of optional signs and digits.
% \item
%   The arguments and return values must be sound.
%   Zero as divisor or factorials of negative numbers will cause errors.
% \end{itemize}
%
% \subsubsection{Postconditions}
%
% Additional properties of the macros apart from calculating
% a correct result (of course \smiley):
% \begin{itemize}
% \item
%   The macros are fully expandable. Thus they can
%   be used inside \cs{edef}, \cs{csname},
%   for example.
% \item
%   Furthermore exactly two expansion steps calculate the result.
% \item
%   The number consists of one optional minus sign and one or more
%   digits. The first digit is larger than zero for
%   numbers that consists of more than one digit.
%
%   In short, the number format is exactly the same as
%   \cs{number} generates, but without its range limitation.
%   And the tokens (minus sign, digits)
%   have catcode 12 (other).
% \item
%   Call by value is simulated. First the arguments are
%   converted to numbers. Then these numbers are used
%   in the calculations.
%
%   Remember that arguments
%   may contain expensive macros or \eTeX\ expressions.
%   This strategy avoids multiple evaluations of such
%   arguments.
% \end{itemize}
%
% \subsection{Error handling}
%
% Some errors are detected by the macros, example: division by zero.
% In this cases an undefined control sequence is called and causes
% a TeX error message, example: \cs{BigIntCalcError:DivisionByZero}.
% The name of the control sequence contains
% the reason for the error. The \TeX\ error may be ignored.
% Then the operation returns zero as result.
% Because the macros are supposed to work in expandible contexts.
% An traditional error message, however, is not expandable and
% would break these contexts.
%
% \subsection{Operations}
%
% Some definition equations below use the function $\opInt$
% that converts a real number to an integer. The number
% is truncated that means rounding to zero:
% \begin{gather*}
%   \opInt(x) \Def
%   \begin{cases}
%     \lfloor x\rfloor & \text{if $x\geq0$}\\
%     \lceil x\rceil & \text{otherwise}
%   \end{cases}
% \end{gather*}
%
% \subsubsection{\op{Num}}
%
% \begin{declcs}{bigintcalcNum} \M{x}
% \end{declcs}
% Macro \cs{bigintcalcNum} converts its argument to a normalized integer
% number without unnecessary leading zeros or signs.
% The result matches the regular expression:
%\begin{quote}
%\begin{verbatim}
%0|-?[1-9][0-9]*
%\end{verbatim}
%\end{quote}
%
% \subsubsection{\op{Inv}, \op{Abs}, \op{Sgn}}
%
% \begin{declcs}{bigintcalcInv} \M{x}
% \end{declcs}
% Macro \cs{bigintcalcInv} switches the sign.
% \begin{gather*}
%   \opInv(x) \Def -x
% \end{gather*}
%
% \begin{declcs}{bigintcalcAbs} \M{x}
% \end{declcs}
% Macro \cs{bigintcalcAbs} returns the absolute value
% of integer \meta{x}.
% \begin{gather*}
%   \opAbs(x) \Def \vert x\vert
% \end{gather*}
%
% \begin{declcs}{bigintcalcSgn} \M{x}
% \end{declcs}
% Macro \cs{bigintcalcSgn} encodes the sign of \meta{x} as number.
% \begin{gather*}
%   \opSgn(x) \Def
%   \begin{cases}
%     -1& \text{if $x<0$}\\
%      0& \text{if $x=0$}\\
%      1& \text{if $x>0$}
%   \end{cases}
% \end{gather*}
% These return values can easily be distinguished by \cs{ifcase}:
%\begin{quote}
%\begin{verbatim}
%\ifcase\bigintcalcSgn{<x>}
%  $x=0$
%\or
%  $x>0$
%\else
%  $x<0$
%\fi
%\end{verbatim}
%\end{quote}
%
% \subsubsection{\op{Min}, \op{Max}, \op{Cmp}}
%
% \begin{declcs}{bigintcalcMin} \M{x} \M{y}
% \end{declcs}
% Macro \cs{bigintcalcMin} returns the smaller of the two integers.
% \begin{gather*}
%   \opMin(x,y) \Def
%   \begin{cases}
%     x & \text{if $x<y$}\\
%     y & \text{otherwise}
%   \end{cases}
% \end{gather*}
%
% \begin{declcs}{bigintcalcMax} \M{x} \M{y}
% \end{declcs}
% Macro \cs{bigintcalcMax} returns the larger of the two integers.
% \begin{gather*}
%   \opMax(x,y) \Def
%   \begin{cases}
%     x & \text{if $x>y$}\\
%     y & \text{otherwise}
%   \end{cases}
% \end{gather*}
%
% \begin{declcs}{bigintcalcCmp} \M{x} \M{y}
% \end{declcs}
% Macro \cs{bigintcalcCmp} encodes the comparison result as number:
% \begin{gather*}
%   \opCmp(x,y) \Def
%   \begin{cases}
%     -1 & \text{if $x<y$}\\
%      0 & \text{if $x=y$}\\
%      1 & \text{if $x>y$}
%   \end{cases}
% \end{gather*}
% These values can be distinguished by \cs{ifcase}:
%\begin{quote}
%\begin{verbatim}
%\ifcase\bigintcalcCmp{<x>}{<y>}
%  $x=y$
%\or
%  $x>y$
%\else
%  $x<y$
%\fi
%\end{verbatim}
%\end{quote}
%
% \subsubsection{\op{Odd}}
%
% \begin{declcs}{bigintcalcOdd} \M{x}
% \end{declcs}
% \begin{gather*}
%   \opOdd(x) \Def
%   \begin{cases}
%     1 & \text{if $x$ is odd}\\
%     0 & \text{if $x$ is even}
%   \end{cases}
% \end{gather*}
%
% \subsubsection{\op{Inc}, \op{Dec}, \op{Add}, \op{Sub}}
%
% \begin{declcs}{bigintcalcInc} \M{x}
% \end{declcs}
% Macro \cs{bigintcalcInc} increments \meta{x} by one.
% \begin{gather*}
%   \opInc(x) \Def x + 1
% \end{gather*}
%
% \begin{declcs}{bigintcalcDec} \M{x}
% \end{declcs}
% Macro \cs{bigintcalcDec} decrements \meta{x} by one.
% \begin{gather*}
%   \opDec(x) \Def x - 1
% \end{gather*}
%
% \begin{declcs}{bigintcalcAdd} \M{x} \M{y}
% \end{declcs}
% Macro \cs{bigintcalcAdd} adds the two numbers.
% \begin{gather*}
%   \opAdd(x, y) \Def x + y
% \end{gather*}
%
% \begin{declcs}{bigintcalcSub} \M{x} \M{y}
% \end{declcs}
% Macro \cs{bigintcalcSub} calculates the difference.
% \begin{gather*}
%   \opSub(x, y) \Def x - y
% \end{gather*}
%
% \subsubsection{\op{Shl}, \op{Shr}}
%
% \begin{declcs}{bigintcalcShl} \M{x}
% \end{declcs}
% Macro \cs{bigintcalcShl} implements shifting to the left that
% means the number is multiplied by two.
% The sign is preserved.
% \begin{gather*}
%   \opShl(x) \Def x*2
% \end{gather*}
%
% \begin{declcs}{bigintcalcShr} \M{x}
% \end{declcs}
% Macro \cs{bigintcalcShr} implements shifting to the right.
% That is equivalent to an integer division by two.
% The sign is preserved.
% \begin{gather*}
%   \opShr(x) \Def \opInt(x/2)
% \end{gather*}
%
% \subsubsection{\op{Mul}, \op{Sqr}, \op{Fac}, \op{Pow}}
%
% \begin{declcs}{bigintcalcMul} \M{x} \M{y}
% \end{declcs}
% Macro \cs{bigintcalcMul} calculates the product of
% \meta{x} and \meta{y}.
% \begin{gather*}
%   \opMul(x,y) \Def x*y
% \end{gather*}
%
% \begin{declcs}{bigintcalcSqr} \M{x}
% \end{declcs}
% Macro \cs{bigintcalcSqr} returns the square product.
% \begin{gather*}
%   \opSqr(x) \Def x^2
% \end{gather*}
%
% \begin{declcs}{bigintcalcFac} \M{x}
% \end{declcs}
% Macro \cs{bigintcalcFac} returns the factorial of \meta{x}.
% Negative numbers are not permitted.
% \begin{gather*}
%   \opFac(x) \Def x!\qquad\text{for $x\geq0$}
% \end{gather*}
% ($0! = 1$)
%
% \begin{declcs}{bigintcalcPow} M{x} M{y}
% \end{declcs}
% Macro \cs{bigintcalcPow} calculates the value of \meta{x} to the
% power of \meta{y}. The error ``division by zero'' is thrown
% if \meta{x} is zero and \meta{y} is negative.
% permitted:
% \begin{gather*}
%   \opPow(x,y) \Def
%   \opInt(x^y)\qquad\text{for $x\neq0$ or $y\geq0$}
% \end{gather*}
% ($0^0 = 1$)
%
% \subsubsection{\op{Div}, \op{Mul}}
%
% \begin{declcs}{bigintcalcDiv} \M{x} \M{y}
% \end{declcs}
% Macro \cs{bigintcalcDiv} performs an integer division.
% Argument \meta{y} must not be zero.
% \begin{gather*}
%   \opDiv(x,y) \Def \opInt(x/y)\qquad\text{for $y\neq0$}
% \end{gather*}
%
% \begin{declcs}{bigintcalcMod} \M{x} \M{y}
% \end{declcs}
% Macro \cs{bigintcalcMod} gets the remainder of the integer
% division. The sign follows the divisor \meta{y}.
% Argument \meta{y} must not be zero.
% \begin{gather*}
%   \opMod(x,y) \Def x\mathrel{\%}y\qquad\text{for $y\neq0$}
% \end{gather*}
% The result ranges:
% \begin{gather*}
%   -\vert y\vert < \opMod(x,y) \leq 0\qquad\text{for $y<0$}\\
%   0 \leq \opMod(x,y) < y\qquad\text{for $y\geq0$}
% \end{gather*}
%
% \subsection{Interface for programmers}
%
% If the programmer can ensure some more properties about
% the arguments of the operations, then the following
% macros are a little more efficient.
%
% In general numbers must obey the following constraints:
% \begin{itemize}
% \item Plain number: digit tokens only, no command tokens.
% \item Non-negative. Signs are forbidden.
% \item Delimited by exclamation mark. Curly braces
%       around the number are not allowed and will
%       break the code.
% \end{itemize}
%
% \begin{declcs}{BigIntCalcOdd} \meta{number} |!|
% \end{declcs}
% |1|/|0| is returned if \meta{number} is odd/even.
%
% \begin{declcs}{BigIntCalcInc} \meta{number} |!|
% \end{declcs}
% Incrementation.
%
% \begin{declcs}{BigIntCalcDec} \meta{number} |!|
% \end{declcs}
% Decrementation, positive number without zero.
%
% \begin{declcs}{BigIntCalcAdd} \meta{number A} |!| \meta{number B} |!|
% \end{declcs}
% Addition, $A\geq B$.
%
% \begin{declcs}{BigIntCalcSub} \meta{number A} |!| \meta{number B} |!|
% \end{declcs}
% Subtraction, $A\geq B$.
%
% \begin{declcs}{BigIntCalcShl} \meta{number} |!|
% \end{declcs}
% Left shift (multiplication with two).
%
% \begin{declcs}{BigIntCalcShr} \meta{number} |!|
% \end{declcs}
% Right shift (integer division by two).
%
% \begin{declcs}{BigIntCalcMul} \meta{number A} |!| \meta{number B} |!|
% \end{declcs}
% Multiplication, $A\geq B$.
%
% \begin{declcs}{BigIntCalcDiv} \meta{number A} |!| \meta{number B} |!|
% \end{declcs}
% Division operation.
%
% \begin{declcs}{BigIntCalcMod} \meta{number A} |!| \meta{number B} |!|
% \end{declcs}
% Modulo operation.
%
%
% \StopEventually{
% }
%
% \section{Implementation}
%
%    \begin{macrocode}
%<*package>
%    \end{macrocode}
%
% \subsection{Reload check and package identification}
%    Reload check, especially if the package is not used with \LaTeX.
%    \begin{macrocode}
\begingroup\catcode61\catcode48\catcode32=10\relax%
  \catcode13=5 % ^^M
  \endlinechar=13 %
  \catcode35=6 % #
  \catcode39=12 % '
  \catcode44=12 % ,
  \catcode45=12 % -
  \catcode46=12 % .
  \catcode58=12 % :
  \catcode64=11 % @
  \catcode123=1 % {
  \catcode125=2 % }
  \expandafter\let\expandafter\x\csname ver@bigintcalc.sty\endcsname
  \ifx\x\relax % plain-TeX, first loading
  \else
    \def\empty{}%
    \ifx\x\empty % LaTeX, first loading,
      % variable is initialized, but \ProvidesPackage not yet seen
    \else
      \expandafter\ifx\csname PackageInfo\endcsname\relax
        \def\x#1#2{%
          \immediate\write-1{Package #1 Info: #2.}%
        }%
      \else
        \def\x#1#2{\PackageInfo{#1}{#2, stopped}}%
      \fi
      \x{bigintcalc}{The package is already loaded}%
      \aftergroup\endinput
    \fi
  \fi
\endgroup%
%    \end{macrocode}
%    Package identification:
%    \begin{macrocode}
\begingroup\catcode61\catcode48\catcode32=10\relax%
  \catcode13=5 % ^^M
  \endlinechar=13 %
  \catcode35=6 % #
  \catcode39=12 % '
  \catcode40=12 % (
  \catcode41=12 % )
  \catcode44=12 % ,
  \catcode45=12 % -
  \catcode46=12 % .
  \catcode47=12 % /
  \catcode58=12 % :
  \catcode64=11 % @
  \catcode91=12 % [
  \catcode93=12 % ]
  \catcode123=1 % {
  \catcode125=2 % }
  \expandafter\ifx\csname ProvidesPackage\endcsname\relax
    \def\x#1#2#3[#4]{\endgroup
      \immediate\write-1{Package: #3 #4}%
      \xdef#1{#4}%
    }%
  \else
    \def\x#1#2[#3]{\endgroup
      #2[{#3}]%
      \ifx#1\@undefined
        \xdef#1{#3}%
      \fi
      \ifx#1\relax
        \xdef#1{#3}%
      \fi
    }%
  \fi
\expandafter\x\csname ver@bigintcalc.sty\endcsname
\ProvidesPackage{bigintcalc}%
  [2016/05/16 v1.4 Expandable calculations on big integers (HO)]%
%    \end{macrocode}
%
% \subsection{Catcodes}
%
%    \begin{macrocode}
\begingroup\catcode61\catcode48\catcode32=10\relax%
  \catcode13=5 % ^^M
  \endlinechar=13 %
  \catcode123=1 % {
  \catcode125=2 % }
  \catcode64=11 % @
  \def\x{\endgroup
    \expandafter\edef\csname BIC@AtEnd\endcsname{%
      \endlinechar=\the\endlinechar\relax
      \catcode13=\the\catcode13\relax
      \catcode32=\the\catcode32\relax
      \catcode35=\the\catcode35\relax
      \catcode61=\the\catcode61\relax
      \catcode64=\the\catcode64\relax
      \catcode123=\the\catcode123\relax
      \catcode125=\the\catcode125\relax
    }%
  }%
\x\catcode61\catcode48\catcode32=10\relax%
\catcode13=5 % ^^M
\endlinechar=13 %
\catcode35=6 % #
\catcode64=11 % @
\catcode123=1 % {
\catcode125=2 % }
\def\TMP@EnsureCode#1#2{%
  \edef\BIC@AtEnd{%
    \BIC@AtEnd
    \catcode#1=\the\catcode#1\relax
  }%
  \catcode#1=#2\relax
}
\TMP@EnsureCode{33}{12}% !
\TMP@EnsureCode{36}{14}% $ (comment!)
\TMP@EnsureCode{38}{14}% & (comment!)
\TMP@EnsureCode{40}{12}% (
\TMP@EnsureCode{41}{12}% )
\TMP@EnsureCode{42}{12}% *
\TMP@EnsureCode{43}{12}% +
\TMP@EnsureCode{45}{12}% -
\TMP@EnsureCode{46}{12}% .
\TMP@EnsureCode{47}{12}% /
\TMP@EnsureCode{58}{11}% : (letter!)
\TMP@EnsureCode{60}{12}% <
\TMP@EnsureCode{62}{12}% >
\TMP@EnsureCode{63}{14}% ? (comment!)
\TMP@EnsureCode{91}{12}% [
\TMP@EnsureCode{93}{12}% ]
\edef\BIC@AtEnd{\BIC@AtEnd\noexpand\endinput}
\begingroup\expandafter\expandafter\expandafter\endgroup
\expandafter\ifx\csname BIC@TestMode\endcsname\relax
\else
  \catcode63=9 % ? (ignore)
\fi
? \let\BIC@@TestMode\BIC@TestMode
%    \end{macrocode}
%
% \subsection{\eTeX\ detection}
%
%    \begin{macrocode}
\begingroup\expandafter\expandafter\expandafter\endgroup
\expandafter\ifx\csname numexpr\endcsname\relax
  \catcode36=9 % $ (ignore)
\else
  \catcode38=9 % & (ignore)
\fi
%    \end{macrocode}
%
% \subsection{Help macros}
%
%    \begin{macro}{\BIC@Fi}
%    \begin{macrocode}
\let\BIC@Fi\fi
%    \end{macrocode}
%    \end{macro}
%    \begin{macro}{\BIC@AfterFi}
%    \begin{macrocode}
\def\BIC@AfterFi#1#2\BIC@Fi{\fi#1}%
%    \end{macrocode}
%    \end{macro}
%    \begin{macro}{\BIC@AfterFiFi}
%    \begin{macrocode}
\def\BIC@AfterFiFi#1#2\BIC@Fi{\fi\fi#1}%
%    \end{macrocode}
%    \end{macro}
%    \begin{macro}{\BIC@AfterFiFiFi}
%    \begin{macrocode}
\def\BIC@AfterFiFiFi#1#2\BIC@Fi{\fi\fi\fi#1}%
%    \end{macrocode}
%    \end{macro}
%
%    \begin{macro}{\BIC@Space}
%    \begin{macrocode}
\begingroup
  \def\x#1{\endgroup
    \let\BIC@Space= #1%
  }%
\x{ }
%    \end{macrocode}
%    \end{macro}
%
% \subsection{Expand number}
%
%    \begin{macrocode}
\begingroup\expandafter\expandafter\expandafter\endgroup
\expandafter\ifx\csname RequirePackage\endcsname\relax
  \def\TMP@RequirePackage#1[#2]{%
    \begingroup\expandafter\expandafter\expandafter\endgroup
    \expandafter\ifx\csname ver@#1.sty\endcsname\relax
      \input #1.sty\relax
    \fi
  }%
  \TMP@RequirePackage{pdftexcmds}[2007/11/11]%
\else
  \RequirePackage{pdftexcmds}[2007/11/11]%
\fi
%    \end{macrocode}
%
%    \begin{macrocode}
\begingroup\expandafter\expandafter\expandafter\endgroup
\expandafter\ifx\csname pdf@escapehex\endcsname\relax
%    \end{macrocode}
%
%    \begin{macro}{\BIC@Expand}
%    \begin{macrocode}
  \def\BIC@Expand#1{%
    \romannumeral0%
    \BIC@@Expand#1!\@nil{}%
  }%
%    \end{macrocode}
%    \end{macro}
%    \begin{macro}{\BIC@@Expand}
%    \begin{macrocode}
  \def\BIC@@Expand#1#2\@nil#3{%
    \expandafter\ifcat\noexpand#1\relax
      \expandafter\@firstoftwo
    \else
      \expandafter\@secondoftwo
    \fi
    {%
      \expandafter\BIC@@Expand#1#2\@nil{#3}%
    }{%
      \ifx#1!%
        \expandafter\@firstoftwo
      \else
        \expandafter\@secondoftwo
      \fi
      { #3}{%
        \BIC@@Expand#2\@nil{#3#1}%
      }%
    }%
  }%
%    \end{macrocode}
%    \end{macro}
%    \begin{macro}{\@firstoftwo}
%    \begin{macrocode}
  \expandafter\ifx\csname @firstoftwo\endcsname\relax
    \long\def\@firstoftwo#1#2{#1}%
  \fi
%    \end{macrocode}
%    \end{macro}
%    \begin{macro}{\@secondoftwo}
%    \begin{macrocode}
  \expandafter\ifx\csname @secondoftwo\endcsname\relax
    \long\def\@secondoftwo#1#2{#2}%
  \fi
%    \end{macrocode}
%    \end{macro}
%
%    \begin{macrocode}
\else
%    \end{macrocode}
%    \begin{macro}{\BIC@Expand}
%    \begin{macrocode}
  \def\BIC@Expand#1{%
    \romannumeral0\expandafter\expandafter\expandafter\BIC@Space
    \pdf@unescapehex{%
      \expandafter\expandafter\expandafter
      \BIC@StripHexSpace\pdf@escapehex{#1}20\@nil
    }%
  }%
%    \end{macrocode}
%    \end{macro}
%    \begin{macro}{\BIC@StripHexSpace}
%    \begin{macrocode}
  \def\BIC@StripHexSpace#120#2\@nil{%
    #1%
    \ifx\\#2\\%
    \else
      \BIC@AfterFi{%
        \BIC@StripHexSpace#2\@nil
      }%
    \BIC@Fi
  }%
%    \end{macrocode}
%    \end{macro}
%    \begin{macrocode}
\fi
%    \end{macrocode}
%
% \subsection{Normalize expanded number}
%
%    \begin{macro}{\BIC@Normalize}
%    |#1|: result sign\\
%    |#2|: first token of number
%    \begin{macrocode}
\def\BIC@Normalize#1#2{%
  \ifx#2-%
    \ifx\\#1\\%
      \BIC@AfterFiFi{%
        \BIC@Normalize-%
      }%
    \else
      \BIC@AfterFiFi{%
        \BIC@Normalize{}%
      }%
    \fi
  \else
    \ifx#2+%
      \BIC@AfterFiFi{%
        \BIC@Normalize{#1}%
      }%
    \else
      \ifx#20%
        \BIC@AfterFiFiFi{%
          \BIC@NormalizeZero{#1}%
        }%
      \else
        \BIC@AfterFiFiFi{%
          \BIC@NormalizeDigits#1#2%
        }%
      \fi
    \fi
  \BIC@Fi
}
%    \end{macrocode}
%    \end{macro}
%    \begin{macro}{\BIC@NormalizeZero}
%    \begin{macrocode}
\def\BIC@NormalizeZero#1#2{%
  \ifx#2!%
    \BIC@AfterFi{ 0}%
  \else
    \ifx#20%
      \BIC@AfterFiFi{%
        \BIC@NormalizeZero{#1}%
      }%
    \else
      \BIC@AfterFiFi{%
        \BIC@NormalizeDigits#1#2%
      }%
    \fi
  \BIC@Fi
}
%    \end{macrocode}
%    \end{macro}
%    \begin{macro}{\BIC@NormalizeDigits}
%    \begin{macrocode}
\def\BIC@NormalizeDigits#1!{ #1}
%    \end{macrocode}
%    \end{macro}
%
% \subsection{\op{Num}}
%
%    \begin{macro}{\bigintcalcNum}
%    \begin{macrocode}
\def\bigintcalcNum#1{%
  \romannumeral0%
  \expandafter\expandafter\expandafter\BIC@Normalize
  \expandafter\expandafter\expandafter{%
  \expandafter\expandafter\expandafter}%
  \BIC@Expand{#1}!%
}
%    \end{macrocode}
%    \end{macro}
%
% \subsection{\op{Inv}, \op{Abs}, \op{Sgn}}
%
%
%    \begin{macro}{\bigintcalcInv}
%    \begin{macrocode}
\def\bigintcalcInv#1{%
  \romannumeral0\expandafter\expandafter\expandafter\BIC@Space
  \bigintcalcNum{-#1}%
}
%    \end{macrocode}
%    \end{macro}
%
%    \begin{macro}{\bigintcalcAbs}
%    \begin{macrocode}
\def\bigintcalcAbs#1{%
  \romannumeral0%
  \expandafter\expandafter\expandafter\BIC@Abs
  \bigintcalcNum{#1}%
}
%    \end{macrocode}
%    \end{macro}
%    \begin{macro}{\BIC@Abs}
%    \begin{macrocode}
\def\BIC@Abs#1{%
  \ifx#1-%
    \expandafter\BIC@Space
  \else
    \expandafter\BIC@Space
    \expandafter#1%
  \fi
}
%    \end{macrocode}
%    \end{macro}
%
%    \begin{macro}{\bigintcalcSgn}
%    \begin{macrocode}
\def\bigintcalcSgn#1{%
  \number
  \expandafter\expandafter\expandafter\BIC@Sgn
  \bigintcalcNum{#1}! %
}
%    \end{macrocode}
%    \end{macro}
%    \begin{macro}{\BIC@Sgn}
%    \begin{macrocode}
\def\BIC@Sgn#1#2!{%
  \ifx#1-%
    -1%
  \else
    \ifx#10%
      0%
    \else
      1%
    \fi
  \fi
}
%    \end{macrocode}
%    \end{macro}
%
% \subsection{\op{Cmp}, \op{Min}, \op{Max}}
%
%    \begin{macro}{\bigintcalcCmp}
%    \begin{macrocode}
\def\bigintcalcCmp#1#2{%
  \number
  \expandafter\expandafter\expandafter\BIC@Cmp
  \bigintcalcNum{#2}!{#1}%
}
%    \end{macrocode}
%    \end{macro}
%    \begin{macro}{\BIC@Cmp}
%    \begin{macrocode}
\def\BIC@Cmp#1!#2{%
  \expandafter\expandafter\expandafter\BIC@@Cmp
  \bigintcalcNum{#2}!#1!%
}
%    \end{macrocode}
%    \end{macro}
%    \begin{macro}{\BIC@@Cmp}
%    \begin{macrocode}
\def\BIC@@Cmp#1#2!#3#4!{%
  \ifx#1-%
    \ifx#3-%
      \BIC@AfterFiFi{%
        \BIC@@Cmp#4!#2!%
      }%
    \else
      \BIC@AfterFiFi{%
        -1 %
      }%
    \fi
  \else
    \ifx#3-%
      \BIC@AfterFiFi{%
        1 %
      }%
    \else
      \BIC@AfterFiFi{%
        \BIC@CmpLength#1#2!#3#4!#1#2!#3#4!%
      }%
    \fi
  \BIC@Fi
}
%    \end{macrocode}
%    \end{macro}
%    \begin{macro}{\BIC@PosCmp}
%    \begin{macrocode}
\def\BIC@PosCmp#1!#2!{%
  \BIC@CmpLength#1!#2!#1!#2!%
}
%    \end{macrocode}
%    \end{macro}
%    \begin{macro}{\BIC@CmpLength}
%    \begin{macrocode}
\def\BIC@CmpLength#1#2!#3#4!{%
  \ifx\\#2\\%
    \ifx\\#4\\%
      \BIC@AfterFiFi\BIC@CmpDiff
    \else
      \BIC@AfterFiFi{%
        \BIC@CmpResult{-1}%
      }%
    \fi
  \else
    \ifx\\#4\\%
      \BIC@AfterFiFi{%
        \BIC@CmpResult1%
      }%
    \else
      \BIC@AfterFiFi{%
        \BIC@CmpLength#2!#4!%
      }%
    \fi
  \BIC@Fi
}
%    \end{macrocode}
%    \end{macro}
%    \begin{macro}{\BIC@CmpResult}
%    \begin{macrocode}
\def\BIC@CmpResult#1#2!#3!{#1 }
%    \end{macrocode}
%    \end{macro}
%    \begin{macro}{\BIC@CmpDiff}
%    \begin{macrocode}
\def\BIC@CmpDiff#1#2!#3#4!{%
  \ifnum#1<#3 %
    \BIC@AfterFi{%
      -1 %
    }%
  \else
    \ifnum#1>#3 %
      \BIC@AfterFiFi{%
        1 %
      }%
    \else
      \ifx\\#2\\%
        \BIC@AfterFiFiFi{%
          0 %
        }%
      \else
        \BIC@AfterFiFiFi{%
          \BIC@CmpDiff#2!#4!%
        }%
      \fi
    \fi
  \BIC@Fi
}
%    \end{macrocode}
%    \end{macro}
%
%    \begin{macro}{\bigintcalcMin}
%    \begin{macrocode}
\def\bigintcalcMin#1{%
  \romannumeral0%
  \expandafter\expandafter\expandafter\BIC@MinMax
  \bigintcalcNum{#1}!-!%
}
%    \end{macrocode}
%    \end{macro}
%    \begin{macro}{\bigintcalcMax}
%    \begin{macrocode}
\def\bigintcalcMax#1{%
  \romannumeral0%
  \expandafter\expandafter\expandafter\BIC@MinMax
  \bigintcalcNum{#1}!!%
}
%    \end{macrocode}
%    \end{macro}
%    \begin{macro}{\BIC@MinMax}
%    |#1|: $x$\\
%    |#2|: sign for comparison\\
%    |#3|: $y$
%    \begin{macrocode}
\def\BIC@MinMax#1!#2!#3{%
  \expandafter\expandafter\expandafter\BIC@@MinMax
  \bigintcalcNum{#3}!#1!#2!%
}
%    \end{macrocode}
%    \end{macro}
%    \begin{macro}{\BIC@@MinMax}
%    |#1|: $y$\\
%    |#2|: $x$\\
%    |#3|: sign for comparison
%    \begin{macrocode}
\def\BIC@@MinMax#1!#2!#3!{%
  \ifnum\BIC@@Cmp#1!#2!=#31 %
    \BIC@AfterFi{ #1}%
  \else
    \BIC@AfterFi{ #2}%
  \BIC@Fi
}
%    \end{macrocode}
%    \end{macro}
%
% \subsection{\op{Odd}}
%
%    \begin{macro}{\bigintcalcOdd}
%    \begin{macrocode}
\def\bigintcalcOdd#1{%
  \romannumeral0%
  \expandafter\expandafter\expandafter\BIC@Odd
  \bigintcalcAbs{#1}!%
}
%    \end{macrocode}
%    \end{macro}
%    \begin{macro}{\BigIntCalcOdd}
%    \begin{macrocode}
\def\BigIntCalcOdd#1!{%
  \romannumeral0%
  \BIC@Odd#1!%
}
%    \end{macrocode}
%    \end{macro}
%    \begin{macro}{\BIC@Odd}
%    |#1|: $x$
%    \begin{macrocode}
\def\BIC@Odd#1#2{%
  \ifx#2!%
    \ifodd#1 %
      \BIC@AfterFiFi{ 1}%
    \else
      \BIC@AfterFiFi{ 0}%
    \fi
  \else
    \expandafter\BIC@Odd\expandafter#2%
  \BIC@Fi
}
%    \end{macrocode}
%    \end{macro}
%
% \subsection{\op{Inc}, \op{Dec}}
%
%    \begin{macro}{\bigintcalcInc}
%    \begin{macrocode}
\def\bigintcalcInc#1{%
  \romannumeral0%
  \expandafter\expandafter\expandafter\BIC@IncSwitch
  \bigintcalcNum{#1}!%
}
%    \end{macrocode}
%    \end{macro}
%    \begin{macro}{\BIC@IncSwitch}
%    \begin{macrocode}
\def\BIC@IncSwitch#1#2!{%
  \ifcase\BIC@@Cmp#1#2!-1!%
    \BIC@AfterFi{ 0}%
  \or
    \BIC@AfterFi{%
      \BIC@Inc#1#2!{}%
    }%
  \else
    \BIC@AfterFi{%
      \expandafter-\romannumeral0%
      \BIC@Dec#2!{}%
    }%
  \BIC@Fi
}
%    \end{macrocode}
%    \end{macro}
%
%    \begin{macro}{\bigintcalcDec}
%    \begin{macrocode}
\def\bigintcalcDec#1{%
  \romannumeral0%
  \expandafter\expandafter\expandafter\BIC@DecSwitch
  \bigintcalcNum{#1}!%
}
%    \end{macrocode}
%    \end{macro}
%    \begin{macro}{\BIC@DecSwitch}
%    \begin{macrocode}
\def\BIC@DecSwitch#1#2!{%
  \ifcase\BIC@Sgn#1#2! %
    \BIC@AfterFi{ -1}%
  \or
    \BIC@AfterFi{%
      \BIC@Dec#1#2!{}%
    }%
  \else
    \BIC@AfterFi{%
      \expandafter-\romannumeral0%
      \BIC@Inc#2!{}%
    }%
  \BIC@Fi
}
%    \end{macrocode}
%    \end{macro}
%
%    \begin{macro}{\BigIntCalcInc}
%    \begin{macrocode}
\def\BigIntCalcInc#1!{%
  \romannumeral0\BIC@Inc#1!{}%
}
%    \end{macrocode}
%    \end{macro}
%    \begin{macro}{\BigIntCalcDec}
%    \begin{macrocode}
\def\BigIntCalcDec#1!{%
  \romannumeral0\BIC@Dec#1!{}%
}
%    \end{macrocode}
%    \end{macro}
%
%    \begin{macro}{\BIC@Inc}
%    \begin{macrocode}
\def\BIC@Inc#1#2!#3{%
  \ifx\\#2\\%
    \BIC@AfterFi{%
      \BIC@@Inc1#1#3!{}%
    }%
  \else
    \BIC@AfterFi{%
      \BIC@Inc#2!{#1#3}%
    }%
  \BIC@Fi
}
%    \end{macrocode}
%    \end{macro}
%    \begin{macro}{\BIC@@Inc}
%    \begin{macrocode}
\def\BIC@@Inc#1#2#3!#4{%
  \ifcase#1 %
    \ifx\\#3\\%
      \BIC@AfterFiFi{ #2#4}%
    \else
      \BIC@AfterFiFi{%
        \BIC@@Inc0#3!{#2#4}%
      }%
    \fi
  \else
    \ifnum#2<9 %
      \BIC@AfterFiFi{%
&       \expandafter\BIC@@@Inc\the\numexpr#2+1\relax
$       \expandafter\expandafter\expandafter\BIC@@@Inc
$       \ifcase#2 \expandafter1%
$       \or\expandafter2%
$       \or\expandafter3%
$       \or\expandafter4%
$       \or\expandafter5%
$       \or\expandafter6%
$       \or\expandafter7%
$       \or\expandafter8%
$       \or\expandafter9%
$?      \else\BigIntCalcError:ThisCannotHappen%
$       \fi
        0#3!{#4}%
      }%
    \else
      \BIC@AfterFiFi{%
        \BIC@@@Inc01#3!{#4}%
      }%
    \fi
  \BIC@Fi
}
%    \end{macrocode}
%    \end{macro}
%    \begin{macro}{\BIC@@@Inc}
%    \begin{macrocode}
\def\BIC@@@Inc#1#2#3!#4{%
  \ifx\\#3\\%
    \ifnum#2=1 %
      \BIC@AfterFiFi{ 1#1#4}%
    \else
      \BIC@AfterFiFi{ #1#4}%
    \fi
  \else
    \BIC@AfterFi{%
      \BIC@@Inc#2#3!{#1#4}%
    }%
  \BIC@Fi
}
%    \end{macrocode}
%    \end{macro}
%
%    \begin{macro}{\BIC@Dec}
%    \begin{macrocode}
\def\BIC@Dec#1#2!#3{%
  \ifx\\#2\\%
    \BIC@AfterFi{%
      \BIC@@Dec1#1#3!{}%
    }%
  \else
    \BIC@AfterFi{%
      \BIC@Dec#2!{#1#3}%
    }%
  \BIC@Fi
}
%    \end{macrocode}
%    \end{macro}
%    \begin{macro}{\BIC@@Dec}
%    \begin{macrocode}
\def\BIC@@Dec#1#2#3!#4{%
  \ifcase#1 %
    \ifx\\#3\\%
      \BIC@AfterFiFi{ #2#4}%
    \else
      \BIC@AfterFiFi{%
        \BIC@@Dec0#3!{#2#4}%
      }%
    \fi
  \else
    \ifnum#2>0 %
      \BIC@AfterFiFi{%
&       \expandafter\BIC@@@Dec\the\numexpr#2-1\relax
$       \expandafter\expandafter\expandafter\BIC@@@Dec
$       \ifcase#2
$?        \BigIntCalcError:ThisCannotHappen%
$       \or\expandafter0%
$       \or\expandafter1%
$       \or\expandafter2%
$       \or\expandafter3%
$       \or\expandafter4%
$       \or\expandafter5%
$       \or\expandafter6%
$       \or\expandafter7%
$       \or\expandafter8%
$?      \else\BigIntCalcError:ThisCannotHappen%
$       \fi
        0#3!{#4}%
      }%
    \else
      \BIC@AfterFiFi{%
        \BIC@@@Dec91#3!{#4}%
      }%
    \fi
  \BIC@Fi
}
%    \end{macrocode}
%    \end{macro}
%    \begin{macro}{\BIC@@@Dec}
%    \begin{macrocode}
\def\BIC@@@Dec#1#2#3!#4{%
  \ifx\\#3\\%
    \ifcase#1 %
      \ifx\\#4\\%
        \BIC@AfterFiFiFi{ 0}%
      \else
        \BIC@AfterFiFiFi{ #4}%
      \fi
    \else
      \BIC@AfterFiFi{ #1#4}%
    \fi
  \else
    \BIC@AfterFi{%
      \BIC@@Dec#2#3!{#1#4}%
    }%
  \BIC@Fi
}
%    \end{macrocode}
%    \end{macro}
%
% \subsection{\op{Add}, \op{Sub}}
%
%    \begin{macro}{\bigintcalcAdd}
%    \begin{macrocode}
\def\bigintcalcAdd#1{%
  \romannumeral0%
  \expandafter\expandafter\expandafter\BIC@Add
  \bigintcalcNum{#1}!%
}
%    \end{macrocode}
%    \end{macro}
%    \begin{macro}{\BIC@Add}
%    \begin{macrocode}
\def\BIC@Add#1!#2{%
  \expandafter\expandafter\expandafter
  \BIC@AddSwitch\bigintcalcNum{#2}!#1!%
}
%    \end{macrocode}
%    \end{macro}
%    \begin{macro}{\bigintcalcSub}
%    \begin{macrocode}
\def\bigintcalcSub#1#2{%
  \romannumeral0%
  \expandafter\expandafter\expandafter\BIC@Add
  \bigintcalcNum{-#2}!{#1}%
}
%    \end{macrocode}
%    \end{macro}
%
%    \begin{macro}{\BIC@AddSwitch}
%    Decision table for \cs{BIC@AddSwitch}.
%    \begin{quote}
%    \begin{tabular}[t]{@{}|l|l|l|l|l|@{}}
%      \hline
%      $x<0$ & $y<0$ & $-x>-y$ & $-$ & $\opAdd(-x,-y)$\\
%      \cline{3-3}\cline{5-5}
%            &       & else    &     & $\opAdd(-y,-x)$\\
%      \cline{2-5}
%            & else  & $-x> y$ & $-$ & $\opSub(-x, y)$\\
%      \cline{3-5}
%            &       & $-x= y$ &     & $0$\\
%      \cline{3-5}
%            &       & else    & $+$ & $\opSub( y,-x)$\\
%      \hline
%      else  & $y<0$ & $ x>-y$ & $+$ & $\opSub( x,-y)$\\
%      \cline{3-5}
%            &       & $ x=-y$ &     & $0$\\
%      \cline{3-5}
%            &       & else    & $-$ & $\opSub(-y, x)$\\
%      \cline{2-5}
%            & else  & $ x> y$ & $+$ & $\opAdd( x, y)$\\
%      \cline{3-3}\cline{5-5}
%            &       & else    &     & $\opAdd( y, x)$\\
%      \hline
%    \end{tabular}
%    \end{quote}
%    \begin{macrocode}
\def\BIC@AddSwitch#1#2!#3#4!{%
  \ifx#1-% x < 0
    \ifx#3-% y < 0
      \expandafter-\romannumeral0%
      \ifnum\BIC@PosCmp#2!#4!=1 % -x > -y
        \BIC@AfterFiFiFi{%
          \BIC@AddXY#2!#4!!!%
        }%
      \else % -x <= -y
        \BIC@AfterFiFiFi{%
          \BIC@AddXY#4!#2!!!%
        }%
      \fi
    \else % y >= 0
      \ifcase\BIC@PosCmp#2!#3#4!% -x = y
        \BIC@AfterFiFiFi{ 0}%
      \or % -x > y
        \expandafter-\romannumeral0%
        \BIC@AfterFiFiFi{%
          \BIC@SubXY#2!#3#4!!!%
        }%
      \else % -x <= y
        \BIC@AfterFiFiFi{%
          \BIC@SubXY#3#4!#2!!!%
        }%
      \fi
    \fi
  \else % x >= 0
    \ifx#3-% y < 0
      \ifcase\BIC@PosCmp#1#2!#4!% x = -y
        \BIC@AfterFiFiFi{ 0}%
      \or % x > -y
        \BIC@AfterFiFiFi{%
          \BIC@SubXY#1#2!#4!!!%
        }%
      \else % x <= -y
        \expandafter-\romannumeral0%
        \BIC@AfterFiFiFi{%
          \BIC@SubXY#4!#1#2!!!%
        }%
      \fi
    \else % y >= 0
      \ifnum\BIC@PosCmp#1#2!#3#4!=1 % x > y
        \BIC@AfterFiFiFi{%
          \BIC@AddXY#1#2!#3#4!!!%
        }%
      \else % x <= y
        \BIC@AfterFiFiFi{%
          \BIC@AddXY#3#4!#1#2!!!%
        }%
      \fi
    \fi
  \BIC@Fi
}
%    \end{macrocode}
%    \end{macro}
%
%    \begin{macro}{\BigIntCalcAdd}
%    \begin{macrocode}
\def\BigIntCalcAdd#1!#2!{%
  \romannumeral0\BIC@AddXY#1!#2!!!%
}
%    \end{macrocode}
%    \end{macro}
%    \begin{macro}{\BigIntCalcSub}
%    \begin{macrocode}
\def\BigIntCalcSub#1!#2!{%
  \romannumeral0\BIC@SubXY#1!#2!!!%
}
%    \end{macrocode}
%    \end{macro}
%
%    \begin{macro}{\BIC@AddXY}
%    \begin{macrocode}
\def\BIC@AddXY#1#2!#3#4!#5!#6!{%
  \ifx\\#2\\%
    \ifx\\#3\\%
      \BIC@AfterFiFi{%
        \BIC@DoAdd0!#1#5!#60!%
      }%
    \else
      \BIC@AfterFiFi{%
        \BIC@DoAdd0!#1#5!#3#6!%
      }%
    \fi
  \else
    \ifx\\#4\\%
      \ifx\\#3\\%
        \BIC@AfterFiFiFi{%
          \BIC@AddXY#2!{}!#1#5!#60!%
        }%
      \else
        \BIC@AfterFiFiFi{%
          \BIC@AddXY#2!{}!#1#5!#3#6!%
        }%
      \fi
    \else
      \BIC@AfterFiFi{%
        \BIC@AddXY#2!#4!#1#5!#3#6!%
      }%
    \fi
  \BIC@Fi
}
%    \end{macrocode}
%    \end{macro}
%    \begin{macro}{\BIC@DoAdd}
%    |#1|: carry\\
%    |#2|: reverted result\\
%    |#3#4|: reverted $x$\\
%    |#5#6|: reverted $y$
%    \begin{macrocode}
\def\BIC@DoAdd#1#2!#3#4!#5#6!{%
  \ifx\\#4\\%
    \BIC@AfterFi{%
&     \expandafter\BIC@Space
&     \the\numexpr#1+#3+#5\relax#2%
$     \expandafter\expandafter\expandafter\BIC@AddResult
$     \BIC@AddDigit#1#3#5#2%
    }%
  \else
    \BIC@AfterFi{%
      \expandafter\expandafter\expandafter\BIC@DoAdd
      \BIC@AddDigit#1#3#5#2!#4!#6!%
    }%
  \BIC@Fi
}
%    \end{macrocode}
%    \end{macro}
%    \begin{macro}{\BIC@AddResult}
%    \begin{macrocode}
$ \def\BIC@AddResult#1{%
$   \ifx#10%
$     \expandafter\BIC@Space
$   \else
$     \expandafter\BIC@Space\expandafter#1%
$   \fi
$ }%
%    \end{macrocode}
%    \end{macro}
%    \begin{macro}{\BIC@AddDigit}
%    |#1|: carry\\
%    |#2|: digit of $x$\\
%    |#3|: digit of $y$
%    \begin{macrocode}
\def\BIC@AddDigit#1#2#3{%
  \romannumeral0%
& \expandafter\BIC@@AddDigit\the\numexpr#1+#2+#3!%
$ \expandafter\BIC@@AddDigit\number%
$ \csname
$   BIC@AddCarry%
$   \ifcase#1 %
$     #2%
$   \else
$     \ifcase#2 1\or2\or3\or4\or5\or6\or7\or8\or9\or10\fi
$   \fi
$ \endcsname#3!%
}
%    \end{macrocode}
%    \end{macro}
%    \begin{macro}{\BIC@@AddDigit}
%    \begin{macrocode}
\def\BIC@@AddDigit#1!{%
  \ifnum#1<10 %
    \BIC@AfterFi{ 0#1}%
  \else
    \BIC@AfterFi{ #1}%
  \BIC@Fi
}
%    \end{macrocode}
%    \end{macro}
%    \begin{macro}{\BIC@AddCarry0}
%    \begin{macrocode}
$ \expandafter\def\csname BIC@AddCarry0\endcsname#1{#1}%
%    \end{macrocode}
%    \end{macro}
%    \begin{macro}{\BIC@AddCarry10}
%    \begin{macrocode}
$ \expandafter\def\csname BIC@AddCarry10\endcsname#1{1#1}%
%    \end{macrocode}
%    \end{macro}
%    \begin{macro}{\BIC@AddCarry[1-9]}
%    \begin{macrocode}
$ \def\BIC@Temp#1#2{%
$   \expandafter\def\csname BIC@AddCarry#1\endcsname##1{%
$     \ifcase##1 #1\or
$     #2%
$?    \else\BigIntCalcError:ThisCannotHappen%
$     \fi
$   }%
$ }%
$ \BIC@Temp 0{1\or2\or3\or4\or5\or6\or7\or8\or9}%
$ \BIC@Temp 1{2\or3\or4\or5\or6\or7\or8\or9\or10}%
$ \BIC@Temp 2{3\or4\or5\or6\or7\or8\or9\or10\or11}%
$ \BIC@Temp 3{4\or5\or6\or7\or8\or9\or10\or11\or12}%
$ \BIC@Temp 4{5\or6\or7\or8\or9\or10\or11\or12\or13}%
$ \BIC@Temp 5{6\or7\or8\or9\or10\or11\or12\or13\or14}%
$ \BIC@Temp 6{7\or8\or9\or10\or11\or12\or13\or14\or15}%
$ \BIC@Temp 7{8\or9\or10\or11\or12\or13\or14\or15\or16}%
$ \BIC@Temp 8{9\or10\or11\or12\or13\or14\or15\or16\or17}%
$ \BIC@Temp 9{10\or11\or12\or13\or14\or15\or16\or17\or18}%
%    \end{macrocode}
%    \end{macro}
%
%    \begin{macro}{\BIC@SubXY}
%    Preconditions:
%    \begin{itemize}
%    \item $x > y$, $x \geq 0$, and $y >= 0$
%    \item digits($x$) = digits($y$)
%    \end{itemize}
%    \begin{macrocode}
\def\BIC@SubXY#1#2!#3#4!#5!#6!{%
  \ifx\\#2\\%
    \ifx\\#3\\%
      \BIC@AfterFiFi{%
        \BIC@DoSub0!#1#5!#60!%
      }%
    \else
      \BIC@AfterFiFi{%
        \BIC@DoSub0!#1#5!#3#6!%
      }%
    \fi
  \else
    \ifx\\#4\\%
      \ifx\\#3\\%
        \BIC@AfterFiFiFi{%
          \BIC@SubXY#2!{}!#1#5!#60!%
        }%
      \else
        \BIC@AfterFiFiFi{%
          \BIC@SubXY#2!{}!#1#5!#3#6!%
        }%
      \fi
    \else
      \BIC@AfterFiFi{%
        \BIC@SubXY#2!#4!#1#5!#3#6!%
      }%
    \fi
  \BIC@Fi
}
%    \end{macrocode}
%    \end{macro}
%    \begin{macro}{\BIC@DoSub}
%    |#1|: carry\\
%    |#2|: reverted result\\
%    |#3#4|: reverted x\\
%    |#5#6|: reverted y
%    \begin{macrocode}
\def\BIC@DoSub#1#2!#3#4!#5#6!{%
  \ifx\\#4\\%
    \BIC@AfterFi{%
      \expandafter\expandafter\expandafter\BIC@SubResult
      \BIC@SubDigit#1#3#5#2%
    }%
  \else
    \BIC@AfterFi{%
      \expandafter\expandafter\expandafter\BIC@DoSub
      \BIC@SubDigit#1#3#5#2!#4!#6!%
    }%
  \BIC@Fi
}
%    \end{macrocode}
%    \end{macro}
%    \begin{macro}{\BIC@SubResult}
%    \begin{macrocode}
\def\BIC@SubResult#1{%
  \ifx#10%
    \expandafter\BIC@SubResult
  \else
    \expandafter\BIC@Space\expandafter#1%
  \fi
}
%    \end{macrocode}
%    \end{macro}
%    \begin{macro}{\BIC@SubDigit}
%    |#1|: carry\\
%    |#2|: digit of $x$\\
%    |#3|: digit of $y$
%    \begin{macrocode}
\def\BIC@SubDigit#1#2#3{%
  \romannumeral0%
& \expandafter\BIC@@SubDigit\the\numexpr#2-#3-#1!%
$ \expandafter\BIC@@AddDigit\number
$   \csname
$     BIC@SubCarry%
$     \ifcase#1 %
$       #3%
$     \else
$       \ifcase#3 1\or2\or3\or4\or5\or6\or7\or8\or9\or10\fi
$     \fi
$   \endcsname#2!%
}
%    \end{macrocode}
%    \end{macro}
%    \begin{macro}{\BIC@@SubDigit}
%    \begin{macrocode}
& \def\BIC@@SubDigit#1!{%
&   \ifnum#1<0 %
&     \BIC@AfterFi{%
&       \expandafter\BIC@Space
&       \expandafter1\the\numexpr#1+10\relax
&     }%
&   \else
&     \BIC@AfterFi{ 0#1}%
&   \BIC@Fi
& }%
%    \end{macrocode}
%    \end{macro}
%    \begin{macro}{\BIC@SubCarry0}
%    \begin{macrocode}
$ \expandafter\def\csname BIC@SubCarry0\endcsname#1{#1}%
%    \end{macrocode}
%    \end{macro}
%    \begin{macro}{\BIC@SubCarry10}%
%    \begin{macrocode}
$ \expandafter\def\csname BIC@SubCarry10\endcsname#1{1#1}%
%    \end{macrocode}
%    \end{macro}
%    \begin{macro}{\BIC@SubCarry[1-9]}
%    \begin{macrocode}
$ \def\BIC@Temp#1#2{%
$   \expandafter\def\csname BIC@SubCarry#1\endcsname##1{%
$     \ifcase##1 #2%
$?    \else\BigIntCalcError:ThisCannotHappen%
$     \fi
$   }%
$ }%
$ \BIC@Temp 1{19\or0\or1\or2\or3\or4\or5\or6\or7\or8}%
$ \BIC@Temp 2{18\or19\or0\or1\or2\or3\or4\or5\or6\or7}%
$ \BIC@Temp 3{17\or18\or19\or0\or1\or2\or3\or4\or5\or6}%
$ \BIC@Temp 4{16\or17\or18\or19\or0\or1\or2\or3\or4\or5}%
$ \BIC@Temp 5{15\or16\or17\or18\or19\or0\or1\or2\or3\or4}%
$ \BIC@Temp 6{14\or15\or16\or17\or18\or19\or0\or1\or2\or3}%
$ \BIC@Temp 7{13\or14\or15\or16\or17\or18\or19\or0\or1\or2}%
$ \BIC@Temp 8{12\or13\or14\or15\or16\or17\or18\or19\or0\or1}%
$ \BIC@Temp 9{11\or12\or13\or14\or15\or16\or17\or18\or19\or0}%
%    \end{macrocode}
%    \end{macro}
%
% \subsection{\op{Shl}, \op{Shr}}
%
%    \begin{macro}{\bigintcalcShl}
%    \begin{macrocode}
\def\bigintcalcShl#1{%
  \romannumeral0%
  \expandafter\expandafter\expandafter\BIC@Shl
  \bigintcalcNum{#1}!%
}
%    \end{macrocode}
%    \end{macro}
%    \begin{macro}{\BIC@Shl}
%    \begin{macrocode}
\def\BIC@Shl#1#2!{%
  \ifx#1-%
    \BIC@AfterFi{%
      \expandafter-\romannumeral0%
&     \BIC@@Shl#2!!%
$     \BIC@AddXY#2!#2!!!%
    }%
  \else
    \BIC@AfterFi{%
&     \BIC@@Shl#1#2!!%
$     \BIC@AddXY#1#2!#1#2!!!%
    }%
  \BIC@Fi
}
%    \end{macrocode}
%    \end{macro}
%    \begin{macro}{\BigIntCalcShl}
%    \begin{macrocode}
\def\BigIntCalcShl#1!{%
  \romannumeral0%
& \BIC@@Shl#1!!%
$ \BIC@AddXY#1!#1!!!%
}
%    \end{macrocode}
%    \end{macro}
%    \begin{macro}{\BIC@@Shl}
%    \begin{macrocode}
& \def\BIC@@Shl#1#2!{%
&   \ifx\\#2\\%
&     \BIC@AfterFi{%
&       \BIC@@@Shl0!#1%
&     }%
&   \else
&     \BIC@AfterFi{%
&       \BIC@@Shl#2!#1%
&     }%
&   \BIC@Fi
& }%
%    \end{macrocode}
%    \end{macro}
%    \begin{macro}{\BIC@@@Shl}
%    |#1|: carry\\
%    |#2|: result\\
%    |#3#4|: reverted number
%    \begin{macrocode}
& \def\BIC@@@Shl#1#2!#3#4!{%
&   \ifx\\#4\\%
&     \BIC@AfterFi{%
&       \expandafter\BIC@Space
&       \the\numexpr#3*2+#1\relax#2%
&     }%
&   \else
&     \BIC@AfterFi{%
&       \expandafter\BIC@@@@Shl\the\numexpr#3*2+#1!#2!#4!%
&     }%
&   \BIC@Fi
& }%
%    \end{macrocode}
%    \end{macro}
%    \begin{macro}{\BIC@@@@Shl}
%    \begin{macrocode}
& \def\BIC@@@@Shl#1!{%
&   \ifnum#1<10 %
&     \BIC@AfterFi{%
&       \BIC@@@Shl0#1%
&     }%
&   \else
&     \BIC@AfterFi{%
&       \BIC@@@Shl#1%
&     }%
&   \BIC@Fi
& }%
%    \end{macrocode}
%    \end{macro}
%
%    \begin{macro}{\bigintcalcShr}
%    \begin{macrocode}
\def\bigintcalcShr#1{%
  \romannumeral0%
  \expandafter\expandafter\expandafter\BIC@Shr
  \bigintcalcNum{#1}!%
}
%    \end{macrocode}
%    \end{macro}
%    \begin{macro}{\BIC@Shr}
%    \begin{macrocode}
\def\BIC@Shr#1#2!{%
  \ifx#1-%
    \expandafter-\romannumeral0%
    \BIC@AfterFi{%
      \BIC@@Shr#2!%
    }%
  \else
    \BIC@AfterFi{%
      \BIC@@Shr#1#2!%
    }%
  \BIC@Fi
}
%    \end{macrocode}
%    \end{macro}
%    \begin{macro}{\BigIntCalcShr}
%    \begin{macrocode}
\def\BigIntCalcShr#1!{%
  \romannumeral0%
  \BIC@@Shr#1!%
}
%    \end{macrocode}
%    \end{macro}
%    \begin{macro}{\BIC@@Shr}
%    \begin{macrocode}
\def\BIC@@Shr#1#2!{%
  \ifcase#1 %
    \BIC@AfterFi{ 0}%
  \or
    \ifx\\#2\\%
      \BIC@AfterFiFi{ 0}%
    \else
      \BIC@AfterFiFi{%
        \BIC@@@Shr#1#2!!%
      }%
    \fi
  \else
    \BIC@AfterFi{%
      \BIC@@@Shr0#1#2!!%
    }%
  \BIC@Fi
}
%    \end{macrocode}
%    \end{macro}
%    \begin{macro}{\BIC@@@Shr}
%    |#1|: carry\\
%    |#2#3|: number\\
%    |#4|: result
%    \begin{macrocode}
\def\BIC@@@Shr#1#2#3!#4!{%
  \ifx\\#3\\%
    \ifodd#1#2 %
      \BIC@AfterFiFi{%
&       \expandafter\BIC@ShrResult\the\numexpr(#1#2-1)/2\relax
$       \expandafter\expandafter\expandafter\BIC@ShrResult
$       \csname BIC@ShrDigit#1#2\endcsname
        #4!%
      }%
    \else
      \BIC@AfterFiFi{%
&       \expandafter\BIC@ShrResult\the\numexpr#1#2/2\relax
$       \expandafter\expandafter\expandafter\BIC@ShrResult
$       \csname BIC@ShrDigit#1#2\endcsname
        #4!%
      }%
    \fi
  \else
    \ifodd#1#2 %
      \BIC@AfterFiFi{%
&       \expandafter\BIC@@@@Shr\the\numexpr(#1#2-1)/2\relax1%
$       \expandafter\expandafter\expandafter\BIC@@@@Shr
$       \csname BIC@ShrDigit#1#2\endcsname
        #3!#4!%
      }%
    \else
      \BIC@AfterFiFi{%
&       \expandafter\BIC@@@@Shr\the\numexpr#1#2/2\relax0%
$       \expandafter\expandafter\expandafter\BIC@@@@Shr
$       \csname BIC@ShrDigit#1#2\endcsname
        #3!#4!%
      }%
    \fi
  \BIC@Fi
}
%    \end{macrocode}
%    \end{macro}
%    \begin{macro}{\BIC@ShrResult}
%    \begin{macrocode}
& \def\BIC@ShrResult#1#2!{ #2#1}%
$ \def\BIC@ShrResult#1#2#3!{ #3#1}%
%    \end{macrocode}
%    \end{macro}
%    \begin{macro}{\BIC@@@@Shr}
%    |#1|: new digit\\
%    |#2|: carry\\
%    |#3|: remaining number\\
%    |#4|: result
%    \begin{macrocode}
\def\BIC@@@@Shr#1#2#3!#4!{%
  \BIC@@@Shr#2#3!#4#1!%
}
%    \end{macrocode}
%    \end{macro}
%    \begin{macro}{\BIC@ShrDigit[00-19]}
%    \begin{macrocode}
$ \def\BIC@Temp#1#2#3#4{%
$   \expandafter\def\csname BIC@ShrDigit#1#2\endcsname{#3#4}%
$ }%
$ \BIC@Temp 0000%
$ \BIC@Temp 0101%
$ \BIC@Temp 0210%
$ \BIC@Temp 0311%
$ \BIC@Temp 0420%
$ \BIC@Temp 0521%
$ \BIC@Temp 0630%
$ \BIC@Temp 0731%
$ \BIC@Temp 0840%
$ \BIC@Temp 0941%
$ \BIC@Temp 1050%
$ \BIC@Temp 1151%
$ \BIC@Temp 1260%
$ \BIC@Temp 1361%
$ \BIC@Temp 1470%
$ \BIC@Temp 1571%
$ \BIC@Temp 1680%
$ \BIC@Temp 1781%
$ \BIC@Temp 1890%
$ \BIC@Temp 1991%
%    \end{macrocode}
%    \end{macro}
%
% \subsection{\cs{BIC@Tim}}
%
%    \begin{macro}{\BIC@Tim}
%    Macro \cs{BIC@Tim} implements ``Number \emph{tim}es digit''.\\
%    |#1|: plain number without sign\\
%    |#2|: digit
\def\BIC@Tim#1!#2{%
  \romannumeral0%
  \ifcase#2 % 0
    \BIC@AfterFi{ 0}%
  \or % 1
    \BIC@AfterFi{ #1}%
  \or % 2
    \BIC@AfterFi{%
      \BIC@Shl#1!%
    }%
  \else % 3-9
    \BIC@AfterFi{%
      \BIC@@Tim#1!!#2%
    }%
  \BIC@Fi
}
%    \begin{macrocode}
%    \end{macrocode}
%    \end{macro}
%    \begin{macro}{\BIC@@Tim}
%    |#1#2|: number\\
%    |#3|: reverted number
%    \begin{macrocode}
\def\BIC@@Tim#1#2!{%
  \ifx\\#2\\%
    \BIC@AfterFi{%
      \BIC@ProcessTim0!#1%
    }%
  \else
    \BIC@AfterFi{%
      \BIC@@Tim#2!#1%
    }%
  \BIC@Fi
}
%    \end{macrocode}
%    \end{macro}
%    \begin{macro}{\BIC@ProcessTim}
%    |#1|: carry\\
%    |#2|: result\\
%    |#3#4|: reverted number\\
%    |#5|: digit
%    \begin{macrocode}
\def\BIC@ProcessTim#1#2!#3#4!#5{%
  \ifx\\#4\\%
    \BIC@AfterFi{%
      \expandafter\BIC@Space
&     \the\numexpr#3*#5+#1\relax
$     \romannumeral0\BIC@TimDigit#3#5#1%
      #2%
    }%
  \else
    \BIC@AfterFi{%
      \expandafter\BIC@@ProcessTim
&     \the\numexpr#3*#5+#1%
$     \romannumeral0\BIC@TimDigit#3#5#1%
      !#2!#4!#5%
    }%
  \BIC@Fi
}
%    \end{macrocode}
%    \end{macro}
%    \begin{macro}{\BIC@@ProcessTim}
%    |#1#2|: carry?, new digit\\
%    |#3|: new number\\
%    |#4|: old number\\
%    |#5|: digit
%    \begin{macrocode}
\def\BIC@@ProcessTim#1#2!{%
  \ifx\\#2\\%
    \BIC@AfterFi{%
      \BIC@ProcessTim0#1%
    }%
  \else
    \BIC@AfterFi{%
      \BIC@ProcessTim#1#2%
    }%
  \BIC@Fi
}
%    \end{macrocode}
%    \end{macro}
%    \begin{macro}{\BIC@TimDigit}
%    |#1|: digit 0--9\\
%    |#2|: digit 3--9\\
%    |#3|: carry 0--9
%    \begin{macrocode}
$ \def\BIC@TimDigit#1#2#3{%
$   \ifcase#1 % 0
$     \BIC@AfterFi{ #3}%
$   \or % 1
$     \BIC@AfterFi{%
$       \expandafter\BIC@Space
$       \number\csname BIC@AddCarry#2\endcsname#3 %
$     }%
$   \else
$     \ifcase#3 %
$       \BIC@AfterFiFi{%
$         \expandafter\BIC@Space
$         \number\csname BIC@MulDigit#2\endcsname#1 %
$       }%
$     \else
$       \BIC@AfterFiFi{%
$         \expandafter\BIC@Space
$         \romannumeral0%
$         \expandafter\BIC@AddXY
$         \number\csname BIC@MulDigit#2\endcsname#1!%
$         #3!!!%
$       }%
$     \fi
$   \BIC@Fi
$ }%
%    \end{macrocode}
%    \end{macro}
%    \begin{macro}{\BIC@MulDigit[3-9]}
%    \begin{macrocode}
$ \def\BIC@Temp#1#2{%
$   \expandafter\def\csname BIC@MulDigit#1\endcsname##1{%
$     \ifcase##1 0%
$     \or ##1%
$     \or #2%
$?    \else\BigIntCalcError:ThisCannotHappen%
$     \fi
$   }%
$ }%
$ \BIC@Temp 3{6\or9\or12\or15\or18\or21\or24\or27}%
$ \BIC@Temp 4{8\or12\or16\or20\or24\or28\or32\or36}%
$ \BIC@Temp 5{10\or15\or20\or25\or30\or35\or40\or45}%
$ \BIC@Temp 6{12\or18\or24\or30\or36\or42\or48\or54}%
$ \BIC@Temp 7{14\or21\or28\or35\or42\or49\or56\or63}%
$ \BIC@Temp 8{16\or24\or32\or40\or48\or56\or64\or72}%
$ \BIC@Temp 9{18\or27\or36\or45\or54\or63\or72\or81}%
%    \end{macrocode}
%    \end{macro}
%
% \subsection{\op{Mul}}
%
%    \begin{macro}{\bigintcalcMul}
%    \begin{macrocode}
\def\bigintcalcMul#1#2{%
  \romannumeral0%
  \expandafter\expandafter\expandafter\BIC@Mul
  \bigintcalcNum{#1}!{#2}%
}
%    \end{macrocode}
%    \end{macro}
%    \begin{macro}{\BIC@Mul}
%    \begin{macrocode}
\def\BIC@Mul#1!#2{%
  \expandafter\expandafter\expandafter\BIC@MulSwitch
  \bigintcalcNum{#2}!#1!%
}
%    \end{macrocode}
%    \end{macro}
%    \begin{macro}{\BIC@MulSwitch}
%    Decision table for \cs{BIC@MulSwitch}.
%    \begin{quote}
%    \begin{tabular}[t]{@{}|l|l|l|l|l|@{}}
%      \hline
%      $x=0$ &\multicolumn{3}{l}{}    &$0$\\
%      \hline
%      $x>0$ & $y=0$ &\multicolumn2l{}&$0$\\
%      \cline{2-5}
%            & $y>0$ & $ x> y$ & $+$ & $\opMul( x, y)$\\
%      \cline{3-3}\cline{5-5}
%            &       & else    &     & $\opMul( y, x)$\\
%      \cline{2-5}
%            & $y<0$ & $ x>-y$ & $-$ & $\opMul( x,-y)$\\
%      \cline{3-3}\cline{5-5}
%            &       & else    &     & $\opMul(-y, x)$\\
%      \hline
%      $x<0$ & $y=0$ &\multicolumn2l{}&$0$\\
%      \cline{2-5}
%            & $y>0$ & $-x> y$ & $-$ & $\opMul(-x, y)$\\
%      \cline{3-3}\cline{5-5}
%            &       & else    &     & $\opMul( y,-x)$\\
%      \cline{2-5}
%            & $y<0$ & $-x>-y$ & $+$ & $\opMul(-x,-y)$\\
%      \cline{3-3}\cline{5-5}
%            &       & else    &     & $\opMul(-y,-x)$\\
%      \hline
%    \end{tabular}
%    \end{quote}
%    \begin{macrocode}
\def\BIC@MulSwitch#1#2!#3#4!{%
  \ifcase\BIC@Sgn#1#2! % x = 0
    \BIC@AfterFi{ 0}%
  \or % x > 0
    \ifcase\BIC@Sgn#3#4! % y = 0
      \BIC@AfterFiFi{ 0}%
    \or % y > 0
      \ifnum\BIC@PosCmp#1#2!#3#4!=1 % x > y
        \BIC@AfterFiFiFi{%
          \BIC@ProcessMul0!#1#2!#3#4!%
        }%
      \else % x <= y
        \BIC@AfterFiFiFi{%
          \BIC@ProcessMul0!#3#4!#1#2!%
        }%
      \fi
    \else % y < 0
      \expandafter-\romannumeral0%
      \ifnum\BIC@PosCmp#1#2!#4!=1 % x > -y
        \BIC@AfterFiFiFi{%
          \BIC@ProcessMul0!#1#2!#4!%
        }%
      \else % x <= -y
        \BIC@AfterFiFiFi{%
          \BIC@ProcessMul0!#4!#1#2!%
        }%
      \fi
    \fi
  \else % x < 0
    \ifcase\BIC@Sgn#3#4! % y = 0
      \BIC@AfterFiFi{ 0}%
    \or % y > 0
      \expandafter-\romannumeral0%
      \ifnum\BIC@PosCmp#2!#3#4!=1 % -x > y
        \BIC@AfterFiFiFi{%
          \BIC@ProcessMul0!#2!#3#4!%
        }%
      \else % -x <= y
        \BIC@AfterFiFiFi{%
          \BIC@ProcessMul0!#3#4!#2!%
        }%
      \fi
    \else % y < 0
      \ifnum\BIC@PosCmp#2!#4!=1 % -x > -y
        \BIC@AfterFiFiFi{%
          \BIC@ProcessMul0!#2!#4!%
        }%
      \else % -x <= -y
        \BIC@AfterFiFiFi{%
          \BIC@ProcessMul0!#4!#2!%
        }%
      \fi
    \fi
  \BIC@Fi
}
%    \end{macrocode}
%    \end{macro}
%    \begin{macro}{\BigIntCalcMul}
%    \begin{macrocode}
\def\BigIntCalcMul#1!#2!{%
  \romannumeral0%
  \BIC@ProcessMul0!#1!#2!%
}
%    \end{macrocode}
%    \end{macro}
%    \begin{macro}{\BIC@ProcessMul}
%    |#1|: result\\
%    |#2|: number $x$\\
%    |#3#4|: number $y$
%    \begin{macrocode}
\def\BIC@ProcessMul#1!#2!#3#4!{%
  \ifx\\#4\\%
    \BIC@AfterFi{%
      \expandafter\expandafter\expandafter\BIC@Space
      \bigintcalcAdd{\BIC@Tim#2!#3}{#10}%
    }%
  \else
    \BIC@AfterFi{%
      \expandafter\expandafter\expandafter\BIC@ProcessMul
      \bigintcalcAdd{\BIC@Tim#2!#3}{#10}!#2!#4!%
    }%
  \BIC@Fi
}
%    \end{macrocode}
%    \end{macro}
%
% \subsection{\op{Sqr}}
%
%    \begin{macro}{\bigintcalcSqr}
%    \begin{macrocode}
\def\bigintcalcSqr#1{%
  \romannumeral0%
  \expandafter\expandafter\expandafter\BIC@Sqr
  \bigintcalcNum{#1}!%
}
%    \end{macrocode}
%    \end{macro}
%    \begin{macro}{\BIC@Sqr}
%    \begin{macrocode}
\def\BIC@Sqr#1{%
   \ifx#1-%
     \expandafter\BIC@@Sqr
   \else
     \expandafter\BIC@@Sqr\expandafter#1%
   \fi
}
%    \end{macrocode}
%    \end{macro}
%    \begin{macro}{\BIC@@Sqr}
%    \begin{macrocode}
\def\BIC@@Sqr#1!{%
  \BIC@ProcessMul0!#1!#1!%
}
%    \end{macrocode}
%    \end{macro}
%
% \subsection{\op{Fac}}
%
%    \begin{macro}{\bigintcalcFac}
%    \begin{macrocode}
\def\bigintcalcFac#1{%
  \romannumeral0%
  \expandafter\expandafter\expandafter\BIC@Fac
  \bigintcalcNum{#1}!%
}
%    \end{macrocode}
%    \end{macro}
%    \begin{macro}{\BIC@Fac}
%    \begin{macrocode}
\def\BIC@Fac#1#2!{%
  \ifx#1-%
    \BIC@AfterFi{ 0\BigIntCalcError:FacNegative}%
  \else
    \ifnum\BIC@PosCmp#1#2!13!<0 %
      \ifcase#1#2 %
         \BIC@AfterFiFiFi{ 1}% 0!
      \or\BIC@AfterFiFiFi{ 1}% 1!
      \or\BIC@AfterFiFiFi{ 2}% 2!
      \or\BIC@AfterFiFiFi{ 6}% 3!
      \or\BIC@AfterFiFiFi{ 24}% 4!
      \or\BIC@AfterFiFiFi{ 120}% 5!
      \or\BIC@AfterFiFiFi{ 720}% 6!
      \or\BIC@AfterFiFiFi{ 5040}% 7!
      \or\BIC@AfterFiFiFi{ 40320}% 8!
      \or\BIC@AfterFiFiFi{ 362880}% 9!
      \or\BIC@AfterFiFiFi{ 3628800}% 10!
      \or\BIC@AfterFiFiFi{ 39916800}% 11!
      \or\BIC@AfterFiFiFi{ 479001600}% 12!
?     \else\BigIntCalcError:ThisCannotHappen%
      \fi
    \else
      \BIC@AfterFiFi{%
        \BIC@ProcessFac#1#2!479001600!%
      }%
    \fi
  \BIC@Fi
}
%    \end{macrocode}
%    \end{macro}
%    \begin{macro}{\BIC@ProcessFac}
%    |#1|: $n$\\
%    |#2|: result
%    \begin{macrocode}
\def\BIC@ProcessFac#1!#2!{%
  \ifnum\BIC@PosCmp#1!12!=0 %
    \BIC@AfterFi{ #2}%
  \else
    \BIC@AfterFi{%
      \expandafter\BIC@@ProcessFac
      \romannumeral0\BIC@ProcessMul0!#2!#1!%
      !#1!%
    }%
  \BIC@Fi
}
%    \end{macrocode}
%    \end{macro}
%    \begin{macro}{\BIC@@ProcessFac}
%    |#1|: result\\
%    |#2|: $n$
%    \begin{macrocode}
\def\BIC@@ProcessFac#1!#2!{%
  \expandafter\BIC@ProcessFac
  \romannumeral0\BIC@Dec#2!{}%
  !#1!%
}
%    \end{macrocode}
%    \end{macro}
%
% \subsection{\op{Pow}}
%
%    \begin{macro}{\bigintcalcPow}
%    |#1|: basis\\
%    |#2|: power
%    \begin{macrocode}
\def\bigintcalcPow#1{%
  \romannumeral0%
  \expandafter\expandafter\expandafter\BIC@Pow
  \bigintcalcNum{#1}!%
}
%    \end{macrocode}
%    \end{macro}
%    \begin{macro}{\BIC@Pow}
%    |#1|: basis\\
%    |#2|: power
%    \begin{macrocode}
\def\BIC@Pow#1!#2{%
  \expandafter\expandafter\expandafter\BIC@PowSwitch
  \bigintcalcNum{#2}!#1!%
}
%    \end{macrocode}
%    \end{macro}
%    \begin{macro}{\BIC@PowSwitch}
%    |#1#2|: power $y$\\
%    |#3#4|: basis $x$\\
%    Decision table for \cs{BIC@PowSwitch}.
%    \begin{quote}
%    \def\M#1{\multicolumn{#1}{l}{}}%
%    \begin{tabular}[t]{@{}|l|l|l|l|@{}}
%    \hline
%    $y=0$ & \M{2}                        & $1$\\
%    \hline
%    $y=1$ & \M{2}                        & $x$\\
%    \hline
%    $y=2$ & $x<0$          & \M{1}       & $\opMul(-x,-x)$\\
%    \cline{2-4}
%          & else           & \M{1}       & $\opMul(x,x)$\\
%    \hline
%    $y<0$ & $x=0$          & \M{1}       & DivisionByZero\\
%    \cline{2-4}
%          & $x=1$          & \M{1}       & $1$\\
%    \cline{2-4}
%          & $x=-1$         & $\opODD(y)$ & $-1$\\
%    \cline{3-4}
%          &                & else        & $1$\\
%    \cline{2-4}
%          & else ($\string|x\string|>1$) & \M{1}       & $0$\\
%    \hline
%    $y>2$ & $x=0$          & \M{1}       & $0$\\
%    \cline{2-4}
%          & $x=1$          & \M{1}       & $1$\\
%    \cline{2-4}
%          & $x=-1$         & $\opODD(y)$ & $-1$\\
%    \cline{3-4}
%          &                & else        & $1$\\
%    \cline{2-4}
%          & $x<-1$ ($x<0$) & $\opODD(y)$ & $-\opPow(-x,y)$\\
%    \cline{3-4}
%          &                & else        & $\opPow(-x,y)$\\
%    \cline{2-4}
%          & else ($x>1$)   & \M{1}       & $\opPow(x,y)$\\
%    \hline
%    \end{tabular}
%    \end{quote}
%    \begin{macrocode}
\def\BIC@PowSwitch#1#2!#3#4!{%
  \ifcase\ifx\\#2\\%
           \ifx#100 % y = 0
           \else\ifx#111 % y = 1
           \else\ifx#122 % y = 2
           \else4 % y > 2
           \fi\fi\fi
         \else
           \ifx#1-3 % y < 0
           \else4 % y > 2
           \fi
         \fi
    \BIC@AfterFi{ 1}% y = 0
  \or % y = 1
    \BIC@AfterFi{ #3#4}%
  \or % y = 2
    \ifx#3-% x < 0
      \BIC@AfterFiFi{%
        \BIC@ProcessMul0!#4!#4!%
      }%
    \else % x >= 0
      \BIC@AfterFiFi{%
        \BIC@ProcessMul0!#3#4!#3#4!%
      }%
    \fi
  \or % y < 0
    \ifcase\ifx\\#4\\%
             \ifx#300 % x = 0
             \else\ifx#311 % x = 1
             \else3 % x > 1
             \fi\fi
           \else
             \ifcase\BIC@MinusOne#3#4! %
               3 % |x| > 1
             \or
               2 % x = -1
?            \else\BigIntCalcError:ThisCannotHappen%
             \fi
           \fi
      \BIC@AfterFiFi{ 0\BigIntCalcError:DivisionByZero}% x = 0
    \or % x = 1
      \BIC@AfterFiFi{ 1}% x = 1
    \or % x = -1
      \ifcase\BIC@ModTwo#2! % even(y)
        \BIC@AfterFiFiFi{ 1}%
      \or % odd(y)
        \BIC@AfterFiFiFi{ -1}%
?     \else\BigIntCalcError:ThisCannotHappen%
      \fi
    \or % |x| > 1
      \BIC@AfterFiFi{ 0}%
?   \else\BigIntCalcError:ThisCannotHappen%
    \fi
  \or % y > 2
    \ifcase\ifx\\#4\\%
             \ifx#300 % x = 0
             \else\ifx#311 % x = 1
             \else4 % x > 1
             \fi\fi
           \else
             \ifx#3-%
               \ifcase\BIC@MinusOne#3#4! %
                 3 % x < -1
               \else
                 2 % x = -1
               \fi
             \else
               4 % x > 1
             \fi
           \fi
      \BIC@AfterFiFi{ 0}% x = 0
    \or % x = 1
      \BIC@AfterFiFi{ 1}% x = 1
    \or % x = -1
      \ifcase\BIC@ModTwo#1#2! % even(y)
        \BIC@AfterFiFiFi{ 1}%
      \or % odd(y)
        \BIC@AfterFiFiFi{ -1}%
?     \else\BigIntCalcError:ThisCannotHappen%
      \fi
    \or % x < -1
      \ifcase\BIC@ModTwo#1#2! % even(y)
        \BIC@AfterFiFiFi{%
          \BIC@PowRec#4!#1#2!1!%
        }%
      \or % odd(y)
        \expandafter-\romannumeral0%
        \BIC@AfterFiFiFi{%
          \BIC@PowRec#4!#1#2!1!%
        }%
?     \else\BigIntCalcError:ThisCannotHappen%
      \fi
    \or % x > 1
      \BIC@AfterFiFi{%
        \BIC@PowRec#3#4!#1#2!1!%
      }%
?   \else\BigIntCalcError:ThisCannotHappen%
    \fi
? \else\BigIntCalcError:ThisCannotHappen%
  \BIC@Fi
}
%    \end{macrocode}
%    \end{macro}
%
% \subsubsection{Help macros}
%
%    \begin{macro}{\BIC@ModTwo}
%    Macro \cs{BIC@ModTwo} expects a number without sign
%    and returns digit |1| or |0| if the number is odd or even.
%    \begin{macrocode}
\def\BIC@ModTwo#1#2!{%
  \ifx\\#2\\%
    \ifodd#1 %
      \BIC@AfterFiFi1%
    \else
      \BIC@AfterFiFi0%
    \fi
  \else
    \BIC@AfterFi{%
      \BIC@ModTwo#2!%
    }%
  \BIC@Fi
}
%    \end{macrocode}
%    \end{macro}
%
%    \begin{macro}{\BIC@MinusOne}
%    Macro \cs{BIC@MinusOne} expects a number and returns
%    digit |1| if the number equals minus one and returns |0| otherwise.
%    \begin{macrocode}
\def\BIC@MinusOne#1#2!{%
  \ifx#1-%
    \BIC@@MinusOne#2!%
  \else
    0%
  \fi
}
%    \end{macrocode}
%    \end{macro}
%    \begin{macro}{\BIC@@MinusOne}
%    \begin{macrocode}
\def\BIC@@MinusOne#1#2!{%
  \ifx#11%
    \ifx\\#2\\%
      1%
    \else
      0%
    \fi
  \else
    0%
  \fi
}
%    \end{macrocode}
%    \end{macro}
%
% \subsubsection{Recursive calculation}
%
%    \begin{macro}{\BIC@PowRec}
%\begin{quote}
%\begin{verbatim}
%Pow(x, y) {
%  PowRec(x, y, 1)
%}
%PowRec(x, y, r) {
%  if y == 1 then
%    return r
%  else
%    ifodd y then
%      return PowRec(x*x, y div 2, r*x) % y div 2 = (y-1)/2
%    else
%      return PowRec(x*x, y div 2, r)
%    fi
%  fi
%}
%\end{verbatim}
%\end{quote}
%    |#1|: $x$ (basis)\\
%    |#2#3|: $y$ (power)\\
%    |#4|: $r$ (result)
%    \begin{macrocode}
\def\BIC@PowRec#1!#2#3!#4!{%
  \ifcase\ifx#21\ifx\\#3\\0 \else1 \fi\else1 \fi % y = 1
    \ifnum\BIC@PosCmp#1!#4!=1 % x > r
      \BIC@AfterFiFi{%
        \BIC@ProcessMul0!#1!#4!%
      }%
    \else
      \BIC@AfterFiFi{%
        \BIC@ProcessMul0!#4!#1!%
      }%
    \fi
  \or
    \ifcase\BIC@ModTwo#2#3! % even(y)
      \BIC@AfterFiFi{%
        \expandafter\BIC@@PowRec\romannumeral0%
        \BIC@@Shr#2#3!%
        !#1!#4!%
      }%
    \or % odd(y)
      \ifnum\BIC@PosCmp#1!#4!=1 % x > r
        \BIC@AfterFiFiFi{%
          \expandafter\BIC@@@PowRec\romannumeral0%
          \BIC@ProcessMul0!#1!#4!%
          !#1!#2#3!%
        }%
      \else
        \BIC@AfterFiFiFi{%
          \expandafter\BIC@@@PowRec\romannumeral0%
          \BIC@ProcessMul0!#1!#4!%
          !#1!#2#3!%
        }%
      \fi
?   \else\BigIntCalcError:ThisCannotHappen%
    \fi
? \else\BigIntCalcError:ThisCannotHappen%
  \BIC@Fi
}
%    \end{macrocode}
%    \end{macro}
%    \begin{macro}{\BIC@@PowRec}
%    |#1|: $y/2$\\
%    |#2|: $x$\\
%    |#3|: new $r$ ($r$ or $r*x$)
%    \begin{macrocode}
\def\BIC@@PowRec#1!#2!#3!{%
  \expandafter\BIC@PowRec\romannumeral0%
  \BIC@ProcessMul0!#2!#2!%
  !#1!#3!%
}
%    \end{macrocode}
%    \end{macro}
%    \begin{macro}{\BIC@@@PowRec}
%    |#1|: $r*x$
%    |#2|: $x$
%    |#3|: $y$
%    \begin{macrocode}
\def\BIC@@@PowRec#1!#2!#3!{%
  \expandafter\BIC@@PowRec\romannumeral0%
  \BIC@@Shr#3!%
  !#2!#1!%
}
%    \end{macrocode}
%    \end{macro}
%
% \subsection{\op{Div}}
%
%    \begin{macro}{\bigintcalcDiv}
%    |#1|: $x$\\
%    |#2|: $y$ (divisor)
%    \begin{macrocode}
\def\bigintcalcDiv#1{%
  \romannumeral0%
  \expandafter\expandafter\expandafter\BIC@Div
  \bigintcalcNum{#1}!%
}
%    \end{macrocode}
%    \end{macro}
%    \begin{macro}{\BIC@Div}
%    |#1|: $x$\\
%    |#2|: $y$
%    \begin{macrocode}
\def\BIC@Div#1!#2{%
  \expandafter\expandafter\expandafter\BIC@DivSwitchSign
  \bigintcalcNum{#2}!#1!%
}
%    \end{macrocode}
%    \end{macro}
%    \begin{macro}{\BigIntCalcDiv}
%    \begin{macrocode}
\def\BigIntCalcDiv#1!#2!{%
  \romannumeral0%
  \BIC@DivSwitchSign#2!#1!%
}
%    \end{macrocode}
%    \end{macro}
%    \begin{macro}{\BIC@DivSwitchSign}
%    Decision table for \cs{BIC@DivSwitchSign}.
%    \begin{quote}
%    \begin{tabular}{@{}|l|l|l|@{}}
%    \hline
%    $y=0$ & \multicolumn{1}{l}{} & DivisionByZero\\
%    \hline
%    $y>0$ & $x=0$ & $0$\\
%    \cline{2-3}
%          & $x>0$ & DivSwitch$(+,x,y)$\\
%    \cline{2-3}
%          & $x<0$ & DivSwitch$(-,-x,y)$\\
%    \hline
%    $y<0$ & $x=0$ & $0$\\
%    \cline{2-3}
%          & $x>0$ & DivSwitch$(-,x,-y)$\\
%    \cline{2-3}
%          & $x<0$ & DivSwitch$(+,-x,-y)$\\
%    \hline
%    \end{tabular}
%    \end{quote}
%    |#1|: $y$ (divisor)\\
%    |#2|: $x$
%    \begin{macrocode}
\def\BIC@DivSwitchSign#1#2!#3#4!{%
  \ifcase\BIC@Sgn#1#2! % y = 0
    \BIC@AfterFi{ 0\BigIntCalcError:DivisionByZero}%
  \or % y > 0
    \ifcase\BIC@Sgn#3#4! % x = 0
      \BIC@AfterFiFi{ 0}%
    \or % x > 0
      \BIC@AfterFiFi{%
        \BIC@DivSwitch{}#3#4!#1#2!%
      }%
    \else % x < 0
      \BIC@AfterFiFi{%
        \BIC@DivSwitch-#4!#1#2!%
      }%
    \fi
  \else % y < 0
    \ifcase\BIC@Sgn#3#4! % x = 0
      \BIC@AfterFiFi{ 0}%
    \or % x > 0
      \BIC@AfterFiFi{%
        \BIC@DivSwitch-#3#4!#2!%
      }%
    \else % x < 0
      \BIC@AfterFiFi{%
        \BIC@DivSwitch{}#4!#2!%
      }%
    \fi
  \BIC@Fi
}
%    \end{macrocode}
%    \end{macro}
%    \begin{macro}{\BIC@DivSwitch}
%    Decision table for \cs{BIC@DivSwitch}.
%    \begin{quote}
%    \begin{tabular}{@{}|l|l|l|@{}}
%    \hline
%    $y=x$ & \multicolumn{1}{l}{} & sign $1$\\
%    \hline
%    $y>x$ & \multicolumn{1}{l}{} & $0$\\
%    \hline
%    $y<x$ & $y=1$                & sign $x$\\
%    \cline{2-3}
%          & $y=2$                & sign Shr$(x)$\\
%    \cline{2-3}
%          & $y=4$                & sign Shr(Shr$(x)$)\\
%    \cline{2-3}
%          & else                 & sign ProcessDiv$(x,y)$\\
%    \hline
%    \end{tabular}
%    \end{quote}
%    |#1|: sign\\
%    |#2|: $x$\\
%    |#3#4|: $y$ ($y\ne 0$)
%    \begin{macrocode}
\def\BIC@DivSwitch#1#2!#3#4!{%
  \ifcase\BIC@PosCmp#3#4!#2!% y = x
    \BIC@AfterFi{ #11}%
  \or % y > x
    \BIC@AfterFi{ 0}%
  \else % y < x
    \ifx\\#1\\%
    \else
      \expandafter-\romannumeral0%
    \fi
    \ifcase\ifx\\#4\\%
             \ifx#310 % y = 1
             \else\ifx#321 % y = 2
             \else\ifx#342 % y = 4
             \else3 % y > 2
             \fi\fi\fi
           \else
             3 % y > 2
           \fi
      \BIC@AfterFiFi{ #2}% y = 1
    \or % y = 2
      \BIC@AfterFiFi{%
        \BIC@@Shr#2!%
      }%
    \or % y = 4
      \BIC@AfterFiFi{%
        \expandafter\BIC@@Shr\romannumeral0%
          \BIC@@Shr#2!!%
      }%
    \or % y > 2
      \BIC@AfterFiFi{%
        \BIC@DivStartX#2!#3#4!!!%
      }%
?   \else\BigIntCalcError:ThisCannotHappen%
    \fi
  \BIC@Fi
}
%    \end{macrocode}
%    \end{macro}
%    \begin{macro}{\BIC@ProcessDiv}
%    |#1#2|: $x$\\
%    |#3#4|: $y$\\
%    |#5|: collect first digits of $x$\\
%    |#6|: corresponding digits of $y$
%    \begin{macrocode}
\def\BIC@DivStartX#1#2!#3#4!#5!#6!{%
  \ifx\\#4\\%
    \BIC@AfterFi{%
      \BIC@DivStartYii#6#3#4!{#5#1}#2=!%
    }%
  \else
    \BIC@AfterFi{%
      \BIC@DivStartX#2!#4!#5#1!#6#3!%
    }%
  \BIC@Fi
}
%    \end{macrocode}
%    \end{macro}
%    \begin{macro}{\BIC@DivStartYii}
%    |#1|: $y$\\
%    |#2|: $x$, |=|
%    \begin{macrocode}
\def\BIC@DivStartYii#1!{%
  \expandafter\BIC@DivStartYiv\romannumeral0%
  \BIC@Shl#1!%
  !#1!%
}
%    \end{macrocode}
%    \end{macro}
%    \begin{macro}{\BIC@DivStartYiv}
%    |#1|: $2y$\\
%    |#2|: $y$\\
%    |#3|: $x$, |=|
%    \begin{macrocode}
\def\BIC@DivStartYiv#1!{%
  \expandafter\BIC@DivStartYvi\romannumeral0%
  \BIC@Shl#1!%
  !#1!%
}
%    \end{macrocode}
%    \end{macro}
%    \begin{macro}{\BIC@DivStartYvi}
%    |#1|: $4y$\\
%    |#2|: $2y$\\
%    |#3|: $y$\\
%    |#4|: $x$, |=|
%    \begin{macrocode}
\def\BIC@DivStartYvi#1!#2!{%
  \expandafter\BIC@DivStartYviii\romannumeral0%
  \BIC@AddXY#1!#2!!!%
  !#1!#2!%
}
%    \end{macrocode}
%    \end{macro}
%    \begin{macro}{\BIC@DivStartYviii}
%    |#1|: $6y$\\
%    |#2|: $4y$\\
%    |#3|: $2y$\\
%    |#4|: $y$\\
%    |#5|: $x$, |=|
%    \begin{macrocode}
\def\BIC@DivStartYviii#1!#2!{%
  \expandafter\BIC@DivStart\romannumeral0%
  \BIC@Shl#2!%
  !#1!#2!%
}
%    \end{macrocode}
%    \end{macro}
%    \begin{macro}{\BIC@DivStart}
%    |#1|: $8y$\\
%    |#2|: $6y$\\
%    |#3|: $4y$\\
%    |#4|: $2y$\\
%    |#5|: $y$\\
%    |#6|: $x$, |=|
%    \begin{macrocode}
\def\BIC@DivStart#1!#2!#3!#4!#5!#6!{%
  \BIC@ProcessDiv#6!!#5!#4!#3!#2!#1!=%
}
%    \end{macrocode}
%    \end{macro}
%    \begin{macro}{\BIC@ProcessDiv}
%    |#1#2#3|: $x$, |=|\\
%    |#4|: result\\
%    |#5|: $y$\\
%    |#6|: $2y$\\
%    |#7|: $4y$\\
%    |#8|: $6y$\\
%    |#9|: $8y$
%    \begin{macrocode}
\def\BIC@ProcessDiv#1#2#3!#4!#5!{%
  \ifcase\BIC@PosCmp#5!#1!% y = #1
    \ifx#2=%
      \BIC@AfterFiFi{\BIC@DivCleanup{#41}}%
    \else
      \BIC@AfterFiFi{%
        \BIC@ProcessDiv#2#3!#41!#5!%
      }%
    \fi
  \or % y > #1
    \ifx#2=%
      \BIC@AfterFiFi{\BIC@DivCleanup{#40}}%
    \else
      \ifx\\#4\\%
        \BIC@AfterFiFiFi{%
          \BIC@ProcessDiv{#1#2}#3!!#5!%
        }%
      \else
        \BIC@AfterFiFiFi{%
          \BIC@ProcessDiv{#1#2}#3!#40!#5!%
        }%
      \fi
    \fi
  \else % y < #1
    \BIC@AfterFi{%
      \BIC@@ProcessDiv{#1}#2#3!#4!#5!%
    }%
  \BIC@Fi
}
%    \end{macrocode}
%    \end{macro}
%    \begin{macro}{\BIC@DivCleanup}
%    |#1|: result\\
%    |#2|: garbage
%    \begin{macrocode}
\def\BIC@DivCleanup#1#2={ #1}%
%    \end{macrocode}
%    \end{macro}
%    \begin{macro}{\BIC@@ProcessDiv}
%    \begin{macrocode}
\def\BIC@@ProcessDiv#1#2#3!#4!#5!#6!#7!{%
  \ifcase\BIC@PosCmp#7!#1!% 4y = #1
    \ifx#2=%
      \BIC@AfterFiFi{\BIC@DivCleanup{#44}}%
    \else
     \BIC@AfterFiFi{%
       \BIC@ProcessDiv#2#3!#44!#5!#6!#7!%
     }%
    \fi
  \or % 4y > #1
    \ifcase\BIC@PosCmp#6!#1!% 2y = #1
      \ifx#2=%
        \BIC@AfterFiFiFi{\BIC@DivCleanup{#42}}%
      \else
        \BIC@AfterFiFiFi{%
          \BIC@ProcessDiv#2#3!#42!#5!#6!#7!%
        }%
      \fi
    \or % 2y > #1
      \ifx#2=%
        \BIC@AfterFiFiFi{\BIC@DivCleanup{#41}}%
      \else
        \BIC@AfterFiFiFi{%
          \BIC@DivSub#1!#5!#2#3!#41!#5!#6!#7!%
        }%
      \fi
    \else % 2y < #1
      \BIC@AfterFiFi{%
        \expandafter\BIC@ProcessDivII\romannumeral0%
        \BIC@SubXY#1!#6!!!%
        !#2#3!#4!#5!23%
        #6!#7!%
      }%
    \fi
  \else % 4y < #1
    \BIC@AfterFi{%
      \BIC@@@ProcessDiv{#1}#2#3!#4!#5!#6!#7!%
    }%
  \BIC@Fi
}
%    \end{macrocode}
%    \end{macro}
%    \begin{macro}{\BIC@DivSub}
%    Next token group: |#1|-|#2| and next digit |#3|.
%    \begin{macrocode}
\def\BIC@DivSub#1!#2!#3{%
  \expandafter\BIC@ProcessDiv\expandafter{%
    \romannumeral0%
    \BIC@SubXY#1!#2!!!%
    #3%
  }%
}
%    \end{macrocode}
%    \end{macro}
%    \begin{macro}{\BIC@ProcessDivII}
%    |#1|: $x'-2y$\\
%    |#2#3|: remaining $x$, |=|\\
%    |#4|: result\\
%    |#5|: $y$\\
%    |#6|: first possible result digit\\
%    |#7|: second possible result digit
%    \begin{macrocode}
\def\BIC@ProcessDivII#1!#2#3!#4!#5!#6#7{%
  \ifcase\BIC@PosCmp#5!#1!% y = #1
    \ifx#2=%
      \BIC@AfterFiFi{\BIC@DivCleanup{#4#7}}%
    \else
      \BIC@AfterFiFi{%
        \BIC@ProcessDiv#2#3!#4#7!#5!%
      }%
    \fi
  \or % y > #1
    \ifx#2=%
      \BIC@AfterFiFi{\BIC@DivCleanup{#4#6}}%
    \else
      \BIC@AfterFiFi{%
        \BIC@ProcessDiv{#1#2}#3!#4#6!#5!%
      }%
    \fi
  \else % y < #1
    \ifx#2=%
      \BIC@AfterFiFi{\BIC@DivCleanup{#4#7}}%
    \else
      \BIC@AfterFiFi{%
        \BIC@DivSub#1!#5!#2#3!#4#7!#5!%
      }%
    \fi
  \BIC@Fi
}
%    \end{macrocode}
%    \end{macro}
%    \begin{macro}{\BIC@ProcessDivIV}
%    |#1#2#3|: $x$, |=|, $x>4y$\\
%    |#4|: result\\
%    |#5|: $y$\\
%    |#6|: $2y$\\
%    |#7|: $4y$\\
%    |#8|: $6y$\\
%    |#9|: $8y$
%    \begin{macrocode}
\def\BIC@@@ProcessDiv#1#2#3!#4!#5!#6!#7!#8!#9!{%
  \ifcase\BIC@PosCmp#8!#1!% 6y = #1
    \ifx#2=%
      \BIC@AfterFiFi{\BIC@DivCleanup{#46}}%
    \else
      \BIC@AfterFiFi{%
        \BIC@ProcessDiv#2#3!#46!#5!#6!#7!#8!#9!%
      }%
    \fi
  \or % 6y > #1
    \BIC@AfterFi{%
      \expandafter\BIC@ProcessDivII\romannumeral0%
      \BIC@SubXY#1!#7!!!%
      !#2#3!#4!#5!45%
      #6!#7!#8!#9!%
    }%
  \else % 6y < #1
    \ifcase\BIC@PosCmp#9!#1!% 8y = #1
      \ifx#2=%
        \BIC@AfterFiFiFi{\BIC@DivCleanup{#48}}%
      \else
        \BIC@AfterFiFiFi{%
          \BIC@ProcessDiv#2#3!#48!#5!#6!#7!#8!#9!%
        }%
      \fi
    \or % 8y > #1
      \BIC@AfterFiFi{%
        \expandafter\BIC@ProcessDivII\romannumeral0%
        \BIC@SubXY#1!#8!!!%
        !#2#3!#4!#5!67%
        #6!#7!#8!#9!%
      }%
    \else % 8y < #1
      \BIC@AfterFiFi{%
        \expandafter\BIC@ProcessDivII\romannumeral0%
        \BIC@SubXY#1!#9!!!%
        !#2#3!#4!#5!89%
        #6!#7!#8!#9!%
      }%
    \fi
  \BIC@Fi
}
%    \end{macrocode}
%    \end{macro}
%
% \subsection{\op{Mod}}
%
%    \begin{macro}{\bigintcalcMod}
%    |#1|: $x$\\
%    |#2|: $y$
%    \begin{macrocode}
\def\bigintcalcMod#1{%
  \romannumeral0%
  \expandafter\expandafter\expandafter\BIC@Mod
  \bigintcalcNum{#1}!%
}
%    \end{macrocode}
%    \end{macro}
%    \begin{macro}{\BIC@Mod}
%    |#1|: $x$\\
%    |#2|: $y$
%    \begin{macrocode}
\def\BIC@Mod#1!#2{%
  \expandafter\expandafter\expandafter\BIC@ModSwitchSign
  \bigintcalcNum{#2}!#1!%
}
%    \end{macrocode}
%    \end{macro}
%    \begin{macro}{\BigIntCalcMod}
%    \begin{macrocode}
\def\BigIntCalcMod#1!#2!{%
  \romannumeral0%
  \BIC@ModSwitchSign#2!#1!%
}
%    \end{macrocode}
%    \end{macro}
%    \begin{macro}{\BIC@ModSwitchSign}
%    Decision table for \cs{BIC@ModSwitchSign}.
%    \begin{quote}
%    \begin{tabular}{@{}|l|l|l|@{}}
%    \hline
%    $y=0$ & \multicolumn{1}{l}{} & DivisionByZero\\
%    \hline
%    $y>0$ & $x=0$ & $0$\\
%    \cline{2-3}
%          & else  & ModSwitch$(+,x,y)$\\
%    \hline
%    $y<0$ & \multicolumn{1}{l}{} & ModSwitch$(-,-x,-y)$\\
%    \hline
%    \end{tabular}
%    \end{quote}
%    |#1#2|: $y$\\
%    |#3#4|: $x$
%    \begin{macrocode}
\def\BIC@ModSwitchSign#1#2!#3#4!{%
  \ifcase\ifx\\#2\\%
           \ifx#100 % y = 0
           \else1 % y > 0
           \fi
          \else
            \ifx#1-2 % y < 0
            \else1 % y > 0
            \fi
          \fi
    \BIC@AfterFi{ 0\BigIntCalcError:DivisionByZero}%
  \or % y > 0
    \ifcase\ifx\\#4\\\ifx#300 \else1 \fi\else1 \fi % x = 0
      \BIC@AfterFiFi{ 0}%
    \else
      \BIC@AfterFiFi{%
        \BIC@ModSwitch{}#3#4!#1#2!%
      }%
    \fi
  \else % y < 0
    \ifcase\ifx\\#4\\%
             \ifx#300 % x = 0
             \else1 % x > 0
             \fi
           \else
             \ifx#3-2 % x < 0
             \else1 % x > 0
             \fi
           \fi
      \BIC@AfterFiFi{ 0}%
    \or % x > 0
      \BIC@AfterFiFi{%
        \BIC@ModSwitch--#3#4!#2!%
      }%
    \else % x < 0
      \BIC@AfterFiFi{%
        \BIC@ModSwitch-#4!#2!%
      }%
    \fi
  \BIC@Fi
}
%    \end{macrocode}
%    \end{macro}
%    \begin{macro}{\BIC@ModSwitch}
%    Decision table for \cs{BIC@ModSwitch}.
%    \begin{quote}
%    \begin{tabular}{@{}|l|l|l|@{}}
%    \hline
%    $y=1$ & \multicolumn{1}{l}{} & $0$\\
%    \hline
%    $y=2$ & ifodd$(x)$ & sign $1$\\
%    \cline{2-3}
%          & else       & $0$\\
%    \hline
%    $y>2$ & $x<0$      &
%        $z\leftarrow x-(x/y)*y;\quad (z<0)\mathbin{?}z+y\mathbin{:}z$\\
%    \cline{2-3}
%          & $x>0$      & $x - (x/y) * y$\\
%    \hline
%    \end{tabular}
%    \end{quote}
%    |#1|: sign\\
%    |#2#3|: $x$\\
%    |#4#5|: $y$
%    \begin{macrocode}
\def\BIC@ModSwitch#1#2#3!#4#5!{%
  \ifcase\ifx\\#5\\%
           \ifx#410 % y = 1
           \else\ifx#421 % y = 2
           \else2 % y > 2
           \fi\fi
         \else2 % y > 2
         \fi
    \BIC@AfterFi{ 0}% y = 1
  \or % y = 2
    \ifcase\BIC@ModTwo#2#3! % even(x)
      \BIC@AfterFiFi{ 0}%
    \or % odd(x)
      \BIC@AfterFiFi{ #11}%
?   \else\BigIntCalcError:ThisCannotHappen%
    \fi
  \or % y > 2
    \ifx\\#1\\%
    \else
      \expandafter\BIC@Space\romannumeral0%
      \expandafter\BIC@ModMinus\romannumeral0%
    \fi
    \ifx#2-% x < 0
      \BIC@AfterFiFi{%
        \expandafter\expandafter\expandafter\BIC@ModX
        \bigintcalcSub{#2#3}{%
          \bigintcalcMul{#4#5}{\bigintcalcDiv{#2#3}{#4#5}}%
        }!#4#5!%
      }%
    \else % x > 0
      \BIC@AfterFiFi{%
        \expandafter\expandafter\expandafter\BIC@Space
        \bigintcalcSub{#2#3}{%
          \bigintcalcMul{#4#5}{\bigintcalcDiv{#2#3}{#4#5}}%
        }%
      }%
    \fi
? \else\BigIntCalcError:ThisCannotHappen%
  \BIC@Fi
}
%    \end{macrocode}
%    \end{macro}
%    \begin{macro}{\BIC@ModMinus}
%    \begin{macrocode}
\def\BIC@ModMinus#1{%
  \ifx#10%
    \BIC@AfterFi{ 0}%
  \else
    \BIC@AfterFi{ -#1}%
  \BIC@Fi
}
%    \end{macrocode}
%    \end{macro}
%    \begin{macro}{\BIC@ModX}
%    |#1#2|: $z$\\
%    |#3|: $x$
%    \begin{macrocode}
\def\BIC@ModX#1#2!#3!{%
  \ifx#1-% z < 0
    \BIC@AfterFi{%
      \expandafter\BIC@Space\romannumeral0%
      \BIC@SubXY#3!#2!!!%
    }%
  \else % z >= 0
    \BIC@AfterFi{ #1#2}%
  \BIC@Fi
}
%    \end{macrocode}
%    \end{macro}
%
%    \begin{macrocode}
\BIC@AtEnd%
%    \end{macrocode}
%
%    \begin{macrocode}
%</package>
%    \end{macrocode}
%
% \section{Test}
%
% \subsection{Catcode checks for loading}
%
%    \begin{macrocode}
%<*test1>
%    \end{macrocode}
%    \begin{macrocode}
\catcode`\{=1 %
\catcode`\}=2 %
\catcode`\#=6 %
\catcode`\@=11 %
\expandafter\ifx\csname count@\endcsname\relax
  \countdef\count@=255 %
\fi
\expandafter\ifx\csname @gobble\endcsname\relax
  \long\def\@gobble#1{}%
\fi
\expandafter\ifx\csname @firstofone\endcsname\relax
  \long\def\@firstofone#1{#1}%
\fi
\expandafter\ifx\csname loop\endcsname\relax
  \expandafter\@firstofone
\else
  \expandafter\@gobble
\fi
{%
  \def\loop#1\repeat{%
    \def\body{#1}%
    \iterate
  }%
  \def\iterate{%
    \body
      \let\next\iterate
    \else
      \let\next\relax
    \fi
    \next
  }%
  \let\repeat=\fi
}%
\def\RestoreCatcodes{}
\count@=0 %
\loop
  \edef\RestoreCatcodes{%
    \RestoreCatcodes
    \catcode\the\count@=\the\catcode\count@\relax
  }%
\ifnum\count@<255 %
  \advance\count@ 1 %
\repeat

\def\RangeCatcodeInvalid#1#2{%
  \count@=#1\relax
  \loop
    \catcode\count@=15 %
  \ifnum\count@<#2\relax
    \advance\count@ 1 %
  \repeat
}
\def\RangeCatcodeCheck#1#2#3{%
  \count@=#1\relax
  \loop
    \ifnum#3=\catcode\count@
    \else
      \errmessage{%
        Character \the\count@\space
        with wrong catcode \the\catcode\count@\space
        instead of \number#3%
      }%
    \fi
  \ifnum\count@<#2\relax
    \advance\count@ 1 %
  \repeat
}
\def\space{ }
\expandafter\ifx\csname LoadCommand\endcsname\relax
  \def\LoadCommand{\input bigintcalc.sty\relax}%
\fi
\def\Test{%
  \RangeCatcodeInvalid{0}{47}%
  \RangeCatcodeInvalid{58}{64}%
  \RangeCatcodeInvalid{91}{96}%
  \RangeCatcodeInvalid{123}{255}%
  \catcode`\@=12 %
  \catcode`\\=0 %
  \catcode`\%=14 %
  \LoadCommand
  \RangeCatcodeCheck{0}{36}{15}%
  \RangeCatcodeCheck{37}{37}{14}%
  \RangeCatcodeCheck{38}{47}{15}%
  \RangeCatcodeCheck{48}{57}{12}%
  \RangeCatcodeCheck{58}{63}{15}%
  \RangeCatcodeCheck{64}{64}{12}%
  \RangeCatcodeCheck{65}{90}{11}%
  \RangeCatcodeCheck{91}{91}{15}%
  \RangeCatcodeCheck{92}{92}{0}%
  \RangeCatcodeCheck{93}{96}{15}%
  \RangeCatcodeCheck{97}{122}{11}%
  \RangeCatcodeCheck{123}{255}{15}%
  \RestoreCatcodes
}
\Test
\csname @@end\endcsname
\end
%    \end{macrocode}
%    \begin{macrocode}
%</test1>
%    \end{macrocode}
%
% \subsection{Macro tests}
%
% \subsubsection{Preamble with test macro definitions}
%
%    \begin{macrocode}
%<*test2>
\NeedsTeXFormat{LaTeX2e}
\nofiles
\documentclass{article}
%<noetex>\let\SavedNumexpr\numexpr
%<noetex>\let\numexpr\UNDEFINED
\makeatletter
\chardef\BIC@TestMode=1 %
\makeatother
\usepackage{bigintcalc}[2016/05/16]
%<noetex>\let\numexpr\SavedNumexpr
\usepackage{qstest}
\IncludeTests{*}
\LogTests{log}{*}{*}
\newcommand*{\TestSpaceAtEnd}[1]{%
%<noetex>  \let\SavedNumexpr\numexpr
%<noetex>  \let\numexpr\UNDEFINED
  \edef\resultA{#1}%
  \edef\resultB{#1 }%
%<noetex>  \let\numexpr\SavedNumexpr
  \Expect*{\resultA\space}*{\resultB}%
}
\newcommand*{\TestResult}[2]{%
%<noetex>  \let\SavedNumexpr\numexpr
%<noetex>  \let\numexpr\UNDEFINED
  \edef\result{#1}%
%<noetex>  \let\numexpr\SavedNumexpr
  \Expect*{\result}{#2}%
}
\newcommand*{\TestResultTwoExpansions}[2]{%
%<*noetex>
  \begingroup
    \let\numexpr\UNDEFINED
    \expandafter\expandafter\expandafter
  \endgroup
%</noetex>
  \expandafter\expandafter\expandafter\Expect
  \expandafter\expandafter\expandafter{#1}{#2}%
}
\newcount\TestCount
%<etex>\newcommand*{\TestArg}[1]{\numexpr#1\relax}
%<noetex>\newcommand*{\TestArg}[1]{#1}
\newcommand*{\TestTeXDivide}[2]{%
  \TestCount=\TestArg{#1}\relax
  \divide\TestCount by \TestArg{#2}\relax
  \Expect*{\bigintcalcDiv{#1}{#2}}*{\the\TestCount}%
}
\newcommand*{\Test}[2]{%
  \TestResult{#1}{#2}%
  \TestResultTwoExpansions{#1}{#2}%
  \TestSpaceAtEnd{#1}%
}
\newcommand*{\TestExch}[2]{\Test{#2}{#1}}
\newcommand*{\TestInv}[2]{%
  \Test{\bigintcalcInv{#1}}{#2}%
}
\newcommand*{\TestAbs}[2]{%
  \Test{\bigintcalcAbs{#1}}{#2}%
}
\newcommand*{\TestSgn}[2]{%
  \Test{\bigintcalcSgn{#1}}{#2}%
}
\newcommand*{\TestMin}[3]{%
  \Test{\bigintcalcMin{#1}{#2}}{#3}%
}
\newcommand*{\TestMax}[3]{%
  \Test{\bigintcalcMax{#1}{#2}}{#3}%
}
\newcommand*{\TestCmp}[3]{%
  \Test{\bigintcalcCmp{#1}{#2}}{#3}%
}
\newcommand*{\TestOdd}[2]{%
  \Test{\bigintcalcOdd{#1}}{#2}%
  \edef\x{%
    \noexpand\Test{%
      \noexpand\BigIntCalcOdd
      \bigintcalcAbs{#1}!%
    }{#2}%
  }%
  \x
}
\newcommand*{\TestInc}[2]{%
  \Test{\bigintcalcInc{#1}}{#2}%
  \ifnum\bigintcalcSgn{#1}>-1 %
    \edef\x{%
      \noexpand\Test{%
        \noexpand\BigIntCalcInc\bigintcalcNum{#1}!%
      }{#2}%
    }%
    \x
  \fi
}
\newcommand*{\TestDec}[2]{%
  \Test{\bigintcalcDec{#1}}{#2}%
  \ifnum\bigintcalcSgn{#1}>0 %
    \edef\x{%
      \noexpand\Test{%
        \noexpand\BigIntCalcDec\bigintcalcNum{#1}!%
      }{#2}%
    }%
    \x
  \fi
}
\newcommand*{\TestAdd}[3]{%
  \Test{\bigintcalcAdd{#1}{#2}}{#3}%
  \ifnum\bigintcalcSgn{#1}>0 %
    \ifnum\bigintcalcSgn{#2}> 0 %
      \ifnum\bigintcalcCmp{#1}{#2}>0 %
        \edef\x{%
          \noexpand\Test{%
            \noexpand\BigIntCalcAdd
            \bigintcalcNum{#1}!\bigintcalcNum{#2}!%
          }{#3}%
        }%
        \x
      \else
        \edef\x{%
          \noexpand\Test{%
            \noexpand\BigIntCalcAdd
            \bigintcalcNum{#2}!\bigintcalcNum{#1}!%
          }{#3}%
        }%
        \x
      \fi
    \fi
  \fi
}
\newcommand*{\TestSub}[3]{%
  \Test{\bigintcalcSub{#1}{#2}}{#3}%
  \ifnum\bigintcalcSgn{#1}>0 %
    \ifnum\bigintcalcSgn{#2}> 0 %
      \ifnum\bigintcalcCmp{#1}{#2}>0 %
        \edef\x{%
          \noexpand\Test{%
            \noexpand\BigIntCalcSub
            \bigintcalcNum{#1}!\bigintcalcNum{#2}!%
          }{#3}%
        }%
        \x
      \fi
    \fi
  \fi
}
\newcommand*{\TestShl}[2]{%
  \Test{\bigintcalcShl{#1}}{#2}%
  \edef\x{%
    \noexpand\Test{%
      \noexpand\BigIntCalcShl\bigintcalcAbs{#1}!%
    }{\bigintcalcAbs{#2}}%
  }%
  \x
}
\newcommand*{\TestShr}[2]{%
  \Test{\bigintcalcShr{#1}}{#2}%
  \edef\x{%
    \noexpand\Test{%
      \noexpand\BigIntCalcShr\bigintcalcAbs{#1}!%
    }{\bigintcalcAbs{#2}}%
  }%
  \x
}
\newcommand*{\TestMul}[3]{%
  \Test{\bigintcalcMul{#1}{#2}}{#3}%
  \edef\x{%
    \noexpand\Test{%
      \noexpand\BigIntCalcMul
      \bigintcalcAbs{#1}!\bigintcalcAbs{#2}!%
    }{\bigintcalcAbs{#3}}%
  }%
  \x
}
\newcommand*{\TestSqr}[2]{%
  \Test{\bigintcalcSqr{#1}}{#2}%
}
\newcommand*{\TestFac}[2]{%
  \expandafter\TestExch\expandafter{%
    \the\numexpr#2%
  }{\bigintcalcFac{#1}}%
}
\newcommand*{\TestFacBig}[2]{%
  \Test{\bigintcalcFac{#1}}{#2}%
}
\newcommand*{\TestPow}[3]{%
  \Test{\bigintcalcPow{#1}{#2}}{#3}%
}
\newcommand*{\TestDiv}[3]{%
  \Test{\bigintcalcDiv{#1}{#2}}{#3}%
  \TestTeXDivide{#1}{#2}%
}
\newcommand*{\TestDivBig}[3]{%
  \Test{\bigintcalcDiv{#1}{#2}}{#3}%
  \edef\x{%
    \noexpand\Test{%
      \noexpand\BigIntCalcDiv\bigintcalcAbs{#1}!\bigintcalcAbs{#2}!%
    }{\bigintcalcAbs{#3}}%
  }%
}
\newcommand*{\TestMod}[3]{%
  \Test{\bigintcalcMod{#1}{#2}}{#3}%
  \ifcase\ifcase\bigintcalcSgn{#1} 0%
         \or
           \ifcase\bigintcalcSgn{#2} 1%
           \or 0%
           \else 1%
           \fi
         \else
           \ifcase\bigintcalcSgn{#2} 1%
           \or 1%
           \else 0%
           \fi
         \fi\relax
    \edef\x{%
      \noexpand\Test{%
        \noexpand\BigIntCalcMod
        \bigintcalcAbs{#1}!\bigintcalcAbs{#2}!%
      }{\bigintcalcAbs{#3}}%
    }%
    \x
  \fi
}
%    \end{macrocode}
%
% \subsubsection{Time}
%
%    \begin{macrocode}
\begingroup\expandafter\expandafter\expandafter\endgroup
\expandafter\ifx\csname pdfresettimer\endcsname\relax
\else
  \makeatletter
  \newcount\SummaryTime
  \newcount\TestTime
  \SummaryTime=\z@
  \newcommand*{\PrintTime}[2]{%
    \typeout{%
      [Time #1: \strip@pt\dimexpr\number#2sp\relax\space s]%
    }%
  }%
  \newcommand*{\StartTime}[1]{%
    \renewcommand*{\TimeDescription}{#1}%
    \pdfresettimer
  }%
  \newcommand*{\TimeDescription}{}%
  \newcommand*{\StopTime}{%
    \TestTime=\pdfelapsedtime
    \global\advance\SummaryTime\TestTime
    \PrintTime\TimeDescription\TestTime
  }%
  \let\saved@qstest\qstest
  \let\saved@endqstest\endqstest
  \def\qstest#1#2{%
    \saved@qstest{#1}{#2}%
    \StartTime{#1}%
  }%
  \def\endqstest{%
    \StopTime
    \saved@endqstest
  }%
  \AtEndDocument{%
    \PrintTime{summary}\SummaryTime
  }%
  \makeatother
\fi
%    \end{macrocode}
%
% \subsubsection{Test sets}
%
%    \begin{macrocode}
\makeatletter

\begin{qstest}{inv}{inv}%
  \TestInv{0}{0}%
  \TestInv{1}{-1}%
  \TestInv{-1}{1}%
  \TestInv{10}{-10}%
  \TestInv{-10}{10}%
  \TestInv{2147483647}{-2147483647}%
  \TestInv{-2147483647}{2147483647}%
  \TestInv{12345678901234567890}{-12345678901234567890}%
  \TestInv{-12345678901234567890}{12345678901234567890}%
  \TestInv{ 0 }{0}%
  \TestInv{ 1 }{-1}%
  \TestInv{--1}{-1}%
  \TestInv{\number\z@}{0}%
  \TestInv{\ifx\relax\relax1\fi}{-1}%
  \TestInv{\ifx\relax\relax-\fi\ifx234\else1\fi}{1}%
\end{qstest}

\begin{qstest}{abs}{abs}%
  \TestAbs{0}{0}%
  \TestAbs{1}{1}%
  \TestAbs{-1}{1}%
  \TestAbs{10}{10}%
  \TestAbs{-10}{10}%
  \TestAbs{2147483647}{2147483647}%
  \TestAbs{-2147483647}{2147483647}%
  \TestAbs{12345678901234567890}{12345678901234567890}%
  \TestAbs{-12345678901234567890}{12345678901234567890}%
  \TestAbs{ 0 }{0}%
  \TestAbs{ 1 }{1}%
  \TestAbs{--1}{1}%
  \TestAbs{-+-+1}{1}%
  \TestAbs{00000000000}{0}%
  \TestAbs{00000001000}{1000}%
  \TestAbs{\ifx\relax\relax 0\else 1\fi}{0}%
\end{qstest}

\begin{qstest}{sign}{sign}%
  \TestSgn{0}{0}%
  \TestSgn{1}{1}%
  \TestSgn{-1}{-1}%
  \TestSgn{10}{1}%
  \TestSgn{-10}{-1}%
  \TestSgn{2147483647}{1}%
  \TestSgn{-2147483647}{-1}%
  \TestSgn{12345678901234567890}{1}%
  \TestSgn{-12345678901234567890}{-1}%
  \TestSgn{ 0 }{0}%
  \TestSgn{ 2 }{1}%
  \TestSgn{ -2 }{-1}%
  \TestSgn{--2}{1}%
  \TestSgn{\number\z@}{0}%
  \TestSgn{\number\@ne}{1}%
  \TestSgn{\number\m@ne}{-1}%
  \TestSgn{%
    -+-+\number\z@\number\z@
    \iftrue1\fi\iftrue2\fi\iftrue3\fi
  }{1}%
\end{qstest}

\begin{qstest}{min}{min}%
  \TestMin{0}{1}{0}%
  \TestMin{1}{0}{0}%
  \TestMin{-10}{-20}{-20}%
  \TestMin{ 1 }{ 2 }{1}%
  \TestMin{ 2 }{ 1 }{1}%
  \TestMin{1}{1}{1}%
  \TestMin{\number\z@}{\number\@ne}{0}%
  \TestMin{\number\@ne}{\number\m@ne}{-1}%
\end{qstest}

\begin{qstest}{max}{max}%
  \TestMax{0}{1}{1}%
  \TestMax{1}{0}{1}%
  \TestMax{-10}{-20}{-10}%
  \TestMax{ 1 }{ 2 }{2}%
  \TestMax{ 2 }{ 1 }{2}%
  \TestMax{1}{1}{1}%
  \TestMax{\number\z@}{\number\@ne}{1}%
  \TestMax{\number\@ne}{\number\m@ne}{1}%
\end{qstest}

\begin{qstest}{cmp}{cmp}%
  \TestCmp{0}{0}{0}%
  \TestCmp{-21}{17}{-1}%
  \TestCmp{3}{4}{-1}%
  \TestCmp{-10}{-10}{0}%
  \TestCmp{-10}{-11}{1}%
  \TestCmp{100}{5}{1}%
  \TestCmp{9}{10}{-1}%
  \TestCmp{10}{9}{1}%
  \TestCmp{ 3 }{ 3 }{0}%
  \TestCmp{-9}{-10}{1}%
  \TestCmp{-10}{-9}{-1}%
  \TestCmp{-3}{-3}{0}%
  \TestCmp{0}{-2}{1}%
  \TestCmp{0}{2}{-1}%
  \TestCmp{2}{0}{1}%
  \TestCmp{-2}{0}{-1}%
  \TestCmp{12}{11}{1}%
  \TestCmp{11}{12}{-1}%
  \TestCmp{2147483647}{-2147483647}{1}%
  \TestCmp{-2147483647}{2147483647}{-1}%
  \TestCmp{2147483647}{2147483647}{0}%
  \TestCmp{\number\z@}{\number\@ne}{-1}%
  \TestCmp{\number\@ne}{\number\m@ne}{1}%
  \TestCmp{ 4 }{ 5 }{-1}%
  \TestCmp{ -3 }{ -7 }{1}%
\end{qstest}

\begin{qstest}{odd}{odd}
\tracingmacros=1
  \TestOdd{0}{0}%
  \TestOdd{1}{1}%
  \TestOdd{2}{0}%
  \TestOdd{3}{1}%
  \TestOdd{14}{0}%
  \TestOdd{15}{1}%
  \TestOdd{12345678901234567896}{0}%
  \TestOdd{12345678901234567897}{1}%
\end{qstest}

\begin{qstest}{inc}{inc}%
  \TestInc{0}{1}%
  \TestInc{1}{2}%
  \TestInc{-1}{0}%
  \TestInc{10}{11}%
  \TestInc{-10}{-9}%
  \TestInc{ 3 }{4}%
  \TestInc{999}{1000}%
  \TestInc{-1000}{-999}%
  \TestInc{129}{130}%
  \TestInc{2147483646}{2147483647}%
  \TestInc{-2147483647}{-2147483646}%
  \TestInc{12345678901234567890}{12345678901234567891}%
  \TestInc{99999999999999999999}{100000000000000000000}%
  \TestInc{-12345678901234567891}{-12345678901234567890}%
  \TestInc{-100000000000000000000}{-99999999999999999999}%
\end{qstest}

\begin{qstest}{dec}{dec}%
  \TestDec{0}{-1}%
  \TestDec{1}{0}%
  \TestDec{-1}{-2}%
  \TestDec{10}{9}%
  \TestDec{-10}{-11}%
  \TestDec{1000}{999}%
  \TestDec{-999}{-1000}%
  \TestDec{130}{129}%
  \TestDec{2147483647}{2147483646}%
  \TestDec{-2147483646}{-2147483647}%
  \TestDec{12345678901234567891}{12345678901234567890}%
  \TestDec{100000000000000000000}{99999999999999999999}%
  \TestDec{-12345678901234567890}{-12345678901234567891}%
  \TestDec{-99999999999999999999}{-100000000000000000000}%
\end{qstest}

\begin{qstest}{add}{add}%
  \TestAdd{0}{0}{0}%
  \TestAdd{1}{0}{1}%
  \TestAdd{0}{1}{1}%
  \TestAdd{1}{2}{3}%
  \TestAdd{-1}{-1}{-2}%
  \TestAdd{2147483646}{1}{2147483647}%
  \TestAdd{-2147483647}{2147483647}{0}%
  \TestAdd{20}{-5}{15}%
  \TestAdd{-4}{-1}{-5}%
  \TestAdd{-1}{-4}{-5}%
  \TestAdd{-4}{1}{-3}%
  \TestAdd{-1}{4}{3}%
  \TestAdd{4}{-1}{3}%
  \TestAdd{1}{-4}{-3}%
  \TestAdd{-4}{-1}{-5}%
  \TestAdd{-1}{-4}{-5}%
  \TestAdd{ -4 }{ -1 }{-5}%
  \TestAdd{ -1 }{ -4 }{-5}%
  \TestAdd{ -4 }{ 1 }{-3}%
  \TestAdd{ -1 }{ 4 }{3}%
  \TestAdd{ 4 }{ -1 }{3}%
  \TestAdd{ 1 }{ -4 }{-3}%
  \TestAdd{ -4 }{ -1 }{-5}%
  \TestAdd{ -1 }{ -4 }{-5}%
  \TestAdd{876543210}{111111111}{987654321}%
  \TestAdd{999999999}{2}{1000000001}%
\end{qstest}

\begin{qstest}{sub}{sub}
  \TestSub{0}{0}{0}%
  \TestSub{1}{0}{1}%
  \TestSub{1}{2}{-1}%
  \TestSub{-1}{-1}{0}%
  \TestSub{2147483646}{-1}{2147483647}%
  \TestSub{-2147483647}{-2147483647}{0}%
  \TestSub{-4}{-1}{-3}%
  \TestSub{-1}{-4}{3}%
  \TestSub{-4}{1}{-5}%
  \TestSub{-1}{4}{-5}%
  \TestSub{4}{-1}{5}%
  \TestSub{1}{-4}{5}%
  \TestSub{-4}{-1}{-3}%
  \TestSub{-1}{-4}{3}%
  \TestSub{ -4 }{ -1 }{-3}%
  \TestSub{ -1 }{ -4 }{3}%
  \TestSub{ -4 }{ 1 }{-5}%
  \TestSub{ -1 }{ 4 }{-5}%
  \TestSub{ 4 }{ -1 }{5}%
  \TestSub{ 1 }{ -4 }{5}%
  \TestSub{ -4 }{ -1 }{-3}%
  \TestSub{ -1 }{ -4 }{3}%
  \TestSub{1000000000}{2}{999999998}%
  \TestSub{987654321}{111111111}{876543210}%
\end{qstest}

\begin{qstest}{shl}{shl}
  \TestShl{0}{0}%
  \TestShl{1}{2}%
  \TestShl{2}{4}%
  \TestShl{5621}{11242}%
  \TestShl{1073741823}{2147483646}%
\end{qstest}

\begin{qstest}{shr}{shr}
  \TestShr{0}{0}%
  \TestShr{1}{0}%
  \TestShr{2}{1}%
  \TestShr{3}{1}%
  \TestShr{4}{2}%
  \TestShr{5}{2}%
  \TestShr{6}{3}%
  \TestShr{7}{3}%
  \TestShr{8}{4}%
  \TestShr{9}{4}%
  \TestShr{10}{5}%
  \TestShr{11}{5}%
  \TestShr{12}{6}%
  \TestShr{13}{6}%
  \TestShr{14}{7}%
  \TestShr{15}{7}%
  \TestShr{16}{8}%
  \TestShr{17}{8}%
  \TestShr{18}{9}%
  \TestShr{19}{9}%
  \TestShr{20}{10}%
  \TestShr{21}{10}%
  \TestShr{22}{11}%
  \TestShr{11241}{5620}%
  \TestShr{73054202}{36527101}%
  \TestShr{2147483646}{1073741823}%
\end{qstest}

\begin{qstest}{mul}{mul}
  \TestMul{0}{0}{0}%
  \TestMul{1}{0}{0}%
  \TestMul{0}{1}{0}%
  \TestMul{1}{1}{1}%
  \TestMul{3}{1}{3}%
  \TestMul{1}{-3}{-3}%
  \TestMul{-4}{-5}{20}%
  \TestMul{3}{7}{21}%
  \TestMul{7}{3}{21}%
  \TestMul{3}{-7}{-21}%
  \TestMul{7}{-3}{-21}%
  \TestMul{-3}{7}{-21}%
  \TestMul{-7}{3}{-21}%
  \TestMul{-3}{-7}{21}%
  \TestMul{-7}{-3}{21}%
  \TestMul{12}{11}{132}%
  \TestMul{999}{333}{332667}%
  \TestMul{1000}{4321}{4321000}%
  \TestMul{12345}{173955}{2147474475}%
  \TestMul{1073741823}{2}{2147483646}%
  \TestMul{2}{1073741823}{2147483646}%
  \TestMul{-1073741823}{2}{-2147483646}%
  \TestMul{2}{-1073741823}{-2147483646}%
  \TestMul{6706022400}{13}{87178291200}%
\end{qstest}

\begin{qstest}{sqr}{sqr}
  \TestSqr{0}{0}%
  \TestSqr{1}{1}%
  \TestSqr{2}{4}%
  \TestSqr{3}{9}%
  \TestSqr{4}{16}%
  \TestSqr{9}{81}%
  \TestSqr{10}{100}%
  \TestSqr{46340}{2147395600}%
  \TestSqr{-1}{1}%
  \TestSqr{-2}{4}%
  \TestSqr{-46340}{2147395600}%
\end{qstest}

\begin{qstest}{fac}{fac}
  \TestFac{0}{1}%
  \TestFac{1}{1}%
  \TestFac{2}{2}%
  \TestFac{3}{2*3}%
  \TestFac{4}{2*3*4}%
  \TestFac{5}{2*3*4*5}%
  \TestFac{6}{2*3*4*5*6}%
  \TestFac{7}{2*3*4*5*6*7}%
  \TestFac{8}{2*3*4*5*6*7*8}%
  \TestFac{9}{2*3*4*5*6*7*8*9}%
  \TestFac{10}{2*3*4*5*6*7*8*9*10}%
  \TestFac{11}{2*3*4*5*6*7*8*9*10*11}%
  \TestFac{12}{2*3*4*5*6*7*8*9*10*11*12}%
  \TestFacBig{13}{6227020800}%
  \TestFacBig{14}{87178291200}%
  \TestFacBig{15}{1307674368000}%
  \TestFacBig{16}{20922789888000}%
  \TestFacBig{17}{355687428096000}%
  \TestFacBig{18}{6402373705728000}%
  \TestFacBig{19}{121645100408832000}%
  \TestFacBig{20}{2432902008176640000}%
  \TestFacBig{21}{51090942171709440000}%
  \TestFacBig{22}{1124000727777607680000}%
\end{qstest}

\begin{qstest}{pow}{pow}
  \TestPow{-2}{0}{1}%
  \TestPow{-1}{0}{1}%
  \TestPow{0}{0}{1}%
  \TestPow{1}{0}{1}%
  \TestPow{2}{0}{1}%
  \TestPow{3}{0}{1}%
  \TestPow{-2}{1}{-2}%
  \TestPow{-1}{1}{-1}%
  \TestPow{1}{1}{1}%
  \TestPow{2}{1}{2}%
  \TestPow{3}{1}{3}%
  \TestPow{-2}{2}{4}%
  \TestPow{-1}{2}{1}%
  \TestPow{0}{2}{0}%
  \TestPow{1}{2}{1}%
  \TestPow{2}{2}{4}%
  \TestPow{3}{2}{9}%
  \TestPow{0}{1}{0}%
  \TestPow{1}{-2}{1}%
  \TestPow{1}{-1}{1}%
  \TestPow{-1}{-2}{1}%
  \TestPow{-1}{-1}{-1}%
  \TestPow{-1}{3}{-1}%
  \TestPow{-1}{4}{1}%
  \TestPow{-2}{-1}{0}%
  \TestPow{-2}{-2}{0}%
  \TestPow{2}{3}{8}%
  \TestPow{2}{4}{16}%
  \TestPow{2}{5}{32}%
  \TestPow{2}{6}{64}%
  \TestPow{2}{7}{128}%
  \TestPow{2}{8}{256}%
  \TestPow{2}{9}{512}%
  \TestPow{2}{10}{1024}%
  \TestPow{-2}{3}{-8}%
  \TestPow{-2}{4}{16}%
  \TestPow{-2}{5}{-32}%
  \TestPow{-2}{6}{64}%
  \TestPow{-2}{7}{-128}%
  \TestPow{-2}{8}{256}%
  \TestPow{-2}{9}{-512}%
  \TestPow{-2}{10}{1024}%
  \TestPow{3}{3}{27}%
  \TestPow{3}{4}{81}%
  \TestPow{3}{5}{243}%
  \TestPow{-3}{3}{-27}%
  \TestPow{-3}{4}{81}%
  \TestPow{-3}{5}{-243}%
  \TestPow{2}{30}{1073741824}%
  \TestPow{-3}{19}{-1162261467}%
  \TestPow{5}{13}{1220703125}%
  \TestPow{-7}{11}{-1977326743}%
\end{qstest}

\begin{qstest}{div}{div}
  \TestDiv{1}{1}{1}%
  \TestDiv{2}{1}{2}%
  \TestDiv{-2}{1}{-2}%
  \TestDiv{2}{-1}{-2}%
  \TestDiv{-2}{-1}{2}%
  \TestDiv{15}{2}{7}%
  \TestDiv{-16}{2}{-8}%
  \TestDiv{1}{2}{0}%
  \TestDiv{1}{3}{0}%
  \TestDiv{2}{3}{0}%
  \TestDiv{-2}{3}{0}%
  \TestDiv{2}{-3}{0}%
  \TestDiv{-2}{-3}{0}%
  \TestDiv{13}{3}{4}%
  \TestDiv{-13}{-3}{4}%
  \TestDiv{-13}{3}{-4}%
  \TestDiv{-6}{5}{-1}%
  \TestDiv{-5}{5}{-1}%
  \TestDiv{-4}{5}{0}%
  \TestDiv{-3}{5}{0}%
  \TestDiv{-2}{5}{0}%
  \TestDiv{-1}{5}{0}%
  \TestDiv{0}{5}{0}%
  \TestDiv{1}{5}{0}%
  \TestDiv{2}{5}{0}%
  \TestDiv{3}{5}{0}%
  \TestDiv{4}{5}{0}%
  \TestDiv{5}{5}{1}%
  \TestDiv{6}{5}{1}%
  \TestDiv{-5}{4}{-1}%
  \TestDiv{-4}{4}{-1}%
  \TestDiv{-3}{4}{0}%
  \TestDiv{-2}{4}{0}%
  \TestDiv{-1}{4}{0}%
  \TestDiv{0}{4}{0}%
  \TestDiv{1}{4}{0}%
  \TestDiv{2}{4}{0}%
  \TestDiv{3}{4}{0}%
  \TestDiv{4}{4}{1}%
  \TestDiv{5}{4}{1}%
  \TestDiv{12345}{678}{18}%
  \TestDiv{32372}{5952}{5}%
  \TestDiv{284271294}{18162}{15651}%
  \TestDiv{217652429}{12561}{17327}%
  \TestDiv{462028434}{5439}{84947}%
  \TestDiv{2147483647}{1000}{2147483}%
  \TestDiv{2147483647}{-1000}{-2147483}%
  \TestDiv{-2147483647}{1000}{-2147483}%
  \TestDiv{-2147483647}{-1000}{2147483}%
  \TestDiv{0}{3}{0}%
  \TestDiv{1}{3}{0}%
  \TestDiv{2}{3}{0}%
  \TestDiv{3}{3}{1}%
  \TestDiv{4}{3}{1}%
  \TestDiv{5}{3}{1}%
  \TestDiv{6}{3}{2}%
  \TestDiv{7}{3}{2}%
  \TestDiv{8}{3}{2}%
  \TestDiv{9}{3}{3}%
  \TestDiv{10}{3}{3}%
  \TestDiv{11}{3}{3}%
  \TestDiv{12}{3}{4}%
  \TestDiv{13}{3}{4}%
  \TestDiv{14}{3}{4}%
  \TestDiv{15}{3}{5}%
  \TestDiv{16}{3}{5}%
  \TestDiv{17}{3}{5}%
  \TestDiv{18}{3}{6}%
  \TestDiv{19}{3}{6}%
  \TestDiv{20}{3}{6}%
  \TestDiv{21}{3}{7}%
  \TestDiv{22}{3}{7}%
  \TestDiv{23}{3}{7}%
  \TestDiv{24}{3}{8}%
  \TestDiv{25}{3}{8}%
  \TestDiv{26}{3}{8}%
  \TestDiv{27}{3}{9}%
  \TestDiv{28}{3}{9}%
  \TestDiv{29}{3}{9}%
  \TestDiv{30}{3}{10}%
  \TestDiv{31}{3}{10}%
  \TestDivBig{17363436332507}{24702}{702916214}%
\end{qstest}

\begin{qstest}{mod}{mod}
  \TestMod{-6}{5}{4}%
  \TestMod{-5}{5}{0}%
  \TestMod{-4}{5}{1}%
  \TestMod{-3}{5}{2}%
  \TestMod{-2}{5}{3}%
  \TestMod{-1}{5}{4}%
  \TestMod{0}{5}{0}%
  \TestMod{1}{5}{1}%
  \TestMod{2}{5}{2}%
  \TestMod{3}{5}{3}%
  \TestMod{4}{5}{4}%
  \TestMod{5}{5}{0}%
  \TestMod{6}{5}{1}%
  \TestMod{-5}{4}{3}%
  \TestMod{-4}{4}{0}%
  \TestMod{-3}{4}{1}%
  \TestMod{-2}{4}{2}%
  \TestMod{-1}{4}{3}%
  \TestMod{0}{4}{0}%
  \TestMod{1}{4}{1}%
  \TestMod{2}{4}{2}%
  \TestMod{3}{4}{3}%
  \TestMod{4}{4}{0}%
  \TestMod{5}{4}{1}%
  \TestMod{-6}{-5}{-1}%
  \TestMod{-5}{-5}{0}%
  \TestMod{-4}{-5}{-4}%
  \TestMod{-3}{-5}{-3}%
  \TestMod{-2}{-5}{-2}%
  \TestMod{-1}{-5}{-1}%
  \TestMod{0}{-5}{0}%
  \TestMod{1}{-5}{-4}%
  \TestMod{2}{-5}{-3}%
  \TestMod{3}{-5}{-2}%
  \TestMod{4}{-5}{-1}%
  \TestMod{5}{-5}{0}%
  \TestMod{6}{-5}{-4}%
  \TestMod{-5}{-4}{-1}%
  \TestMod{-4}{-4}{0}%
  \TestMod{-3}{-4}{-3}%
  \TestMod{-2}{-4}{-2}%
  \TestMod{-1}{-4}{-1}%
  \TestMod{0}{-4}{0}%
  \TestMod{1}{-4}{-3}%
  \TestMod{2}{-4}{-2}%
  \TestMod{3}{-4}{-1}%
  \TestMod{4}{-4}{0}%
  \TestMod{5}{-4}{-3}%
  \TestMod{2147483647}{1000}{647}%
  \TestMod{2147483647}{-1000}{-353}%
  \TestMod{-2147483647}{1000}{353}%
  \TestMod{-2147483647}{-1000}{-647}%
  \TestMod{ 0 }{ 4 }{0}%
  \TestMod{ 1 }{ 4 }{1}%
  \TestMod{ -1 }{ 4 }{3}%
  \TestMod{ 0 }{ -4 }{0}%
  \TestMod{ 1 }{ -4 }{-3}%
  \TestMod{ -1 }{ -4 }{-1}%
  \TestMod{18362}{25}{12}%
\end{qstest}

\newcommand*{\TestError}[2]{%
  \begingroup
    \expandafter\def\csname BigIntCalcError:#1\endcsname{}%
    \Expect*{#2}{0}%
    \expandafter\def\csname BigIntCalcError:#1\endcsname{ERROR}%
    \Expect*{#2}{0ERROR}%
  \endgroup
}
\begin{qstest}{error}{error}
  \TestError{FacNegative}{\bigintcalcFac{-1}}%
  \TestError{FacNegative}{\bigintcalcFac{-2147483647}}%
  \TestError{DivisionByZero}{\bigintcalcPow{0}{-1}}%
  \TestError{DivisionByZero}{\bigintcalcDiv{1}{0}}%
  \TestError{DivisionByZero}{\bigintcalcMod{1}{0}}%
\end{qstest}

\begin{document}
\end{document}
%</test2>
%    \end{macrocode}
%
% \section{Installation}
%
% \subsection{Download}
%
% \paragraph{Package.} This package is available on
% CTAN\footnote{\url{http://ctan.org/pkg/bigintcalc}}:
% \begin{description}
% \item[\CTAN{macros/latex/contrib/oberdiek/bigintcalc.dtx}] The source file.
% \item[\CTAN{macros/latex/contrib/oberdiek/bigintcalc.pdf}] Documentation.
% \end{description}
%
%
% \paragraph{Bundle.} All the packages of the bundle `oberdiek'
% are also available in a TDS compliant ZIP archive. There
% the packages are already unpacked and the documentation files
% are generated. The files and directories obey the TDS standard.
% \begin{description}
% \item[\CTAN{install/macros/latex/contrib/oberdiek.tds.zip}]
% \end{description}
% \emph{TDS} refers to the standard ``A Directory Structure
% for \TeX\ Files'' (\CTAN{tds/tds.pdf}). Directories
% with \xfile{texmf} in their name are usually organized this way.
%
% \subsection{Bundle installation}
%
% \paragraph{Unpacking.} Unpack the \xfile{oberdiek.tds.zip} in the
% TDS tree (also known as \xfile{texmf} tree) of your choice.
% Example (linux):
% \begin{quote}
%   |unzip oberdiek.tds.zip -d ~/texmf|
% \end{quote}
%
% \paragraph{Script installation.}
% Check the directory \xfile{TDS:scripts/oberdiek/} for
% scripts that need further installation steps.
% Package \xpackage{attachfile2} comes with the Perl script
% \xfile{pdfatfi.pl} that should be installed in such a way
% that it can be called as \texttt{pdfatfi}.
% Example (linux):
% \begin{quote}
%   |chmod +x scripts/oberdiek/pdfatfi.pl|\\
%   |cp scripts/oberdiek/pdfatfi.pl /usr/local/bin/|
% \end{quote}
%
% \subsection{Package installation}
%
% \paragraph{Unpacking.} The \xfile{.dtx} file is a self-extracting
% \docstrip\ archive. The files are extracted by running the
% \xfile{.dtx} through \plainTeX:
% \begin{quote}
%   \verb|tex bigintcalc.dtx|
% \end{quote}
%
% \paragraph{TDS.} Now the different files must be moved into
% the different directories in your installation TDS tree
% (also known as \xfile{texmf} tree):
% \begin{quote}
% \def\t{^^A
% \begin{tabular}{@{}>{\ttfamily}l@{ $\rightarrow$ }>{\ttfamily}l@{}}
%   bigintcalc.sty & tex/generic/oberdiek/bigintcalc.sty\\
%   bigintcalc.pdf & doc/latex/oberdiek/bigintcalc.pdf\\
%   test/bigintcalc-test1.tex & doc/latex/oberdiek/test/bigintcalc-test1.tex\\
%   test/bigintcalc-test2.tex & doc/latex/oberdiek/test/bigintcalc-test2.tex\\
%   test/bigintcalc-test3.tex & doc/latex/oberdiek/test/bigintcalc-test3.tex\\
%   bigintcalc.dtx & source/latex/oberdiek/bigintcalc.dtx\\
% \end{tabular}^^A
% }^^A
% \sbox0{\t}^^A
% \ifdim\wd0>\linewidth
%   \begingroup
%     \advance\linewidth by\leftmargin
%     \advance\linewidth by\rightmargin
%   \edef\x{\endgroup
%     \def\noexpand\lw{\the\linewidth}^^A
%   }\x
%   \def\lwbox{^^A
%     \leavevmode
%     \hbox to \linewidth{^^A
%       \kern-\leftmargin\relax
%       \hss
%       \usebox0
%       \hss
%       \kern-\rightmargin\relax
%     }^^A
%   }^^A
%   \ifdim\wd0>\lw
%     \sbox0{\small\t}^^A
%     \ifdim\wd0>\linewidth
%       \ifdim\wd0>\lw
%         \sbox0{\footnotesize\t}^^A
%         \ifdim\wd0>\linewidth
%           \ifdim\wd0>\lw
%             \sbox0{\scriptsize\t}^^A
%             \ifdim\wd0>\linewidth
%               \ifdim\wd0>\lw
%                 \sbox0{\tiny\t}^^A
%                 \ifdim\wd0>\linewidth
%                   \lwbox
%                 \else
%                   \usebox0
%                 \fi
%               \else
%                 \lwbox
%               \fi
%             \else
%               \usebox0
%             \fi
%           \else
%             \lwbox
%           \fi
%         \else
%           \usebox0
%         \fi
%       \else
%         \lwbox
%       \fi
%     \else
%       \usebox0
%     \fi
%   \else
%     \lwbox
%   \fi
% \else
%   \usebox0
% \fi
% \end{quote}
% If you have a \xfile{docstrip.cfg} that configures and enables \docstrip's
% TDS installing feature, then some files can already be in the right
% place, see the documentation of \docstrip.
%
% \subsection{Refresh file name databases}
%
% If your \TeX~distribution
% (\teTeX, \mikTeX, \dots) relies on file name databases, you must refresh
% these. For example, \teTeX\ users run \verb|texhash| or
% \verb|mktexlsr|.
%
% \subsection{Some details for the interested}
%
% \paragraph{Attached source.}
%
% The PDF documentation on CTAN also includes the
% \xfile{.dtx} source file. It can be extracted by
% AcrobatReader 6 or higher. Another option is \textsf{pdftk},
% e.g. unpack the file into the current directory:
% \begin{quote}
%   \verb|pdftk bigintcalc.pdf unpack_files output .|
% \end{quote}
%
% \paragraph{Unpacking with \LaTeX.}
% The \xfile{.dtx} chooses its action depending on the format:
% \begin{description}
% \item[\plainTeX:] Run \docstrip\ and extract the files.
% \item[\LaTeX:] Generate the documentation.
% \end{description}
% If you insist on using \LaTeX\ for \docstrip\ (really,
% \docstrip\ does not need \LaTeX), then inform the autodetect routine
% about your intention:
% \begin{quote}
%   \verb|latex \let\install=y% \iffalse meta-comment
%
% File: bigintcalc.dtx
% Version: 2016/05/16 v1.4
% Info: Expandable calculations on big integers
%
% Copyright (C) 2007, 2011, 2012 by
%    Heiko Oberdiek <heiko.oberdiek at googlemail.com>
%    2016
%    https://github.com/ho-tex/oberdiek/issues
%
% This work may be distributed and/or modified under the
% conditions of the LaTeX Project Public License, either
% version 1.3c of this license or (at your option) any later
% version. This version of this license is in
%    http://www.latex-project.org/lppl/lppl-1-3c.txt
% and the latest version of this license is in
%    http://www.latex-project.org/lppl.txt
% and version 1.3 or later is part of all distributions of
% LaTeX version 2005/12/01 or later.
%
% This work has the LPPL maintenance status "maintained".
%
% This Current Maintainer of this work is Heiko Oberdiek.
%
% The Base Interpreter refers to any `TeX-Format',
% because some files are installed in TDS:tex/generic//.
%
% This work consists of the main source file bigintcalc.dtx
% and the derived files
%    bigintcalc.sty, bigintcalc.pdf, bigintcalc.ins, bigintcalc.drv,
%    bigintcalc-test1.tex, bigintcalc-test2.tex,
%    bigintcalc-test3.tex.
%
% Distribution:
%    CTAN:macros/latex/contrib/oberdiek/bigintcalc.dtx
%    CTAN:macros/latex/contrib/oberdiek/bigintcalc.pdf
%
% Unpacking:
%    (a) If bigintcalc.ins is present:
%           tex bigintcalc.ins
%    (b) Without bigintcalc.ins:
%           tex bigintcalc.dtx
%    (c) If you insist on using LaTeX
%           latex \let\install=y% \iffalse meta-comment
%
% File: bigintcalc.dtx
% Version: 2016/05/16 v1.4
% Info: Expandable calculations on big integers
%
% Copyright (C) 2007, 2011, 2012 by
%    Heiko Oberdiek <heiko.oberdiek at googlemail.com>
%    2016
%    https://github.com/ho-tex/oberdiek/issues
%
% This work may be distributed and/or modified under the
% conditions of the LaTeX Project Public License, either
% version 1.3c of this license or (at your option) any later
% version. This version of this license is in
%    http://www.latex-project.org/lppl/lppl-1-3c.txt
% and the latest version of this license is in
%    http://www.latex-project.org/lppl.txt
% and version 1.3 or later is part of all distributions of
% LaTeX version 2005/12/01 or later.
%
% This work has the LPPL maintenance status "maintained".
%
% This Current Maintainer of this work is Heiko Oberdiek.
%
% The Base Interpreter refers to any `TeX-Format',
% because some files are installed in TDS:tex/generic//.
%
% This work consists of the main source file bigintcalc.dtx
% and the derived files
%    bigintcalc.sty, bigintcalc.pdf, bigintcalc.ins, bigintcalc.drv,
%    bigintcalc-test1.tex, bigintcalc-test2.tex,
%    bigintcalc-test3.tex.
%
% Distribution:
%    CTAN:macros/latex/contrib/oberdiek/bigintcalc.dtx
%    CTAN:macros/latex/contrib/oberdiek/bigintcalc.pdf
%
% Unpacking:
%    (a) If bigintcalc.ins is present:
%           tex bigintcalc.ins
%    (b) Without bigintcalc.ins:
%           tex bigintcalc.dtx
%    (c) If you insist on using LaTeX
%           latex \let\install=y\input{bigintcalc.dtx}
%        (quote the arguments according to the demands of your shell)
%
% Documentation:
%    (a) If bigintcalc.drv is present:
%           latex bigintcalc.drv
%    (b) Without bigintcalc.drv:
%           latex bigintcalc.dtx; ...
%    The class ltxdoc loads the configuration file ltxdoc.cfg
%    if available. Here you can specify further options, e.g.
%    use A4 as paper format:
%       \PassOptionsToClass{a4paper}{article}
%
%    Programm calls to get the documentation (example):
%       pdflatex bigintcalc.dtx
%       makeindex -s gind.ist bigintcalc.idx
%       pdflatex bigintcalc.dtx
%       makeindex -s gind.ist bigintcalc.idx
%       pdflatex bigintcalc.dtx
%
% Installation:
%    TDS:tex/generic/oberdiek/bigintcalc.sty
%    TDS:doc/latex/oberdiek/bigintcalc.pdf
%    TDS:doc/latex/oberdiek/test/bigintcalc-test1.tex
%    TDS:doc/latex/oberdiek/test/bigintcalc-test2.tex
%    TDS:doc/latex/oberdiek/test/bigintcalc-test3.tex
%    TDS:source/latex/oberdiek/bigintcalc.dtx
%
%<*ignore>
\begingroup
  \catcode123=1 %
  \catcode125=2 %
  \def\x{LaTeX2e}%
\expandafter\endgroup
\ifcase 0\ifx\install y1\fi\expandafter
         \ifx\csname processbatchFile\endcsname\relax\else1\fi
         \ifx\fmtname\x\else 1\fi\relax
\else\csname fi\endcsname
%</ignore>
%<*install>
\input docstrip.tex
\Msg{************************************************************************}
\Msg{* Installation}
\Msg{* Package: bigintcalc 2016/05/16 v1.4 Expandable calculations on big integers (HO)}
\Msg{************************************************************************}

\keepsilent
\askforoverwritefalse

\let\MetaPrefix\relax
\preamble

This is a generated file.

Project: bigintcalc
Version: 2016/05/16 v1.4

Copyright (C) 2007, 2011, 2012 by
   Heiko Oberdiek <heiko.oberdiek at googlemail.com>

This work may be distributed and/or modified under the
conditions of the LaTeX Project Public License, either
version 1.3c of this license or (at your option) any later
version. This version of this license is in
   http://www.latex-project.org/lppl/lppl-1-3c.txt
and the latest version of this license is in
   http://www.latex-project.org/lppl.txt
and version 1.3 or later is part of all distributions of
LaTeX version 2005/12/01 or later.

This work has the LPPL maintenance status "maintained".

This Current Maintainer of this work is Heiko Oberdiek.

The Base Interpreter refers to any `TeX-Format',
because some files are installed in TDS:tex/generic//.

This work consists of the main source file bigintcalc.dtx
and the derived files
   bigintcalc.sty, bigintcalc.pdf, bigintcalc.ins, bigintcalc.drv,
   bigintcalc-test1.tex, bigintcalc-test2.tex,
   bigintcalc-test3.tex.

\endpreamble
\let\MetaPrefix\DoubleperCent

\generate{%
  \file{bigintcalc.ins}{\from{bigintcalc.dtx}{install}}%
  \file{bigintcalc.drv}{\from{bigintcalc.dtx}{driver}}%
  \usedir{tex/generic/oberdiek}%
  \file{bigintcalc.sty}{\from{bigintcalc.dtx}{package}}%
%  \usedir{doc/latex/oberdiek/test}%
%  \file{bigintcalc-test1.tex}{\from{bigintcalc.dtx}{test1}}%
%  \file{bigintcalc-test2.tex}{\from{bigintcalc.dtx}{test2,etex}}%
%  \file{bigintcalc-test3.tex}{\from{bigintcalc.dtx}{test2,noetex}}%
  \nopreamble
  \nopostamble
  \usedir{source/latex/oberdiek/catalogue}%
  \file{bigintcalc.xml}{\from{bigintcalc.dtx}{catalogue}}%
}

\catcode32=13\relax% active space
\let =\space%
\Msg{************************************************************************}
\Msg{*}
\Msg{* To finish the installation you have to move the following}
\Msg{* file into a directory searched by TeX:}
\Msg{*}
\Msg{*     bigintcalc.sty}
\Msg{*}
\Msg{* To produce the documentation run the file `bigintcalc.drv'}
\Msg{* through LaTeX.}
\Msg{*}
\Msg{* Happy TeXing!}
\Msg{*}
\Msg{************************************************************************}

\endbatchfile
%</install>
%<*ignore>
\fi
%</ignore>
%<*driver>
\NeedsTeXFormat{LaTeX2e}
\ProvidesFile{bigintcalc.drv}%
  [2016/05/16 v1.4 Expandable calculations on big integers (HO)]%
\documentclass{ltxdoc}
\usepackage{wasysym}
\let\iint\relax
\let\iiint\relax
\usepackage[fleqn]{amsmath}

\DeclareMathOperator{\opInv}{Inv}
\DeclareMathOperator{\opAbs}{Abs}
\DeclareMathOperator{\opSgn}{Sgn}
\DeclareMathOperator{\opMin}{Min}
\DeclareMathOperator{\opMax}{Max}
\DeclareMathOperator{\opCmp}{Cmp}
\DeclareMathOperator{\opOdd}{Odd}
\DeclareMathOperator{\opInc}{Inc}
\DeclareMathOperator{\opDec}{Dec}
\DeclareMathOperator{\opAdd}{Add}
\DeclareMathOperator{\opSub}{Sub}
\DeclareMathOperator{\opShl}{Shl}
\DeclareMathOperator{\opShr}{Shr}
\DeclareMathOperator{\opMul}{Mul}
\DeclareMathOperator{\opSqr}{Sqr}
\DeclareMathOperator{\opFac}{Fac}
\DeclareMathOperator{\opPow}{Pow}
\DeclareMathOperator{\opDiv}{Div}
\DeclareMathOperator{\opMod}{Mod}
\DeclareMathOperator{\opInt}{Int}
\DeclareMathOperator{\opODD}{ifodd}

\newcommand*{\Def}{%
  \ensuremath{%
    \mathrel{\mathop{:}}=%
  }%
}
\newcommand*{\op}[1]{%
  \textsf{#1}%
}
\usepackage{holtxdoc}[2011/11/22]
\begin{document}
  \DocInput{bigintcalc.dtx}%
\end{document}
%</driver>
% \fi
%
%
% \CharacterTable
%  {Upper-case    \A\B\C\D\E\F\G\H\I\J\K\L\M\N\O\P\Q\R\S\T\U\V\W\X\Y\Z
%   Lower-case    \a\b\c\d\e\f\g\h\i\j\k\l\m\n\o\p\q\r\s\t\u\v\w\x\y\z
%   Digits        \0\1\2\3\4\5\6\7\8\9
%   Exclamation   \!     Double quote  \"     Hash (number) \#
%   Dollar        \$     Percent       \%     Ampersand     \&
%   Acute accent  \'     Left paren    \(     Right paren   \)
%   Asterisk      \*     Plus          \+     Comma         \,
%   Minus         \-     Point         \.     Solidus       \/
%   Colon         \:     Semicolon     \;     Less than     \<
%   Equals        \=     Greater than  \>     Question mark \?
%   Commercial at \@     Left bracket  \[     Backslash     \\
%   Right bracket \]     Circumflex    \^     Underscore    \_
%   Grave accent  \`     Left brace    \{     Vertical bar  \|
%   Right brace   \}     Tilde         \~}
%
% \GetFileInfo{bigintcalc.drv}
%
% \title{The \xpackage{bigintcalc} package}
% \date{2016/05/16 v1.4}
% \author{Heiko Oberdiek\thanks
% {Please report any issues at https://github.com/ho-tex/oberdiek/issues}\\
% \xemail{heiko.oberdiek at googlemail.com}}
%
% \maketitle
%
% \begin{abstract}
% This package provides expandable arithmetic operations
% with big integers that can exceed \TeX's number limits.
% \end{abstract}
%
% \tableofcontents
%
% \section{Documentation}
%
% \subsection{Introduction}
%
% Package \xpackage{bigintcalc} defines arithmetic operations
% that deal with big integers. Big integers can be given
% either as explicit integer number or as macro code
% that expands to an explicit number. \emph{Big} means that
% there is no limit on the size of the number. Big integers
% may exceed \TeX's range limitation of -2147483647 and 2147483647.
% Only memory issues will limit the usable range.
%
% In opposite to package \xpackage{intcalc}
% unexpandable command tokens are not supported, even if
% they are valid \TeX\ numbers like count registers or
% commands created by \cs{chardef}. Nevertheless they may
% be used, if they are prefixed by \cs{number}.
%
% Also \eTeX's \cs{numexpr} expressions are not supported
% directly in the manner of package \xpackage{intcalc}.
% However they can be given if \cs{the}\cs{numexpr}
% or \cs{number}\cs{numexpr} are used.
%
% The operations have the form of macros that take one or
% two integers as parameter and return the integer result.
% The macro name is a three letter operation name prefixed
% by the package name, e.g. \cs{bigintcalcAdd}|{10}{43}| returns
% |53|.
%
% The macros are fully expandable, exactly two expansion
% steps generate the result. Therefore the operations
% may be used nearly everywhere in \TeX, even inside
% \cs{csname}, file names,  or other
% expandable contexts.
%
% \subsection{Conditions}
%
% \subsubsection{Preconditions}
%
% \begin{itemize}
% \item
%   Arguments can be anything that expands to a number that consists
%   of optional signs and digits.
% \item
%   The arguments and return values must be sound.
%   Zero as divisor or factorials of negative numbers will cause errors.
% \end{itemize}
%
% \subsubsection{Postconditions}
%
% Additional properties of the macros apart from calculating
% a correct result (of course \smiley):
% \begin{itemize}
% \item
%   The macros are fully expandable. Thus they can
%   be used inside \cs{edef}, \cs{csname},
%   for example.
% \item
%   Furthermore exactly two expansion steps calculate the result.
% \item
%   The number consists of one optional minus sign and one or more
%   digits. The first digit is larger than zero for
%   numbers that consists of more than one digit.
%
%   In short, the number format is exactly the same as
%   \cs{number} generates, but without its range limitation.
%   And the tokens (minus sign, digits)
%   have catcode 12 (other).
% \item
%   Call by value is simulated. First the arguments are
%   converted to numbers. Then these numbers are used
%   in the calculations.
%
%   Remember that arguments
%   may contain expensive macros or \eTeX\ expressions.
%   This strategy avoids multiple evaluations of such
%   arguments.
% \end{itemize}
%
% \subsection{Error handling}
%
% Some errors are detected by the macros, example: division by zero.
% In this cases an undefined control sequence is called and causes
% a TeX error message, example: \cs{BigIntCalcError:DivisionByZero}.
% The name of the control sequence contains
% the reason for the error. The \TeX\ error may be ignored.
% Then the operation returns zero as result.
% Because the macros are supposed to work in expandible contexts.
% An traditional error message, however, is not expandable and
% would break these contexts.
%
% \subsection{Operations}
%
% Some definition equations below use the function $\opInt$
% that converts a real number to an integer. The number
% is truncated that means rounding to zero:
% \begin{gather*}
%   \opInt(x) \Def
%   \begin{cases}
%     \lfloor x\rfloor & \text{if $x\geq0$}\\
%     \lceil x\rceil & \text{otherwise}
%   \end{cases}
% \end{gather*}
%
% \subsubsection{\op{Num}}
%
% \begin{declcs}{bigintcalcNum} \M{x}
% \end{declcs}
% Macro \cs{bigintcalcNum} converts its argument to a normalized integer
% number without unnecessary leading zeros or signs.
% The result matches the regular expression:
%\begin{quote}
%\begin{verbatim}
%0|-?[1-9][0-9]*
%\end{verbatim}
%\end{quote}
%
% \subsubsection{\op{Inv}, \op{Abs}, \op{Sgn}}
%
% \begin{declcs}{bigintcalcInv} \M{x}
% \end{declcs}
% Macro \cs{bigintcalcInv} switches the sign.
% \begin{gather*}
%   \opInv(x) \Def -x
% \end{gather*}
%
% \begin{declcs}{bigintcalcAbs} \M{x}
% \end{declcs}
% Macro \cs{bigintcalcAbs} returns the absolute value
% of integer \meta{x}.
% \begin{gather*}
%   \opAbs(x) \Def \vert x\vert
% \end{gather*}
%
% \begin{declcs}{bigintcalcSgn} \M{x}
% \end{declcs}
% Macro \cs{bigintcalcSgn} encodes the sign of \meta{x} as number.
% \begin{gather*}
%   \opSgn(x) \Def
%   \begin{cases}
%     -1& \text{if $x<0$}\\
%      0& \text{if $x=0$}\\
%      1& \text{if $x>0$}
%   \end{cases}
% \end{gather*}
% These return values can easily be distinguished by \cs{ifcase}:
%\begin{quote}
%\begin{verbatim}
%\ifcase\bigintcalcSgn{<x>}
%  $x=0$
%\or
%  $x>0$
%\else
%  $x<0$
%\fi
%\end{verbatim}
%\end{quote}
%
% \subsubsection{\op{Min}, \op{Max}, \op{Cmp}}
%
% \begin{declcs}{bigintcalcMin} \M{x} \M{y}
% \end{declcs}
% Macro \cs{bigintcalcMin} returns the smaller of the two integers.
% \begin{gather*}
%   \opMin(x,y) \Def
%   \begin{cases}
%     x & \text{if $x<y$}\\
%     y & \text{otherwise}
%   \end{cases}
% \end{gather*}
%
% \begin{declcs}{bigintcalcMax} \M{x} \M{y}
% \end{declcs}
% Macro \cs{bigintcalcMax} returns the larger of the two integers.
% \begin{gather*}
%   \opMax(x,y) \Def
%   \begin{cases}
%     x & \text{if $x>y$}\\
%     y & \text{otherwise}
%   \end{cases}
% \end{gather*}
%
% \begin{declcs}{bigintcalcCmp} \M{x} \M{y}
% \end{declcs}
% Macro \cs{bigintcalcCmp} encodes the comparison result as number:
% \begin{gather*}
%   \opCmp(x,y) \Def
%   \begin{cases}
%     -1 & \text{if $x<y$}\\
%      0 & \text{if $x=y$}\\
%      1 & \text{if $x>y$}
%   \end{cases}
% \end{gather*}
% These values can be distinguished by \cs{ifcase}:
%\begin{quote}
%\begin{verbatim}
%\ifcase\bigintcalcCmp{<x>}{<y>}
%  $x=y$
%\or
%  $x>y$
%\else
%  $x<y$
%\fi
%\end{verbatim}
%\end{quote}
%
% \subsubsection{\op{Odd}}
%
% \begin{declcs}{bigintcalcOdd} \M{x}
% \end{declcs}
% \begin{gather*}
%   \opOdd(x) \Def
%   \begin{cases}
%     1 & \text{if $x$ is odd}\\
%     0 & \text{if $x$ is even}
%   \end{cases}
% \end{gather*}
%
% \subsubsection{\op{Inc}, \op{Dec}, \op{Add}, \op{Sub}}
%
% \begin{declcs}{bigintcalcInc} \M{x}
% \end{declcs}
% Macro \cs{bigintcalcInc} increments \meta{x} by one.
% \begin{gather*}
%   \opInc(x) \Def x + 1
% \end{gather*}
%
% \begin{declcs}{bigintcalcDec} \M{x}
% \end{declcs}
% Macro \cs{bigintcalcDec} decrements \meta{x} by one.
% \begin{gather*}
%   \opDec(x) \Def x - 1
% \end{gather*}
%
% \begin{declcs}{bigintcalcAdd} \M{x} \M{y}
% \end{declcs}
% Macro \cs{bigintcalcAdd} adds the two numbers.
% \begin{gather*}
%   \opAdd(x, y) \Def x + y
% \end{gather*}
%
% \begin{declcs}{bigintcalcSub} \M{x} \M{y}
% \end{declcs}
% Macro \cs{bigintcalcSub} calculates the difference.
% \begin{gather*}
%   \opSub(x, y) \Def x - y
% \end{gather*}
%
% \subsubsection{\op{Shl}, \op{Shr}}
%
% \begin{declcs}{bigintcalcShl} \M{x}
% \end{declcs}
% Macro \cs{bigintcalcShl} implements shifting to the left that
% means the number is multiplied by two.
% The sign is preserved.
% \begin{gather*}
%   \opShl(x) \Def x*2
% \end{gather*}
%
% \begin{declcs}{bigintcalcShr} \M{x}
% \end{declcs}
% Macro \cs{bigintcalcShr} implements shifting to the right.
% That is equivalent to an integer division by two.
% The sign is preserved.
% \begin{gather*}
%   \opShr(x) \Def \opInt(x/2)
% \end{gather*}
%
% \subsubsection{\op{Mul}, \op{Sqr}, \op{Fac}, \op{Pow}}
%
% \begin{declcs}{bigintcalcMul} \M{x} \M{y}
% \end{declcs}
% Macro \cs{bigintcalcMul} calculates the product of
% \meta{x} and \meta{y}.
% \begin{gather*}
%   \opMul(x,y) \Def x*y
% \end{gather*}
%
% \begin{declcs}{bigintcalcSqr} \M{x}
% \end{declcs}
% Macro \cs{bigintcalcSqr} returns the square product.
% \begin{gather*}
%   \opSqr(x) \Def x^2
% \end{gather*}
%
% \begin{declcs}{bigintcalcFac} \M{x}
% \end{declcs}
% Macro \cs{bigintcalcFac} returns the factorial of \meta{x}.
% Negative numbers are not permitted.
% \begin{gather*}
%   \opFac(x) \Def x!\qquad\text{for $x\geq0$}
% \end{gather*}
% ($0! = 1$)
%
% \begin{declcs}{bigintcalcPow} M{x} M{y}
% \end{declcs}
% Macro \cs{bigintcalcPow} calculates the value of \meta{x} to the
% power of \meta{y}. The error ``division by zero'' is thrown
% if \meta{x} is zero and \meta{y} is negative.
% permitted:
% \begin{gather*}
%   \opPow(x,y) \Def
%   \opInt(x^y)\qquad\text{for $x\neq0$ or $y\geq0$}
% \end{gather*}
% ($0^0 = 1$)
%
% \subsubsection{\op{Div}, \op{Mul}}
%
% \begin{declcs}{bigintcalcDiv} \M{x} \M{y}
% \end{declcs}
% Macro \cs{bigintcalcDiv} performs an integer division.
% Argument \meta{y} must not be zero.
% \begin{gather*}
%   \opDiv(x,y) \Def \opInt(x/y)\qquad\text{for $y\neq0$}
% \end{gather*}
%
% \begin{declcs}{bigintcalcMod} \M{x} \M{y}
% \end{declcs}
% Macro \cs{bigintcalcMod} gets the remainder of the integer
% division. The sign follows the divisor \meta{y}.
% Argument \meta{y} must not be zero.
% \begin{gather*}
%   \opMod(x,y) \Def x\mathrel{\%}y\qquad\text{for $y\neq0$}
% \end{gather*}
% The result ranges:
% \begin{gather*}
%   -\vert y\vert < \opMod(x,y) \leq 0\qquad\text{for $y<0$}\\
%   0 \leq \opMod(x,y) < y\qquad\text{for $y\geq0$}
% \end{gather*}
%
% \subsection{Interface for programmers}
%
% If the programmer can ensure some more properties about
% the arguments of the operations, then the following
% macros are a little more efficient.
%
% In general numbers must obey the following constraints:
% \begin{itemize}
% \item Plain number: digit tokens only, no command tokens.
% \item Non-negative. Signs are forbidden.
% \item Delimited by exclamation mark. Curly braces
%       around the number are not allowed and will
%       break the code.
% \end{itemize}
%
% \begin{declcs}{BigIntCalcOdd} \meta{number} |!|
% \end{declcs}
% |1|/|0| is returned if \meta{number} is odd/even.
%
% \begin{declcs}{BigIntCalcInc} \meta{number} |!|
% \end{declcs}
% Incrementation.
%
% \begin{declcs}{BigIntCalcDec} \meta{number} |!|
% \end{declcs}
% Decrementation, positive number without zero.
%
% \begin{declcs}{BigIntCalcAdd} \meta{number A} |!| \meta{number B} |!|
% \end{declcs}
% Addition, $A\geq B$.
%
% \begin{declcs}{BigIntCalcSub} \meta{number A} |!| \meta{number B} |!|
% \end{declcs}
% Subtraction, $A\geq B$.
%
% \begin{declcs}{BigIntCalcShl} \meta{number} |!|
% \end{declcs}
% Left shift (multiplication with two).
%
% \begin{declcs}{BigIntCalcShr} \meta{number} |!|
% \end{declcs}
% Right shift (integer division by two).
%
% \begin{declcs}{BigIntCalcMul} \meta{number A} |!| \meta{number B} |!|
% \end{declcs}
% Multiplication, $A\geq B$.
%
% \begin{declcs}{BigIntCalcDiv} \meta{number A} |!| \meta{number B} |!|
% \end{declcs}
% Division operation.
%
% \begin{declcs}{BigIntCalcMod} \meta{number A} |!| \meta{number B} |!|
% \end{declcs}
% Modulo operation.
%
%
% \StopEventually{
% }
%
% \section{Implementation}
%
%    \begin{macrocode}
%<*package>
%    \end{macrocode}
%
% \subsection{Reload check and package identification}
%    Reload check, especially if the package is not used with \LaTeX.
%    \begin{macrocode}
\begingroup\catcode61\catcode48\catcode32=10\relax%
  \catcode13=5 % ^^M
  \endlinechar=13 %
  \catcode35=6 % #
  \catcode39=12 % '
  \catcode44=12 % ,
  \catcode45=12 % -
  \catcode46=12 % .
  \catcode58=12 % :
  \catcode64=11 % @
  \catcode123=1 % {
  \catcode125=2 % }
  \expandafter\let\expandafter\x\csname ver@bigintcalc.sty\endcsname
  \ifx\x\relax % plain-TeX, first loading
  \else
    \def\empty{}%
    \ifx\x\empty % LaTeX, first loading,
      % variable is initialized, but \ProvidesPackage not yet seen
    \else
      \expandafter\ifx\csname PackageInfo\endcsname\relax
        \def\x#1#2{%
          \immediate\write-1{Package #1 Info: #2.}%
        }%
      \else
        \def\x#1#2{\PackageInfo{#1}{#2, stopped}}%
      \fi
      \x{bigintcalc}{The package is already loaded}%
      \aftergroup\endinput
    \fi
  \fi
\endgroup%
%    \end{macrocode}
%    Package identification:
%    \begin{macrocode}
\begingroup\catcode61\catcode48\catcode32=10\relax%
  \catcode13=5 % ^^M
  \endlinechar=13 %
  \catcode35=6 % #
  \catcode39=12 % '
  \catcode40=12 % (
  \catcode41=12 % )
  \catcode44=12 % ,
  \catcode45=12 % -
  \catcode46=12 % .
  \catcode47=12 % /
  \catcode58=12 % :
  \catcode64=11 % @
  \catcode91=12 % [
  \catcode93=12 % ]
  \catcode123=1 % {
  \catcode125=2 % }
  \expandafter\ifx\csname ProvidesPackage\endcsname\relax
    \def\x#1#2#3[#4]{\endgroup
      \immediate\write-1{Package: #3 #4}%
      \xdef#1{#4}%
    }%
  \else
    \def\x#1#2[#3]{\endgroup
      #2[{#3}]%
      \ifx#1\@undefined
        \xdef#1{#3}%
      \fi
      \ifx#1\relax
        \xdef#1{#3}%
      \fi
    }%
  \fi
\expandafter\x\csname ver@bigintcalc.sty\endcsname
\ProvidesPackage{bigintcalc}%
  [2016/05/16 v1.4 Expandable calculations on big integers (HO)]%
%    \end{macrocode}
%
% \subsection{Catcodes}
%
%    \begin{macrocode}
\begingroup\catcode61\catcode48\catcode32=10\relax%
  \catcode13=5 % ^^M
  \endlinechar=13 %
  \catcode123=1 % {
  \catcode125=2 % }
  \catcode64=11 % @
  \def\x{\endgroup
    \expandafter\edef\csname BIC@AtEnd\endcsname{%
      \endlinechar=\the\endlinechar\relax
      \catcode13=\the\catcode13\relax
      \catcode32=\the\catcode32\relax
      \catcode35=\the\catcode35\relax
      \catcode61=\the\catcode61\relax
      \catcode64=\the\catcode64\relax
      \catcode123=\the\catcode123\relax
      \catcode125=\the\catcode125\relax
    }%
  }%
\x\catcode61\catcode48\catcode32=10\relax%
\catcode13=5 % ^^M
\endlinechar=13 %
\catcode35=6 % #
\catcode64=11 % @
\catcode123=1 % {
\catcode125=2 % }
\def\TMP@EnsureCode#1#2{%
  \edef\BIC@AtEnd{%
    \BIC@AtEnd
    \catcode#1=\the\catcode#1\relax
  }%
  \catcode#1=#2\relax
}
\TMP@EnsureCode{33}{12}% !
\TMP@EnsureCode{36}{14}% $ (comment!)
\TMP@EnsureCode{38}{14}% & (comment!)
\TMP@EnsureCode{40}{12}% (
\TMP@EnsureCode{41}{12}% )
\TMP@EnsureCode{42}{12}% *
\TMP@EnsureCode{43}{12}% +
\TMP@EnsureCode{45}{12}% -
\TMP@EnsureCode{46}{12}% .
\TMP@EnsureCode{47}{12}% /
\TMP@EnsureCode{58}{11}% : (letter!)
\TMP@EnsureCode{60}{12}% <
\TMP@EnsureCode{62}{12}% >
\TMP@EnsureCode{63}{14}% ? (comment!)
\TMP@EnsureCode{91}{12}% [
\TMP@EnsureCode{93}{12}% ]
\edef\BIC@AtEnd{\BIC@AtEnd\noexpand\endinput}
\begingroup\expandafter\expandafter\expandafter\endgroup
\expandafter\ifx\csname BIC@TestMode\endcsname\relax
\else
  \catcode63=9 % ? (ignore)
\fi
? \let\BIC@@TestMode\BIC@TestMode
%    \end{macrocode}
%
% \subsection{\eTeX\ detection}
%
%    \begin{macrocode}
\begingroup\expandafter\expandafter\expandafter\endgroup
\expandafter\ifx\csname numexpr\endcsname\relax
  \catcode36=9 % $ (ignore)
\else
  \catcode38=9 % & (ignore)
\fi
%    \end{macrocode}
%
% \subsection{Help macros}
%
%    \begin{macro}{\BIC@Fi}
%    \begin{macrocode}
\let\BIC@Fi\fi
%    \end{macrocode}
%    \end{macro}
%    \begin{macro}{\BIC@AfterFi}
%    \begin{macrocode}
\def\BIC@AfterFi#1#2\BIC@Fi{\fi#1}%
%    \end{macrocode}
%    \end{macro}
%    \begin{macro}{\BIC@AfterFiFi}
%    \begin{macrocode}
\def\BIC@AfterFiFi#1#2\BIC@Fi{\fi\fi#1}%
%    \end{macrocode}
%    \end{macro}
%    \begin{macro}{\BIC@AfterFiFiFi}
%    \begin{macrocode}
\def\BIC@AfterFiFiFi#1#2\BIC@Fi{\fi\fi\fi#1}%
%    \end{macrocode}
%    \end{macro}
%
%    \begin{macro}{\BIC@Space}
%    \begin{macrocode}
\begingroup
  \def\x#1{\endgroup
    \let\BIC@Space= #1%
  }%
\x{ }
%    \end{macrocode}
%    \end{macro}
%
% \subsection{Expand number}
%
%    \begin{macrocode}
\begingroup\expandafter\expandafter\expandafter\endgroup
\expandafter\ifx\csname RequirePackage\endcsname\relax
  \def\TMP@RequirePackage#1[#2]{%
    \begingroup\expandafter\expandafter\expandafter\endgroup
    \expandafter\ifx\csname ver@#1.sty\endcsname\relax
      \input #1.sty\relax
    \fi
  }%
  \TMP@RequirePackage{pdftexcmds}[2007/11/11]%
\else
  \RequirePackage{pdftexcmds}[2007/11/11]%
\fi
%    \end{macrocode}
%
%    \begin{macrocode}
\begingroup\expandafter\expandafter\expandafter\endgroup
\expandafter\ifx\csname pdf@escapehex\endcsname\relax
%    \end{macrocode}
%
%    \begin{macro}{\BIC@Expand}
%    \begin{macrocode}
  \def\BIC@Expand#1{%
    \romannumeral0%
    \BIC@@Expand#1!\@nil{}%
  }%
%    \end{macrocode}
%    \end{macro}
%    \begin{macro}{\BIC@@Expand}
%    \begin{macrocode}
  \def\BIC@@Expand#1#2\@nil#3{%
    \expandafter\ifcat\noexpand#1\relax
      \expandafter\@firstoftwo
    \else
      \expandafter\@secondoftwo
    \fi
    {%
      \expandafter\BIC@@Expand#1#2\@nil{#3}%
    }{%
      \ifx#1!%
        \expandafter\@firstoftwo
      \else
        \expandafter\@secondoftwo
      \fi
      { #3}{%
        \BIC@@Expand#2\@nil{#3#1}%
      }%
    }%
  }%
%    \end{macrocode}
%    \end{macro}
%    \begin{macro}{\@firstoftwo}
%    \begin{macrocode}
  \expandafter\ifx\csname @firstoftwo\endcsname\relax
    \long\def\@firstoftwo#1#2{#1}%
  \fi
%    \end{macrocode}
%    \end{macro}
%    \begin{macro}{\@secondoftwo}
%    \begin{macrocode}
  \expandafter\ifx\csname @secondoftwo\endcsname\relax
    \long\def\@secondoftwo#1#2{#2}%
  \fi
%    \end{macrocode}
%    \end{macro}
%
%    \begin{macrocode}
\else
%    \end{macrocode}
%    \begin{macro}{\BIC@Expand}
%    \begin{macrocode}
  \def\BIC@Expand#1{%
    \romannumeral0\expandafter\expandafter\expandafter\BIC@Space
    \pdf@unescapehex{%
      \expandafter\expandafter\expandafter
      \BIC@StripHexSpace\pdf@escapehex{#1}20\@nil
    }%
  }%
%    \end{macrocode}
%    \end{macro}
%    \begin{macro}{\BIC@StripHexSpace}
%    \begin{macrocode}
  \def\BIC@StripHexSpace#120#2\@nil{%
    #1%
    \ifx\\#2\\%
    \else
      \BIC@AfterFi{%
        \BIC@StripHexSpace#2\@nil
      }%
    \BIC@Fi
  }%
%    \end{macrocode}
%    \end{macro}
%    \begin{macrocode}
\fi
%    \end{macrocode}
%
% \subsection{Normalize expanded number}
%
%    \begin{macro}{\BIC@Normalize}
%    |#1|: result sign\\
%    |#2|: first token of number
%    \begin{macrocode}
\def\BIC@Normalize#1#2{%
  \ifx#2-%
    \ifx\\#1\\%
      \BIC@AfterFiFi{%
        \BIC@Normalize-%
      }%
    \else
      \BIC@AfterFiFi{%
        \BIC@Normalize{}%
      }%
    \fi
  \else
    \ifx#2+%
      \BIC@AfterFiFi{%
        \BIC@Normalize{#1}%
      }%
    \else
      \ifx#20%
        \BIC@AfterFiFiFi{%
          \BIC@NormalizeZero{#1}%
        }%
      \else
        \BIC@AfterFiFiFi{%
          \BIC@NormalizeDigits#1#2%
        }%
      \fi
    \fi
  \BIC@Fi
}
%    \end{macrocode}
%    \end{macro}
%    \begin{macro}{\BIC@NormalizeZero}
%    \begin{macrocode}
\def\BIC@NormalizeZero#1#2{%
  \ifx#2!%
    \BIC@AfterFi{ 0}%
  \else
    \ifx#20%
      \BIC@AfterFiFi{%
        \BIC@NormalizeZero{#1}%
      }%
    \else
      \BIC@AfterFiFi{%
        \BIC@NormalizeDigits#1#2%
      }%
    \fi
  \BIC@Fi
}
%    \end{macrocode}
%    \end{macro}
%    \begin{macro}{\BIC@NormalizeDigits}
%    \begin{macrocode}
\def\BIC@NormalizeDigits#1!{ #1}
%    \end{macrocode}
%    \end{macro}
%
% \subsection{\op{Num}}
%
%    \begin{macro}{\bigintcalcNum}
%    \begin{macrocode}
\def\bigintcalcNum#1{%
  \romannumeral0%
  \expandafter\expandafter\expandafter\BIC@Normalize
  \expandafter\expandafter\expandafter{%
  \expandafter\expandafter\expandafter}%
  \BIC@Expand{#1}!%
}
%    \end{macrocode}
%    \end{macro}
%
% \subsection{\op{Inv}, \op{Abs}, \op{Sgn}}
%
%
%    \begin{macro}{\bigintcalcInv}
%    \begin{macrocode}
\def\bigintcalcInv#1{%
  \romannumeral0\expandafter\expandafter\expandafter\BIC@Space
  \bigintcalcNum{-#1}%
}
%    \end{macrocode}
%    \end{macro}
%
%    \begin{macro}{\bigintcalcAbs}
%    \begin{macrocode}
\def\bigintcalcAbs#1{%
  \romannumeral0%
  \expandafter\expandafter\expandafter\BIC@Abs
  \bigintcalcNum{#1}%
}
%    \end{macrocode}
%    \end{macro}
%    \begin{macro}{\BIC@Abs}
%    \begin{macrocode}
\def\BIC@Abs#1{%
  \ifx#1-%
    \expandafter\BIC@Space
  \else
    \expandafter\BIC@Space
    \expandafter#1%
  \fi
}
%    \end{macrocode}
%    \end{macro}
%
%    \begin{macro}{\bigintcalcSgn}
%    \begin{macrocode}
\def\bigintcalcSgn#1{%
  \number
  \expandafter\expandafter\expandafter\BIC@Sgn
  \bigintcalcNum{#1}! %
}
%    \end{macrocode}
%    \end{macro}
%    \begin{macro}{\BIC@Sgn}
%    \begin{macrocode}
\def\BIC@Sgn#1#2!{%
  \ifx#1-%
    -1%
  \else
    \ifx#10%
      0%
    \else
      1%
    \fi
  \fi
}
%    \end{macrocode}
%    \end{macro}
%
% \subsection{\op{Cmp}, \op{Min}, \op{Max}}
%
%    \begin{macro}{\bigintcalcCmp}
%    \begin{macrocode}
\def\bigintcalcCmp#1#2{%
  \number
  \expandafter\expandafter\expandafter\BIC@Cmp
  \bigintcalcNum{#2}!{#1}%
}
%    \end{macrocode}
%    \end{macro}
%    \begin{macro}{\BIC@Cmp}
%    \begin{macrocode}
\def\BIC@Cmp#1!#2{%
  \expandafter\expandafter\expandafter\BIC@@Cmp
  \bigintcalcNum{#2}!#1!%
}
%    \end{macrocode}
%    \end{macro}
%    \begin{macro}{\BIC@@Cmp}
%    \begin{macrocode}
\def\BIC@@Cmp#1#2!#3#4!{%
  \ifx#1-%
    \ifx#3-%
      \BIC@AfterFiFi{%
        \BIC@@Cmp#4!#2!%
      }%
    \else
      \BIC@AfterFiFi{%
        -1 %
      }%
    \fi
  \else
    \ifx#3-%
      \BIC@AfterFiFi{%
        1 %
      }%
    \else
      \BIC@AfterFiFi{%
        \BIC@CmpLength#1#2!#3#4!#1#2!#3#4!%
      }%
    \fi
  \BIC@Fi
}
%    \end{macrocode}
%    \end{macro}
%    \begin{macro}{\BIC@PosCmp}
%    \begin{macrocode}
\def\BIC@PosCmp#1!#2!{%
  \BIC@CmpLength#1!#2!#1!#2!%
}
%    \end{macrocode}
%    \end{macro}
%    \begin{macro}{\BIC@CmpLength}
%    \begin{macrocode}
\def\BIC@CmpLength#1#2!#3#4!{%
  \ifx\\#2\\%
    \ifx\\#4\\%
      \BIC@AfterFiFi\BIC@CmpDiff
    \else
      \BIC@AfterFiFi{%
        \BIC@CmpResult{-1}%
      }%
    \fi
  \else
    \ifx\\#4\\%
      \BIC@AfterFiFi{%
        \BIC@CmpResult1%
      }%
    \else
      \BIC@AfterFiFi{%
        \BIC@CmpLength#2!#4!%
      }%
    \fi
  \BIC@Fi
}
%    \end{macrocode}
%    \end{macro}
%    \begin{macro}{\BIC@CmpResult}
%    \begin{macrocode}
\def\BIC@CmpResult#1#2!#3!{#1 }
%    \end{macrocode}
%    \end{macro}
%    \begin{macro}{\BIC@CmpDiff}
%    \begin{macrocode}
\def\BIC@CmpDiff#1#2!#3#4!{%
  \ifnum#1<#3 %
    \BIC@AfterFi{%
      -1 %
    }%
  \else
    \ifnum#1>#3 %
      \BIC@AfterFiFi{%
        1 %
      }%
    \else
      \ifx\\#2\\%
        \BIC@AfterFiFiFi{%
          0 %
        }%
      \else
        \BIC@AfterFiFiFi{%
          \BIC@CmpDiff#2!#4!%
        }%
      \fi
    \fi
  \BIC@Fi
}
%    \end{macrocode}
%    \end{macro}
%
%    \begin{macro}{\bigintcalcMin}
%    \begin{macrocode}
\def\bigintcalcMin#1{%
  \romannumeral0%
  \expandafter\expandafter\expandafter\BIC@MinMax
  \bigintcalcNum{#1}!-!%
}
%    \end{macrocode}
%    \end{macro}
%    \begin{macro}{\bigintcalcMax}
%    \begin{macrocode}
\def\bigintcalcMax#1{%
  \romannumeral0%
  \expandafter\expandafter\expandafter\BIC@MinMax
  \bigintcalcNum{#1}!!%
}
%    \end{macrocode}
%    \end{macro}
%    \begin{macro}{\BIC@MinMax}
%    |#1|: $x$\\
%    |#2|: sign for comparison\\
%    |#3|: $y$
%    \begin{macrocode}
\def\BIC@MinMax#1!#2!#3{%
  \expandafter\expandafter\expandafter\BIC@@MinMax
  \bigintcalcNum{#3}!#1!#2!%
}
%    \end{macrocode}
%    \end{macro}
%    \begin{macro}{\BIC@@MinMax}
%    |#1|: $y$\\
%    |#2|: $x$\\
%    |#3|: sign for comparison
%    \begin{macrocode}
\def\BIC@@MinMax#1!#2!#3!{%
  \ifnum\BIC@@Cmp#1!#2!=#31 %
    \BIC@AfterFi{ #1}%
  \else
    \BIC@AfterFi{ #2}%
  \BIC@Fi
}
%    \end{macrocode}
%    \end{macro}
%
% \subsection{\op{Odd}}
%
%    \begin{macro}{\bigintcalcOdd}
%    \begin{macrocode}
\def\bigintcalcOdd#1{%
  \romannumeral0%
  \expandafter\expandafter\expandafter\BIC@Odd
  \bigintcalcAbs{#1}!%
}
%    \end{macrocode}
%    \end{macro}
%    \begin{macro}{\BigIntCalcOdd}
%    \begin{macrocode}
\def\BigIntCalcOdd#1!{%
  \romannumeral0%
  \BIC@Odd#1!%
}
%    \end{macrocode}
%    \end{macro}
%    \begin{macro}{\BIC@Odd}
%    |#1|: $x$
%    \begin{macrocode}
\def\BIC@Odd#1#2{%
  \ifx#2!%
    \ifodd#1 %
      \BIC@AfterFiFi{ 1}%
    \else
      \BIC@AfterFiFi{ 0}%
    \fi
  \else
    \expandafter\BIC@Odd\expandafter#2%
  \BIC@Fi
}
%    \end{macrocode}
%    \end{macro}
%
% \subsection{\op{Inc}, \op{Dec}}
%
%    \begin{macro}{\bigintcalcInc}
%    \begin{macrocode}
\def\bigintcalcInc#1{%
  \romannumeral0%
  \expandafter\expandafter\expandafter\BIC@IncSwitch
  \bigintcalcNum{#1}!%
}
%    \end{macrocode}
%    \end{macro}
%    \begin{macro}{\BIC@IncSwitch}
%    \begin{macrocode}
\def\BIC@IncSwitch#1#2!{%
  \ifcase\BIC@@Cmp#1#2!-1!%
    \BIC@AfterFi{ 0}%
  \or
    \BIC@AfterFi{%
      \BIC@Inc#1#2!{}%
    }%
  \else
    \BIC@AfterFi{%
      \expandafter-\romannumeral0%
      \BIC@Dec#2!{}%
    }%
  \BIC@Fi
}
%    \end{macrocode}
%    \end{macro}
%
%    \begin{macro}{\bigintcalcDec}
%    \begin{macrocode}
\def\bigintcalcDec#1{%
  \romannumeral0%
  \expandafter\expandafter\expandafter\BIC@DecSwitch
  \bigintcalcNum{#1}!%
}
%    \end{macrocode}
%    \end{macro}
%    \begin{macro}{\BIC@DecSwitch}
%    \begin{macrocode}
\def\BIC@DecSwitch#1#2!{%
  \ifcase\BIC@Sgn#1#2! %
    \BIC@AfterFi{ -1}%
  \or
    \BIC@AfterFi{%
      \BIC@Dec#1#2!{}%
    }%
  \else
    \BIC@AfterFi{%
      \expandafter-\romannumeral0%
      \BIC@Inc#2!{}%
    }%
  \BIC@Fi
}
%    \end{macrocode}
%    \end{macro}
%
%    \begin{macro}{\BigIntCalcInc}
%    \begin{macrocode}
\def\BigIntCalcInc#1!{%
  \romannumeral0\BIC@Inc#1!{}%
}
%    \end{macrocode}
%    \end{macro}
%    \begin{macro}{\BigIntCalcDec}
%    \begin{macrocode}
\def\BigIntCalcDec#1!{%
  \romannumeral0\BIC@Dec#1!{}%
}
%    \end{macrocode}
%    \end{macro}
%
%    \begin{macro}{\BIC@Inc}
%    \begin{macrocode}
\def\BIC@Inc#1#2!#3{%
  \ifx\\#2\\%
    \BIC@AfterFi{%
      \BIC@@Inc1#1#3!{}%
    }%
  \else
    \BIC@AfterFi{%
      \BIC@Inc#2!{#1#3}%
    }%
  \BIC@Fi
}
%    \end{macrocode}
%    \end{macro}
%    \begin{macro}{\BIC@@Inc}
%    \begin{macrocode}
\def\BIC@@Inc#1#2#3!#4{%
  \ifcase#1 %
    \ifx\\#3\\%
      \BIC@AfterFiFi{ #2#4}%
    \else
      \BIC@AfterFiFi{%
        \BIC@@Inc0#3!{#2#4}%
      }%
    \fi
  \else
    \ifnum#2<9 %
      \BIC@AfterFiFi{%
&       \expandafter\BIC@@@Inc\the\numexpr#2+1\relax
$       \expandafter\expandafter\expandafter\BIC@@@Inc
$       \ifcase#2 \expandafter1%
$       \or\expandafter2%
$       \or\expandafter3%
$       \or\expandafter4%
$       \or\expandafter5%
$       \or\expandafter6%
$       \or\expandafter7%
$       \or\expandafter8%
$       \or\expandafter9%
$?      \else\BigIntCalcError:ThisCannotHappen%
$       \fi
        0#3!{#4}%
      }%
    \else
      \BIC@AfterFiFi{%
        \BIC@@@Inc01#3!{#4}%
      }%
    \fi
  \BIC@Fi
}
%    \end{macrocode}
%    \end{macro}
%    \begin{macro}{\BIC@@@Inc}
%    \begin{macrocode}
\def\BIC@@@Inc#1#2#3!#4{%
  \ifx\\#3\\%
    \ifnum#2=1 %
      \BIC@AfterFiFi{ 1#1#4}%
    \else
      \BIC@AfterFiFi{ #1#4}%
    \fi
  \else
    \BIC@AfterFi{%
      \BIC@@Inc#2#3!{#1#4}%
    }%
  \BIC@Fi
}
%    \end{macrocode}
%    \end{macro}
%
%    \begin{macro}{\BIC@Dec}
%    \begin{macrocode}
\def\BIC@Dec#1#2!#3{%
  \ifx\\#2\\%
    \BIC@AfterFi{%
      \BIC@@Dec1#1#3!{}%
    }%
  \else
    \BIC@AfterFi{%
      \BIC@Dec#2!{#1#3}%
    }%
  \BIC@Fi
}
%    \end{macrocode}
%    \end{macro}
%    \begin{macro}{\BIC@@Dec}
%    \begin{macrocode}
\def\BIC@@Dec#1#2#3!#4{%
  \ifcase#1 %
    \ifx\\#3\\%
      \BIC@AfterFiFi{ #2#4}%
    \else
      \BIC@AfterFiFi{%
        \BIC@@Dec0#3!{#2#4}%
      }%
    \fi
  \else
    \ifnum#2>0 %
      \BIC@AfterFiFi{%
&       \expandafter\BIC@@@Dec\the\numexpr#2-1\relax
$       \expandafter\expandafter\expandafter\BIC@@@Dec
$       \ifcase#2
$?        \BigIntCalcError:ThisCannotHappen%
$       \or\expandafter0%
$       \or\expandafter1%
$       \or\expandafter2%
$       \or\expandafter3%
$       \or\expandafter4%
$       \or\expandafter5%
$       \or\expandafter6%
$       \or\expandafter7%
$       \or\expandafter8%
$?      \else\BigIntCalcError:ThisCannotHappen%
$       \fi
        0#3!{#4}%
      }%
    \else
      \BIC@AfterFiFi{%
        \BIC@@@Dec91#3!{#4}%
      }%
    \fi
  \BIC@Fi
}
%    \end{macrocode}
%    \end{macro}
%    \begin{macro}{\BIC@@@Dec}
%    \begin{macrocode}
\def\BIC@@@Dec#1#2#3!#4{%
  \ifx\\#3\\%
    \ifcase#1 %
      \ifx\\#4\\%
        \BIC@AfterFiFiFi{ 0}%
      \else
        \BIC@AfterFiFiFi{ #4}%
      \fi
    \else
      \BIC@AfterFiFi{ #1#4}%
    \fi
  \else
    \BIC@AfterFi{%
      \BIC@@Dec#2#3!{#1#4}%
    }%
  \BIC@Fi
}
%    \end{macrocode}
%    \end{macro}
%
% \subsection{\op{Add}, \op{Sub}}
%
%    \begin{macro}{\bigintcalcAdd}
%    \begin{macrocode}
\def\bigintcalcAdd#1{%
  \romannumeral0%
  \expandafter\expandafter\expandafter\BIC@Add
  \bigintcalcNum{#1}!%
}
%    \end{macrocode}
%    \end{macro}
%    \begin{macro}{\BIC@Add}
%    \begin{macrocode}
\def\BIC@Add#1!#2{%
  \expandafter\expandafter\expandafter
  \BIC@AddSwitch\bigintcalcNum{#2}!#1!%
}
%    \end{macrocode}
%    \end{macro}
%    \begin{macro}{\bigintcalcSub}
%    \begin{macrocode}
\def\bigintcalcSub#1#2{%
  \romannumeral0%
  \expandafter\expandafter\expandafter\BIC@Add
  \bigintcalcNum{-#2}!{#1}%
}
%    \end{macrocode}
%    \end{macro}
%
%    \begin{macro}{\BIC@AddSwitch}
%    Decision table for \cs{BIC@AddSwitch}.
%    \begin{quote}
%    \begin{tabular}[t]{@{}|l|l|l|l|l|@{}}
%      \hline
%      $x<0$ & $y<0$ & $-x>-y$ & $-$ & $\opAdd(-x,-y)$\\
%      \cline{3-3}\cline{5-5}
%            &       & else    &     & $\opAdd(-y,-x)$\\
%      \cline{2-5}
%            & else  & $-x> y$ & $-$ & $\opSub(-x, y)$\\
%      \cline{3-5}
%            &       & $-x= y$ &     & $0$\\
%      \cline{3-5}
%            &       & else    & $+$ & $\opSub( y,-x)$\\
%      \hline
%      else  & $y<0$ & $ x>-y$ & $+$ & $\opSub( x,-y)$\\
%      \cline{3-5}
%            &       & $ x=-y$ &     & $0$\\
%      \cline{3-5}
%            &       & else    & $-$ & $\opSub(-y, x)$\\
%      \cline{2-5}
%            & else  & $ x> y$ & $+$ & $\opAdd( x, y)$\\
%      \cline{3-3}\cline{5-5}
%            &       & else    &     & $\opAdd( y, x)$\\
%      \hline
%    \end{tabular}
%    \end{quote}
%    \begin{macrocode}
\def\BIC@AddSwitch#1#2!#3#4!{%
  \ifx#1-% x < 0
    \ifx#3-% y < 0
      \expandafter-\romannumeral0%
      \ifnum\BIC@PosCmp#2!#4!=1 % -x > -y
        \BIC@AfterFiFiFi{%
          \BIC@AddXY#2!#4!!!%
        }%
      \else % -x <= -y
        \BIC@AfterFiFiFi{%
          \BIC@AddXY#4!#2!!!%
        }%
      \fi
    \else % y >= 0
      \ifcase\BIC@PosCmp#2!#3#4!% -x = y
        \BIC@AfterFiFiFi{ 0}%
      \or % -x > y
        \expandafter-\romannumeral0%
        \BIC@AfterFiFiFi{%
          \BIC@SubXY#2!#3#4!!!%
        }%
      \else % -x <= y
        \BIC@AfterFiFiFi{%
          \BIC@SubXY#3#4!#2!!!%
        }%
      \fi
    \fi
  \else % x >= 0
    \ifx#3-% y < 0
      \ifcase\BIC@PosCmp#1#2!#4!% x = -y
        \BIC@AfterFiFiFi{ 0}%
      \or % x > -y
        \BIC@AfterFiFiFi{%
          \BIC@SubXY#1#2!#4!!!%
        }%
      \else % x <= -y
        \expandafter-\romannumeral0%
        \BIC@AfterFiFiFi{%
          \BIC@SubXY#4!#1#2!!!%
        }%
      \fi
    \else % y >= 0
      \ifnum\BIC@PosCmp#1#2!#3#4!=1 % x > y
        \BIC@AfterFiFiFi{%
          \BIC@AddXY#1#2!#3#4!!!%
        }%
      \else % x <= y
        \BIC@AfterFiFiFi{%
          \BIC@AddXY#3#4!#1#2!!!%
        }%
      \fi
    \fi
  \BIC@Fi
}
%    \end{macrocode}
%    \end{macro}
%
%    \begin{macro}{\BigIntCalcAdd}
%    \begin{macrocode}
\def\BigIntCalcAdd#1!#2!{%
  \romannumeral0\BIC@AddXY#1!#2!!!%
}
%    \end{macrocode}
%    \end{macro}
%    \begin{macro}{\BigIntCalcSub}
%    \begin{macrocode}
\def\BigIntCalcSub#1!#2!{%
  \romannumeral0\BIC@SubXY#1!#2!!!%
}
%    \end{macrocode}
%    \end{macro}
%
%    \begin{macro}{\BIC@AddXY}
%    \begin{macrocode}
\def\BIC@AddXY#1#2!#3#4!#5!#6!{%
  \ifx\\#2\\%
    \ifx\\#3\\%
      \BIC@AfterFiFi{%
        \BIC@DoAdd0!#1#5!#60!%
      }%
    \else
      \BIC@AfterFiFi{%
        \BIC@DoAdd0!#1#5!#3#6!%
      }%
    \fi
  \else
    \ifx\\#4\\%
      \ifx\\#3\\%
        \BIC@AfterFiFiFi{%
          \BIC@AddXY#2!{}!#1#5!#60!%
        }%
      \else
        \BIC@AfterFiFiFi{%
          \BIC@AddXY#2!{}!#1#5!#3#6!%
        }%
      \fi
    \else
      \BIC@AfterFiFi{%
        \BIC@AddXY#2!#4!#1#5!#3#6!%
      }%
    \fi
  \BIC@Fi
}
%    \end{macrocode}
%    \end{macro}
%    \begin{macro}{\BIC@DoAdd}
%    |#1|: carry\\
%    |#2|: reverted result\\
%    |#3#4|: reverted $x$\\
%    |#5#6|: reverted $y$
%    \begin{macrocode}
\def\BIC@DoAdd#1#2!#3#4!#5#6!{%
  \ifx\\#4\\%
    \BIC@AfterFi{%
&     \expandafter\BIC@Space
&     \the\numexpr#1+#3+#5\relax#2%
$     \expandafter\expandafter\expandafter\BIC@AddResult
$     \BIC@AddDigit#1#3#5#2%
    }%
  \else
    \BIC@AfterFi{%
      \expandafter\expandafter\expandafter\BIC@DoAdd
      \BIC@AddDigit#1#3#5#2!#4!#6!%
    }%
  \BIC@Fi
}
%    \end{macrocode}
%    \end{macro}
%    \begin{macro}{\BIC@AddResult}
%    \begin{macrocode}
$ \def\BIC@AddResult#1{%
$   \ifx#10%
$     \expandafter\BIC@Space
$   \else
$     \expandafter\BIC@Space\expandafter#1%
$   \fi
$ }%
%    \end{macrocode}
%    \end{macro}
%    \begin{macro}{\BIC@AddDigit}
%    |#1|: carry\\
%    |#2|: digit of $x$\\
%    |#3|: digit of $y$
%    \begin{macrocode}
\def\BIC@AddDigit#1#2#3{%
  \romannumeral0%
& \expandafter\BIC@@AddDigit\the\numexpr#1+#2+#3!%
$ \expandafter\BIC@@AddDigit\number%
$ \csname
$   BIC@AddCarry%
$   \ifcase#1 %
$     #2%
$   \else
$     \ifcase#2 1\or2\or3\or4\or5\or6\or7\or8\or9\or10\fi
$   \fi
$ \endcsname#3!%
}
%    \end{macrocode}
%    \end{macro}
%    \begin{macro}{\BIC@@AddDigit}
%    \begin{macrocode}
\def\BIC@@AddDigit#1!{%
  \ifnum#1<10 %
    \BIC@AfterFi{ 0#1}%
  \else
    \BIC@AfterFi{ #1}%
  \BIC@Fi
}
%    \end{macrocode}
%    \end{macro}
%    \begin{macro}{\BIC@AddCarry0}
%    \begin{macrocode}
$ \expandafter\def\csname BIC@AddCarry0\endcsname#1{#1}%
%    \end{macrocode}
%    \end{macro}
%    \begin{macro}{\BIC@AddCarry10}
%    \begin{macrocode}
$ \expandafter\def\csname BIC@AddCarry10\endcsname#1{1#1}%
%    \end{macrocode}
%    \end{macro}
%    \begin{macro}{\BIC@AddCarry[1-9]}
%    \begin{macrocode}
$ \def\BIC@Temp#1#2{%
$   \expandafter\def\csname BIC@AddCarry#1\endcsname##1{%
$     \ifcase##1 #1\or
$     #2%
$?    \else\BigIntCalcError:ThisCannotHappen%
$     \fi
$   }%
$ }%
$ \BIC@Temp 0{1\or2\or3\or4\or5\or6\or7\or8\or9}%
$ \BIC@Temp 1{2\or3\or4\or5\or6\or7\or8\or9\or10}%
$ \BIC@Temp 2{3\or4\or5\or6\or7\or8\or9\or10\or11}%
$ \BIC@Temp 3{4\or5\or6\or7\or8\or9\or10\or11\or12}%
$ \BIC@Temp 4{5\or6\or7\or8\or9\or10\or11\or12\or13}%
$ \BIC@Temp 5{6\or7\or8\or9\or10\or11\or12\or13\or14}%
$ \BIC@Temp 6{7\or8\or9\or10\or11\or12\or13\or14\or15}%
$ \BIC@Temp 7{8\or9\or10\or11\or12\or13\or14\or15\or16}%
$ \BIC@Temp 8{9\or10\or11\or12\or13\or14\or15\or16\or17}%
$ \BIC@Temp 9{10\or11\or12\or13\or14\or15\or16\or17\or18}%
%    \end{macrocode}
%    \end{macro}
%
%    \begin{macro}{\BIC@SubXY}
%    Preconditions:
%    \begin{itemize}
%    \item $x > y$, $x \geq 0$, and $y >= 0$
%    \item digits($x$) = digits($y$)
%    \end{itemize}
%    \begin{macrocode}
\def\BIC@SubXY#1#2!#3#4!#5!#6!{%
  \ifx\\#2\\%
    \ifx\\#3\\%
      \BIC@AfterFiFi{%
        \BIC@DoSub0!#1#5!#60!%
      }%
    \else
      \BIC@AfterFiFi{%
        \BIC@DoSub0!#1#5!#3#6!%
      }%
    \fi
  \else
    \ifx\\#4\\%
      \ifx\\#3\\%
        \BIC@AfterFiFiFi{%
          \BIC@SubXY#2!{}!#1#5!#60!%
        }%
      \else
        \BIC@AfterFiFiFi{%
          \BIC@SubXY#2!{}!#1#5!#3#6!%
        }%
      \fi
    \else
      \BIC@AfterFiFi{%
        \BIC@SubXY#2!#4!#1#5!#3#6!%
      }%
    \fi
  \BIC@Fi
}
%    \end{macrocode}
%    \end{macro}
%    \begin{macro}{\BIC@DoSub}
%    |#1|: carry\\
%    |#2|: reverted result\\
%    |#3#4|: reverted x\\
%    |#5#6|: reverted y
%    \begin{macrocode}
\def\BIC@DoSub#1#2!#3#4!#5#6!{%
  \ifx\\#4\\%
    \BIC@AfterFi{%
      \expandafter\expandafter\expandafter\BIC@SubResult
      \BIC@SubDigit#1#3#5#2%
    }%
  \else
    \BIC@AfterFi{%
      \expandafter\expandafter\expandafter\BIC@DoSub
      \BIC@SubDigit#1#3#5#2!#4!#6!%
    }%
  \BIC@Fi
}
%    \end{macrocode}
%    \end{macro}
%    \begin{macro}{\BIC@SubResult}
%    \begin{macrocode}
\def\BIC@SubResult#1{%
  \ifx#10%
    \expandafter\BIC@SubResult
  \else
    \expandafter\BIC@Space\expandafter#1%
  \fi
}
%    \end{macrocode}
%    \end{macro}
%    \begin{macro}{\BIC@SubDigit}
%    |#1|: carry\\
%    |#2|: digit of $x$\\
%    |#3|: digit of $y$
%    \begin{macrocode}
\def\BIC@SubDigit#1#2#3{%
  \romannumeral0%
& \expandafter\BIC@@SubDigit\the\numexpr#2-#3-#1!%
$ \expandafter\BIC@@AddDigit\number
$   \csname
$     BIC@SubCarry%
$     \ifcase#1 %
$       #3%
$     \else
$       \ifcase#3 1\or2\or3\or4\or5\or6\or7\or8\or9\or10\fi
$     \fi
$   \endcsname#2!%
}
%    \end{macrocode}
%    \end{macro}
%    \begin{macro}{\BIC@@SubDigit}
%    \begin{macrocode}
& \def\BIC@@SubDigit#1!{%
&   \ifnum#1<0 %
&     \BIC@AfterFi{%
&       \expandafter\BIC@Space
&       \expandafter1\the\numexpr#1+10\relax
&     }%
&   \else
&     \BIC@AfterFi{ 0#1}%
&   \BIC@Fi
& }%
%    \end{macrocode}
%    \end{macro}
%    \begin{macro}{\BIC@SubCarry0}
%    \begin{macrocode}
$ \expandafter\def\csname BIC@SubCarry0\endcsname#1{#1}%
%    \end{macrocode}
%    \end{macro}
%    \begin{macro}{\BIC@SubCarry10}%
%    \begin{macrocode}
$ \expandafter\def\csname BIC@SubCarry10\endcsname#1{1#1}%
%    \end{macrocode}
%    \end{macro}
%    \begin{macro}{\BIC@SubCarry[1-9]}
%    \begin{macrocode}
$ \def\BIC@Temp#1#2{%
$   \expandafter\def\csname BIC@SubCarry#1\endcsname##1{%
$     \ifcase##1 #2%
$?    \else\BigIntCalcError:ThisCannotHappen%
$     \fi
$   }%
$ }%
$ \BIC@Temp 1{19\or0\or1\or2\or3\or4\or5\or6\or7\or8}%
$ \BIC@Temp 2{18\or19\or0\or1\or2\or3\or4\or5\or6\or7}%
$ \BIC@Temp 3{17\or18\or19\or0\or1\or2\or3\or4\or5\or6}%
$ \BIC@Temp 4{16\or17\or18\or19\or0\or1\or2\or3\or4\or5}%
$ \BIC@Temp 5{15\or16\or17\or18\or19\or0\or1\or2\or3\or4}%
$ \BIC@Temp 6{14\or15\or16\or17\or18\or19\or0\or1\or2\or3}%
$ \BIC@Temp 7{13\or14\or15\or16\or17\or18\or19\or0\or1\or2}%
$ \BIC@Temp 8{12\or13\or14\or15\or16\or17\or18\or19\or0\or1}%
$ \BIC@Temp 9{11\or12\or13\or14\or15\or16\or17\or18\or19\or0}%
%    \end{macrocode}
%    \end{macro}
%
% \subsection{\op{Shl}, \op{Shr}}
%
%    \begin{macro}{\bigintcalcShl}
%    \begin{macrocode}
\def\bigintcalcShl#1{%
  \romannumeral0%
  \expandafter\expandafter\expandafter\BIC@Shl
  \bigintcalcNum{#1}!%
}
%    \end{macrocode}
%    \end{macro}
%    \begin{macro}{\BIC@Shl}
%    \begin{macrocode}
\def\BIC@Shl#1#2!{%
  \ifx#1-%
    \BIC@AfterFi{%
      \expandafter-\romannumeral0%
&     \BIC@@Shl#2!!%
$     \BIC@AddXY#2!#2!!!%
    }%
  \else
    \BIC@AfterFi{%
&     \BIC@@Shl#1#2!!%
$     \BIC@AddXY#1#2!#1#2!!!%
    }%
  \BIC@Fi
}
%    \end{macrocode}
%    \end{macro}
%    \begin{macro}{\BigIntCalcShl}
%    \begin{macrocode}
\def\BigIntCalcShl#1!{%
  \romannumeral0%
& \BIC@@Shl#1!!%
$ \BIC@AddXY#1!#1!!!%
}
%    \end{macrocode}
%    \end{macro}
%    \begin{macro}{\BIC@@Shl}
%    \begin{macrocode}
& \def\BIC@@Shl#1#2!{%
&   \ifx\\#2\\%
&     \BIC@AfterFi{%
&       \BIC@@@Shl0!#1%
&     }%
&   \else
&     \BIC@AfterFi{%
&       \BIC@@Shl#2!#1%
&     }%
&   \BIC@Fi
& }%
%    \end{macrocode}
%    \end{macro}
%    \begin{macro}{\BIC@@@Shl}
%    |#1|: carry\\
%    |#2|: result\\
%    |#3#4|: reverted number
%    \begin{macrocode}
& \def\BIC@@@Shl#1#2!#3#4!{%
&   \ifx\\#4\\%
&     \BIC@AfterFi{%
&       \expandafter\BIC@Space
&       \the\numexpr#3*2+#1\relax#2%
&     }%
&   \else
&     \BIC@AfterFi{%
&       \expandafter\BIC@@@@Shl\the\numexpr#3*2+#1!#2!#4!%
&     }%
&   \BIC@Fi
& }%
%    \end{macrocode}
%    \end{macro}
%    \begin{macro}{\BIC@@@@Shl}
%    \begin{macrocode}
& \def\BIC@@@@Shl#1!{%
&   \ifnum#1<10 %
&     \BIC@AfterFi{%
&       \BIC@@@Shl0#1%
&     }%
&   \else
&     \BIC@AfterFi{%
&       \BIC@@@Shl#1%
&     }%
&   \BIC@Fi
& }%
%    \end{macrocode}
%    \end{macro}
%
%    \begin{macro}{\bigintcalcShr}
%    \begin{macrocode}
\def\bigintcalcShr#1{%
  \romannumeral0%
  \expandafter\expandafter\expandafter\BIC@Shr
  \bigintcalcNum{#1}!%
}
%    \end{macrocode}
%    \end{macro}
%    \begin{macro}{\BIC@Shr}
%    \begin{macrocode}
\def\BIC@Shr#1#2!{%
  \ifx#1-%
    \expandafter-\romannumeral0%
    \BIC@AfterFi{%
      \BIC@@Shr#2!%
    }%
  \else
    \BIC@AfterFi{%
      \BIC@@Shr#1#2!%
    }%
  \BIC@Fi
}
%    \end{macrocode}
%    \end{macro}
%    \begin{macro}{\BigIntCalcShr}
%    \begin{macrocode}
\def\BigIntCalcShr#1!{%
  \romannumeral0%
  \BIC@@Shr#1!%
}
%    \end{macrocode}
%    \end{macro}
%    \begin{macro}{\BIC@@Shr}
%    \begin{macrocode}
\def\BIC@@Shr#1#2!{%
  \ifcase#1 %
    \BIC@AfterFi{ 0}%
  \or
    \ifx\\#2\\%
      \BIC@AfterFiFi{ 0}%
    \else
      \BIC@AfterFiFi{%
        \BIC@@@Shr#1#2!!%
      }%
    \fi
  \else
    \BIC@AfterFi{%
      \BIC@@@Shr0#1#2!!%
    }%
  \BIC@Fi
}
%    \end{macrocode}
%    \end{macro}
%    \begin{macro}{\BIC@@@Shr}
%    |#1|: carry\\
%    |#2#3|: number\\
%    |#4|: result
%    \begin{macrocode}
\def\BIC@@@Shr#1#2#3!#4!{%
  \ifx\\#3\\%
    \ifodd#1#2 %
      \BIC@AfterFiFi{%
&       \expandafter\BIC@ShrResult\the\numexpr(#1#2-1)/2\relax
$       \expandafter\expandafter\expandafter\BIC@ShrResult
$       \csname BIC@ShrDigit#1#2\endcsname
        #4!%
      }%
    \else
      \BIC@AfterFiFi{%
&       \expandafter\BIC@ShrResult\the\numexpr#1#2/2\relax
$       \expandafter\expandafter\expandafter\BIC@ShrResult
$       \csname BIC@ShrDigit#1#2\endcsname
        #4!%
      }%
    \fi
  \else
    \ifodd#1#2 %
      \BIC@AfterFiFi{%
&       \expandafter\BIC@@@@Shr\the\numexpr(#1#2-1)/2\relax1%
$       \expandafter\expandafter\expandafter\BIC@@@@Shr
$       \csname BIC@ShrDigit#1#2\endcsname
        #3!#4!%
      }%
    \else
      \BIC@AfterFiFi{%
&       \expandafter\BIC@@@@Shr\the\numexpr#1#2/2\relax0%
$       \expandafter\expandafter\expandafter\BIC@@@@Shr
$       \csname BIC@ShrDigit#1#2\endcsname
        #3!#4!%
      }%
    \fi
  \BIC@Fi
}
%    \end{macrocode}
%    \end{macro}
%    \begin{macro}{\BIC@ShrResult}
%    \begin{macrocode}
& \def\BIC@ShrResult#1#2!{ #2#1}%
$ \def\BIC@ShrResult#1#2#3!{ #3#1}%
%    \end{macrocode}
%    \end{macro}
%    \begin{macro}{\BIC@@@@Shr}
%    |#1|: new digit\\
%    |#2|: carry\\
%    |#3|: remaining number\\
%    |#4|: result
%    \begin{macrocode}
\def\BIC@@@@Shr#1#2#3!#4!{%
  \BIC@@@Shr#2#3!#4#1!%
}
%    \end{macrocode}
%    \end{macro}
%    \begin{macro}{\BIC@ShrDigit[00-19]}
%    \begin{macrocode}
$ \def\BIC@Temp#1#2#3#4{%
$   \expandafter\def\csname BIC@ShrDigit#1#2\endcsname{#3#4}%
$ }%
$ \BIC@Temp 0000%
$ \BIC@Temp 0101%
$ \BIC@Temp 0210%
$ \BIC@Temp 0311%
$ \BIC@Temp 0420%
$ \BIC@Temp 0521%
$ \BIC@Temp 0630%
$ \BIC@Temp 0731%
$ \BIC@Temp 0840%
$ \BIC@Temp 0941%
$ \BIC@Temp 1050%
$ \BIC@Temp 1151%
$ \BIC@Temp 1260%
$ \BIC@Temp 1361%
$ \BIC@Temp 1470%
$ \BIC@Temp 1571%
$ \BIC@Temp 1680%
$ \BIC@Temp 1781%
$ \BIC@Temp 1890%
$ \BIC@Temp 1991%
%    \end{macrocode}
%    \end{macro}
%
% \subsection{\cs{BIC@Tim}}
%
%    \begin{macro}{\BIC@Tim}
%    Macro \cs{BIC@Tim} implements ``Number \emph{tim}es digit''.\\
%    |#1|: plain number without sign\\
%    |#2|: digit
\def\BIC@Tim#1!#2{%
  \romannumeral0%
  \ifcase#2 % 0
    \BIC@AfterFi{ 0}%
  \or % 1
    \BIC@AfterFi{ #1}%
  \or % 2
    \BIC@AfterFi{%
      \BIC@Shl#1!%
    }%
  \else % 3-9
    \BIC@AfterFi{%
      \BIC@@Tim#1!!#2%
    }%
  \BIC@Fi
}
%    \begin{macrocode}
%    \end{macrocode}
%    \end{macro}
%    \begin{macro}{\BIC@@Tim}
%    |#1#2|: number\\
%    |#3|: reverted number
%    \begin{macrocode}
\def\BIC@@Tim#1#2!{%
  \ifx\\#2\\%
    \BIC@AfterFi{%
      \BIC@ProcessTim0!#1%
    }%
  \else
    \BIC@AfterFi{%
      \BIC@@Tim#2!#1%
    }%
  \BIC@Fi
}
%    \end{macrocode}
%    \end{macro}
%    \begin{macro}{\BIC@ProcessTim}
%    |#1|: carry\\
%    |#2|: result\\
%    |#3#4|: reverted number\\
%    |#5|: digit
%    \begin{macrocode}
\def\BIC@ProcessTim#1#2!#3#4!#5{%
  \ifx\\#4\\%
    \BIC@AfterFi{%
      \expandafter\BIC@Space
&     \the\numexpr#3*#5+#1\relax
$     \romannumeral0\BIC@TimDigit#3#5#1%
      #2%
    }%
  \else
    \BIC@AfterFi{%
      \expandafter\BIC@@ProcessTim
&     \the\numexpr#3*#5+#1%
$     \romannumeral0\BIC@TimDigit#3#5#1%
      !#2!#4!#5%
    }%
  \BIC@Fi
}
%    \end{macrocode}
%    \end{macro}
%    \begin{macro}{\BIC@@ProcessTim}
%    |#1#2|: carry?, new digit\\
%    |#3|: new number\\
%    |#4|: old number\\
%    |#5|: digit
%    \begin{macrocode}
\def\BIC@@ProcessTim#1#2!{%
  \ifx\\#2\\%
    \BIC@AfterFi{%
      \BIC@ProcessTim0#1%
    }%
  \else
    \BIC@AfterFi{%
      \BIC@ProcessTim#1#2%
    }%
  \BIC@Fi
}
%    \end{macrocode}
%    \end{macro}
%    \begin{macro}{\BIC@TimDigit}
%    |#1|: digit 0--9\\
%    |#2|: digit 3--9\\
%    |#3|: carry 0--9
%    \begin{macrocode}
$ \def\BIC@TimDigit#1#2#3{%
$   \ifcase#1 % 0
$     \BIC@AfterFi{ #3}%
$   \or % 1
$     \BIC@AfterFi{%
$       \expandafter\BIC@Space
$       \number\csname BIC@AddCarry#2\endcsname#3 %
$     }%
$   \else
$     \ifcase#3 %
$       \BIC@AfterFiFi{%
$         \expandafter\BIC@Space
$         \number\csname BIC@MulDigit#2\endcsname#1 %
$       }%
$     \else
$       \BIC@AfterFiFi{%
$         \expandafter\BIC@Space
$         \romannumeral0%
$         \expandafter\BIC@AddXY
$         \number\csname BIC@MulDigit#2\endcsname#1!%
$         #3!!!%
$       }%
$     \fi
$   \BIC@Fi
$ }%
%    \end{macrocode}
%    \end{macro}
%    \begin{macro}{\BIC@MulDigit[3-9]}
%    \begin{macrocode}
$ \def\BIC@Temp#1#2{%
$   \expandafter\def\csname BIC@MulDigit#1\endcsname##1{%
$     \ifcase##1 0%
$     \or ##1%
$     \or #2%
$?    \else\BigIntCalcError:ThisCannotHappen%
$     \fi
$   }%
$ }%
$ \BIC@Temp 3{6\or9\or12\or15\or18\or21\or24\or27}%
$ \BIC@Temp 4{8\or12\or16\or20\or24\or28\or32\or36}%
$ \BIC@Temp 5{10\or15\or20\or25\or30\or35\or40\or45}%
$ \BIC@Temp 6{12\or18\or24\or30\or36\or42\or48\or54}%
$ \BIC@Temp 7{14\or21\or28\or35\or42\or49\or56\or63}%
$ \BIC@Temp 8{16\or24\or32\or40\or48\or56\or64\or72}%
$ \BIC@Temp 9{18\or27\or36\or45\or54\or63\or72\or81}%
%    \end{macrocode}
%    \end{macro}
%
% \subsection{\op{Mul}}
%
%    \begin{macro}{\bigintcalcMul}
%    \begin{macrocode}
\def\bigintcalcMul#1#2{%
  \romannumeral0%
  \expandafter\expandafter\expandafter\BIC@Mul
  \bigintcalcNum{#1}!{#2}%
}
%    \end{macrocode}
%    \end{macro}
%    \begin{macro}{\BIC@Mul}
%    \begin{macrocode}
\def\BIC@Mul#1!#2{%
  \expandafter\expandafter\expandafter\BIC@MulSwitch
  \bigintcalcNum{#2}!#1!%
}
%    \end{macrocode}
%    \end{macro}
%    \begin{macro}{\BIC@MulSwitch}
%    Decision table for \cs{BIC@MulSwitch}.
%    \begin{quote}
%    \begin{tabular}[t]{@{}|l|l|l|l|l|@{}}
%      \hline
%      $x=0$ &\multicolumn{3}{l}{}    &$0$\\
%      \hline
%      $x>0$ & $y=0$ &\multicolumn2l{}&$0$\\
%      \cline{2-5}
%            & $y>0$ & $ x> y$ & $+$ & $\opMul( x, y)$\\
%      \cline{3-3}\cline{5-5}
%            &       & else    &     & $\opMul( y, x)$\\
%      \cline{2-5}
%            & $y<0$ & $ x>-y$ & $-$ & $\opMul( x,-y)$\\
%      \cline{3-3}\cline{5-5}
%            &       & else    &     & $\opMul(-y, x)$\\
%      \hline
%      $x<0$ & $y=0$ &\multicolumn2l{}&$0$\\
%      \cline{2-5}
%            & $y>0$ & $-x> y$ & $-$ & $\opMul(-x, y)$\\
%      \cline{3-3}\cline{5-5}
%            &       & else    &     & $\opMul( y,-x)$\\
%      \cline{2-5}
%            & $y<0$ & $-x>-y$ & $+$ & $\opMul(-x,-y)$\\
%      \cline{3-3}\cline{5-5}
%            &       & else    &     & $\opMul(-y,-x)$\\
%      \hline
%    \end{tabular}
%    \end{quote}
%    \begin{macrocode}
\def\BIC@MulSwitch#1#2!#3#4!{%
  \ifcase\BIC@Sgn#1#2! % x = 0
    \BIC@AfterFi{ 0}%
  \or % x > 0
    \ifcase\BIC@Sgn#3#4! % y = 0
      \BIC@AfterFiFi{ 0}%
    \or % y > 0
      \ifnum\BIC@PosCmp#1#2!#3#4!=1 % x > y
        \BIC@AfterFiFiFi{%
          \BIC@ProcessMul0!#1#2!#3#4!%
        }%
      \else % x <= y
        \BIC@AfterFiFiFi{%
          \BIC@ProcessMul0!#3#4!#1#2!%
        }%
      \fi
    \else % y < 0
      \expandafter-\romannumeral0%
      \ifnum\BIC@PosCmp#1#2!#4!=1 % x > -y
        \BIC@AfterFiFiFi{%
          \BIC@ProcessMul0!#1#2!#4!%
        }%
      \else % x <= -y
        \BIC@AfterFiFiFi{%
          \BIC@ProcessMul0!#4!#1#2!%
        }%
      \fi
    \fi
  \else % x < 0
    \ifcase\BIC@Sgn#3#4! % y = 0
      \BIC@AfterFiFi{ 0}%
    \or % y > 0
      \expandafter-\romannumeral0%
      \ifnum\BIC@PosCmp#2!#3#4!=1 % -x > y
        \BIC@AfterFiFiFi{%
          \BIC@ProcessMul0!#2!#3#4!%
        }%
      \else % -x <= y
        \BIC@AfterFiFiFi{%
          \BIC@ProcessMul0!#3#4!#2!%
        }%
      \fi
    \else % y < 0
      \ifnum\BIC@PosCmp#2!#4!=1 % -x > -y
        \BIC@AfterFiFiFi{%
          \BIC@ProcessMul0!#2!#4!%
        }%
      \else % -x <= -y
        \BIC@AfterFiFiFi{%
          \BIC@ProcessMul0!#4!#2!%
        }%
      \fi
    \fi
  \BIC@Fi
}
%    \end{macrocode}
%    \end{macro}
%    \begin{macro}{\BigIntCalcMul}
%    \begin{macrocode}
\def\BigIntCalcMul#1!#2!{%
  \romannumeral0%
  \BIC@ProcessMul0!#1!#2!%
}
%    \end{macrocode}
%    \end{macro}
%    \begin{macro}{\BIC@ProcessMul}
%    |#1|: result\\
%    |#2|: number $x$\\
%    |#3#4|: number $y$
%    \begin{macrocode}
\def\BIC@ProcessMul#1!#2!#3#4!{%
  \ifx\\#4\\%
    \BIC@AfterFi{%
      \expandafter\expandafter\expandafter\BIC@Space
      \bigintcalcAdd{\BIC@Tim#2!#3}{#10}%
    }%
  \else
    \BIC@AfterFi{%
      \expandafter\expandafter\expandafter\BIC@ProcessMul
      \bigintcalcAdd{\BIC@Tim#2!#3}{#10}!#2!#4!%
    }%
  \BIC@Fi
}
%    \end{macrocode}
%    \end{macro}
%
% \subsection{\op{Sqr}}
%
%    \begin{macro}{\bigintcalcSqr}
%    \begin{macrocode}
\def\bigintcalcSqr#1{%
  \romannumeral0%
  \expandafter\expandafter\expandafter\BIC@Sqr
  \bigintcalcNum{#1}!%
}
%    \end{macrocode}
%    \end{macro}
%    \begin{macro}{\BIC@Sqr}
%    \begin{macrocode}
\def\BIC@Sqr#1{%
   \ifx#1-%
     \expandafter\BIC@@Sqr
   \else
     \expandafter\BIC@@Sqr\expandafter#1%
   \fi
}
%    \end{macrocode}
%    \end{macro}
%    \begin{macro}{\BIC@@Sqr}
%    \begin{macrocode}
\def\BIC@@Sqr#1!{%
  \BIC@ProcessMul0!#1!#1!%
}
%    \end{macrocode}
%    \end{macro}
%
% \subsection{\op{Fac}}
%
%    \begin{macro}{\bigintcalcFac}
%    \begin{macrocode}
\def\bigintcalcFac#1{%
  \romannumeral0%
  \expandafter\expandafter\expandafter\BIC@Fac
  \bigintcalcNum{#1}!%
}
%    \end{macrocode}
%    \end{macro}
%    \begin{macro}{\BIC@Fac}
%    \begin{macrocode}
\def\BIC@Fac#1#2!{%
  \ifx#1-%
    \BIC@AfterFi{ 0\BigIntCalcError:FacNegative}%
  \else
    \ifnum\BIC@PosCmp#1#2!13!<0 %
      \ifcase#1#2 %
         \BIC@AfterFiFiFi{ 1}% 0!
      \or\BIC@AfterFiFiFi{ 1}% 1!
      \or\BIC@AfterFiFiFi{ 2}% 2!
      \or\BIC@AfterFiFiFi{ 6}% 3!
      \or\BIC@AfterFiFiFi{ 24}% 4!
      \or\BIC@AfterFiFiFi{ 120}% 5!
      \or\BIC@AfterFiFiFi{ 720}% 6!
      \or\BIC@AfterFiFiFi{ 5040}% 7!
      \or\BIC@AfterFiFiFi{ 40320}% 8!
      \or\BIC@AfterFiFiFi{ 362880}% 9!
      \or\BIC@AfterFiFiFi{ 3628800}% 10!
      \or\BIC@AfterFiFiFi{ 39916800}% 11!
      \or\BIC@AfterFiFiFi{ 479001600}% 12!
?     \else\BigIntCalcError:ThisCannotHappen%
      \fi
    \else
      \BIC@AfterFiFi{%
        \BIC@ProcessFac#1#2!479001600!%
      }%
    \fi
  \BIC@Fi
}
%    \end{macrocode}
%    \end{macro}
%    \begin{macro}{\BIC@ProcessFac}
%    |#1|: $n$\\
%    |#2|: result
%    \begin{macrocode}
\def\BIC@ProcessFac#1!#2!{%
  \ifnum\BIC@PosCmp#1!12!=0 %
    \BIC@AfterFi{ #2}%
  \else
    \BIC@AfterFi{%
      \expandafter\BIC@@ProcessFac
      \romannumeral0\BIC@ProcessMul0!#2!#1!%
      !#1!%
    }%
  \BIC@Fi
}
%    \end{macrocode}
%    \end{macro}
%    \begin{macro}{\BIC@@ProcessFac}
%    |#1|: result\\
%    |#2|: $n$
%    \begin{macrocode}
\def\BIC@@ProcessFac#1!#2!{%
  \expandafter\BIC@ProcessFac
  \romannumeral0\BIC@Dec#2!{}%
  !#1!%
}
%    \end{macrocode}
%    \end{macro}
%
% \subsection{\op{Pow}}
%
%    \begin{macro}{\bigintcalcPow}
%    |#1|: basis\\
%    |#2|: power
%    \begin{macrocode}
\def\bigintcalcPow#1{%
  \romannumeral0%
  \expandafter\expandafter\expandafter\BIC@Pow
  \bigintcalcNum{#1}!%
}
%    \end{macrocode}
%    \end{macro}
%    \begin{macro}{\BIC@Pow}
%    |#1|: basis\\
%    |#2|: power
%    \begin{macrocode}
\def\BIC@Pow#1!#2{%
  \expandafter\expandafter\expandafter\BIC@PowSwitch
  \bigintcalcNum{#2}!#1!%
}
%    \end{macrocode}
%    \end{macro}
%    \begin{macro}{\BIC@PowSwitch}
%    |#1#2|: power $y$\\
%    |#3#4|: basis $x$\\
%    Decision table for \cs{BIC@PowSwitch}.
%    \begin{quote}
%    \def\M#1{\multicolumn{#1}{l}{}}%
%    \begin{tabular}[t]{@{}|l|l|l|l|@{}}
%    \hline
%    $y=0$ & \M{2}                        & $1$\\
%    \hline
%    $y=1$ & \M{2}                        & $x$\\
%    \hline
%    $y=2$ & $x<0$          & \M{1}       & $\opMul(-x,-x)$\\
%    \cline{2-4}
%          & else           & \M{1}       & $\opMul(x,x)$\\
%    \hline
%    $y<0$ & $x=0$          & \M{1}       & DivisionByZero\\
%    \cline{2-4}
%          & $x=1$          & \M{1}       & $1$\\
%    \cline{2-4}
%          & $x=-1$         & $\opODD(y)$ & $-1$\\
%    \cline{3-4}
%          &                & else        & $1$\\
%    \cline{2-4}
%          & else ($\string|x\string|>1$) & \M{1}       & $0$\\
%    \hline
%    $y>2$ & $x=0$          & \M{1}       & $0$\\
%    \cline{2-4}
%          & $x=1$          & \M{1}       & $1$\\
%    \cline{2-4}
%          & $x=-1$         & $\opODD(y)$ & $-1$\\
%    \cline{3-4}
%          &                & else        & $1$\\
%    \cline{2-4}
%          & $x<-1$ ($x<0$) & $\opODD(y)$ & $-\opPow(-x,y)$\\
%    \cline{3-4}
%          &                & else        & $\opPow(-x,y)$\\
%    \cline{2-4}
%          & else ($x>1$)   & \M{1}       & $\opPow(x,y)$\\
%    \hline
%    \end{tabular}
%    \end{quote}
%    \begin{macrocode}
\def\BIC@PowSwitch#1#2!#3#4!{%
  \ifcase\ifx\\#2\\%
           \ifx#100 % y = 0
           \else\ifx#111 % y = 1
           \else\ifx#122 % y = 2
           \else4 % y > 2
           \fi\fi\fi
         \else
           \ifx#1-3 % y < 0
           \else4 % y > 2
           \fi
         \fi
    \BIC@AfterFi{ 1}% y = 0
  \or % y = 1
    \BIC@AfterFi{ #3#4}%
  \or % y = 2
    \ifx#3-% x < 0
      \BIC@AfterFiFi{%
        \BIC@ProcessMul0!#4!#4!%
      }%
    \else % x >= 0
      \BIC@AfterFiFi{%
        \BIC@ProcessMul0!#3#4!#3#4!%
      }%
    \fi
  \or % y < 0
    \ifcase\ifx\\#4\\%
             \ifx#300 % x = 0
             \else\ifx#311 % x = 1
             \else3 % x > 1
             \fi\fi
           \else
             \ifcase\BIC@MinusOne#3#4! %
               3 % |x| > 1
             \or
               2 % x = -1
?            \else\BigIntCalcError:ThisCannotHappen%
             \fi
           \fi
      \BIC@AfterFiFi{ 0\BigIntCalcError:DivisionByZero}% x = 0
    \or % x = 1
      \BIC@AfterFiFi{ 1}% x = 1
    \or % x = -1
      \ifcase\BIC@ModTwo#2! % even(y)
        \BIC@AfterFiFiFi{ 1}%
      \or % odd(y)
        \BIC@AfterFiFiFi{ -1}%
?     \else\BigIntCalcError:ThisCannotHappen%
      \fi
    \or % |x| > 1
      \BIC@AfterFiFi{ 0}%
?   \else\BigIntCalcError:ThisCannotHappen%
    \fi
  \or % y > 2
    \ifcase\ifx\\#4\\%
             \ifx#300 % x = 0
             \else\ifx#311 % x = 1
             \else4 % x > 1
             \fi\fi
           \else
             \ifx#3-%
               \ifcase\BIC@MinusOne#3#4! %
                 3 % x < -1
               \else
                 2 % x = -1
               \fi
             \else
               4 % x > 1
             \fi
           \fi
      \BIC@AfterFiFi{ 0}% x = 0
    \or % x = 1
      \BIC@AfterFiFi{ 1}% x = 1
    \or % x = -1
      \ifcase\BIC@ModTwo#1#2! % even(y)
        \BIC@AfterFiFiFi{ 1}%
      \or % odd(y)
        \BIC@AfterFiFiFi{ -1}%
?     \else\BigIntCalcError:ThisCannotHappen%
      \fi
    \or % x < -1
      \ifcase\BIC@ModTwo#1#2! % even(y)
        \BIC@AfterFiFiFi{%
          \BIC@PowRec#4!#1#2!1!%
        }%
      \or % odd(y)
        \expandafter-\romannumeral0%
        \BIC@AfterFiFiFi{%
          \BIC@PowRec#4!#1#2!1!%
        }%
?     \else\BigIntCalcError:ThisCannotHappen%
      \fi
    \or % x > 1
      \BIC@AfterFiFi{%
        \BIC@PowRec#3#4!#1#2!1!%
      }%
?   \else\BigIntCalcError:ThisCannotHappen%
    \fi
? \else\BigIntCalcError:ThisCannotHappen%
  \BIC@Fi
}
%    \end{macrocode}
%    \end{macro}
%
% \subsubsection{Help macros}
%
%    \begin{macro}{\BIC@ModTwo}
%    Macro \cs{BIC@ModTwo} expects a number without sign
%    and returns digit |1| or |0| if the number is odd or even.
%    \begin{macrocode}
\def\BIC@ModTwo#1#2!{%
  \ifx\\#2\\%
    \ifodd#1 %
      \BIC@AfterFiFi1%
    \else
      \BIC@AfterFiFi0%
    \fi
  \else
    \BIC@AfterFi{%
      \BIC@ModTwo#2!%
    }%
  \BIC@Fi
}
%    \end{macrocode}
%    \end{macro}
%
%    \begin{macro}{\BIC@MinusOne}
%    Macro \cs{BIC@MinusOne} expects a number and returns
%    digit |1| if the number equals minus one and returns |0| otherwise.
%    \begin{macrocode}
\def\BIC@MinusOne#1#2!{%
  \ifx#1-%
    \BIC@@MinusOne#2!%
  \else
    0%
  \fi
}
%    \end{macrocode}
%    \end{macro}
%    \begin{macro}{\BIC@@MinusOne}
%    \begin{macrocode}
\def\BIC@@MinusOne#1#2!{%
  \ifx#11%
    \ifx\\#2\\%
      1%
    \else
      0%
    \fi
  \else
    0%
  \fi
}
%    \end{macrocode}
%    \end{macro}
%
% \subsubsection{Recursive calculation}
%
%    \begin{macro}{\BIC@PowRec}
%\begin{quote}
%\begin{verbatim}
%Pow(x, y) {
%  PowRec(x, y, 1)
%}
%PowRec(x, y, r) {
%  if y == 1 then
%    return r
%  else
%    ifodd y then
%      return PowRec(x*x, y div 2, r*x) % y div 2 = (y-1)/2
%    else
%      return PowRec(x*x, y div 2, r)
%    fi
%  fi
%}
%\end{verbatim}
%\end{quote}
%    |#1|: $x$ (basis)\\
%    |#2#3|: $y$ (power)\\
%    |#4|: $r$ (result)
%    \begin{macrocode}
\def\BIC@PowRec#1!#2#3!#4!{%
  \ifcase\ifx#21\ifx\\#3\\0 \else1 \fi\else1 \fi % y = 1
    \ifnum\BIC@PosCmp#1!#4!=1 % x > r
      \BIC@AfterFiFi{%
        \BIC@ProcessMul0!#1!#4!%
      }%
    \else
      \BIC@AfterFiFi{%
        \BIC@ProcessMul0!#4!#1!%
      }%
    \fi
  \or
    \ifcase\BIC@ModTwo#2#3! % even(y)
      \BIC@AfterFiFi{%
        \expandafter\BIC@@PowRec\romannumeral0%
        \BIC@@Shr#2#3!%
        !#1!#4!%
      }%
    \or % odd(y)
      \ifnum\BIC@PosCmp#1!#4!=1 % x > r
        \BIC@AfterFiFiFi{%
          \expandafter\BIC@@@PowRec\romannumeral0%
          \BIC@ProcessMul0!#1!#4!%
          !#1!#2#3!%
        }%
      \else
        \BIC@AfterFiFiFi{%
          \expandafter\BIC@@@PowRec\romannumeral0%
          \BIC@ProcessMul0!#1!#4!%
          !#1!#2#3!%
        }%
      \fi
?   \else\BigIntCalcError:ThisCannotHappen%
    \fi
? \else\BigIntCalcError:ThisCannotHappen%
  \BIC@Fi
}
%    \end{macrocode}
%    \end{macro}
%    \begin{macro}{\BIC@@PowRec}
%    |#1|: $y/2$\\
%    |#2|: $x$\\
%    |#3|: new $r$ ($r$ or $r*x$)
%    \begin{macrocode}
\def\BIC@@PowRec#1!#2!#3!{%
  \expandafter\BIC@PowRec\romannumeral0%
  \BIC@ProcessMul0!#2!#2!%
  !#1!#3!%
}
%    \end{macrocode}
%    \end{macro}
%    \begin{macro}{\BIC@@@PowRec}
%    |#1|: $r*x$
%    |#2|: $x$
%    |#3|: $y$
%    \begin{macrocode}
\def\BIC@@@PowRec#1!#2!#3!{%
  \expandafter\BIC@@PowRec\romannumeral0%
  \BIC@@Shr#3!%
  !#2!#1!%
}
%    \end{macrocode}
%    \end{macro}
%
% \subsection{\op{Div}}
%
%    \begin{macro}{\bigintcalcDiv}
%    |#1|: $x$\\
%    |#2|: $y$ (divisor)
%    \begin{macrocode}
\def\bigintcalcDiv#1{%
  \romannumeral0%
  \expandafter\expandafter\expandafter\BIC@Div
  \bigintcalcNum{#1}!%
}
%    \end{macrocode}
%    \end{macro}
%    \begin{macro}{\BIC@Div}
%    |#1|: $x$\\
%    |#2|: $y$
%    \begin{macrocode}
\def\BIC@Div#1!#2{%
  \expandafter\expandafter\expandafter\BIC@DivSwitchSign
  \bigintcalcNum{#2}!#1!%
}
%    \end{macrocode}
%    \end{macro}
%    \begin{macro}{\BigIntCalcDiv}
%    \begin{macrocode}
\def\BigIntCalcDiv#1!#2!{%
  \romannumeral0%
  \BIC@DivSwitchSign#2!#1!%
}
%    \end{macrocode}
%    \end{macro}
%    \begin{macro}{\BIC@DivSwitchSign}
%    Decision table for \cs{BIC@DivSwitchSign}.
%    \begin{quote}
%    \begin{tabular}{@{}|l|l|l|@{}}
%    \hline
%    $y=0$ & \multicolumn{1}{l}{} & DivisionByZero\\
%    \hline
%    $y>0$ & $x=0$ & $0$\\
%    \cline{2-3}
%          & $x>0$ & DivSwitch$(+,x,y)$\\
%    \cline{2-3}
%          & $x<0$ & DivSwitch$(-,-x,y)$\\
%    \hline
%    $y<0$ & $x=0$ & $0$\\
%    \cline{2-3}
%          & $x>0$ & DivSwitch$(-,x,-y)$\\
%    \cline{2-3}
%          & $x<0$ & DivSwitch$(+,-x,-y)$\\
%    \hline
%    \end{tabular}
%    \end{quote}
%    |#1|: $y$ (divisor)\\
%    |#2|: $x$
%    \begin{macrocode}
\def\BIC@DivSwitchSign#1#2!#3#4!{%
  \ifcase\BIC@Sgn#1#2! % y = 0
    \BIC@AfterFi{ 0\BigIntCalcError:DivisionByZero}%
  \or % y > 0
    \ifcase\BIC@Sgn#3#4! % x = 0
      \BIC@AfterFiFi{ 0}%
    \or % x > 0
      \BIC@AfterFiFi{%
        \BIC@DivSwitch{}#3#4!#1#2!%
      }%
    \else % x < 0
      \BIC@AfterFiFi{%
        \BIC@DivSwitch-#4!#1#2!%
      }%
    \fi
  \else % y < 0
    \ifcase\BIC@Sgn#3#4! % x = 0
      \BIC@AfterFiFi{ 0}%
    \or % x > 0
      \BIC@AfterFiFi{%
        \BIC@DivSwitch-#3#4!#2!%
      }%
    \else % x < 0
      \BIC@AfterFiFi{%
        \BIC@DivSwitch{}#4!#2!%
      }%
    \fi
  \BIC@Fi
}
%    \end{macrocode}
%    \end{macro}
%    \begin{macro}{\BIC@DivSwitch}
%    Decision table for \cs{BIC@DivSwitch}.
%    \begin{quote}
%    \begin{tabular}{@{}|l|l|l|@{}}
%    \hline
%    $y=x$ & \multicolumn{1}{l}{} & sign $1$\\
%    \hline
%    $y>x$ & \multicolumn{1}{l}{} & $0$\\
%    \hline
%    $y<x$ & $y=1$                & sign $x$\\
%    \cline{2-3}
%          & $y=2$                & sign Shr$(x)$\\
%    \cline{2-3}
%          & $y=4$                & sign Shr(Shr$(x)$)\\
%    \cline{2-3}
%          & else                 & sign ProcessDiv$(x,y)$\\
%    \hline
%    \end{tabular}
%    \end{quote}
%    |#1|: sign\\
%    |#2|: $x$\\
%    |#3#4|: $y$ ($y\ne 0$)
%    \begin{macrocode}
\def\BIC@DivSwitch#1#2!#3#4!{%
  \ifcase\BIC@PosCmp#3#4!#2!% y = x
    \BIC@AfterFi{ #11}%
  \or % y > x
    \BIC@AfterFi{ 0}%
  \else % y < x
    \ifx\\#1\\%
    \else
      \expandafter-\romannumeral0%
    \fi
    \ifcase\ifx\\#4\\%
             \ifx#310 % y = 1
             \else\ifx#321 % y = 2
             \else\ifx#342 % y = 4
             \else3 % y > 2
             \fi\fi\fi
           \else
             3 % y > 2
           \fi
      \BIC@AfterFiFi{ #2}% y = 1
    \or % y = 2
      \BIC@AfterFiFi{%
        \BIC@@Shr#2!%
      }%
    \or % y = 4
      \BIC@AfterFiFi{%
        \expandafter\BIC@@Shr\romannumeral0%
          \BIC@@Shr#2!!%
      }%
    \or % y > 2
      \BIC@AfterFiFi{%
        \BIC@DivStartX#2!#3#4!!!%
      }%
?   \else\BigIntCalcError:ThisCannotHappen%
    \fi
  \BIC@Fi
}
%    \end{macrocode}
%    \end{macro}
%    \begin{macro}{\BIC@ProcessDiv}
%    |#1#2|: $x$\\
%    |#3#4|: $y$\\
%    |#5|: collect first digits of $x$\\
%    |#6|: corresponding digits of $y$
%    \begin{macrocode}
\def\BIC@DivStartX#1#2!#3#4!#5!#6!{%
  \ifx\\#4\\%
    \BIC@AfterFi{%
      \BIC@DivStartYii#6#3#4!{#5#1}#2=!%
    }%
  \else
    \BIC@AfterFi{%
      \BIC@DivStartX#2!#4!#5#1!#6#3!%
    }%
  \BIC@Fi
}
%    \end{macrocode}
%    \end{macro}
%    \begin{macro}{\BIC@DivStartYii}
%    |#1|: $y$\\
%    |#2|: $x$, |=|
%    \begin{macrocode}
\def\BIC@DivStartYii#1!{%
  \expandafter\BIC@DivStartYiv\romannumeral0%
  \BIC@Shl#1!%
  !#1!%
}
%    \end{macrocode}
%    \end{macro}
%    \begin{macro}{\BIC@DivStartYiv}
%    |#1|: $2y$\\
%    |#2|: $y$\\
%    |#3|: $x$, |=|
%    \begin{macrocode}
\def\BIC@DivStartYiv#1!{%
  \expandafter\BIC@DivStartYvi\romannumeral0%
  \BIC@Shl#1!%
  !#1!%
}
%    \end{macrocode}
%    \end{macro}
%    \begin{macro}{\BIC@DivStartYvi}
%    |#1|: $4y$\\
%    |#2|: $2y$\\
%    |#3|: $y$\\
%    |#4|: $x$, |=|
%    \begin{macrocode}
\def\BIC@DivStartYvi#1!#2!{%
  \expandafter\BIC@DivStartYviii\romannumeral0%
  \BIC@AddXY#1!#2!!!%
  !#1!#2!%
}
%    \end{macrocode}
%    \end{macro}
%    \begin{macro}{\BIC@DivStartYviii}
%    |#1|: $6y$\\
%    |#2|: $4y$\\
%    |#3|: $2y$\\
%    |#4|: $y$\\
%    |#5|: $x$, |=|
%    \begin{macrocode}
\def\BIC@DivStartYviii#1!#2!{%
  \expandafter\BIC@DivStart\romannumeral0%
  \BIC@Shl#2!%
  !#1!#2!%
}
%    \end{macrocode}
%    \end{macro}
%    \begin{macro}{\BIC@DivStart}
%    |#1|: $8y$\\
%    |#2|: $6y$\\
%    |#3|: $4y$\\
%    |#4|: $2y$\\
%    |#5|: $y$\\
%    |#6|: $x$, |=|
%    \begin{macrocode}
\def\BIC@DivStart#1!#2!#3!#4!#5!#6!{%
  \BIC@ProcessDiv#6!!#5!#4!#3!#2!#1!=%
}
%    \end{macrocode}
%    \end{macro}
%    \begin{macro}{\BIC@ProcessDiv}
%    |#1#2#3|: $x$, |=|\\
%    |#4|: result\\
%    |#5|: $y$\\
%    |#6|: $2y$\\
%    |#7|: $4y$\\
%    |#8|: $6y$\\
%    |#9|: $8y$
%    \begin{macrocode}
\def\BIC@ProcessDiv#1#2#3!#4!#5!{%
  \ifcase\BIC@PosCmp#5!#1!% y = #1
    \ifx#2=%
      \BIC@AfterFiFi{\BIC@DivCleanup{#41}}%
    \else
      \BIC@AfterFiFi{%
        \BIC@ProcessDiv#2#3!#41!#5!%
      }%
    \fi
  \or % y > #1
    \ifx#2=%
      \BIC@AfterFiFi{\BIC@DivCleanup{#40}}%
    \else
      \ifx\\#4\\%
        \BIC@AfterFiFiFi{%
          \BIC@ProcessDiv{#1#2}#3!!#5!%
        }%
      \else
        \BIC@AfterFiFiFi{%
          \BIC@ProcessDiv{#1#2}#3!#40!#5!%
        }%
      \fi
    \fi
  \else % y < #1
    \BIC@AfterFi{%
      \BIC@@ProcessDiv{#1}#2#3!#4!#5!%
    }%
  \BIC@Fi
}
%    \end{macrocode}
%    \end{macro}
%    \begin{macro}{\BIC@DivCleanup}
%    |#1|: result\\
%    |#2|: garbage
%    \begin{macrocode}
\def\BIC@DivCleanup#1#2={ #1}%
%    \end{macrocode}
%    \end{macro}
%    \begin{macro}{\BIC@@ProcessDiv}
%    \begin{macrocode}
\def\BIC@@ProcessDiv#1#2#3!#4!#5!#6!#7!{%
  \ifcase\BIC@PosCmp#7!#1!% 4y = #1
    \ifx#2=%
      \BIC@AfterFiFi{\BIC@DivCleanup{#44}}%
    \else
     \BIC@AfterFiFi{%
       \BIC@ProcessDiv#2#3!#44!#5!#6!#7!%
     }%
    \fi
  \or % 4y > #1
    \ifcase\BIC@PosCmp#6!#1!% 2y = #1
      \ifx#2=%
        \BIC@AfterFiFiFi{\BIC@DivCleanup{#42}}%
      \else
        \BIC@AfterFiFiFi{%
          \BIC@ProcessDiv#2#3!#42!#5!#6!#7!%
        }%
      \fi
    \or % 2y > #1
      \ifx#2=%
        \BIC@AfterFiFiFi{\BIC@DivCleanup{#41}}%
      \else
        \BIC@AfterFiFiFi{%
          \BIC@DivSub#1!#5!#2#3!#41!#5!#6!#7!%
        }%
      \fi
    \else % 2y < #1
      \BIC@AfterFiFi{%
        \expandafter\BIC@ProcessDivII\romannumeral0%
        \BIC@SubXY#1!#6!!!%
        !#2#3!#4!#5!23%
        #6!#7!%
      }%
    \fi
  \else % 4y < #1
    \BIC@AfterFi{%
      \BIC@@@ProcessDiv{#1}#2#3!#4!#5!#6!#7!%
    }%
  \BIC@Fi
}
%    \end{macrocode}
%    \end{macro}
%    \begin{macro}{\BIC@DivSub}
%    Next token group: |#1|-|#2| and next digit |#3|.
%    \begin{macrocode}
\def\BIC@DivSub#1!#2!#3{%
  \expandafter\BIC@ProcessDiv\expandafter{%
    \romannumeral0%
    \BIC@SubXY#1!#2!!!%
    #3%
  }%
}
%    \end{macrocode}
%    \end{macro}
%    \begin{macro}{\BIC@ProcessDivII}
%    |#1|: $x'-2y$\\
%    |#2#3|: remaining $x$, |=|\\
%    |#4|: result\\
%    |#5|: $y$\\
%    |#6|: first possible result digit\\
%    |#7|: second possible result digit
%    \begin{macrocode}
\def\BIC@ProcessDivII#1!#2#3!#4!#5!#6#7{%
  \ifcase\BIC@PosCmp#5!#1!% y = #1
    \ifx#2=%
      \BIC@AfterFiFi{\BIC@DivCleanup{#4#7}}%
    \else
      \BIC@AfterFiFi{%
        \BIC@ProcessDiv#2#3!#4#7!#5!%
      }%
    \fi
  \or % y > #1
    \ifx#2=%
      \BIC@AfterFiFi{\BIC@DivCleanup{#4#6}}%
    \else
      \BIC@AfterFiFi{%
        \BIC@ProcessDiv{#1#2}#3!#4#6!#5!%
      }%
    \fi
  \else % y < #1
    \ifx#2=%
      \BIC@AfterFiFi{\BIC@DivCleanup{#4#7}}%
    \else
      \BIC@AfterFiFi{%
        \BIC@DivSub#1!#5!#2#3!#4#7!#5!%
      }%
    \fi
  \BIC@Fi
}
%    \end{macrocode}
%    \end{macro}
%    \begin{macro}{\BIC@ProcessDivIV}
%    |#1#2#3|: $x$, |=|, $x>4y$\\
%    |#4|: result\\
%    |#5|: $y$\\
%    |#6|: $2y$\\
%    |#7|: $4y$\\
%    |#8|: $6y$\\
%    |#9|: $8y$
%    \begin{macrocode}
\def\BIC@@@ProcessDiv#1#2#3!#4!#5!#6!#7!#8!#9!{%
  \ifcase\BIC@PosCmp#8!#1!% 6y = #1
    \ifx#2=%
      \BIC@AfterFiFi{\BIC@DivCleanup{#46}}%
    \else
      \BIC@AfterFiFi{%
        \BIC@ProcessDiv#2#3!#46!#5!#6!#7!#8!#9!%
      }%
    \fi
  \or % 6y > #1
    \BIC@AfterFi{%
      \expandafter\BIC@ProcessDivII\romannumeral0%
      \BIC@SubXY#1!#7!!!%
      !#2#3!#4!#5!45%
      #6!#7!#8!#9!%
    }%
  \else % 6y < #1
    \ifcase\BIC@PosCmp#9!#1!% 8y = #1
      \ifx#2=%
        \BIC@AfterFiFiFi{\BIC@DivCleanup{#48}}%
      \else
        \BIC@AfterFiFiFi{%
          \BIC@ProcessDiv#2#3!#48!#5!#6!#7!#8!#9!%
        }%
      \fi
    \or % 8y > #1
      \BIC@AfterFiFi{%
        \expandafter\BIC@ProcessDivII\romannumeral0%
        \BIC@SubXY#1!#8!!!%
        !#2#3!#4!#5!67%
        #6!#7!#8!#9!%
      }%
    \else % 8y < #1
      \BIC@AfterFiFi{%
        \expandafter\BIC@ProcessDivII\romannumeral0%
        \BIC@SubXY#1!#9!!!%
        !#2#3!#4!#5!89%
        #6!#7!#8!#9!%
      }%
    \fi
  \BIC@Fi
}
%    \end{macrocode}
%    \end{macro}
%
% \subsection{\op{Mod}}
%
%    \begin{macro}{\bigintcalcMod}
%    |#1|: $x$\\
%    |#2|: $y$
%    \begin{macrocode}
\def\bigintcalcMod#1{%
  \romannumeral0%
  \expandafter\expandafter\expandafter\BIC@Mod
  \bigintcalcNum{#1}!%
}
%    \end{macrocode}
%    \end{macro}
%    \begin{macro}{\BIC@Mod}
%    |#1|: $x$\\
%    |#2|: $y$
%    \begin{macrocode}
\def\BIC@Mod#1!#2{%
  \expandafter\expandafter\expandafter\BIC@ModSwitchSign
  \bigintcalcNum{#2}!#1!%
}
%    \end{macrocode}
%    \end{macro}
%    \begin{macro}{\BigIntCalcMod}
%    \begin{macrocode}
\def\BigIntCalcMod#1!#2!{%
  \romannumeral0%
  \BIC@ModSwitchSign#2!#1!%
}
%    \end{macrocode}
%    \end{macro}
%    \begin{macro}{\BIC@ModSwitchSign}
%    Decision table for \cs{BIC@ModSwitchSign}.
%    \begin{quote}
%    \begin{tabular}{@{}|l|l|l|@{}}
%    \hline
%    $y=0$ & \multicolumn{1}{l}{} & DivisionByZero\\
%    \hline
%    $y>0$ & $x=0$ & $0$\\
%    \cline{2-3}
%          & else  & ModSwitch$(+,x,y)$\\
%    \hline
%    $y<0$ & \multicolumn{1}{l}{} & ModSwitch$(-,-x,-y)$\\
%    \hline
%    \end{tabular}
%    \end{quote}
%    |#1#2|: $y$\\
%    |#3#4|: $x$
%    \begin{macrocode}
\def\BIC@ModSwitchSign#1#2!#3#4!{%
  \ifcase\ifx\\#2\\%
           \ifx#100 % y = 0
           \else1 % y > 0
           \fi
          \else
            \ifx#1-2 % y < 0
            \else1 % y > 0
            \fi
          \fi
    \BIC@AfterFi{ 0\BigIntCalcError:DivisionByZero}%
  \or % y > 0
    \ifcase\ifx\\#4\\\ifx#300 \else1 \fi\else1 \fi % x = 0
      \BIC@AfterFiFi{ 0}%
    \else
      \BIC@AfterFiFi{%
        \BIC@ModSwitch{}#3#4!#1#2!%
      }%
    \fi
  \else % y < 0
    \ifcase\ifx\\#4\\%
             \ifx#300 % x = 0
             \else1 % x > 0
             \fi
           \else
             \ifx#3-2 % x < 0
             \else1 % x > 0
             \fi
           \fi
      \BIC@AfterFiFi{ 0}%
    \or % x > 0
      \BIC@AfterFiFi{%
        \BIC@ModSwitch--#3#4!#2!%
      }%
    \else % x < 0
      \BIC@AfterFiFi{%
        \BIC@ModSwitch-#4!#2!%
      }%
    \fi
  \BIC@Fi
}
%    \end{macrocode}
%    \end{macro}
%    \begin{macro}{\BIC@ModSwitch}
%    Decision table for \cs{BIC@ModSwitch}.
%    \begin{quote}
%    \begin{tabular}{@{}|l|l|l|@{}}
%    \hline
%    $y=1$ & \multicolumn{1}{l}{} & $0$\\
%    \hline
%    $y=2$ & ifodd$(x)$ & sign $1$\\
%    \cline{2-3}
%          & else       & $0$\\
%    \hline
%    $y>2$ & $x<0$      &
%        $z\leftarrow x-(x/y)*y;\quad (z<0)\mathbin{?}z+y\mathbin{:}z$\\
%    \cline{2-3}
%          & $x>0$      & $x - (x/y) * y$\\
%    \hline
%    \end{tabular}
%    \end{quote}
%    |#1|: sign\\
%    |#2#3|: $x$\\
%    |#4#5|: $y$
%    \begin{macrocode}
\def\BIC@ModSwitch#1#2#3!#4#5!{%
  \ifcase\ifx\\#5\\%
           \ifx#410 % y = 1
           \else\ifx#421 % y = 2
           \else2 % y > 2
           \fi\fi
         \else2 % y > 2
         \fi
    \BIC@AfterFi{ 0}% y = 1
  \or % y = 2
    \ifcase\BIC@ModTwo#2#3! % even(x)
      \BIC@AfterFiFi{ 0}%
    \or % odd(x)
      \BIC@AfterFiFi{ #11}%
?   \else\BigIntCalcError:ThisCannotHappen%
    \fi
  \or % y > 2
    \ifx\\#1\\%
    \else
      \expandafter\BIC@Space\romannumeral0%
      \expandafter\BIC@ModMinus\romannumeral0%
    \fi
    \ifx#2-% x < 0
      \BIC@AfterFiFi{%
        \expandafter\expandafter\expandafter\BIC@ModX
        \bigintcalcSub{#2#3}{%
          \bigintcalcMul{#4#5}{\bigintcalcDiv{#2#3}{#4#5}}%
        }!#4#5!%
      }%
    \else % x > 0
      \BIC@AfterFiFi{%
        \expandafter\expandafter\expandafter\BIC@Space
        \bigintcalcSub{#2#3}{%
          \bigintcalcMul{#4#5}{\bigintcalcDiv{#2#3}{#4#5}}%
        }%
      }%
    \fi
? \else\BigIntCalcError:ThisCannotHappen%
  \BIC@Fi
}
%    \end{macrocode}
%    \end{macro}
%    \begin{macro}{\BIC@ModMinus}
%    \begin{macrocode}
\def\BIC@ModMinus#1{%
  \ifx#10%
    \BIC@AfterFi{ 0}%
  \else
    \BIC@AfterFi{ -#1}%
  \BIC@Fi
}
%    \end{macrocode}
%    \end{macro}
%    \begin{macro}{\BIC@ModX}
%    |#1#2|: $z$\\
%    |#3|: $x$
%    \begin{macrocode}
\def\BIC@ModX#1#2!#3!{%
  \ifx#1-% z < 0
    \BIC@AfterFi{%
      \expandafter\BIC@Space\romannumeral0%
      \BIC@SubXY#3!#2!!!%
    }%
  \else % z >= 0
    \BIC@AfterFi{ #1#2}%
  \BIC@Fi
}
%    \end{macrocode}
%    \end{macro}
%
%    \begin{macrocode}
\BIC@AtEnd%
%    \end{macrocode}
%
%    \begin{macrocode}
%</package>
%    \end{macrocode}
%
% \section{Test}
%
% \subsection{Catcode checks for loading}
%
%    \begin{macrocode}
%<*test1>
%    \end{macrocode}
%    \begin{macrocode}
\catcode`\{=1 %
\catcode`\}=2 %
\catcode`\#=6 %
\catcode`\@=11 %
\expandafter\ifx\csname count@\endcsname\relax
  \countdef\count@=255 %
\fi
\expandafter\ifx\csname @gobble\endcsname\relax
  \long\def\@gobble#1{}%
\fi
\expandafter\ifx\csname @firstofone\endcsname\relax
  \long\def\@firstofone#1{#1}%
\fi
\expandafter\ifx\csname loop\endcsname\relax
  \expandafter\@firstofone
\else
  \expandafter\@gobble
\fi
{%
  \def\loop#1\repeat{%
    \def\body{#1}%
    \iterate
  }%
  \def\iterate{%
    \body
      \let\next\iterate
    \else
      \let\next\relax
    \fi
    \next
  }%
  \let\repeat=\fi
}%
\def\RestoreCatcodes{}
\count@=0 %
\loop
  \edef\RestoreCatcodes{%
    \RestoreCatcodes
    \catcode\the\count@=\the\catcode\count@\relax
  }%
\ifnum\count@<255 %
  \advance\count@ 1 %
\repeat

\def\RangeCatcodeInvalid#1#2{%
  \count@=#1\relax
  \loop
    \catcode\count@=15 %
  \ifnum\count@<#2\relax
    \advance\count@ 1 %
  \repeat
}
\def\RangeCatcodeCheck#1#2#3{%
  \count@=#1\relax
  \loop
    \ifnum#3=\catcode\count@
    \else
      \errmessage{%
        Character \the\count@\space
        with wrong catcode \the\catcode\count@\space
        instead of \number#3%
      }%
    \fi
  \ifnum\count@<#2\relax
    \advance\count@ 1 %
  \repeat
}
\def\space{ }
\expandafter\ifx\csname LoadCommand\endcsname\relax
  \def\LoadCommand{\input bigintcalc.sty\relax}%
\fi
\def\Test{%
  \RangeCatcodeInvalid{0}{47}%
  \RangeCatcodeInvalid{58}{64}%
  \RangeCatcodeInvalid{91}{96}%
  \RangeCatcodeInvalid{123}{255}%
  \catcode`\@=12 %
  \catcode`\\=0 %
  \catcode`\%=14 %
  \LoadCommand
  \RangeCatcodeCheck{0}{36}{15}%
  \RangeCatcodeCheck{37}{37}{14}%
  \RangeCatcodeCheck{38}{47}{15}%
  \RangeCatcodeCheck{48}{57}{12}%
  \RangeCatcodeCheck{58}{63}{15}%
  \RangeCatcodeCheck{64}{64}{12}%
  \RangeCatcodeCheck{65}{90}{11}%
  \RangeCatcodeCheck{91}{91}{15}%
  \RangeCatcodeCheck{92}{92}{0}%
  \RangeCatcodeCheck{93}{96}{15}%
  \RangeCatcodeCheck{97}{122}{11}%
  \RangeCatcodeCheck{123}{255}{15}%
  \RestoreCatcodes
}
\Test
\csname @@end\endcsname
\end
%    \end{macrocode}
%    \begin{macrocode}
%</test1>
%    \end{macrocode}
%
% \subsection{Macro tests}
%
% \subsubsection{Preamble with test macro definitions}
%
%    \begin{macrocode}
%<*test2>
\NeedsTeXFormat{LaTeX2e}
\nofiles
\documentclass{article}
%<noetex>\let\SavedNumexpr\numexpr
%<noetex>\let\numexpr\UNDEFINED
\makeatletter
\chardef\BIC@TestMode=1 %
\makeatother
\usepackage{bigintcalc}[2016/05/16]
%<noetex>\let\numexpr\SavedNumexpr
\usepackage{qstest}
\IncludeTests{*}
\LogTests{log}{*}{*}
\newcommand*{\TestSpaceAtEnd}[1]{%
%<noetex>  \let\SavedNumexpr\numexpr
%<noetex>  \let\numexpr\UNDEFINED
  \edef\resultA{#1}%
  \edef\resultB{#1 }%
%<noetex>  \let\numexpr\SavedNumexpr
  \Expect*{\resultA\space}*{\resultB}%
}
\newcommand*{\TestResult}[2]{%
%<noetex>  \let\SavedNumexpr\numexpr
%<noetex>  \let\numexpr\UNDEFINED
  \edef\result{#1}%
%<noetex>  \let\numexpr\SavedNumexpr
  \Expect*{\result}{#2}%
}
\newcommand*{\TestResultTwoExpansions}[2]{%
%<*noetex>
  \begingroup
    \let\numexpr\UNDEFINED
    \expandafter\expandafter\expandafter
  \endgroup
%</noetex>
  \expandafter\expandafter\expandafter\Expect
  \expandafter\expandafter\expandafter{#1}{#2}%
}
\newcount\TestCount
%<etex>\newcommand*{\TestArg}[1]{\numexpr#1\relax}
%<noetex>\newcommand*{\TestArg}[1]{#1}
\newcommand*{\TestTeXDivide}[2]{%
  \TestCount=\TestArg{#1}\relax
  \divide\TestCount by \TestArg{#2}\relax
  \Expect*{\bigintcalcDiv{#1}{#2}}*{\the\TestCount}%
}
\newcommand*{\Test}[2]{%
  \TestResult{#1}{#2}%
  \TestResultTwoExpansions{#1}{#2}%
  \TestSpaceAtEnd{#1}%
}
\newcommand*{\TestExch}[2]{\Test{#2}{#1}}
\newcommand*{\TestInv}[2]{%
  \Test{\bigintcalcInv{#1}}{#2}%
}
\newcommand*{\TestAbs}[2]{%
  \Test{\bigintcalcAbs{#1}}{#2}%
}
\newcommand*{\TestSgn}[2]{%
  \Test{\bigintcalcSgn{#1}}{#2}%
}
\newcommand*{\TestMin}[3]{%
  \Test{\bigintcalcMin{#1}{#2}}{#3}%
}
\newcommand*{\TestMax}[3]{%
  \Test{\bigintcalcMax{#1}{#2}}{#3}%
}
\newcommand*{\TestCmp}[3]{%
  \Test{\bigintcalcCmp{#1}{#2}}{#3}%
}
\newcommand*{\TestOdd}[2]{%
  \Test{\bigintcalcOdd{#1}}{#2}%
  \edef\x{%
    \noexpand\Test{%
      \noexpand\BigIntCalcOdd
      \bigintcalcAbs{#1}!%
    }{#2}%
  }%
  \x
}
\newcommand*{\TestInc}[2]{%
  \Test{\bigintcalcInc{#1}}{#2}%
  \ifnum\bigintcalcSgn{#1}>-1 %
    \edef\x{%
      \noexpand\Test{%
        \noexpand\BigIntCalcInc\bigintcalcNum{#1}!%
      }{#2}%
    }%
    \x
  \fi
}
\newcommand*{\TestDec}[2]{%
  \Test{\bigintcalcDec{#1}}{#2}%
  \ifnum\bigintcalcSgn{#1}>0 %
    \edef\x{%
      \noexpand\Test{%
        \noexpand\BigIntCalcDec\bigintcalcNum{#1}!%
      }{#2}%
    }%
    \x
  \fi
}
\newcommand*{\TestAdd}[3]{%
  \Test{\bigintcalcAdd{#1}{#2}}{#3}%
  \ifnum\bigintcalcSgn{#1}>0 %
    \ifnum\bigintcalcSgn{#2}> 0 %
      \ifnum\bigintcalcCmp{#1}{#2}>0 %
        \edef\x{%
          \noexpand\Test{%
            \noexpand\BigIntCalcAdd
            \bigintcalcNum{#1}!\bigintcalcNum{#2}!%
          }{#3}%
        }%
        \x
      \else
        \edef\x{%
          \noexpand\Test{%
            \noexpand\BigIntCalcAdd
            \bigintcalcNum{#2}!\bigintcalcNum{#1}!%
          }{#3}%
        }%
        \x
      \fi
    \fi
  \fi
}
\newcommand*{\TestSub}[3]{%
  \Test{\bigintcalcSub{#1}{#2}}{#3}%
  \ifnum\bigintcalcSgn{#1}>0 %
    \ifnum\bigintcalcSgn{#2}> 0 %
      \ifnum\bigintcalcCmp{#1}{#2}>0 %
        \edef\x{%
          \noexpand\Test{%
            \noexpand\BigIntCalcSub
            \bigintcalcNum{#1}!\bigintcalcNum{#2}!%
          }{#3}%
        }%
        \x
      \fi
    \fi
  \fi
}
\newcommand*{\TestShl}[2]{%
  \Test{\bigintcalcShl{#1}}{#2}%
  \edef\x{%
    \noexpand\Test{%
      \noexpand\BigIntCalcShl\bigintcalcAbs{#1}!%
    }{\bigintcalcAbs{#2}}%
  }%
  \x
}
\newcommand*{\TestShr}[2]{%
  \Test{\bigintcalcShr{#1}}{#2}%
  \edef\x{%
    \noexpand\Test{%
      \noexpand\BigIntCalcShr\bigintcalcAbs{#1}!%
    }{\bigintcalcAbs{#2}}%
  }%
  \x
}
\newcommand*{\TestMul}[3]{%
  \Test{\bigintcalcMul{#1}{#2}}{#3}%
  \edef\x{%
    \noexpand\Test{%
      \noexpand\BigIntCalcMul
      \bigintcalcAbs{#1}!\bigintcalcAbs{#2}!%
    }{\bigintcalcAbs{#3}}%
  }%
  \x
}
\newcommand*{\TestSqr}[2]{%
  \Test{\bigintcalcSqr{#1}}{#2}%
}
\newcommand*{\TestFac}[2]{%
  \expandafter\TestExch\expandafter{%
    \the\numexpr#2%
  }{\bigintcalcFac{#1}}%
}
\newcommand*{\TestFacBig}[2]{%
  \Test{\bigintcalcFac{#1}}{#2}%
}
\newcommand*{\TestPow}[3]{%
  \Test{\bigintcalcPow{#1}{#2}}{#3}%
}
\newcommand*{\TestDiv}[3]{%
  \Test{\bigintcalcDiv{#1}{#2}}{#3}%
  \TestTeXDivide{#1}{#2}%
}
\newcommand*{\TestDivBig}[3]{%
  \Test{\bigintcalcDiv{#1}{#2}}{#3}%
  \edef\x{%
    \noexpand\Test{%
      \noexpand\BigIntCalcDiv\bigintcalcAbs{#1}!\bigintcalcAbs{#2}!%
    }{\bigintcalcAbs{#3}}%
  }%
}
\newcommand*{\TestMod}[3]{%
  \Test{\bigintcalcMod{#1}{#2}}{#3}%
  \ifcase\ifcase\bigintcalcSgn{#1} 0%
         \or
           \ifcase\bigintcalcSgn{#2} 1%
           \or 0%
           \else 1%
           \fi
         \else
           \ifcase\bigintcalcSgn{#2} 1%
           \or 1%
           \else 0%
           \fi
         \fi\relax
    \edef\x{%
      \noexpand\Test{%
        \noexpand\BigIntCalcMod
        \bigintcalcAbs{#1}!\bigintcalcAbs{#2}!%
      }{\bigintcalcAbs{#3}}%
    }%
    \x
  \fi
}
%    \end{macrocode}
%
% \subsubsection{Time}
%
%    \begin{macrocode}
\begingroup\expandafter\expandafter\expandafter\endgroup
\expandafter\ifx\csname pdfresettimer\endcsname\relax
\else
  \makeatletter
  \newcount\SummaryTime
  \newcount\TestTime
  \SummaryTime=\z@
  \newcommand*{\PrintTime}[2]{%
    \typeout{%
      [Time #1: \strip@pt\dimexpr\number#2sp\relax\space s]%
    }%
  }%
  \newcommand*{\StartTime}[1]{%
    \renewcommand*{\TimeDescription}{#1}%
    \pdfresettimer
  }%
  \newcommand*{\TimeDescription}{}%
  \newcommand*{\StopTime}{%
    \TestTime=\pdfelapsedtime
    \global\advance\SummaryTime\TestTime
    \PrintTime\TimeDescription\TestTime
  }%
  \let\saved@qstest\qstest
  \let\saved@endqstest\endqstest
  \def\qstest#1#2{%
    \saved@qstest{#1}{#2}%
    \StartTime{#1}%
  }%
  \def\endqstest{%
    \StopTime
    \saved@endqstest
  }%
  \AtEndDocument{%
    \PrintTime{summary}\SummaryTime
  }%
  \makeatother
\fi
%    \end{macrocode}
%
% \subsubsection{Test sets}
%
%    \begin{macrocode}
\makeatletter

\begin{qstest}{inv}{inv}%
  \TestInv{0}{0}%
  \TestInv{1}{-1}%
  \TestInv{-1}{1}%
  \TestInv{10}{-10}%
  \TestInv{-10}{10}%
  \TestInv{2147483647}{-2147483647}%
  \TestInv{-2147483647}{2147483647}%
  \TestInv{12345678901234567890}{-12345678901234567890}%
  \TestInv{-12345678901234567890}{12345678901234567890}%
  \TestInv{ 0 }{0}%
  \TestInv{ 1 }{-1}%
  \TestInv{--1}{-1}%
  \TestInv{\number\z@}{0}%
  \TestInv{\ifx\relax\relax1\fi}{-1}%
  \TestInv{\ifx\relax\relax-\fi\ifx234\else1\fi}{1}%
\end{qstest}

\begin{qstest}{abs}{abs}%
  \TestAbs{0}{0}%
  \TestAbs{1}{1}%
  \TestAbs{-1}{1}%
  \TestAbs{10}{10}%
  \TestAbs{-10}{10}%
  \TestAbs{2147483647}{2147483647}%
  \TestAbs{-2147483647}{2147483647}%
  \TestAbs{12345678901234567890}{12345678901234567890}%
  \TestAbs{-12345678901234567890}{12345678901234567890}%
  \TestAbs{ 0 }{0}%
  \TestAbs{ 1 }{1}%
  \TestAbs{--1}{1}%
  \TestAbs{-+-+1}{1}%
  \TestAbs{00000000000}{0}%
  \TestAbs{00000001000}{1000}%
  \TestAbs{\ifx\relax\relax 0\else 1\fi}{0}%
\end{qstest}

\begin{qstest}{sign}{sign}%
  \TestSgn{0}{0}%
  \TestSgn{1}{1}%
  \TestSgn{-1}{-1}%
  \TestSgn{10}{1}%
  \TestSgn{-10}{-1}%
  \TestSgn{2147483647}{1}%
  \TestSgn{-2147483647}{-1}%
  \TestSgn{12345678901234567890}{1}%
  \TestSgn{-12345678901234567890}{-1}%
  \TestSgn{ 0 }{0}%
  \TestSgn{ 2 }{1}%
  \TestSgn{ -2 }{-1}%
  \TestSgn{--2}{1}%
  \TestSgn{\number\z@}{0}%
  \TestSgn{\number\@ne}{1}%
  \TestSgn{\number\m@ne}{-1}%
  \TestSgn{%
    -+-+\number\z@\number\z@
    \iftrue1\fi\iftrue2\fi\iftrue3\fi
  }{1}%
\end{qstest}

\begin{qstest}{min}{min}%
  \TestMin{0}{1}{0}%
  \TestMin{1}{0}{0}%
  \TestMin{-10}{-20}{-20}%
  \TestMin{ 1 }{ 2 }{1}%
  \TestMin{ 2 }{ 1 }{1}%
  \TestMin{1}{1}{1}%
  \TestMin{\number\z@}{\number\@ne}{0}%
  \TestMin{\number\@ne}{\number\m@ne}{-1}%
\end{qstest}

\begin{qstest}{max}{max}%
  \TestMax{0}{1}{1}%
  \TestMax{1}{0}{1}%
  \TestMax{-10}{-20}{-10}%
  \TestMax{ 1 }{ 2 }{2}%
  \TestMax{ 2 }{ 1 }{2}%
  \TestMax{1}{1}{1}%
  \TestMax{\number\z@}{\number\@ne}{1}%
  \TestMax{\number\@ne}{\number\m@ne}{1}%
\end{qstest}

\begin{qstest}{cmp}{cmp}%
  \TestCmp{0}{0}{0}%
  \TestCmp{-21}{17}{-1}%
  \TestCmp{3}{4}{-1}%
  \TestCmp{-10}{-10}{0}%
  \TestCmp{-10}{-11}{1}%
  \TestCmp{100}{5}{1}%
  \TestCmp{9}{10}{-1}%
  \TestCmp{10}{9}{1}%
  \TestCmp{ 3 }{ 3 }{0}%
  \TestCmp{-9}{-10}{1}%
  \TestCmp{-10}{-9}{-1}%
  \TestCmp{-3}{-3}{0}%
  \TestCmp{0}{-2}{1}%
  \TestCmp{0}{2}{-1}%
  \TestCmp{2}{0}{1}%
  \TestCmp{-2}{0}{-1}%
  \TestCmp{12}{11}{1}%
  \TestCmp{11}{12}{-1}%
  \TestCmp{2147483647}{-2147483647}{1}%
  \TestCmp{-2147483647}{2147483647}{-1}%
  \TestCmp{2147483647}{2147483647}{0}%
  \TestCmp{\number\z@}{\number\@ne}{-1}%
  \TestCmp{\number\@ne}{\number\m@ne}{1}%
  \TestCmp{ 4 }{ 5 }{-1}%
  \TestCmp{ -3 }{ -7 }{1}%
\end{qstest}

\begin{qstest}{odd}{odd}
\tracingmacros=1
  \TestOdd{0}{0}%
  \TestOdd{1}{1}%
  \TestOdd{2}{0}%
  \TestOdd{3}{1}%
  \TestOdd{14}{0}%
  \TestOdd{15}{1}%
  \TestOdd{12345678901234567896}{0}%
  \TestOdd{12345678901234567897}{1}%
\end{qstest}

\begin{qstest}{inc}{inc}%
  \TestInc{0}{1}%
  \TestInc{1}{2}%
  \TestInc{-1}{0}%
  \TestInc{10}{11}%
  \TestInc{-10}{-9}%
  \TestInc{ 3 }{4}%
  \TestInc{999}{1000}%
  \TestInc{-1000}{-999}%
  \TestInc{129}{130}%
  \TestInc{2147483646}{2147483647}%
  \TestInc{-2147483647}{-2147483646}%
  \TestInc{12345678901234567890}{12345678901234567891}%
  \TestInc{99999999999999999999}{100000000000000000000}%
  \TestInc{-12345678901234567891}{-12345678901234567890}%
  \TestInc{-100000000000000000000}{-99999999999999999999}%
\end{qstest}

\begin{qstest}{dec}{dec}%
  \TestDec{0}{-1}%
  \TestDec{1}{0}%
  \TestDec{-1}{-2}%
  \TestDec{10}{9}%
  \TestDec{-10}{-11}%
  \TestDec{1000}{999}%
  \TestDec{-999}{-1000}%
  \TestDec{130}{129}%
  \TestDec{2147483647}{2147483646}%
  \TestDec{-2147483646}{-2147483647}%
  \TestDec{12345678901234567891}{12345678901234567890}%
  \TestDec{100000000000000000000}{99999999999999999999}%
  \TestDec{-12345678901234567890}{-12345678901234567891}%
  \TestDec{-99999999999999999999}{-100000000000000000000}%
\end{qstest}

\begin{qstest}{add}{add}%
  \TestAdd{0}{0}{0}%
  \TestAdd{1}{0}{1}%
  \TestAdd{0}{1}{1}%
  \TestAdd{1}{2}{3}%
  \TestAdd{-1}{-1}{-2}%
  \TestAdd{2147483646}{1}{2147483647}%
  \TestAdd{-2147483647}{2147483647}{0}%
  \TestAdd{20}{-5}{15}%
  \TestAdd{-4}{-1}{-5}%
  \TestAdd{-1}{-4}{-5}%
  \TestAdd{-4}{1}{-3}%
  \TestAdd{-1}{4}{3}%
  \TestAdd{4}{-1}{3}%
  \TestAdd{1}{-4}{-3}%
  \TestAdd{-4}{-1}{-5}%
  \TestAdd{-1}{-4}{-5}%
  \TestAdd{ -4 }{ -1 }{-5}%
  \TestAdd{ -1 }{ -4 }{-5}%
  \TestAdd{ -4 }{ 1 }{-3}%
  \TestAdd{ -1 }{ 4 }{3}%
  \TestAdd{ 4 }{ -1 }{3}%
  \TestAdd{ 1 }{ -4 }{-3}%
  \TestAdd{ -4 }{ -1 }{-5}%
  \TestAdd{ -1 }{ -4 }{-5}%
  \TestAdd{876543210}{111111111}{987654321}%
  \TestAdd{999999999}{2}{1000000001}%
\end{qstest}

\begin{qstest}{sub}{sub}
  \TestSub{0}{0}{0}%
  \TestSub{1}{0}{1}%
  \TestSub{1}{2}{-1}%
  \TestSub{-1}{-1}{0}%
  \TestSub{2147483646}{-1}{2147483647}%
  \TestSub{-2147483647}{-2147483647}{0}%
  \TestSub{-4}{-1}{-3}%
  \TestSub{-1}{-4}{3}%
  \TestSub{-4}{1}{-5}%
  \TestSub{-1}{4}{-5}%
  \TestSub{4}{-1}{5}%
  \TestSub{1}{-4}{5}%
  \TestSub{-4}{-1}{-3}%
  \TestSub{-1}{-4}{3}%
  \TestSub{ -4 }{ -1 }{-3}%
  \TestSub{ -1 }{ -4 }{3}%
  \TestSub{ -4 }{ 1 }{-5}%
  \TestSub{ -1 }{ 4 }{-5}%
  \TestSub{ 4 }{ -1 }{5}%
  \TestSub{ 1 }{ -4 }{5}%
  \TestSub{ -4 }{ -1 }{-3}%
  \TestSub{ -1 }{ -4 }{3}%
  \TestSub{1000000000}{2}{999999998}%
  \TestSub{987654321}{111111111}{876543210}%
\end{qstest}

\begin{qstest}{shl}{shl}
  \TestShl{0}{0}%
  \TestShl{1}{2}%
  \TestShl{2}{4}%
  \TestShl{5621}{11242}%
  \TestShl{1073741823}{2147483646}%
\end{qstest}

\begin{qstest}{shr}{shr}
  \TestShr{0}{0}%
  \TestShr{1}{0}%
  \TestShr{2}{1}%
  \TestShr{3}{1}%
  \TestShr{4}{2}%
  \TestShr{5}{2}%
  \TestShr{6}{3}%
  \TestShr{7}{3}%
  \TestShr{8}{4}%
  \TestShr{9}{4}%
  \TestShr{10}{5}%
  \TestShr{11}{5}%
  \TestShr{12}{6}%
  \TestShr{13}{6}%
  \TestShr{14}{7}%
  \TestShr{15}{7}%
  \TestShr{16}{8}%
  \TestShr{17}{8}%
  \TestShr{18}{9}%
  \TestShr{19}{9}%
  \TestShr{20}{10}%
  \TestShr{21}{10}%
  \TestShr{22}{11}%
  \TestShr{11241}{5620}%
  \TestShr{73054202}{36527101}%
  \TestShr{2147483646}{1073741823}%
\end{qstest}

\begin{qstest}{mul}{mul}
  \TestMul{0}{0}{0}%
  \TestMul{1}{0}{0}%
  \TestMul{0}{1}{0}%
  \TestMul{1}{1}{1}%
  \TestMul{3}{1}{3}%
  \TestMul{1}{-3}{-3}%
  \TestMul{-4}{-5}{20}%
  \TestMul{3}{7}{21}%
  \TestMul{7}{3}{21}%
  \TestMul{3}{-7}{-21}%
  \TestMul{7}{-3}{-21}%
  \TestMul{-3}{7}{-21}%
  \TestMul{-7}{3}{-21}%
  \TestMul{-3}{-7}{21}%
  \TestMul{-7}{-3}{21}%
  \TestMul{12}{11}{132}%
  \TestMul{999}{333}{332667}%
  \TestMul{1000}{4321}{4321000}%
  \TestMul{12345}{173955}{2147474475}%
  \TestMul{1073741823}{2}{2147483646}%
  \TestMul{2}{1073741823}{2147483646}%
  \TestMul{-1073741823}{2}{-2147483646}%
  \TestMul{2}{-1073741823}{-2147483646}%
  \TestMul{6706022400}{13}{87178291200}%
\end{qstest}

\begin{qstest}{sqr}{sqr}
  \TestSqr{0}{0}%
  \TestSqr{1}{1}%
  \TestSqr{2}{4}%
  \TestSqr{3}{9}%
  \TestSqr{4}{16}%
  \TestSqr{9}{81}%
  \TestSqr{10}{100}%
  \TestSqr{46340}{2147395600}%
  \TestSqr{-1}{1}%
  \TestSqr{-2}{4}%
  \TestSqr{-46340}{2147395600}%
\end{qstest}

\begin{qstest}{fac}{fac}
  \TestFac{0}{1}%
  \TestFac{1}{1}%
  \TestFac{2}{2}%
  \TestFac{3}{2*3}%
  \TestFac{4}{2*3*4}%
  \TestFac{5}{2*3*4*5}%
  \TestFac{6}{2*3*4*5*6}%
  \TestFac{7}{2*3*4*5*6*7}%
  \TestFac{8}{2*3*4*5*6*7*8}%
  \TestFac{9}{2*3*4*5*6*7*8*9}%
  \TestFac{10}{2*3*4*5*6*7*8*9*10}%
  \TestFac{11}{2*3*4*5*6*7*8*9*10*11}%
  \TestFac{12}{2*3*4*5*6*7*8*9*10*11*12}%
  \TestFacBig{13}{6227020800}%
  \TestFacBig{14}{87178291200}%
  \TestFacBig{15}{1307674368000}%
  \TestFacBig{16}{20922789888000}%
  \TestFacBig{17}{355687428096000}%
  \TestFacBig{18}{6402373705728000}%
  \TestFacBig{19}{121645100408832000}%
  \TestFacBig{20}{2432902008176640000}%
  \TestFacBig{21}{51090942171709440000}%
  \TestFacBig{22}{1124000727777607680000}%
\end{qstest}

\begin{qstest}{pow}{pow}
  \TestPow{-2}{0}{1}%
  \TestPow{-1}{0}{1}%
  \TestPow{0}{0}{1}%
  \TestPow{1}{0}{1}%
  \TestPow{2}{0}{1}%
  \TestPow{3}{0}{1}%
  \TestPow{-2}{1}{-2}%
  \TestPow{-1}{1}{-1}%
  \TestPow{1}{1}{1}%
  \TestPow{2}{1}{2}%
  \TestPow{3}{1}{3}%
  \TestPow{-2}{2}{4}%
  \TestPow{-1}{2}{1}%
  \TestPow{0}{2}{0}%
  \TestPow{1}{2}{1}%
  \TestPow{2}{2}{4}%
  \TestPow{3}{2}{9}%
  \TestPow{0}{1}{0}%
  \TestPow{1}{-2}{1}%
  \TestPow{1}{-1}{1}%
  \TestPow{-1}{-2}{1}%
  \TestPow{-1}{-1}{-1}%
  \TestPow{-1}{3}{-1}%
  \TestPow{-1}{4}{1}%
  \TestPow{-2}{-1}{0}%
  \TestPow{-2}{-2}{0}%
  \TestPow{2}{3}{8}%
  \TestPow{2}{4}{16}%
  \TestPow{2}{5}{32}%
  \TestPow{2}{6}{64}%
  \TestPow{2}{7}{128}%
  \TestPow{2}{8}{256}%
  \TestPow{2}{9}{512}%
  \TestPow{2}{10}{1024}%
  \TestPow{-2}{3}{-8}%
  \TestPow{-2}{4}{16}%
  \TestPow{-2}{5}{-32}%
  \TestPow{-2}{6}{64}%
  \TestPow{-2}{7}{-128}%
  \TestPow{-2}{8}{256}%
  \TestPow{-2}{9}{-512}%
  \TestPow{-2}{10}{1024}%
  \TestPow{3}{3}{27}%
  \TestPow{3}{4}{81}%
  \TestPow{3}{5}{243}%
  \TestPow{-3}{3}{-27}%
  \TestPow{-3}{4}{81}%
  \TestPow{-3}{5}{-243}%
  \TestPow{2}{30}{1073741824}%
  \TestPow{-3}{19}{-1162261467}%
  \TestPow{5}{13}{1220703125}%
  \TestPow{-7}{11}{-1977326743}%
\end{qstest}

\begin{qstest}{div}{div}
  \TestDiv{1}{1}{1}%
  \TestDiv{2}{1}{2}%
  \TestDiv{-2}{1}{-2}%
  \TestDiv{2}{-1}{-2}%
  \TestDiv{-2}{-1}{2}%
  \TestDiv{15}{2}{7}%
  \TestDiv{-16}{2}{-8}%
  \TestDiv{1}{2}{0}%
  \TestDiv{1}{3}{0}%
  \TestDiv{2}{3}{0}%
  \TestDiv{-2}{3}{0}%
  \TestDiv{2}{-3}{0}%
  \TestDiv{-2}{-3}{0}%
  \TestDiv{13}{3}{4}%
  \TestDiv{-13}{-3}{4}%
  \TestDiv{-13}{3}{-4}%
  \TestDiv{-6}{5}{-1}%
  \TestDiv{-5}{5}{-1}%
  \TestDiv{-4}{5}{0}%
  \TestDiv{-3}{5}{0}%
  \TestDiv{-2}{5}{0}%
  \TestDiv{-1}{5}{0}%
  \TestDiv{0}{5}{0}%
  \TestDiv{1}{5}{0}%
  \TestDiv{2}{5}{0}%
  \TestDiv{3}{5}{0}%
  \TestDiv{4}{5}{0}%
  \TestDiv{5}{5}{1}%
  \TestDiv{6}{5}{1}%
  \TestDiv{-5}{4}{-1}%
  \TestDiv{-4}{4}{-1}%
  \TestDiv{-3}{4}{0}%
  \TestDiv{-2}{4}{0}%
  \TestDiv{-1}{4}{0}%
  \TestDiv{0}{4}{0}%
  \TestDiv{1}{4}{0}%
  \TestDiv{2}{4}{0}%
  \TestDiv{3}{4}{0}%
  \TestDiv{4}{4}{1}%
  \TestDiv{5}{4}{1}%
  \TestDiv{12345}{678}{18}%
  \TestDiv{32372}{5952}{5}%
  \TestDiv{284271294}{18162}{15651}%
  \TestDiv{217652429}{12561}{17327}%
  \TestDiv{462028434}{5439}{84947}%
  \TestDiv{2147483647}{1000}{2147483}%
  \TestDiv{2147483647}{-1000}{-2147483}%
  \TestDiv{-2147483647}{1000}{-2147483}%
  \TestDiv{-2147483647}{-1000}{2147483}%
  \TestDiv{0}{3}{0}%
  \TestDiv{1}{3}{0}%
  \TestDiv{2}{3}{0}%
  \TestDiv{3}{3}{1}%
  \TestDiv{4}{3}{1}%
  \TestDiv{5}{3}{1}%
  \TestDiv{6}{3}{2}%
  \TestDiv{7}{3}{2}%
  \TestDiv{8}{3}{2}%
  \TestDiv{9}{3}{3}%
  \TestDiv{10}{3}{3}%
  \TestDiv{11}{3}{3}%
  \TestDiv{12}{3}{4}%
  \TestDiv{13}{3}{4}%
  \TestDiv{14}{3}{4}%
  \TestDiv{15}{3}{5}%
  \TestDiv{16}{3}{5}%
  \TestDiv{17}{3}{5}%
  \TestDiv{18}{3}{6}%
  \TestDiv{19}{3}{6}%
  \TestDiv{20}{3}{6}%
  \TestDiv{21}{3}{7}%
  \TestDiv{22}{3}{7}%
  \TestDiv{23}{3}{7}%
  \TestDiv{24}{3}{8}%
  \TestDiv{25}{3}{8}%
  \TestDiv{26}{3}{8}%
  \TestDiv{27}{3}{9}%
  \TestDiv{28}{3}{9}%
  \TestDiv{29}{3}{9}%
  \TestDiv{30}{3}{10}%
  \TestDiv{31}{3}{10}%
  \TestDivBig{17363436332507}{24702}{702916214}%
\end{qstest}

\begin{qstest}{mod}{mod}
  \TestMod{-6}{5}{4}%
  \TestMod{-5}{5}{0}%
  \TestMod{-4}{5}{1}%
  \TestMod{-3}{5}{2}%
  \TestMod{-2}{5}{3}%
  \TestMod{-1}{5}{4}%
  \TestMod{0}{5}{0}%
  \TestMod{1}{5}{1}%
  \TestMod{2}{5}{2}%
  \TestMod{3}{5}{3}%
  \TestMod{4}{5}{4}%
  \TestMod{5}{5}{0}%
  \TestMod{6}{5}{1}%
  \TestMod{-5}{4}{3}%
  \TestMod{-4}{4}{0}%
  \TestMod{-3}{4}{1}%
  \TestMod{-2}{4}{2}%
  \TestMod{-1}{4}{3}%
  \TestMod{0}{4}{0}%
  \TestMod{1}{4}{1}%
  \TestMod{2}{4}{2}%
  \TestMod{3}{4}{3}%
  \TestMod{4}{4}{0}%
  \TestMod{5}{4}{1}%
  \TestMod{-6}{-5}{-1}%
  \TestMod{-5}{-5}{0}%
  \TestMod{-4}{-5}{-4}%
  \TestMod{-3}{-5}{-3}%
  \TestMod{-2}{-5}{-2}%
  \TestMod{-1}{-5}{-1}%
  \TestMod{0}{-5}{0}%
  \TestMod{1}{-5}{-4}%
  \TestMod{2}{-5}{-3}%
  \TestMod{3}{-5}{-2}%
  \TestMod{4}{-5}{-1}%
  \TestMod{5}{-5}{0}%
  \TestMod{6}{-5}{-4}%
  \TestMod{-5}{-4}{-1}%
  \TestMod{-4}{-4}{0}%
  \TestMod{-3}{-4}{-3}%
  \TestMod{-2}{-4}{-2}%
  \TestMod{-1}{-4}{-1}%
  \TestMod{0}{-4}{0}%
  \TestMod{1}{-4}{-3}%
  \TestMod{2}{-4}{-2}%
  \TestMod{3}{-4}{-1}%
  \TestMod{4}{-4}{0}%
  \TestMod{5}{-4}{-3}%
  \TestMod{2147483647}{1000}{647}%
  \TestMod{2147483647}{-1000}{-353}%
  \TestMod{-2147483647}{1000}{353}%
  \TestMod{-2147483647}{-1000}{-647}%
  \TestMod{ 0 }{ 4 }{0}%
  \TestMod{ 1 }{ 4 }{1}%
  \TestMod{ -1 }{ 4 }{3}%
  \TestMod{ 0 }{ -4 }{0}%
  \TestMod{ 1 }{ -4 }{-3}%
  \TestMod{ -1 }{ -4 }{-1}%
  \TestMod{18362}{25}{12}%
\end{qstest}

\newcommand*{\TestError}[2]{%
  \begingroup
    \expandafter\def\csname BigIntCalcError:#1\endcsname{}%
    \Expect*{#2}{0}%
    \expandafter\def\csname BigIntCalcError:#1\endcsname{ERROR}%
    \Expect*{#2}{0ERROR}%
  \endgroup
}
\begin{qstest}{error}{error}
  \TestError{FacNegative}{\bigintcalcFac{-1}}%
  \TestError{FacNegative}{\bigintcalcFac{-2147483647}}%
  \TestError{DivisionByZero}{\bigintcalcPow{0}{-1}}%
  \TestError{DivisionByZero}{\bigintcalcDiv{1}{0}}%
  \TestError{DivisionByZero}{\bigintcalcMod{1}{0}}%
\end{qstest}

\begin{document}
\end{document}
%</test2>
%    \end{macrocode}
%
% \section{Installation}
%
% \subsection{Download}
%
% \paragraph{Package.} This package is available on
% CTAN\footnote{\url{http://ctan.org/pkg/bigintcalc}}:
% \begin{description}
% \item[\CTAN{macros/latex/contrib/oberdiek/bigintcalc.dtx}] The source file.
% \item[\CTAN{macros/latex/contrib/oberdiek/bigintcalc.pdf}] Documentation.
% \end{description}
%
%
% \paragraph{Bundle.} All the packages of the bundle `oberdiek'
% are also available in a TDS compliant ZIP archive. There
% the packages are already unpacked and the documentation files
% are generated. The files and directories obey the TDS standard.
% \begin{description}
% \item[\CTAN{install/macros/latex/contrib/oberdiek.tds.zip}]
% \end{description}
% \emph{TDS} refers to the standard ``A Directory Structure
% for \TeX\ Files'' (\CTAN{tds/tds.pdf}). Directories
% with \xfile{texmf} in their name are usually organized this way.
%
% \subsection{Bundle installation}
%
% \paragraph{Unpacking.} Unpack the \xfile{oberdiek.tds.zip} in the
% TDS tree (also known as \xfile{texmf} tree) of your choice.
% Example (linux):
% \begin{quote}
%   |unzip oberdiek.tds.zip -d ~/texmf|
% \end{quote}
%
% \paragraph{Script installation.}
% Check the directory \xfile{TDS:scripts/oberdiek/} for
% scripts that need further installation steps.
% Package \xpackage{attachfile2} comes with the Perl script
% \xfile{pdfatfi.pl} that should be installed in such a way
% that it can be called as \texttt{pdfatfi}.
% Example (linux):
% \begin{quote}
%   |chmod +x scripts/oberdiek/pdfatfi.pl|\\
%   |cp scripts/oberdiek/pdfatfi.pl /usr/local/bin/|
% \end{quote}
%
% \subsection{Package installation}
%
% \paragraph{Unpacking.} The \xfile{.dtx} file is a self-extracting
% \docstrip\ archive. The files are extracted by running the
% \xfile{.dtx} through \plainTeX:
% \begin{quote}
%   \verb|tex bigintcalc.dtx|
% \end{quote}
%
% \paragraph{TDS.} Now the different files must be moved into
% the different directories in your installation TDS tree
% (also known as \xfile{texmf} tree):
% \begin{quote}
% \def\t{^^A
% \begin{tabular}{@{}>{\ttfamily}l@{ $\rightarrow$ }>{\ttfamily}l@{}}
%   bigintcalc.sty & tex/generic/oberdiek/bigintcalc.sty\\
%   bigintcalc.pdf & doc/latex/oberdiek/bigintcalc.pdf\\
%   test/bigintcalc-test1.tex & doc/latex/oberdiek/test/bigintcalc-test1.tex\\
%   test/bigintcalc-test2.tex & doc/latex/oberdiek/test/bigintcalc-test2.tex\\
%   test/bigintcalc-test3.tex & doc/latex/oberdiek/test/bigintcalc-test3.tex\\
%   bigintcalc.dtx & source/latex/oberdiek/bigintcalc.dtx\\
% \end{tabular}^^A
% }^^A
% \sbox0{\t}^^A
% \ifdim\wd0>\linewidth
%   \begingroup
%     \advance\linewidth by\leftmargin
%     \advance\linewidth by\rightmargin
%   \edef\x{\endgroup
%     \def\noexpand\lw{\the\linewidth}^^A
%   }\x
%   \def\lwbox{^^A
%     \leavevmode
%     \hbox to \linewidth{^^A
%       \kern-\leftmargin\relax
%       \hss
%       \usebox0
%       \hss
%       \kern-\rightmargin\relax
%     }^^A
%   }^^A
%   \ifdim\wd0>\lw
%     \sbox0{\small\t}^^A
%     \ifdim\wd0>\linewidth
%       \ifdim\wd0>\lw
%         \sbox0{\footnotesize\t}^^A
%         \ifdim\wd0>\linewidth
%           \ifdim\wd0>\lw
%             \sbox0{\scriptsize\t}^^A
%             \ifdim\wd0>\linewidth
%               \ifdim\wd0>\lw
%                 \sbox0{\tiny\t}^^A
%                 \ifdim\wd0>\linewidth
%                   \lwbox
%                 \else
%                   \usebox0
%                 \fi
%               \else
%                 \lwbox
%               \fi
%             \else
%               \usebox0
%             \fi
%           \else
%             \lwbox
%           \fi
%         \else
%           \usebox0
%         \fi
%       \else
%         \lwbox
%       \fi
%     \else
%       \usebox0
%     \fi
%   \else
%     \lwbox
%   \fi
% \else
%   \usebox0
% \fi
% \end{quote}
% If you have a \xfile{docstrip.cfg} that configures and enables \docstrip's
% TDS installing feature, then some files can already be in the right
% place, see the documentation of \docstrip.
%
% \subsection{Refresh file name databases}
%
% If your \TeX~distribution
% (\teTeX, \mikTeX, \dots) relies on file name databases, you must refresh
% these. For example, \teTeX\ users run \verb|texhash| or
% \verb|mktexlsr|.
%
% \subsection{Some details for the interested}
%
% \paragraph{Attached source.}
%
% The PDF documentation on CTAN also includes the
% \xfile{.dtx} source file. It can be extracted by
% AcrobatReader 6 or higher. Another option is \textsf{pdftk},
% e.g. unpack the file into the current directory:
% \begin{quote}
%   \verb|pdftk bigintcalc.pdf unpack_files output .|
% \end{quote}
%
% \paragraph{Unpacking with \LaTeX.}
% The \xfile{.dtx} chooses its action depending on the format:
% \begin{description}
% \item[\plainTeX:] Run \docstrip\ and extract the files.
% \item[\LaTeX:] Generate the documentation.
% \end{description}
% If you insist on using \LaTeX\ for \docstrip\ (really,
% \docstrip\ does not need \LaTeX), then inform the autodetect routine
% about your intention:
% \begin{quote}
%   \verb|latex \let\install=y\input{bigintcalc.dtx}|
% \end{quote}
% Do not forget to quote the argument according to the demands
% of your shell.
%
% \paragraph{Generating the documentation.}
% You can use both the \xfile{.dtx} or the \xfile{.drv} to generate
% the documentation. The process can be configured by the
% configuration file \xfile{ltxdoc.cfg}. For instance, put this
% line into this file, if you want to have A4 as paper format:
% \begin{quote}
%   \verb|\PassOptionsToClass{a4paper}{article}|
% \end{quote}
% An example follows how to generate the
% documentation with pdf\LaTeX:
% \begin{quote}
%\begin{verbatim}
%pdflatex bigintcalc.dtx
%makeindex -s gind.ist bigintcalc.idx
%pdflatex bigintcalc.dtx
%makeindex -s gind.ist bigintcalc.idx
%pdflatex bigintcalc.dtx
%\end{verbatim}
% \end{quote}
%
% \section{Catalogue}
%
% The following XML file can be used as source for the
% \href{http://mirror.ctan.org/help/Catalogue/catalogue.html}{\TeX\ Catalogue}.
% The elements \texttt{caption} and \texttt{description} are imported
% from the original XML file from the Catalogue.
% The name of the XML file in the Catalogue is \xfile{bigintcalc.xml}.
%    \begin{macrocode}
%<*catalogue>
<?xml version='1.0' encoding='us-ascii'?>
<!DOCTYPE entry SYSTEM 'catalogue.dtd'>
<entry datestamp='$Date$' modifier='$Author$' id='bigintcalc'>
  <name>bigintcalc</name>
  <caption>Integer calculations on very large numbers.</caption>
  <authorref id='auth:oberdiek'/>
  <copyright owner='Heiko Oberdiek' year='2007,2011,2012'/>
  <license type='lppl1.3'/>
  <version number='1.4'/>
  <description>
    This package provides expandable arithmetic operations
    with big integers that can exceed TeX's number limits.
    <p/>
    The package is part of the <xref refid='oberdiek'>oberdiek</xref> bundle.
  </description>
  <documentation details='Package documentation'
      href='ctan:/macros/latex/contrib/oberdiek/bigintcalc.pdf'/>
  <ctan file='true' path='/macros/latex/contrib/oberdiek/bigintcalc.dtx'/>
  <miktex location='oberdiek'/>
  <texlive location='oberdiek'/>
  <install path='/macros/latex/contrib/oberdiek/oberdiek.tds.zip'/>
</entry>
%</catalogue>
%    \end{macrocode}
%
% \begin{History}
%   \begin{Version}{2007/09/27 v1.0}
%   \item
%     First version.
%   \end{Version}
%   \begin{Version}{2007/11/11 v1.1}
%   \item
%     Use of package \xpackage{pdftexcmds} for \LuaTeX\ support.
%   \end{Version}
%   \begin{Version}{2011/01/30 v1.2}
%   \item
%     Already loaded package files are not input in \hologo{plainTeX}.
%   \end{Version}
%   \begin{Version}{2012/04/08 v1.3}
%   \item
%     Fix: \xpackage{pdftexcmds} wasn't loaded in case of \hologo{LaTeX}.
%   \end{Version}
%   \begin{Version}{2016/05/16 v1.4}
%   \item
%     Documentation updates.
%   \end{Version}
% \end{History}
%
% \PrintIndex
%
% \Finale
\endinput

%        (quote the arguments according to the demands of your shell)
%
% Documentation:
%    (a) If bigintcalc.drv is present:
%           latex bigintcalc.drv
%    (b) Without bigintcalc.drv:
%           latex bigintcalc.dtx; ...
%    The class ltxdoc loads the configuration file ltxdoc.cfg
%    if available. Here you can specify further options, e.g.
%    use A4 as paper format:
%       \PassOptionsToClass{a4paper}{article}
%
%    Programm calls to get the documentation (example):
%       pdflatex bigintcalc.dtx
%       makeindex -s gind.ist bigintcalc.idx
%       pdflatex bigintcalc.dtx
%       makeindex -s gind.ist bigintcalc.idx
%       pdflatex bigintcalc.dtx
%
% Installation:
%    TDS:tex/generic/oberdiek/bigintcalc.sty
%    TDS:doc/latex/oberdiek/bigintcalc.pdf
%    TDS:doc/latex/oberdiek/test/bigintcalc-test1.tex
%    TDS:doc/latex/oberdiek/test/bigintcalc-test2.tex
%    TDS:doc/latex/oberdiek/test/bigintcalc-test3.tex
%    TDS:source/latex/oberdiek/bigintcalc.dtx
%
%<*ignore>
\begingroup
  \catcode123=1 %
  \catcode125=2 %
  \def\x{LaTeX2e}%
\expandafter\endgroup
\ifcase 0\ifx\install y1\fi\expandafter
         \ifx\csname processbatchFile\endcsname\relax\else1\fi
         \ifx\fmtname\x\else 1\fi\relax
\else\csname fi\endcsname
%</ignore>
%<*install>
\input docstrip.tex
\Msg{************************************************************************}
\Msg{* Installation}
\Msg{* Package: bigintcalc 2016/05/16 v1.4 Expandable calculations on big integers (HO)}
\Msg{************************************************************************}

\keepsilent
\askforoverwritefalse

\let\MetaPrefix\relax
\preamble

This is a generated file.

Project: bigintcalc
Version: 2016/05/16 v1.4

Copyright (C) 2007, 2011, 2012 by
   Heiko Oberdiek <heiko.oberdiek at googlemail.com>

This work may be distributed and/or modified under the
conditions of the LaTeX Project Public License, either
version 1.3c of this license or (at your option) any later
version. This version of this license is in
   http://www.latex-project.org/lppl/lppl-1-3c.txt
and the latest version of this license is in
   http://www.latex-project.org/lppl.txt
and version 1.3 or later is part of all distributions of
LaTeX version 2005/12/01 or later.

This work has the LPPL maintenance status "maintained".

This Current Maintainer of this work is Heiko Oberdiek.

The Base Interpreter refers to any `TeX-Format',
because some files are installed in TDS:tex/generic//.

This work consists of the main source file bigintcalc.dtx
and the derived files
   bigintcalc.sty, bigintcalc.pdf, bigintcalc.ins, bigintcalc.drv,
   bigintcalc-test1.tex, bigintcalc-test2.tex,
   bigintcalc-test3.tex.

\endpreamble
\let\MetaPrefix\DoubleperCent

\generate{%
  \file{bigintcalc.ins}{\from{bigintcalc.dtx}{install}}%
  \file{bigintcalc.drv}{\from{bigintcalc.dtx}{driver}}%
  \usedir{tex/generic/oberdiek}%
  \file{bigintcalc.sty}{\from{bigintcalc.dtx}{package}}%
%  \usedir{doc/latex/oberdiek/test}%
%  \file{bigintcalc-test1.tex}{\from{bigintcalc.dtx}{test1}}%
%  \file{bigintcalc-test2.tex}{\from{bigintcalc.dtx}{test2,etex}}%
%  \file{bigintcalc-test3.tex}{\from{bigintcalc.dtx}{test2,noetex}}%
  \nopreamble
  \nopostamble
  \usedir{source/latex/oberdiek/catalogue}%
  \file{bigintcalc.xml}{\from{bigintcalc.dtx}{catalogue}}%
}

\catcode32=13\relax% active space
\let =\space%
\Msg{************************************************************************}
\Msg{*}
\Msg{* To finish the installation you have to move the following}
\Msg{* file into a directory searched by TeX:}
\Msg{*}
\Msg{*     bigintcalc.sty}
\Msg{*}
\Msg{* To produce the documentation run the file `bigintcalc.drv'}
\Msg{* through LaTeX.}
\Msg{*}
\Msg{* Happy TeXing!}
\Msg{*}
\Msg{************************************************************************}

\endbatchfile
%</install>
%<*ignore>
\fi
%</ignore>
%<*driver>
\NeedsTeXFormat{LaTeX2e}
\ProvidesFile{bigintcalc.drv}%
  [2016/05/16 v1.4 Expandable calculations on big integers (HO)]%
\documentclass{ltxdoc}
\usepackage{wasysym}
\let\iint\relax
\let\iiint\relax
\usepackage[fleqn]{amsmath}

\DeclareMathOperator{\opInv}{Inv}
\DeclareMathOperator{\opAbs}{Abs}
\DeclareMathOperator{\opSgn}{Sgn}
\DeclareMathOperator{\opMin}{Min}
\DeclareMathOperator{\opMax}{Max}
\DeclareMathOperator{\opCmp}{Cmp}
\DeclareMathOperator{\opOdd}{Odd}
\DeclareMathOperator{\opInc}{Inc}
\DeclareMathOperator{\opDec}{Dec}
\DeclareMathOperator{\opAdd}{Add}
\DeclareMathOperator{\opSub}{Sub}
\DeclareMathOperator{\opShl}{Shl}
\DeclareMathOperator{\opShr}{Shr}
\DeclareMathOperator{\opMul}{Mul}
\DeclareMathOperator{\opSqr}{Sqr}
\DeclareMathOperator{\opFac}{Fac}
\DeclareMathOperator{\opPow}{Pow}
\DeclareMathOperator{\opDiv}{Div}
\DeclareMathOperator{\opMod}{Mod}
\DeclareMathOperator{\opInt}{Int}
\DeclareMathOperator{\opODD}{ifodd}

\newcommand*{\Def}{%
  \ensuremath{%
    \mathrel{\mathop{:}}=%
  }%
}
\newcommand*{\op}[1]{%
  \textsf{#1}%
}
\usepackage{holtxdoc}[2011/11/22]
\begin{document}
  \DocInput{bigintcalc.dtx}%
\end{document}
%</driver>
% \fi
%
%
% \CharacterTable
%  {Upper-case    \A\B\C\D\E\F\G\H\I\J\K\L\M\N\O\P\Q\R\S\T\U\V\W\X\Y\Z
%   Lower-case    \a\b\c\d\e\f\g\h\i\j\k\l\m\n\o\p\q\r\s\t\u\v\w\x\y\z
%   Digits        \0\1\2\3\4\5\6\7\8\9
%   Exclamation   \!     Double quote  \"     Hash (number) \#
%   Dollar        \$     Percent       \%     Ampersand     \&
%   Acute accent  \'     Left paren    \(     Right paren   \)
%   Asterisk      \*     Plus          \+     Comma         \,
%   Minus         \-     Point         \.     Solidus       \/
%   Colon         \:     Semicolon     \;     Less than     \<
%   Equals        \=     Greater than  \>     Question mark \?
%   Commercial at \@     Left bracket  \[     Backslash     \\
%   Right bracket \]     Circumflex    \^     Underscore    \_
%   Grave accent  \`     Left brace    \{     Vertical bar  \|
%   Right brace   \}     Tilde         \~}
%
% \GetFileInfo{bigintcalc.drv}
%
% \title{The \xpackage{bigintcalc} package}
% \date{2016/05/16 v1.4}
% \author{Heiko Oberdiek\thanks
% {Please report any issues at https://github.com/ho-tex/oberdiek/issues}\\
% \xemail{heiko.oberdiek at googlemail.com}}
%
% \maketitle
%
% \begin{abstract}
% This package provides expandable arithmetic operations
% with big integers that can exceed \TeX's number limits.
% \end{abstract}
%
% \tableofcontents
%
% \section{Documentation}
%
% \subsection{Introduction}
%
% Package \xpackage{bigintcalc} defines arithmetic operations
% that deal with big integers. Big integers can be given
% either as explicit integer number or as macro code
% that expands to an explicit number. \emph{Big} means that
% there is no limit on the size of the number. Big integers
% may exceed \TeX's range limitation of -2147483647 and 2147483647.
% Only memory issues will limit the usable range.
%
% In opposite to package \xpackage{intcalc}
% unexpandable command tokens are not supported, even if
% they are valid \TeX\ numbers like count registers or
% commands created by \cs{chardef}. Nevertheless they may
% be used, if they are prefixed by \cs{number}.
%
% Also \eTeX's \cs{numexpr} expressions are not supported
% directly in the manner of package \xpackage{intcalc}.
% However they can be given if \cs{the}\cs{numexpr}
% or \cs{number}\cs{numexpr} are used.
%
% The operations have the form of macros that take one or
% two integers as parameter and return the integer result.
% The macro name is a three letter operation name prefixed
% by the package name, e.g. \cs{bigintcalcAdd}|{10}{43}| returns
% |53|.
%
% The macros are fully expandable, exactly two expansion
% steps generate the result. Therefore the operations
% may be used nearly everywhere in \TeX, even inside
% \cs{csname}, file names,  or other
% expandable contexts.
%
% \subsection{Conditions}
%
% \subsubsection{Preconditions}
%
% \begin{itemize}
% \item
%   Arguments can be anything that expands to a number that consists
%   of optional signs and digits.
% \item
%   The arguments and return values must be sound.
%   Zero as divisor or factorials of negative numbers will cause errors.
% \end{itemize}
%
% \subsubsection{Postconditions}
%
% Additional properties of the macros apart from calculating
% a correct result (of course \smiley):
% \begin{itemize}
% \item
%   The macros are fully expandable. Thus they can
%   be used inside \cs{edef}, \cs{csname},
%   for example.
% \item
%   Furthermore exactly two expansion steps calculate the result.
% \item
%   The number consists of one optional minus sign and one or more
%   digits. The first digit is larger than zero for
%   numbers that consists of more than one digit.
%
%   In short, the number format is exactly the same as
%   \cs{number} generates, but without its range limitation.
%   And the tokens (minus sign, digits)
%   have catcode 12 (other).
% \item
%   Call by value is simulated. First the arguments are
%   converted to numbers. Then these numbers are used
%   in the calculations.
%
%   Remember that arguments
%   may contain expensive macros or \eTeX\ expressions.
%   This strategy avoids multiple evaluations of such
%   arguments.
% \end{itemize}
%
% \subsection{Error handling}
%
% Some errors are detected by the macros, example: division by zero.
% In this cases an undefined control sequence is called and causes
% a TeX error message, example: \cs{BigIntCalcError:DivisionByZero}.
% The name of the control sequence contains
% the reason for the error. The \TeX\ error may be ignored.
% Then the operation returns zero as result.
% Because the macros are supposed to work in expandible contexts.
% An traditional error message, however, is not expandable and
% would break these contexts.
%
% \subsection{Operations}
%
% Some definition equations below use the function $\opInt$
% that converts a real number to an integer. The number
% is truncated that means rounding to zero:
% \begin{gather*}
%   \opInt(x) \Def
%   \begin{cases}
%     \lfloor x\rfloor & \text{if $x\geq0$}\\
%     \lceil x\rceil & \text{otherwise}
%   \end{cases}
% \end{gather*}
%
% \subsubsection{\op{Num}}
%
% \begin{declcs}{bigintcalcNum} \M{x}
% \end{declcs}
% Macro \cs{bigintcalcNum} converts its argument to a normalized integer
% number without unnecessary leading zeros or signs.
% The result matches the regular expression:
%\begin{quote}
%\begin{verbatim}
%0|-?[1-9][0-9]*
%\end{verbatim}
%\end{quote}
%
% \subsubsection{\op{Inv}, \op{Abs}, \op{Sgn}}
%
% \begin{declcs}{bigintcalcInv} \M{x}
% \end{declcs}
% Macro \cs{bigintcalcInv} switches the sign.
% \begin{gather*}
%   \opInv(x) \Def -x
% \end{gather*}
%
% \begin{declcs}{bigintcalcAbs} \M{x}
% \end{declcs}
% Macro \cs{bigintcalcAbs} returns the absolute value
% of integer \meta{x}.
% \begin{gather*}
%   \opAbs(x) \Def \vert x\vert
% \end{gather*}
%
% \begin{declcs}{bigintcalcSgn} \M{x}
% \end{declcs}
% Macro \cs{bigintcalcSgn} encodes the sign of \meta{x} as number.
% \begin{gather*}
%   \opSgn(x) \Def
%   \begin{cases}
%     -1& \text{if $x<0$}\\
%      0& \text{if $x=0$}\\
%      1& \text{if $x>0$}
%   \end{cases}
% \end{gather*}
% These return values can easily be distinguished by \cs{ifcase}:
%\begin{quote}
%\begin{verbatim}
%\ifcase\bigintcalcSgn{<x>}
%  $x=0$
%\or
%  $x>0$
%\else
%  $x<0$
%\fi
%\end{verbatim}
%\end{quote}
%
% \subsubsection{\op{Min}, \op{Max}, \op{Cmp}}
%
% \begin{declcs}{bigintcalcMin} \M{x} \M{y}
% \end{declcs}
% Macro \cs{bigintcalcMin} returns the smaller of the two integers.
% \begin{gather*}
%   \opMin(x,y) \Def
%   \begin{cases}
%     x & \text{if $x<y$}\\
%     y & \text{otherwise}
%   \end{cases}
% \end{gather*}
%
% \begin{declcs}{bigintcalcMax} \M{x} \M{y}
% \end{declcs}
% Macro \cs{bigintcalcMax} returns the larger of the two integers.
% \begin{gather*}
%   \opMax(x,y) \Def
%   \begin{cases}
%     x & \text{if $x>y$}\\
%     y & \text{otherwise}
%   \end{cases}
% \end{gather*}
%
% \begin{declcs}{bigintcalcCmp} \M{x} \M{y}
% \end{declcs}
% Macro \cs{bigintcalcCmp} encodes the comparison result as number:
% \begin{gather*}
%   \opCmp(x,y) \Def
%   \begin{cases}
%     -1 & \text{if $x<y$}\\
%      0 & \text{if $x=y$}\\
%      1 & \text{if $x>y$}
%   \end{cases}
% \end{gather*}
% These values can be distinguished by \cs{ifcase}:
%\begin{quote}
%\begin{verbatim}
%\ifcase\bigintcalcCmp{<x>}{<y>}
%  $x=y$
%\or
%  $x>y$
%\else
%  $x<y$
%\fi
%\end{verbatim}
%\end{quote}
%
% \subsubsection{\op{Odd}}
%
% \begin{declcs}{bigintcalcOdd} \M{x}
% \end{declcs}
% \begin{gather*}
%   \opOdd(x) \Def
%   \begin{cases}
%     1 & \text{if $x$ is odd}\\
%     0 & \text{if $x$ is even}
%   \end{cases}
% \end{gather*}
%
% \subsubsection{\op{Inc}, \op{Dec}, \op{Add}, \op{Sub}}
%
% \begin{declcs}{bigintcalcInc} \M{x}
% \end{declcs}
% Macro \cs{bigintcalcInc} increments \meta{x} by one.
% \begin{gather*}
%   \opInc(x) \Def x + 1
% \end{gather*}
%
% \begin{declcs}{bigintcalcDec} \M{x}
% \end{declcs}
% Macro \cs{bigintcalcDec} decrements \meta{x} by one.
% \begin{gather*}
%   \opDec(x) \Def x - 1
% \end{gather*}
%
% \begin{declcs}{bigintcalcAdd} \M{x} \M{y}
% \end{declcs}
% Macro \cs{bigintcalcAdd} adds the two numbers.
% \begin{gather*}
%   \opAdd(x, y) \Def x + y
% \end{gather*}
%
% \begin{declcs}{bigintcalcSub} \M{x} \M{y}
% \end{declcs}
% Macro \cs{bigintcalcSub} calculates the difference.
% \begin{gather*}
%   \opSub(x, y) \Def x - y
% \end{gather*}
%
% \subsubsection{\op{Shl}, \op{Shr}}
%
% \begin{declcs}{bigintcalcShl} \M{x}
% \end{declcs}
% Macro \cs{bigintcalcShl} implements shifting to the left that
% means the number is multiplied by two.
% The sign is preserved.
% \begin{gather*}
%   \opShl(x) \Def x*2
% \end{gather*}
%
% \begin{declcs}{bigintcalcShr} \M{x}
% \end{declcs}
% Macro \cs{bigintcalcShr} implements shifting to the right.
% That is equivalent to an integer division by two.
% The sign is preserved.
% \begin{gather*}
%   \opShr(x) \Def \opInt(x/2)
% \end{gather*}
%
% \subsubsection{\op{Mul}, \op{Sqr}, \op{Fac}, \op{Pow}}
%
% \begin{declcs}{bigintcalcMul} \M{x} \M{y}
% \end{declcs}
% Macro \cs{bigintcalcMul} calculates the product of
% \meta{x} and \meta{y}.
% \begin{gather*}
%   \opMul(x,y) \Def x*y
% \end{gather*}
%
% \begin{declcs}{bigintcalcSqr} \M{x}
% \end{declcs}
% Macro \cs{bigintcalcSqr} returns the square product.
% \begin{gather*}
%   \opSqr(x) \Def x^2
% \end{gather*}
%
% \begin{declcs}{bigintcalcFac} \M{x}
% \end{declcs}
% Macro \cs{bigintcalcFac} returns the factorial of \meta{x}.
% Negative numbers are not permitted.
% \begin{gather*}
%   \opFac(x) \Def x!\qquad\text{for $x\geq0$}
% \end{gather*}
% ($0! = 1$)
%
% \begin{declcs}{bigintcalcPow} M{x} M{y}
% \end{declcs}
% Macro \cs{bigintcalcPow} calculates the value of \meta{x} to the
% power of \meta{y}. The error ``division by zero'' is thrown
% if \meta{x} is zero and \meta{y} is negative.
% permitted:
% \begin{gather*}
%   \opPow(x,y) \Def
%   \opInt(x^y)\qquad\text{for $x\neq0$ or $y\geq0$}
% \end{gather*}
% ($0^0 = 1$)
%
% \subsubsection{\op{Div}, \op{Mul}}
%
% \begin{declcs}{bigintcalcDiv} \M{x} \M{y}
% \end{declcs}
% Macro \cs{bigintcalcDiv} performs an integer division.
% Argument \meta{y} must not be zero.
% \begin{gather*}
%   \opDiv(x,y) \Def \opInt(x/y)\qquad\text{for $y\neq0$}
% \end{gather*}
%
% \begin{declcs}{bigintcalcMod} \M{x} \M{y}
% \end{declcs}
% Macro \cs{bigintcalcMod} gets the remainder of the integer
% division. The sign follows the divisor \meta{y}.
% Argument \meta{y} must not be zero.
% \begin{gather*}
%   \opMod(x,y) \Def x\mathrel{\%}y\qquad\text{for $y\neq0$}
% \end{gather*}
% The result ranges:
% \begin{gather*}
%   -\vert y\vert < \opMod(x,y) \leq 0\qquad\text{for $y<0$}\\
%   0 \leq \opMod(x,y) < y\qquad\text{for $y\geq0$}
% \end{gather*}
%
% \subsection{Interface for programmers}
%
% If the programmer can ensure some more properties about
% the arguments of the operations, then the following
% macros are a little more efficient.
%
% In general numbers must obey the following constraints:
% \begin{itemize}
% \item Plain number: digit tokens only, no command tokens.
% \item Non-negative. Signs are forbidden.
% \item Delimited by exclamation mark. Curly braces
%       around the number are not allowed and will
%       break the code.
% \end{itemize}
%
% \begin{declcs}{BigIntCalcOdd} \meta{number} |!|
% \end{declcs}
% |1|/|0| is returned if \meta{number} is odd/even.
%
% \begin{declcs}{BigIntCalcInc} \meta{number} |!|
% \end{declcs}
% Incrementation.
%
% \begin{declcs}{BigIntCalcDec} \meta{number} |!|
% \end{declcs}
% Decrementation, positive number without zero.
%
% \begin{declcs}{BigIntCalcAdd} \meta{number A} |!| \meta{number B} |!|
% \end{declcs}
% Addition, $A\geq B$.
%
% \begin{declcs}{BigIntCalcSub} \meta{number A} |!| \meta{number B} |!|
% \end{declcs}
% Subtraction, $A\geq B$.
%
% \begin{declcs}{BigIntCalcShl} \meta{number} |!|
% \end{declcs}
% Left shift (multiplication with two).
%
% \begin{declcs}{BigIntCalcShr} \meta{number} |!|
% \end{declcs}
% Right shift (integer division by two).
%
% \begin{declcs}{BigIntCalcMul} \meta{number A} |!| \meta{number B} |!|
% \end{declcs}
% Multiplication, $A\geq B$.
%
% \begin{declcs}{BigIntCalcDiv} \meta{number A} |!| \meta{number B} |!|
% \end{declcs}
% Division operation.
%
% \begin{declcs}{BigIntCalcMod} \meta{number A} |!| \meta{number B} |!|
% \end{declcs}
% Modulo operation.
%
%
% \StopEventually{
% }
%
% \section{Implementation}
%
%    \begin{macrocode}
%<*package>
%    \end{macrocode}
%
% \subsection{Reload check and package identification}
%    Reload check, especially if the package is not used with \LaTeX.
%    \begin{macrocode}
\begingroup\catcode61\catcode48\catcode32=10\relax%
  \catcode13=5 % ^^M
  \endlinechar=13 %
  \catcode35=6 % #
  \catcode39=12 % '
  \catcode44=12 % ,
  \catcode45=12 % -
  \catcode46=12 % .
  \catcode58=12 % :
  \catcode64=11 % @
  \catcode123=1 % {
  \catcode125=2 % }
  \expandafter\let\expandafter\x\csname ver@bigintcalc.sty\endcsname
  \ifx\x\relax % plain-TeX, first loading
  \else
    \def\empty{}%
    \ifx\x\empty % LaTeX, first loading,
      % variable is initialized, but \ProvidesPackage not yet seen
    \else
      \expandafter\ifx\csname PackageInfo\endcsname\relax
        \def\x#1#2{%
          \immediate\write-1{Package #1 Info: #2.}%
        }%
      \else
        \def\x#1#2{\PackageInfo{#1}{#2, stopped}}%
      \fi
      \x{bigintcalc}{The package is already loaded}%
      \aftergroup\endinput
    \fi
  \fi
\endgroup%
%    \end{macrocode}
%    Package identification:
%    \begin{macrocode}
\begingroup\catcode61\catcode48\catcode32=10\relax%
  \catcode13=5 % ^^M
  \endlinechar=13 %
  \catcode35=6 % #
  \catcode39=12 % '
  \catcode40=12 % (
  \catcode41=12 % )
  \catcode44=12 % ,
  \catcode45=12 % -
  \catcode46=12 % .
  \catcode47=12 % /
  \catcode58=12 % :
  \catcode64=11 % @
  \catcode91=12 % [
  \catcode93=12 % ]
  \catcode123=1 % {
  \catcode125=2 % }
  \expandafter\ifx\csname ProvidesPackage\endcsname\relax
    \def\x#1#2#3[#4]{\endgroup
      \immediate\write-1{Package: #3 #4}%
      \xdef#1{#4}%
    }%
  \else
    \def\x#1#2[#3]{\endgroup
      #2[{#3}]%
      \ifx#1\@undefined
        \xdef#1{#3}%
      \fi
      \ifx#1\relax
        \xdef#1{#3}%
      \fi
    }%
  \fi
\expandafter\x\csname ver@bigintcalc.sty\endcsname
\ProvidesPackage{bigintcalc}%
  [2016/05/16 v1.4 Expandable calculations on big integers (HO)]%
%    \end{macrocode}
%
% \subsection{Catcodes}
%
%    \begin{macrocode}
\begingroup\catcode61\catcode48\catcode32=10\relax%
  \catcode13=5 % ^^M
  \endlinechar=13 %
  \catcode123=1 % {
  \catcode125=2 % }
  \catcode64=11 % @
  \def\x{\endgroup
    \expandafter\edef\csname BIC@AtEnd\endcsname{%
      \endlinechar=\the\endlinechar\relax
      \catcode13=\the\catcode13\relax
      \catcode32=\the\catcode32\relax
      \catcode35=\the\catcode35\relax
      \catcode61=\the\catcode61\relax
      \catcode64=\the\catcode64\relax
      \catcode123=\the\catcode123\relax
      \catcode125=\the\catcode125\relax
    }%
  }%
\x\catcode61\catcode48\catcode32=10\relax%
\catcode13=5 % ^^M
\endlinechar=13 %
\catcode35=6 % #
\catcode64=11 % @
\catcode123=1 % {
\catcode125=2 % }
\def\TMP@EnsureCode#1#2{%
  \edef\BIC@AtEnd{%
    \BIC@AtEnd
    \catcode#1=\the\catcode#1\relax
  }%
  \catcode#1=#2\relax
}
\TMP@EnsureCode{33}{12}% !
\TMP@EnsureCode{36}{14}% $ (comment!)
\TMP@EnsureCode{38}{14}% & (comment!)
\TMP@EnsureCode{40}{12}% (
\TMP@EnsureCode{41}{12}% )
\TMP@EnsureCode{42}{12}% *
\TMP@EnsureCode{43}{12}% +
\TMP@EnsureCode{45}{12}% -
\TMP@EnsureCode{46}{12}% .
\TMP@EnsureCode{47}{12}% /
\TMP@EnsureCode{58}{11}% : (letter!)
\TMP@EnsureCode{60}{12}% <
\TMP@EnsureCode{62}{12}% >
\TMP@EnsureCode{63}{14}% ? (comment!)
\TMP@EnsureCode{91}{12}% [
\TMP@EnsureCode{93}{12}% ]
\edef\BIC@AtEnd{\BIC@AtEnd\noexpand\endinput}
\begingroup\expandafter\expandafter\expandafter\endgroup
\expandafter\ifx\csname BIC@TestMode\endcsname\relax
\else
  \catcode63=9 % ? (ignore)
\fi
? \let\BIC@@TestMode\BIC@TestMode
%    \end{macrocode}
%
% \subsection{\eTeX\ detection}
%
%    \begin{macrocode}
\begingroup\expandafter\expandafter\expandafter\endgroup
\expandafter\ifx\csname numexpr\endcsname\relax
  \catcode36=9 % $ (ignore)
\else
  \catcode38=9 % & (ignore)
\fi
%    \end{macrocode}
%
% \subsection{Help macros}
%
%    \begin{macro}{\BIC@Fi}
%    \begin{macrocode}
\let\BIC@Fi\fi
%    \end{macrocode}
%    \end{macro}
%    \begin{macro}{\BIC@AfterFi}
%    \begin{macrocode}
\def\BIC@AfterFi#1#2\BIC@Fi{\fi#1}%
%    \end{macrocode}
%    \end{macro}
%    \begin{macro}{\BIC@AfterFiFi}
%    \begin{macrocode}
\def\BIC@AfterFiFi#1#2\BIC@Fi{\fi\fi#1}%
%    \end{macrocode}
%    \end{macro}
%    \begin{macro}{\BIC@AfterFiFiFi}
%    \begin{macrocode}
\def\BIC@AfterFiFiFi#1#2\BIC@Fi{\fi\fi\fi#1}%
%    \end{macrocode}
%    \end{macro}
%
%    \begin{macro}{\BIC@Space}
%    \begin{macrocode}
\begingroup
  \def\x#1{\endgroup
    \let\BIC@Space= #1%
  }%
\x{ }
%    \end{macrocode}
%    \end{macro}
%
% \subsection{Expand number}
%
%    \begin{macrocode}
\begingroup\expandafter\expandafter\expandafter\endgroup
\expandafter\ifx\csname RequirePackage\endcsname\relax
  \def\TMP@RequirePackage#1[#2]{%
    \begingroup\expandafter\expandafter\expandafter\endgroup
    \expandafter\ifx\csname ver@#1.sty\endcsname\relax
      \input #1.sty\relax
    \fi
  }%
  \TMP@RequirePackage{pdftexcmds}[2007/11/11]%
\else
  \RequirePackage{pdftexcmds}[2007/11/11]%
\fi
%    \end{macrocode}
%
%    \begin{macrocode}
\begingroup\expandafter\expandafter\expandafter\endgroup
\expandafter\ifx\csname pdf@escapehex\endcsname\relax
%    \end{macrocode}
%
%    \begin{macro}{\BIC@Expand}
%    \begin{macrocode}
  \def\BIC@Expand#1{%
    \romannumeral0%
    \BIC@@Expand#1!\@nil{}%
  }%
%    \end{macrocode}
%    \end{macro}
%    \begin{macro}{\BIC@@Expand}
%    \begin{macrocode}
  \def\BIC@@Expand#1#2\@nil#3{%
    \expandafter\ifcat\noexpand#1\relax
      \expandafter\@firstoftwo
    \else
      \expandafter\@secondoftwo
    \fi
    {%
      \expandafter\BIC@@Expand#1#2\@nil{#3}%
    }{%
      \ifx#1!%
        \expandafter\@firstoftwo
      \else
        \expandafter\@secondoftwo
      \fi
      { #3}{%
        \BIC@@Expand#2\@nil{#3#1}%
      }%
    }%
  }%
%    \end{macrocode}
%    \end{macro}
%    \begin{macro}{\@firstoftwo}
%    \begin{macrocode}
  \expandafter\ifx\csname @firstoftwo\endcsname\relax
    \long\def\@firstoftwo#1#2{#1}%
  \fi
%    \end{macrocode}
%    \end{macro}
%    \begin{macro}{\@secondoftwo}
%    \begin{macrocode}
  \expandafter\ifx\csname @secondoftwo\endcsname\relax
    \long\def\@secondoftwo#1#2{#2}%
  \fi
%    \end{macrocode}
%    \end{macro}
%
%    \begin{macrocode}
\else
%    \end{macrocode}
%    \begin{macro}{\BIC@Expand}
%    \begin{macrocode}
  \def\BIC@Expand#1{%
    \romannumeral0\expandafter\expandafter\expandafter\BIC@Space
    \pdf@unescapehex{%
      \expandafter\expandafter\expandafter
      \BIC@StripHexSpace\pdf@escapehex{#1}20\@nil
    }%
  }%
%    \end{macrocode}
%    \end{macro}
%    \begin{macro}{\BIC@StripHexSpace}
%    \begin{macrocode}
  \def\BIC@StripHexSpace#120#2\@nil{%
    #1%
    \ifx\\#2\\%
    \else
      \BIC@AfterFi{%
        \BIC@StripHexSpace#2\@nil
      }%
    \BIC@Fi
  }%
%    \end{macrocode}
%    \end{macro}
%    \begin{macrocode}
\fi
%    \end{macrocode}
%
% \subsection{Normalize expanded number}
%
%    \begin{macro}{\BIC@Normalize}
%    |#1|: result sign\\
%    |#2|: first token of number
%    \begin{macrocode}
\def\BIC@Normalize#1#2{%
  \ifx#2-%
    \ifx\\#1\\%
      \BIC@AfterFiFi{%
        \BIC@Normalize-%
      }%
    \else
      \BIC@AfterFiFi{%
        \BIC@Normalize{}%
      }%
    \fi
  \else
    \ifx#2+%
      \BIC@AfterFiFi{%
        \BIC@Normalize{#1}%
      }%
    \else
      \ifx#20%
        \BIC@AfterFiFiFi{%
          \BIC@NormalizeZero{#1}%
        }%
      \else
        \BIC@AfterFiFiFi{%
          \BIC@NormalizeDigits#1#2%
        }%
      \fi
    \fi
  \BIC@Fi
}
%    \end{macrocode}
%    \end{macro}
%    \begin{macro}{\BIC@NormalizeZero}
%    \begin{macrocode}
\def\BIC@NormalizeZero#1#2{%
  \ifx#2!%
    \BIC@AfterFi{ 0}%
  \else
    \ifx#20%
      \BIC@AfterFiFi{%
        \BIC@NormalizeZero{#1}%
      }%
    \else
      \BIC@AfterFiFi{%
        \BIC@NormalizeDigits#1#2%
      }%
    \fi
  \BIC@Fi
}
%    \end{macrocode}
%    \end{macro}
%    \begin{macro}{\BIC@NormalizeDigits}
%    \begin{macrocode}
\def\BIC@NormalizeDigits#1!{ #1}
%    \end{macrocode}
%    \end{macro}
%
% \subsection{\op{Num}}
%
%    \begin{macro}{\bigintcalcNum}
%    \begin{macrocode}
\def\bigintcalcNum#1{%
  \romannumeral0%
  \expandafter\expandafter\expandafter\BIC@Normalize
  \expandafter\expandafter\expandafter{%
  \expandafter\expandafter\expandafter}%
  \BIC@Expand{#1}!%
}
%    \end{macrocode}
%    \end{macro}
%
% \subsection{\op{Inv}, \op{Abs}, \op{Sgn}}
%
%
%    \begin{macro}{\bigintcalcInv}
%    \begin{macrocode}
\def\bigintcalcInv#1{%
  \romannumeral0\expandafter\expandafter\expandafter\BIC@Space
  \bigintcalcNum{-#1}%
}
%    \end{macrocode}
%    \end{macro}
%
%    \begin{macro}{\bigintcalcAbs}
%    \begin{macrocode}
\def\bigintcalcAbs#1{%
  \romannumeral0%
  \expandafter\expandafter\expandafter\BIC@Abs
  \bigintcalcNum{#1}%
}
%    \end{macrocode}
%    \end{macro}
%    \begin{macro}{\BIC@Abs}
%    \begin{macrocode}
\def\BIC@Abs#1{%
  \ifx#1-%
    \expandafter\BIC@Space
  \else
    \expandafter\BIC@Space
    \expandafter#1%
  \fi
}
%    \end{macrocode}
%    \end{macro}
%
%    \begin{macro}{\bigintcalcSgn}
%    \begin{macrocode}
\def\bigintcalcSgn#1{%
  \number
  \expandafter\expandafter\expandafter\BIC@Sgn
  \bigintcalcNum{#1}! %
}
%    \end{macrocode}
%    \end{macro}
%    \begin{macro}{\BIC@Sgn}
%    \begin{macrocode}
\def\BIC@Sgn#1#2!{%
  \ifx#1-%
    -1%
  \else
    \ifx#10%
      0%
    \else
      1%
    \fi
  \fi
}
%    \end{macrocode}
%    \end{macro}
%
% \subsection{\op{Cmp}, \op{Min}, \op{Max}}
%
%    \begin{macro}{\bigintcalcCmp}
%    \begin{macrocode}
\def\bigintcalcCmp#1#2{%
  \number
  \expandafter\expandafter\expandafter\BIC@Cmp
  \bigintcalcNum{#2}!{#1}%
}
%    \end{macrocode}
%    \end{macro}
%    \begin{macro}{\BIC@Cmp}
%    \begin{macrocode}
\def\BIC@Cmp#1!#2{%
  \expandafter\expandafter\expandafter\BIC@@Cmp
  \bigintcalcNum{#2}!#1!%
}
%    \end{macrocode}
%    \end{macro}
%    \begin{macro}{\BIC@@Cmp}
%    \begin{macrocode}
\def\BIC@@Cmp#1#2!#3#4!{%
  \ifx#1-%
    \ifx#3-%
      \BIC@AfterFiFi{%
        \BIC@@Cmp#4!#2!%
      }%
    \else
      \BIC@AfterFiFi{%
        -1 %
      }%
    \fi
  \else
    \ifx#3-%
      \BIC@AfterFiFi{%
        1 %
      }%
    \else
      \BIC@AfterFiFi{%
        \BIC@CmpLength#1#2!#3#4!#1#2!#3#4!%
      }%
    \fi
  \BIC@Fi
}
%    \end{macrocode}
%    \end{macro}
%    \begin{macro}{\BIC@PosCmp}
%    \begin{macrocode}
\def\BIC@PosCmp#1!#2!{%
  \BIC@CmpLength#1!#2!#1!#2!%
}
%    \end{macrocode}
%    \end{macro}
%    \begin{macro}{\BIC@CmpLength}
%    \begin{macrocode}
\def\BIC@CmpLength#1#2!#3#4!{%
  \ifx\\#2\\%
    \ifx\\#4\\%
      \BIC@AfterFiFi\BIC@CmpDiff
    \else
      \BIC@AfterFiFi{%
        \BIC@CmpResult{-1}%
      }%
    \fi
  \else
    \ifx\\#4\\%
      \BIC@AfterFiFi{%
        \BIC@CmpResult1%
      }%
    \else
      \BIC@AfterFiFi{%
        \BIC@CmpLength#2!#4!%
      }%
    \fi
  \BIC@Fi
}
%    \end{macrocode}
%    \end{macro}
%    \begin{macro}{\BIC@CmpResult}
%    \begin{macrocode}
\def\BIC@CmpResult#1#2!#3!{#1 }
%    \end{macrocode}
%    \end{macro}
%    \begin{macro}{\BIC@CmpDiff}
%    \begin{macrocode}
\def\BIC@CmpDiff#1#2!#3#4!{%
  \ifnum#1<#3 %
    \BIC@AfterFi{%
      -1 %
    }%
  \else
    \ifnum#1>#3 %
      \BIC@AfterFiFi{%
        1 %
      }%
    \else
      \ifx\\#2\\%
        \BIC@AfterFiFiFi{%
          0 %
        }%
      \else
        \BIC@AfterFiFiFi{%
          \BIC@CmpDiff#2!#4!%
        }%
      \fi
    \fi
  \BIC@Fi
}
%    \end{macrocode}
%    \end{macro}
%
%    \begin{macro}{\bigintcalcMin}
%    \begin{macrocode}
\def\bigintcalcMin#1{%
  \romannumeral0%
  \expandafter\expandafter\expandafter\BIC@MinMax
  \bigintcalcNum{#1}!-!%
}
%    \end{macrocode}
%    \end{macro}
%    \begin{macro}{\bigintcalcMax}
%    \begin{macrocode}
\def\bigintcalcMax#1{%
  \romannumeral0%
  \expandafter\expandafter\expandafter\BIC@MinMax
  \bigintcalcNum{#1}!!%
}
%    \end{macrocode}
%    \end{macro}
%    \begin{macro}{\BIC@MinMax}
%    |#1|: $x$\\
%    |#2|: sign for comparison\\
%    |#3|: $y$
%    \begin{macrocode}
\def\BIC@MinMax#1!#2!#3{%
  \expandafter\expandafter\expandafter\BIC@@MinMax
  \bigintcalcNum{#3}!#1!#2!%
}
%    \end{macrocode}
%    \end{macro}
%    \begin{macro}{\BIC@@MinMax}
%    |#1|: $y$\\
%    |#2|: $x$\\
%    |#3|: sign for comparison
%    \begin{macrocode}
\def\BIC@@MinMax#1!#2!#3!{%
  \ifnum\BIC@@Cmp#1!#2!=#31 %
    \BIC@AfterFi{ #1}%
  \else
    \BIC@AfterFi{ #2}%
  \BIC@Fi
}
%    \end{macrocode}
%    \end{macro}
%
% \subsection{\op{Odd}}
%
%    \begin{macro}{\bigintcalcOdd}
%    \begin{macrocode}
\def\bigintcalcOdd#1{%
  \romannumeral0%
  \expandafter\expandafter\expandafter\BIC@Odd
  \bigintcalcAbs{#1}!%
}
%    \end{macrocode}
%    \end{macro}
%    \begin{macro}{\BigIntCalcOdd}
%    \begin{macrocode}
\def\BigIntCalcOdd#1!{%
  \romannumeral0%
  \BIC@Odd#1!%
}
%    \end{macrocode}
%    \end{macro}
%    \begin{macro}{\BIC@Odd}
%    |#1|: $x$
%    \begin{macrocode}
\def\BIC@Odd#1#2{%
  \ifx#2!%
    \ifodd#1 %
      \BIC@AfterFiFi{ 1}%
    \else
      \BIC@AfterFiFi{ 0}%
    \fi
  \else
    \expandafter\BIC@Odd\expandafter#2%
  \BIC@Fi
}
%    \end{macrocode}
%    \end{macro}
%
% \subsection{\op{Inc}, \op{Dec}}
%
%    \begin{macro}{\bigintcalcInc}
%    \begin{macrocode}
\def\bigintcalcInc#1{%
  \romannumeral0%
  \expandafter\expandafter\expandafter\BIC@IncSwitch
  \bigintcalcNum{#1}!%
}
%    \end{macrocode}
%    \end{macro}
%    \begin{macro}{\BIC@IncSwitch}
%    \begin{macrocode}
\def\BIC@IncSwitch#1#2!{%
  \ifcase\BIC@@Cmp#1#2!-1!%
    \BIC@AfterFi{ 0}%
  \or
    \BIC@AfterFi{%
      \BIC@Inc#1#2!{}%
    }%
  \else
    \BIC@AfterFi{%
      \expandafter-\romannumeral0%
      \BIC@Dec#2!{}%
    }%
  \BIC@Fi
}
%    \end{macrocode}
%    \end{macro}
%
%    \begin{macro}{\bigintcalcDec}
%    \begin{macrocode}
\def\bigintcalcDec#1{%
  \romannumeral0%
  \expandafter\expandafter\expandafter\BIC@DecSwitch
  \bigintcalcNum{#1}!%
}
%    \end{macrocode}
%    \end{macro}
%    \begin{macro}{\BIC@DecSwitch}
%    \begin{macrocode}
\def\BIC@DecSwitch#1#2!{%
  \ifcase\BIC@Sgn#1#2! %
    \BIC@AfterFi{ -1}%
  \or
    \BIC@AfterFi{%
      \BIC@Dec#1#2!{}%
    }%
  \else
    \BIC@AfterFi{%
      \expandafter-\romannumeral0%
      \BIC@Inc#2!{}%
    }%
  \BIC@Fi
}
%    \end{macrocode}
%    \end{macro}
%
%    \begin{macro}{\BigIntCalcInc}
%    \begin{macrocode}
\def\BigIntCalcInc#1!{%
  \romannumeral0\BIC@Inc#1!{}%
}
%    \end{macrocode}
%    \end{macro}
%    \begin{macro}{\BigIntCalcDec}
%    \begin{macrocode}
\def\BigIntCalcDec#1!{%
  \romannumeral0\BIC@Dec#1!{}%
}
%    \end{macrocode}
%    \end{macro}
%
%    \begin{macro}{\BIC@Inc}
%    \begin{macrocode}
\def\BIC@Inc#1#2!#3{%
  \ifx\\#2\\%
    \BIC@AfterFi{%
      \BIC@@Inc1#1#3!{}%
    }%
  \else
    \BIC@AfterFi{%
      \BIC@Inc#2!{#1#3}%
    }%
  \BIC@Fi
}
%    \end{macrocode}
%    \end{macro}
%    \begin{macro}{\BIC@@Inc}
%    \begin{macrocode}
\def\BIC@@Inc#1#2#3!#4{%
  \ifcase#1 %
    \ifx\\#3\\%
      \BIC@AfterFiFi{ #2#4}%
    \else
      \BIC@AfterFiFi{%
        \BIC@@Inc0#3!{#2#4}%
      }%
    \fi
  \else
    \ifnum#2<9 %
      \BIC@AfterFiFi{%
&       \expandafter\BIC@@@Inc\the\numexpr#2+1\relax
$       \expandafter\expandafter\expandafter\BIC@@@Inc
$       \ifcase#2 \expandafter1%
$       \or\expandafter2%
$       \or\expandafter3%
$       \or\expandafter4%
$       \or\expandafter5%
$       \or\expandafter6%
$       \or\expandafter7%
$       \or\expandafter8%
$       \or\expandafter9%
$?      \else\BigIntCalcError:ThisCannotHappen%
$       \fi
        0#3!{#4}%
      }%
    \else
      \BIC@AfterFiFi{%
        \BIC@@@Inc01#3!{#4}%
      }%
    \fi
  \BIC@Fi
}
%    \end{macrocode}
%    \end{macro}
%    \begin{macro}{\BIC@@@Inc}
%    \begin{macrocode}
\def\BIC@@@Inc#1#2#3!#4{%
  \ifx\\#3\\%
    \ifnum#2=1 %
      \BIC@AfterFiFi{ 1#1#4}%
    \else
      \BIC@AfterFiFi{ #1#4}%
    \fi
  \else
    \BIC@AfterFi{%
      \BIC@@Inc#2#3!{#1#4}%
    }%
  \BIC@Fi
}
%    \end{macrocode}
%    \end{macro}
%
%    \begin{macro}{\BIC@Dec}
%    \begin{macrocode}
\def\BIC@Dec#1#2!#3{%
  \ifx\\#2\\%
    \BIC@AfterFi{%
      \BIC@@Dec1#1#3!{}%
    }%
  \else
    \BIC@AfterFi{%
      \BIC@Dec#2!{#1#3}%
    }%
  \BIC@Fi
}
%    \end{macrocode}
%    \end{macro}
%    \begin{macro}{\BIC@@Dec}
%    \begin{macrocode}
\def\BIC@@Dec#1#2#3!#4{%
  \ifcase#1 %
    \ifx\\#3\\%
      \BIC@AfterFiFi{ #2#4}%
    \else
      \BIC@AfterFiFi{%
        \BIC@@Dec0#3!{#2#4}%
      }%
    \fi
  \else
    \ifnum#2>0 %
      \BIC@AfterFiFi{%
&       \expandafter\BIC@@@Dec\the\numexpr#2-1\relax
$       \expandafter\expandafter\expandafter\BIC@@@Dec
$       \ifcase#2
$?        \BigIntCalcError:ThisCannotHappen%
$       \or\expandafter0%
$       \or\expandafter1%
$       \or\expandafter2%
$       \or\expandafter3%
$       \or\expandafter4%
$       \or\expandafter5%
$       \or\expandafter6%
$       \or\expandafter7%
$       \or\expandafter8%
$?      \else\BigIntCalcError:ThisCannotHappen%
$       \fi
        0#3!{#4}%
      }%
    \else
      \BIC@AfterFiFi{%
        \BIC@@@Dec91#3!{#4}%
      }%
    \fi
  \BIC@Fi
}
%    \end{macrocode}
%    \end{macro}
%    \begin{macro}{\BIC@@@Dec}
%    \begin{macrocode}
\def\BIC@@@Dec#1#2#3!#4{%
  \ifx\\#3\\%
    \ifcase#1 %
      \ifx\\#4\\%
        \BIC@AfterFiFiFi{ 0}%
      \else
        \BIC@AfterFiFiFi{ #4}%
      \fi
    \else
      \BIC@AfterFiFi{ #1#4}%
    \fi
  \else
    \BIC@AfterFi{%
      \BIC@@Dec#2#3!{#1#4}%
    }%
  \BIC@Fi
}
%    \end{macrocode}
%    \end{macro}
%
% \subsection{\op{Add}, \op{Sub}}
%
%    \begin{macro}{\bigintcalcAdd}
%    \begin{macrocode}
\def\bigintcalcAdd#1{%
  \romannumeral0%
  \expandafter\expandafter\expandafter\BIC@Add
  \bigintcalcNum{#1}!%
}
%    \end{macrocode}
%    \end{macro}
%    \begin{macro}{\BIC@Add}
%    \begin{macrocode}
\def\BIC@Add#1!#2{%
  \expandafter\expandafter\expandafter
  \BIC@AddSwitch\bigintcalcNum{#2}!#1!%
}
%    \end{macrocode}
%    \end{macro}
%    \begin{macro}{\bigintcalcSub}
%    \begin{macrocode}
\def\bigintcalcSub#1#2{%
  \romannumeral0%
  \expandafter\expandafter\expandafter\BIC@Add
  \bigintcalcNum{-#2}!{#1}%
}
%    \end{macrocode}
%    \end{macro}
%
%    \begin{macro}{\BIC@AddSwitch}
%    Decision table for \cs{BIC@AddSwitch}.
%    \begin{quote}
%    \begin{tabular}[t]{@{}|l|l|l|l|l|@{}}
%      \hline
%      $x<0$ & $y<0$ & $-x>-y$ & $-$ & $\opAdd(-x,-y)$\\
%      \cline{3-3}\cline{5-5}
%            &       & else    &     & $\opAdd(-y,-x)$\\
%      \cline{2-5}
%            & else  & $-x> y$ & $-$ & $\opSub(-x, y)$\\
%      \cline{3-5}
%            &       & $-x= y$ &     & $0$\\
%      \cline{3-5}
%            &       & else    & $+$ & $\opSub( y,-x)$\\
%      \hline
%      else  & $y<0$ & $ x>-y$ & $+$ & $\opSub( x,-y)$\\
%      \cline{3-5}
%            &       & $ x=-y$ &     & $0$\\
%      \cline{3-5}
%            &       & else    & $-$ & $\opSub(-y, x)$\\
%      \cline{2-5}
%            & else  & $ x> y$ & $+$ & $\opAdd( x, y)$\\
%      \cline{3-3}\cline{5-5}
%            &       & else    &     & $\opAdd( y, x)$\\
%      \hline
%    \end{tabular}
%    \end{quote}
%    \begin{macrocode}
\def\BIC@AddSwitch#1#2!#3#4!{%
  \ifx#1-% x < 0
    \ifx#3-% y < 0
      \expandafter-\romannumeral0%
      \ifnum\BIC@PosCmp#2!#4!=1 % -x > -y
        \BIC@AfterFiFiFi{%
          \BIC@AddXY#2!#4!!!%
        }%
      \else % -x <= -y
        \BIC@AfterFiFiFi{%
          \BIC@AddXY#4!#2!!!%
        }%
      \fi
    \else % y >= 0
      \ifcase\BIC@PosCmp#2!#3#4!% -x = y
        \BIC@AfterFiFiFi{ 0}%
      \or % -x > y
        \expandafter-\romannumeral0%
        \BIC@AfterFiFiFi{%
          \BIC@SubXY#2!#3#4!!!%
        }%
      \else % -x <= y
        \BIC@AfterFiFiFi{%
          \BIC@SubXY#3#4!#2!!!%
        }%
      \fi
    \fi
  \else % x >= 0
    \ifx#3-% y < 0
      \ifcase\BIC@PosCmp#1#2!#4!% x = -y
        \BIC@AfterFiFiFi{ 0}%
      \or % x > -y
        \BIC@AfterFiFiFi{%
          \BIC@SubXY#1#2!#4!!!%
        }%
      \else % x <= -y
        \expandafter-\romannumeral0%
        \BIC@AfterFiFiFi{%
          \BIC@SubXY#4!#1#2!!!%
        }%
      \fi
    \else % y >= 0
      \ifnum\BIC@PosCmp#1#2!#3#4!=1 % x > y
        \BIC@AfterFiFiFi{%
          \BIC@AddXY#1#2!#3#4!!!%
        }%
      \else % x <= y
        \BIC@AfterFiFiFi{%
          \BIC@AddXY#3#4!#1#2!!!%
        }%
      \fi
    \fi
  \BIC@Fi
}
%    \end{macrocode}
%    \end{macro}
%
%    \begin{macro}{\BigIntCalcAdd}
%    \begin{macrocode}
\def\BigIntCalcAdd#1!#2!{%
  \romannumeral0\BIC@AddXY#1!#2!!!%
}
%    \end{macrocode}
%    \end{macro}
%    \begin{macro}{\BigIntCalcSub}
%    \begin{macrocode}
\def\BigIntCalcSub#1!#2!{%
  \romannumeral0\BIC@SubXY#1!#2!!!%
}
%    \end{macrocode}
%    \end{macro}
%
%    \begin{macro}{\BIC@AddXY}
%    \begin{macrocode}
\def\BIC@AddXY#1#2!#3#4!#5!#6!{%
  \ifx\\#2\\%
    \ifx\\#3\\%
      \BIC@AfterFiFi{%
        \BIC@DoAdd0!#1#5!#60!%
      }%
    \else
      \BIC@AfterFiFi{%
        \BIC@DoAdd0!#1#5!#3#6!%
      }%
    \fi
  \else
    \ifx\\#4\\%
      \ifx\\#3\\%
        \BIC@AfterFiFiFi{%
          \BIC@AddXY#2!{}!#1#5!#60!%
        }%
      \else
        \BIC@AfterFiFiFi{%
          \BIC@AddXY#2!{}!#1#5!#3#6!%
        }%
      \fi
    \else
      \BIC@AfterFiFi{%
        \BIC@AddXY#2!#4!#1#5!#3#6!%
      }%
    \fi
  \BIC@Fi
}
%    \end{macrocode}
%    \end{macro}
%    \begin{macro}{\BIC@DoAdd}
%    |#1|: carry\\
%    |#2|: reverted result\\
%    |#3#4|: reverted $x$\\
%    |#5#6|: reverted $y$
%    \begin{macrocode}
\def\BIC@DoAdd#1#2!#3#4!#5#6!{%
  \ifx\\#4\\%
    \BIC@AfterFi{%
&     \expandafter\BIC@Space
&     \the\numexpr#1+#3+#5\relax#2%
$     \expandafter\expandafter\expandafter\BIC@AddResult
$     \BIC@AddDigit#1#3#5#2%
    }%
  \else
    \BIC@AfterFi{%
      \expandafter\expandafter\expandafter\BIC@DoAdd
      \BIC@AddDigit#1#3#5#2!#4!#6!%
    }%
  \BIC@Fi
}
%    \end{macrocode}
%    \end{macro}
%    \begin{macro}{\BIC@AddResult}
%    \begin{macrocode}
$ \def\BIC@AddResult#1{%
$   \ifx#10%
$     \expandafter\BIC@Space
$   \else
$     \expandafter\BIC@Space\expandafter#1%
$   \fi
$ }%
%    \end{macrocode}
%    \end{macro}
%    \begin{macro}{\BIC@AddDigit}
%    |#1|: carry\\
%    |#2|: digit of $x$\\
%    |#3|: digit of $y$
%    \begin{macrocode}
\def\BIC@AddDigit#1#2#3{%
  \romannumeral0%
& \expandafter\BIC@@AddDigit\the\numexpr#1+#2+#3!%
$ \expandafter\BIC@@AddDigit\number%
$ \csname
$   BIC@AddCarry%
$   \ifcase#1 %
$     #2%
$   \else
$     \ifcase#2 1\or2\or3\or4\or5\or6\or7\or8\or9\or10\fi
$   \fi
$ \endcsname#3!%
}
%    \end{macrocode}
%    \end{macro}
%    \begin{macro}{\BIC@@AddDigit}
%    \begin{macrocode}
\def\BIC@@AddDigit#1!{%
  \ifnum#1<10 %
    \BIC@AfterFi{ 0#1}%
  \else
    \BIC@AfterFi{ #1}%
  \BIC@Fi
}
%    \end{macrocode}
%    \end{macro}
%    \begin{macro}{\BIC@AddCarry0}
%    \begin{macrocode}
$ \expandafter\def\csname BIC@AddCarry0\endcsname#1{#1}%
%    \end{macrocode}
%    \end{macro}
%    \begin{macro}{\BIC@AddCarry10}
%    \begin{macrocode}
$ \expandafter\def\csname BIC@AddCarry10\endcsname#1{1#1}%
%    \end{macrocode}
%    \end{macro}
%    \begin{macro}{\BIC@AddCarry[1-9]}
%    \begin{macrocode}
$ \def\BIC@Temp#1#2{%
$   \expandafter\def\csname BIC@AddCarry#1\endcsname##1{%
$     \ifcase##1 #1\or
$     #2%
$?    \else\BigIntCalcError:ThisCannotHappen%
$     \fi
$   }%
$ }%
$ \BIC@Temp 0{1\or2\or3\or4\or5\or6\or7\or8\or9}%
$ \BIC@Temp 1{2\or3\or4\or5\or6\or7\or8\or9\or10}%
$ \BIC@Temp 2{3\or4\or5\or6\or7\or8\or9\or10\or11}%
$ \BIC@Temp 3{4\or5\or6\or7\or8\or9\or10\or11\or12}%
$ \BIC@Temp 4{5\or6\or7\or8\or9\or10\or11\or12\or13}%
$ \BIC@Temp 5{6\or7\or8\or9\or10\or11\or12\or13\or14}%
$ \BIC@Temp 6{7\or8\or9\or10\or11\or12\or13\or14\or15}%
$ \BIC@Temp 7{8\or9\or10\or11\or12\or13\or14\or15\or16}%
$ \BIC@Temp 8{9\or10\or11\or12\or13\or14\or15\or16\or17}%
$ \BIC@Temp 9{10\or11\or12\or13\or14\or15\or16\or17\or18}%
%    \end{macrocode}
%    \end{macro}
%
%    \begin{macro}{\BIC@SubXY}
%    Preconditions:
%    \begin{itemize}
%    \item $x > y$, $x \geq 0$, and $y >= 0$
%    \item digits($x$) = digits($y$)
%    \end{itemize}
%    \begin{macrocode}
\def\BIC@SubXY#1#2!#3#4!#5!#6!{%
  \ifx\\#2\\%
    \ifx\\#3\\%
      \BIC@AfterFiFi{%
        \BIC@DoSub0!#1#5!#60!%
      }%
    \else
      \BIC@AfterFiFi{%
        \BIC@DoSub0!#1#5!#3#6!%
      }%
    \fi
  \else
    \ifx\\#4\\%
      \ifx\\#3\\%
        \BIC@AfterFiFiFi{%
          \BIC@SubXY#2!{}!#1#5!#60!%
        }%
      \else
        \BIC@AfterFiFiFi{%
          \BIC@SubXY#2!{}!#1#5!#3#6!%
        }%
      \fi
    \else
      \BIC@AfterFiFi{%
        \BIC@SubXY#2!#4!#1#5!#3#6!%
      }%
    \fi
  \BIC@Fi
}
%    \end{macrocode}
%    \end{macro}
%    \begin{macro}{\BIC@DoSub}
%    |#1|: carry\\
%    |#2|: reverted result\\
%    |#3#4|: reverted x\\
%    |#5#6|: reverted y
%    \begin{macrocode}
\def\BIC@DoSub#1#2!#3#4!#5#6!{%
  \ifx\\#4\\%
    \BIC@AfterFi{%
      \expandafter\expandafter\expandafter\BIC@SubResult
      \BIC@SubDigit#1#3#5#2%
    }%
  \else
    \BIC@AfterFi{%
      \expandafter\expandafter\expandafter\BIC@DoSub
      \BIC@SubDigit#1#3#5#2!#4!#6!%
    }%
  \BIC@Fi
}
%    \end{macrocode}
%    \end{macro}
%    \begin{macro}{\BIC@SubResult}
%    \begin{macrocode}
\def\BIC@SubResult#1{%
  \ifx#10%
    \expandafter\BIC@SubResult
  \else
    \expandafter\BIC@Space\expandafter#1%
  \fi
}
%    \end{macrocode}
%    \end{macro}
%    \begin{macro}{\BIC@SubDigit}
%    |#1|: carry\\
%    |#2|: digit of $x$\\
%    |#3|: digit of $y$
%    \begin{macrocode}
\def\BIC@SubDigit#1#2#3{%
  \romannumeral0%
& \expandafter\BIC@@SubDigit\the\numexpr#2-#3-#1!%
$ \expandafter\BIC@@AddDigit\number
$   \csname
$     BIC@SubCarry%
$     \ifcase#1 %
$       #3%
$     \else
$       \ifcase#3 1\or2\or3\or4\or5\or6\or7\or8\or9\or10\fi
$     \fi
$   \endcsname#2!%
}
%    \end{macrocode}
%    \end{macro}
%    \begin{macro}{\BIC@@SubDigit}
%    \begin{macrocode}
& \def\BIC@@SubDigit#1!{%
&   \ifnum#1<0 %
&     \BIC@AfterFi{%
&       \expandafter\BIC@Space
&       \expandafter1\the\numexpr#1+10\relax
&     }%
&   \else
&     \BIC@AfterFi{ 0#1}%
&   \BIC@Fi
& }%
%    \end{macrocode}
%    \end{macro}
%    \begin{macro}{\BIC@SubCarry0}
%    \begin{macrocode}
$ \expandafter\def\csname BIC@SubCarry0\endcsname#1{#1}%
%    \end{macrocode}
%    \end{macro}
%    \begin{macro}{\BIC@SubCarry10}%
%    \begin{macrocode}
$ \expandafter\def\csname BIC@SubCarry10\endcsname#1{1#1}%
%    \end{macrocode}
%    \end{macro}
%    \begin{macro}{\BIC@SubCarry[1-9]}
%    \begin{macrocode}
$ \def\BIC@Temp#1#2{%
$   \expandafter\def\csname BIC@SubCarry#1\endcsname##1{%
$     \ifcase##1 #2%
$?    \else\BigIntCalcError:ThisCannotHappen%
$     \fi
$   }%
$ }%
$ \BIC@Temp 1{19\or0\or1\or2\or3\or4\or5\or6\or7\or8}%
$ \BIC@Temp 2{18\or19\or0\or1\or2\or3\or4\or5\or6\or7}%
$ \BIC@Temp 3{17\or18\or19\or0\or1\or2\or3\or4\or5\or6}%
$ \BIC@Temp 4{16\or17\or18\or19\or0\or1\or2\or3\or4\or5}%
$ \BIC@Temp 5{15\or16\or17\or18\or19\or0\or1\or2\or3\or4}%
$ \BIC@Temp 6{14\or15\or16\or17\or18\or19\or0\or1\or2\or3}%
$ \BIC@Temp 7{13\or14\or15\or16\or17\or18\or19\or0\or1\or2}%
$ \BIC@Temp 8{12\or13\or14\or15\or16\or17\or18\or19\or0\or1}%
$ \BIC@Temp 9{11\or12\or13\or14\or15\or16\or17\or18\or19\or0}%
%    \end{macrocode}
%    \end{macro}
%
% \subsection{\op{Shl}, \op{Shr}}
%
%    \begin{macro}{\bigintcalcShl}
%    \begin{macrocode}
\def\bigintcalcShl#1{%
  \romannumeral0%
  \expandafter\expandafter\expandafter\BIC@Shl
  \bigintcalcNum{#1}!%
}
%    \end{macrocode}
%    \end{macro}
%    \begin{macro}{\BIC@Shl}
%    \begin{macrocode}
\def\BIC@Shl#1#2!{%
  \ifx#1-%
    \BIC@AfterFi{%
      \expandafter-\romannumeral0%
&     \BIC@@Shl#2!!%
$     \BIC@AddXY#2!#2!!!%
    }%
  \else
    \BIC@AfterFi{%
&     \BIC@@Shl#1#2!!%
$     \BIC@AddXY#1#2!#1#2!!!%
    }%
  \BIC@Fi
}
%    \end{macrocode}
%    \end{macro}
%    \begin{macro}{\BigIntCalcShl}
%    \begin{macrocode}
\def\BigIntCalcShl#1!{%
  \romannumeral0%
& \BIC@@Shl#1!!%
$ \BIC@AddXY#1!#1!!!%
}
%    \end{macrocode}
%    \end{macro}
%    \begin{macro}{\BIC@@Shl}
%    \begin{macrocode}
& \def\BIC@@Shl#1#2!{%
&   \ifx\\#2\\%
&     \BIC@AfterFi{%
&       \BIC@@@Shl0!#1%
&     }%
&   \else
&     \BIC@AfterFi{%
&       \BIC@@Shl#2!#1%
&     }%
&   \BIC@Fi
& }%
%    \end{macrocode}
%    \end{macro}
%    \begin{macro}{\BIC@@@Shl}
%    |#1|: carry\\
%    |#2|: result\\
%    |#3#4|: reverted number
%    \begin{macrocode}
& \def\BIC@@@Shl#1#2!#3#4!{%
&   \ifx\\#4\\%
&     \BIC@AfterFi{%
&       \expandafter\BIC@Space
&       \the\numexpr#3*2+#1\relax#2%
&     }%
&   \else
&     \BIC@AfterFi{%
&       \expandafter\BIC@@@@Shl\the\numexpr#3*2+#1!#2!#4!%
&     }%
&   \BIC@Fi
& }%
%    \end{macrocode}
%    \end{macro}
%    \begin{macro}{\BIC@@@@Shl}
%    \begin{macrocode}
& \def\BIC@@@@Shl#1!{%
&   \ifnum#1<10 %
&     \BIC@AfterFi{%
&       \BIC@@@Shl0#1%
&     }%
&   \else
&     \BIC@AfterFi{%
&       \BIC@@@Shl#1%
&     }%
&   \BIC@Fi
& }%
%    \end{macrocode}
%    \end{macro}
%
%    \begin{macro}{\bigintcalcShr}
%    \begin{macrocode}
\def\bigintcalcShr#1{%
  \romannumeral0%
  \expandafter\expandafter\expandafter\BIC@Shr
  \bigintcalcNum{#1}!%
}
%    \end{macrocode}
%    \end{macro}
%    \begin{macro}{\BIC@Shr}
%    \begin{macrocode}
\def\BIC@Shr#1#2!{%
  \ifx#1-%
    \expandafter-\romannumeral0%
    \BIC@AfterFi{%
      \BIC@@Shr#2!%
    }%
  \else
    \BIC@AfterFi{%
      \BIC@@Shr#1#2!%
    }%
  \BIC@Fi
}
%    \end{macrocode}
%    \end{macro}
%    \begin{macro}{\BigIntCalcShr}
%    \begin{macrocode}
\def\BigIntCalcShr#1!{%
  \romannumeral0%
  \BIC@@Shr#1!%
}
%    \end{macrocode}
%    \end{macro}
%    \begin{macro}{\BIC@@Shr}
%    \begin{macrocode}
\def\BIC@@Shr#1#2!{%
  \ifcase#1 %
    \BIC@AfterFi{ 0}%
  \or
    \ifx\\#2\\%
      \BIC@AfterFiFi{ 0}%
    \else
      \BIC@AfterFiFi{%
        \BIC@@@Shr#1#2!!%
      }%
    \fi
  \else
    \BIC@AfterFi{%
      \BIC@@@Shr0#1#2!!%
    }%
  \BIC@Fi
}
%    \end{macrocode}
%    \end{macro}
%    \begin{macro}{\BIC@@@Shr}
%    |#1|: carry\\
%    |#2#3|: number\\
%    |#4|: result
%    \begin{macrocode}
\def\BIC@@@Shr#1#2#3!#4!{%
  \ifx\\#3\\%
    \ifodd#1#2 %
      \BIC@AfterFiFi{%
&       \expandafter\BIC@ShrResult\the\numexpr(#1#2-1)/2\relax
$       \expandafter\expandafter\expandafter\BIC@ShrResult
$       \csname BIC@ShrDigit#1#2\endcsname
        #4!%
      }%
    \else
      \BIC@AfterFiFi{%
&       \expandafter\BIC@ShrResult\the\numexpr#1#2/2\relax
$       \expandafter\expandafter\expandafter\BIC@ShrResult
$       \csname BIC@ShrDigit#1#2\endcsname
        #4!%
      }%
    \fi
  \else
    \ifodd#1#2 %
      \BIC@AfterFiFi{%
&       \expandafter\BIC@@@@Shr\the\numexpr(#1#2-1)/2\relax1%
$       \expandafter\expandafter\expandafter\BIC@@@@Shr
$       \csname BIC@ShrDigit#1#2\endcsname
        #3!#4!%
      }%
    \else
      \BIC@AfterFiFi{%
&       \expandafter\BIC@@@@Shr\the\numexpr#1#2/2\relax0%
$       \expandafter\expandafter\expandafter\BIC@@@@Shr
$       \csname BIC@ShrDigit#1#2\endcsname
        #3!#4!%
      }%
    \fi
  \BIC@Fi
}
%    \end{macrocode}
%    \end{macro}
%    \begin{macro}{\BIC@ShrResult}
%    \begin{macrocode}
& \def\BIC@ShrResult#1#2!{ #2#1}%
$ \def\BIC@ShrResult#1#2#3!{ #3#1}%
%    \end{macrocode}
%    \end{macro}
%    \begin{macro}{\BIC@@@@Shr}
%    |#1|: new digit\\
%    |#2|: carry\\
%    |#3|: remaining number\\
%    |#4|: result
%    \begin{macrocode}
\def\BIC@@@@Shr#1#2#3!#4!{%
  \BIC@@@Shr#2#3!#4#1!%
}
%    \end{macrocode}
%    \end{macro}
%    \begin{macro}{\BIC@ShrDigit[00-19]}
%    \begin{macrocode}
$ \def\BIC@Temp#1#2#3#4{%
$   \expandafter\def\csname BIC@ShrDigit#1#2\endcsname{#3#4}%
$ }%
$ \BIC@Temp 0000%
$ \BIC@Temp 0101%
$ \BIC@Temp 0210%
$ \BIC@Temp 0311%
$ \BIC@Temp 0420%
$ \BIC@Temp 0521%
$ \BIC@Temp 0630%
$ \BIC@Temp 0731%
$ \BIC@Temp 0840%
$ \BIC@Temp 0941%
$ \BIC@Temp 1050%
$ \BIC@Temp 1151%
$ \BIC@Temp 1260%
$ \BIC@Temp 1361%
$ \BIC@Temp 1470%
$ \BIC@Temp 1571%
$ \BIC@Temp 1680%
$ \BIC@Temp 1781%
$ \BIC@Temp 1890%
$ \BIC@Temp 1991%
%    \end{macrocode}
%    \end{macro}
%
% \subsection{\cs{BIC@Tim}}
%
%    \begin{macro}{\BIC@Tim}
%    Macro \cs{BIC@Tim} implements ``Number \emph{tim}es digit''.\\
%    |#1|: plain number without sign\\
%    |#2|: digit
\def\BIC@Tim#1!#2{%
  \romannumeral0%
  \ifcase#2 % 0
    \BIC@AfterFi{ 0}%
  \or % 1
    \BIC@AfterFi{ #1}%
  \or % 2
    \BIC@AfterFi{%
      \BIC@Shl#1!%
    }%
  \else % 3-9
    \BIC@AfterFi{%
      \BIC@@Tim#1!!#2%
    }%
  \BIC@Fi
}
%    \begin{macrocode}
%    \end{macrocode}
%    \end{macro}
%    \begin{macro}{\BIC@@Tim}
%    |#1#2|: number\\
%    |#3|: reverted number
%    \begin{macrocode}
\def\BIC@@Tim#1#2!{%
  \ifx\\#2\\%
    \BIC@AfterFi{%
      \BIC@ProcessTim0!#1%
    }%
  \else
    \BIC@AfterFi{%
      \BIC@@Tim#2!#1%
    }%
  \BIC@Fi
}
%    \end{macrocode}
%    \end{macro}
%    \begin{macro}{\BIC@ProcessTim}
%    |#1|: carry\\
%    |#2|: result\\
%    |#3#4|: reverted number\\
%    |#5|: digit
%    \begin{macrocode}
\def\BIC@ProcessTim#1#2!#3#4!#5{%
  \ifx\\#4\\%
    \BIC@AfterFi{%
      \expandafter\BIC@Space
&     \the\numexpr#3*#5+#1\relax
$     \romannumeral0\BIC@TimDigit#3#5#1%
      #2%
    }%
  \else
    \BIC@AfterFi{%
      \expandafter\BIC@@ProcessTim
&     \the\numexpr#3*#5+#1%
$     \romannumeral0\BIC@TimDigit#3#5#1%
      !#2!#4!#5%
    }%
  \BIC@Fi
}
%    \end{macrocode}
%    \end{macro}
%    \begin{macro}{\BIC@@ProcessTim}
%    |#1#2|: carry?, new digit\\
%    |#3|: new number\\
%    |#4|: old number\\
%    |#5|: digit
%    \begin{macrocode}
\def\BIC@@ProcessTim#1#2!{%
  \ifx\\#2\\%
    \BIC@AfterFi{%
      \BIC@ProcessTim0#1%
    }%
  \else
    \BIC@AfterFi{%
      \BIC@ProcessTim#1#2%
    }%
  \BIC@Fi
}
%    \end{macrocode}
%    \end{macro}
%    \begin{macro}{\BIC@TimDigit}
%    |#1|: digit 0--9\\
%    |#2|: digit 3--9\\
%    |#3|: carry 0--9
%    \begin{macrocode}
$ \def\BIC@TimDigit#1#2#3{%
$   \ifcase#1 % 0
$     \BIC@AfterFi{ #3}%
$   \or % 1
$     \BIC@AfterFi{%
$       \expandafter\BIC@Space
$       \number\csname BIC@AddCarry#2\endcsname#3 %
$     }%
$   \else
$     \ifcase#3 %
$       \BIC@AfterFiFi{%
$         \expandafter\BIC@Space
$         \number\csname BIC@MulDigit#2\endcsname#1 %
$       }%
$     \else
$       \BIC@AfterFiFi{%
$         \expandafter\BIC@Space
$         \romannumeral0%
$         \expandafter\BIC@AddXY
$         \number\csname BIC@MulDigit#2\endcsname#1!%
$         #3!!!%
$       }%
$     \fi
$   \BIC@Fi
$ }%
%    \end{macrocode}
%    \end{macro}
%    \begin{macro}{\BIC@MulDigit[3-9]}
%    \begin{macrocode}
$ \def\BIC@Temp#1#2{%
$   \expandafter\def\csname BIC@MulDigit#1\endcsname##1{%
$     \ifcase##1 0%
$     \or ##1%
$     \or #2%
$?    \else\BigIntCalcError:ThisCannotHappen%
$     \fi
$   }%
$ }%
$ \BIC@Temp 3{6\or9\or12\or15\or18\or21\or24\or27}%
$ \BIC@Temp 4{8\or12\or16\or20\or24\or28\or32\or36}%
$ \BIC@Temp 5{10\or15\or20\or25\or30\or35\or40\or45}%
$ \BIC@Temp 6{12\or18\or24\or30\or36\or42\or48\or54}%
$ \BIC@Temp 7{14\or21\or28\or35\or42\or49\or56\or63}%
$ \BIC@Temp 8{16\or24\or32\or40\or48\or56\or64\or72}%
$ \BIC@Temp 9{18\or27\or36\or45\or54\or63\or72\or81}%
%    \end{macrocode}
%    \end{macro}
%
% \subsection{\op{Mul}}
%
%    \begin{macro}{\bigintcalcMul}
%    \begin{macrocode}
\def\bigintcalcMul#1#2{%
  \romannumeral0%
  \expandafter\expandafter\expandafter\BIC@Mul
  \bigintcalcNum{#1}!{#2}%
}
%    \end{macrocode}
%    \end{macro}
%    \begin{macro}{\BIC@Mul}
%    \begin{macrocode}
\def\BIC@Mul#1!#2{%
  \expandafter\expandafter\expandafter\BIC@MulSwitch
  \bigintcalcNum{#2}!#1!%
}
%    \end{macrocode}
%    \end{macro}
%    \begin{macro}{\BIC@MulSwitch}
%    Decision table for \cs{BIC@MulSwitch}.
%    \begin{quote}
%    \begin{tabular}[t]{@{}|l|l|l|l|l|@{}}
%      \hline
%      $x=0$ &\multicolumn{3}{l}{}    &$0$\\
%      \hline
%      $x>0$ & $y=0$ &\multicolumn2l{}&$0$\\
%      \cline{2-5}
%            & $y>0$ & $ x> y$ & $+$ & $\opMul( x, y)$\\
%      \cline{3-3}\cline{5-5}
%            &       & else    &     & $\opMul( y, x)$\\
%      \cline{2-5}
%            & $y<0$ & $ x>-y$ & $-$ & $\opMul( x,-y)$\\
%      \cline{3-3}\cline{5-5}
%            &       & else    &     & $\opMul(-y, x)$\\
%      \hline
%      $x<0$ & $y=0$ &\multicolumn2l{}&$0$\\
%      \cline{2-5}
%            & $y>0$ & $-x> y$ & $-$ & $\opMul(-x, y)$\\
%      \cline{3-3}\cline{5-5}
%            &       & else    &     & $\opMul( y,-x)$\\
%      \cline{2-5}
%            & $y<0$ & $-x>-y$ & $+$ & $\opMul(-x,-y)$\\
%      \cline{3-3}\cline{5-5}
%            &       & else    &     & $\opMul(-y,-x)$\\
%      \hline
%    \end{tabular}
%    \end{quote}
%    \begin{macrocode}
\def\BIC@MulSwitch#1#2!#3#4!{%
  \ifcase\BIC@Sgn#1#2! % x = 0
    \BIC@AfterFi{ 0}%
  \or % x > 0
    \ifcase\BIC@Sgn#3#4! % y = 0
      \BIC@AfterFiFi{ 0}%
    \or % y > 0
      \ifnum\BIC@PosCmp#1#2!#3#4!=1 % x > y
        \BIC@AfterFiFiFi{%
          \BIC@ProcessMul0!#1#2!#3#4!%
        }%
      \else % x <= y
        \BIC@AfterFiFiFi{%
          \BIC@ProcessMul0!#3#4!#1#2!%
        }%
      \fi
    \else % y < 0
      \expandafter-\romannumeral0%
      \ifnum\BIC@PosCmp#1#2!#4!=1 % x > -y
        \BIC@AfterFiFiFi{%
          \BIC@ProcessMul0!#1#2!#4!%
        }%
      \else % x <= -y
        \BIC@AfterFiFiFi{%
          \BIC@ProcessMul0!#4!#1#2!%
        }%
      \fi
    \fi
  \else % x < 0
    \ifcase\BIC@Sgn#3#4! % y = 0
      \BIC@AfterFiFi{ 0}%
    \or % y > 0
      \expandafter-\romannumeral0%
      \ifnum\BIC@PosCmp#2!#3#4!=1 % -x > y
        \BIC@AfterFiFiFi{%
          \BIC@ProcessMul0!#2!#3#4!%
        }%
      \else % -x <= y
        \BIC@AfterFiFiFi{%
          \BIC@ProcessMul0!#3#4!#2!%
        }%
      \fi
    \else % y < 0
      \ifnum\BIC@PosCmp#2!#4!=1 % -x > -y
        \BIC@AfterFiFiFi{%
          \BIC@ProcessMul0!#2!#4!%
        }%
      \else % -x <= -y
        \BIC@AfterFiFiFi{%
          \BIC@ProcessMul0!#4!#2!%
        }%
      \fi
    \fi
  \BIC@Fi
}
%    \end{macrocode}
%    \end{macro}
%    \begin{macro}{\BigIntCalcMul}
%    \begin{macrocode}
\def\BigIntCalcMul#1!#2!{%
  \romannumeral0%
  \BIC@ProcessMul0!#1!#2!%
}
%    \end{macrocode}
%    \end{macro}
%    \begin{macro}{\BIC@ProcessMul}
%    |#1|: result\\
%    |#2|: number $x$\\
%    |#3#4|: number $y$
%    \begin{macrocode}
\def\BIC@ProcessMul#1!#2!#3#4!{%
  \ifx\\#4\\%
    \BIC@AfterFi{%
      \expandafter\expandafter\expandafter\BIC@Space
      \bigintcalcAdd{\BIC@Tim#2!#3}{#10}%
    }%
  \else
    \BIC@AfterFi{%
      \expandafter\expandafter\expandafter\BIC@ProcessMul
      \bigintcalcAdd{\BIC@Tim#2!#3}{#10}!#2!#4!%
    }%
  \BIC@Fi
}
%    \end{macrocode}
%    \end{macro}
%
% \subsection{\op{Sqr}}
%
%    \begin{macro}{\bigintcalcSqr}
%    \begin{macrocode}
\def\bigintcalcSqr#1{%
  \romannumeral0%
  \expandafter\expandafter\expandafter\BIC@Sqr
  \bigintcalcNum{#1}!%
}
%    \end{macrocode}
%    \end{macro}
%    \begin{macro}{\BIC@Sqr}
%    \begin{macrocode}
\def\BIC@Sqr#1{%
   \ifx#1-%
     \expandafter\BIC@@Sqr
   \else
     \expandafter\BIC@@Sqr\expandafter#1%
   \fi
}
%    \end{macrocode}
%    \end{macro}
%    \begin{macro}{\BIC@@Sqr}
%    \begin{macrocode}
\def\BIC@@Sqr#1!{%
  \BIC@ProcessMul0!#1!#1!%
}
%    \end{macrocode}
%    \end{macro}
%
% \subsection{\op{Fac}}
%
%    \begin{macro}{\bigintcalcFac}
%    \begin{macrocode}
\def\bigintcalcFac#1{%
  \romannumeral0%
  \expandafter\expandafter\expandafter\BIC@Fac
  \bigintcalcNum{#1}!%
}
%    \end{macrocode}
%    \end{macro}
%    \begin{macro}{\BIC@Fac}
%    \begin{macrocode}
\def\BIC@Fac#1#2!{%
  \ifx#1-%
    \BIC@AfterFi{ 0\BigIntCalcError:FacNegative}%
  \else
    \ifnum\BIC@PosCmp#1#2!13!<0 %
      \ifcase#1#2 %
         \BIC@AfterFiFiFi{ 1}% 0!
      \or\BIC@AfterFiFiFi{ 1}% 1!
      \or\BIC@AfterFiFiFi{ 2}% 2!
      \or\BIC@AfterFiFiFi{ 6}% 3!
      \or\BIC@AfterFiFiFi{ 24}% 4!
      \or\BIC@AfterFiFiFi{ 120}% 5!
      \or\BIC@AfterFiFiFi{ 720}% 6!
      \or\BIC@AfterFiFiFi{ 5040}% 7!
      \or\BIC@AfterFiFiFi{ 40320}% 8!
      \or\BIC@AfterFiFiFi{ 362880}% 9!
      \or\BIC@AfterFiFiFi{ 3628800}% 10!
      \or\BIC@AfterFiFiFi{ 39916800}% 11!
      \or\BIC@AfterFiFiFi{ 479001600}% 12!
?     \else\BigIntCalcError:ThisCannotHappen%
      \fi
    \else
      \BIC@AfterFiFi{%
        \BIC@ProcessFac#1#2!479001600!%
      }%
    \fi
  \BIC@Fi
}
%    \end{macrocode}
%    \end{macro}
%    \begin{macro}{\BIC@ProcessFac}
%    |#1|: $n$\\
%    |#2|: result
%    \begin{macrocode}
\def\BIC@ProcessFac#1!#2!{%
  \ifnum\BIC@PosCmp#1!12!=0 %
    \BIC@AfterFi{ #2}%
  \else
    \BIC@AfterFi{%
      \expandafter\BIC@@ProcessFac
      \romannumeral0\BIC@ProcessMul0!#2!#1!%
      !#1!%
    }%
  \BIC@Fi
}
%    \end{macrocode}
%    \end{macro}
%    \begin{macro}{\BIC@@ProcessFac}
%    |#1|: result\\
%    |#2|: $n$
%    \begin{macrocode}
\def\BIC@@ProcessFac#1!#2!{%
  \expandafter\BIC@ProcessFac
  \romannumeral0\BIC@Dec#2!{}%
  !#1!%
}
%    \end{macrocode}
%    \end{macro}
%
% \subsection{\op{Pow}}
%
%    \begin{macro}{\bigintcalcPow}
%    |#1|: basis\\
%    |#2|: power
%    \begin{macrocode}
\def\bigintcalcPow#1{%
  \romannumeral0%
  \expandafter\expandafter\expandafter\BIC@Pow
  \bigintcalcNum{#1}!%
}
%    \end{macrocode}
%    \end{macro}
%    \begin{macro}{\BIC@Pow}
%    |#1|: basis\\
%    |#2|: power
%    \begin{macrocode}
\def\BIC@Pow#1!#2{%
  \expandafter\expandafter\expandafter\BIC@PowSwitch
  \bigintcalcNum{#2}!#1!%
}
%    \end{macrocode}
%    \end{macro}
%    \begin{macro}{\BIC@PowSwitch}
%    |#1#2|: power $y$\\
%    |#3#4|: basis $x$\\
%    Decision table for \cs{BIC@PowSwitch}.
%    \begin{quote}
%    \def\M#1{\multicolumn{#1}{l}{}}%
%    \begin{tabular}[t]{@{}|l|l|l|l|@{}}
%    \hline
%    $y=0$ & \M{2}                        & $1$\\
%    \hline
%    $y=1$ & \M{2}                        & $x$\\
%    \hline
%    $y=2$ & $x<0$          & \M{1}       & $\opMul(-x,-x)$\\
%    \cline{2-4}
%          & else           & \M{1}       & $\opMul(x,x)$\\
%    \hline
%    $y<0$ & $x=0$          & \M{1}       & DivisionByZero\\
%    \cline{2-4}
%          & $x=1$          & \M{1}       & $1$\\
%    \cline{2-4}
%          & $x=-1$         & $\opODD(y)$ & $-1$\\
%    \cline{3-4}
%          &                & else        & $1$\\
%    \cline{2-4}
%          & else ($\string|x\string|>1$) & \M{1}       & $0$\\
%    \hline
%    $y>2$ & $x=0$          & \M{1}       & $0$\\
%    \cline{2-4}
%          & $x=1$          & \M{1}       & $1$\\
%    \cline{2-4}
%          & $x=-1$         & $\opODD(y)$ & $-1$\\
%    \cline{3-4}
%          &                & else        & $1$\\
%    \cline{2-4}
%          & $x<-1$ ($x<0$) & $\opODD(y)$ & $-\opPow(-x,y)$\\
%    \cline{3-4}
%          &                & else        & $\opPow(-x,y)$\\
%    \cline{2-4}
%          & else ($x>1$)   & \M{1}       & $\opPow(x,y)$\\
%    \hline
%    \end{tabular}
%    \end{quote}
%    \begin{macrocode}
\def\BIC@PowSwitch#1#2!#3#4!{%
  \ifcase\ifx\\#2\\%
           \ifx#100 % y = 0
           \else\ifx#111 % y = 1
           \else\ifx#122 % y = 2
           \else4 % y > 2
           \fi\fi\fi
         \else
           \ifx#1-3 % y < 0
           \else4 % y > 2
           \fi
         \fi
    \BIC@AfterFi{ 1}% y = 0
  \or % y = 1
    \BIC@AfterFi{ #3#4}%
  \or % y = 2
    \ifx#3-% x < 0
      \BIC@AfterFiFi{%
        \BIC@ProcessMul0!#4!#4!%
      }%
    \else % x >= 0
      \BIC@AfterFiFi{%
        \BIC@ProcessMul0!#3#4!#3#4!%
      }%
    \fi
  \or % y < 0
    \ifcase\ifx\\#4\\%
             \ifx#300 % x = 0
             \else\ifx#311 % x = 1
             \else3 % x > 1
             \fi\fi
           \else
             \ifcase\BIC@MinusOne#3#4! %
               3 % |x| > 1
             \or
               2 % x = -1
?            \else\BigIntCalcError:ThisCannotHappen%
             \fi
           \fi
      \BIC@AfterFiFi{ 0\BigIntCalcError:DivisionByZero}% x = 0
    \or % x = 1
      \BIC@AfterFiFi{ 1}% x = 1
    \or % x = -1
      \ifcase\BIC@ModTwo#2! % even(y)
        \BIC@AfterFiFiFi{ 1}%
      \or % odd(y)
        \BIC@AfterFiFiFi{ -1}%
?     \else\BigIntCalcError:ThisCannotHappen%
      \fi
    \or % |x| > 1
      \BIC@AfterFiFi{ 0}%
?   \else\BigIntCalcError:ThisCannotHappen%
    \fi
  \or % y > 2
    \ifcase\ifx\\#4\\%
             \ifx#300 % x = 0
             \else\ifx#311 % x = 1
             \else4 % x > 1
             \fi\fi
           \else
             \ifx#3-%
               \ifcase\BIC@MinusOne#3#4! %
                 3 % x < -1
               \else
                 2 % x = -1
               \fi
             \else
               4 % x > 1
             \fi
           \fi
      \BIC@AfterFiFi{ 0}% x = 0
    \or % x = 1
      \BIC@AfterFiFi{ 1}% x = 1
    \or % x = -1
      \ifcase\BIC@ModTwo#1#2! % even(y)
        \BIC@AfterFiFiFi{ 1}%
      \or % odd(y)
        \BIC@AfterFiFiFi{ -1}%
?     \else\BigIntCalcError:ThisCannotHappen%
      \fi
    \or % x < -1
      \ifcase\BIC@ModTwo#1#2! % even(y)
        \BIC@AfterFiFiFi{%
          \BIC@PowRec#4!#1#2!1!%
        }%
      \or % odd(y)
        \expandafter-\romannumeral0%
        \BIC@AfterFiFiFi{%
          \BIC@PowRec#4!#1#2!1!%
        }%
?     \else\BigIntCalcError:ThisCannotHappen%
      \fi
    \or % x > 1
      \BIC@AfterFiFi{%
        \BIC@PowRec#3#4!#1#2!1!%
      }%
?   \else\BigIntCalcError:ThisCannotHappen%
    \fi
? \else\BigIntCalcError:ThisCannotHappen%
  \BIC@Fi
}
%    \end{macrocode}
%    \end{macro}
%
% \subsubsection{Help macros}
%
%    \begin{macro}{\BIC@ModTwo}
%    Macro \cs{BIC@ModTwo} expects a number without sign
%    and returns digit |1| or |0| if the number is odd or even.
%    \begin{macrocode}
\def\BIC@ModTwo#1#2!{%
  \ifx\\#2\\%
    \ifodd#1 %
      \BIC@AfterFiFi1%
    \else
      \BIC@AfterFiFi0%
    \fi
  \else
    \BIC@AfterFi{%
      \BIC@ModTwo#2!%
    }%
  \BIC@Fi
}
%    \end{macrocode}
%    \end{macro}
%
%    \begin{macro}{\BIC@MinusOne}
%    Macro \cs{BIC@MinusOne} expects a number and returns
%    digit |1| if the number equals minus one and returns |0| otherwise.
%    \begin{macrocode}
\def\BIC@MinusOne#1#2!{%
  \ifx#1-%
    \BIC@@MinusOne#2!%
  \else
    0%
  \fi
}
%    \end{macrocode}
%    \end{macro}
%    \begin{macro}{\BIC@@MinusOne}
%    \begin{macrocode}
\def\BIC@@MinusOne#1#2!{%
  \ifx#11%
    \ifx\\#2\\%
      1%
    \else
      0%
    \fi
  \else
    0%
  \fi
}
%    \end{macrocode}
%    \end{macro}
%
% \subsubsection{Recursive calculation}
%
%    \begin{macro}{\BIC@PowRec}
%\begin{quote}
%\begin{verbatim}
%Pow(x, y) {
%  PowRec(x, y, 1)
%}
%PowRec(x, y, r) {
%  if y == 1 then
%    return r
%  else
%    ifodd y then
%      return PowRec(x*x, y div 2, r*x) % y div 2 = (y-1)/2
%    else
%      return PowRec(x*x, y div 2, r)
%    fi
%  fi
%}
%\end{verbatim}
%\end{quote}
%    |#1|: $x$ (basis)\\
%    |#2#3|: $y$ (power)\\
%    |#4|: $r$ (result)
%    \begin{macrocode}
\def\BIC@PowRec#1!#2#3!#4!{%
  \ifcase\ifx#21\ifx\\#3\\0 \else1 \fi\else1 \fi % y = 1
    \ifnum\BIC@PosCmp#1!#4!=1 % x > r
      \BIC@AfterFiFi{%
        \BIC@ProcessMul0!#1!#4!%
      }%
    \else
      \BIC@AfterFiFi{%
        \BIC@ProcessMul0!#4!#1!%
      }%
    \fi
  \or
    \ifcase\BIC@ModTwo#2#3! % even(y)
      \BIC@AfterFiFi{%
        \expandafter\BIC@@PowRec\romannumeral0%
        \BIC@@Shr#2#3!%
        !#1!#4!%
      }%
    \or % odd(y)
      \ifnum\BIC@PosCmp#1!#4!=1 % x > r
        \BIC@AfterFiFiFi{%
          \expandafter\BIC@@@PowRec\romannumeral0%
          \BIC@ProcessMul0!#1!#4!%
          !#1!#2#3!%
        }%
      \else
        \BIC@AfterFiFiFi{%
          \expandafter\BIC@@@PowRec\romannumeral0%
          \BIC@ProcessMul0!#1!#4!%
          !#1!#2#3!%
        }%
      \fi
?   \else\BigIntCalcError:ThisCannotHappen%
    \fi
? \else\BigIntCalcError:ThisCannotHappen%
  \BIC@Fi
}
%    \end{macrocode}
%    \end{macro}
%    \begin{macro}{\BIC@@PowRec}
%    |#1|: $y/2$\\
%    |#2|: $x$\\
%    |#3|: new $r$ ($r$ or $r*x$)
%    \begin{macrocode}
\def\BIC@@PowRec#1!#2!#3!{%
  \expandafter\BIC@PowRec\romannumeral0%
  \BIC@ProcessMul0!#2!#2!%
  !#1!#3!%
}
%    \end{macrocode}
%    \end{macro}
%    \begin{macro}{\BIC@@@PowRec}
%    |#1|: $r*x$
%    |#2|: $x$
%    |#3|: $y$
%    \begin{macrocode}
\def\BIC@@@PowRec#1!#2!#3!{%
  \expandafter\BIC@@PowRec\romannumeral0%
  \BIC@@Shr#3!%
  !#2!#1!%
}
%    \end{macrocode}
%    \end{macro}
%
% \subsection{\op{Div}}
%
%    \begin{macro}{\bigintcalcDiv}
%    |#1|: $x$\\
%    |#2|: $y$ (divisor)
%    \begin{macrocode}
\def\bigintcalcDiv#1{%
  \romannumeral0%
  \expandafter\expandafter\expandafter\BIC@Div
  \bigintcalcNum{#1}!%
}
%    \end{macrocode}
%    \end{macro}
%    \begin{macro}{\BIC@Div}
%    |#1|: $x$\\
%    |#2|: $y$
%    \begin{macrocode}
\def\BIC@Div#1!#2{%
  \expandafter\expandafter\expandafter\BIC@DivSwitchSign
  \bigintcalcNum{#2}!#1!%
}
%    \end{macrocode}
%    \end{macro}
%    \begin{macro}{\BigIntCalcDiv}
%    \begin{macrocode}
\def\BigIntCalcDiv#1!#2!{%
  \romannumeral0%
  \BIC@DivSwitchSign#2!#1!%
}
%    \end{macrocode}
%    \end{macro}
%    \begin{macro}{\BIC@DivSwitchSign}
%    Decision table for \cs{BIC@DivSwitchSign}.
%    \begin{quote}
%    \begin{tabular}{@{}|l|l|l|@{}}
%    \hline
%    $y=0$ & \multicolumn{1}{l}{} & DivisionByZero\\
%    \hline
%    $y>0$ & $x=0$ & $0$\\
%    \cline{2-3}
%          & $x>0$ & DivSwitch$(+,x,y)$\\
%    \cline{2-3}
%          & $x<0$ & DivSwitch$(-,-x,y)$\\
%    \hline
%    $y<0$ & $x=0$ & $0$\\
%    \cline{2-3}
%          & $x>0$ & DivSwitch$(-,x,-y)$\\
%    \cline{2-3}
%          & $x<0$ & DivSwitch$(+,-x,-y)$\\
%    \hline
%    \end{tabular}
%    \end{quote}
%    |#1|: $y$ (divisor)\\
%    |#2|: $x$
%    \begin{macrocode}
\def\BIC@DivSwitchSign#1#2!#3#4!{%
  \ifcase\BIC@Sgn#1#2! % y = 0
    \BIC@AfterFi{ 0\BigIntCalcError:DivisionByZero}%
  \or % y > 0
    \ifcase\BIC@Sgn#3#4! % x = 0
      \BIC@AfterFiFi{ 0}%
    \or % x > 0
      \BIC@AfterFiFi{%
        \BIC@DivSwitch{}#3#4!#1#2!%
      }%
    \else % x < 0
      \BIC@AfterFiFi{%
        \BIC@DivSwitch-#4!#1#2!%
      }%
    \fi
  \else % y < 0
    \ifcase\BIC@Sgn#3#4! % x = 0
      \BIC@AfterFiFi{ 0}%
    \or % x > 0
      \BIC@AfterFiFi{%
        \BIC@DivSwitch-#3#4!#2!%
      }%
    \else % x < 0
      \BIC@AfterFiFi{%
        \BIC@DivSwitch{}#4!#2!%
      }%
    \fi
  \BIC@Fi
}
%    \end{macrocode}
%    \end{macro}
%    \begin{macro}{\BIC@DivSwitch}
%    Decision table for \cs{BIC@DivSwitch}.
%    \begin{quote}
%    \begin{tabular}{@{}|l|l|l|@{}}
%    \hline
%    $y=x$ & \multicolumn{1}{l}{} & sign $1$\\
%    \hline
%    $y>x$ & \multicolumn{1}{l}{} & $0$\\
%    \hline
%    $y<x$ & $y=1$                & sign $x$\\
%    \cline{2-3}
%          & $y=2$                & sign Shr$(x)$\\
%    \cline{2-3}
%          & $y=4$                & sign Shr(Shr$(x)$)\\
%    \cline{2-3}
%          & else                 & sign ProcessDiv$(x,y)$\\
%    \hline
%    \end{tabular}
%    \end{quote}
%    |#1|: sign\\
%    |#2|: $x$\\
%    |#3#4|: $y$ ($y\ne 0$)
%    \begin{macrocode}
\def\BIC@DivSwitch#1#2!#3#4!{%
  \ifcase\BIC@PosCmp#3#4!#2!% y = x
    \BIC@AfterFi{ #11}%
  \or % y > x
    \BIC@AfterFi{ 0}%
  \else % y < x
    \ifx\\#1\\%
    \else
      \expandafter-\romannumeral0%
    \fi
    \ifcase\ifx\\#4\\%
             \ifx#310 % y = 1
             \else\ifx#321 % y = 2
             \else\ifx#342 % y = 4
             \else3 % y > 2
             \fi\fi\fi
           \else
             3 % y > 2
           \fi
      \BIC@AfterFiFi{ #2}% y = 1
    \or % y = 2
      \BIC@AfterFiFi{%
        \BIC@@Shr#2!%
      }%
    \or % y = 4
      \BIC@AfterFiFi{%
        \expandafter\BIC@@Shr\romannumeral0%
          \BIC@@Shr#2!!%
      }%
    \or % y > 2
      \BIC@AfterFiFi{%
        \BIC@DivStartX#2!#3#4!!!%
      }%
?   \else\BigIntCalcError:ThisCannotHappen%
    \fi
  \BIC@Fi
}
%    \end{macrocode}
%    \end{macro}
%    \begin{macro}{\BIC@ProcessDiv}
%    |#1#2|: $x$\\
%    |#3#4|: $y$\\
%    |#5|: collect first digits of $x$\\
%    |#6|: corresponding digits of $y$
%    \begin{macrocode}
\def\BIC@DivStartX#1#2!#3#4!#5!#6!{%
  \ifx\\#4\\%
    \BIC@AfterFi{%
      \BIC@DivStartYii#6#3#4!{#5#1}#2=!%
    }%
  \else
    \BIC@AfterFi{%
      \BIC@DivStartX#2!#4!#5#1!#6#3!%
    }%
  \BIC@Fi
}
%    \end{macrocode}
%    \end{macro}
%    \begin{macro}{\BIC@DivStartYii}
%    |#1|: $y$\\
%    |#2|: $x$, |=|
%    \begin{macrocode}
\def\BIC@DivStartYii#1!{%
  \expandafter\BIC@DivStartYiv\romannumeral0%
  \BIC@Shl#1!%
  !#1!%
}
%    \end{macrocode}
%    \end{macro}
%    \begin{macro}{\BIC@DivStartYiv}
%    |#1|: $2y$\\
%    |#2|: $y$\\
%    |#3|: $x$, |=|
%    \begin{macrocode}
\def\BIC@DivStartYiv#1!{%
  \expandafter\BIC@DivStartYvi\romannumeral0%
  \BIC@Shl#1!%
  !#1!%
}
%    \end{macrocode}
%    \end{macro}
%    \begin{macro}{\BIC@DivStartYvi}
%    |#1|: $4y$\\
%    |#2|: $2y$\\
%    |#3|: $y$\\
%    |#4|: $x$, |=|
%    \begin{macrocode}
\def\BIC@DivStartYvi#1!#2!{%
  \expandafter\BIC@DivStartYviii\romannumeral0%
  \BIC@AddXY#1!#2!!!%
  !#1!#2!%
}
%    \end{macrocode}
%    \end{macro}
%    \begin{macro}{\BIC@DivStartYviii}
%    |#1|: $6y$\\
%    |#2|: $4y$\\
%    |#3|: $2y$\\
%    |#4|: $y$\\
%    |#5|: $x$, |=|
%    \begin{macrocode}
\def\BIC@DivStartYviii#1!#2!{%
  \expandafter\BIC@DivStart\romannumeral0%
  \BIC@Shl#2!%
  !#1!#2!%
}
%    \end{macrocode}
%    \end{macro}
%    \begin{macro}{\BIC@DivStart}
%    |#1|: $8y$\\
%    |#2|: $6y$\\
%    |#3|: $4y$\\
%    |#4|: $2y$\\
%    |#5|: $y$\\
%    |#6|: $x$, |=|
%    \begin{macrocode}
\def\BIC@DivStart#1!#2!#3!#4!#5!#6!{%
  \BIC@ProcessDiv#6!!#5!#4!#3!#2!#1!=%
}
%    \end{macrocode}
%    \end{macro}
%    \begin{macro}{\BIC@ProcessDiv}
%    |#1#2#3|: $x$, |=|\\
%    |#4|: result\\
%    |#5|: $y$\\
%    |#6|: $2y$\\
%    |#7|: $4y$\\
%    |#8|: $6y$\\
%    |#9|: $8y$
%    \begin{macrocode}
\def\BIC@ProcessDiv#1#2#3!#4!#5!{%
  \ifcase\BIC@PosCmp#5!#1!% y = #1
    \ifx#2=%
      \BIC@AfterFiFi{\BIC@DivCleanup{#41}}%
    \else
      \BIC@AfterFiFi{%
        \BIC@ProcessDiv#2#3!#41!#5!%
      }%
    \fi
  \or % y > #1
    \ifx#2=%
      \BIC@AfterFiFi{\BIC@DivCleanup{#40}}%
    \else
      \ifx\\#4\\%
        \BIC@AfterFiFiFi{%
          \BIC@ProcessDiv{#1#2}#3!!#5!%
        }%
      \else
        \BIC@AfterFiFiFi{%
          \BIC@ProcessDiv{#1#2}#3!#40!#5!%
        }%
      \fi
    \fi
  \else % y < #1
    \BIC@AfterFi{%
      \BIC@@ProcessDiv{#1}#2#3!#4!#5!%
    }%
  \BIC@Fi
}
%    \end{macrocode}
%    \end{macro}
%    \begin{macro}{\BIC@DivCleanup}
%    |#1|: result\\
%    |#2|: garbage
%    \begin{macrocode}
\def\BIC@DivCleanup#1#2={ #1}%
%    \end{macrocode}
%    \end{macro}
%    \begin{macro}{\BIC@@ProcessDiv}
%    \begin{macrocode}
\def\BIC@@ProcessDiv#1#2#3!#4!#5!#6!#7!{%
  \ifcase\BIC@PosCmp#7!#1!% 4y = #1
    \ifx#2=%
      \BIC@AfterFiFi{\BIC@DivCleanup{#44}}%
    \else
     \BIC@AfterFiFi{%
       \BIC@ProcessDiv#2#3!#44!#5!#6!#7!%
     }%
    \fi
  \or % 4y > #1
    \ifcase\BIC@PosCmp#6!#1!% 2y = #1
      \ifx#2=%
        \BIC@AfterFiFiFi{\BIC@DivCleanup{#42}}%
      \else
        \BIC@AfterFiFiFi{%
          \BIC@ProcessDiv#2#3!#42!#5!#6!#7!%
        }%
      \fi
    \or % 2y > #1
      \ifx#2=%
        \BIC@AfterFiFiFi{\BIC@DivCleanup{#41}}%
      \else
        \BIC@AfterFiFiFi{%
          \BIC@DivSub#1!#5!#2#3!#41!#5!#6!#7!%
        }%
      \fi
    \else % 2y < #1
      \BIC@AfterFiFi{%
        \expandafter\BIC@ProcessDivII\romannumeral0%
        \BIC@SubXY#1!#6!!!%
        !#2#3!#4!#5!23%
        #6!#7!%
      }%
    \fi
  \else % 4y < #1
    \BIC@AfterFi{%
      \BIC@@@ProcessDiv{#1}#2#3!#4!#5!#6!#7!%
    }%
  \BIC@Fi
}
%    \end{macrocode}
%    \end{macro}
%    \begin{macro}{\BIC@DivSub}
%    Next token group: |#1|-|#2| and next digit |#3|.
%    \begin{macrocode}
\def\BIC@DivSub#1!#2!#3{%
  \expandafter\BIC@ProcessDiv\expandafter{%
    \romannumeral0%
    \BIC@SubXY#1!#2!!!%
    #3%
  }%
}
%    \end{macrocode}
%    \end{macro}
%    \begin{macro}{\BIC@ProcessDivII}
%    |#1|: $x'-2y$\\
%    |#2#3|: remaining $x$, |=|\\
%    |#4|: result\\
%    |#5|: $y$\\
%    |#6|: first possible result digit\\
%    |#7|: second possible result digit
%    \begin{macrocode}
\def\BIC@ProcessDivII#1!#2#3!#4!#5!#6#7{%
  \ifcase\BIC@PosCmp#5!#1!% y = #1
    \ifx#2=%
      \BIC@AfterFiFi{\BIC@DivCleanup{#4#7}}%
    \else
      \BIC@AfterFiFi{%
        \BIC@ProcessDiv#2#3!#4#7!#5!%
      }%
    \fi
  \or % y > #1
    \ifx#2=%
      \BIC@AfterFiFi{\BIC@DivCleanup{#4#6}}%
    \else
      \BIC@AfterFiFi{%
        \BIC@ProcessDiv{#1#2}#3!#4#6!#5!%
      }%
    \fi
  \else % y < #1
    \ifx#2=%
      \BIC@AfterFiFi{\BIC@DivCleanup{#4#7}}%
    \else
      \BIC@AfterFiFi{%
        \BIC@DivSub#1!#5!#2#3!#4#7!#5!%
      }%
    \fi
  \BIC@Fi
}
%    \end{macrocode}
%    \end{macro}
%    \begin{macro}{\BIC@ProcessDivIV}
%    |#1#2#3|: $x$, |=|, $x>4y$\\
%    |#4|: result\\
%    |#5|: $y$\\
%    |#6|: $2y$\\
%    |#7|: $4y$\\
%    |#8|: $6y$\\
%    |#9|: $8y$
%    \begin{macrocode}
\def\BIC@@@ProcessDiv#1#2#3!#4!#5!#6!#7!#8!#9!{%
  \ifcase\BIC@PosCmp#8!#1!% 6y = #1
    \ifx#2=%
      \BIC@AfterFiFi{\BIC@DivCleanup{#46}}%
    \else
      \BIC@AfterFiFi{%
        \BIC@ProcessDiv#2#3!#46!#5!#6!#7!#8!#9!%
      }%
    \fi
  \or % 6y > #1
    \BIC@AfterFi{%
      \expandafter\BIC@ProcessDivII\romannumeral0%
      \BIC@SubXY#1!#7!!!%
      !#2#3!#4!#5!45%
      #6!#7!#8!#9!%
    }%
  \else % 6y < #1
    \ifcase\BIC@PosCmp#9!#1!% 8y = #1
      \ifx#2=%
        \BIC@AfterFiFiFi{\BIC@DivCleanup{#48}}%
      \else
        \BIC@AfterFiFiFi{%
          \BIC@ProcessDiv#2#3!#48!#5!#6!#7!#8!#9!%
        }%
      \fi
    \or % 8y > #1
      \BIC@AfterFiFi{%
        \expandafter\BIC@ProcessDivII\romannumeral0%
        \BIC@SubXY#1!#8!!!%
        !#2#3!#4!#5!67%
        #6!#7!#8!#9!%
      }%
    \else % 8y < #1
      \BIC@AfterFiFi{%
        \expandafter\BIC@ProcessDivII\romannumeral0%
        \BIC@SubXY#1!#9!!!%
        !#2#3!#4!#5!89%
        #6!#7!#8!#9!%
      }%
    \fi
  \BIC@Fi
}
%    \end{macrocode}
%    \end{macro}
%
% \subsection{\op{Mod}}
%
%    \begin{macro}{\bigintcalcMod}
%    |#1|: $x$\\
%    |#2|: $y$
%    \begin{macrocode}
\def\bigintcalcMod#1{%
  \romannumeral0%
  \expandafter\expandafter\expandafter\BIC@Mod
  \bigintcalcNum{#1}!%
}
%    \end{macrocode}
%    \end{macro}
%    \begin{macro}{\BIC@Mod}
%    |#1|: $x$\\
%    |#2|: $y$
%    \begin{macrocode}
\def\BIC@Mod#1!#2{%
  \expandafter\expandafter\expandafter\BIC@ModSwitchSign
  \bigintcalcNum{#2}!#1!%
}
%    \end{macrocode}
%    \end{macro}
%    \begin{macro}{\BigIntCalcMod}
%    \begin{macrocode}
\def\BigIntCalcMod#1!#2!{%
  \romannumeral0%
  \BIC@ModSwitchSign#2!#1!%
}
%    \end{macrocode}
%    \end{macro}
%    \begin{macro}{\BIC@ModSwitchSign}
%    Decision table for \cs{BIC@ModSwitchSign}.
%    \begin{quote}
%    \begin{tabular}{@{}|l|l|l|@{}}
%    \hline
%    $y=0$ & \multicolumn{1}{l}{} & DivisionByZero\\
%    \hline
%    $y>0$ & $x=0$ & $0$\\
%    \cline{2-3}
%          & else  & ModSwitch$(+,x,y)$\\
%    \hline
%    $y<0$ & \multicolumn{1}{l}{} & ModSwitch$(-,-x,-y)$\\
%    \hline
%    \end{tabular}
%    \end{quote}
%    |#1#2|: $y$\\
%    |#3#4|: $x$
%    \begin{macrocode}
\def\BIC@ModSwitchSign#1#2!#3#4!{%
  \ifcase\ifx\\#2\\%
           \ifx#100 % y = 0
           \else1 % y > 0
           \fi
          \else
            \ifx#1-2 % y < 0
            \else1 % y > 0
            \fi
          \fi
    \BIC@AfterFi{ 0\BigIntCalcError:DivisionByZero}%
  \or % y > 0
    \ifcase\ifx\\#4\\\ifx#300 \else1 \fi\else1 \fi % x = 0
      \BIC@AfterFiFi{ 0}%
    \else
      \BIC@AfterFiFi{%
        \BIC@ModSwitch{}#3#4!#1#2!%
      }%
    \fi
  \else % y < 0
    \ifcase\ifx\\#4\\%
             \ifx#300 % x = 0
             \else1 % x > 0
             \fi
           \else
             \ifx#3-2 % x < 0
             \else1 % x > 0
             \fi
           \fi
      \BIC@AfterFiFi{ 0}%
    \or % x > 0
      \BIC@AfterFiFi{%
        \BIC@ModSwitch--#3#4!#2!%
      }%
    \else % x < 0
      \BIC@AfterFiFi{%
        \BIC@ModSwitch-#4!#2!%
      }%
    \fi
  \BIC@Fi
}
%    \end{macrocode}
%    \end{macro}
%    \begin{macro}{\BIC@ModSwitch}
%    Decision table for \cs{BIC@ModSwitch}.
%    \begin{quote}
%    \begin{tabular}{@{}|l|l|l|@{}}
%    \hline
%    $y=1$ & \multicolumn{1}{l}{} & $0$\\
%    \hline
%    $y=2$ & ifodd$(x)$ & sign $1$\\
%    \cline{2-3}
%          & else       & $0$\\
%    \hline
%    $y>2$ & $x<0$      &
%        $z\leftarrow x-(x/y)*y;\quad (z<0)\mathbin{?}z+y\mathbin{:}z$\\
%    \cline{2-3}
%          & $x>0$      & $x - (x/y) * y$\\
%    \hline
%    \end{tabular}
%    \end{quote}
%    |#1|: sign\\
%    |#2#3|: $x$\\
%    |#4#5|: $y$
%    \begin{macrocode}
\def\BIC@ModSwitch#1#2#3!#4#5!{%
  \ifcase\ifx\\#5\\%
           \ifx#410 % y = 1
           \else\ifx#421 % y = 2
           \else2 % y > 2
           \fi\fi
         \else2 % y > 2
         \fi
    \BIC@AfterFi{ 0}% y = 1
  \or % y = 2
    \ifcase\BIC@ModTwo#2#3! % even(x)
      \BIC@AfterFiFi{ 0}%
    \or % odd(x)
      \BIC@AfterFiFi{ #11}%
?   \else\BigIntCalcError:ThisCannotHappen%
    \fi
  \or % y > 2
    \ifx\\#1\\%
    \else
      \expandafter\BIC@Space\romannumeral0%
      \expandafter\BIC@ModMinus\romannumeral0%
    \fi
    \ifx#2-% x < 0
      \BIC@AfterFiFi{%
        \expandafter\expandafter\expandafter\BIC@ModX
        \bigintcalcSub{#2#3}{%
          \bigintcalcMul{#4#5}{\bigintcalcDiv{#2#3}{#4#5}}%
        }!#4#5!%
      }%
    \else % x > 0
      \BIC@AfterFiFi{%
        \expandafter\expandafter\expandafter\BIC@Space
        \bigintcalcSub{#2#3}{%
          \bigintcalcMul{#4#5}{\bigintcalcDiv{#2#3}{#4#5}}%
        }%
      }%
    \fi
? \else\BigIntCalcError:ThisCannotHappen%
  \BIC@Fi
}
%    \end{macrocode}
%    \end{macro}
%    \begin{macro}{\BIC@ModMinus}
%    \begin{macrocode}
\def\BIC@ModMinus#1{%
  \ifx#10%
    \BIC@AfterFi{ 0}%
  \else
    \BIC@AfterFi{ -#1}%
  \BIC@Fi
}
%    \end{macrocode}
%    \end{macro}
%    \begin{macro}{\BIC@ModX}
%    |#1#2|: $z$\\
%    |#3|: $x$
%    \begin{macrocode}
\def\BIC@ModX#1#2!#3!{%
  \ifx#1-% z < 0
    \BIC@AfterFi{%
      \expandafter\BIC@Space\romannumeral0%
      \BIC@SubXY#3!#2!!!%
    }%
  \else % z >= 0
    \BIC@AfterFi{ #1#2}%
  \BIC@Fi
}
%    \end{macrocode}
%    \end{macro}
%
%    \begin{macrocode}
\BIC@AtEnd%
%    \end{macrocode}
%
%    \begin{macrocode}
%</package>
%    \end{macrocode}
%
% \section{Test}
%
% \subsection{Catcode checks for loading}
%
%    \begin{macrocode}
%<*test1>
%    \end{macrocode}
%    \begin{macrocode}
\catcode`\{=1 %
\catcode`\}=2 %
\catcode`\#=6 %
\catcode`\@=11 %
\expandafter\ifx\csname count@\endcsname\relax
  \countdef\count@=255 %
\fi
\expandafter\ifx\csname @gobble\endcsname\relax
  \long\def\@gobble#1{}%
\fi
\expandafter\ifx\csname @firstofone\endcsname\relax
  \long\def\@firstofone#1{#1}%
\fi
\expandafter\ifx\csname loop\endcsname\relax
  \expandafter\@firstofone
\else
  \expandafter\@gobble
\fi
{%
  \def\loop#1\repeat{%
    \def\body{#1}%
    \iterate
  }%
  \def\iterate{%
    \body
      \let\next\iterate
    \else
      \let\next\relax
    \fi
    \next
  }%
  \let\repeat=\fi
}%
\def\RestoreCatcodes{}
\count@=0 %
\loop
  \edef\RestoreCatcodes{%
    \RestoreCatcodes
    \catcode\the\count@=\the\catcode\count@\relax
  }%
\ifnum\count@<255 %
  \advance\count@ 1 %
\repeat

\def\RangeCatcodeInvalid#1#2{%
  \count@=#1\relax
  \loop
    \catcode\count@=15 %
  \ifnum\count@<#2\relax
    \advance\count@ 1 %
  \repeat
}
\def\RangeCatcodeCheck#1#2#3{%
  \count@=#1\relax
  \loop
    \ifnum#3=\catcode\count@
    \else
      \errmessage{%
        Character \the\count@\space
        with wrong catcode \the\catcode\count@\space
        instead of \number#3%
      }%
    \fi
  \ifnum\count@<#2\relax
    \advance\count@ 1 %
  \repeat
}
\def\space{ }
\expandafter\ifx\csname LoadCommand\endcsname\relax
  \def\LoadCommand{\input bigintcalc.sty\relax}%
\fi
\def\Test{%
  \RangeCatcodeInvalid{0}{47}%
  \RangeCatcodeInvalid{58}{64}%
  \RangeCatcodeInvalid{91}{96}%
  \RangeCatcodeInvalid{123}{255}%
  \catcode`\@=12 %
  \catcode`\\=0 %
  \catcode`\%=14 %
  \LoadCommand
  \RangeCatcodeCheck{0}{36}{15}%
  \RangeCatcodeCheck{37}{37}{14}%
  \RangeCatcodeCheck{38}{47}{15}%
  \RangeCatcodeCheck{48}{57}{12}%
  \RangeCatcodeCheck{58}{63}{15}%
  \RangeCatcodeCheck{64}{64}{12}%
  \RangeCatcodeCheck{65}{90}{11}%
  \RangeCatcodeCheck{91}{91}{15}%
  \RangeCatcodeCheck{92}{92}{0}%
  \RangeCatcodeCheck{93}{96}{15}%
  \RangeCatcodeCheck{97}{122}{11}%
  \RangeCatcodeCheck{123}{255}{15}%
  \RestoreCatcodes
}
\Test
\csname @@end\endcsname
\end
%    \end{macrocode}
%    \begin{macrocode}
%</test1>
%    \end{macrocode}
%
% \subsection{Macro tests}
%
% \subsubsection{Preamble with test macro definitions}
%
%    \begin{macrocode}
%<*test2>
\NeedsTeXFormat{LaTeX2e}
\nofiles
\documentclass{article}
%<noetex>\let\SavedNumexpr\numexpr
%<noetex>\let\numexpr\UNDEFINED
\makeatletter
\chardef\BIC@TestMode=1 %
\makeatother
\usepackage{bigintcalc}[2016/05/16]
%<noetex>\let\numexpr\SavedNumexpr
\usepackage{qstest}
\IncludeTests{*}
\LogTests{log}{*}{*}
\newcommand*{\TestSpaceAtEnd}[1]{%
%<noetex>  \let\SavedNumexpr\numexpr
%<noetex>  \let\numexpr\UNDEFINED
  \edef\resultA{#1}%
  \edef\resultB{#1 }%
%<noetex>  \let\numexpr\SavedNumexpr
  \Expect*{\resultA\space}*{\resultB}%
}
\newcommand*{\TestResult}[2]{%
%<noetex>  \let\SavedNumexpr\numexpr
%<noetex>  \let\numexpr\UNDEFINED
  \edef\result{#1}%
%<noetex>  \let\numexpr\SavedNumexpr
  \Expect*{\result}{#2}%
}
\newcommand*{\TestResultTwoExpansions}[2]{%
%<*noetex>
  \begingroup
    \let\numexpr\UNDEFINED
    \expandafter\expandafter\expandafter
  \endgroup
%</noetex>
  \expandafter\expandafter\expandafter\Expect
  \expandafter\expandafter\expandafter{#1}{#2}%
}
\newcount\TestCount
%<etex>\newcommand*{\TestArg}[1]{\numexpr#1\relax}
%<noetex>\newcommand*{\TestArg}[1]{#1}
\newcommand*{\TestTeXDivide}[2]{%
  \TestCount=\TestArg{#1}\relax
  \divide\TestCount by \TestArg{#2}\relax
  \Expect*{\bigintcalcDiv{#1}{#2}}*{\the\TestCount}%
}
\newcommand*{\Test}[2]{%
  \TestResult{#1}{#2}%
  \TestResultTwoExpansions{#1}{#2}%
  \TestSpaceAtEnd{#1}%
}
\newcommand*{\TestExch}[2]{\Test{#2}{#1}}
\newcommand*{\TestInv}[2]{%
  \Test{\bigintcalcInv{#1}}{#2}%
}
\newcommand*{\TestAbs}[2]{%
  \Test{\bigintcalcAbs{#1}}{#2}%
}
\newcommand*{\TestSgn}[2]{%
  \Test{\bigintcalcSgn{#1}}{#2}%
}
\newcommand*{\TestMin}[3]{%
  \Test{\bigintcalcMin{#1}{#2}}{#3}%
}
\newcommand*{\TestMax}[3]{%
  \Test{\bigintcalcMax{#1}{#2}}{#3}%
}
\newcommand*{\TestCmp}[3]{%
  \Test{\bigintcalcCmp{#1}{#2}}{#3}%
}
\newcommand*{\TestOdd}[2]{%
  \Test{\bigintcalcOdd{#1}}{#2}%
  \edef\x{%
    \noexpand\Test{%
      \noexpand\BigIntCalcOdd
      \bigintcalcAbs{#1}!%
    }{#2}%
  }%
  \x
}
\newcommand*{\TestInc}[2]{%
  \Test{\bigintcalcInc{#1}}{#2}%
  \ifnum\bigintcalcSgn{#1}>-1 %
    \edef\x{%
      \noexpand\Test{%
        \noexpand\BigIntCalcInc\bigintcalcNum{#1}!%
      }{#2}%
    }%
    \x
  \fi
}
\newcommand*{\TestDec}[2]{%
  \Test{\bigintcalcDec{#1}}{#2}%
  \ifnum\bigintcalcSgn{#1}>0 %
    \edef\x{%
      \noexpand\Test{%
        \noexpand\BigIntCalcDec\bigintcalcNum{#1}!%
      }{#2}%
    }%
    \x
  \fi
}
\newcommand*{\TestAdd}[3]{%
  \Test{\bigintcalcAdd{#1}{#2}}{#3}%
  \ifnum\bigintcalcSgn{#1}>0 %
    \ifnum\bigintcalcSgn{#2}> 0 %
      \ifnum\bigintcalcCmp{#1}{#2}>0 %
        \edef\x{%
          \noexpand\Test{%
            \noexpand\BigIntCalcAdd
            \bigintcalcNum{#1}!\bigintcalcNum{#2}!%
          }{#3}%
        }%
        \x
      \else
        \edef\x{%
          \noexpand\Test{%
            \noexpand\BigIntCalcAdd
            \bigintcalcNum{#2}!\bigintcalcNum{#1}!%
          }{#3}%
        }%
        \x
      \fi
    \fi
  \fi
}
\newcommand*{\TestSub}[3]{%
  \Test{\bigintcalcSub{#1}{#2}}{#3}%
  \ifnum\bigintcalcSgn{#1}>0 %
    \ifnum\bigintcalcSgn{#2}> 0 %
      \ifnum\bigintcalcCmp{#1}{#2}>0 %
        \edef\x{%
          \noexpand\Test{%
            \noexpand\BigIntCalcSub
            \bigintcalcNum{#1}!\bigintcalcNum{#2}!%
          }{#3}%
        }%
        \x
      \fi
    \fi
  \fi
}
\newcommand*{\TestShl}[2]{%
  \Test{\bigintcalcShl{#1}}{#2}%
  \edef\x{%
    \noexpand\Test{%
      \noexpand\BigIntCalcShl\bigintcalcAbs{#1}!%
    }{\bigintcalcAbs{#2}}%
  }%
  \x
}
\newcommand*{\TestShr}[2]{%
  \Test{\bigintcalcShr{#1}}{#2}%
  \edef\x{%
    \noexpand\Test{%
      \noexpand\BigIntCalcShr\bigintcalcAbs{#1}!%
    }{\bigintcalcAbs{#2}}%
  }%
  \x
}
\newcommand*{\TestMul}[3]{%
  \Test{\bigintcalcMul{#1}{#2}}{#3}%
  \edef\x{%
    \noexpand\Test{%
      \noexpand\BigIntCalcMul
      \bigintcalcAbs{#1}!\bigintcalcAbs{#2}!%
    }{\bigintcalcAbs{#3}}%
  }%
  \x
}
\newcommand*{\TestSqr}[2]{%
  \Test{\bigintcalcSqr{#1}}{#2}%
}
\newcommand*{\TestFac}[2]{%
  \expandafter\TestExch\expandafter{%
    \the\numexpr#2%
  }{\bigintcalcFac{#1}}%
}
\newcommand*{\TestFacBig}[2]{%
  \Test{\bigintcalcFac{#1}}{#2}%
}
\newcommand*{\TestPow}[3]{%
  \Test{\bigintcalcPow{#1}{#2}}{#3}%
}
\newcommand*{\TestDiv}[3]{%
  \Test{\bigintcalcDiv{#1}{#2}}{#3}%
  \TestTeXDivide{#1}{#2}%
}
\newcommand*{\TestDivBig}[3]{%
  \Test{\bigintcalcDiv{#1}{#2}}{#3}%
  \edef\x{%
    \noexpand\Test{%
      \noexpand\BigIntCalcDiv\bigintcalcAbs{#1}!\bigintcalcAbs{#2}!%
    }{\bigintcalcAbs{#3}}%
  }%
}
\newcommand*{\TestMod}[3]{%
  \Test{\bigintcalcMod{#1}{#2}}{#3}%
  \ifcase\ifcase\bigintcalcSgn{#1} 0%
         \or
           \ifcase\bigintcalcSgn{#2} 1%
           \or 0%
           \else 1%
           \fi
         \else
           \ifcase\bigintcalcSgn{#2} 1%
           \or 1%
           \else 0%
           \fi
         \fi\relax
    \edef\x{%
      \noexpand\Test{%
        \noexpand\BigIntCalcMod
        \bigintcalcAbs{#1}!\bigintcalcAbs{#2}!%
      }{\bigintcalcAbs{#3}}%
    }%
    \x
  \fi
}
%    \end{macrocode}
%
% \subsubsection{Time}
%
%    \begin{macrocode}
\begingroup\expandafter\expandafter\expandafter\endgroup
\expandafter\ifx\csname pdfresettimer\endcsname\relax
\else
  \makeatletter
  \newcount\SummaryTime
  \newcount\TestTime
  \SummaryTime=\z@
  \newcommand*{\PrintTime}[2]{%
    \typeout{%
      [Time #1: \strip@pt\dimexpr\number#2sp\relax\space s]%
    }%
  }%
  \newcommand*{\StartTime}[1]{%
    \renewcommand*{\TimeDescription}{#1}%
    \pdfresettimer
  }%
  \newcommand*{\TimeDescription}{}%
  \newcommand*{\StopTime}{%
    \TestTime=\pdfelapsedtime
    \global\advance\SummaryTime\TestTime
    \PrintTime\TimeDescription\TestTime
  }%
  \let\saved@qstest\qstest
  \let\saved@endqstest\endqstest
  \def\qstest#1#2{%
    \saved@qstest{#1}{#2}%
    \StartTime{#1}%
  }%
  \def\endqstest{%
    \StopTime
    \saved@endqstest
  }%
  \AtEndDocument{%
    \PrintTime{summary}\SummaryTime
  }%
  \makeatother
\fi
%    \end{macrocode}
%
% \subsubsection{Test sets}
%
%    \begin{macrocode}
\makeatletter

\begin{qstest}{inv}{inv}%
  \TestInv{0}{0}%
  \TestInv{1}{-1}%
  \TestInv{-1}{1}%
  \TestInv{10}{-10}%
  \TestInv{-10}{10}%
  \TestInv{2147483647}{-2147483647}%
  \TestInv{-2147483647}{2147483647}%
  \TestInv{12345678901234567890}{-12345678901234567890}%
  \TestInv{-12345678901234567890}{12345678901234567890}%
  \TestInv{ 0 }{0}%
  \TestInv{ 1 }{-1}%
  \TestInv{--1}{-1}%
  \TestInv{\number\z@}{0}%
  \TestInv{\ifx\relax\relax1\fi}{-1}%
  \TestInv{\ifx\relax\relax-\fi\ifx234\else1\fi}{1}%
\end{qstest}

\begin{qstest}{abs}{abs}%
  \TestAbs{0}{0}%
  \TestAbs{1}{1}%
  \TestAbs{-1}{1}%
  \TestAbs{10}{10}%
  \TestAbs{-10}{10}%
  \TestAbs{2147483647}{2147483647}%
  \TestAbs{-2147483647}{2147483647}%
  \TestAbs{12345678901234567890}{12345678901234567890}%
  \TestAbs{-12345678901234567890}{12345678901234567890}%
  \TestAbs{ 0 }{0}%
  \TestAbs{ 1 }{1}%
  \TestAbs{--1}{1}%
  \TestAbs{-+-+1}{1}%
  \TestAbs{00000000000}{0}%
  \TestAbs{00000001000}{1000}%
  \TestAbs{\ifx\relax\relax 0\else 1\fi}{0}%
\end{qstest}

\begin{qstest}{sign}{sign}%
  \TestSgn{0}{0}%
  \TestSgn{1}{1}%
  \TestSgn{-1}{-1}%
  \TestSgn{10}{1}%
  \TestSgn{-10}{-1}%
  \TestSgn{2147483647}{1}%
  \TestSgn{-2147483647}{-1}%
  \TestSgn{12345678901234567890}{1}%
  \TestSgn{-12345678901234567890}{-1}%
  \TestSgn{ 0 }{0}%
  \TestSgn{ 2 }{1}%
  \TestSgn{ -2 }{-1}%
  \TestSgn{--2}{1}%
  \TestSgn{\number\z@}{0}%
  \TestSgn{\number\@ne}{1}%
  \TestSgn{\number\m@ne}{-1}%
  \TestSgn{%
    -+-+\number\z@\number\z@
    \iftrue1\fi\iftrue2\fi\iftrue3\fi
  }{1}%
\end{qstest}

\begin{qstest}{min}{min}%
  \TestMin{0}{1}{0}%
  \TestMin{1}{0}{0}%
  \TestMin{-10}{-20}{-20}%
  \TestMin{ 1 }{ 2 }{1}%
  \TestMin{ 2 }{ 1 }{1}%
  \TestMin{1}{1}{1}%
  \TestMin{\number\z@}{\number\@ne}{0}%
  \TestMin{\number\@ne}{\number\m@ne}{-1}%
\end{qstest}

\begin{qstest}{max}{max}%
  \TestMax{0}{1}{1}%
  \TestMax{1}{0}{1}%
  \TestMax{-10}{-20}{-10}%
  \TestMax{ 1 }{ 2 }{2}%
  \TestMax{ 2 }{ 1 }{2}%
  \TestMax{1}{1}{1}%
  \TestMax{\number\z@}{\number\@ne}{1}%
  \TestMax{\number\@ne}{\number\m@ne}{1}%
\end{qstest}

\begin{qstest}{cmp}{cmp}%
  \TestCmp{0}{0}{0}%
  \TestCmp{-21}{17}{-1}%
  \TestCmp{3}{4}{-1}%
  \TestCmp{-10}{-10}{0}%
  \TestCmp{-10}{-11}{1}%
  \TestCmp{100}{5}{1}%
  \TestCmp{9}{10}{-1}%
  \TestCmp{10}{9}{1}%
  \TestCmp{ 3 }{ 3 }{0}%
  \TestCmp{-9}{-10}{1}%
  \TestCmp{-10}{-9}{-1}%
  \TestCmp{-3}{-3}{0}%
  \TestCmp{0}{-2}{1}%
  \TestCmp{0}{2}{-1}%
  \TestCmp{2}{0}{1}%
  \TestCmp{-2}{0}{-1}%
  \TestCmp{12}{11}{1}%
  \TestCmp{11}{12}{-1}%
  \TestCmp{2147483647}{-2147483647}{1}%
  \TestCmp{-2147483647}{2147483647}{-1}%
  \TestCmp{2147483647}{2147483647}{0}%
  \TestCmp{\number\z@}{\number\@ne}{-1}%
  \TestCmp{\number\@ne}{\number\m@ne}{1}%
  \TestCmp{ 4 }{ 5 }{-1}%
  \TestCmp{ -3 }{ -7 }{1}%
\end{qstest}

\begin{qstest}{odd}{odd}
\tracingmacros=1
  \TestOdd{0}{0}%
  \TestOdd{1}{1}%
  \TestOdd{2}{0}%
  \TestOdd{3}{1}%
  \TestOdd{14}{0}%
  \TestOdd{15}{1}%
  \TestOdd{12345678901234567896}{0}%
  \TestOdd{12345678901234567897}{1}%
\end{qstest}

\begin{qstest}{inc}{inc}%
  \TestInc{0}{1}%
  \TestInc{1}{2}%
  \TestInc{-1}{0}%
  \TestInc{10}{11}%
  \TestInc{-10}{-9}%
  \TestInc{ 3 }{4}%
  \TestInc{999}{1000}%
  \TestInc{-1000}{-999}%
  \TestInc{129}{130}%
  \TestInc{2147483646}{2147483647}%
  \TestInc{-2147483647}{-2147483646}%
  \TestInc{12345678901234567890}{12345678901234567891}%
  \TestInc{99999999999999999999}{100000000000000000000}%
  \TestInc{-12345678901234567891}{-12345678901234567890}%
  \TestInc{-100000000000000000000}{-99999999999999999999}%
\end{qstest}

\begin{qstest}{dec}{dec}%
  \TestDec{0}{-1}%
  \TestDec{1}{0}%
  \TestDec{-1}{-2}%
  \TestDec{10}{9}%
  \TestDec{-10}{-11}%
  \TestDec{1000}{999}%
  \TestDec{-999}{-1000}%
  \TestDec{130}{129}%
  \TestDec{2147483647}{2147483646}%
  \TestDec{-2147483646}{-2147483647}%
  \TestDec{12345678901234567891}{12345678901234567890}%
  \TestDec{100000000000000000000}{99999999999999999999}%
  \TestDec{-12345678901234567890}{-12345678901234567891}%
  \TestDec{-99999999999999999999}{-100000000000000000000}%
\end{qstest}

\begin{qstest}{add}{add}%
  \TestAdd{0}{0}{0}%
  \TestAdd{1}{0}{1}%
  \TestAdd{0}{1}{1}%
  \TestAdd{1}{2}{3}%
  \TestAdd{-1}{-1}{-2}%
  \TestAdd{2147483646}{1}{2147483647}%
  \TestAdd{-2147483647}{2147483647}{0}%
  \TestAdd{20}{-5}{15}%
  \TestAdd{-4}{-1}{-5}%
  \TestAdd{-1}{-4}{-5}%
  \TestAdd{-4}{1}{-3}%
  \TestAdd{-1}{4}{3}%
  \TestAdd{4}{-1}{3}%
  \TestAdd{1}{-4}{-3}%
  \TestAdd{-4}{-1}{-5}%
  \TestAdd{-1}{-4}{-5}%
  \TestAdd{ -4 }{ -1 }{-5}%
  \TestAdd{ -1 }{ -4 }{-5}%
  \TestAdd{ -4 }{ 1 }{-3}%
  \TestAdd{ -1 }{ 4 }{3}%
  \TestAdd{ 4 }{ -1 }{3}%
  \TestAdd{ 1 }{ -4 }{-3}%
  \TestAdd{ -4 }{ -1 }{-5}%
  \TestAdd{ -1 }{ -4 }{-5}%
  \TestAdd{876543210}{111111111}{987654321}%
  \TestAdd{999999999}{2}{1000000001}%
\end{qstest}

\begin{qstest}{sub}{sub}
  \TestSub{0}{0}{0}%
  \TestSub{1}{0}{1}%
  \TestSub{1}{2}{-1}%
  \TestSub{-1}{-1}{0}%
  \TestSub{2147483646}{-1}{2147483647}%
  \TestSub{-2147483647}{-2147483647}{0}%
  \TestSub{-4}{-1}{-3}%
  \TestSub{-1}{-4}{3}%
  \TestSub{-4}{1}{-5}%
  \TestSub{-1}{4}{-5}%
  \TestSub{4}{-1}{5}%
  \TestSub{1}{-4}{5}%
  \TestSub{-4}{-1}{-3}%
  \TestSub{-1}{-4}{3}%
  \TestSub{ -4 }{ -1 }{-3}%
  \TestSub{ -1 }{ -4 }{3}%
  \TestSub{ -4 }{ 1 }{-5}%
  \TestSub{ -1 }{ 4 }{-5}%
  \TestSub{ 4 }{ -1 }{5}%
  \TestSub{ 1 }{ -4 }{5}%
  \TestSub{ -4 }{ -1 }{-3}%
  \TestSub{ -1 }{ -4 }{3}%
  \TestSub{1000000000}{2}{999999998}%
  \TestSub{987654321}{111111111}{876543210}%
\end{qstest}

\begin{qstest}{shl}{shl}
  \TestShl{0}{0}%
  \TestShl{1}{2}%
  \TestShl{2}{4}%
  \TestShl{5621}{11242}%
  \TestShl{1073741823}{2147483646}%
\end{qstest}

\begin{qstest}{shr}{shr}
  \TestShr{0}{0}%
  \TestShr{1}{0}%
  \TestShr{2}{1}%
  \TestShr{3}{1}%
  \TestShr{4}{2}%
  \TestShr{5}{2}%
  \TestShr{6}{3}%
  \TestShr{7}{3}%
  \TestShr{8}{4}%
  \TestShr{9}{4}%
  \TestShr{10}{5}%
  \TestShr{11}{5}%
  \TestShr{12}{6}%
  \TestShr{13}{6}%
  \TestShr{14}{7}%
  \TestShr{15}{7}%
  \TestShr{16}{8}%
  \TestShr{17}{8}%
  \TestShr{18}{9}%
  \TestShr{19}{9}%
  \TestShr{20}{10}%
  \TestShr{21}{10}%
  \TestShr{22}{11}%
  \TestShr{11241}{5620}%
  \TestShr{73054202}{36527101}%
  \TestShr{2147483646}{1073741823}%
\end{qstest}

\begin{qstest}{mul}{mul}
  \TestMul{0}{0}{0}%
  \TestMul{1}{0}{0}%
  \TestMul{0}{1}{0}%
  \TestMul{1}{1}{1}%
  \TestMul{3}{1}{3}%
  \TestMul{1}{-3}{-3}%
  \TestMul{-4}{-5}{20}%
  \TestMul{3}{7}{21}%
  \TestMul{7}{3}{21}%
  \TestMul{3}{-7}{-21}%
  \TestMul{7}{-3}{-21}%
  \TestMul{-3}{7}{-21}%
  \TestMul{-7}{3}{-21}%
  \TestMul{-3}{-7}{21}%
  \TestMul{-7}{-3}{21}%
  \TestMul{12}{11}{132}%
  \TestMul{999}{333}{332667}%
  \TestMul{1000}{4321}{4321000}%
  \TestMul{12345}{173955}{2147474475}%
  \TestMul{1073741823}{2}{2147483646}%
  \TestMul{2}{1073741823}{2147483646}%
  \TestMul{-1073741823}{2}{-2147483646}%
  \TestMul{2}{-1073741823}{-2147483646}%
  \TestMul{6706022400}{13}{87178291200}%
\end{qstest}

\begin{qstest}{sqr}{sqr}
  \TestSqr{0}{0}%
  \TestSqr{1}{1}%
  \TestSqr{2}{4}%
  \TestSqr{3}{9}%
  \TestSqr{4}{16}%
  \TestSqr{9}{81}%
  \TestSqr{10}{100}%
  \TestSqr{46340}{2147395600}%
  \TestSqr{-1}{1}%
  \TestSqr{-2}{4}%
  \TestSqr{-46340}{2147395600}%
\end{qstest}

\begin{qstest}{fac}{fac}
  \TestFac{0}{1}%
  \TestFac{1}{1}%
  \TestFac{2}{2}%
  \TestFac{3}{2*3}%
  \TestFac{4}{2*3*4}%
  \TestFac{5}{2*3*4*5}%
  \TestFac{6}{2*3*4*5*6}%
  \TestFac{7}{2*3*4*5*6*7}%
  \TestFac{8}{2*3*4*5*6*7*8}%
  \TestFac{9}{2*3*4*5*6*7*8*9}%
  \TestFac{10}{2*3*4*5*6*7*8*9*10}%
  \TestFac{11}{2*3*4*5*6*7*8*9*10*11}%
  \TestFac{12}{2*3*4*5*6*7*8*9*10*11*12}%
  \TestFacBig{13}{6227020800}%
  \TestFacBig{14}{87178291200}%
  \TestFacBig{15}{1307674368000}%
  \TestFacBig{16}{20922789888000}%
  \TestFacBig{17}{355687428096000}%
  \TestFacBig{18}{6402373705728000}%
  \TestFacBig{19}{121645100408832000}%
  \TestFacBig{20}{2432902008176640000}%
  \TestFacBig{21}{51090942171709440000}%
  \TestFacBig{22}{1124000727777607680000}%
\end{qstest}

\begin{qstest}{pow}{pow}
  \TestPow{-2}{0}{1}%
  \TestPow{-1}{0}{1}%
  \TestPow{0}{0}{1}%
  \TestPow{1}{0}{1}%
  \TestPow{2}{0}{1}%
  \TestPow{3}{0}{1}%
  \TestPow{-2}{1}{-2}%
  \TestPow{-1}{1}{-1}%
  \TestPow{1}{1}{1}%
  \TestPow{2}{1}{2}%
  \TestPow{3}{1}{3}%
  \TestPow{-2}{2}{4}%
  \TestPow{-1}{2}{1}%
  \TestPow{0}{2}{0}%
  \TestPow{1}{2}{1}%
  \TestPow{2}{2}{4}%
  \TestPow{3}{2}{9}%
  \TestPow{0}{1}{0}%
  \TestPow{1}{-2}{1}%
  \TestPow{1}{-1}{1}%
  \TestPow{-1}{-2}{1}%
  \TestPow{-1}{-1}{-1}%
  \TestPow{-1}{3}{-1}%
  \TestPow{-1}{4}{1}%
  \TestPow{-2}{-1}{0}%
  \TestPow{-2}{-2}{0}%
  \TestPow{2}{3}{8}%
  \TestPow{2}{4}{16}%
  \TestPow{2}{5}{32}%
  \TestPow{2}{6}{64}%
  \TestPow{2}{7}{128}%
  \TestPow{2}{8}{256}%
  \TestPow{2}{9}{512}%
  \TestPow{2}{10}{1024}%
  \TestPow{-2}{3}{-8}%
  \TestPow{-2}{4}{16}%
  \TestPow{-2}{5}{-32}%
  \TestPow{-2}{6}{64}%
  \TestPow{-2}{7}{-128}%
  \TestPow{-2}{8}{256}%
  \TestPow{-2}{9}{-512}%
  \TestPow{-2}{10}{1024}%
  \TestPow{3}{3}{27}%
  \TestPow{3}{4}{81}%
  \TestPow{3}{5}{243}%
  \TestPow{-3}{3}{-27}%
  \TestPow{-3}{4}{81}%
  \TestPow{-3}{5}{-243}%
  \TestPow{2}{30}{1073741824}%
  \TestPow{-3}{19}{-1162261467}%
  \TestPow{5}{13}{1220703125}%
  \TestPow{-7}{11}{-1977326743}%
\end{qstest}

\begin{qstest}{div}{div}
  \TestDiv{1}{1}{1}%
  \TestDiv{2}{1}{2}%
  \TestDiv{-2}{1}{-2}%
  \TestDiv{2}{-1}{-2}%
  \TestDiv{-2}{-1}{2}%
  \TestDiv{15}{2}{7}%
  \TestDiv{-16}{2}{-8}%
  \TestDiv{1}{2}{0}%
  \TestDiv{1}{3}{0}%
  \TestDiv{2}{3}{0}%
  \TestDiv{-2}{3}{0}%
  \TestDiv{2}{-3}{0}%
  \TestDiv{-2}{-3}{0}%
  \TestDiv{13}{3}{4}%
  \TestDiv{-13}{-3}{4}%
  \TestDiv{-13}{3}{-4}%
  \TestDiv{-6}{5}{-1}%
  \TestDiv{-5}{5}{-1}%
  \TestDiv{-4}{5}{0}%
  \TestDiv{-3}{5}{0}%
  \TestDiv{-2}{5}{0}%
  \TestDiv{-1}{5}{0}%
  \TestDiv{0}{5}{0}%
  \TestDiv{1}{5}{0}%
  \TestDiv{2}{5}{0}%
  \TestDiv{3}{5}{0}%
  \TestDiv{4}{5}{0}%
  \TestDiv{5}{5}{1}%
  \TestDiv{6}{5}{1}%
  \TestDiv{-5}{4}{-1}%
  \TestDiv{-4}{4}{-1}%
  \TestDiv{-3}{4}{0}%
  \TestDiv{-2}{4}{0}%
  \TestDiv{-1}{4}{0}%
  \TestDiv{0}{4}{0}%
  \TestDiv{1}{4}{0}%
  \TestDiv{2}{4}{0}%
  \TestDiv{3}{4}{0}%
  \TestDiv{4}{4}{1}%
  \TestDiv{5}{4}{1}%
  \TestDiv{12345}{678}{18}%
  \TestDiv{32372}{5952}{5}%
  \TestDiv{284271294}{18162}{15651}%
  \TestDiv{217652429}{12561}{17327}%
  \TestDiv{462028434}{5439}{84947}%
  \TestDiv{2147483647}{1000}{2147483}%
  \TestDiv{2147483647}{-1000}{-2147483}%
  \TestDiv{-2147483647}{1000}{-2147483}%
  \TestDiv{-2147483647}{-1000}{2147483}%
  \TestDiv{0}{3}{0}%
  \TestDiv{1}{3}{0}%
  \TestDiv{2}{3}{0}%
  \TestDiv{3}{3}{1}%
  \TestDiv{4}{3}{1}%
  \TestDiv{5}{3}{1}%
  \TestDiv{6}{3}{2}%
  \TestDiv{7}{3}{2}%
  \TestDiv{8}{3}{2}%
  \TestDiv{9}{3}{3}%
  \TestDiv{10}{3}{3}%
  \TestDiv{11}{3}{3}%
  \TestDiv{12}{3}{4}%
  \TestDiv{13}{3}{4}%
  \TestDiv{14}{3}{4}%
  \TestDiv{15}{3}{5}%
  \TestDiv{16}{3}{5}%
  \TestDiv{17}{3}{5}%
  \TestDiv{18}{3}{6}%
  \TestDiv{19}{3}{6}%
  \TestDiv{20}{3}{6}%
  \TestDiv{21}{3}{7}%
  \TestDiv{22}{3}{7}%
  \TestDiv{23}{3}{7}%
  \TestDiv{24}{3}{8}%
  \TestDiv{25}{3}{8}%
  \TestDiv{26}{3}{8}%
  \TestDiv{27}{3}{9}%
  \TestDiv{28}{3}{9}%
  \TestDiv{29}{3}{9}%
  \TestDiv{30}{3}{10}%
  \TestDiv{31}{3}{10}%
  \TestDivBig{17363436332507}{24702}{702916214}%
\end{qstest}

\begin{qstest}{mod}{mod}
  \TestMod{-6}{5}{4}%
  \TestMod{-5}{5}{0}%
  \TestMod{-4}{5}{1}%
  \TestMod{-3}{5}{2}%
  \TestMod{-2}{5}{3}%
  \TestMod{-1}{5}{4}%
  \TestMod{0}{5}{0}%
  \TestMod{1}{5}{1}%
  \TestMod{2}{5}{2}%
  \TestMod{3}{5}{3}%
  \TestMod{4}{5}{4}%
  \TestMod{5}{5}{0}%
  \TestMod{6}{5}{1}%
  \TestMod{-5}{4}{3}%
  \TestMod{-4}{4}{0}%
  \TestMod{-3}{4}{1}%
  \TestMod{-2}{4}{2}%
  \TestMod{-1}{4}{3}%
  \TestMod{0}{4}{0}%
  \TestMod{1}{4}{1}%
  \TestMod{2}{4}{2}%
  \TestMod{3}{4}{3}%
  \TestMod{4}{4}{0}%
  \TestMod{5}{4}{1}%
  \TestMod{-6}{-5}{-1}%
  \TestMod{-5}{-5}{0}%
  \TestMod{-4}{-5}{-4}%
  \TestMod{-3}{-5}{-3}%
  \TestMod{-2}{-5}{-2}%
  \TestMod{-1}{-5}{-1}%
  \TestMod{0}{-5}{0}%
  \TestMod{1}{-5}{-4}%
  \TestMod{2}{-5}{-3}%
  \TestMod{3}{-5}{-2}%
  \TestMod{4}{-5}{-1}%
  \TestMod{5}{-5}{0}%
  \TestMod{6}{-5}{-4}%
  \TestMod{-5}{-4}{-1}%
  \TestMod{-4}{-4}{0}%
  \TestMod{-3}{-4}{-3}%
  \TestMod{-2}{-4}{-2}%
  \TestMod{-1}{-4}{-1}%
  \TestMod{0}{-4}{0}%
  \TestMod{1}{-4}{-3}%
  \TestMod{2}{-4}{-2}%
  \TestMod{3}{-4}{-1}%
  \TestMod{4}{-4}{0}%
  \TestMod{5}{-4}{-3}%
  \TestMod{2147483647}{1000}{647}%
  \TestMod{2147483647}{-1000}{-353}%
  \TestMod{-2147483647}{1000}{353}%
  \TestMod{-2147483647}{-1000}{-647}%
  \TestMod{ 0 }{ 4 }{0}%
  \TestMod{ 1 }{ 4 }{1}%
  \TestMod{ -1 }{ 4 }{3}%
  \TestMod{ 0 }{ -4 }{0}%
  \TestMod{ 1 }{ -4 }{-3}%
  \TestMod{ -1 }{ -4 }{-1}%
  \TestMod{18362}{25}{12}%
\end{qstest}

\newcommand*{\TestError}[2]{%
  \begingroup
    \expandafter\def\csname BigIntCalcError:#1\endcsname{}%
    \Expect*{#2}{0}%
    \expandafter\def\csname BigIntCalcError:#1\endcsname{ERROR}%
    \Expect*{#2}{0ERROR}%
  \endgroup
}
\begin{qstest}{error}{error}
  \TestError{FacNegative}{\bigintcalcFac{-1}}%
  \TestError{FacNegative}{\bigintcalcFac{-2147483647}}%
  \TestError{DivisionByZero}{\bigintcalcPow{0}{-1}}%
  \TestError{DivisionByZero}{\bigintcalcDiv{1}{0}}%
  \TestError{DivisionByZero}{\bigintcalcMod{1}{0}}%
\end{qstest}

\begin{document}
\end{document}
%</test2>
%    \end{macrocode}
%
% \section{Installation}
%
% \subsection{Download}
%
% \paragraph{Package.} This package is available on
% CTAN\footnote{\url{http://ctan.org/pkg/bigintcalc}}:
% \begin{description}
% \item[\CTAN{macros/latex/contrib/oberdiek/bigintcalc.dtx}] The source file.
% \item[\CTAN{macros/latex/contrib/oberdiek/bigintcalc.pdf}] Documentation.
% \end{description}
%
%
% \paragraph{Bundle.} All the packages of the bundle `oberdiek'
% are also available in a TDS compliant ZIP archive. There
% the packages are already unpacked and the documentation files
% are generated. The files and directories obey the TDS standard.
% \begin{description}
% \item[\CTAN{install/macros/latex/contrib/oberdiek.tds.zip}]
% \end{description}
% \emph{TDS} refers to the standard ``A Directory Structure
% for \TeX\ Files'' (\CTAN{tds/tds.pdf}). Directories
% with \xfile{texmf} in their name are usually organized this way.
%
% \subsection{Bundle installation}
%
% \paragraph{Unpacking.} Unpack the \xfile{oberdiek.tds.zip} in the
% TDS tree (also known as \xfile{texmf} tree) of your choice.
% Example (linux):
% \begin{quote}
%   |unzip oberdiek.tds.zip -d ~/texmf|
% \end{quote}
%
% \paragraph{Script installation.}
% Check the directory \xfile{TDS:scripts/oberdiek/} for
% scripts that need further installation steps.
% Package \xpackage{attachfile2} comes with the Perl script
% \xfile{pdfatfi.pl} that should be installed in such a way
% that it can be called as \texttt{pdfatfi}.
% Example (linux):
% \begin{quote}
%   |chmod +x scripts/oberdiek/pdfatfi.pl|\\
%   |cp scripts/oberdiek/pdfatfi.pl /usr/local/bin/|
% \end{quote}
%
% \subsection{Package installation}
%
% \paragraph{Unpacking.} The \xfile{.dtx} file is a self-extracting
% \docstrip\ archive. The files are extracted by running the
% \xfile{.dtx} through \plainTeX:
% \begin{quote}
%   \verb|tex bigintcalc.dtx|
% \end{quote}
%
% \paragraph{TDS.} Now the different files must be moved into
% the different directories in your installation TDS tree
% (also known as \xfile{texmf} tree):
% \begin{quote}
% \def\t{^^A
% \begin{tabular}{@{}>{\ttfamily}l@{ $\rightarrow$ }>{\ttfamily}l@{}}
%   bigintcalc.sty & tex/generic/oberdiek/bigintcalc.sty\\
%   bigintcalc.pdf & doc/latex/oberdiek/bigintcalc.pdf\\
%   test/bigintcalc-test1.tex & doc/latex/oberdiek/test/bigintcalc-test1.tex\\
%   test/bigintcalc-test2.tex & doc/latex/oberdiek/test/bigintcalc-test2.tex\\
%   test/bigintcalc-test3.tex & doc/latex/oberdiek/test/bigintcalc-test3.tex\\
%   bigintcalc.dtx & source/latex/oberdiek/bigintcalc.dtx\\
% \end{tabular}^^A
% }^^A
% \sbox0{\t}^^A
% \ifdim\wd0>\linewidth
%   \begingroup
%     \advance\linewidth by\leftmargin
%     \advance\linewidth by\rightmargin
%   \edef\x{\endgroup
%     \def\noexpand\lw{\the\linewidth}^^A
%   }\x
%   \def\lwbox{^^A
%     \leavevmode
%     \hbox to \linewidth{^^A
%       \kern-\leftmargin\relax
%       \hss
%       \usebox0
%       \hss
%       \kern-\rightmargin\relax
%     }^^A
%   }^^A
%   \ifdim\wd0>\lw
%     \sbox0{\small\t}^^A
%     \ifdim\wd0>\linewidth
%       \ifdim\wd0>\lw
%         \sbox0{\footnotesize\t}^^A
%         \ifdim\wd0>\linewidth
%           \ifdim\wd0>\lw
%             \sbox0{\scriptsize\t}^^A
%             \ifdim\wd0>\linewidth
%               \ifdim\wd0>\lw
%                 \sbox0{\tiny\t}^^A
%                 \ifdim\wd0>\linewidth
%                   \lwbox
%                 \else
%                   \usebox0
%                 \fi
%               \else
%                 \lwbox
%               \fi
%             \else
%               \usebox0
%             \fi
%           \else
%             \lwbox
%           \fi
%         \else
%           \usebox0
%         \fi
%       \else
%         \lwbox
%       \fi
%     \else
%       \usebox0
%     \fi
%   \else
%     \lwbox
%   \fi
% \else
%   \usebox0
% \fi
% \end{quote}
% If you have a \xfile{docstrip.cfg} that configures and enables \docstrip's
% TDS installing feature, then some files can already be in the right
% place, see the documentation of \docstrip.
%
% \subsection{Refresh file name databases}
%
% If your \TeX~distribution
% (\teTeX, \mikTeX, \dots) relies on file name databases, you must refresh
% these. For example, \teTeX\ users run \verb|texhash| or
% \verb|mktexlsr|.
%
% \subsection{Some details for the interested}
%
% \paragraph{Attached source.}
%
% The PDF documentation on CTAN also includes the
% \xfile{.dtx} source file. It can be extracted by
% AcrobatReader 6 or higher. Another option is \textsf{pdftk},
% e.g. unpack the file into the current directory:
% \begin{quote}
%   \verb|pdftk bigintcalc.pdf unpack_files output .|
% \end{quote}
%
% \paragraph{Unpacking with \LaTeX.}
% The \xfile{.dtx} chooses its action depending on the format:
% \begin{description}
% \item[\plainTeX:] Run \docstrip\ and extract the files.
% \item[\LaTeX:] Generate the documentation.
% \end{description}
% If you insist on using \LaTeX\ for \docstrip\ (really,
% \docstrip\ does not need \LaTeX), then inform the autodetect routine
% about your intention:
% \begin{quote}
%   \verb|latex \let\install=y% \iffalse meta-comment
%
% File: bigintcalc.dtx
% Version: 2016/05/16 v1.4
% Info: Expandable calculations on big integers
%
% Copyright (C) 2007, 2011, 2012 by
%    Heiko Oberdiek <heiko.oberdiek at googlemail.com>
%    2016
%    https://github.com/ho-tex/oberdiek/issues
%
% This work may be distributed and/or modified under the
% conditions of the LaTeX Project Public License, either
% version 1.3c of this license or (at your option) any later
% version. This version of this license is in
%    http://www.latex-project.org/lppl/lppl-1-3c.txt
% and the latest version of this license is in
%    http://www.latex-project.org/lppl.txt
% and version 1.3 or later is part of all distributions of
% LaTeX version 2005/12/01 or later.
%
% This work has the LPPL maintenance status "maintained".
%
% This Current Maintainer of this work is Heiko Oberdiek.
%
% The Base Interpreter refers to any `TeX-Format',
% because some files are installed in TDS:tex/generic//.
%
% This work consists of the main source file bigintcalc.dtx
% and the derived files
%    bigintcalc.sty, bigintcalc.pdf, bigintcalc.ins, bigintcalc.drv,
%    bigintcalc-test1.tex, bigintcalc-test2.tex,
%    bigintcalc-test3.tex.
%
% Distribution:
%    CTAN:macros/latex/contrib/oberdiek/bigintcalc.dtx
%    CTAN:macros/latex/contrib/oberdiek/bigintcalc.pdf
%
% Unpacking:
%    (a) If bigintcalc.ins is present:
%           tex bigintcalc.ins
%    (b) Without bigintcalc.ins:
%           tex bigintcalc.dtx
%    (c) If you insist on using LaTeX
%           latex \let\install=y\input{bigintcalc.dtx}
%        (quote the arguments according to the demands of your shell)
%
% Documentation:
%    (a) If bigintcalc.drv is present:
%           latex bigintcalc.drv
%    (b) Without bigintcalc.drv:
%           latex bigintcalc.dtx; ...
%    The class ltxdoc loads the configuration file ltxdoc.cfg
%    if available. Here you can specify further options, e.g.
%    use A4 as paper format:
%       \PassOptionsToClass{a4paper}{article}
%
%    Programm calls to get the documentation (example):
%       pdflatex bigintcalc.dtx
%       makeindex -s gind.ist bigintcalc.idx
%       pdflatex bigintcalc.dtx
%       makeindex -s gind.ist bigintcalc.idx
%       pdflatex bigintcalc.dtx
%
% Installation:
%    TDS:tex/generic/oberdiek/bigintcalc.sty
%    TDS:doc/latex/oberdiek/bigintcalc.pdf
%    TDS:doc/latex/oberdiek/test/bigintcalc-test1.tex
%    TDS:doc/latex/oberdiek/test/bigintcalc-test2.tex
%    TDS:doc/latex/oberdiek/test/bigintcalc-test3.tex
%    TDS:source/latex/oberdiek/bigintcalc.dtx
%
%<*ignore>
\begingroup
  \catcode123=1 %
  \catcode125=2 %
  \def\x{LaTeX2e}%
\expandafter\endgroup
\ifcase 0\ifx\install y1\fi\expandafter
         \ifx\csname processbatchFile\endcsname\relax\else1\fi
         \ifx\fmtname\x\else 1\fi\relax
\else\csname fi\endcsname
%</ignore>
%<*install>
\input docstrip.tex
\Msg{************************************************************************}
\Msg{* Installation}
\Msg{* Package: bigintcalc 2016/05/16 v1.4 Expandable calculations on big integers (HO)}
\Msg{************************************************************************}

\keepsilent
\askforoverwritefalse

\let\MetaPrefix\relax
\preamble

This is a generated file.

Project: bigintcalc
Version: 2016/05/16 v1.4

Copyright (C) 2007, 2011, 2012 by
   Heiko Oberdiek <heiko.oberdiek at googlemail.com>

This work may be distributed and/or modified under the
conditions of the LaTeX Project Public License, either
version 1.3c of this license or (at your option) any later
version. This version of this license is in
   http://www.latex-project.org/lppl/lppl-1-3c.txt
and the latest version of this license is in
   http://www.latex-project.org/lppl.txt
and version 1.3 or later is part of all distributions of
LaTeX version 2005/12/01 or later.

This work has the LPPL maintenance status "maintained".

This Current Maintainer of this work is Heiko Oberdiek.

The Base Interpreter refers to any `TeX-Format',
because some files are installed in TDS:tex/generic//.

This work consists of the main source file bigintcalc.dtx
and the derived files
   bigintcalc.sty, bigintcalc.pdf, bigintcalc.ins, bigintcalc.drv,
   bigintcalc-test1.tex, bigintcalc-test2.tex,
   bigintcalc-test3.tex.

\endpreamble
\let\MetaPrefix\DoubleperCent

\generate{%
  \file{bigintcalc.ins}{\from{bigintcalc.dtx}{install}}%
  \file{bigintcalc.drv}{\from{bigintcalc.dtx}{driver}}%
  \usedir{tex/generic/oberdiek}%
  \file{bigintcalc.sty}{\from{bigintcalc.dtx}{package}}%
%  \usedir{doc/latex/oberdiek/test}%
%  \file{bigintcalc-test1.tex}{\from{bigintcalc.dtx}{test1}}%
%  \file{bigintcalc-test2.tex}{\from{bigintcalc.dtx}{test2,etex}}%
%  \file{bigintcalc-test3.tex}{\from{bigintcalc.dtx}{test2,noetex}}%
  \nopreamble
  \nopostamble
  \usedir{source/latex/oberdiek/catalogue}%
  \file{bigintcalc.xml}{\from{bigintcalc.dtx}{catalogue}}%
}

\catcode32=13\relax% active space
\let =\space%
\Msg{************************************************************************}
\Msg{*}
\Msg{* To finish the installation you have to move the following}
\Msg{* file into a directory searched by TeX:}
\Msg{*}
\Msg{*     bigintcalc.sty}
\Msg{*}
\Msg{* To produce the documentation run the file `bigintcalc.drv'}
\Msg{* through LaTeX.}
\Msg{*}
\Msg{* Happy TeXing!}
\Msg{*}
\Msg{************************************************************************}

\endbatchfile
%</install>
%<*ignore>
\fi
%</ignore>
%<*driver>
\NeedsTeXFormat{LaTeX2e}
\ProvidesFile{bigintcalc.drv}%
  [2016/05/16 v1.4 Expandable calculations on big integers (HO)]%
\documentclass{ltxdoc}
\usepackage{wasysym}
\let\iint\relax
\let\iiint\relax
\usepackage[fleqn]{amsmath}

\DeclareMathOperator{\opInv}{Inv}
\DeclareMathOperator{\opAbs}{Abs}
\DeclareMathOperator{\opSgn}{Sgn}
\DeclareMathOperator{\opMin}{Min}
\DeclareMathOperator{\opMax}{Max}
\DeclareMathOperator{\opCmp}{Cmp}
\DeclareMathOperator{\opOdd}{Odd}
\DeclareMathOperator{\opInc}{Inc}
\DeclareMathOperator{\opDec}{Dec}
\DeclareMathOperator{\opAdd}{Add}
\DeclareMathOperator{\opSub}{Sub}
\DeclareMathOperator{\opShl}{Shl}
\DeclareMathOperator{\opShr}{Shr}
\DeclareMathOperator{\opMul}{Mul}
\DeclareMathOperator{\opSqr}{Sqr}
\DeclareMathOperator{\opFac}{Fac}
\DeclareMathOperator{\opPow}{Pow}
\DeclareMathOperator{\opDiv}{Div}
\DeclareMathOperator{\opMod}{Mod}
\DeclareMathOperator{\opInt}{Int}
\DeclareMathOperator{\opODD}{ifodd}

\newcommand*{\Def}{%
  \ensuremath{%
    \mathrel{\mathop{:}}=%
  }%
}
\newcommand*{\op}[1]{%
  \textsf{#1}%
}
\usepackage{holtxdoc}[2011/11/22]
\begin{document}
  \DocInput{bigintcalc.dtx}%
\end{document}
%</driver>
% \fi
%
%
% \CharacterTable
%  {Upper-case    \A\B\C\D\E\F\G\H\I\J\K\L\M\N\O\P\Q\R\S\T\U\V\W\X\Y\Z
%   Lower-case    \a\b\c\d\e\f\g\h\i\j\k\l\m\n\o\p\q\r\s\t\u\v\w\x\y\z
%   Digits        \0\1\2\3\4\5\6\7\8\9
%   Exclamation   \!     Double quote  \"     Hash (number) \#
%   Dollar        \$     Percent       \%     Ampersand     \&
%   Acute accent  \'     Left paren    \(     Right paren   \)
%   Asterisk      \*     Plus          \+     Comma         \,
%   Minus         \-     Point         \.     Solidus       \/
%   Colon         \:     Semicolon     \;     Less than     \<
%   Equals        \=     Greater than  \>     Question mark \?
%   Commercial at \@     Left bracket  \[     Backslash     \\
%   Right bracket \]     Circumflex    \^     Underscore    \_
%   Grave accent  \`     Left brace    \{     Vertical bar  \|
%   Right brace   \}     Tilde         \~}
%
% \GetFileInfo{bigintcalc.drv}
%
% \title{The \xpackage{bigintcalc} package}
% \date{2016/05/16 v1.4}
% \author{Heiko Oberdiek\thanks
% {Please report any issues at https://github.com/ho-tex/oberdiek/issues}\\
% \xemail{heiko.oberdiek at googlemail.com}}
%
% \maketitle
%
% \begin{abstract}
% This package provides expandable arithmetic operations
% with big integers that can exceed \TeX's number limits.
% \end{abstract}
%
% \tableofcontents
%
% \section{Documentation}
%
% \subsection{Introduction}
%
% Package \xpackage{bigintcalc} defines arithmetic operations
% that deal with big integers. Big integers can be given
% either as explicit integer number or as macro code
% that expands to an explicit number. \emph{Big} means that
% there is no limit on the size of the number. Big integers
% may exceed \TeX's range limitation of -2147483647 and 2147483647.
% Only memory issues will limit the usable range.
%
% In opposite to package \xpackage{intcalc}
% unexpandable command tokens are not supported, even if
% they are valid \TeX\ numbers like count registers or
% commands created by \cs{chardef}. Nevertheless they may
% be used, if they are prefixed by \cs{number}.
%
% Also \eTeX's \cs{numexpr} expressions are not supported
% directly in the manner of package \xpackage{intcalc}.
% However they can be given if \cs{the}\cs{numexpr}
% or \cs{number}\cs{numexpr} are used.
%
% The operations have the form of macros that take one or
% two integers as parameter and return the integer result.
% The macro name is a three letter operation name prefixed
% by the package name, e.g. \cs{bigintcalcAdd}|{10}{43}| returns
% |53|.
%
% The macros are fully expandable, exactly two expansion
% steps generate the result. Therefore the operations
% may be used nearly everywhere in \TeX, even inside
% \cs{csname}, file names,  or other
% expandable contexts.
%
% \subsection{Conditions}
%
% \subsubsection{Preconditions}
%
% \begin{itemize}
% \item
%   Arguments can be anything that expands to a number that consists
%   of optional signs and digits.
% \item
%   The arguments and return values must be sound.
%   Zero as divisor or factorials of negative numbers will cause errors.
% \end{itemize}
%
% \subsubsection{Postconditions}
%
% Additional properties of the macros apart from calculating
% a correct result (of course \smiley):
% \begin{itemize}
% \item
%   The macros are fully expandable. Thus they can
%   be used inside \cs{edef}, \cs{csname},
%   for example.
% \item
%   Furthermore exactly two expansion steps calculate the result.
% \item
%   The number consists of one optional minus sign and one or more
%   digits. The first digit is larger than zero for
%   numbers that consists of more than one digit.
%
%   In short, the number format is exactly the same as
%   \cs{number} generates, but without its range limitation.
%   And the tokens (minus sign, digits)
%   have catcode 12 (other).
% \item
%   Call by value is simulated. First the arguments are
%   converted to numbers. Then these numbers are used
%   in the calculations.
%
%   Remember that arguments
%   may contain expensive macros or \eTeX\ expressions.
%   This strategy avoids multiple evaluations of such
%   arguments.
% \end{itemize}
%
% \subsection{Error handling}
%
% Some errors are detected by the macros, example: division by zero.
% In this cases an undefined control sequence is called and causes
% a TeX error message, example: \cs{BigIntCalcError:DivisionByZero}.
% The name of the control sequence contains
% the reason for the error. The \TeX\ error may be ignored.
% Then the operation returns zero as result.
% Because the macros are supposed to work in expandible contexts.
% An traditional error message, however, is not expandable and
% would break these contexts.
%
% \subsection{Operations}
%
% Some definition equations below use the function $\opInt$
% that converts a real number to an integer. The number
% is truncated that means rounding to zero:
% \begin{gather*}
%   \opInt(x) \Def
%   \begin{cases}
%     \lfloor x\rfloor & \text{if $x\geq0$}\\
%     \lceil x\rceil & \text{otherwise}
%   \end{cases}
% \end{gather*}
%
% \subsubsection{\op{Num}}
%
% \begin{declcs}{bigintcalcNum} \M{x}
% \end{declcs}
% Macro \cs{bigintcalcNum} converts its argument to a normalized integer
% number without unnecessary leading zeros or signs.
% The result matches the regular expression:
%\begin{quote}
%\begin{verbatim}
%0|-?[1-9][0-9]*
%\end{verbatim}
%\end{quote}
%
% \subsubsection{\op{Inv}, \op{Abs}, \op{Sgn}}
%
% \begin{declcs}{bigintcalcInv} \M{x}
% \end{declcs}
% Macro \cs{bigintcalcInv} switches the sign.
% \begin{gather*}
%   \opInv(x) \Def -x
% \end{gather*}
%
% \begin{declcs}{bigintcalcAbs} \M{x}
% \end{declcs}
% Macro \cs{bigintcalcAbs} returns the absolute value
% of integer \meta{x}.
% \begin{gather*}
%   \opAbs(x) \Def \vert x\vert
% \end{gather*}
%
% \begin{declcs}{bigintcalcSgn} \M{x}
% \end{declcs}
% Macro \cs{bigintcalcSgn} encodes the sign of \meta{x} as number.
% \begin{gather*}
%   \opSgn(x) \Def
%   \begin{cases}
%     -1& \text{if $x<0$}\\
%      0& \text{if $x=0$}\\
%      1& \text{if $x>0$}
%   \end{cases}
% \end{gather*}
% These return values can easily be distinguished by \cs{ifcase}:
%\begin{quote}
%\begin{verbatim}
%\ifcase\bigintcalcSgn{<x>}
%  $x=0$
%\or
%  $x>0$
%\else
%  $x<0$
%\fi
%\end{verbatim}
%\end{quote}
%
% \subsubsection{\op{Min}, \op{Max}, \op{Cmp}}
%
% \begin{declcs}{bigintcalcMin} \M{x} \M{y}
% \end{declcs}
% Macro \cs{bigintcalcMin} returns the smaller of the two integers.
% \begin{gather*}
%   \opMin(x,y) \Def
%   \begin{cases}
%     x & \text{if $x<y$}\\
%     y & \text{otherwise}
%   \end{cases}
% \end{gather*}
%
% \begin{declcs}{bigintcalcMax} \M{x} \M{y}
% \end{declcs}
% Macro \cs{bigintcalcMax} returns the larger of the two integers.
% \begin{gather*}
%   \opMax(x,y) \Def
%   \begin{cases}
%     x & \text{if $x>y$}\\
%     y & \text{otherwise}
%   \end{cases}
% \end{gather*}
%
% \begin{declcs}{bigintcalcCmp} \M{x} \M{y}
% \end{declcs}
% Macro \cs{bigintcalcCmp} encodes the comparison result as number:
% \begin{gather*}
%   \opCmp(x,y) \Def
%   \begin{cases}
%     -1 & \text{if $x<y$}\\
%      0 & \text{if $x=y$}\\
%      1 & \text{if $x>y$}
%   \end{cases}
% \end{gather*}
% These values can be distinguished by \cs{ifcase}:
%\begin{quote}
%\begin{verbatim}
%\ifcase\bigintcalcCmp{<x>}{<y>}
%  $x=y$
%\or
%  $x>y$
%\else
%  $x<y$
%\fi
%\end{verbatim}
%\end{quote}
%
% \subsubsection{\op{Odd}}
%
% \begin{declcs}{bigintcalcOdd} \M{x}
% \end{declcs}
% \begin{gather*}
%   \opOdd(x) \Def
%   \begin{cases}
%     1 & \text{if $x$ is odd}\\
%     0 & \text{if $x$ is even}
%   \end{cases}
% \end{gather*}
%
% \subsubsection{\op{Inc}, \op{Dec}, \op{Add}, \op{Sub}}
%
% \begin{declcs}{bigintcalcInc} \M{x}
% \end{declcs}
% Macro \cs{bigintcalcInc} increments \meta{x} by one.
% \begin{gather*}
%   \opInc(x) \Def x + 1
% \end{gather*}
%
% \begin{declcs}{bigintcalcDec} \M{x}
% \end{declcs}
% Macro \cs{bigintcalcDec} decrements \meta{x} by one.
% \begin{gather*}
%   \opDec(x) \Def x - 1
% \end{gather*}
%
% \begin{declcs}{bigintcalcAdd} \M{x} \M{y}
% \end{declcs}
% Macro \cs{bigintcalcAdd} adds the two numbers.
% \begin{gather*}
%   \opAdd(x, y) \Def x + y
% \end{gather*}
%
% \begin{declcs}{bigintcalcSub} \M{x} \M{y}
% \end{declcs}
% Macro \cs{bigintcalcSub} calculates the difference.
% \begin{gather*}
%   \opSub(x, y) \Def x - y
% \end{gather*}
%
% \subsubsection{\op{Shl}, \op{Shr}}
%
% \begin{declcs}{bigintcalcShl} \M{x}
% \end{declcs}
% Macro \cs{bigintcalcShl} implements shifting to the left that
% means the number is multiplied by two.
% The sign is preserved.
% \begin{gather*}
%   \opShl(x) \Def x*2
% \end{gather*}
%
% \begin{declcs}{bigintcalcShr} \M{x}
% \end{declcs}
% Macro \cs{bigintcalcShr} implements shifting to the right.
% That is equivalent to an integer division by two.
% The sign is preserved.
% \begin{gather*}
%   \opShr(x) \Def \opInt(x/2)
% \end{gather*}
%
% \subsubsection{\op{Mul}, \op{Sqr}, \op{Fac}, \op{Pow}}
%
% \begin{declcs}{bigintcalcMul} \M{x} \M{y}
% \end{declcs}
% Macro \cs{bigintcalcMul} calculates the product of
% \meta{x} and \meta{y}.
% \begin{gather*}
%   \opMul(x,y) \Def x*y
% \end{gather*}
%
% \begin{declcs}{bigintcalcSqr} \M{x}
% \end{declcs}
% Macro \cs{bigintcalcSqr} returns the square product.
% \begin{gather*}
%   \opSqr(x) \Def x^2
% \end{gather*}
%
% \begin{declcs}{bigintcalcFac} \M{x}
% \end{declcs}
% Macro \cs{bigintcalcFac} returns the factorial of \meta{x}.
% Negative numbers are not permitted.
% \begin{gather*}
%   \opFac(x) \Def x!\qquad\text{for $x\geq0$}
% \end{gather*}
% ($0! = 1$)
%
% \begin{declcs}{bigintcalcPow} M{x} M{y}
% \end{declcs}
% Macro \cs{bigintcalcPow} calculates the value of \meta{x} to the
% power of \meta{y}. The error ``division by zero'' is thrown
% if \meta{x} is zero and \meta{y} is negative.
% permitted:
% \begin{gather*}
%   \opPow(x,y) \Def
%   \opInt(x^y)\qquad\text{for $x\neq0$ or $y\geq0$}
% \end{gather*}
% ($0^0 = 1$)
%
% \subsubsection{\op{Div}, \op{Mul}}
%
% \begin{declcs}{bigintcalcDiv} \M{x} \M{y}
% \end{declcs}
% Macro \cs{bigintcalcDiv} performs an integer division.
% Argument \meta{y} must not be zero.
% \begin{gather*}
%   \opDiv(x,y) \Def \opInt(x/y)\qquad\text{for $y\neq0$}
% \end{gather*}
%
% \begin{declcs}{bigintcalcMod} \M{x} \M{y}
% \end{declcs}
% Macro \cs{bigintcalcMod} gets the remainder of the integer
% division. The sign follows the divisor \meta{y}.
% Argument \meta{y} must not be zero.
% \begin{gather*}
%   \opMod(x,y) \Def x\mathrel{\%}y\qquad\text{for $y\neq0$}
% \end{gather*}
% The result ranges:
% \begin{gather*}
%   -\vert y\vert < \opMod(x,y) \leq 0\qquad\text{for $y<0$}\\
%   0 \leq \opMod(x,y) < y\qquad\text{for $y\geq0$}
% \end{gather*}
%
% \subsection{Interface for programmers}
%
% If the programmer can ensure some more properties about
% the arguments of the operations, then the following
% macros are a little more efficient.
%
% In general numbers must obey the following constraints:
% \begin{itemize}
% \item Plain number: digit tokens only, no command tokens.
% \item Non-negative. Signs are forbidden.
% \item Delimited by exclamation mark. Curly braces
%       around the number are not allowed and will
%       break the code.
% \end{itemize}
%
% \begin{declcs}{BigIntCalcOdd} \meta{number} |!|
% \end{declcs}
% |1|/|0| is returned if \meta{number} is odd/even.
%
% \begin{declcs}{BigIntCalcInc} \meta{number} |!|
% \end{declcs}
% Incrementation.
%
% \begin{declcs}{BigIntCalcDec} \meta{number} |!|
% \end{declcs}
% Decrementation, positive number without zero.
%
% \begin{declcs}{BigIntCalcAdd} \meta{number A} |!| \meta{number B} |!|
% \end{declcs}
% Addition, $A\geq B$.
%
% \begin{declcs}{BigIntCalcSub} \meta{number A} |!| \meta{number B} |!|
% \end{declcs}
% Subtraction, $A\geq B$.
%
% \begin{declcs}{BigIntCalcShl} \meta{number} |!|
% \end{declcs}
% Left shift (multiplication with two).
%
% \begin{declcs}{BigIntCalcShr} \meta{number} |!|
% \end{declcs}
% Right shift (integer division by two).
%
% \begin{declcs}{BigIntCalcMul} \meta{number A} |!| \meta{number B} |!|
% \end{declcs}
% Multiplication, $A\geq B$.
%
% \begin{declcs}{BigIntCalcDiv} \meta{number A} |!| \meta{number B} |!|
% \end{declcs}
% Division operation.
%
% \begin{declcs}{BigIntCalcMod} \meta{number A} |!| \meta{number B} |!|
% \end{declcs}
% Modulo operation.
%
%
% \StopEventually{
% }
%
% \section{Implementation}
%
%    \begin{macrocode}
%<*package>
%    \end{macrocode}
%
% \subsection{Reload check and package identification}
%    Reload check, especially if the package is not used with \LaTeX.
%    \begin{macrocode}
\begingroup\catcode61\catcode48\catcode32=10\relax%
  \catcode13=5 % ^^M
  \endlinechar=13 %
  \catcode35=6 % #
  \catcode39=12 % '
  \catcode44=12 % ,
  \catcode45=12 % -
  \catcode46=12 % .
  \catcode58=12 % :
  \catcode64=11 % @
  \catcode123=1 % {
  \catcode125=2 % }
  \expandafter\let\expandafter\x\csname ver@bigintcalc.sty\endcsname
  \ifx\x\relax % plain-TeX, first loading
  \else
    \def\empty{}%
    \ifx\x\empty % LaTeX, first loading,
      % variable is initialized, but \ProvidesPackage not yet seen
    \else
      \expandafter\ifx\csname PackageInfo\endcsname\relax
        \def\x#1#2{%
          \immediate\write-1{Package #1 Info: #2.}%
        }%
      \else
        \def\x#1#2{\PackageInfo{#1}{#2, stopped}}%
      \fi
      \x{bigintcalc}{The package is already loaded}%
      \aftergroup\endinput
    \fi
  \fi
\endgroup%
%    \end{macrocode}
%    Package identification:
%    \begin{macrocode}
\begingroup\catcode61\catcode48\catcode32=10\relax%
  \catcode13=5 % ^^M
  \endlinechar=13 %
  \catcode35=6 % #
  \catcode39=12 % '
  \catcode40=12 % (
  \catcode41=12 % )
  \catcode44=12 % ,
  \catcode45=12 % -
  \catcode46=12 % .
  \catcode47=12 % /
  \catcode58=12 % :
  \catcode64=11 % @
  \catcode91=12 % [
  \catcode93=12 % ]
  \catcode123=1 % {
  \catcode125=2 % }
  \expandafter\ifx\csname ProvidesPackage\endcsname\relax
    \def\x#1#2#3[#4]{\endgroup
      \immediate\write-1{Package: #3 #4}%
      \xdef#1{#4}%
    }%
  \else
    \def\x#1#2[#3]{\endgroup
      #2[{#3}]%
      \ifx#1\@undefined
        \xdef#1{#3}%
      \fi
      \ifx#1\relax
        \xdef#1{#3}%
      \fi
    }%
  \fi
\expandafter\x\csname ver@bigintcalc.sty\endcsname
\ProvidesPackage{bigintcalc}%
  [2016/05/16 v1.4 Expandable calculations on big integers (HO)]%
%    \end{macrocode}
%
% \subsection{Catcodes}
%
%    \begin{macrocode}
\begingroup\catcode61\catcode48\catcode32=10\relax%
  \catcode13=5 % ^^M
  \endlinechar=13 %
  \catcode123=1 % {
  \catcode125=2 % }
  \catcode64=11 % @
  \def\x{\endgroup
    \expandafter\edef\csname BIC@AtEnd\endcsname{%
      \endlinechar=\the\endlinechar\relax
      \catcode13=\the\catcode13\relax
      \catcode32=\the\catcode32\relax
      \catcode35=\the\catcode35\relax
      \catcode61=\the\catcode61\relax
      \catcode64=\the\catcode64\relax
      \catcode123=\the\catcode123\relax
      \catcode125=\the\catcode125\relax
    }%
  }%
\x\catcode61\catcode48\catcode32=10\relax%
\catcode13=5 % ^^M
\endlinechar=13 %
\catcode35=6 % #
\catcode64=11 % @
\catcode123=1 % {
\catcode125=2 % }
\def\TMP@EnsureCode#1#2{%
  \edef\BIC@AtEnd{%
    \BIC@AtEnd
    \catcode#1=\the\catcode#1\relax
  }%
  \catcode#1=#2\relax
}
\TMP@EnsureCode{33}{12}% !
\TMP@EnsureCode{36}{14}% $ (comment!)
\TMP@EnsureCode{38}{14}% & (comment!)
\TMP@EnsureCode{40}{12}% (
\TMP@EnsureCode{41}{12}% )
\TMP@EnsureCode{42}{12}% *
\TMP@EnsureCode{43}{12}% +
\TMP@EnsureCode{45}{12}% -
\TMP@EnsureCode{46}{12}% .
\TMP@EnsureCode{47}{12}% /
\TMP@EnsureCode{58}{11}% : (letter!)
\TMP@EnsureCode{60}{12}% <
\TMP@EnsureCode{62}{12}% >
\TMP@EnsureCode{63}{14}% ? (comment!)
\TMP@EnsureCode{91}{12}% [
\TMP@EnsureCode{93}{12}% ]
\edef\BIC@AtEnd{\BIC@AtEnd\noexpand\endinput}
\begingroup\expandafter\expandafter\expandafter\endgroup
\expandafter\ifx\csname BIC@TestMode\endcsname\relax
\else
  \catcode63=9 % ? (ignore)
\fi
? \let\BIC@@TestMode\BIC@TestMode
%    \end{macrocode}
%
% \subsection{\eTeX\ detection}
%
%    \begin{macrocode}
\begingroup\expandafter\expandafter\expandafter\endgroup
\expandafter\ifx\csname numexpr\endcsname\relax
  \catcode36=9 % $ (ignore)
\else
  \catcode38=9 % & (ignore)
\fi
%    \end{macrocode}
%
% \subsection{Help macros}
%
%    \begin{macro}{\BIC@Fi}
%    \begin{macrocode}
\let\BIC@Fi\fi
%    \end{macrocode}
%    \end{macro}
%    \begin{macro}{\BIC@AfterFi}
%    \begin{macrocode}
\def\BIC@AfterFi#1#2\BIC@Fi{\fi#1}%
%    \end{macrocode}
%    \end{macro}
%    \begin{macro}{\BIC@AfterFiFi}
%    \begin{macrocode}
\def\BIC@AfterFiFi#1#2\BIC@Fi{\fi\fi#1}%
%    \end{macrocode}
%    \end{macro}
%    \begin{macro}{\BIC@AfterFiFiFi}
%    \begin{macrocode}
\def\BIC@AfterFiFiFi#1#2\BIC@Fi{\fi\fi\fi#1}%
%    \end{macrocode}
%    \end{macro}
%
%    \begin{macro}{\BIC@Space}
%    \begin{macrocode}
\begingroup
  \def\x#1{\endgroup
    \let\BIC@Space= #1%
  }%
\x{ }
%    \end{macrocode}
%    \end{macro}
%
% \subsection{Expand number}
%
%    \begin{macrocode}
\begingroup\expandafter\expandafter\expandafter\endgroup
\expandafter\ifx\csname RequirePackage\endcsname\relax
  \def\TMP@RequirePackage#1[#2]{%
    \begingroup\expandafter\expandafter\expandafter\endgroup
    \expandafter\ifx\csname ver@#1.sty\endcsname\relax
      \input #1.sty\relax
    \fi
  }%
  \TMP@RequirePackage{pdftexcmds}[2007/11/11]%
\else
  \RequirePackage{pdftexcmds}[2007/11/11]%
\fi
%    \end{macrocode}
%
%    \begin{macrocode}
\begingroup\expandafter\expandafter\expandafter\endgroup
\expandafter\ifx\csname pdf@escapehex\endcsname\relax
%    \end{macrocode}
%
%    \begin{macro}{\BIC@Expand}
%    \begin{macrocode}
  \def\BIC@Expand#1{%
    \romannumeral0%
    \BIC@@Expand#1!\@nil{}%
  }%
%    \end{macrocode}
%    \end{macro}
%    \begin{macro}{\BIC@@Expand}
%    \begin{macrocode}
  \def\BIC@@Expand#1#2\@nil#3{%
    \expandafter\ifcat\noexpand#1\relax
      \expandafter\@firstoftwo
    \else
      \expandafter\@secondoftwo
    \fi
    {%
      \expandafter\BIC@@Expand#1#2\@nil{#3}%
    }{%
      \ifx#1!%
        \expandafter\@firstoftwo
      \else
        \expandafter\@secondoftwo
      \fi
      { #3}{%
        \BIC@@Expand#2\@nil{#3#1}%
      }%
    }%
  }%
%    \end{macrocode}
%    \end{macro}
%    \begin{macro}{\@firstoftwo}
%    \begin{macrocode}
  \expandafter\ifx\csname @firstoftwo\endcsname\relax
    \long\def\@firstoftwo#1#2{#1}%
  \fi
%    \end{macrocode}
%    \end{macro}
%    \begin{macro}{\@secondoftwo}
%    \begin{macrocode}
  \expandafter\ifx\csname @secondoftwo\endcsname\relax
    \long\def\@secondoftwo#1#2{#2}%
  \fi
%    \end{macrocode}
%    \end{macro}
%
%    \begin{macrocode}
\else
%    \end{macrocode}
%    \begin{macro}{\BIC@Expand}
%    \begin{macrocode}
  \def\BIC@Expand#1{%
    \romannumeral0\expandafter\expandafter\expandafter\BIC@Space
    \pdf@unescapehex{%
      \expandafter\expandafter\expandafter
      \BIC@StripHexSpace\pdf@escapehex{#1}20\@nil
    }%
  }%
%    \end{macrocode}
%    \end{macro}
%    \begin{macro}{\BIC@StripHexSpace}
%    \begin{macrocode}
  \def\BIC@StripHexSpace#120#2\@nil{%
    #1%
    \ifx\\#2\\%
    \else
      \BIC@AfterFi{%
        \BIC@StripHexSpace#2\@nil
      }%
    \BIC@Fi
  }%
%    \end{macrocode}
%    \end{macro}
%    \begin{macrocode}
\fi
%    \end{macrocode}
%
% \subsection{Normalize expanded number}
%
%    \begin{macro}{\BIC@Normalize}
%    |#1|: result sign\\
%    |#2|: first token of number
%    \begin{macrocode}
\def\BIC@Normalize#1#2{%
  \ifx#2-%
    \ifx\\#1\\%
      \BIC@AfterFiFi{%
        \BIC@Normalize-%
      }%
    \else
      \BIC@AfterFiFi{%
        \BIC@Normalize{}%
      }%
    \fi
  \else
    \ifx#2+%
      \BIC@AfterFiFi{%
        \BIC@Normalize{#1}%
      }%
    \else
      \ifx#20%
        \BIC@AfterFiFiFi{%
          \BIC@NormalizeZero{#1}%
        }%
      \else
        \BIC@AfterFiFiFi{%
          \BIC@NormalizeDigits#1#2%
        }%
      \fi
    \fi
  \BIC@Fi
}
%    \end{macrocode}
%    \end{macro}
%    \begin{macro}{\BIC@NormalizeZero}
%    \begin{macrocode}
\def\BIC@NormalizeZero#1#2{%
  \ifx#2!%
    \BIC@AfterFi{ 0}%
  \else
    \ifx#20%
      \BIC@AfterFiFi{%
        \BIC@NormalizeZero{#1}%
      }%
    \else
      \BIC@AfterFiFi{%
        \BIC@NormalizeDigits#1#2%
      }%
    \fi
  \BIC@Fi
}
%    \end{macrocode}
%    \end{macro}
%    \begin{macro}{\BIC@NormalizeDigits}
%    \begin{macrocode}
\def\BIC@NormalizeDigits#1!{ #1}
%    \end{macrocode}
%    \end{macro}
%
% \subsection{\op{Num}}
%
%    \begin{macro}{\bigintcalcNum}
%    \begin{macrocode}
\def\bigintcalcNum#1{%
  \romannumeral0%
  \expandafter\expandafter\expandafter\BIC@Normalize
  \expandafter\expandafter\expandafter{%
  \expandafter\expandafter\expandafter}%
  \BIC@Expand{#1}!%
}
%    \end{macrocode}
%    \end{macro}
%
% \subsection{\op{Inv}, \op{Abs}, \op{Sgn}}
%
%
%    \begin{macro}{\bigintcalcInv}
%    \begin{macrocode}
\def\bigintcalcInv#1{%
  \romannumeral0\expandafter\expandafter\expandafter\BIC@Space
  \bigintcalcNum{-#1}%
}
%    \end{macrocode}
%    \end{macro}
%
%    \begin{macro}{\bigintcalcAbs}
%    \begin{macrocode}
\def\bigintcalcAbs#1{%
  \romannumeral0%
  \expandafter\expandafter\expandafter\BIC@Abs
  \bigintcalcNum{#1}%
}
%    \end{macrocode}
%    \end{macro}
%    \begin{macro}{\BIC@Abs}
%    \begin{macrocode}
\def\BIC@Abs#1{%
  \ifx#1-%
    \expandafter\BIC@Space
  \else
    \expandafter\BIC@Space
    \expandafter#1%
  \fi
}
%    \end{macrocode}
%    \end{macro}
%
%    \begin{macro}{\bigintcalcSgn}
%    \begin{macrocode}
\def\bigintcalcSgn#1{%
  \number
  \expandafter\expandafter\expandafter\BIC@Sgn
  \bigintcalcNum{#1}! %
}
%    \end{macrocode}
%    \end{macro}
%    \begin{macro}{\BIC@Sgn}
%    \begin{macrocode}
\def\BIC@Sgn#1#2!{%
  \ifx#1-%
    -1%
  \else
    \ifx#10%
      0%
    \else
      1%
    \fi
  \fi
}
%    \end{macrocode}
%    \end{macro}
%
% \subsection{\op{Cmp}, \op{Min}, \op{Max}}
%
%    \begin{macro}{\bigintcalcCmp}
%    \begin{macrocode}
\def\bigintcalcCmp#1#2{%
  \number
  \expandafter\expandafter\expandafter\BIC@Cmp
  \bigintcalcNum{#2}!{#1}%
}
%    \end{macrocode}
%    \end{macro}
%    \begin{macro}{\BIC@Cmp}
%    \begin{macrocode}
\def\BIC@Cmp#1!#2{%
  \expandafter\expandafter\expandafter\BIC@@Cmp
  \bigintcalcNum{#2}!#1!%
}
%    \end{macrocode}
%    \end{macro}
%    \begin{macro}{\BIC@@Cmp}
%    \begin{macrocode}
\def\BIC@@Cmp#1#2!#3#4!{%
  \ifx#1-%
    \ifx#3-%
      \BIC@AfterFiFi{%
        \BIC@@Cmp#4!#2!%
      }%
    \else
      \BIC@AfterFiFi{%
        -1 %
      }%
    \fi
  \else
    \ifx#3-%
      \BIC@AfterFiFi{%
        1 %
      }%
    \else
      \BIC@AfterFiFi{%
        \BIC@CmpLength#1#2!#3#4!#1#2!#3#4!%
      }%
    \fi
  \BIC@Fi
}
%    \end{macrocode}
%    \end{macro}
%    \begin{macro}{\BIC@PosCmp}
%    \begin{macrocode}
\def\BIC@PosCmp#1!#2!{%
  \BIC@CmpLength#1!#2!#1!#2!%
}
%    \end{macrocode}
%    \end{macro}
%    \begin{macro}{\BIC@CmpLength}
%    \begin{macrocode}
\def\BIC@CmpLength#1#2!#3#4!{%
  \ifx\\#2\\%
    \ifx\\#4\\%
      \BIC@AfterFiFi\BIC@CmpDiff
    \else
      \BIC@AfterFiFi{%
        \BIC@CmpResult{-1}%
      }%
    \fi
  \else
    \ifx\\#4\\%
      \BIC@AfterFiFi{%
        \BIC@CmpResult1%
      }%
    \else
      \BIC@AfterFiFi{%
        \BIC@CmpLength#2!#4!%
      }%
    \fi
  \BIC@Fi
}
%    \end{macrocode}
%    \end{macro}
%    \begin{macro}{\BIC@CmpResult}
%    \begin{macrocode}
\def\BIC@CmpResult#1#2!#3!{#1 }
%    \end{macrocode}
%    \end{macro}
%    \begin{macro}{\BIC@CmpDiff}
%    \begin{macrocode}
\def\BIC@CmpDiff#1#2!#3#4!{%
  \ifnum#1<#3 %
    \BIC@AfterFi{%
      -1 %
    }%
  \else
    \ifnum#1>#3 %
      \BIC@AfterFiFi{%
        1 %
      }%
    \else
      \ifx\\#2\\%
        \BIC@AfterFiFiFi{%
          0 %
        }%
      \else
        \BIC@AfterFiFiFi{%
          \BIC@CmpDiff#2!#4!%
        }%
      \fi
    \fi
  \BIC@Fi
}
%    \end{macrocode}
%    \end{macro}
%
%    \begin{macro}{\bigintcalcMin}
%    \begin{macrocode}
\def\bigintcalcMin#1{%
  \romannumeral0%
  \expandafter\expandafter\expandafter\BIC@MinMax
  \bigintcalcNum{#1}!-!%
}
%    \end{macrocode}
%    \end{macro}
%    \begin{macro}{\bigintcalcMax}
%    \begin{macrocode}
\def\bigintcalcMax#1{%
  \romannumeral0%
  \expandafter\expandafter\expandafter\BIC@MinMax
  \bigintcalcNum{#1}!!%
}
%    \end{macrocode}
%    \end{macro}
%    \begin{macro}{\BIC@MinMax}
%    |#1|: $x$\\
%    |#2|: sign for comparison\\
%    |#3|: $y$
%    \begin{macrocode}
\def\BIC@MinMax#1!#2!#3{%
  \expandafter\expandafter\expandafter\BIC@@MinMax
  \bigintcalcNum{#3}!#1!#2!%
}
%    \end{macrocode}
%    \end{macro}
%    \begin{macro}{\BIC@@MinMax}
%    |#1|: $y$\\
%    |#2|: $x$\\
%    |#3|: sign for comparison
%    \begin{macrocode}
\def\BIC@@MinMax#1!#2!#3!{%
  \ifnum\BIC@@Cmp#1!#2!=#31 %
    \BIC@AfterFi{ #1}%
  \else
    \BIC@AfterFi{ #2}%
  \BIC@Fi
}
%    \end{macrocode}
%    \end{macro}
%
% \subsection{\op{Odd}}
%
%    \begin{macro}{\bigintcalcOdd}
%    \begin{macrocode}
\def\bigintcalcOdd#1{%
  \romannumeral0%
  \expandafter\expandafter\expandafter\BIC@Odd
  \bigintcalcAbs{#1}!%
}
%    \end{macrocode}
%    \end{macro}
%    \begin{macro}{\BigIntCalcOdd}
%    \begin{macrocode}
\def\BigIntCalcOdd#1!{%
  \romannumeral0%
  \BIC@Odd#1!%
}
%    \end{macrocode}
%    \end{macro}
%    \begin{macro}{\BIC@Odd}
%    |#1|: $x$
%    \begin{macrocode}
\def\BIC@Odd#1#2{%
  \ifx#2!%
    \ifodd#1 %
      \BIC@AfterFiFi{ 1}%
    \else
      \BIC@AfterFiFi{ 0}%
    \fi
  \else
    \expandafter\BIC@Odd\expandafter#2%
  \BIC@Fi
}
%    \end{macrocode}
%    \end{macro}
%
% \subsection{\op{Inc}, \op{Dec}}
%
%    \begin{macro}{\bigintcalcInc}
%    \begin{macrocode}
\def\bigintcalcInc#1{%
  \romannumeral0%
  \expandafter\expandafter\expandafter\BIC@IncSwitch
  \bigintcalcNum{#1}!%
}
%    \end{macrocode}
%    \end{macro}
%    \begin{macro}{\BIC@IncSwitch}
%    \begin{macrocode}
\def\BIC@IncSwitch#1#2!{%
  \ifcase\BIC@@Cmp#1#2!-1!%
    \BIC@AfterFi{ 0}%
  \or
    \BIC@AfterFi{%
      \BIC@Inc#1#2!{}%
    }%
  \else
    \BIC@AfterFi{%
      \expandafter-\romannumeral0%
      \BIC@Dec#2!{}%
    }%
  \BIC@Fi
}
%    \end{macrocode}
%    \end{macro}
%
%    \begin{macro}{\bigintcalcDec}
%    \begin{macrocode}
\def\bigintcalcDec#1{%
  \romannumeral0%
  \expandafter\expandafter\expandafter\BIC@DecSwitch
  \bigintcalcNum{#1}!%
}
%    \end{macrocode}
%    \end{macro}
%    \begin{macro}{\BIC@DecSwitch}
%    \begin{macrocode}
\def\BIC@DecSwitch#1#2!{%
  \ifcase\BIC@Sgn#1#2! %
    \BIC@AfterFi{ -1}%
  \or
    \BIC@AfterFi{%
      \BIC@Dec#1#2!{}%
    }%
  \else
    \BIC@AfterFi{%
      \expandafter-\romannumeral0%
      \BIC@Inc#2!{}%
    }%
  \BIC@Fi
}
%    \end{macrocode}
%    \end{macro}
%
%    \begin{macro}{\BigIntCalcInc}
%    \begin{macrocode}
\def\BigIntCalcInc#1!{%
  \romannumeral0\BIC@Inc#1!{}%
}
%    \end{macrocode}
%    \end{macro}
%    \begin{macro}{\BigIntCalcDec}
%    \begin{macrocode}
\def\BigIntCalcDec#1!{%
  \romannumeral0\BIC@Dec#1!{}%
}
%    \end{macrocode}
%    \end{macro}
%
%    \begin{macro}{\BIC@Inc}
%    \begin{macrocode}
\def\BIC@Inc#1#2!#3{%
  \ifx\\#2\\%
    \BIC@AfterFi{%
      \BIC@@Inc1#1#3!{}%
    }%
  \else
    \BIC@AfterFi{%
      \BIC@Inc#2!{#1#3}%
    }%
  \BIC@Fi
}
%    \end{macrocode}
%    \end{macro}
%    \begin{macro}{\BIC@@Inc}
%    \begin{macrocode}
\def\BIC@@Inc#1#2#3!#4{%
  \ifcase#1 %
    \ifx\\#3\\%
      \BIC@AfterFiFi{ #2#4}%
    \else
      \BIC@AfterFiFi{%
        \BIC@@Inc0#3!{#2#4}%
      }%
    \fi
  \else
    \ifnum#2<9 %
      \BIC@AfterFiFi{%
&       \expandafter\BIC@@@Inc\the\numexpr#2+1\relax
$       \expandafter\expandafter\expandafter\BIC@@@Inc
$       \ifcase#2 \expandafter1%
$       \or\expandafter2%
$       \or\expandafter3%
$       \or\expandafter4%
$       \or\expandafter5%
$       \or\expandafter6%
$       \or\expandafter7%
$       \or\expandafter8%
$       \or\expandafter9%
$?      \else\BigIntCalcError:ThisCannotHappen%
$       \fi
        0#3!{#4}%
      }%
    \else
      \BIC@AfterFiFi{%
        \BIC@@@Inc01#3!{#4}%
      }%
    \fi
  \BIC@Fi
}
%    \end{macrocode}
%    \end{macro}
%    \begin{macro}{\BIC@@@Inc}
%    \begin{macrocode}
\def\BIC@@@Inc#1#2#3!#4{%
  \ifx\\#3\\%
    \ifnum#2=1 %
      \BIC@AfterFiFi{ 1#1#4}%
    \else
      \BIC@AfterFiFi{ #1#4}%
    \fi
  \else
    \BIC@AfterFi{%
      \BIC@@Inc#2#3!{#1#4}%
    }%
  \BIC@Fi
}
%    \end{macrocode}
%    \end{macro}
%
%    \begin{macro}{\BIC@Dec}
%    \begin{macrocode}
\def\BIC@Dec#1#2!#3{%
  \ifx\\#2\\%
    \BIC@AfterFi{%
      \BIC@@Dec1#1#3!{}%
    }%
  \else
    \BIC@AfterFi{%
      \BIC@Dec#2!{#1#3}%
    }%
  \BIC@Fi
}
%    \end{macrocode}
%    \end{macro}
%    \begin{macro}{\BIC@@Dec}
%    \begin{macrocode}
\def\BIC@@Dec#1#2#3!#4{%
  \ifcase#1 %
    \ifx\\#3\\%
      \BIC@AfterFiFi{ #2#4}%
    \else
      \BIC@AfterFiFi{%
        \BIC@@Dec0#3!{#2#4}%
      }%
    \fi
  \else
    \ifnum#2>0 %
      \BIC@AfterFiFi{%
&       \expandafter\BIC@@@Dec\the\numexpr#2-1\relax
$       \expandafter\expandafter\expandafter\BIC@@@Dec
$       \ifcase#2
$?        \BigIntCalcError:ThisCannotHappen%
$       \or\expandafter0%
$       \or\expandafter1%
$       \or\expandafter2%
$       \or\expandafter3%
$       \or\expandafter4%
$       \or\expandafter5%
$       \or\expandafter6%
$       \or\expandafter7%
$       \or\expandafter8%
$?      \else\BigIntCalcError:ThisCannotHappen%
$       \fi
        0#3!{#4}%
      }%
    \else
      \BIC@AfterFiFi{%
        \BIC@@@Dec91#3!{#4}%
      }%
    \fi
  \BIC@Fi
}
%    \end{macrocode}
%    \end{macro}
%    \begin{macro}{\BIC@@@Dec}
%    \begin{macrocode}
\def\BIC@@@Dec#1#2#3!#4{%
  \ifx\\#3\\%
    \ifcase#1 %
      \ifx\\#4\\%
        \BIC@AfterFiFiFi{ 0}%
      \else
        \BIC@AfterFiFiFi{ #4}%
      \fi
    \else
      \BIC@AfterFiFi{ #1#4}%
    \fi
  \else
    \BIC@AfterFi{%
      \BIC@@Dec#2#3!{#1#4}%
    }%
  \BIC@Fi
}
%    \end{macrocode}
%    \end{macro}
%
% \subsection{\op{Add}, \op{Sub}}
%
%    \begin{macro}{\bigintcalcAdd}
%    \begin{macrocode}
\def\bigintcalcAdd#1{%
  \romannumeral0%
  \expandafter\expandafter\expandafter\BIC@Add
  \bigintcalcNum{#1}!%
}
%    \end{macrocode}
%    \end{macro}
%    \begin{macro}{\BIC@Add}
%    \begin{macrocode}
\def\BIC@Add#1!#2{%
  \expandafter\expandafter\expandafter
  \BIC@AddSwitch\bigintcalcNum{#2}!#1!%
}
%    \end{macrocode}
%    \end{macro}
%    \begin{macro}{\bigintcalcSub}
%    \begin{macrocode}
\def\bigintcalcSub#1#2{%
  \romannumeral0%
  \expandafter\expandafter\expandafter\BIC@Add
  \bigintcalcNum{-#2}!{#1}%
}
%    \end{macrocode}
%    \end{macro}
%
%    \begin{macro}{\BIC@AddSwitch}
%    Decision table for \cs{BIC@AddSwitch}.
%    \begin{quote}
%    \begin{tabular}[t]{@{}|l|l|l|l|l|@{}}
%      \hline
%      $x<0$ & $y<0$ & $-x>-y$ & $-$ & $\opAdd(-x,-y)$\\
%      \cline{3-3}\cline{5-5}
%            &       & else    &     & $\opAdd(-y,-x)$\\
%      \cline{2-5}
%            & else  & $-x> y$ & $-$ & $\opSub(-x, y)$\\
%      \cline{3-5}
%            &       & $-x= y$ &     & $0$\\
%      \cline{3-5}
%            &       & else    & $+$ & $\opSub( y,-x)$\\
%      \hline
%      else  & $y<0$ & $ x>-y$ & $+$ & $\opSub( x,-y)$\\
%      \cline{3-5}
%            &       & $ x=-y$ &     & $0$\\
%      \cline{3-5}
%            &       & else    & $-$ & $\opSub(-y, x)$\\
%      \cline{2-5}
%            & else  & $ x> y$ & $+$ & $\opAdd( x, y)$\\
%      \cline{3-3}\cline{5-5}
%            &       & else    &     & $\opAdd( y, x)$\\
%      \hline
%    \end{tabular}
%    \end{quote}
%    \begin{macrocode}
\def\BIC@AddSwitch#1#2!#3#4!{%
  \ifx#1-% x < 0
    \ifx#3-% y < 0
      \expandafter-\romannumeral0%
      \ifnum\BIC@PosCmp#2!#4!=1 % -x > -y
        \BIC@AfterFiFiFi{%
          \BIC@AddXY#2!#4!!!%
        }%
      \else % -x <= -y
        \BIC@AfterFiFiFi{%
          \BIC@AddXY#4!#2!!!%
        }%
      \fi
    \else % y >= 0
      \ifcase\BIC@PosCmp#2!#3#4!% -x = y
        \BIC@AfterFiFiFi{ 0}%
      \or % -x > y
        \expandafter-\romannumeral0%
        \BIC@AfterFiFiFi{%
          \BIC@SubXY#2!#3#4!!!%
        }%
      \else % -x <= y
        \BIC@AfterFiFiFi{%
          \BIC@SubXY#3#4!#2!!!%
        }%
      \fi
    \fi
  \else % x >= 0
    \ifx#3-% y < 0
      \ifcase\BIC@PosCmp#1#2!#4!% x = -y
        \BIC@AfterFiFiFi{ 0}%
      \or % x > -y
        \BIC@AfterFiFiFi{%
          \BIC@SubXY#1#2!#4!!!%
        }%
      \else % x <= -y
        \expandafter-\romannumeral0%
        \BIC@AfterFiFiFi{%
          \BIC@SubXY#4!#1#2!!!%
        }%
      \fi
    \else % y >= 0
      \ifnum\BIC@PosCmp#1#2!#3#4!=1 % x > y
        \BIC@AfterFiFiFi{%
          \BIC@AddXY#1#2!#3#4!!!%
        }%
      \else % x <= y
        \BIC@AfterFiFiFi{%
          \BIC@AddXY#3#4!#1#2!!!%
        }%
      \fi
    \fi
  \BIC@Fi
}
%    \end{macrocode}
%    \end{macro}
%
%    \begin{macro}{\BigIntCalcAdd}
%    \begin{macrocode}
\def\BigIntCalcAdd#1!#2!{%
  \romannumeral0\BIC@AddXY#1!#2!!!%
}
%    \end{macrocode}
%    \end{macro}
%    \begin{macro}{\BigIntCalcSub}
%    \begin{macrocode}
\def\BigIntCalcSub#1!#2!{%
  \romannumeral0\BIC@SubXY#1!#2!!!%
}
%    \end{macrocode}
%    \end{macro}
%
%    \begin{macro}{\BIC@AddXY}
%    \begin{macrocode}
\def\BIC@AddXY#1#2!#3#4!#5!#6!{%
  \ifx\\#2\\%
    \ifx\\#3\\%
      \BIC@AfterFiFi{%
        \BIC@DoAdd0!#1#5!#60!%
      }%
    \else
      \BIC@AfterFiFi{%
        \BIC@DoAdd0!#1#5!#3#6!%
      }%
    \fi
  \else
    \ifx\\#4\\%
      \ifx\\#3\\%
        \BIC@AfterFiFiFi{%
          \BIC@AddXY#2!{}!#1#5!#60!%
        }%
      \else
        \BIC@AfterFiFiFi{%
          \BIC@AddXY#2!{}!#1#5!#3#6!%
        }%
      \fi
    \else
      \BIC@AfterFiFi{%
        \BIC@AddXY#2!#4!#1#5!#3#6!%
      }%
    \fi
  \BIC@Fi
}
%    \end{macrocode}
%    \end{macro}
%    \begin{macro}{\BIC@DoAdd}
%    |#1|: carry\\
%    |#2|: reverted result\\
%    |#3#4|: reverted $x$\\
%    |#5#6|: reverted $y$
%    \begin{macrocode}
\def\BIC@DoAdd#1#2!#3#4!#5#6!{%
  \ifx\\#4\\%
    \BIC@AfterFi{%
&     \expandafter\BIC@Space
&     \the\numexpr#1+#3+#5\relax#2%
$     \expandafter\expandafter\expandafter\BIC@AddResult
$     \BIC@AddDigit#1#3#5#2%
    }%
  \else
    \BIC@AfterFi{%
      \expandafter\expandafter\expandafter\BIC@DoAdd
      \BIC@AddDigit#1#3#5#2!#4!#6!%
    }%
  \BIC@Fi
}
%    \end{macrocode}
%    \end{macro}
%    \begin{macro}{\BIC@AddResult}
%    \begin{macrocode}
$ \def\BIC@AddResult#1{%
$   \ifx#10%
$     \expandafter\BIC@Space
$   \else
$     \expandafter\BIC@Space\expandafter#1%
$   \fi
$ }%
%    \end{macrocode}
%    \end{macro}
%    \begin{macro}{\BIC@AddDigit}
%    |#1|: carry\\
%    |#2|: digit of $x$\\
%    |#3|: digit of $y$
%    \begin{macrocode}
\def\BIC@AddDigit#1#2#3{%
  \romannumeral0%
& \expandafter\BIC@@AddDigit\the\numexpr#1+#2+#3!%
$ \expandafter\BIC@@AddDigit\number%
$ \csname
$   BIC@AddCarry%
$   \ifcase#1 %
$     #2%
$   \else
$     \ifcase#2 1\or2\or3\or4\or5\or6\or7\or8\or9\or10\fi
$   \fi
$ \endcsname#3!%
}
%    \end{macrocode}
%    \end{macro}
%    \begin{macro}{\BIC@@AddDigit}
%    \begin{macrocode}
\def\BIC@@AddDigit#1!{%
  \ifnum#1<10 %
    \BIC@AfterFi{ 0#1}%
  \else
    \BIC@AfterFi{ #1}%
  \BIC@Fi
}
%    \end{macrocode}
%    \end{macro}
%    \begin{macro}{\BIC@AddCarry0}
%    \begin{macrocode}
$ \expandafter\def\csname BIC@AddCarry0\endcsname#1{#1}%
%    \end{macrocode}
%    \end{macro}
%    \begin{macro}{\BIC@AddCarry10}
%    \begin{macrocode}
$ \expandafter\def\csname BIC@AddCarry10\endcsname#1{1#1}%
%    \end{macrocode}
%    \end{macro}
%    \begin{macro}{\BIC@AddCarry[1-9]}
%    \begin{macrocode}
$ \def\BIC@Temp#1#2{%
$   \expandafter\def\csname BIC@AddCarry#1\endcsname##1{%
$     \ifcase##1 #1\or
$     #2%
$?    \else\BigIntCalcError:ThisCannotHappen%
$     \fi
$   }%
$ }%
$ \BIC@Temp 0{1\or2\or3\or4\or5\or6\or7\or8\or9}%
$ \BIC@Temp 1{2\or3\or4\or5\or6\or7\or8\or9\or10}%
$ \BIC@Temp 2{3\or4\or5\or6\or7\or8\or9\or10\or11}%
$ \BIC@Temp 3{4\or5\or6\or7\or8\or9\or10\or11\or12}%
$ \BIC@Temp 4{5\or6\or7\or8\or9\or10\or11\or12\or13}%
$ \BIC@Temp 5{6\or7\or8\or9\or10\or11\or12\or13\or14}%
$ \BIC@Temp 6{7\or8\or9\or10\or11\or12\or13\or14\or15}%
$ \BIC@Temp 7{8\or9\or10\or11\or12\or13\or14\or15\or16}%
$ \BIC@Temp 8{9\or10\or11\or12\or13\or14\or15\or16\or17}%
$ \BIC@Temp 9{10\or11\or12\or13\or14\or15\or16\or17\or18}%
%    \end{macrocode}
%    \end{macro}
%
%    \begin{macro}{\BIC@SubXY}
%    Preconditions:
%    \begin{itemize}
%    \item $x > y$, $x \geq 0$, and $y >= 0$
%    \item digits($x$) = digits($y$)
%    \end{itemize}
%    \begin{macrocode}
\def\BIC@SubXY#1#2!#3#4!#5!#6!{%
  \ifx\\#2\\%
    \ifx\\#3\\%
      \BIC@AfterFiFi{%
        \BIC@DoSub0!#1#5!#60!%
      }%
    \else
      \BIC@AfterFiFi{%
        \BIC@DoSub0!#1#5!#3#6!%
      }%
    \fi
  \else
    \ifx\\#4\\%
      \ifx\\#3\\%
        \BIC@AfterFiFiFi{%
          \BIC@SubXY#2!{}!#1#5!#60!%
        }%
      \else
        \BIC@AfterFiFiFi{%
          \BIC@SubXY#2!{}!#1#5!#3#6!%
        }%
      \fi
    \else
      \BIC@AfterFiFi{%
        \BIC@SubXY#2!#4!#1#5!#3#6!%
      }%
    \fi
  \BIC@Fi
}
%    \end{macrocode}
%    \end{macro}
%    \begin{macro}{\BIC@DoSub}
%    |#1|: carry\\
%    |#2|: reverted result\\
%    |#3#4|: reverted x\\
%    |#5#6|: reverted y
%    \begin{macrocode}
\def\BIC@DoSub#1#2!#3#4!#5#6!{%
  \ifx\\#4\\%
    \BIC@AfterFi{%
      \expandafter\expandafter\expandafter\BIC@SubResult
      \BIC@SubDigit#1#3#5#2%
    }%
  \else
    \BIC@AfterFi{%
      \expandafter\expandafter\expandafter\BIC@DoSub
      \BIC@SubDigit#1#3#5#2!#4!#6!%
    }%
  \BIC@Fi
}
%    \end{macrocode}
%    \end{macro}
%    \begin{macro}{\BIC@SubResult}
%    \begin{macrocode}
\def\BIC@SubResult#1{%
  \ifx#10%
    \expandafter\BIC@SubResult
  \else
    \expandafter\BIC@Space\expandafter#1%
  \fi
}
%    \end{macrocode}
%    \end{macro}
%    \begin{macro}{\BIC@SubDigit}
%    |#1|: carry\\
%    |#2|: digit of $x$\\
%    |#3|: digit of $y$
%    \begin{macrocode}
\def\BIC@SubDigit#1#2#3{%
  \romannumeral0%
& \expandafter\BIC@@SubDigit\the\numexpr#2-#3-#1!%
$ \expandafter\BIC@@AddDigit\number
$   \csname
$     BIC@SubCarry%
$     \ifcase#1 %
$       #3%
$     \else
$       \ifcase#3 1\or2\or3\or4\or5\or6\or7\or8\or9\or10\fi
$     \fi
$   \endcsname#2!%
}
%    \end{macrocode}
%    \end{macro}
%    \begin{macro}{\BIC@@SubDigit}
%    \begin{macrocode}
& \def\BIC@@SubDigit#1!{%
&   \ifnum#1<0 %
&     \BIC@AfterFi{%
&       \expandafter\BIC@Space
&       \expandafter1\the\numexpr#1+10\relax
&     }%
&   \else
&     \BIC@AfterFi{ 0#1}%
&   \BIC@Fi
& }%
%    \end{macrocode}
%    \end{macro}
%    \begin{macro}{\BIC@SubCarry0}
%    \begin{macrocode}
$ \expandafter\def\csname BIC@SubCarry0\endcsname#1{#1}%
%    \end{macrocode}
%    \end{macro}
%    \begin{macro}{\BIC@SubCarry10}%
%    \begin{macrocode}
$ \expandafter\def\csname BIC@SubCarry10\endcsname#1{1#1}%
%    \end{macrocode}
%    \end{macro}
%    \begin{macro}{\BIC@SubCarry[1-9]}
%    \begin{macrocode}
$ \def\BIC@Temp#1#2{%
$   \expandafter\def\csname BIC@SubCarry#1\endcsname##1{%
$     \ifcase##1 #2%
$?    \else\BigIntCalcError:ThisCannotHappen%
$     \fi
$   }%
$ }%
$ \BIC@Temp 1{19\or0\or1\or2\or3\or4\or5\or6\or7\or8}%
$ \BIC@Temp 2{18\or19\or0\or1\or2\or3\or4\or5\or6\or7}%
$ \BIC@Temp 3{17\or18\or19\or0\or1\or2\or3\or4\or5\or6}%
$ \BIC@Temp 4{16\or17\or18\or19\or0\or1\or2\or3\or4\or5}%
$ \BIC@Temp 5{15\or16\or17\or18\or19\or0\or1\or2\or3\or4}%
$ \BIC@Temp 6{14\or15\or16\or17\or18\or19\or0\or1\or2\or3}%
$ \BIC@Temp 7{13\or14\or15\or16\or17\or18\or19\or0\or1\or2}%
$ \BIC@Temp 8{12\or13\or14\or15\or16\or17\or18\or19\or0\or1}%
$ \BIC@Temp 9{11\or12\or13\or14\or15\or16\or17\or18\or19\or0}%
%    \end{macrocode}
%    \end{macro}
%
% \subsection{\op{Shl}, \op{Shr}}
%
%    \begin{macro}{\bigintcalcShl}
%    \begin{macrocode}
\def\bigintcalcShl#1{%
  \romannumeral0%
  \expandafter\expandafter\expandafter\BIC@Shl
  \bigintcalcNum{#1}!%
}
%    \end{macrocode}
%    \end{macro}
%    \begin{macro}{\BIC@Shl}
%    \begin{macrocode}
\def\BIC@Shl#1#2!{%
  \ifx#1-%
    \BIC@AfterFi{%
      \expandafter-\romannumeral0%
&     \BIC@@Shl#2!!%
$     \BIC@AddXY#2!#2!!!%
    }%
  \else
    \BIC@AfterFi{%
&     \BIC@@Shl#1#2!!%
$     \BIC@AddXY#1#2!#1#2!!!%
    }%
  \BIC@Fi
}
%    \end{macrocode}
%    \end{macro}
%    \begin{macro}{\BigIntCalcShl}
%    \begin{macrocode}
\def\BigIntCalcShl#1!{%
  \romannumeral0%
& \BIC@@Shl#1!!%
$ \BIC@AddXY#1!#1!!!%
}
%    \end{macrocode}
%    \end{macro}
%    \begin{macro}{\BIC@@Shl}
%    \begin{macrocode}
& \def\BIC@@Shl#1#2!{%
&   \ifx\\#2\\%
&     \BIC@AfterFi{%
&       \BIC@@@Shl0!#1%
&     }%
&   \else
&     \BIC@AfterFi{%
&       \BIC@@Shl#2!#1%
&     }%
&   \BIC@Fi
& }%
%    \end{macrocode}
%    \end{macro}
%    \begin{macro}{\BIC@@@Shl}
%    |#1|: carry\\
%    |#2|: result\\
%    |#3#4|: reverted number
%    \begin{macrocode}
& \def\BIC@@@Shl#1#2!#3#4!{%
&   \ifx\\#4\\%
&     \BIC@AfterFi{%
&       \expandafter\BIC@Space
&       \the\numexpr#3*2+#1\relax#2%
&     }%
&   \else
&     \BIC@AfterFi{%
&       \expandafter\BIC@@@@Shl\the\numexpr#3*2+#1!#2!#4!%
&     }%
&   \BIC@Fi
& }%
%    \end{macrocode}
%    \end{macro}
%    \begin{macro}{\BIC@@@@Shl}
%    \begin{macrocode}
& \def\BIC@@@@Shl#1!{%
&   \ifnum#1<10 %
&     \BIC@AfterFi{%
&       \BIC@@@Shl0#1%
&     }%
&   \else
&     \BIC@AfterFi{%
&       \BIC@@@Shl#1%
&     }%
&   \BIC@Fi
& }%
%    \end{macrocode}
%    \end{macro}
%
%    \begin{macro}{\bigintcalcShr}
%    \begin{macrocode}
\def\bigintcalcShr#1{%
  \romannumeral0%
  \expandafter\expandafter\expandafter\BIC@Shr
  \bigintcalcNum{#1}!%
}
%    \end{macrocode}
%    \end{macro}
%    \begin{macro}{\BIC@Shr}
%    \begin{macrocode}
\def\BIC@Shr#1#2!{%
  \ifx#1-%
    \expandafter-\romannumeral0%
    \BIC@AfterFi{%
      \BIC@@Shr#2!%
    }%
  \else
    \BIC@AfterFi{%
      \BIC@@Shr#1#2!%
    }%
  \BIC@Fi
}
%    \end{macrocode}
%    \end{macro}
%    \begin{macro}{\BigIntCalcShr}
%    \begin{macrocode}
\def\BigIntCalcShr#1!{%
  \romannumeral0%
  \BIC@@Shr#1!%
}
%    \end{macrocode}
%    \end{macro}
%    \begin{macro}{\BIC@@Shr}
%    \begin{macrocode}
\def\BIC@@Shr#1#2!{%
  \ifcase#1 %
    \BIC@AfterFi{ 0}%
  \or
    \ifx\\#2\\%
      \BIC@AfterFiFi{ 0}%
    \else
      \BIC@AfterFiFi{%
        \BIC@@@Shr#1#2!!%
      }%
    \fi
  \else
    \BIC@AfterFi{%
      \BIC@@@Shr0#1#2!!%
    }%
  \BIC@Fi
}
%    \end{macrocode}
%    \end{macro}
%    \begin{macro}{\BIC@@@Shr}
%    |#1|: carry\\
%    |#2#3|: number\\
%    |#4|: result
%    \begin{macrocode}
\def\BIC@@@Shr#1#2#3!#4!{%
  \ifx\\#3\\%
    \ifodd#1#2 %
      \BIC@AfterFiFi{%
&       \expandafter\BIC@ShrResult\the\numexpr(#1#2-1)/2\relax
$       \expandafter\expandafter\expandafter\BIC@ShrResult
$       \csname BIC@ShrDigit#1#2\endcsname
        #4!%
      }%
    \else
      \BIC@AfterFiFi{%
&       \expandafter\BIC@ShrResult\the\numexpr#1#2/2\relax
$       \expandafter\expandafter\expandafter\BIC@ShrResult
$       \csname BIC@ShrDigit#1#2\endcsname
        #4!%
      }%
    \fi
  \else
    \ifodd#1#2 %
      \BIC@AfterFiFi{%
&       \expandafter\BIC@@@@Shr\the\numexpr(#1#2-1)/2\relax1%
$       \expandafter\expandafter\expandafter\BIC@@@@Shr
$       \csname BIC@ShrDigit#1#2\endcsname
        #3!#4!%
      }%
    \else
      \BIC@AfterFiFi{%
&       \expandafter\BIC@@@@Shr\the\numexpr#1#2/2\relax0%
$       \expandafter\expandafter\expandafter\BIC@@@@Shr
$       \csname BIC@ShrDigit#1#2\endcsname
        #3!#4!%
      }%
    \fi
  \BIC@Fi
}
%    \end{macrocode}
%    \end{macro}
%    \begin{macro}{\BIC@ShrResult}
%    \begin{macrocode}
& \def\BIC@ShrResult#1#2!{ #2#1}%
$ \def\BIC@ShrResult#1#2#3!{ #3#1}%
%    \end{macrocode}
%    \end{macro}
%    \begin{macro}{\BIC@@@@Shr}
%    |#1|: new digit\\
%    |#2|: carry\\
%    |#3|: remaining number\\
%    |#4|: result
%    \begin{macrocode}
\def\BIC@@@@Shr#1#2#3!#4!{%
  \BIC@@@Shr#2#3!#4#1!%
}
%    \end{macrocode}
%    \end{macro}
%    \begin{macro}{\BIC@ShrDigit[00-19]}
%    \begin{macrocode}
$ \def\BIC@Temp#1#2#3#4{%
$   \expandafter\def\csname BIC@ShrDigit#1#2\endcsname{#3#4}%
$ }%
$ \BIC@Temp 0000%
$ \BIC@Temp 0101%
$ \BIC@Temp 0210%
$ \BIC@Temp 0311%
$ \BIC@Temp 0420%
$ \BIC@Temp 0521%
$ \BIC@Temp 0630%
$ \BIC@Temp 0731%
$ \BIC@Temp 0840%
$ \BIC@Temp 0941%
$ \BIC@Temp 1050%
$ \BIC@Temp 1151%
$ \BIC@Temp 1260%
$ \BIC@Temp 1361%
$ \BIC@Temp 1470%
$ \BIC@Temp 1571%
$ \BIC@Temp 1680%
$ \BIC@Temp 1781%
$ \BIC@Temp 1890%
$ \BIC@Temp 1991%
%    \end{macrocode}
%    \end{macro}
%
% \subsection{\cs{BIC@Tim}}
%
%    \begin{macro}{\BIC@Tim}
%    Macro \cs{BIC@Tim} implements ``Number \emph{tim}es digit''.\\
%    |#1|: plain number without sign\\
%    |#2|: digit
\def\BIC@Tim#1!#2{%
  \romannumeral0%
  \ifcase#2 % 0
    \BIC@AfterFi{ 0}%
  \or % 1
    \BIC@AfterFi{ #1}%
  \or % 2
    \BIC@AfterFi{%
      \BIC@Shl#1!%
    }%
  \else % 3-9
    \BIC@AfterFi{%
      \BIC@@Tim#1!!#2%
    }%
  \BIC@Fi
}
%    \begin{macrocode}
%    \end{macrocode}
%    \end{macro}
%    \begin{macro}{\BIC@@Tim}
%    |#1#2|: number\\
%    |#3|: reverted number
%    \begin{macrocode}
\def\BIC@@Tim#1#2!{%
  \ifx\\#2\\%
    \BIC@AfterFi{%
      \BIC@ProcessTim0!#1%
    }%
  \else
    \BIC@AfterFi{%
      \BIC@@Tim#2!#1%
    }%
  \BIC@Fi
}
%    \end{macrocode}
%    \end{macro}
%    \begin{macro}{\BIC@ProcessTim}
%    |#1|: carry\\
%    |#2|: result\\
%    |#3#4|: reverted number\\
%    |#5|: digit
%    \begin{macrocode}
\def\BIC@ProcessTim#1#2!#3#4!#5{%
  \ifx\\#4\\%
    \BIC@AfterFi{%
      \expandafter\BIC@Space
&     \the\numexpr#3*#5+#1\relax
$     \romannumeral0\BIC@TimDigit#3#5#1%
      #2%
    }%
  \else
    \BIC@AfterFi{%
      \expandafter\BIC@@ProcessTim
&     \the\numexpr#3*#5+#1%
$     \romannumeral0\BIC@TimDigit#3#5#1%
      !#2!#4!#5%
    }%
  \BIC@Fi
}
%    \end{macrocode}
%    \end{macro}
%    \begin{macro}{\BIC@@ProcessTim}
%    |#1#2|: carry?, new digit\\
%    |#3|: new number\\
%    |#4|: old number\\
%    |#5|: digit
%    \begin{macrocode}
\def\BIC@@ProcessTim#1#2!{%
  \ifx\\#2\\%
    \BIC@AfterFi{%
      \BIC@ProcessTim0#1%
    }%
  \else
    \BIC@AfterFi{%
      \BIC@ProcessTim#1#2%
    }%
  \BIC@Fi
}
%    \end{macrocode}
%    \end{macro}
%    \begin{macro}{\BIC@TimDigit}
%    |#1|: digit 0--9\\
%    |#2|: digit 3--9\\
%    |#3|: carry 0--9
%    \begin{macrocode}
$ \def\BIC@TimDigit#1#2#3{%
$   \ifcase#1 % 0
$     \BIC@AfterFi{ #3}%
$   \or % 1
$     \BIC@AfterFi{%
$       \expandafter\BIC@Space
$       \number\csname BIC@AddCarry#2\endcsname#3 %
$     }%
$   \else
$     \ifcase#3 %
$       \BIC@AfterFiFi{%
$         \expandafter\BIC@Space
$         \number\csname BIC@MulDigit#2\endcsname#1 %
$       }%
$     \else
$       \BIC@AfterFiFi{%
$         \expandafter\BIC@Space
$         \romannumeral0%
$         \expandafter\BIC@AddXY
$         \number\csname BIC@MulDigit#2\endcsname#1!%
$         #3!!!%
$       }%
$     \fi
$   \BIC@Fi
$ }%
%    \end{macrocode}
%    \end{macro}
%    \begin{macro}{\BIC@MulDigit[3-9]}
%    \begin{macrocode}
$ \def\BIC@Temp#1#2{%
$   \expandafter\def\csname BIC@MulDigit#1\endcsname##1{%
$     \ifcase##1 0%
$     \or ##1%
$     \or #2%
$?    \else\BigIntCalcError:ThisCannotHappen%
$     \fi
$   }%
$ }%
$ \BIC@Temp 3{6\or9\or12\or15\or18\or21\or24\or27}%
$ \BIC@Temp 4{8\or12\or16\or20\or24\or28\or32\or36}%
$ \BIC@Temp 5{10\or15\or20\or25\or30\or35\or40\or45}%
$ \BIC@Temp 6{12\or18\or24\or30\or36\or42\or48\or54}%
$ \BIC@Temp 7{14\or21\or28\or35\or42\or49\or56\or63}%
$ \BIC@Temp 8{16\or24\or32\or40\or48\or56\or64\or72}%
$ \BIC@Temp 9{18\or27\or36\or45\or54\or63\or72\or81}%
%    \end{macrocode}
%    \end{macro}
%
% \subsection{\op{Mul}}
%
%    \begin{macro}{\bigintcalcMul}
%    \begin{macrocode}
\def\bigintcalcMul#1#2{%
  \romannumeral0%
  \expandafter\expandafter\expandafter\BIC@Mul
  \bigintcalcNum{#1}!{#2}%
}
%    \end{macrocode}
%    \end{macro}
%    \begin{macro}{\BIC@Mul}
%    \begin{macrocode}
\def\BIC@Mul#1!#2{%
  \expandafter\expandafter\expandafter\BIC@MulSwitch
  \bigintcalcNum{#2}!#1!%
}
%    \end{macrocode}
%    \end{macro}
%    \begin{macro}{\BIC@MulSwitch}
%    Decision table for \cs{BIC@MulSwitch}.
%    \begin{quote}
%    \begin{tabular}[t]{@{}|l|l|l|l|l|@{}}
%      \hline
%      $x=0$ &\multicolumn{3}{l}{}    &$0$\\
%      \hline
%      $x>0$ & $y=0$ &\multicolumn2l{}&$0$\\
%      \cline{2-5}
%            & $y>0$ & $ x> y$ & $+$ & $\opMul( x, y)$\\
%      \cline{3-3}\cline{5-5}
%            &       & else    &     & $\opMul( y, x)$\\
%      \cline{2-5}
%            & $y<0$ & $ x>-y$ & $-$ & $\opMul( x,-y)$\\
%      \cline{3-3}\cline{5-5}
%            &       & else    &     & $\opMul(-y, x)$\\
%      \hline
%      $x<0$ & $y=0$ &\multicolumn2l{}&$0$\\
%      \cline{2-5}
%            & $y>0$ & $-x> y$ & $-$ & $\opMul(-x, y)$\\
%      \cline{3-3}\cline{5-5}
%            &       & else    &     & $\opMul( y,-x)$\\
%      \cline{2-5}
%            & $y<0$ & $-x>-y$ & $+$ & $\opMul(-x,-y)$\\
%      \cline{3-3}\cline{5-5}
%            &       & else    &     & $\opMul(-y,-x)$\\
%      \hline
%    \end{tabular}
%    \end{quote}
%    \begin{macrocode}
\def\BIC@MulSwitch#1#2!#3#4!{%
  \ifcase\BIC@Sgn#1#2! % x = 0
    \BIC@AfterFi{ 0}%
  \or % x > 0
    \ifcase\BIC@Sgn#3#4! % y = 0
      \BIC@AfterFiFi{ 0}%
    \or % y > 0
      \ifnum\BIC@PosCmp#1#2!#3#4!=1 % x > y
        \BIC@AfterFiFiFi{%
          \BIC@ProcessMul0!#1#2!#3#4!%
        }%
      \else % x <= y
        \BIC@AfterFiFiFi{%
          \BIC@ProcessMul0!#3#4!#1#2!%
        }%
      \fi
    \else % y < 0
      \expandafter-\romannumeral0%
      \ifnum\BIC@PosCmp#1#2!#4!=1 % x > -y
        \BIC@AfterFiFiFi{%
          \BIC@ProcessMul0!#1#2!#4!%
        }%
      \else % x <= -y
        \BIC@AfterFiFiFi{%
          \BIC@ProcessMul0!#4!#1#2!%
        }%
      \fi
    \fi
  \else % x < 0
    \ifcase\BIC@Sgn#3#4! % y = 0
      \BIC@AfterFiFi{ 0}%
    \or % y > 0
      \expandafter-\romannumeral0%
      \ifnum\BIC@PosCmp#2!#3#4!=1 % -x > y
        \BIC@AfterFiFiFi{%
          \BIC@ProcessMul0!#2!#3#4!%
        }%
      \else % -x <= y
        \BIC@AfterFiFiFi{%
          \BIC@ProcessMul0!#3#4!#2!%
        }%
      \fi
    \else % y < 0
      \ifnum\BIC@PosCmp#2!#4!=1 % -x > -y
        \BIC@AfterFiFiFi{%
          \BIC@ProcessMul0!#2!#4!%
        }%
      \else % -x <= -y
        \BIC@AfterFiFiFi{%
          \BIC@ProcessMul0!#4!#2!%
        }%
      \fi
    \fi
  \BIC@Fi
}
%    \end{macrocode}
%    \end{macro}
%    \begin{macro}{\BigIntCalcMul}
%    \begin{macrocode}
\def\BigIntCalcMul#1!#2!{%
  \romannumeral0%
  \BIC@ProcessMul0!#1!#2!%
}
%    \end{macrocode}
%    \end{macro}
%    \begin{macro}{\BIC@ProcessMul}
%    |#1|: result\\
%    |#2|: number $x$\\
%    |#3#4|: number $y$
%    \begin{macrocode}
\def\BIC@ProcessMul#1!#2!#3#4!{%
  \ifx\\#4\\%
    \BIC@AfterFi{%
      \expandafter\expandafter\expandafter\BIC@Space
      \bigintcalcAdd{\BIC@Tim#2!#3}{#10}%
    }%
  \else
    \BIC@AfterFi{%
      \expandafter\expandafter\expandafter\BIC@ProcessMul
      \bigintcalcAdd{\BIC@Tim#2!#3}{#10}!#2!#4!%
    }%
  \BIC@Fi
}
%    \end{macrocode}
%    \end{macro}
%
% \subsection{\op{Sqr}}
%
%    \begin{macro}{\bigintcalcSqr}
%    \begin{macrocode}
\def\bigintcalcSqr#1{%
  \romannumeral0%
  \expandafter\expandafter\expandafter\BIC@Sqr
  \bigintcalcNum{#1}!%
}
%    \end{macrocode}
%    \end{macro}
%    \begin{macro}{\BIC@Sqr}
%    \begin{macrocode}
\def\BIC@Sqr#1{%
   \ifx#1-%
     \expandafter\BIC@@Sqr
   \else
     \expandafter\BIC@@Sqr\expandafter#1%
   \fi
}
%    \end{macrocode}
%    \end{macro}
%    \begin{macro}{\BIC@@Sqr}
%    \begin{macrocode}
\def\BIC@@Sqr#1!{%
  \BIC@ProcessMul0!#1!#1!%
}
%    \end{macrocode}
%    \end{macro}
%
% \subsection{\op{Fac}}
%
%    \begin{macro}{\bigintcalcFac}
%    \begin{macrocode}
\def\bigintcalcFac#1{%
  \romannumeral0%
  \expandafter\expandafter\expandafter\BIC@Fac
  \bigintcalcNum{#1}!%
}
%    \end{macrocode}
%    \end{macro}
%    \begin{macro}{\BIC@Fac}
%    \begin{macrocode}
\def\BIC@Fac#1#2!{%
  \ifx#1-%
    \BIC@AfterFi{ 0\BigIntCalcError:FacNegative}%
  \else
    \ifnum\BIC@PosCmp#1#2!13!<0 %
      \ifcase#1#2 %
         \BIC@AfterFiFiFi{ 1}% 0!
      \or\BIC@AfterFiFiFi{ 1}% 1!
      \or\BIC@AfterFiFiFi{ 2}% 2!
      \or\BIC@AfterFiFiFi{ 6}% 3!
      \or\BIC@AfterFiFiFi{ 24}% 4!
      \or\BIC@AfterFiFiFi{ 120}% 5!
      \or\BIC@AfterFiFiFi{ 720}% 6!
      \or\BIC@AfterFiFiFi{ 5040}% 7!
      \or\BIC@AfterFiFiFi{ 40320}% 8!
      \or\BIC@AfterFiFiFi{ 362880}% 9!
      \or\BIC@AfterFiFiFi{ 3628800}% 10!
      \or\BIC@AfterFiFiFi{ 39916800}% 11!
      \or\BIC@AfterFiFiFi{ 479001600}% 12!
?     \else\BigIntCalcError:ThisCannotHappen%
      \fi
    \else
      \BIC@AfterFiFi{%
        \BIC@ProcessFac#1#2!479001600!%
      }%
    \fi
  \BIC@Fi
}
%    \end{macrocode}
%    \end{macro}
%    \begin{macro}{\BIC@ProcessFac}
%    |#1|: $n$\\
%    |#2|: result
%    \begin{macrocode}
\def\BIC@ProcessFac#1!#2!{%
  \ifnum\BIC@PosCmp#1!12!=0 %
    \BIC@AfterFi{ #2}%
  \else
    \BIC@AfterFi{%
      \expandafter\BIC@@ProcessFac
      \romannumeral0\BIC@ProcessMul0!#2!#1!%
      !#1!%
    }%
  \BIC@Fi
}
%    \end{macrocode}
%    \end{macro}
%    \begin{macro}{\BIC@@ProcessFac}
%    |#1|: result\\
%    |#2|: $n$
%    \begin{macrocode}
\def\BIC@@ProcessFac#1!#2!{%
  \expandafter\BIC@ProcessFac
  \romannumeral0\BIC@Dec#2!{}%
  !#1!%
}
%    \end{macrocode}
%    \end{macro}
%
% \subsection{\op{Pow}}
%
%    \begin{macro}{\bigintcalcPow}
%    |#1|: basis\\
%    |#2|: power
%    \begin{macrocode}
\def\bigintcalcPow#1{%
  \romannumeral0%
  \expandafter\expandafter\expandafter\BIC@Pow
  \bigintcalcNum{#1}!%
}
%    \end{macrocode}
%    \end{macro}
%    \begin{macro}{\BIC@Pow}
%    |#1|: basis\\
%    |#2|: power
%    \begin{macrocode}
\def\BIC@Pow#1!#2{%
  \expandafter\expandafter\expandafter\BIC@PowSwitch
  \bigintcalcNum{#2}!#1!%
}
%    \end{macrocode}
%    \end{macro}
%    \begin{macro}{\BIC@PowSwitch}
%    |#1#2|: power $y$\\
%    |#3#4|: basis $x$\\
%    Decision table for \cs{BIC@PowSwitch}.
%    \begin{quote}
%    \def\M#1{\multicolumn{#1}{l}{}}%
%    \begin{tabular}[t]{@{}|l|l|l|l|@{}}
%    \hline
%    $y=0$ & \M{2}                        & $1$\\
%    \hline
%    $y=1$ & \M{2}                        & $x$\\
%    \hline
%    $y=2$ & $x<0$          & \M{1}       & $\opMul(-x,-x)$\\
%    \cline{2-4}
%          & else           & \M{1}       & $\opMul(x,x)$\\
%    \hline
%    $y<0$ & $x=0$          & \M{1}       & DivisionByZero\\
%    \cline{2-4}
%          & $x=1$          & \M{1}       & $1$\\
%    \cline{2-4}
%          & $x=-1$         & $\opODD(y)$ & $-1$\\
%    \cline{3-4}
%          &                & else        & $1$\\
%    \cline{2-4}
%          & else ($\string|x\string|>1$) & \M{1}       & $0$\\
%    \hline
%    $y>2$ & $x=0$          & \M{1}       & $0$\\
%    \cline{2-4}
%          & $x=1$          & \M{1}       & $1$\\
%    \cline{2-4}
%          & $x=-1$         & $\opODD(y)$ & $-1$\\
%    \cline{3-4}
%          &                & else        & $1$\\
%    \cline{2-4}
%          & $x<-1$ ($x<0$) & $\opODD(y)$ & $-\opPow(-x,y)$\\
%    \cline{3-4}
%          &                & else        & $\opPow(-x,y)$\\
%    \cline{2-4}
%          & else ($x>1$)   & \M{1}       & $\opPow(x,y)$\\
%    \hline
%    \end{tabular}
%    \end{quote}
%    \begin{macrocode}
\def\BIC@PowSwitch#1#2!#3#4!{%
  \ifcase\ifx\\#2\\%
           \ifx#100 % y = 0
           \else\ifx#111 % y = 1
           \else\ifx#122 % y = 2
           \else4 % y > 2
           \fi\fi\fi
         \else
           \ifx#1-3 % y < 0
           \else4 % y > 2
           \fi
         \fi
    \BIC@AfterFi{ 1}% y = 0
  \or % y = 1
    \BIC@AfterFi{ #3#4}%
  \or % y = 2
    \ifx#3-% x < 0
      \BIC@AfterFiFi{%
        \BIC@ProcessMul0!#4!#4!%
      }%
    \else % x >= 0
      \BIC@AfterFiFi{%
        \BIC@ProcessMul0!#3#4!#3#4!%
      }%
    \fi
  \or % y < 0
    \ifcase\ifx\\#4\\%
             \ifx#300 % x = 0
             \else\ifx#311 % x = 1
             \else3 % x > 1
             \fi\fi
           \else
             \ifcase\BIC@MinusOne#3#4! %
               3 % |x| > 1
             \or
               2 % x = -1
?            \else\BigIntCalcError:ThisCannotHappen%
             \fi
           \fi
      \BIC@AfterFiFi{ 0\BigIntCalcError:DivisionByZero}% x = 0
    \or % x = 1
      \BIC@AfterFiFi{ 1}% x = 1
    \or % x = -1
      \ifcase\BIC@ModTwo#2! % even(y)
        \BIC@AfterFiFiFi{ 1}%
      \or % odd(y)
        \BIC@AfterFiFiFi{ -1}%
?     \else\BigIntCalcError:ThisCannotHappen%
      \fi
    \or % |x| > 1
      \BIC@AfterFiFi{ 0}%
?   \else\BigIntCalcError:ThisCannotHappen%
    \fi
  \or % y > 2
    \ifcase\ifx\\#4\\%
             \ifx#300 % x = 0
             \else\ifx#311 % x = 1
             \else4 % x > 1
             \fi\fi
           \else
             \ifx#3-%
               \ifcase\BIC@MinusOne#3#4! %
                 3 % x < -1
               \else
                 2 % x = -1
               \fi
             \else
               4 % x > 1
             \fi
           \fi
      \BIC@AfterFiFi{ 0}% x = 0
    \or % x = 1
      \BIC@AfterFiFi{ 1}% x = 1
    \or % x = -1
      \ifcase\BIC@ModTwo#1#2! % even(y)
        \BIC@AfterFiFiFi{ 1}%
      \or % odd(y)
        \BIC@AfterFiFiFi{ -1}%
?     \else\BigIntCalcError:ThisCannotHappen%
      \fi
    \or % x < -1
      \ifcase\BIC@ModTwo#1#2! % even(y)
        \BIC@AfterFiFiFi{%
          \BIC@PowRec#4!#1#2!1!%
        }%
      \or % odd(y)
        \expandafter-\romannumeral0%
        \BIC@AfterFiFiFi{%
          \BIC@PowRec#4!#1#2!1!%
        }%
?     \else\BigIntCalcError:ThisCannotHappen%
      \fi
    \or % x > 1
      \BIC@AfterFiFi{%
        \BIC@PowRec#3#4!#1#2!1!%
      }%
?   \else\BigIntCalcError:ThisCannotHappen%
    \fi
? \else\BigIntCalcError:ThisCannotHappen%
  \BIC@Fi
}
%    \end{macrocode}
%    \end{macro}
%
% \subsubsection{Help macros}
%
%    \begin{macro}{\BIC@ModTwo}
%    Macro \cs{BIC@ModTwo} expects a number without sign
%    and returns digit |1| or |0| if the number is odd or even.
%    \begin{macrocode}
\def\BIC@ModTwo#1#2!{%
  \ifx\\#2\\%
    \ifodd#1 %
      \BIC@AfterFiFi1%
    \else
      \BIC@AfterFiFi0%
    \fi
  \else
    \BIC@AfterFi{%
      \BIC@ModTwo#2!%
    }%
  \BIC@Fi
}
%    \end{macrocode}
%    \end{macro}
%
%    \begin{macro}{\BIC@MinusOne}
%    Macro \cs{BIC@MinusOne} expects a number and returns
%    digit |1| if the number equals minus one and returns |0| otherwise.
%    \begin{macrocode}
\def\BIC@MinusOne#1#2!{%
  \ifx#1-%
    \BIC@@MinusOne#2!%
  \else
    0%
  \fi
}
%    \end{macrocode}
%    \end{macro}
%    \begin{macro}{\BIC@@MinusOne}
%    \begin{macrocode}
\def\BIC@@MinusOne#1#2!{%
  \ifx#11%
    \ifx\\#2\\%
      1%
    \else
      0%
    \fi
  \else
    0%
  \fi
}
%    \end{macrocode}
%    \end{macro}
%
% \subsubsection{Recursive calculation}
%
%    \begin{macro}{\BIC@PowRec}
%\begin{quote}
%\begin{verbatim}
%Pow(x, y) {
%  PowRec(x, y, 1)
%}
%PowRec(x, y, r) {
%  if y == 1 then
%    return r
%  else
%    ifodd y then
%      return PowRec(x*x, y div 2, r*x) % y div 2 = (y-1)/2
%    else
%      return PowRec(x*x, y div 2, r)
%    fi
%  fi
%}
%\end{verbatim}
%\end{quote}
%    |#1|: $x$ (basis)\\
%    |#2#3|: $y$ (power)\\
%    |#4|: $r$ (result)
%    \begin{macrocode}
\def\BIC@PowRec#1!#2#3!#4!{%
  \ifcase\ifx#21\ifx\\#3\\0 \else1 \fi\else1 \fi % y = 1
    \ifnum\BIC@PosCmp#1!#4!=1 % x > r
      \BIC@AfterFiFi{%
        \BIC@ProcessMul0!#1!#4!%
      }%
    \else
      \BIC@AfterFiFi{%
        \BIC@ProcessMul0!#4!#1!%
      }%
    \fi
  \or
    \ifcase\BIC@ModTwo#2#3! % even(y)
      \BIC@AfterFiFi{%
        \expandafter\BIC@@PowRec\romannumeral0%
        \BIC@@Shr#2#3!%
        !#1!#4!%
      }%
    \or % odd(y)
      \ifnum\BIC@PosCmp#1!#4!=1 % x > r
        \BIC@AfterFiFiFi{%
          \expandafter\BIC@@@PowRec\romannumeral0%
          \BIC@ProcessMul0!#1!#4!%
          !#1!#2#3!%
        }%
      \else
        \BIC@AfterFiFiFi{%
          \expandafter\BIC@@@PowRec\romannumeral0%
          \BIC@ProcessMul0!#1!#4!%
          !#1!#2#3!%
        }%
      \fi
?   \else\BigIntCalcError:ThisCannotHappen%
    \fi
? \else\BigIntCalcError:ThisCannotHappen%
  \BIC@Fi
}
%    \end{macrocode}
%    \end{macro}
%    \begin{macro}{\BIC@@PowRec}
%    |#1|: $y/2$\\
%    |#2|: $x$\\
%    |#3|: new $r$ ($r$ or $r*x$)
%    \begin{macrocode}
\def\BIC@@PowRec#1!#2!#3!{%
  \expandafter\BIC@PowRec\romannumeral0%
  \BIC@ProcessMul0!#2!#2!%
  !#1!#3!%
}
%    \end{macrocode}
%    \end{macro}
%    \begin{macro}{\BIC@@@PowRec}
%    |#1|: $r*x$
%    |#2|: $x$
%    |#3|: $y$
%    \begin{macrocode}
\def\BIC@@@PowRec#1!#2!#3!{%
  \expandafter\BIC@@PowRec\romannumeral0%
  \BIC@@Shr#3!%
  !#2!#1!%
}
%    \end{macrocode}
%    \end{macro}
%
% \subsection{\op{Div}}
%
%    \begin{macro}{\bigintcalcDiv}
%    |#1|: $x$\\
%    |#2|: $y$ (divisor)
%    \begin{macrocode}
\def\bigintcalcDiv#1{%
  \romannumeral0%
  \expandafter\expandafter\expandafter\BIC@Div
  \bigintcalcNum{#1}!%
}
%    \end{macrocode}
%    \end{macro}
%    \begin{macro}{\BIC@Div}
%    |#1|: $x$\\
%    |#2|: $y$
%    \begin{macrocode}
\def\BIC@Div#1!#2{%
  \expandafter\expandafter\expandafter\BIC@DivSwitchSign
  \bigintcalcNum{#2}!#1!%
}
%    \end{macrocode}
%    \end{macro}
%    \begin{macro}{\BigIntCalcDiv}
%    \begin{macrocode}
\def\BigIntCalcDiv#1!#2!{%
  \romannumeral0%
  \BIC@DivSwitchSign#2!#1!%
}
%    \end{macrocode}
%    \end{macro}
%    \begin{macro}{\BIC@DivSwitchSign}
%    Decision table for \cs{BIC@DivSwitchSign}.
%    \begin{quote}
%    \begin{tabular}{@{}|l|l|l|@{}}
%    \hline
%    $y=0$ & \multicolumn{1}{l}{} & DivisionByZero\\
%    \hline
%    $y>0$ & $x=0$ & $0$\\
%    \cline{2-3}
%          & $x>0$ & DivSwitch$(+,x,y)$\\
%    \cline{2-3}
%          & $x<0$ & DivSwitch$(-,-x,y)$\\
%    \hline
%    $y<0$ & $x=0$ & $0$\\
%    \cline{2-3}
%          & $x>0$ & DivSwitch$(-,x,-y)$\\
%    \cline{2-3}
%          & $x<0$ & DivSwitch$(+,-x,-y)$\\
%    \hline
%    \end{tabular}
%    \end{quote}
%    |#1|: $y$ (divisor)\\
%    |#2|: $x$
%    \begin{macrocode}
\def\BIC@DivSwitchSign#1#2!#3#4!{%
  \ifcase\BIC@Sgn#1#2! % y = 0
    \BIC@AfterFi{ 0\BigIntCalcError:DivisionByZero}%
  \or % y > 0
    \ifcase\BIC@Sgn#3#4! % x = 0
      \BIC@AfterFiFi{ 0}%
    \or % x > 0
      \BIC@AfterFiFi{%
        \BIC@DivSwitch{}#3#4!#1#2!%
      }%
    \else % x < 0
      \BIC@AfterFiFi{%
        \BIC@DivSwitch-#4!#1#2!%
      }%
    \fi
  \else % y < 0
    \ifcase\BIC@Sgn#3#4! % x = 0
      \BIC@AfterFiFi{ 0}%
    \or % x > 0
      \BIC@AfterFiFi{%
        \BIC@DivSwitch-#3#4!#2!%
      }%
    \else % x < 0
      \BIC@AfterFiFi{%
        \BIC@DivSwitch{}#4!#2!%
      }%
    \fi
  \BIC@Fi
}
%    \end{macrocode}
%    \end{macro}
%    \begin{macro}{\BIC@DivSwitch}
%    Decision table for \cs{BIC@DivSwitch}.
%    \begin{quote}
%    \begin{tabular}{@{}|l|l|l|@{}}
%    \hline
%    $y=x$ & \multicolumn{1}{l}{} & sign $1$\\
%    \hline
%    $y>x$ & \multicolumn{1}{l}{} & $0$\\
%    \hline
%    $y<x$ & $y=1$                & sign $x$\\
%    \cline{2-3}
%          & $y=2$                & sign Shr$(x)$\\
%    \cline{2-3}
%          & $y=4$                & sign Shr(Shr$(x)$)\\
%    \cline{2-3}
%          & else                 & sign ProcessDiv$(x,y)$\\
%    \hline
%    \end{tabular}
%    \end{quote}
%    |#1|: sign\\
%    |#2|: $x$\\
%    |#3#4|: $y$ ($y\ne 0$)
%    \begin{macrocode}
\def\BIC@DivSwitch#1#2!#3#4!{%
  \ifcase\BIC@PosCmp#3#4!#2!% y = x
    \BIC@AfterFi{ #11}%
  \or % y > x
    \BIC@AfterFi{ 0}%
  \else % y < x
    \ifx\\#1\\%
    \else
      \expandafter-\romannumeral0%
    \fi
    \ifcase\ifx\\#4\\%
             \ifx#310 % y = 1
             \else\ifx#321 % y = 2
             \else\ifx#342 % y = 4
             \else3 % y > 2
             \fi\fi\fi
           \else
             3 % y > 2
           \fi
      \BIC@AfterFiFi{ #2}% y = 1
    \or % y = 2
      \BIC@AfterFiFi{%
        \BIC@@Shr#2!%
      }%
    \or % y = 4
      \BIC@AfterFiFi{%
        \expandafter\BIC@@Shr\romannumeral0%
          \BIC@@Shr#2!!%
      }%
    \or % y > 2
      \BIC@AfterFiFi{%
        \BIC@DivStartX#2!#3#4!!!%
      }%
?   \else\BigIntCalcError:ThisCannotHappen%
    \fi
  \BIC@Fi
}
%    \end{macrocode}
%    \end{macro}
%    \begin{macro}{\BIC@ProcessDiv}
%    |#1#2|: $x$\\
%    |#3#4|: $y$\\
%    |#5|: collect first digits of $x$\\
%    |#6|: corresponding digits of $y$
%    \begin{macrocode}
\def\BIC@DivStartX#1#2!#3#4!#5!#6!{%
  \ifx\\#4\\%
    \BIC@AfterFi{%
      \BIC@DivStartYii#6#3#4!{#5#1}#2=!%
    }%
  \else
    \BIC@AfterFi{%
      \BIC@DivStartX#2!#4!#5#1!#6#3!%
    }%
  \BIC@Fi
}
%    \end{macrocode}
%    \end{macro}
%    \begin{macro}{\BIC@DivStartYii}
%    |#1|: $y$\\
%    |#2|: $x$, |=|
%    \begin{macrocode}
\def\BIC@DivStartYii#1!{%
  \expandafter\BIC@DivStartYiv\romannumeral0%
  \BIC@Shl#1!%
  !#1!%
}
%    \end{macrocode}
%    \end{macro}
%    \begin{macro}{\BIC@DivStartYiv}
%    |#1|: $2y$\\
%    |#2|: $y$\\
%    |#3|: $x$, |=|
%    \begin{macrocode}
\def\BIC@DivStartYiv#1!{%
  \expandafter\BIC@DivStartYvi\romannumeral0%
  \BIC@Shl#1!%
  !#1!%
}
%    \end{macrocode}
%    \end{macro}
%    \begin{macro}{\BIC@DivStartYvi}
%    |#1|: $4y$\\
%    |#2|: $2y$\\
%    |#3|: $y$\\
%    |#4|: $x$, |=|
%    \begin{macrocode}
\def\BIC@DivStartYvi#1!#2!{%
  \expandafter\BIC@DivStartYviii\romannumeral0%
  \BIC@AddXY#1!#2!!!%
  !#1!#2!%
}
%    \end{macrocode}
%    \end{macro}
%    \begin{macro}{\BIC@DivStartYviii}
%    |#1|: $6y$\\
%    |#2|: $4y$\\
%    |#3|: $2y$\\
%    |#4|: $y$\\
%    |#5|: $x$, |=|
%    \begin{macrocode}
\def\BIC@DivStartYviii#1!#2!{%
  \expandafter\BIC@DivStart\romannumeral0%
  \BIC@Shl#2!%
  !#1!#2!%
}
%    \end{macrocode}
%    \end{macro}
%    \begin{macro}{\BIC@DivStart}
%    |#1|: $8y$\\
%    |#2|: $6y$\\
%    |#3|: $4y$\\
%    |#4|: $2y$\\
%    |#5|: $y$\\
%    |#6|: $x$, |=|
%    \begin{macrocode}
\def\BIC@DivStart#1!#2!#3!#4!#5!#6!{%
  \BIC@ProcessDiv#6!!#5!#4!#3!#2!#1!=%
}
%    \end{macrocode}
%    \end{macro}
%    \begin{macro}{\BIC@ProcessDiv}
%    |#1#2#3|: $x$, |=|\\
%    |#4|: result\\
%    |#5|: $y$\\
%    |#6|: $2y$\\
%    |#7|: $4y$\\
%    |#8|: $6y$\\
%    |#9|: $8y$
%    \begin{macrocode}
\def\BIC@ProcessDiv#1#2#3!#4!#5!{%
  \ifcase\BIC@PosCmp#5!#1!% y = #1
    \ifx#2=%
      \BIC@AfterFiFi{\BIC@DivCleanup{#41}}%
    \else
      \BIC@AfterFiFi{%
        \BIC@ProcessDiv#2#3!#41!#5!%
      }%
    \fi
  \or % y > #1
    \ifx#2=%
      \BIC@AfterFiFi{\BIC@DivCleanup{#40}}%
    \else
      \ifx\\#4\\%
        \BIC@AfterFiFiFi{%
          \BIC@ProcessDiv{#1#2}#3!!#5!%
        }%
      \else
        \BIC@AfterFiFiFi{%
          \BIC@ProcessDiv{#1#2}#3!#40!#5!%
        }%
      \fi
    \fi
  \else % y < #1
    \BIC@AfterFi{%
      \BIC@@ProcessDiv{#1}#2#3!#4!#5!%
    }%
  \BIC@Fi
}
%    \end{macrocode}
%    \end{macro}
%    \begin{macro}{\BIC@DivCleanup}
%    |#1|: result\\
%    |#2|: garbage
%    \begin{macrocode}
\def\BIC@DivCleanup#1#2={ #1}%
%    \end{macrocode}
%    \end{macro}
%    \begin{macro}{\BIC@@ProcessDiv}
%    \begin{macrocode}
\def\BIC@@ProcessDiv#1#2#3!#4!#5!#6!#7!{%
  \ifcase\BIC@PosCmp#7!#1!% 4y = #1
    \ifx#2=%
      \BIC@AfterFiFi{\BIC@DivCleanup{#44}}%
    \else
     \BIC@AfterFiFi{%
       \BIC@ProcessDiv#2#3!#44!#5!#6!#7!%
     }%
    \fi
  \or % 4y > #1
    \ifcase\BIC@PosCmp#6!#1!% 2y = #1
      \ifx#2=%
        \BIC@AfterFiFiFi{\BIC@DivCleanup{#42}}%
      \else
        \BIC@AfterFiFiFi{%
          \BIC@ProcessDiv#2#3!#42!#5!#6!#7!%
        }%
      \fi
    \or % 2y > #1
      \ifx#2=%
        \BIC@AfterFiFiFi{\BIC@DivCleanup{#41}}%
      \else
        \BIC@AfterFiFiFi{%
          \BIC@DivSub#1!#5!#2#3!#41!#5!#6!#7!%
        }%
      \fi
    \else % 2y < #1
      \BIC@AfterFiFi{%
        \expandafter\BIC@ProcessDivII\romannumeral0%
        \BIC@SubXY#1!#6!!!%
        !#2#3!#4!#5!23%
        #6!#7!%
      }%
    \fi
  \else % 4y < #1
    \BIC@AfterFi{%
      \BIC@@@ProcessDiv{#1}#2#3!#4!#5!#6!#7!%
    }%
  \BIC@Fi
}
%    \end{macrocode}
%    \end{macro}
%    \begin{macro}{\BIC@DivSub}
%    Next token group: |#1|-|#2| and next digit |#3|.
%    \begin{macrocode}
\def\BIC@DivSub#1!#2!#3{%
  \expandafter\BIC@ProcessDiv\expandafter{%
    \romannumeral0%
    \BIC@SubXY#1!#2!!!%
    #3%
  }%
}
%    \end{macrocode}
%    \end{macro}
%    \begin{macro}{\BIC@ProcessDivII}
%    |#1|: $x'-2y$\\
%    |#2#3|: remaining $x$, |=|\\
%    |#4|: result\\
%    |#5|: $y$\\
%    |#6|: first possible result digit\\
%    |#7|: second possible result digit
%    \begin{macrocode}
\def\BIC@ProcessDivII#1!#2#3!#4!#5!#6#7{%
  \ifcase\BIC@PosCmp#5!#1!% y = #1
    \ifx#2=%
      \BIC@AfterFiFi{\BIC@DivCleanup{#4#7}}%
    \else
      \BIC@AfterFiFi{%
        \BIC@ProcessDiv#2#3!#4#7!#5!%
      }%
    \fi
  \or % y > #1
    \ifx#2=%
      \BIC@AfterFiFi{\BIC@DivCleanup{#4#6}}%
    \else
      \BIC@AfterFiFi{%
        \BIC@ProcessDiv{#1#2}#3!#4#6!#5!%
      }%
    \fi
  \else % y < #1
    \ifx#2=%
      \BIC@AfterFiFi{\BIC@DivCleanup{#4#7}}%
    \else
      \BIC@AfterFiFi{%
        \BIC@DivSub#1!#5!#2#3!#4#7!#5!%
      }%
    \fi
  \BIC@Fi
}
%    \end{macrocode}
%    \end{macro}
%    \begin{macro}{\BIC@ProcessDivIV}
%    |#1#2#3|: $x$, |=|, $x>4y$\\
%    |#4|: result\\
%    |#5|: $y$\\
%    |#6|: $2y$\\
%    |#7|: $4y$\\
%    |#8|: $6y$\\
%    |#9|: $8y$
%    \begin{macrocode}
\def\BIC@@@ProcessDiv#1#2#3!#4!#5!#6!#7!#8!#9!{%
  \ifcase\BIC@PosCmp#8!#1!% 6y = #1
    \ifx#2=%
      \BIC@AfterFiFi{\BIC@DivCleanup{#46}}%
    \else
      \BIC@AfterFiFi{%
        \BIC@ProcessDiv#2#3!#46!#5!#6!#7!#8!#9!%
      }%
    \fi
  \or % 6y > #1
    \BIC@AfterFi{%
      \expandafter\BIC@ProcessDivII\romannumeral0%
      \BIC@SubXY#1!#7!!!%
      !#2#3!#4!#5!45%
      #6!#7!#8!#9!%
    }%
  \else % 6y < #1
    \ifcase\BIC@PosCmp#9!#1!% 8y = #1
      \ifx#2=%
        \BIC@AfterFiFiFi{\BIC@DivCleanup{#48}}%
      \else
        \BIC@AfterFiFiFi{%
          \BIC@ProcessDiv#2#3!#48!#5!#6!#7!#8!#9!%
        }%
      \fi
    \or % 8y > #1
      \BIC@AfterFiFi{%
        \expandafter\BIC@ProcessDivII\romannumeral0%
        \BIC@SubXY#1!#8!!!%
        !#2#3!#4!#5!67%
        #6!#7!#8!#9!%
      }%
    \else % 8y < #1
      \BIC@AfterFiFi{%
        \expandafter\BIC@ProcessDivII\romannumeral0%
        \BIC@SubXY#1!#9!!!%
        !#2#3!#4!#5!89%
        #6!#7!#8!#9!%
      }%
    \fi
  \BIC@Fi
}
%    \end{macrocode}
%    \end{macro}
%
% \subsection{\op{Mod}}
%
%    \begin{macro}{\bigintcalcMod}
%    |#1|: $x$\\
%    |#2|: $y$
%    \begin{macrocode}
\def\bigintcalcMod#1{%
  \romannumeral0%
  \expandafter\expandafter\expandafter\BIC@Mod
  \bigintcalcNum{#1}!%
}
%    \end{macrocode}
%    \end{macro}
%    \begin{macro}{\BIC@Mod}
%    |#1|: $x$\\
%    |#2|: $y$
%    \begin{macrocode}
\def\BIC@Mod#1!#2{%
  \expandafter\expandafter\expandafter\BIC@ModSwitchSign
  \bigintcalcNum{#2}!#1!%
}
%    \end{macrocode}
%    \end{macro}
%    \begin{macro}{\BigIntCalcMod}
%    \begin{macrocode}
\def\BigIntCalcMod#1!#2!{%
  \romannumeral0%
  \BIC@ModSwitchSign#2!#1!%
}
%    \end{macrocode}
%    \end{macro}
%    \begin{macro}{\BIC@ModSwitchSign}
%    Decision table for \cs{BIC@ModSwitchSign}.
%    \begin{quote}
%    \begin{tabular}{@{}|l|l|l|@{}}
%    \hline
%    $y=0$ & \multicolumn{1}{l}{} & DivisionByZero\\
%    \hline
%    $y>0$ & $x=0$ & $0$\\
%    \cline{2-3}
%          & else  & ModSwitch$(+,x,y)$\\
%    \hline
%    $y<0$ & \multicolumn{1}{l}{} & ModSwitch$(-,-x,-y)$\\
%    \hline
%    \end{tabular}
%    \end{quote}
%    |#1#2|: $y$\\
%    |#3#4|: $x$
%    \begin{macrocode}
\def\BIC@ModSwitchSign#1#2!#3#4!{%
  \ifcase\ifx\\#2\\%
           \ifx#100 % y = 0
           \else1 % y > 0
           \fi
          \else
            \ifx#1-2 % y < 0
            \else1 % y > 0
            \fi
          \fi
    \BIC@AfterFi{ 0\BigIntCalcError:DivisionByZero}%
  \or % y > 0
    \ifcase\ifx\\#4\\\ifx#300 \else1 \fi\else1 \fi % x = 0
      \BIC@AfterFiFi{ 0}%
    \else
      \BIC@AfterFiFi{%
        \BIC@ModSwitch{}#3#4!#1#2!%
      }%
    \fi
  \else % y < 0
    \ifcase\ifx\\#4\\%
             \ifx#300 % x = 0
             \else1 % x > 0
             \fi
           \else
             \ifx#3-2 % x < 0
             \else1 % x > 0
             \fi
           \fi
      \BIC@AfterFiFi{ 0}%
    \or % x > 0
      \BIC@AfterFiFi{%
        \BIC@ModSwitch--#3#4!#2!%
      }%
    \else % x < 0
      \BIC@AfterFiFi{%
        \BIC@ModSwitch-#4!#2!%
      }%
    \fi
  \BIC@Fi
}
%    \end{macrocode}
%    \end{macro}
%    \begin{macro}{\BIC@ModSwitch}
%    Decision table for \cs{BIC@ModSwitch}.
%    \begin{quote}
%    \begin{tabular}{@{}|l|l|l|@{}}
%    \hline
%    $y=1$ & \multicolumn{1}{l}{} & $0$\\
%    \hline
%    $y=2$ & ifodd$(x)$ & sign $1$\\
%    \cline{2-3}
%          & else       & $0$\\
%    \hline
%    $y>2$ & $x<0$      &
%        $z\leftarrow x-(x/y)*y;\quad (z<0)\mathbin{?}z+y\mathbin{:}z$\\
%    \cline{2-3}
%          & $x>0$      & $x - (x/y) * y$\\
%    \hline
%    \end{tabular}
%    \end{quote}
%    |#1|: sign\\
%    |#2#3|: $x$\\
%    |#4#5|: $y$
%    \begin{macrocode}
\def\BIC@ModSwitch#1#2#3!#4#5!{%
  \ifcase\ifx\\#5\\%
           \ifx#410 % y = 1
           \else\ifx#421 % y = 2
           \else2 % y > 2
           \fi\fi
         \else2 % y > 2
         \fi
    \BIC@AfterFi{ 0}% y = 1
  \or % y = 2
    \ifcase\BIC@ModTwo#2#3! % even(x)
      \BIC@AfterFiFi{ 0}%
    \or % odd(x)
      \BIC@AfterFiFi{ #11}%
?   \else\BigIntCalcError:ThisCannotHappen%
    \fi
  \or % y > 2
    \ifx\\#1\\%
    \else
      \expandafter\BIC@Space\romannumeral0%
      \expandafter\BIC@ModMinus\romannumeral0%
    \fi
    \ifx#2-% x < 0
      \BIC@AfterFiFi{%
        \expandafter\expandafter\expandafter\BIC@ModX
        \bigintcalcSub{#2#3}{%
          \bigintcalcMul{#4#5}{\bigintcalcDiv{#2#3}{#4#5}}%
        }!#4#5!%
      }%
    \else % x > 0
      \BIC@AfterFiFi{%
        \expandafter\expandafter\expandafter\BIC@Space
        \bigintcalcSub{#2#3}{%
          \bigintcalcMul{#4#5}{\bigintcalcDiv{#2#3}{#4#5}}%
        }%
      }%
    \fi
? \else\BigIntCalcError:ThisCannotHappen%
  \BIC@Fi
}
%    \end{macrocode}
%    \end{macro}
%    \begin{macro}{\BIC@ModMinus}
%    \begin{macrocode}
\def\BIC@ModMinus#1{%
  \ifx#10%
    \BIC@AfterFi{ 0}%
  \else
    \BIC@AfterFi{ -#1}%
  \BIC@Fi
}
%    \end{macrocode}
%    \end{macro}
%    \begin{macro}{\BIC@ModX}
%    |#1#2|: $z$\\
%    |#3|: $x$
%    \begin{macrocode}
\def\BIC@ModX#1#2!#3!{%
  \ifx#1-% z < 0
    \BIC@AfterFi{%
      \expandafter\BIC@Space\romannumeral0%
      \BIC@SubXY#3!#2!!!%
    }%
  \else % z >= 0
    \BIC@AfterFi{ #1#2}%
  \BIC@Fi
}
%    \end{macrocode}
%    \end{macro}
%
%    \begin{macrocode}
\BIC@AtEnd%
%    \end{macrocode}
%
%    \begin{macrocode}
%</package>
%    \end{macrocode}
%
% \section{Test}
%
% \subsection{Catcode checks for loading}
%
%    \begin{macrocode}
%<*test1>
%    \end{macrocode}
%    \begin{macrocode}
\catcode`\{=1 %
\catcode`\}=2 %
\catcode`\#=6 %
\catcode`\@=11 %
\expandafter\ifx\csname count@\endcsname\relax
  \countdef\count@=255 %
\fi
\expandafter\ifx\csname @gobble\endcsname\relax
  \long\def\@gobble#1{}%
\fi
\expandafter\ifx\csname @firstofone\endcsname\relax
  \long\def\@firstofone#1{#1}%
\fi
\expandafter\ifx\csname loop\endcsname\relax
  \expandafter\@firstofone
\else
  \expandafter\@gobble
\fi
{%
  \def\loop#1\repeat{%
    \def\body{#1}%
    \iterate
  }%
  \def\iterate{%
    \body
      \let\next\iterate
    \else
      \let\next\relax
    \fi
    \next
  }%
  \let\repeat=\fi
}%
\def\RestoreCatcodes{}
\count@=0 %
\loop
  \edef\RestoreCatcodes{%
    \RestoreCatcodes
    \catcode\the\count@=\the\catcode\count@\relax
  }%
\ifnum\count@<255 %
  \advance\count@ 1 %
\repeat

\def\RangeCatcodeInvalid#1#2{%
  \count@=#1\relax
  \loop
    \catcode\count@=15 %
  \ifnum\count@<#2\relax
    \advance\count@ 1 %
  \repeat
}
\def\RangeCatcodeCheck#1#2#3{%
  \count@=#1\relax
  \loop
    \ifnum#3=\catcode\count@
    \else
      \errmessage{%
        Character \the\count@\space
        with wrong catcode \the\catcode\count@\space
        instead of \number#3%
      }%
    \fi
  \ifnum\count@<#2\relax
    \advance\count@ 1 %
  \repeat
}
\def\space{ }
\expandafter\ifx\csname LoadCommand\endcsname\relax
  \def\LoadCommand{\input bigintcalc.sty\relax}%
\fi
\def\Test{%
  \RangeCatcodeInvalid{0}{47}%
  \RangeCatcodeInvalid{58}{64}%
  \RangeCatcodeInvalid{91}{96}%
  \RangeCatcodeInvalid{123}{255}%
  \catcode`\@=12 %
  \catcode`\\=0 %
  \catcode`\%=14 %
  \LoadCommand
  \RangeCatcodeCheck{0}{36}{15}%
  \RangeCatcodeCheck{37}{37}{14}%
  \RangeCatcodeCheck{38}{47}{15}%
  \RangeCatcodeCheck{48}{57}{12}%
  \RangeCatcodeCheck{58}{63}{15}%
  \RangeCatcodeCheck{64}{64}{12}%
  \RangeCatcodeCheck{65}{90}{11}%
  \RangeCatcodeCheck{91}{91}{15}%
  \RangeCatcodeCheck{92}{92}{0}%
  \RangeCatcodeCheck{93}{96}{15}%
  \RangeCatcodeCheck{97}{122}{11}%
  \RangeCatcodeCheck{123}{255}{15}%
  \RestoreCatcodes
}
\Test
\csname @@end\endcsname
\end
%    \end{macrocode}
%    \begin{macrocode}
%</test1>
%    \end{macrocode}
%
% \subsection{Macro tests}
%
% \subsubsection{Preamble with test macro definitions}
%
%    \begin{macrocode}
%<*test2>
\NeedsTeXFormat{LaTeX2e}
\nofiles
\documentclass{article}
%<noetex>\let\SavedNumexpr\numexpr
%<noetex>\let\numexpr\UNDEFINED
\makeatletter
\chardef\BIC@TestMode=1 %
\makeatother
\usepackage{bigintcalc}[2016/05/16]
%<noetex>\let\numexpr\SavedNumexpr
\usepackage{qstest}
\IncludeTests{*}
\LogTests{log}{*}{*}
\newcommand*{\TestSpaceAtEnd}[1]{%
%<noetex>  \let\SavedNumexpr\numexpr
%<noetex>  \let\numexpr\UNDEFINED
  \edef\resultA{#1}%
  \edef\resultB{#1 }%
%<noetex>  \let\numexpr\SavedNumexpr
  \Expect*{\resultA\space}*{\resultB}%
}
\newcommand*{\TestResult}[2]{%
%<noetex>  \let\SavedNumexpr\numexpr
%<noetex>  \let\numexpr\UNDEFINED
  \edef\result{#1}%
%<noetex>  \let\numexpr\SavedNumexpr
  \Expect*{\result}{#2}%
}
\newcommand*{\TestResultTwoExpansions}[2]{%
%<*noetex>
  \begingroup
    \let\numexpr\UNDEFINED
    \expandafter\expandafter\expandafter
  \endgroup
%</noetex>
  \expandafter\expandafter\expandafter\Expect
  \expandafter\expandafter\expandafter{#1}{#2}%
}
\newcount\TestCount
%<etex>\newcommand*{\TestArg}[1]{\numexpr#1\relax}
%<noetex>\newcommand*{\TestArg}[1]{#1}
\newcommand*{\TestTeXDivide}[2]{%
  \TestCount=\TestArg{#1}\relax
  \divide\TestCount by \TestArg{#2}\relax
  \Expect*{\bigintcalcDiv{#1}{#2}}*{\the\TestCount}%
}
\newcommand*{\Test}[2]{%
  \TestResult{#1}{#2}%
  \TestResultTwoExpansions{#1}{#2}%
  \TestSpaceAtEnd{#1}%
}
\newcommand*{\TestExch}[2]{\Test{#2}{#1}}
\newcommand*{\TestInv}[2]{%
  \Test{\bigintcalcInv{#1}}{#2}%
}
\newcommand*{\TestAbs}[2]{%
  \Test{\bigintcalcAbs{#1}}{#2}%
}
\newcommand*{\TestSgn}[2]{%
  \Test{\bigintcalcSgn{#1}}{#2}%
}
\newcommand*{\TestMin}[3]{%
  \Test{\bigintcalcMin{#1}{#2}}{#3}%
}
\newcommand*{\TestMax}[3]{%
  \Test{\bigintcalcMax{#1}{#2}}{#3}%
}
\newcommand*{\TestCmp}[3]{%
  \Test{\bigintcalcCmp{#1}{#2}}{#3}%
}
\newcommand*{\TestOdd}[2]{%
  \Test{\bigintcalcOdd{#1}}{#2}%
  \edef\x{%
    \noexpand\Test{%
      \noexpand\BigIntCalcOdd
      \bigintcalcAbs{#1}!%
    }{#2}%
  }%
  \x
}
\newcommand*{\TestInc}[2]{%
  \Test{\bigintcalcInc{#1}}{#2}%
  \ifnum\bigintcalcSgn{#1}>-1 %
    \edef\x{%
      \noexpand\Test{%
        \noexpand\BigIntCalcInc\bigintcalcNum{#1}!%
      }{#2}%
    }%
    \x
  \fi
}
\newcommand*{\TestDec}[2]{%
  \Test{\bigintcalcDec{#1}}{#2}%
  \ifnum\bigintcalcSgn{#1}>0 %
    \edef\x{%
      \noexpand\Test{%
        \noexpand\BigIntCalcDec\bigintcalcNum{#1}!%
      }{#2}%
    }%
    \x
  \fi
}
\newcommand*{\TestAdd}[3]{%
  \Test{\bigintcalcAdd{#1}{#2}}{#3}%
  \ifnum\bigintcalcSgn{#1}>0 %
    \ifnum\bigintcalcSgn{#2}> 0 %
      \ifnum\bigintcalcCmp{#1}{#2}>0 %
        \edef\x{%
          \noexpand\Test{%
            \noexpand\BigIntCalcAdd
            \bigintcalcNum{#1}!\bigintcalcNum{#2}!%
          }{#3}%
        }%
        \x
      \else
        \edef\x{%
          \noexpand\Test{%
            \noexpand\BigIntCalcAdd
            \bigintcalcNum{#2}!\bigintcalcNum{#1}!%
          }{#3}%
        }%
        \x
      \fi
    \fi
  \fi
}
\newcommand*{\TestSub}[3]{%
  \Test{\bigintcalcSub{#1}{#2}}{#3}%
  \ifnum\bigintcalcSgn{#1}>0 %
    \ifnum\bigintcalcSgn{#2}> 0 %
      \ifnum\bigintcalcCmp{#1}{#2}>0 %
        \edef\x{%
          \noexpand\Test{%
            \noexpand\BigIntCalcSub
            \bigintcalcNum{#1}!\bigintcalcNum{#2}!%
          }{#3}%
        }%
        \x
      \fi
    \fi
  \fi
}
\newcommand*{\TestShl}[2]{%
  \Test{\bigintcalcShl{#1}}{#2}%
  \edef\x{%
    \noexpand\Test{%
      \noexpand\BigIntCalcShl\bigintcalcAbs{#1}!%
    }{\bigintcalcAbs{#2}}%
  }%
  \x
}
\newcommand*{\TestShr}[2]{%
  \Test{\bigintcalcShr{#1}}{#2}%
  \edef\x{%
    \noexpand\Test{%
      \noexpand\BigIntCalcShr\bigintcalcAbs{#1}!%
    }{\bigintcalcAbs{#2}}%
  }%
  \x
}
\newcommand*{\TestMul}[3]{%
  \Test{\bigintcalcMul{#1}{#2}}{#3}%
  \edef\x{%
    \noexpand\Test{%
      \noexpand\BigIntCalcMul
      \bigintcalcAbs{#1}!\bigintcalcAbs{#2}!%
    }{\bigintcalcAbs{#3}}%
  }%
  \x
}
\newcommand*{\TestSqr}[2]{%
  \Test{\bigintcalcSqr{#1}}{#2}%
}
\newcommand*{\TestFac}[2]{%
  \expandafter\TestExch\expandafter{%
    \the\numexpr#2%
  }{\bigintcalcFac{#1}}%
}
\newcommand*{\TestFacBig}[2]{%
  \Test{\bigintcalcFac{#1}}{#2}%
}
\newcommand*{\TestPow}[3]{%
  \Test{\bigintcalcPow{#1}{#2}}{#3}%
}
\newcommand*{\TestDiv}[3]{%
  \Test{\bigintcalcDiv{#1}{#2}}{#3}%
  \TestTeXDivide{#1}{#2}%
}
\newcommand*{\TestDivBig}[3]{%
  \Test{\bigintcalcDiv{#1}{#2}}{#3}%
  \edef\x{%
    \noexpand\Test{%
      \noexpand\BigIntCalcDiv\bigintcalcAbs{#1}!\bigintcalcAbs{#2}!%
    }{\bigintcalcAbs{#3}}%
  }%
}
\newcommand*{\TestMod}[3]{%
  \Test{\bigintcalcMod{#1}{#2}}{#3}%
  \ifcase\ifcase\bigintcalcSgn{#1} 0%
         \or
           \ifcase\bigintcalcSgn{#2} 1%
           \or 0%
           \else 1%
           \fi
         \else
           \ifcase\bigintcalcSgn{#2} 1%
           \or 1%
           \else 0%
           \fi
         \fi\relax
    \edef\x{%
      \noexpand\Test{%
        \noexpand\BigIntCalcMod
        \bigintcalcAbs{#1}!\bigintcalcAbs{#2}!%
      }{\bigintcalcAbs{#3}}%
    }%
    \x
  \fi
}
%    \end{macrocode}
%
% \subsubsection{Time}
%
%    \begin{macrocode}
\begingroup\expandafter\expandafter\expandafter\endgroup
\expandafter\ifx\csname pdfresettimer\endcsname\relax
\else
  \makeatletter
  \newcount\SummaryTime
  \newcount\TestTime
  \SummaryTime=\z@
  \newcommand*{\PrintTime}[2]{%
    \typeout{%
      [Time #1: \strip@pt\dimexpr\number#2sp\relax\space s]%
    }%
  }%
  \newcommand*{\StartTime}[1]{%
    \renewcommand*{\TimeDescription}{#1}%
    \pdfresettimer
  }%
  \newcommand*{\TimeDescription}{}%
  \newcommand*{\StopTime}{%
    \TestTime=\pdfelapsedtime
    \global\advance\SummaryTime\TestTime
    \PrintTime\TimeDescription\TestTime
  }%
  \let\saved@qstest\qstest
  \let\saved@endqstest\endqstest
  \def\qstest#1#2{%
    \saved@qstest{#1}{#2}%
    \StartTime{#1}%
  }%
  \def\endqstest{%
    \StopTime
    \saved@endqstest
  }%
  \AtEndDocument{%
    \PrintTime{summary}\SummaryTime
  }%
  \makeatother
\fi
%    \end{macrocode}
%
% \subsubsection{Test sets}
%
%    \begin{macrocode}
\makeatletter

\begin{qstest}{inv}{inv}%
  \TestInv{0}{0}%
  \TestInv{1}{-1}%
  \TestInv{-1}{1}%
  \TestInv{10}{-10}%
  \TestInv{-10}{10}%
  \TestInv{2147483647}{-2147483647}%
  \TestInv{-2147483647}{2147483647}%
  \TestInv{12345678901234567890}{-12345678901234567890}%
  \TestInv{-12345678901234567890}{12345678901234567890}%
  \TestInv{ 0 }{0}%
  \TestInv{ 1 }{-1}%
  \TestInv{--1}{-1}%
  \TestInv{\number\z@}{0}%
  \TestInv{\ifx\relax\relax1\fi}{-1}%
  \TestInv{\ifx\relax\relax-\fi\ifx234\else1\fi}{1}%
\end{qstest}

\begin{qstest}{abs}{abs}%
  \TestAbs{0}{0}%
  \TestAbs{1}{1}%
  \TestAbs{-1}{1}%
  \TestAbs{10}{10}%
  \TestAbs{-10}{10}%
  \TestAbs{2147483647}{2147483647}%
  \TestAbs{-2147483647}{2147483647}%
  \TestAbs{12345678901234567890}{12345678901234567890}%
  \TestAbs{-12345678901234567890}{12345678901234567890}%
  \TestAbs{ 0 }{0}%
  \TestAbs{ 1 }{1}%
  \TestAbs{--1}{1}%
  \TestAbs{-+-+1}{1}%
  \TestAbs{00000000000}{0}%
  \TestAbs{00000001000}{1000}%
  \TestAbs{\ifx\relax\relax 0\else 1\fi}{0}%
\end{qstest}

\begin{qstest}{sign}{sign}%
  \TestSgn{0}{0}%
  \TestSgn{1}{1}%
  \TestSgn{-1}{-1}%
  \TestSgn{10}{1}%
  \TestSgn{-10}{-1}%
  \TestSgn{2147483647}{1}%
  \TestSgn{-2147483647}{-1}%
  \TestSgn{12345678901234567890}{1}%
  \TestSgn{-12345678901234567890}{-1}%
  \TestSgn{ 0 }{0}%
  \TestSgn{ 2 }{1}%
  \TestSgn{ -2 }{-1}%
  \TestSgn{--2}{1}%
  \TestSgn{\number\z@}{0}%
  \TestSgn{\number\@ne}{1}%
  \TestSgn{\number\m@ne}{-1}%
  \TestSgn{%
    -+-+\number\z@\number\z@
    \iftrue1\fi\iftrue2\fi\iftrue3\fi
  }{1}%
\end{qstest}

\begin{qstest}{min}{min}%
  \TestMin{0}{1}{0}%
  \TestMin{1}{0}{0}%
  \TestMin{-10}{-20}{-20}%
  \TestMin{ 1 }{ 2 }{1}%
  \TestMin{ 2 }{ 1 }{1}%
  \TestMin{1}{1}{1}%
  \TestMin{\number\z@}{\number\@ne}{0}%
  \TestMin{\number\@ne}{\number\m@ne}{-1}%
\end{qstest}

\begin{qstest}{max}{max}%
  \TestMax{0}{1}{1}%
  \TestMax{1}{0}{1}%
  \TestMax{-10}{-20}{-10}%
  \TestMax{ 1 }{ 2 }{2}%
  \TestMax{ 2 }{ 1 }{2}%
  \TestMax{1}{1}{1}%
  \TestMax{\number\z@}{\number\@ne}{1}%
  \TestMax{\number\@ne}{\number\m@ne}{1}%
\end{qstest}

\begin{qstest}{cmp}{cmp}%
  \TestCmp{0}{0}{0}%
  \TestCmp{-21}{17}{-1}%
  \TestCmp{3}{4}{-1}%
  \TestCmp{-10}{-10}{0}%
  \TestCmp{-10}{-11}{1}%
  \TestCmp{100}{5}{1}%
  \TestCmp{9}{10}{-1}%
  \TestCmp{10}{9}{1}%
  \TestCmp{ 3 }{ 3 }{0}%
  \TestCmp{-9}{-10}{1}%
  \TestCmp{-10}{-9}{-1}%
  \TestCmp{-3}{-3}{0}%
  \TestCmp{0}{-2}{1}%
  \TestCmp{0}{2}{-1}%
  \TestCmp{2}{0}{1}%
  \TestCmp{-2}{0}{-1}%
  \TestCmp{12}{11}{1}%
  \TestCmp{11}{12}{-1}%
  \TestCmp{2147483647}{-2147483647}{1}%
  \TestCmp{-2147483647}{2147483647}{-1}%
  \TestCmp{2147483647}{2147483647}{0}%
  \TestCmp{\number\z@}{\number\@ne}{-1}%
  \TestCmp{\number\@ne}{\number\m@ne}{1}%
  \TestCmp{ 4 }{ 5 }{-1}%
  \TestCmp{ -3 }{ -7 }{1}%
\end{qstest}

\begin{qstest}{odd}{odd}
\tracingmacros=1
  \TestOdd{0}{0}%
  \TestOdd{1}{1}%
  \TestOdd{2}{0}%
  \TestOdd{3}{1}%
  \TestOdd{14}{0}%
  \TestOdd{15}{1}%
  \TestOdd{12345678901234567896}{0}%
  \TestOdd{12345678901234567897}{1}%
\end{qstest}

\begin{qstest}{inc}{inc}%
  \TestInc{0}{1}%
  \TestInc{1}{2}%
  \TestInc{-1}{0}%
  \TestInc{10}{11}%
  \TestInc{-10}{-9}%
  \TestInc{ 3 }{4}%
  \TestInc{999}{1000}%
  \TestInc{-1000}{-999}%
  \TestInc{129}{130}%
  \TestInc{2147483646}{2147483647}%
  \TestInc{-2147483647}{-2147483646}%
  \TestInc{12345678901234567890}{12345678901234567891}%
  \TestInc{99999999999999999999}{100000000000000000000}%
  \TestInc{-12345678901234567891}{-12345678901234567890}%
  \TestInc{-100000000000000000000}{-99999999999999999999}%
\end{qstest}

\begin{qstest}{dec}{dec}%
  \TestDec{0}{-1}%
  \TestDec{1}{0}%
  \TestDec{-1}{-2}%
  \TestDec{10}{9}%
  \TestDec{-10}{-11}%
  \TestDec{1000}{999}%
  \TestDec{-999}{-1000}%
  \TestDec{130}{129}%
  \TestDec{2147483647}{2147483646}%
  \TestDec{-2147483646}{-2147483647}%
  \TestDec{12345678901234567891}{12345678901234567890}%
  \TestDec{100000000000000000000}{99999999999999999999}%
  \TestDec{-12345678901234567890}{-12345678901234567891}%
  \TestDec{-99999999999999999999}{-100000000000000000000}%
\end{qstest}

\begin{qstest}{add}{add}%
  \TestAdd{0}{0}{0}%
  \TestAdd{1}{0}{1}%
  \TestAdd{0}{1}{1}%
  \TestAdd{1}{2}{3}%
  \TestAdd{-1}{-1}{-2}%
  \TestAdd{2147483646}{1}{2147483647}%
  \TestAdd{-2147483647}{2147483647}{0}%
  \TestAdd{20}{-5}{15}%
  \TestAdd{-4}{-1}{-5}%
  \TestAdd{-1}{-4}{-5}%
  \TestAdd{-4}{1}{-3}%
  \TestAdd{-1}{4}{3}%
  \TestAdd{4}{-1}{3}%
  \TestAdd{1}{-4}{-3}%
  \TestAdd{-4}{-1}{-5}%
  \TestAdd{-1}{-4}{-5}%
  \TestAdd{ -4 }{ -1 }{-5}%
  \TestAdd{ -1 }{ -4 }{-5}%
  \TestAdd{ -4 }{ 1 }{-3}%
  \TestAdd{ -1 }{ 4 }{3}%
  \TestAdd{ 4 }{ -1 }{3}%
  \TestAdd{ 1 }{ -4 }{-3}%
  \TestAdd{ -4 }{ -1 }{-5}%
  \TestAdd{ -1 }{ -4 }{-5}%
  \TestAdd{876543210}{111111111}{987654321}%
  \TestAdd{999999999}{2}{1000000001}%
\end{qstest}

\begin{qstest}{sub}{sub}
  \TestSub{0}{0}{0}%
  \TestSub{1}{0}{1}%
  \TestSub{1}{2}{-1}%
  \TestSub{-1}{-1}{0}%
  \TestSub{2147483646}{-1}{2147483647}%
  \TestSub{-2147483647}{-2147483647}{0}%
  \TestSub{-4}{-1}{-3}%
  \TestSub{-1}{-4}{3}%
  \TestSub{-4}{1}{-5}%
  \TestSub{-1}{4}{-5}%
  \TestSub{4}{-1}{5}%
  \TestSub{1}{-4}{5}%
  \TestSub{-4}{-1}{-3}%
  \TestSub{-1}{-4}{3}%
  \TestSub{ -4 }{ -1 }{-3}%
  \TestSub{ -1 }{ -4 }{3}%
  \TestSub{ -4 }{ 1 }{-5}%
  \TestSub{ -1 }{ 4 }{-5}%
  \TestSub{ 4 }{ -1 }{5}%
  \TestSub{ 1 }{ -4 }{5}%
  \TestSub{ -4 }{ -1 }{-3}%
  \TestSub{ -1 }{ -4 }{3}%
  \TestSub{1000000000}{2}{999999998}%
  \TestSub{987654321}{111111111}{876543210}%
\end{qstest}

\begin{qstest}{shl}{shl}
  \TestShl{0}{0}%
  \TestShl{1}{2}%
  \TestShl{2}{4}%
  \TestShl{5621}{11242}%
  \TestShl{1073741823}{2147483646}%
\end{qstest}

\begin{qstest}{shr}{shr}
  \TestShr{0}{0}%
  \TestShr{1}{0}%
  \TestShr{2}{1}%
  \TestShr{3}{1}%
  \TestShr{4}{2}%
  \TestShr{5}{2}%
  \TestShr{6}{3}%
  \TestShr{7}{3}%
  \TestShr{8}{4}%
  \TestShr{9}{4}%
  \TestShr{10}{5}%
  \TestShr{11}{5}%
  \TestShr{12}{6}%
  \TestShr{13}{6}%
  \TestShr{14}{7}%
  \TestShr{15}{7}%
  \TestShr{16}{8}%
  \TestShr{17}{8}%
  \TestShr{18}{9}%
  \TestShr{19}{9}%
  \TestShr{20}{10}%
  \TestShr{21}{10}%
  \TestShr{22}{11}%
  \TestShr{11241}{5620}%
  \TestShr{73054202}{36527101}%
  \TestShr{2147483646}{1073741823}%
\end{qstest}

\begin{qstest}{mul}{mul}
  \TestMul{0}{0}{0}%
  \TestMul{1}{0}{0}%
  \TestMul{0}{1}{0}%
  \TestMul{1}{1}{1}%
  \TestMul{3}{1}{3}%
  \TestMul{1}{-3}{-3}%
  \TestMul{-4}{-5}{20}%
  \TestMul{3}{7}{21}%
  \TestMul{7}{3}{21}%
  \TestMul{3}{-7}{-21}%
  \TestMul{7}{-3}{-21}%
  \TestMul{-3}{7}{-21}%
  \TestMul{-7}{3}{-21}%
  \TestMul{-3}{-7}{21}%
  \TestMul{-7}{-3}{21}%
  \TestMul{12}{11}{132}%
  \TestMul{999}{333}{332667}%
  \TestMul{1000}{4321}{4321000}%
  \TestMul{12345}{173955}{2147474475}%
  \TestMul{1073741823}{2}{2147483646}%
  \TestMul{2}{1073741823}{2147483646}%
  \TestMul{-1073741823}{2}{-2147483646}%
  \TestMul{2}{-1073741823}{-2147483646}%
  \TestMul{6706022400}{13}{87178291200}%
\end{qstest}

\begin{qstest}{sqr}{sqr}
  \TestSqr{0}{0}%
  \TestSqr{1}{1}%
  \TestSqr{2}{4}%
  \TestSqr{3}{9}%
  \TestSqr{4}{16}%
  \TestSqr{9}{81}%
  \TestSqr{10}{100}%
  \TestSqr{46340}{2147395600}%
  \TestSqr{-1}{1}%
  \TestSqr{-2}{4}%
  \TestSqr{-46340}{2147395600}%
\end{qstest}

\begin{qstest}{fac}{fac}
  \TestFac{0}{1}%
  \TestFac{1}{1}%
  \TestFac{2}{2}%
  \TestFac{3}{2*3}%
  \TestFac{4}{2*3*4}%
  \TestFac{5}{2*3*4*5}%
  \TestFac{6}{2*3*4*5*6}%
  \TestFac{7}{2*3*4*5*6*7}%
  \TestFac{8}{2*3*4*5*6*7*8}%
  \TestFac{9}{2*3*4*5*6*7*8*9}%
  \TestFac{10}{2*3*4*5*6*7*8*9*10}%
  \TestFac{11}{2*3*4*5*6*7*8*9*10*11}%
  \TestFac{12}{2*3*4*5*6*7*8*9*10*11*12}%
  \TestFacBig{13}{6227020800}%
  \TestFacBig{14}{87178291200}%
  \TestFacBig{15}{1307674368000}%
  \TestFacBig{16}{20922789888000}%
  \TestFacBig{17}{355687428096000}%
  \TestFacBig{18}{6402373705728000}%
  \TestFacBig{19}{121645100408832000}%
  \TestFacBig{20}{2432902008176640000}%
  \TestFacBig{21}{51090942171709440000}%
  \TestFacBig{22}{1124000727777607680000}%
\end{qstest}

\begin{qstest}{pow}{pow}
  \TestPow{-2}{0}{1}%
  \TestPow{-1}{0}{1}%
  \TestPow{0}{0}{1}%
  \TestPow{1}{0}{1}%
  \TestPow{2}{0}{1}%
  \TestPow{3}{0}{1}%
  \TestPow{-2}{1}{-2}%
  \TestPow{-1}{1}{-1}%
  \TestPow{1}{1}{1}%
  \TestPow{2}{1}{2}%
  \TestPow{3}{1}{3}%
  \TestPow{-2}{2}{4}%
  \TestPow{-1}{2}{1}%
  \TestPow{0}{2}{0}%
  \TestPow{1}{2}{1}%
  \TestPow{2}{2}{4}%
  \TestPow{3}{2}{9}%
  \TestPow{0}{1}{0}%
  \TestPow{1}{-2}{1}%
  \TestPow{1}{-1}{1}%
  \TestPow{-1}{-2}{1}%
  \TestPow{-1}{-1}{-1}%
  \TestPow{-1}{3}{-1}%
  \TestPow{-1}{4}{1}%
  \TestPow{-2}{-1}{0}%
  \TestPow{-2}{-2}{0}%
  \TestPow{2}{3}{8}%
  \TestPow{2}{4}{16}%
  \TestPow{2}{5}{32}%
  \TestPow{2}{6}{64}%
  \TestPow{2}{7}{128}%
  \TestPow{2}{8}{256}%
  \TestPow{2}{9}{512}%
  \TestPow{2}{10}{1024}%
  \TestPow{-2}{3}{-8}%
  \TestPow{-2}{4}{16}%
  \TestPow{-2}{5}{-32}%
  \TestPow{-2}{6}{64}%
  \TestPow{-2}{7}{-128}%
  \TestPow{-2}{8}{256}%
  \TestPow{-2}{9}{-512}%
  \TestPow{-2}{10}{1024}%
  \TestPow{3}{3}{27}%
  \TestPow{3}{4}{81}%
  \TestPow{3}{5}{243}%
  \TestPow{-3}{3}{-27}%
  \TestPow{-3}{4}{81}%
  \TestPow{-3}{5}{-243}%
  \TestPow{2}{30}{1073741824}%
  \TestPow{-3}{19}{-1162261467}%
  \TestPow{5}{13}{1220703125}%
  \TestPow{-7}{11}{-1977326743}%
\end{qstest}

\begin{qstest}{div}{div}
  \TestDiv{1}{1}{1}%
  \TestDiv{2}{1}{2}%
  \TestDiv{-2}{1}{-2}%
  \TestDiv{2}{-1}{-2}%
  \TestDiv{-2}{-1}{2}%
  \TestDiv{15}{2}{7}%
  \TestDiv{-16}{2}{-8}%
  \TestDiv{1}{2}{0}%
  \TestDiv{1}{3}{0}%
  \TestDiv{2}{3}{0}%
  \TestDiv{-2}{3}{0}%
  \TestDiv{2}{-3}{0}%
  \TestDiv{-2}{-3}{0}%
  \TestDiv{13}{3}{4}%
  \TestDiv{-13}{-3}{4}%
  \TestDiv{-13}{3}{-4}%
  \TestDiv{-6}{5}{-1}%
  \TestDiv{-5}{5}{-1}%
  \TestDiv{-4}{5}{0}%
  \TestDiv{-3}{5}{0}%
  \TestDiv{-2}{5}{0}%
  \TestDiv{-1}{5}{0}%
  \TestDiv{0}{5}{0}%
  \TestDiv{1}{5}{0}%
  \TestDiv{2}{5}{0}%
  \TestDiv{3}{5}{0}%
  \TestDiv{4}{5}{0}%
  \TestDiv{5}{5}{1}%
  \TestDiv{6}{5}{1}%
  \TestDiv{-5}{4}{-1}%
  \TestDiv{-4}{4}{-1}%
  \TestDiv{-3}{4}{0}%
  \TestDiv{-2}{4}{0}%
  \TestDiv{-1}{4}{0}%
  \TestDiv{0}{4}{0}%
  \TestDiv{1}{4}{0}%
  \TestDiv{2}{4}{0}%
  \TestDiv{3}{4}{0}%
  \TestDiv{4}{4}{1}%
  \TestDiv{5}{4}{1}%
  \TestDiv{12345}{678}{18}%
  \TestDiv{32372}{5952}{5}%
  \TestDiv{284271294}{18162}{15651}%
  \TestDiv{217652429}{12561}{17327}%
  \TestDiv{462028434}{5439}{84947}%
  \TestDiv{2147483647}{1000}{2147483}%
  \TestDiv{2147483647}{-1000}{-2147483}%
  \TestDiv{-2147483647}{1000}{-2147483}%
  \TestDiv{-2147483647}{-1000}{2147483}%
  \TestDiv{0}{3}{0}%
  \TestDiv{1}{3}{0}%
  \TestDiv{2}{3}{0}%
  \TestDiv{3}{3}{1}%
  \TestDiv{4}{3}{1}%
  \TestDiv{5}{3}{1}%
  \TestDiv{6}{3}{2}%
  \TestDiv{7}{3}{2}%
  \TestDiv{8}{3}{2}%
  \TestDiv{9}{3}{3}%
  \TestDiv{10}{3}{3}%
  \TestDiv{11}{3}{3}%
  \TestDiv{12}{3}{4}%
  \TestDiv{13}{3}{4}%
  \TestDiv{14}{3}{4}%
  \TestDiv{15}{3}{5}%
  \TestDiv{16}{3}{5}%
  \TestDiv{17}{3}{5}%
  \TestDiv{18}{3}{6}%
  \TestDiv{19}{3}{6}%
  \TestDiv{20}{3}{6}%
  \TestDiv{21}{3}{7}%
  \TestDiv{22}{3}{7}%
  \TestDiv{23}{3}{7}%
  \TestDiv{24}{3}{8}%
  \TestDiv{25}{3}{8}%
  \TestDiv{26}{3}{8}%
  \TestDiv{27}{3}{9}%
  \TestDiv{28}{3}{9}%
  \TestDiv{29}{3}{9}%
  \TestDiv{30}{3}{10}%
  \TestDiv{31}{3}{10}%
  \TestDivBig{17363436332507}{24702}{702916214}%
\end{qstest}

\begin{qstest}{mod}{mod}
  \TestMod{-6}{5}{4}%
  \TestMod{-5}{5}{0}%
  \TestMod{-4}{5}{1}%
  \TestMod{-3}{5}{2}%
  \TestMod{-2}{5}{3}%
  \TestMod{-1}{5}{4}%
  \TestMod{0}{5}{0}%
  \TestMod{1}{5}{1}%
  \TestMod{2}{5}{2}%
  \TestMod{3}{5}{3}%
  \TestMod{4}{5}{4}%
  \TestMod{5}{5}{0}%
  \TestMod{6}{5}{1}%
  \TestMod{-5}{4}{3}%
  \TestMod{-4}{4}{0}%
  \TestMod{-3}{4}{1}%
  \TestMod{-2}{4}{2}%
  \TestMod{-1}{4}{3}%
  \TestMod{0}{4}{0}%
  \TestMod{1}{4}{1}%
  \TestMod{2}{4}{2}%
  \TestMod{3}{4}{3}%
  \TestMod{4}{4}{0}%
  \TestMod{5}{4}{1}%
  \TestMod{-6}{-5}{-1}%
  \TestMod{-5}{-5}{0}%
  \TestMod{-4}{-5}{-4}%
  \TestMod{-3}{-5}{-3}%
  \TestMod{-2}{-5}{-2}%
  \TestMod{-1}{-5}{-1}%
  \TestMod{0}{-5}{0}%
  \TestMod{1}{-5}{-4}%
  \TestMod{2}{-5}{-3}%
  \TestMod{3}{-5}{-2}%
  \TestMod{4}{-5}{-1}%
  \TestMod{5}{-5}{0}%
  \TestMod{6}{-5}{-4}%
  \TestMod{-5}{-4}{-1}%
  \TestMod{-4}{-4}{0}%
  \TestMod{-3}{-4}{-3}%
  \TestMod{-2}{-4}{-2}%
  \TestMod{-1}{-4}{-1}%
  \TestMod{0}{-4}{0}%
  \TestMod{1}{-4}{-3}%
  \TestMod{2}{-4}{-2}%
  \TestMod{3}{-4}{-1}%
  \TestMod{4}{-4}{0}%
  \TestMod{5}{-4}{-3}%
  \TestMod{2147483647}{1000}{647}%
  \TestMod{2147483647}{-1000}{-353}%
  \TestMod{-2147483647}{1000}{353}%
  \TestMod{-2147483647}{-1000}{-647}%
  \TestMod{ 0 }{ 4 }{0}%
  \TestMod{ 1 }{ 4 }{1}%
  \TestMod{ -1 }{ 4 }{3}%
  \TestMod{ 0 }{ -4 }{0}%
  \TestMod{ 1 }{ -4 }{-3}%
  \TestMod{ -1 }{ -4 }{-1}%
  \TestMod{18362}{25}{12}%
\end{qstest}

\newcommand*{\TestError}[2]{%
  \begingroup
    \expandafter\def\csname BigIntCalcError:#1\endcsname{}%
    \Expect*{#2}{0}%
    \expandafter\def\csname BigIntCalcError:#1\endcsname{ERROR}%
    \Expect*{#2}{0ERROR}%
  \endgroup
}
\begin{qstest}{error}{error}
  \TestError{FacNegative}{\bigintcalcFac{-1}}%
  \TestError{FacNegative}{\bigintcalcFac{-2147483647}}%
  \TestError{DivisionByZero}{\bigintcalcPow{0}{-1}}%
  \TestError{DivisionByZero}{\bigintcalcDiv{1}{0}}%
  \TestError{DivisionByZero}{\bigintcalcMod{1}{0}}%
\end{qstest}

\begin{document}
\end{document}
%</test2>
%    \end{macrocode}
%
% \section{Installation}
%
% \subsection{Download}
%
% \paragraph{Package.} This package is available on
% CTAN\footnote{\url{http://ctan.org/pkg/bigintcalc}}:
% \begin{description}
% \item[\CTAN{macros/latex/contrib/oberdiek/bigintcalc.dtx}] The source file.
% \item[\CTAN{macros/latex/contrib/oberdiek/bigintcalc.pdf}] Documentation.
% \end{description}
%
%
% \paragraph{Bundle.} All the packages of the bundle `oberdiek'
% are also available in a TDS compliant ZIP archive. There
% the packages are already unpacked and the documentation files
% are generated. The files and directories obey the TDS standard.
% \begin{description}
% \item[\CTAN{install/macros/latex/contrib/oberdiek.tds.zip}]
% \end{description}
% \emph{TDS} refers to the standard ``A Directory Structure
% for \TeX\ Files'' (\CTAN{tds/tds.pdf}). Directories
% with \xfile{texmf} in their name are usually organized this way.
%
% \subsection{Bundle installation}
%
% \paragraph{Unpacking.} Unpack the \xfile{oberdiek.tds.zip} in the
% TDS tree (also known as \xfile{texmf} tree) of your choice.
% Example (linux):
% \begin{quote}
%   |unzip oberdiek.tds.zip -d ~/texmf|
% \end{quote}
%
% \paragraph{Script installation.}
% Check the directory \xfile{TDS:scripts/oberdiek/} for
% scripts that need further installation steps.
% Package \xpackage{attachfile2} comes with the Perl script
% \xfile{pdfatfi.pl} that should be installed in such a way
% that it can be called as \texttt{pdfatfi}.
% Example (linux):
% \begin{quote}
%   |chmod +x scripts/oberdiek/pdfatfi.pl|\\
%   |cp scripts/oberdiek/pdfatfi.pl /usr/local/bin/|
% \end{quote}
%
% \subsection{Package installation}
%
% \paragraph{Unpacking.} The \xfile{.dtx} file is a self-extracting
% \docstrip\ archive. The files are extracted by running the
% \xfile{.dtx} through \plainTeX:
% \begin{quote}
%   \verb|tex bigintcalc.dtx|
% \end{quote}
%
% \paragraph{TDS.} Now the different files must be moved into
% the different directories in your installation TDS tree
% (also known as \xfile{texmf} tree):
% \begin{quote}
% \def\t{^^A
% \begin{tabular}{@{}>{\ttfamily}l@{ $\rightarrow$ }>{\ttfamily}l@{}}
%   bigintcalc.sty & tex/generic/oberdiek/bigintcalc.sty\\
%   bigintcalc.pdf & doc/latex/oberdiek/bigintcalc.pdf\\
%   test/bigintcalc-test1.tex & doc/latex/oberdiek/test/bigintcalc-test1.tex\\
%   test/bigintcalc-test2.tex & doc/latex/oberdiek/test/bigintcalc-test2.tex\\
%   test/bigintcalc-test3.tex & doc/latex/oberdiek/test/bigintcalc-test3.tex\\
%   bigintcalc.dtx & source/latex/oberdiek/bigintcalc.dtx\\
% \end{tabular}^^A
% }^^A
% \sbox0{\t}^^A
% \ifdim\wd0>\linewidth
%   \begingroup
%     \advance\linewidth by\leftmargin
%     \advance\linewidth by\rightmargin
%   \edef\x{\endgroup
%     \def\noexpand\lw{\the\linewidth}^^A
%   }\x
%   \def\lwbox{^^A
%     \leavevmode
%     \hbox to \linewidth{^^A
%       \kern-\leftmargin\relax
%       \hss
%       \usebox0
%       \hss
%       \kern-\rightmargin\relax
%     }^^A
%   }^^A
%   \ifdim\wd0>\lw
%     \sbox0{\small\t}^^A
%     \ifdim\wd0>\linewidth
%       \ifdim\wd0>\lw
%         \sbox0{\footnotesize\t}^^A
%         \ifdim\wd0>\linewidth
%           \ifdim\wd0>\lw
%             \sbox0{\scriptsize\t}^^A
%             \ifdim\wd0>\linewidth
%               \ifdim\wd0>\lw
%                 \sbox0{\tiny\t}^^A
%                 \ifdim\wd0>\linewidth
%                   \lwbox
%                 \else
%                   \usebox0
%                 \fi
%               \else
%                 \lwbox
%               \fi
%             \else
%               \usebox0
%             \fi
%           \else
%             \lwbox
%           \fi
%         \else
%           \usebox0
%         \fi
%       \else
%         \lwbox
%       \fi
%     \else
%       \usebox0
%     \fi
%   \else
%     \lwbox
%   \fi
% \else
%   \usebox0
% \fi
% \end{quote}
% If you have a \xfile{docstrip.cfg} that configures and enables \docstrip's
% TDS installing feature, then some files can already be in the right
% place, see the documentation of \docstrip.
%
% \subsection{Refresh file name databases}
%
% If your \TeX~distribution
% (\teTeX, \mikTeX, \dots) relies on file name databases, you must refresh
% these. For example, \teTeX\ users run \verb|texhash| or
% \verb|mktexlsr|.
%
% \subsection{Some details for the interested}
%
% \paragraph{Attached source.}
%
% The PDF documentation on CTAN also includes the
% \xfile{.dtx} source file. It can be extracted by
% AcrobatReader 6 or higher. Another option is \textsf{pdftk},
% e.g. unpack the file into the current directory:
% \begin{quote}
%   \verb|pdftk bigintcalc.pdf unpack_files output .|
% \end{quote}
%
% \paragraph{Unpacking with \LaTeX.}
% The \xfile{.dtx} chooses its action depending on the format:
% \begin{description}
% \item[\plainTeX:] Run \docstrip\ and extract the files.
% \item[\LaTeX:] Generate the documentation.
% \end{description}
% If you insist on using \LaTeX\ for \docstrip\ (really,
% \docstrip\ does not need \LaTeX), then inform the autodetect routine
% about your intention:
% \begin{quote}
%   \verb|latex \let\install=y\input{bigintcalc.dtx}|
% \end{quote}
% Do not forget to quote the argument according to the demands
% of your shell.
%
% \paragraph{Generating the documentation.}
% You can use both the \xfile{.dtx} or the \xfile{.drv} to generate
% the documentation. The process can be configured by the
% configuration file \xfile{ltxdoc.cfg}. For instance, put this
% line into this file, if you want to have A4 as paper format:
% \begin{quote}
%   \verb|\PassOptionsToClass{a4paper}{article}|
% \end{quote}
% An example follows how to generate the
% documentation with pdf\LaTeX:
% \begin{quote}
%\begin{verbatim}
%pdflatex bigintcalc.dtx
%makeindex -s gind.ist bigintcalc.idx
%pdflatex bigintcalc.dtx
%makeindex -s gind.ist bigintcalc.idx
%pdflatex bigintcalc.dtx
%\end{verbatim}
% \end{quote}
%
% \section{Catalogue}
%
% The following XML file can be used as source for the
% \href{http://mirror.ctan.org/help/Catalogue/catalogue.html}{\TeX\ Catalogue}.
% The elements \texttt{caption} and \texttt{description} are imported
% from the original XML file from the Catalogue.
% The name of the XML file in the Catalogue is \xfile{bigintcalc.xml}.
%    \begin{macrocode}
%<*catalogue>
<?xml version='1.0' encoding='us-ascii'?>
<!DOCTYPE entry SYSTEM 'catalogue.dtd'>
<entry datestamp='$Date$' modifier='$Author$' id='bigintcalc'>
  <name>bigintcalc</name>
  <caption>Integer calculations on very large numbers.</caption>
  <authorref id='auth:oberdiek'/>
  <copyright owner='Heiko Oberdiek' year='2007,2011,2012'/>
  <license type='lppl1.3'/>
  <version number='1.4'/>
  <description>
    This package provides expandable arithmetic operations
    with big integers that can exceed TeX's number limits.
    <p/>
    The package is part of the <xref refid='oberdiek'>oberdiek</xref> bundle.
  </description>
  <documentation details='Package documentation'
      href='ctan:/macros/latex/contrib/oberdiek/bigintcalc.pdf'/>
  <ctan file='true' path='/macros/latex/contrib/oberdiek/bigintcalc.dtx'/>
  <miktex location='oberdiek'/>
  <texlive location='oberdiek'/>
  <install path='/macros/latex/contrib/oberdiek/oberdiek.tds.zip'/>
</entry>
%</catalogue>
%    \end{macrocode}
%
% \begin{History}
%   \begin{Version}{2007/09/27 v1.0}
%   \item
%     First version.
%   \end{Version}
%   \begin{Version}{2007/11/11 v1.1}
%   \item
%     Use of package \xpackage{pdftexcmds} for \LuaTeX\ support.
%   \end{Version}
%   \begin{Version}{2011/01/30 v1.2}
%   \item
%     Already loaded package files are not input in \hologo{plainTeX}.
%   \end{Version}
%   \begin{Version}{2012/04/08 v1.3}
%   \item
%     Fix: \xpackage{pdftexcmds} wasn't loaded in case of \hologo{LaTeX}.
%   \end{Version}
%   \begin{Version}{2016/05/16 v1.4}
%   \item
%     Documentation updates.
%   \end{Version}
% \end{History}
%
% \PrintIndex
%
% \Finale
\endinput
|
% \end{quote}
% Do not forget to quote the argument according to the demands
% of your shell.
%
% \paragraph{Generating the documentation.}
% You can use both the \xfile{.dtx} or the \xfile{.drv} to generate
% the documentation. The process can be configured by the
% configuration file \xfile{ltxdoc.cfg}. For instance, put this
% line into this file, if you want to have A4 as paper format:
% \begin{quote}
%   \verb|\PassOptionsToClass{a4paper}{article}|
% \end{quote}
% An example follows how to generate the
% documentation with pdf\LaTeX:
% \begin{quote}
%\begin{verbatim}
%pdflatex bigintcalc.dtx
%makeindex -s gind.ist bigintcalc.idx
%pdflatex bigintcalc.dtx
%makeindex -s gind.ist bigintcalc.idx
%pdflatex bigintcalc.dtx
%\end{verbatim}
% \end{quote}
%
% \section{Catalogue}
%
% The following XML file can be used as source for the
% \href{http://mirror.ctan.org/help/Catalogue/catalogue.html}{\TeX\ Catalogue}.
% The elements \texttt{caption} and \texttt{description} are imported
% from the original XML file from the Catalogue.
% The name of the XML file in the Catalogue is \xfile{bigintcalc.xml}.
%    \begin{macrocode}
%<*catalogue>
<?xml version='1.0' encoding='us-ascii'?>
<!DOCTYPE entry SYSTEM 'catalogue.dtd'>
<entry datestamp='$Date$' modifier='$Author$' id='bigintcalc'>
  <name>bigintcalc</name>
  <caption>Integer calculations on very large numbers.</caption>
  <authorref id='auth:oberdiek'/>
  <copyright owner='Heiko Oberdiek' year='2007,2011,2012'/>
  <license type='lppl1.3'/>
  <version number='1.4'/>
  <description>
    This package provides expandable arithmetic operations
    with big integers that can exceed TeX's number limits.
    <p/>
    The package is part of the <xref refid='oberdiek'>oberdiek</xref> bundle.
  </description>
  <documentation details='Package documentation'
      href='ctan:/macros/latex/contrib/oberdiek/bigintcalc.pdf'/>
  <ctan file='true' path='/macros/latex/contrib/oberdiek/bigintcalc.dtx'/>
  <miktex location='oberdiek'/>
  <texlive location='oberdiek'/>
  <install path='/macros/latex/contrib/oberdiek/oberdiek.tds.zip'/>
</entry>
%</catalogue>
%    \end{macrocode}
%
% \begin{History}
%   \begin{Version}{2007/09/27 v1.0}
%   \item
%     First version.
%   \end{Version}
%   \begin{Version}{2007/11/11 v1.1}
%   \item
%     Use of package \xpackage{pdftexcmds} for \LuaTeX\ support.
%   \end{Version}
%   \begin{Version}{2011/01/30 v1.2}
%   \item
%     Already loaded package files are not input in \hologo{plainTeX}.
%   \end{Version}
%   \begin{Version}{2012/04/08 v1.3}
%   \item
%     Fix: \xpackage{pdftexcmds} wasn't loaded in case of \hologo{LaTeX}.
%   \end{Version}
%   \begin{Version}{2016/05/16 v1.4}
%   \item
%     Documentation updates.
%   \end{Version}
% \end{History}
%
% \PrintIndex
%
% \Finale
\endinput
|
% \end{quote}
% Do not forget to quote the argument according to the demands
% of your shell.
%
% \paragraph{Generating the documentation.}
% You can use both the \xfile{.dtx} or the \xfile{.drv} to generate
% the documentation. The process can be configured by the
% configuration file \xfile{ltxdoc.cfg}. For instance, put this
% line into this file, if you want to have A4 as paper format:
% \begin{quote}
%   \verb|\PassOptionsToClass{a4paper}{article}|
% \end{quote}
% An example follows how to generate the
% documentation with pdf\LaTeX:
% \begin{quote}
%\begin{verbatim}
%pdflatex bigintcalc.dtx
%makeindex -s gind.ist bigintcalc.idx
%pdflatex bigintcalc.dtx
%makeindex -s gind.ist bigintcalc.idx
%pdflatex bigintcalc.dtx
%\end{verbatim}
% \end{quote}
%
% \section{Catalogue}
%
% The following XML file can be used as source for the
% \href{http://mirror.ctan.org/help/Catalogue/catalogue.html}{\TeX\ Catalogue}.
% The elements \texttt{caption} and \texttt{description} are imported
% from the original XML file from the Catalogue.
% The name of the XML file in the Catalogue is \xfile{bigintcalc.xml}.
%    \begin{macrocode}
%<*catalogue>
<?xml version='1.0' encoding='us-ascii'?>
<!DOCTYPE entry SYSTEM 'catalogue.dtd'>
<entry datestamp='$Date$' modifier='$Author$' id='bigintcalc'>
  <name>bigintcalc</name>
  <caption>Integer calculations on very large numbers.</caption>
  <authorref id='auth:oberdiek'/>
  <copyright owner='Heiko Oberdiek' year='2007,2011,2012'/>
  <license type='lppl1.3'/>
  <version number='1.4'/>
  <description>
    This package provides expandable arithmetic operations
    with big integers that can exceed TeX's number limits.
    <p/>
    The package is part of the <xref refid='oberdiek'>oberdiek</xref> bundle.
  </description>
  <documentation details='Package documentation'
      href='ctan:/macros/latex/contrib/oberdiek/bigintcalc.pdf'/>
  <ctan file='true' path='/macros/latex/contrib/oberdiek/bigintcalc.dtx'/>
  <miktex location='oberdiek'/>
  <texlive location='oberdiek'/>
  <install path='/macros/latex/contrib/oberdiek/oberdiek.tds.zip'/>
</entry>
%</catalogue>
%    \end{macrocode}
%
% \begin{History}
%   \begin{Version}{2007/09/27 v1.0}
%   \item
%     First version.
%   \end{Version}
%   \begin{Version}{2007/11/11 v1.1}
%   \item
%     Use of package \xpackage{pdftexcmds} for \LuaTeX\ support.
%   \end{Version}
%   \begin{Version}{2011/01/30 v1.2}
%   \item
%     Already loaded package files are not input in \hologo{plainTeX}.
%   \end{Version}
%   \begin{Version}{2012/04/08 v1.3}
%   \item
%     Fix: \xpackage{pdftexcmds} wasn't loaded in case of \hologo{LaTeX}.
%   \end{Version}
%   \begin{Version}{2016/05/16 v1.4}
%   \item
%     Documentation updates.
%   \end{Version}
% \end{History}
%
% \PrintIndex
%
% \Finale
\endinput
|
% \end{quote}
% Do not forget to quote the argument according to the demands
% of your shell.
%
% \paragraph{Generating the documentation.}
% You can use both the \xfile{.dtx} or the \xfile{.drv} to generate
% the documentation. The process can be configured by the
% configuration file \xfile{ltxdoc.cfg}. For instance, put this
% line into this file, if you want to have A4 as paper format:
% \begin{quote}
%   \verb|\PassOptionsToClass{a4paper}{article}|
% \end{quote}
% An example follows how to generate the
% documentation with pdf\LaTeX:
% \begin{quote}
%\begin{verbatim}
%pdflatex bigintcalc.dtx
%makeindex -s gind.ist bigintcalc.idx
%pdflatex bigintcalc.dtx
%makeindex -s gind.ist bigintcalc.idx
%pdflatex bigintcalc.dtx
%\end{verbatim}
% \end{quote}
%
% \section{Catalogue}
%
% The following XML file can be used as source for the
% \href{http://mirror.ctan.org/help/Catalogue/catalogue.html}{\TeX\ Catalogue}.
% The elements \texttt{caption} and \texttt{description} are imported
% from the original XML file from the Catalogue.
% The name of the XML file in the Catalogue is \xfile{bigintcalc.xml}.
%    \begin{macrocode}
%<*catalogue>
<?xml version='1.0' encoding='us-ascii'?>
<!DOCTYPE entry SYSTEM 'catalogue.dtd'>
<entry datestamp='$Date$' modifier='$Author$' id='bigintcalc'>
  <name>bigintcalc</name>
  <caption>Integer calculations on very large numbers.</caption>
  <authorref id='auth:oberdiek'/>
  <copyright owner='Heiko Oberdiek' year='2007,2011,2012'/>
  <license type='lppl1.3'/>
  <version number='1.3'/>
  <description>
    This package provides expandable arithmetic operations
    with big integers that can exceed TeX's number limits.
    <p/>
    The package is part of the <xref refid='oberdiek'>oberdiek</xref> bundle.
  </description>
  <documentation details='Package documentation'
      href='ctan:/macros/latex/contrib/oberdiek/bigintcalc.pdf'/>
  <ctan file='true' path='/macros/latex/contrib/oberdiek/bigintcalc.dtx'/>
  <miktex location='oberdiek'/>
  <texlive location='oberdiek'/>
  <install path='/macros/latex/contrib/oberdiek/oberdiek.tds.zip'/>
</entry>
%</catalogue>
%    \end{macrocode}
%
% \begin{History}
%   \begin{Version}{2007/09/27 v1.0}
%   \item
%     First version.
%   \end{Version}
%   \begin{Version}{2007/11/11 v1.1}
%   \item
%     Use of package \xpackage{pdftexcmds} for \LuaTeX\ support.
%   \end{Version}
%   \begin{Version}{2011/01/30 v1.2}
%   \item
%     Already loaded package files are not input in \hologo{plainTeX}.
%   \end{Version}
%   \begin{Version}{2012/04/08 v1.3}
%   \item
%     Fix: \xpackage{pdftexcmds} wasn't loaded in case of \hologo{LaTeX}.
%   \end{Version}
% \end{History}
%
% \PrintIndex
%
% \Finale
\endinput
